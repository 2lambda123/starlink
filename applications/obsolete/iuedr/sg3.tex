\documentstyle[11pt,twoside]{article}
\pagestyle{myheadings}
\makeindex

%------------------------------------------------------------------------------
\newcommand{\stardoccategory}  {Starlink Guide}
\newcommand{\stardocinitials}  {SG}
\newcommand{\stardocsource}    {sg3.5}
\newcommand{\stardocnumber}    {3.5}
\newcommand{\stardocauthors}   {Paul Rees, Jack Giddings, Dave Mills \& Martin Clayton}
\newcommand{\stardocdate}      {12 March 1996}
\newcommand{\stardoctitle}     {IUEDR---Reference Manual}
%------------------------------------------------------------------------------


\newcommand{\stardocname}{\stardocinitials /\stardocnumber}
\newcommand{\numcir}[1]{\mbox{\hspace{3ex}$\bigcirc$\hspace{-1.7ex}{\small #1}}}
\newcommand{\lsk}{\raisebox{-0.4ex}{\rm *}}
%\renewcommand{\_}{{\tt\char'137}}     % re-centres the underscore - DONE LATER
\markright{\stardocname}
\setlength{\textwidth}{160mm}
\setlength{\textheight}{230mm}
\setlength{\topmargin}{-2mm}
\setlength{\oddsidemargin}{0mm}
\setlength{\evensidemargin}{0mm}
\setlength{\parindent}{0mm}
\setlength{\parskip}{\medskipamount}
\setlength{\unitlength}{1mm}


% -----------------------------------------------------------------------------
%  Hypertext definitions.
%  ======================
%  These are used by the LaTeX2HTML translator in conjuction with star2html.

%  Comment.sty: version 2.0, 19 June 1992
%  Selectively in/exclude pieces of text.
%
%  Author
%    Victor Eijkhout                                      <eijkhout@cs.utk.edu>
%    Department of Computer Science
%    University Tennessee at Knoxville
%    104 Ayres Hall
%    Knoxville, TN 37996
%    USA

%  Do not remove the %\begin{rawtex} and %\end{rawtex} lines (used by
%  star2html to signify raw TeX that latex2html cannot process).
%\begin{rawtex}
\makeatletter
\def\makeinnocent#1{\catcode`#1=12 }
\def\csarg#1#2{\expandafter#1\csname#2\endcsname}

\def\ThrowAwayComment#1{\begingroup
    \def\CurrentComment{#1}%
    \let\do\makeinnocent \dospecials
    \makeinnocent\^^L% and whatever other special cases
    \endlinechar`\^^M \catcode`\^^M=12 \xComment}
{\catcode`\^^M=12 \endlinechar=-1 %
 \gdef\xComment#1^^M{\def\test{#1}
      \csarg\ifx{PlainEnd\CurrentComment Test}\test
          \let\html@next\endgroup
      \else \csarg\ifx{LaLaEnd\CurrentComment Test}\test
            \edef\html@next{\endgroup\noexpand\end{\CurrentComment}}
      \else \let\html@next\xComment
      \fi \fi \html@next}
}
\makeatother

\def\includecomment
 #1{\expandafter\def\csname#1\endcsname{}%
    \expandafter\def\csname end#1\endcsname{}}
\def\excludecomment
 #1{\expandafter\def\csname#1\endcsname{\ThrowAwayComment{#1}}%
    {\escapechar=-1\relax
     \csarg\xdef{PlainEnd#1Test}{\string\\end#1}%
     \csarg\xdef{LaLaEnd#1Test}{\string\\end\string\{#1\string\}}%
    }}

%  Define environments that ignore their contents.
\excludecomment{comment}
\excludecomment{rawhtml}
\excludecomment{htmlonly}
%\end{rawtex}

%  Hypertext commands etc. This is a condensed version of the html.sty
%  file supplied with LaTeX2HTML by: Nikos Drakos <nikos@cbl.leeds.ac.uk> &
%  Jelle van Zeijl <jvzeijl@isou17.estec.esa.nl>. The LaTeX2HTML documentation
%  should be consulted about all commands (and the environments defined above)
%  except \xref and \xlabel which are Starlink specific.

\newcommand{\htmladdnormallinkfoot}[2]{#1\footnote{#2}}
\newcommand{\htmladdnormallink}[2]{#1}
\newcommand{\htmladdimg}[1]{}
\newenvironment{latexonly}{}{}
\newcommand{\hyperref}[4]{#2\ref{#4}#3}
\newcommand{\htmlref}[2]{#1}
\newcommand{\htmlimage}[1]{}
\newcommand{\htmladdtonavigation}[1]{}

%  Starlink cross-references and labels.
\newcommand{\xref}[3]{#1}
\newcommand{\xlabel}[1]{}

%  LaTeX2HTML symbol.
\newcommand{\latextohtml}{{\bf LaTeX}{2}{\tt{HTML}}}

%  Define command to recentre underscore for Latex and leave as normal
%  for HTML (severe problems with \_ in tabbing environments and \_\_
%  generally otherwise).
\newcommand{\latex}[1]{#1}
\newcommand{\setunderscore}{\renewcommand{\_}{{\tt\symbol{95}}}}
\latex{\setunderscore}

%  Redefine the \tableofcontents command. This procrastination is necessary
%  to stop the automatic creation of a second table of contents page
%  by latex2html.
\newcommand{\latexonlytoc}[0]{\tableofcontents}

% -----------------------------------------------------------------------------
%  Debugging.
%  =========
%  Un-comment the following to debug links in the HTML version using Latex.

% \newcommand{\hotlink}[2]{\fbox{\begin{tabular}[t]{@{}c@{}}#1\\\hline{\footnotesize #2}\end{tabular}}}
% \renewcommand{\htmladdnormallinkfoot}[2]{\hotlink{#1}{#2}}
% \renewcommand{\htmladdnormallink}[2]{\hotlink{#1}{#2}}
% \renewcommand{\hyperref}[4]{\hotlink{#1}{\S\ref{#4}}}
% \renewcommand{\htmlref}[2]{\hotlink{#1}{\S\ref{#2}}}
% \renewcommand{\xref}[3]{\hotlink{#1}{#2 -- #3}}
% -----------------------------------------------------------------------------
%  Add any document specific \newcommand or \newenvironment commands here

\newcommand{\lmbox}
{
    \mbox{} \\
}

\newcommand{\cpar}[2]
{
    \makebox[30mm][l]{\bf #1} & #2 (p~\pageref{#1}.)\\
}

\newcommand{\cparc}[1]
{
    \makebox[30mm][l]{ } & #1\\
}

\newcommand{\npar}[1]
{
    \makebox[30mm][l]{\bf #1} & \\
}

\newcommand{\iueparlist}[1]{
   \begin{description}
      #1
   \end{description}
}

\newcommand{\iueparameter}[3]
{
   \item [\label{#1}\index{#1}#1 = #2] \mbox{}\\
   #3
}

\newcommand{\indexentry}[2]
{
{\bf #1}\dotfill #2 \hspace*{15mm}\\
}

\newcommand{\findexentry}[3]
{
   \hspace*{\fill}\vspace*{3mm}\\
   \hspace*{\fill}{\large\bf --- #1 ---}\hspace*{\fill} \hspace*{15mm}\\
   \hspace*{\fill}\vspace*{-3mm}\\
   {\bf #2}\dotfill #3 \hspace*{15mm}\\
}

\newcommand{\comdescenv}[1]
{
\begin {tabular}{ll}
  #1
\end {tabular}
}

\newcommand{\comdesc}[2]
{
   \makebox[27mm][l]{\bf #1} & #2 \\
}

\newcommand{\comdescc}[1]
{
   \makebox[27mm][l]{ } & #1 \\
}


%% Redefine commands for hypertext version.

\begin{htmlonly}

\renewcommand{\lmbox}
{ }

\renewcommand{\cpar}[2]
{
    \item [\htmlref{#1}{#1}] #2
}

\renewcommand{\cparc}[1]
{
  #1
}

\rerenewcommand{\npar}[1]
{
    \item [#1]
}

\renewcommand{\indexentry}[2]
{
    {\bf \htmlref{#1}{#1}}\\
}

\renewcommand{\findexentry}[3]
{
    {\bf \htmlref{#2}{#2}}\\
}

\renewcommand{\comdescenv}[1]
{
   \begin{description}
       #1
   \end{description}
}

\renewcommand{\comdesc}[2]
{
   \item [{\bf \htmlref{#1}{#1}}] #2
}

\newcommand{\comdescc}[1]
{
  #1
}

\renewcommand{\iueparlist}[1]
{
      #1
}

\renewcommand{\iueparameter}[3]
{
\subsection{\xlabel{#1}\label{#1}#1}
   \begin{description}
   \item [{\bf Type:}] #2
   \item [{\bf Description:}] #3
   \end{description}
}

\end{htmlonly}

%+
%  Name:
%     SST.TEX

%  Purpose:
%     Define LaTeX commands for laying out Starlink routine descriptions.

%  Language:
%     LaTeX

%  Type of Module:
%     LaTeX data file.

%  Description:
%     This file defines LaTeX commands which allow routine documentation
%     produced by the SST application PROLAT to be processed by LaTeX and
%     by LaTeX2html. The contents of this file should be included in the
%     source prior to any statements that make of the sst commnds.

%  Notes:
%     The commands defined in the style file html.sty provided with LaTeX2html
%     are used. These should either be made available by using the appropriate
%     sun.tex (with hypertext extensions) or by putting the file html.sty
%     on your TEXINPUTS path (and including the name as part of the
%     documentstyle declaration).

%  Authors:
%     RFWS: R.F. Warren-Smith (STARLINK)
%     PDRAPER: P.W. Draper (Starlink - Durham University)

%  History:
%     10-SEP-1990 (RFWS):
%        Original version.
%     10-SEP-1990 (RFWS):
%        Added the implementation status section.
%     12-SEP-1990 (RFWS):
%        Added support for the usage section and adjusted various spacings.
%     8-DEC-1994 (PDRAPER):
%        Added support for simplified formatting using LaTeX2html.
%     {enter_further_changes_here}

%  Bugs:
%     {note_any_bugs_here}

% -

%  Define length variables.
\newlength{\sstbannerlength}
\newlength{\sstcaptionlength}
\newlength{\sstexampleslength}
\newlength{\sstexampleswidth}

%  Define a \tt font of the required size.
\newfont{\ssttt}{cmtt10 scaled 1095}

%  Define a command to produce a routine header, including its name,
%  a purpose description and the rest of the routine's documentation.
\newcommand{\sstroutine}[3]{
   \goodbreak
   \rule{\textwidth}{0.5mm}
   \vspace{-7ex}
   \newline
   \settowidth{\sstbannerlength}{{\Large {\bf #1}}}
   \setlength{\sstcaptionlength}{\textwidth}
   \setlength{\sstexampleslength}{\textwidth}
   \addtolength{\sstbannerlength}{0.5em}
   \addtolength{\sstcaptionlength}{-2.0\sstbannerlength}
   \addtolength{\sstcaptionlength}{-5.0pt}
   \settowidth{\sstexampleswidth}{{\bf Examples:}}
   \addtolength{\sstexampleslength}{-\sstexampleswidth}
   \parbox[t]{\sstbannerlength}{\flushleft{\Large {\bf #1}}}
   \parbox[t]{\sstcaptionlength}{\center{\Large #2}}
   \parbox[t]{\sstbannerlength}{\flushright{\Large {\bf #1}}}
   \label{#1}\index{#1}
   \begin{description}
      #3
   \end{description}
}

%  Format the description section.
\newcommand{\sstdescription}[1]{\item {\bf Description:}\vspace*{6pt}\\ #1}

%  Format the usage section.
\newcommand{\sstusage}[1]{\item[Usage:] \mbox{} \\[1.3ex] {\ssttt #1}}

%  Format the invocation section.
\newcommand{\sstinvocation}[1]{\item[Invocation:]\hspace{0.4em}{\tt #1}}

%  Format the arguments section.
\newcommand{\sstarguments}[1]
{
   \item[Arguments:] \mbox{} \\
   \vspace{-3.5ex}
   \begin{description}
      #1
   \end{description}
}

%  Format the returned value section (for a function).
\newcommand{\sstreturnedvalue}[1]{
   \item[Returned Value:] \mbox{} \\
   \vspace{-3.5ex}
   \begin{description}
      #1
   \end{description}
}

%  Format the parameters section (for an application).
\newcommand{\sstparameters}[1]{
\item {\bf Parameters:\vspace*{6pt}\\}
    \begin{tabular}{ll}
    #1
    \end{tabular}
}

%  Format the examples section.
\newcommand{\sstexamples}[1]{
   \item[Examples:] \mbox{} \\
   \vspace{-3.5ex}
   \begin{description}
      #1
   \end{description}
}

%  Define the format of a subsection in a normal section.
\newcommand{\sstsubsection}[1]{ \item[{#1}] \mbox{} \\}

%  Define the format of a subsection in the examples section.
\newcommand{\sstexamplesubsection}[2]{\sloppy
\item[\parbox{\sstexampleslength}{\ssttt #1}] \mbox{} \\ #2 }

%  Format the notes section.
\newcommand{\sstnotes}[1]{\item[Notes:] \mbox{} \\[1.3ex] #1}

%  Provide a general-purpose format for additional (DIY) sections.
\newcommand{\sstdiytopic}[2]{\item[{\hspace{-0.35em}#1\hspace{-0.35em}:}] \mbox{} \\[1.3ex] #2}

%  Format the implementation status section.
\newcommand{\sstimplementationstatus}[1]{
   \item[{Implementation Status:}] \mbox{} \\[1.3ex] #1}

%  Format the bugs section.
\newcommand{\sstbugs}[1]{\item[Bugs:] #1}

%  Format a list of items while in paragraph mode.
\newcommand{\sstitemlist}[1]{
  \mbox{} \\
  \vspace{-3.5ex}
  \begin{itemize}
     #1
  \end{itemize}
}

%  Define the format of an item.
\newcommand{\sstitem}{\item}

%% Now define html equivalents of those already set. These are used by
%  latex2html and are defined in the html.sty files.

\begin{htmlonly}

%  Re-define \ssttt.
   \newcommand{\ssttt}{\tt}

%  sstroutine.
   \renewcommand{\sstroutine}[3]{
\subsection{\xlabel{#1}\label{#1}#1}
      \begin{description}
         \item[{\bf Purpose:}] #2
         #3
      \end{description}
   }

%  sstdescription
   \renewcommand{\sstdescription}[1]{
      \item[{\bf Description:}]
      \begin{description}
         #1
      \end{description}
   }

%  sstusage
   \renewcommand{\sstusage}[1]{\item[Usage:]
      \begin{description}
         {\ssttt #1}
      \end{description}
   }

%  sstinvocation
   \renewcommand{\sstinvocation}[1]{\item[Invocation:]
      \begin{description}
         {\ssttt #1}
      \end{description}
   }

%  sstarguments
   \renewcommand{\sstarguments}[1]{
      \item[Arguments:]
      \begin{description}
         #1
      \end{description}
   }

%  sstreturnedvalue
   \renewcommand{\sstreturnedvalue}[1]{
      \item[Returned Value:]
      \begin{description}
         #1
      \end{description}
   }

%  sstparameters
   \renewcommand{\sstparameters}[1]{
      \item[{\bf Parameters:}]
      \begin{description}
         #1
      \end{description}
   }

%  sstexamples
   \renewcommand{\sstexamples}[1]{
      \item[Examples:]
      \begin{description}
         #1
      \end{description}
   }

%  sstsubsection
   \renewcommand{\sstsubsection}[1]{\item[{#1}]}

%  sstexamplesubsection
   \renewcommand{\sstexamplesubsection}[2]{\item[{\ssttt #1}] \\ #2}

%  sstnotes
   \renewcommand{\sstnotes}[1]{\item[Notes:]
      \begin{description}
         #1
      \end{description}
   }

%  sstdiytopic
   \renewcommand{\sstdiytopic}[2]{\item[{#1}]
      \begin{description}
         #2
      \end{description}
   }

%  sstimplementationstatus
   \renewcommand{\sstimplementationstatus}[1]{\item[Implementation Status:]
      \begin{description}
         #1
      \end{description}
   }

%  sstitemlist
   \newcommand{\sstitemlist}[1]{
      \begin{itemize}
         #1
      \end{itemize}
   }
\end{htmlonly}

%  End of "sst.tex" layout definitions.

% -----------------------------------------------------------------------------
%  Title Page.
%  ===========
\renewcommand{\thepage}{\roman{page}}
\begin{document}
\thispagestyle{empty}
%  Latex document header.
%  ======================
\begin{latexonly}
   CCLRC / {\sc Rutherford Appleton Laboratory} \hfill {\bf \stardocname}\\
   {\large Particle Physics \& Astronomy Research Council}\\
   {\large Starlink Project\\}
   {\large \stardoccategory\ \stardocnumber}
   \begin{flushright}
   \stardocauthors\\
   \stardocdate
   \end{flushright}
   \vspace{-4mm}
   \rule{\textwidth}{0.5mm}
   \vspace{5mm}
   \begin{center}
   {\Large\bf \stardoctitle}
   \end{center}
   \vspace{5mm}

%  Add heading for abstract if used.
%   \vspace{10mm}
%   \begin{center}
%      {\Large\bf Description}
%   \end{center}
\end{latexonly}

%  HTML documentation header.
%  ==========================
\begin{htmlonly}
   \xlabel{}
   \begin{rawhtml} <H1> \end{rawhtml}
      \stardoctitle
   \begin{rawhtml} </H1> \end{rawhtml}

%  Add picture here if required.

   \begin{rawhtml} <P> <I> \end{rawhtml}
   \stardoccategory \stardocnumber \\
   \stardocauthors \\
   \stardocdate
   \begin{rawhtml} </I> </P> <H3> \end{rawhtml}
      \htmladdnormallink{CCLRC}{http://www.cclrc.ac.uk} /
      \htmladdnormallink{Rutherford Appleton Laboratory}
                        {http://www.cclrc.ac.uk/ral} \\
      Particle Physics \& Astronomy Research Council \\
   \begin{rawhtml} </H3> <H2> \end{rawhtml}
      \htmladdnormallink{Starlink Project}{http://www.starlink.ac.uk/}
   \begin{rawhtml} </H2> \end{rawhtml}
   \htmladdnormallink{\htmladdimg{source.gif} Retrieve hardcopy}
      {http://www.starlink.ac.uk/cgi-bin/hcserver?\stardocsource}\\

%  HTML document table of contents.
%  ================================
%  Add table of contents header and a navigation button to return to this
%  point in the document (this should always go before the abstract \section).
  \label{stardoccontents}
  \begin{rawhtml}
    <HR>
    <H2>Contents</H2>
  \end{rawhtml}
  \renewcommand{\latexonlytoc}[0]{}
  \htmladdtonavigation{\htmlref{\htmladdimg{contents_motif.gif}}
        {stardoccontents}}

%  Start new section for abstract if used.
%  \section{\xlabel{abstract}Abstract}

\end{htmlonly}

% -----------------------------------------------------------------------------
%  Document Abstract. (if used)
%  ==================
% -----------------------------------------------------------------------------
%  Latex document Table of Contents (if used).
%  ===========================================
\begin{latexonly}
   \setlength{\parskip}{0mm}
   \latexonlytoc
   \setlength{\parskip}{\medskipamount}
   \markright{\stardocname}
\end{latexonly}
% -----------------------------------------------------------------------------

%%%%%%%%%%%%%%%%%%%%%%%%%%%%%%%%%%%%%%%%%%%%%%%%%%%%%%%%%%%%%%%%%%%%%%%%%%%
\newpage
\renewcommand{\thepage}{\arabic{page}}
\setcounter{page}{1}
\section{\xlabel{introduction}\label{se:introduction}Introduction }
\markboth{Introduction}{\stardocname}

This manual describes the commands and parameters used by IUEDR\@.
It is intended as a reference aid for people using IUEDR\@.

If you are new to IUE data reduction, you may like to read
\xref{{\sl IUE Analysis
a Tutorial}}{sg7}{} (SG/7) and the
\xref{{\sl IUEDR User Guide}}{mud45}{} (MUD/45) before proceeding
any further.
The Starlink User Note \xref{SUN/37}{sun37}{} contains a general description
of IUEDR which
overlaps with the early sections of this manual and also contains any notes on
the most recent release of the program.

Commands are described in Section~\ref{se:commands}, with parameters described
in more detail in Section~\ref{se:parameters}\@.
A list of default parameter behaviour and values is given in
Appendix~\ref{se:parameter_defaults}\@.
Details of transferring old-style IUEDR files from VMS systems to UNIX systems
and the conversion process are given in Appendix~\ref{se:vmsunix}\@.

\begin{latexonly}
An index of both commands and parameters is given in Appendix~\ref{se:index}\@.
\end{latexonly}

IUEDR functions fall into a number of specific categories:

\begin {itemize}
   \item IUE GO tape or file inspection and reading.
   \item Data display and manipulation.
   \item Spectrum extraction and calibration.
   \item Extraction product inspection and manipulation.
   \item Extraction product output.
   \item General operational commands.
\end {itemize}

These functions are controlled by over fifty commands, with nearly one hundred
global parameters within IUEDR\@.
There follows a summary of the commands available in each of the categories
listed above.

\subsection {IUE GO tape or file inspection and reading}

\comdescenv{
   \comdesc{LISTIUE}{Analyse the contents of one or more IUE tape files.}
   \comdesc{MTMOVE}{Move to the start of a tape file.}
   \comdesc{MTREW}{Rewind to the start of the tape.}
   \comdesc{MTSHOW}{Show the current tape position.}
   \comdesc{MTSKIPEOV}{Skip over the end-of-volume mark.}
   \comdesc{MTSKIPF}{Skip over NSKIP tape marks.}
   \comdesc{READIUE}{Read a RAW, GPHOT or PHOT IUE image from the tape/file.}
   \comdesc{READSIPS}{Read the MELO or MEHI IUESIPS product from the tape/file.}
}

\subsection {Data display and manipulation}

\comdescenv{
   \comdesc{CULIMITS}{Delineate the graphical display limits using the
                      graphics cursor.}
   \comdesc{CURSOR}{Determine display coordinates using the graphics cursor}
           \comdescc{and print them at the terminal.}
   \comdesc{DRIMAGE}{Display an IUE image on a suitable graphics workstation.}
   \comdesc{EDIMAGE}{Edit the image data quality using the graphics cursor.}
   \comdesc{MODIMAGE}{Modify image pixel intensities interactively.}
   \comdesc{CLEAN}{Mark as `bad' pixels with value below a given threshold.}
   \comdesc{SHOW}{Print information relating to the current dataset at the
                  terminal.}
   \comdesc{ERASE}{Erase the display screen of the current graphics
                   workstation.}
}

Image displays are colour coded to provide data quality information.
The colour codes used by IUEDR are as follows:

\begin{latexonly}
\begin {quote}
\begin {description}
   \item [Green] pixels affected by reseau marks
   \item [Red] pixels which are saturated (DN=255)
   \item [Orange] pixels affected by ITF truncation
   \item [Yellow] pixels marked bad by the user
\end {description}
\end {quote}
\end{latexonly}

\begin{htmlonly}
\begin{rawhtml}
<PRE>
   <B>Green</B>  - pixels affected by reseau marks
   <B>Red</B>    - pixels which are saturated (DN=255)
   <B>Orange</B> - pixels affected by ITF truncation
   <B>Yellow</B> - pixels marked bad by the user
</PRE>
\end{rawhtml}
\end{htmlonly}

The colour {\bf Blue} is used to indicate a pixel which has a value above the
maximum that can be displayed using the linear greyscale image display colour
look-up table.

When using a mouse or tracker-ball with the graphics cursor, the cursor hit
buttons are normally numbered in increasing order from left to right.
For example the left mouse button corresponds to cursor key hit 1, middle
button to cursor key hit 2 and so on.
Many terminals allow left-handed users to reverse the mouse button order.

\subsection {Spectrum extraction and calibration}

\comdescenv{
   \comdesc{AESHIFT}{Determine (HIRES) spectrum ESHIFT automatically.}
   \comdesc{AGSHIFT}{Determine spectrum template shift automatically.}
   \comdesc{BARKER}{Correct the extracted data for \'{e}chelle ripple}
           \comdescc{using a method based upon that of Barker (1984).}
   \comdesc{CGSHIFT}{Determine spectrum template shift using the cursor
                     on a SCAN plot.}
   \comdesc{LBLS}{Extract a line-by-line-spectrum array from the image.}
   \comdesc{NEWABS}{Associate a new absolute flux calibration with the
                  current  dataset.}
   \comdesc{NEWCUT}{Associate new \'{e}chelle order wavelength limits with the
                  current  dataset.}
   \comdesc{NEWDISP}{Associate new spectrograph dispersion data with the
                   current  dataset.}
   \comdesc{NEWFID}{Associate new fiducial positions with the current
                  dataset.}
   \comdesc{NEWRIP}{Associate new ripple calibration data with the current
                  dataset.}
   \comdesc{NEWTEM}{Associate new spectrum centroid template data with the
                    current dataset.}
   \comdesc{SCAN}{Perform a scan of the image data perpendicular}
           \comdescc{to the spectrograph dispersion.}
   \comdesc{SETA}{Set dataset parameters which are aperture specific.}
   \comdesc{SETD}{Set dataset parameters which are independent of order and
                aperture.}
   \comdesc{SETM}{Set dataset parameters which are order specific.}
   \comdesc{TRAK}{Extract a spectrum from the image.}
}

\subsection {Extraction product inspection and manipulation}

\comdescenv{
   \comdesc{EDMEAN}{Edit the mean extracted spectrum using the graphics
                  cursor.}
   \comdesc{EDSPEC}{Edit the net extracted spectrum using the graphics
                  cursor.}
   \comdesc{MAP}{Map and merge extracted spectrum components to produce}
           \comdescc{a mean spectrum.}
   \comdesc{PLCEN}{Plot the smoothed spectrum centroid shifts.}
   \comdesc{PLFLUX}{Plot the calibrated flux spectrum.}
   \comdesc{PLGRS}{Plot the pseudo-gross and background resulting from the}
           \comdescc{spectrum extraction.}
   \comdesc{PLMEAN}{Plot the mean spectrum.}
   \comdesc{PLNET}{Plot the uncalibrated net spectrum.}
   \comdesc{PLSCAN}{Plot the image scan perpendicular to the dispersion.}
   \comdesc{SGS}{Print the names of the available SGS graphics devices at
               the  terminal.}
}

Plots of extracted IUE spectra and image scans include data quality information
flags for bad data.
The data quality codes used by IUEDR are as follows:

\begin{latexonly}
\begin {quote}
\begin {description}
   \item [1] affected by extrapolated ITF
   \item [2] affected by microphonics
   \item [3] affected by noise spike
   \item [4] affected by bright point (or user)
   \item [5] affected by reseau mark
   \item [6] affected by ITF truncation
   \item [7] affected by saturation
   \item [U] affected by user edit
\end {description}
\end {quote}
\end{latexonly}

\begin{htmlonly}
\begin{rawhtml}
<PRE>
   <B>1</B> - affected by extrapolated ITF
   <B>2</B> - affected by microphonics
   <B>3</B> - affected by noise spike
   <B>4</B> - affected by bright point (or user)
   <B>5</B> - affected by reseau mark
   <B>6</B> - affected by ITF truncation
   <B>7</B> - affected by saturation
   <B>U</B> - affected by user edit
</PRE>
\end{rawhtml}
\end{htmlonly}

\subsection {Extraction product output}

\comdescenv{
   \comdesc{OUTEM}{Output the current spectrum template data to a formatted
                 data  file.}
   \comdesc{OUTLBLS}{Output the current LBLS array to a binary data file.}
   \comdesc{OUTMEAN}{Output the current mean spectrum to a DIPSO SP
                   format  data file.}
   \comdesc{OUTNET}{Output the current net spectrum to a DIPSO SP
                  format data file.}
   \comdesc{OUTRAK}{Output the current uncalibrated spectrum to a}
           \comdescc{``TRAK'' formatted data file.}
   \comdesc{OUTSCAN}{Output the current scan data to a DIPSO SP format
                   data  file.}
   \comdesc{OUTSPEC}{Output the current aperture (LORES) or order (HIRES)}
           \comdescc{spectrum to a DIPSO SP format data file.}
   \comdesc{PRGRS}{Print the current extracted aperture or order spectrum
                 in tabular form.}
   \comdesc{PRLBLS}{Print the current LBLS array in tabular form.}
   \comdesc{PRMEAN}{Print the current mean spectrum in tabular form.}
   \comdesc{PRSCAN}{Print the intensities of the current image scan in
                  tabular  form.}
   \comdesc{PRSPEC}{Print the current aperture or order spectrum in tabular
                  form.}
}

\subsection {General operational commands}

\comdescenv{
   \comdesc{EXIT}{Leave IUEDR and update any files altered during the}
           \comdescc{current IUEDR session.}
   \comdesc{QUIT}{Leave IUEDR and update any files altered during the}
           \comdescc{current IUEDR session.}
   \comdesc{SAVE}{Overwrite any files that have had their contents}
           \comdescc{updated during the current IUEDR session.}
}

A log of all commands typed during an IUEDR session and program output at the
terminal can be found in the file {\tt session.lis}.
This is particularly useful when investigating the contents of IUE tapes.

\newpage
\section{\xlabel{user_interface}User interface}
\markboth{User interface}{\stardocname}

The IUEDR user interface uses the Starlink ADAM parameter system.
The interface will be familiar to users of the VMS IUEDR, with a few changes
to reach a level of consistency with other Starlink packages.

\subsection {Starting IUEDR}

To initialise for IUEDR type
\begin{verbatim}
   % iuedr
\end{verbatim}

at the shell prompt \verb+%+.
The first time you type the command, IUEDR environment variables are set up in
your session.
You can now start the program by typing
\begin{verbatim}
   % iuedr
\end{verbatim}
again.
Edit your \verb+.login+ file if you want to avoid having to type the command
that extra time.

The command line interface prompt is \verb+>+\@.
This should appear after a welcome message and you can then type commands as
you would in a typical command shell.

\subsection {Response to command prompts}

Instructions to IUEDR are given as command lines.

Command lines begin with a command and an optional list of parameter
assignments. For example:
\begin{verbatim}
   > DRIMAGE DATASET=SWP14931 DEVICE=xw
\end{verbatim}

Usually IUEDR will only prompt for parameters required by commands if
they have no currently defined value.
However, some parameters are either cancelled during the execution of a
command or are set so that the user is always prompted for a value.
A command can be forced to prompt for all required parameter values thus:

\begin{verbatim}
   > READIUE PROMPT
\end{verbatim}

The \verb+PROMPT+ may be abbreviated to \verb+PR+\@.

In a similar way commands which always prompt for parameter values can be made
to accept default values thus:

\begin{verbatim}
   > READIUE ACCEPT
\end{verbatim}

Command input and output printed at the terminal is also copied to the
file \verb+session.lis+ in the working directory.
Note that this file is rewritten each time IUEDR is run.

\subsection {Response to parameter prompts}

Help about a parameter can be obtained by responding to the parameter prompt
with a question mark, {\it{e.g.,}}

\begin{verbatim}
   DATASET - Dataset Name. > ?
\end{verbatim}

Help information will then be printed at the terminal and the prompt repeated.

Sometimes an undefined parameter value is interpreted by a command in
a specific way ({\it{e.g.,}}\ auto-scaling within plotting commands)\@.
A parameter can be set undefined by responding to the prompt with an
exclamation mark, {\it{e.g.,}}

\begin{verbatim}
   XL - X-axis plotting limits, [0,0] means auto-scale. /[1150,1950]/ > !
\end{verbatim}

A command may be aborted by responding to a parameter prompt with a
double exclamation mark, {\it{e.g.,}}

\begin{verbatim}
   XL - X-axis plotting limits, [0,0] means auto-scale. /[1150,1950]/ > !!
\end{verbatim}

To prevent a command from prompting for further parameter values (once in
prompt mode) type a backslash, {\it{e.g.,}}

\begin{verbatim}
   XL - X-axis plotting limits, [0,0] means auto-scale. /[1150,1950]/ > \
\end{verbatim}

The command will then only prompt for parameters for which it cannot generate
a suitable value.

\subsection {Getting HELP}

\begin{latexonly}
Type HELP at the IUEDR command line prompt.
You may optionally append a detailed description of the topic on which help
is required. See page~\pageref{com: HELP } for further details.
\end{latexonly}

\begin{htmlonly}
Type HELP at the IUEDR command line prompt.
You may optionally append a detailed description of the topic on which help
is required. See HELP for further details.
\end{htmlonly}

\subsection {IUEDR in script and batch modes}

It is possible to run IUEDR in script mode, where command and
parameter input originates from a file instead of the terminal.
The file from which the command input is to be taken is piped into IUEDR:

\begin{verbatim}
   % iuedr < script_file
\end{verbatim}

This will result in the command input being taken from the file
\verb+script_file+ and the text output being written to the \verb+session.lis+
file.

Alternatively the output can be directed to a specific file:

\begin{verbatim}
   % iuedr < script_file > script_log
\end{verbatim}

In this case \verb+script_log+ contains the output log \verb+session.lis+ is
{\bf not} over-written.

\newpage
\section{\xlabel{IUEDR_data_files}IUEDR data files}
\markboth{IUEDR data files}{\stardocname}

Once an IUE GO format file has been read by IUEDR two files are created.  One
is a text file containing information about the data set calibration data.
This file has a name constructed
\begin{verbatim}
   <dataset>.UEC
\end{verbatim}
where \verb+<dataset>+ corresponds to the value of the IUEDR \verb+DATASET+
parameter.

when a GPHOT, PHOT or RAW image is read the data produced are stored in an
Image and Data Quality file
\begin{verbatim}
   <dataset>_UED.sdf
\end{verbatim}
After the spectrum extraction process the uncalibrated spectral data are stored
in a file
\begin{verbatim}
   <dataset>_UES.sdf
\end{verbatim}
This file is also produced when an IUESIPS MELO or MEHI file is read with
\verb+READSIPS+\@.  No \verb+_UED.sdf+ being created in this case.

Calibrated spectra produced by the \verb+MAP+ command are stored in a file
\begin{verbatim}
   <dataset>_UEM.sdf
\end{verbatim}

All these \verb+.sdf+ files are Starlink NDF format files (See
\xref{SUN/33}{sun33}{}
for details of access to NDFs) which can be read by any of the standard
packages (KAPPA, FIGARO etc.).  The contents of these files can be examined
outside of IUEDR using the \verb+hdstrace+ command.

These files are in addition to the IUEDR log file \verb+session.lis+ and the
files generated by output commands.  They should {\bf not} be deleted until the
data reduction is complete and the output spectra obtained.

The \verb+OUT*+ family of IUEDR output commands also generate NDFs which can
be read by programs such as DIPSO.

In summary:
\begin {description}
   \item \verb+<dataset>.UEC+ --- calibration file.
   \item \verb+<dataset>_UED.SDF+ --- image data and quality file.
   \item \verb+<dataset>_UES.SDF+ --- uncalibrated spectrum file.
   \item \verb+<dataset>_UEM.SDF+ --- calibrated mean spectrum file.
\end {description}

{\bf Refer to Appendix~\ref{se:vmsunix} for VMS to UNIX file conversion.}

\subsection{\label{subap:ndf}NDF Components in IUEDR files}

This section gives a summary of the NDF components present in IUEDR files for
those who may wish to access the files from their own programs.
The structure and content of an NDF can be inspected using the {\tt hdstrace}
utility (See \xref{SUN/102}{sun102}{})\@.  Values for components have been
given where they are constant for all files of the particular type.

\subsubsection{Image data and quality file {\tt \_UED}}

\begin{latexonly}
A simple NDF, each point in the $768\times 768$ image is described by a datum
and a quality flag.  The image is given the generic title `IUE image'\@.
\end{latexonly}

\begin{htmlonly}
A simple NDF, each point in the 768x768 image is described by a datum
and a quality flag.  The image is given the generic title `IUE image'\@.
\end{htmlonly}

\begin{verbatim}
IUEDR  <NDF>

   DATA_ARRAY(768,768)  <_WORD>

   QUALITY        <QUALITY>       {structure}
      QUALITY(768,768)  <_UBYTE>

   TITLE          <_CHAR*9>       'IUE image'
\end{verbatim}

\subsubsection{Uncalibrated spectrum file {\tt \_UES}}

This NDF contains IUEDR specific extensions, which are written and read by the
program when processing spectra.  In the description below {\tt no} is
the number of orders processed with the \verb+TRAK+ command.  {\tt mo} is
the number of data points in the longest order processed.
{\tt WAVES} contains the wavelengths of each of the flux data in each order.
{\tt ORDERS} holds a list of the order numbers processed.
{\tt NWAVS} stores the actual number of points in each of the {\tt no} orders.

\begin{verbatim}
IUEDR  <NDF>

   DATA_ARRAY(mo,no)   <_REAL>

   QUALITY        <QUALITY>       {structure}
      QUALITY(mo,no)   <_UBYTE>

   MORE           <EXT>           {structure}
      IUEDR_EXTRA    <EXTENSION>     {structure}
         WAVES(mo,no)   <_REAL>
         ORDERS(no)     <_INTEGER>
         NWAVS(no)      <_INTEGER>

   TITLE          <_CHAR*80>
   LABEL          <_CHAR*4>       'Flux'
\end{verbatim}

\subsubsection{Calibrated mean spectrum file {\tt \_UEM}}

The mean spectrum is basically an array of flux values against a wavelength
scale.  Quality information is included in the data file.  The axes units are
also included.
This NDF type is again IUEDR specific, the unusual components are as follows.
{\tt WAVES} are wavelengths of flux data points.
{\tt WEIGHTS} are weights applied to the flux.
{\tt XCOMB1} is the start wavelength.
{\tt DXCOMB} is the wavelength step from point to point.



\begin{verbatim}
IUEDR  <NDF>

   DATA_ARRAY(17001)  <_REAL>

   QUALITY        <QUALITY>       {structure}
      QUALITY(17001)  <_UBYTE>

   MORE           <EXT>           {structure}
      IUEDR_EXTRA    <EXTENSION>     {structure}
         WAVES(17001)   <_REAL>
         WEIGHTS(17001)  <_REAL>
         XCOMB1         <_DOUBLE>
         DXCOMB         <_DOUBLE>

   AXIS(1)        <AXIS>          {structure}
      DATA_ARRAY(17001)  <_REAL>
      UNITS          <_CHAR*40>      '(A)'
      LABEL          <_CHAR*40>      'Wavelength'

   TITLE          <_CHAR*80>
   UNITS          <_CHAR*40>      '(FN/s)'
   LABEL          <_CHAR*40>      'Flux'
\end{verbatim}

\subsubsection{\label{se:spectrum}SPECTRUM format output files}

SPECTRUM is a data analysis programme written by Steve Adams at UCL\@.
Although the programme is no longer used (I guess it might be in use
somewhere\ldots) the file formats it introduced were adopted by the popular
spectrum analysis programme DIPSO (described in \xref{SUN/50}{sun50}{})\@.

The basic input to a SPECTRUM file is a single spectrum (wavelength, flux)\@.
The wavelengths should be in increasing order, and evenly spaced.

There are three variants of the SPECTRUM file format.  Using its terminology:

\begin{latexonly}
\begin{tabular}{ll}
Format number & File characteristics\\
0             & Unformatted (Binary)\\
1             & Fixed Format Text\\
2             & Free-field Format Text\\
\end{tabular}
\end{latexonly}

\begin{htmlonly}
\begin{rawhtml}
<PRE>
<B>Format number     File characteristics</B>
      0           Unformatted (Binary)
      1           Fixed Format Text
      2           Free-field Format Text
</PRE>
\end{rawhtml}
\end{htmlonly}

IUEDR {\bf no longer} produces output of the SP0 type.  Instead NDFs are used.
In practice this is invisible to the user as the DIPSO SP0RD command (read
SPECTRUM format 0 file) now reads NDFs!  The other two formats are still
available.  A Description of the old SP0 format is included here in case
anyone needs to read an existing file in this format (DIPSO can still read
SP0 format via the SP0RD command)\@.

Using a FORTRAN77 notation, the contents of a SPECTRUM file can be
expressed as:

\begin{verbatim}
   PARAMETER(MAXWAV=8000) ! maximum number of wavelengths
   CHARACTER*79 CLINE1    ! first line of text
   CHARACTER*79 CLINE2    ! second line of text
   INTEGER NWAV           ! number of wavelengths
   REAL WAV(MAXWAV)       ! wavelengths
   REAL FLUX(MAXWAV)      ! fluxes
\end{verbatim}

Both \verb+CLINE1+ and \verb+CLINE2+ are totally unstructured text strings,
and are used to describe the spectrum.  The convention is that
\verb+FLUX(I)=0.0+ when its value is undefined.

Here, briefly, is the code needed to read the SPECTRUM formats:

Format number 0 is an unformatted (binary) file read by:

\begin{verbatim}
   OPEN(UNIT=1, ACCESS='SEQUENTIAL', FORM='UNFORMATTED')
   READ(1) CLINE1(1:79)
   READ(1) CLINE2(1:79)
   READ(1) NWAV
   READ(1) (WAV(I),FLUX(I),I=1,NWAV)
   CLOSE(UNIT=1)
\end{verbatim}

Format number 1 is a fixed format text file read by:

\begin{verbatim}
   OPEN(UNIT=1, ACCESS='SEQUENTIAL')
   READ(1,'(A79)') CLINE1(1:79)
   READ(1,'(A79)') CLINE2(1:79)
   READ(1,'(20X,I6)') NWAV
   READ(1,'(4(F8.3,E10.3))') (WAV(I),FLUX(I),I=1,NWAV)
   CLOSE(UNIT=1)
\end{verbatim}

Format number 2 is a free-field text file read by:

\begin{verbatim}
   OPEN(UNIT=1, ACCESS='SEQUENTIAL')
   READ(1,'(A79)') CLINE1(1:79)
   READ(1,'(A79)') CLINE2(1:79)
   READ(1,*) NWAV
   READ(1,*) (WAV(I),FLUX(I),I=1,NWAV)
   CLOSE(UNIT=1)
\end{verbatim}

The sections of FORTRAN 77 code shown above are not intended to be serious
attempts to write a SPECTRUM file reading programme.  Instead they are designed
to define the contents as succinctly as possible.

\newpage
\section{\xlabel{porting_changes}Changes during the port to UNIX}
\markboth{Changes during the port to UNIX}{\stardocname}

This section describes the main changes that have been made to IUEDR
during its conversion to an ADAM based application which runs on
all Starlink supported platforms.  \xref{SUN/37}{sun37}{} gives notes on the
very latest version of IUEDR.

If you are a seasoned IUEDR user then you should study this section
especially carefully.

The most significant change from the scientific point of view is that
the precision of all floating point calculations has been upgraded to
DOUBLE PRECISION. This was done after it was noticed that for high
resolution extraction the output spectra were subject to rounding
noise at the 1\% level.

The format of the calibration file ({\tt .UEC}) created by IUEDR has
been  changed to make it more readable. A VMS program to convert
IUEDR datasets to the new format is available, see Appendix~\ref{se:vmsunix}
for details.

The functionality of the package has been enhanced to allow image data
to be read directly from disk.

The general operation of IUEDR, and all the command and parameter
names, are identical to those used in previous versions.

\subsection{IUEDR command files}

The \verb+.CMD+ style of VMS IUEDR command files is not directly supported by
UNIX IUEDR, and neither is the associated input/output redirection using
\verb+<+ and \verb+>+\@.

It is very easy to convert a {\tt .CMD} file into a UNIX IUEDR command
script.
{\it{e.g.,}}\ a {\tt DEMO.CMD} procedure:

\begin{verbatim}
   DATASET=SWP03196
   SHOW
   SCAN ORDERS=(125,66)
   TRAK APERTURE=LAP
   SHOW V=S
\end{verbatim}

would become a UNIX IUEDR command script \verb+demo.cmd+, thus:

\begin{verbatim}
   SHOW DATASET=SWP03196
   SCAN ORDERS=[125,66]
   TRAK APERTURE=LAP
   SHOW V=S
\end{verbatim}

The only changes which need be made are to move any parameter
specifications ({\it{e.g.,}}\ \verb+DATASET=+) onto the same line as the command
they apply to, and to change vector parameter specifications to use square
\verb+[]+ brackets instead of the old-style round \verb+()+ brackets.

This script can then be read into IUEDR by
\begin{verbatim}
   % iuedr < demo.cmd
\end{verbatim}

\subsection{Interaction with DIPSO}

As part of the port to UNIX the
format of the default DIPSO spectrum format SP0 files has been
changed to use the STARLINK NDF data format. This means that these
files can be read by any standard STARLINK package.

The IUEDR/DIPSO user should notice no difference, as both IUEDR and
DIPSO understand the new format.

\subsection{DRIVE parameter options}

The use of the DRIVE parameter has been enhanced to allow
specification of disk files containing IUE datasets. This is intended
for use with  files obtained from online archives (RAL and NASA).

The syntax is to provide the full filename and extension in response to
the DRIVE prompt:

\begin{verbatim}
   DRIVE> SWP12345.RAW
\end{verbatim}

\subsection{Specifying vector parameters}

Some IUEDR parameters ({\it{e.g.,}}\ \verb+XP+, \verb+YP+) require the
specification of a pair of numbers defining the limits of a range of values
({\it{e.g.,}}\ pixels).

The method of setting such values has changed to the ADAM style:

\begin{verbatim}
   XP=[100,300]
\end{verbatim}

Note that the square brackets are only necessary when vector
parameters are specified on the command line. They are not required
when IUEDR prompts the user for a vector parameter.

\subsection{Calibration files and the NEW* family of commands}

The missing Calibration file for SWP camera HIRES data has been added to the
IUEDR package.

The action of the \verb+NEW*+ family of commands for updating IUEDR
calibrations, geometry and so on have been altered  to improve functionality.
Users will find that the previous style of file name entry still works,
and that the following features have been added:
\begin{itemize}
   \item Both Logical Name and Environment Variable style file specifications
   may be given as parameter values, for example
   \begin{verbatim}
      > NEWABS ABSFILE=$IUEDR_DATA/swphi
   \end{verbatim}
   and
   \begin{verbatim}
      > NEWABS ABSFILE=IUEDR_DATA:swphi
   \end{verbatim}
   are equivalent and allowed on all platforms.
   \item The default file name extension need not be given, however if the
   file to be read has a different extension this {\bf should} be given.
   For example,
   \begin{verbatim}
      > NEWABS ABSFILE=$IUEDR_DATA/swphi
   \end{verbatim}
   and
   \begin{verbatim}
      > NEWABS ABSFILE=$IUEDR_DATA/swphi.abs
   \end{verbatim}
   are {\bf both} valid.

   This behaviour is a change to the VMS-only IUEDR where the extension had
   to be omitted and the default value for the appropriate command was always
   taken.
   \item For case-sensitive file systems (like UNIX) if the Enviroment Variable
   \verb+$IUEDR_DATA+ is used as part of the file specification then the case
   of the file name itself is always converted to {\bf lower case}.  All the
   files available in the \verb+$IUEDR_DATA+ directory have lower case names
   so this is not a problem, rather it allows default calibration file names
   to be upper- or lower-case in IUEDR command scripts.
\end{itemize}

\subsection{Documentation}

The excellent introduction to IUEDR
\xref{{\sl IUE Analysis---A Tutorial}}{sg7}{}
(SG/7) by Richard Tweedy has been updated for UNIX and included as a standard
part of IUEDR\@.  Some special calibration corrections described in this
document have also been added to the IUEDR package.

\newpage
\section{\xlabel{commands}\label{se:commands}Commands}
\markboth{Commands}{\stardocname}

This section contains a detailed description of each of the commands
available in IUEDR.  A list of the parameters used by each command is given,
along with a brief description of each.  The pages on which you will find
full parameter descriptions are given at the end of each line in the parameter
list.

\sstroutine{AESHIFT}
{
   Determine (HIRES) spectrum ESHIFT automatically.
}{
   \sstparameters{
   \cpar{DATASET}{Dataset name.}
   \cpar{CENTREWAVE}{Line central wavelengths (A).}
   \cpar{DELTAWAVE}{Half-width of line search windows (A).}
}
\sstdescription{
   \verb+AESHIFT+ can be used to measure the global \'{e}chelle shift for a
   HIRES
   spectrum.  A set of laboratory wavelengths of absorption features which
   should be present in the spectrum are located in the spectrum and the
   \verb+ESHIFT+ for each is calculated.  The median of these \verb+ESHIFT+s
   is then applied to the whole dataset.
}
}

\sstroutine{AGSHIFT}
{
   Determine spectrum shift for HIRES automatically.
}{
   \sstparameters{
   \cpar{DATASET}{Dataset name.}
   \cpar{ORDERS}{This delineates a range of \'{e}chelle orders.}
}
\sstdescription{
   \verb+AGSHIFT+ can be used to measure the global geometric shift for a HIRES
   spectrum.  A scan of the \verb+DATASET+ must be made available using the
   \verb+SCAN+ command.  The scan data is traversed starting at the lowest
   numbered order and a probable site for the peak of each order is found.
   The central position of each order is then estimated using a centroiding
   algorithm.  An estimate of the geometric shift is made for each order and
   these are recorded and displayed.

   A weighted mean in which the shifts determined for orders 100 to 110
   inclusive are given greater weight than other shifts is calculated.
   The individual order shifts are compared to the mean and any shift
   greater than 3 pixels from the mean position is rejected and a new value
   for the mean shift calculated from the remaining orders.
   The mean shift is displayed and the \verb+GSHIFT+ parameter is set.

   Using \verb+AGSHIFT+ it is possible to automate the spectrum extraction
   process.  It should be noted that objects with no continuum may break
   the \verb+AGSHIFT+ mechanism, giving poor shift values, in these cases
   the interactive \verb+CGSHIFT+ command should be used.
}
}

\sstroutine{BARKER}
{
   Correct spectrum data for \'{e}chelle ripple using a method based
   upon that of Barker (1984).
}{
   \sstparameters{
   \cpar{DATASET}{Dataset name.}
   \cpar{ORDERS}{This delineates a range of \'{e}chelle orders.}
}
\sstdescription{
   The spectrum data in \verb+DATASET+ are corrected for residual \'{e}chelle
   ripple using  the method described by Barker (1984. Astronomical Journal,
   \underline{89},  899). Orders in the range \verb+ORDERS+ are used in the
   ripple correction optimisation.  Note that this optimisation method is
   only applicable for SWP spectra.
}
}


\sstroutine{CGSHIFT}
{
   Determine spectrum template shift using the cursor on a scan plot.
}{
   \sstparameters{
   \cpar{DATASET}{Dataset name.}
   \cpar{APERTURE}{Aperture name (\verb+SAP+ or \verb+LAP+).}
   \cpar{ORDERS}{This delineates a range of \'{e}chelle orders.}
   \cpar{DEVICE}{GKS/SGS graphics device name.}
}
\sstdescription{
   This command allows the graphics cursor to be used to provide
   information about spectrum template registration shifts.

   A plot of the current spectrum scan must be available on the graphics
   \verb+DEVICE+\@.

   A cycle consisting of any  number of left or middle mouse button hits is
   used to mark the position of the spectrum. Each hit is used to calculate
   a linear geometric shift of the spectrum template relative to the image.
   The cycle is terminated by pressing the right mouse button.

   Keyboard keys 1, 2 and 3 can be used  in place of left, middle and right
   mouse buttons respectively.

   For LORES, when the cycle is complete the last geometric shift
   determined is adopted and the scan is revoked.

   For HIRES, each cursor hit is automatically associated with an \'{e}chelle
   order in the range defined by the \verb+ORDERS+ parameter. The last shift is
   again adopted, but the scan is available for further display or
   measurement.
}
}

\newpage
\sstroutine{CLEAN}
{
   Mark pixels with values below a selected threshold as BAD.
}{
   \sstparameters{
   \cpar{DATASET}{Dataset name.}
   \cpar{THRESH}{Smallest pixel value to be accepted as GOOD.}
}
\sstdescription{
   Some IUE  datasets are effected  by horizontal  bars of low pixel values.
   These are caused by a weak  signal from the IUE craft at the time of data
   download.  When there are  a few bars  they can be marked as bad with the
   \verb+EDIMAGE+ command.  In the case of many bars a quicker solution is to
   mark all pixels in  the image below  a user selected  threshold value as BAD.

   Successive \verb+CLEAN+ and \verb+DRIMAGE+ commands starting with a value of
   \verb+THRESH=-1000+ and increasing \verb+THRESH+ towards zero will  allow
   the user to chose a value suitable for the problem image.
}
}

\sstroutine{CULIMITS}
{
   Set display limits with the cursor.
}{
   \sstparameters{
   \cpar{DEVICE}{GKS/SGS graphics device name.}
   \cpar{XL}{$x$-axis plotting limits, undefined or [0, 0] means auto-scale.}
   \cpar{YL}{$y$-axis plotting limits, undefined or [0, 0] means auto-scale.}
   \cpar{XP}{$x$-axis pixel limits, undefined or [0, 0] means full extent.}
   \cpar{YP}{$y$-axis pixel limits, undefined or [0, 0] means full extent.}
}
\sstdescription{
   This command uses the cursor to delineate part of a current display,
   graph or image, to be displayed in some subsequent command
   ({\it{e.g.,}}\ \verb+PLFLUX+, \verb+DRIMAGE+\ldots ).

   The two cursor positions should be at the corners of the required
   rectangular subset. The relation between cursor position sequences and
   axis reversals for graphs is:

   \begin{tabular}{llll}
   {\bf Position 1} & {\bf Position 2} & {\bf x-reversed} & {\bf y-reversed}\\
   bottom/left  & top/right    & NO  & NO \\
   bottom/right & top/left     & YES & NO \\
   top/left     & bottom/right & NO  & YES \\
   top/right    & bottom/left  & YES & YES
   \end{tabular}

   The \verb+XL+ and \verb+YL+ values are changed accordingly.

   In the case of an image display, the \verb+XP+ and \verb+YP+ parameter values
   are changed.
   The image will {\bf always} be drawn without axis reversals.
}
}

\sstroutine{CURSOR}
{
   Find display coordinates using the cursor and print them at the terminal.
}{
   \sstparameters{
   \npar{None.}
}
\sstdescription{
   This command uses the graphics cursor to find coordinates on a
   displayed graph or image.

   Pressing the left or middle  mouse button displays information about the
   pixel being  pointed to.  Pressing the right mouse button terminates the
   \verb+CURSOR+ cycle.  The coordinates for each hit are printed on the
   terminal; they correspond to the unit scale of the axes prevailing on the
   current diagram,  ({\it{e.g.,}}\ (wavelength,  flux)).
   If meaningful,  additional coordinate information is also printed.
   Keyboard keys 1, 2 and 3 may be used in place of left, middle and right
   mouse buttons respectively.
}
}

\sstroutine{DRIMAGE}
{
   Display an IUE image on an suitable graphics workstation.
}{
   \sstparameters{
   \cpar{DATASET}{Dataset name.}
   \cpar{DEVICE}{GKS/SGS graphics device name.}
   \cpar{XP}{$x$-axis pixel limits, undefined or [0, 0] means full extent.}
   \cpar{YP}{$y$-axis pixel limits, undefined or [0, 0] means full extent.}
   \cpar{ZL}{Data limits for image display, undefined means full range.}
   \cpar{COLOUR}{Whether a false colour look-up table is used.}
   \cpar{ZONE}{Zone to be used for plotting.}
   \cpar{FLAG}{Whether data quality for faulty pixels are displayed.}
}
\sstdescription{
   This command displays the image specified by the \verb+DATASET+
   parameter on the device specified by the \verb+DEVICE+ parameter.

   The part of the image displayed is specified by the \verb+XP+ and \verb+YP+
   parameter values.
   If unspecified, \verb+XP+ and \verb+YP+ default to the entire image extent,
   {\it{i.e.}}

   \begin {quote}
      \verb+XP = [1,768], YP = [1,768]+
   \end {quote}

   If the values of \verb+XP+ or \verb+YP+ are specified in decreasing order,
   the image will {\bf not} be reversed along the appropriate axis.

   The range of data values displayed as a grey scale is limited
   by the two values of the \verb+ZL+ parameter.
   Data values at or below \verb+ZL[1]+ will appear {\bf black},
   those at \verb+ZL[2]+ will appear {\bf white} and those above \verb+ZL[2]+
   will appear {\bf blue}.
   If the \verb+ZL+ values are given in decreasing order, then high data
   values will be represented by low (dark) display intensities,
   and vice-versa.
   If the values are undefined, then the full intensity range of the
   image will be used.
   The full intensity range of the image can be found using the command

   \begin {quote}
      \verb+> SHOW V=I+
   \end {quote}

   The \verb+FLAG+ parameter specifies whether faulty pixels are flagged using
   the following colour scheme:

   \begin{description}
      \item GREEN --- pixels affected by reseau marks
      \item RED --- pixels which are saturated (DN=255)
      \item ORANGE --- pixels affected by ITF truncation
      \item YELLOW --- pixels marked bad by the user
   \end{description}

   If a pixel is affected by more than one of the above faults, then
   the first in the list is adopted for display.

   The \verb+ZONE+ parameter is accepted by \verb+DRIMAGE+ but is ignored, the
   display always using \verb+ZONE=0+\@.
}
}

\sstroutine{EDIMAGE}
{
   Edit the image data quality using the graphics cursor.
}{
   \sstparameters{
   \cpar{DATASET}{Dataset name.}
   \cpar{DEVICE}{GKS/SGS graphics device name.}
}
\sstdescription{
   This command uses the image display cursor to mark pixels and
   regions of the current image that are ``bad'' or ``good''.
   The image should have previously been displayed using the
   \verb+DRIMAGE+ command.
   So that faulty pixels can be seen, the \verb+FLAG=TRUE+ option in
   \verb+DRIMAGE+ should be used.

   The image display is specified by the \verb+DEVICE+ parameter and the
   associated dataset by the \verb+DATASET+ parameter.

   The following cursor hit sequences can be used in a cycle:

   \begin {description}
      \item 1 then 1 --- marks all pixels in the rectangle GOOD.
      \item 2 then 2 --- marks all points in the rectangle BAD.
      \item 1 --- marks the nearest pixel GOOD.
      \item 2 --- marks the nearest pixel BAD.
      \item 3 --- causes the cursor cycle to terminate.
   \end {description}

   Mouse buttons can be used for cursor hits where:

   \begin {description}
      \item {\bf left} mouse button     is  hit 1.
      \item {\bf middle} mouse button   is  hit 2.
      \item {\bf right} mouse button    is  hit 3.
   \end{description}

   Alternatively, keyboard keys 1, 2 and 3 can be used to mark hits.

   The pixels or ranges changed are printed on the terminal.
   The term ``rectangle'' is used above to indicate a rectangular
   set of pixels delineated by the two cursor positions.
   Thus, for the first hit, the cursor can be positioned at the
   bottom left corner, and for the second at the top right corner.

   Only the user-defined data quality bit can be changed by this
   command.
   Initially, all faulty pixels have this bit set BAD, so that
   spectrum extraction (say) can ignore these where appropriate.
   However, the user-defined data quality can also be set GOOD.

   See the IUEDR User Guide (MUD/45) for further information on data quality.
}
}

\sstroutine{EDMEAN}
{
   Edit the mean extracted spectrum using the graphics cursor.
}{
   \sstparameters{
   \cpar{DATASET}{Dataset name.}
   \cpar{DEVICE}{GKS/SGS graphics device name.}
}
\sstdescription{
   This command uses the graphics cursor to mark points and
   regions of the mean spectrum that are ``bad'' or ``good''.

   The following cursor hit sequences can be used in a cycle:

   \begin {description}
      \item 1 then 1 --- marks all points in the $x$-range GOOD.
      \item 2 then 2 --- marks all points in the $x$-range BAD.
      \item 1 --- marks the nearest point in $x$-direction GOOD.
      \item 2 --- marks the nearest point in $x$-direction BAD.
      \item 3 --- causes the cursor cycle to terminate.
   \end {description}

   The points or ranges changed are printed on the terminal.

   Mouse buttons can be used for cursor hits where:

   \begin {description}
      \item {\bf left} mouse button     is  hit 1.
      \item {\bf middle} mouse button   is  hit 2.
      \item {\bf right} mouse button    is  hit 3.
   \end{description}

   Alternatively, keyboard keys 1, 2 and 3 can be used to mark hits.

   See the IUEDR User Guide (MUD/45) for further information on data quality.
}
}

\sstroutine{EDSPEC}
{
   Edit the net extracted spectrum using the graphics cursor.
}{
   \sstparameters{
   \cpar{DATASET}{Dataset name.}
   \cpar{ORDER}{\'{E}chelle order number.}
   \cpar{APERTURE}{Aperture name (\verb+SAP+ or \verb+LAP+).}
   \cpar{DEVICE}{GKS/SGS graphics device name.}
}
\sstdescription{
   This command uses the graphics cursor to mark points and
   regions of the current net spectrum that are ``bad'' or ``good''.
   A plot of the \verb+APERTURE+ or \verb+ORDER+ spectrum is required before
   this command can be used.

   The following cursor hit sequences can be used in a cycle:

   \begin {description}
      \item 1 then 1 --- marks all points in the $x$-range GOOD.
      \item 2 then 2 --- marks all points in the $x$-range BAD.
      \item 1 --- marks the nearest point in $x$-direction GOOD.
      \item 2 --- marks the nearest point in $x$-direction BAD.
      \item 3 --- causes the cursor cycle to terminate.
   \end {description}

   The points or ranges changed are printed on the terminal.

   Mouse buttons can be used for cursor hits where:

   \begin {description}
      \item {\bf left} mouse button     is  hit 1.
      \item {\bf middle} mouse button   is  hit 2.
      \item {\bf right} mouse button    is  hit 3.
   \end{description}

   Alternatively, keyboard keys 1, 2 and 3 can be used to mark hits.

   Only the user-defined data quality bit can be changed by this
   command.
   Initially, all faulty points have this bit set BAD ({\it{e.g.,}}\ by
   \verb+TRAK+)\@. However, whether they are considered bad ({\it{e.g.,}}\ when
   plotting or creating output files) is determined by the user-defined
   bit, which can be changed at will.

   See the IUEDR User Guide (MUD/45) for further information on data quality.
}
}

\sstroutine{ERASE}
{
   Erase the display screen of the graphics device.
}{
   \sstparameters{
   \cpar{DEVICE}{GKS/SGS graphics device name.}
}
\sstdescription{
   The display screen of the specified graphics device is erased.
}
}

\sstroutine{EXIT}
{
   Quit IUEDR.
}{
   \sstparameters{
   \npar{None.}
}
\sstdescription{
   This command quits IUEDR\@.
   Any files that require new versions will be written by this command.
   This command is a synonym for the \verb+QUIT+ command.
}
}

\sstroutine{HELP}
{
   Find out about IUEDR commands and parameters.
}{
   \sstparameters{
   \npar{None.}
}
\sstdescription{
   By simply typing \verb+HELP+ \label{com: HELP }the user is presented with
   a brief introduction to IUEDR, a list of the commands available and some
   general information topics. Users familiar with he VMS help system will
   find this facility very essentially the same to use.

   The \verb+HELP+ system provides a list of topics which can be
   selected from by typing enough characters of a topic name to uniquely
   identify it and pressing return.  Pressing the return key with no topic
   chosen takes the \verb+HELP+ system back one topic-level.
   At any time, pressing the return key a few times will return you to the
   IUEDR prompt.

   You may optionally give a specific topic to the \verb+HELP+ command at the
   IUEDR prompt, for example
   \begin{quote}
      \verb+> HELP DRIMAGE+
   \end{quote}
   or even
   \begin{quote}
      \verb+> HELP DRIMAGE COLOUR+
   \end{quote}
}
}

\sstroutine{LBLS}
{
   Extracts a line-by-line-spectrum array from the image.
}{
   \sstparameters{
   \cpar{DATASET}{Dataset name.}
   \cpar{ORDER}{\'{E}chelle order number.}
   \cpar{APERTURE}{Aperture name (\verb+SAP+ or \verb+LAP+).}
   \cpar{GSAMP}{Spectrum grid sampling rate (geometric pixels).}
   \cpar{CUTWV}{Whether wavelength cutoff data used for extraction grid.}
   \cpar{CENTM}{Whether pre-existing centroid template is used.}
   \cpar{RL}{Limits across spectrum for LBLS array (pixels).}
   \cpar{RSAMP}{Radial coordinate sampling rate for LBLS grid (pixels).}
}
\sstdescription{
   This command creates a line-by-line-spectrum (LBLS) array from the
   image defined by \verb+DATASET+\@.
   The array consists of intensities $F(IR, I\lambda )$ for a grid of
   wavelengths, $W(I\lambda)$, and radial coordinates, $R(IR)$\@.
   The wavelength grid, $\lambda$, is determined in a similar way to the
   \verb+TRAK+ command, using the \verb+CUTWV+ (HIRES) and \verb+GSAMP+
   (HIRES/LORES) parameters.

   The radial coordinates are distances from the centre of the spectrum,
   derived from the template data,
   along a line perpendicular to the dispersion direction and
   measured in geometric pixels.
   The radial grid, $R$, is determined by the \verb+RL+ and \verb+RSAMP+
   parameters.

   The value of each pixel in the array corresponds to the surface
   over the image of a rectangle centred on its $(R, \lambda )$ coordinates,
   and extents

   \begin {equation}
      (R(IR) - dR / 2, R(IR) + dR / 2)
   \end {equation}
   and

   \begin {equation}
      (W(I\lambda ) - d\lambda / 2, W(I\lambda ) + d\lambda / 2)
   \end {equation}
   $dR$ is the distance between $R$ values, and $d\lambda$ is the wavelength
   step between $\lambda$ values.

   This surface integral is scaled along the $\lambda$ direction to
   correspond to an interval of 1.414 geometric pixels.
   The reason for this is to make LBLS intensities consistent with
   those produced by the \verb+TRAK+ command.
   For a particular wavelength, $W(I\lambda )$, the sum of LBLS intensities
   after removal of background should correspond to the net
   flux as measured by \verb+TRAK+\@.
}
}

\sstroutine{LISTIUE}
{
   Analyse the contents of IUE tapes or files.
}{
   \sstparameters{
   \cpar{DRIVE}{Tape drive or file name.}
   \cpar{FILE}{Tape file number.}
   \cpar{NFILE}{Number of tape files to be processed.}
   \cpar{NLINE}{Number of IUE header lines printed.}
   \cpar{SKIPNEXT}{Whether skip to next tape file.}
}
\sstdescription{
   This performs an analysis of \verb+NFILE+ IUE tape files, starting at
   the file specified by the \verb+FILE+ parameter.
   \verb+NFILE=-1+ means list all files until the end of the tape.
   \verb+NLINE=-1+ means print all lines in file header.

   \verb+LISTIUE+ can also be used to list the header of a GO format disk file.
}
}

\newpage
\sstroutine{MAP}
{
   Map and merge the extracted spectrum components to produce a mean spectrum.
}{
   \sstparameters{
   \cpar{DATASET}{Dataset name.}
   \cpar{ORDERS}{This delineates a range of \'{e}chelle orders.}
   \cpar{APERTURE}{Aperture name (\verb+SAP+ or \verb+LAP+).}
   \cpar{RM}{Whether mean spectrum is reset before averaging.}
   \cpar{ML}{Wavelength grid limits for mean spectrum.}
   \cpar{MSAMP}{Wavelength sampling rate for mean spectrum grid.}
   \cpar{FILLGAP}{Whether gaps can be filled within order.}
   \cpar{COVERGAP}{Whether gaps can be filled by covering orders.}
}
\sstdescription{
   This command can be used to produce a mean spectrum with contributions
   from several \'{e}chelle orders (HIRES), or from several apertures (LORES).

   If \verb+RM=TRUE+, or if there is no existing mean spectrum, then an
   evenly spaced wavelength grid is constructed between the
   limits specified by the \verb+ML+ parameter using the sampling rate
   specified by the \verb+MSAMP+ parameter.

   If \verb+RM=FALSE+ and there {\bf is}
   an existing mean spectrum, then the
   wavelength grid {\bf and contents} are retained.
   New components will be averaged with what is already there.

   In the case of HIRES, the \verb+ORDERS+ parameter is used to delimit the
   range of \'{e}chelle orders that are allowed to contribute to the mean.

   In the case of LORES, only a single aperture specified by the
   \verb+APERTURE+ parameter is mapped at a given time.
}
}

\sstroutine{MODIMAGE}
{
   Modifies image pixel intensities interactively.
}{
   \sstparameters{
   \cpar{DATASET}{Dataset name.}
   \cpar{DEVICE}{GKS/SGS graphics device name.}
   \cpar{FN}{Replacement Flux Number for pixel.}
}
\sstdescription{
   This command uses the image display cursor to modify image data.
   The image should already have been displayed using the \verb+DRIMAGE+
   command.

   The following cursor sequences are adopted:

   \begin {description}
      \item 1 then 2 --- copy intensity of first picked pixel to the second.
      \item 2 --- prompt for replacement pixel intensity.
      \item 3 --- finish.
   \end {description}

   Mouse buttons can be used for cursor hits where:

   \begin {description}
      \item {\bf left} mouse button     is  hit 1.
      \item {\bf middle} mouse button   is  hit 2.
      \item {\bf right} mouse button    is  hit 3.
   \end{description}

   Alternatively, keyboard keys 1, 2 and 3 can be used to mark hits.

   If the data or data qualities change after a session, then the file is
   saved on disk.

   The assumption is made that the current image displayed corresponds
   to the current dataset!
}
}

\sstroutine{MTMOVE}
{
   Move to the start of a tape file.
}{
   \sstparameters{
   \cpar{DRIVE}{Tape drive.}
   \cpar{FILE}{Tape file number.}
}
\sstdescription{
   Move to the start of the file specified by the \verb+FILE+ parameter on the
   tape specified by the \verb+DRIVE+ parameter.
}
}

\sstroutine{MTREW}
{
   Rewind to the start of the tape.
}{
   \sstparameters{
   \cpar{DRIVE}{Tape drive.}
}
\sstdescription{
   This command rewinds the tape specified by the \verb+DRIVE+ parameter.
   The \verb+FILE+ parameter is also set to 1 by this command.
}
}

\sstroutine{MTSHOW}
{
   Show the current tape position.
}{
   \sstparameters{
   \cpar{DRIVE}{Tape drive.}
}
\sstdescription{
   This command displays the current tape position.
   This includes the file number and the block position relative to either
   the start or the end of the file.

   Note that the actual file position may differ from the
   value of the \verb+FILE+ parameter.
}
}

\sstroutine{MTSKIPEOV}
{
   Skip over end-of-volume (EOV) mark.
}{
   \sstparameters{
   \cpar{DRIVE}{Tape drive.}
}
\sstdescription{
   This command skips over an end-of-volume (EOV) mark on the tape specified
   by the DRIVE parameter.
   An EOV condition is where there are two consecutive tape marks.
   When attempting to skip across an EOV, an error will be reported
   and the tape left positioned between the two marks.
   Subsequent attempts to skip forward will fail and
   only this command can be used to move forward beyond the
   second tape mark.
}
}

\sstroutine{MTSKIPF}
{
   Skip over NSKIP tape marks.
}{
   \sstparameters{
   \cpar{DRIVE}{Tape drive.}
   \cpar{NSKIP}{Number of tape marks to be skipped over.}
}
\sstdescription{
   This command skips over \verb+NSKIP+ tape marks on the tape specified
   by the \verb+DRIVE+ parameter.
   If \verb+NSKIP+ is negative this means that tape marks are skipped in the
   reverse direction, {\it{i.e.}}\ towards the start of the tape.
}
}

\sstroutine{NEWABS}
{
   Associate a new absolute flux calibration with the current dataset.
}{
   \sstparameters{
   \cpar{DATASET}{Dataset name.}
   \cpar{ABSFILE}{Name of file containing absolute flux calibration.}
}
\sstdescription{
   This command reads the absolute flux calibration from a text file
   specified by the \verb+ABSFILE+ parameter and stores it in the dataset
   specified by \verb+DATASET+\@.

   The file type is assumed to be \verb+.abs+ and need not be
   specified as part of the \verb+ABSFILE+ parameter.

   The calibration of any current spectrum is automatically updated.
}
}

\newpage
\sstroutine{NEWCUT}
{
   Associate new \'{e}chelle order wavelength limits with the current
   dataset.
}{
   \sstparameters{
   \cpar{DATASET}{Dataset name.}
   \cpar{CUTFILE}{Name of file containing \'{e}chelle order wavelength limits.}
}
\sstdescription{
   This command reads the \'{e}chelle order wavelength limits from a text file
   specified by the \verb+CUTFILE+ parameter and stores them in the dataset
   specified by \verb+DATASET+\@.

   The file type is assumed to be \verb+.cut+ and need not be
   specified as part of the \verb+CUTFILE+ parameter.

   The calibration of any current spectrum is automatically updated.
}
}

\sstroutine{NEWDISP}
{
   Associate new spectrograph dispersion data with the current dataset.
}{
   \sstparameters{
   \cpar{DATASET}{Dataset name.}
   \cpar{DISPFILE}{Name of file containing dispersion data.}
}
\sstdescription{
   This command reads spectrograph dispersion data from the text file
   specified by the \verb+DISPFILE+ parameter and stores them in the dataset
   specified by \verb+DATASET+\@.

   The file type is assumed to be \verb+.dsp+ and need not be specified
   as part of the \verb+DISPFILE+ parameter.
}
}

\sstroutine{NEWFID}
{
   Read IUE fiducial positions from text file.
}{
   \sstparameters{
   \cpar{DATASET}{Dataset name.}
   \cpar{FIDFILE}{Name of file containing fiducial positions.}
   \cpar{NGEOM}{Number of Chebyshev terms used to represent geometry.}
}
\sstdescription{
   This command reads IUE fiducial positions from a text file
   specified by the \verb+FIDFILE+ parameter and stores them in the dataset
   specified by \verb+DATASET+\@.

   The file type is assumed to be \verb+.fid+ and need not be specified
   as part of the FIDFILE parameter.

   The image data quality and geometry representation are updated to
   account for any changes that these fiducial positions imply.
   In the case of datasets containing image distortion, the \verb+NGEOM+
   parameter is used to specify the number of terms used for the Chebyshev
   representation along each axis.
}
}

\sstroutine{NEWRIP}
{
   Read \'{e}chelle ripple calibration from text file.
}{
   \sstparameters{
   \cpar{DATASET}{Dataset name.}
   \cpar{RIPFILE}{Name of file containing \'{e}chelle ripple calibration.}
}
\sstdescription{
   This command reads an \'{e}chelle ripple calibration from a text file
   specified by the \verb+RIPFILE+ parameter and stores it in the dataset
   specified by \verb+DATASET+\@.

   The file type is assumed to be \verb+.rip+ and need not be specified
   as part of the \verb+RIPFILE+ parameter.

   The calibration of any current spectrum is automatically updated.
}
}

\sstroutine{NEWTEM}
{
   Read spectrum centroid template data from text file.
}{
   \sstparameters{
   \cpar{DATASET}{Dataset name.}
   \cpar{TEMFILE}{Name of file containing spectrum template data.}
}
\sstdescription{
   This command reads the spectrum centroid template data into \verb+DATASET+
   from a text file with name specified by \verb+TEMFILE+\@.

   The file type is assumed to be \verb+.tem+ and need not be specified
   as part of the \verb+TEMFILE+\@.
}
}

\sstroutine{OUTEM}
{
   Output the current spectrum template data to a formatted data file.
}{
   \sstparameters{
   \cpar{DATASET}{Dataset name.}
   \cpar{TEMFILE}{Name of file containing spectrum template data.}
}
\sstdescription{
   This command outputs the templates stored with the current dataset to a
   text file.
   If not specified, the file name is constructed as:

   \begin {quote}
      \verb+<CAMERA>HI<APERTURE>.TEM+
   \end {quote}
   or

   \begin {quote}
      \verb+<CAMERA>LO.TEM+
   \end {quote}

   for the HIRES and LORES cases respectively.
}
}

\sstroutine{OUTLBLS}
{
   Output the current LBLS array to a binary data file.
}{
   \sstparameters{
   \cpar{DATASET}{Dataset name.}
   \cpar{OUTFILE}{Name of output file.}
}
\sstdescription{
   This command outputs the current LBLS array to a file.
   If not specified by the \verb+OUTFILE+ parameter, the file name is
   constructed as:

   \begin {quote}
      \verb+<CAMERA><IMAGE>R.DAT+
   \end {quote}
   The format of this file is described by the Fortran 77 routine, RDLBLS, which
   can be found in the file:

   \begin {quote}
      {\tt \$IUEDR\_USER/rdlbls.for}
   \end {quote}
   The directory {\tt \$IUEDR\_USER} also contains a test
   program for using RDLBLS and other helpful items.
}
}

\sstroutine{OUTMEAN}
{
   Output current mean spectrum to a DIPSO SP format file.
}{
   \sstparameters{
   \cpar{DATASET}{Dataset name.}
   \cpar{OUTFILE}{Name of output file.}
   \cpar{SPECTYPE}{DIPSO SP file type (0, 1 or 2).}
}
\sstdescription{
   This command outputs the mean spectrum associated with
   \verb+DATASET+ to a file that can be read into DIPSO
   (see \xref{SUN/50}{sun50}{}).

   This file is created with type specified by the \verb+SPECTYPE+ parameter
   (see Section~\ref{se:spectrum}
   for SP options).
   If not specified, the file name is constructed as:

   \begin {quote}
      \verb+<CAMERA><IMAGE>M.sdf+ \hspace*{8mm} for SP0\\
      \verb+<CAMERA><IMAGE>M.DAT+ \hspace*{8mm} for SP1 or SP2
   \end {quote}
   In DIPSO SP format, bad points are indicated by having zero intensities.
   In determining which points in the output file are to be marked
   ``bad'', the user-defined data quality bit is used.
   Since this bit can be arbitrarily edited,
   faulty data values can be written to the output file
   without subsequent information being retained.
}
}

\sstroutine{OUTNET}
{
   Output the current net spectrum to a DIPSO  SP format data file.
}{
   \sstparameters{
   \cpar{DATASET}{Dataset name.}
   \cpar{APERTURE}{Aperture name (\verb+SAP+ or \verb+LAP+).}
   \cpar{ORDER}{\'{E}chelle order number.}
   \cpar{OUTFILE}{Name of output file.}
   \cpar{SPECTYPE}{DIPSO SP file type (0, 1 or 2).}
}
\sstdescription{
   This command outputs the net spectrum associated with \verb+ORDER+ or
   \verb+APERTURE+ and \verb+DATASET+ to a file that can be read into DIPSO
   (see \xref{SUN/50}{sun50}{}).

   The file is created with type specified
   by the \verb+SPECTYPE+ parameter (see Section~\ref{se:spectrum}
   for SP
   options).
   If not specified, the file name is constructed as:

   \begin {quote}
      \verb+<CAMERA><IMAGE>+\_\verb+<APERTURE>.sdf+ \hspace*{8mm} for SP0\\
      \verb+<CAMERA><IMAGE>.<APERTURE>+ \hspace*{8mm} for SP1 or SP2
   \end {quote}
   in the case of LORES and

   \begin {quote}
      \verb+<CAMERA><IMAGE>+\_\verb+<ORDER>.sdf+ \hspace*{8mm} for SP0\\
      \verb+<CAMERA><IMAGE>.<ORDER>+ \hspace*{8mm} for SP1 or SP2
   \end {quote}
   in the case of HIRES.
   Here, \verb+<APERTURE>+ is the aperture name (\verb+SAP+ or \verb+LAP+),
   or index, and \verb+<ORDER>+ is the \'{e}chelle order number.

   In DIPSO SP format, bad points are indicated by having zero intensities.
   In determining which points in the output file are to be marked
   ``bad'', the user-defined data quality bit is used.
   Since this bit can be arbitrarily edited,
   faulty data values can be written to the output file
   without subsequent information being retained.
}
}

\sstroutine{OUTRAK}
{
   Output the current uncalibrated spectrum to a ``TRAK'' formatted data file.
}{
   \sstparameters{
   \cpar{DATASET}{Dataset name.}
   \cpar{OUTFILE}{Name of output file.}
}
\sstdescription{
   This command outputs the uncalibrated spectrum associated with
   \verb+DATASET+ to a formatted file that is compatible with output
   from the old ``TRAK'' program.
   The default file name is of the form:

   \begin {quote}
      \verb+<CAMERA><IMAGE>.TRK+
   \end {quote}
   The main difference from an actual ``TRAK'' file is that the background
   level is uniformly zero, so that GROSS=NET.
}
}

\sstroutine{OUTSCAN}
{
   Output the current scan data to a DIPSO  SP format data file.
}{
   \sstparameters{
   \cpar{DATASET}{Dataset name.}
   \cpar{OUTFILE}{Name of output file.}
   \cpar{SPECTYPE}{DIPSO SP file type (0, 1 or 2).}
}
\sstdescription{
   This command outputs the current scan associated with
   \verb+DATASET+ to a file which can be read into DIPSO
   (see \xref{SUN/50}{sun50}{}).

   The file is created with type specified
   by the \verb+SPECTYPE+ parameter
   (see Section~\ref{se:spectrum}
   for SP options).
   If not specified, the file name is constructed as:

   \begin {quote}
      \verb+<CAMERA><IMAGE>P.sdf+ \hspace*{8mm} for SP0\\
      \verb+<CAMERA><IMAGE>P.DAT+ \hspace*{8mm} for SP1 or SP2
   \end {quote}
   In DIPSO SP format, bad points are indicated by having zero intensities.
   In determining which points in the output file are to be marked
   ``bad'', the user-defined data quality bit is used.
   Since this bit can be arbitrarily edited,
   faulty data values can be written to the output file
   without subsequent information being retained.
}
}

\sstroutine{OUTSPEC}
{
   Output the current aperture or order spectrum to a DIPSO SP format data file.
}{
   \sstparameters{
   \cpar{DATASET}{Dataset name.}
   \cpar{APERTURE}{Aperture name (\verb+SAP+ or \verb+LAP+).}
   \cpar{ORDER}{\'{E}chelle order number.}
   \cpar{OUTFILE}{Name of output file.}
   \cpar{SPECTYPE}{DIPSO SP file type (0, 1 or 2).}
}
\sstdescription{
   This command outputs the spectrum associated with the \verb+ORDER+ or
   \verb+APERTURE+ and \verb+DATASET+ to a file which can be read into DIPSO
   (see \xref{SUN/50}{sun50}{}).

   The file is created with type specified
   by the \verb+SPECTYPE+ parameter
   (see Section~\ref{se:spectrum}
   for SP options).
   If not specified, the file name is constructed as:

   \begin {quote}
      \verb+<CAMERA><IMAGE>+\_\verb+<APERTURE>.sdf+ \hspace*{8mm} for SP0\\
      \verb+<CAMERA><IMAGE>.<APERTURE>+ \hspace*{8mm} for SP1 or SP2
   \end {quote}
   in the case of LORES and

   \begin {quote}
      \verb+<CAMERA><IMAGE>+\_\verb+<ORDER>.sdf+ \hspace*{8mm} for SP0\\
      \verb+<CAMERA><IMAGE>.<ORDER>+ \hspace*{8mm} for SP1 or SP2
   \end {quote}
   in the case of HIRES.
   Here, \verb+<APERTURE>+ is the aperture name (\verb+SAP+ or \verb+LAP+), or
   index, and \verb+<ORDER>+ is the \'{e}chelle order number.

   In DIPSO SP format, bad points are indicated by having zero intensities.
   In determining which points in the output file are to be marked
   ``bad'', the user-defined data quality bit is used.
   Since this bit can be arbitrarily edited,
   faulty data values can be written to the output file
   without subsequent information being retained.
}
}

\sstroutine{PLCEN}
{
   Plot smoothed centroid shifts.
}{
   \sstparameters{
   \cpar{DATASET}{Dataset name.}
   \cpar{ORDER}{\'{E}chelle order number.}
   \cpar{APERTURE}{Aperture name (\verb+SAP+ or \verb+LAP+).}
   \cpar{RS}{Whether display is reset before plotting.}
   \cpar{DEVICE}{GKS/SGS graphics device name.}
   \cpar{ZONE}{Zone to be used for plotting.}
   \cpar{LINE}{Plotting line style (\verb+SOLID+, \verb+DASH+, \verb+DOTDASH+
               or \verb+DOT+).}
   \cpar{LINEROT}{Whether line style is changed after next plot.}
   \cpar{COL}{Plotting line colour (1, 2, 3, \dots 10).}
   \cpar{COLROT}{Whether line colour is changed after next plot.}
   \cpar{XL}{$x$-axis plotting limits, undefined or [0, 0] means auto-scale.}
   \cpar{YL}{$y$-axis plotting limits, undefined or [0, 0] means auto-scale.}
}
\sstdescription{
   This command plots the smoothed centroid shifts produced during
   the most recent spectrum extraction from \verb+DATASET+
   on the graphics device and zone specified by the \verb+DEVICE+ and
   \verb+ZONE+ parameters respectively.

   In the case of a LORES spectrum, if there is more than a single
   aperture available, then the \verb+APERTURE+ parameter needs to be specified.

   In the case of a HIRES spectrum, if there is more than a single
   \'{e}chelle order, then the \verb+ORDER+ parameter needs to be specified.

   The \verb+RS+ parameter specifies whether a new plot is started, or whether
   the data can be plotted over an existing plot.

   The \verb+LINE+ and \verb+LINEROT+ parameters determine the line style
   which will be used for plotting.

   The \verb+COL+ and \verb+COLROT+ parameters determine the line colour which
   will be used for plotting if the \verb+DEVICE+ supports colour graphics.

   The diagram limits are specified by the \verb+XL+ and \verb+YL+ parameter
   values.
   If \verb+XL+ and \verb+YL+ have values

   \begin {quote}
      \verb+XL=[0,0], YL=[0,0]+
   \end {quote}
   then the plot limits along each axis are determined so that the whole
   spectrum is visible.
   If the values of \verb+XL+ or \verb+YL+ are specified in decreasing order,
   then the coordinates will be reversed along the appropriate axis.
}
}

\sstroutine{PLFLUX}
{
   Plot calibrated flux spectrum.
}{
   \sstparameters{
   \cpar{DATASET}{Dataset name.}
   \cpar{ORDER}{\'{E}chelle order number.}
   \cpar{APERTURE}{Aperture name (\verb+SAP+ or \verb+LAP+).}
   \cpar{RS}{Whether display is reset before plotting.}
   \cpar{DEVICE}{GKS/SGS graphics device name.}
   \cpar{ZONE}{Zone to be used for plotting.}
   \cpar{LINE}{Plotting line style (\verb+SOLID+, \verb+DASH+, \verb+DOTDASH+
               or \verb+DOT+).}
   \cpar{LINEROT}{Whether line style is changed after next plot.}
   \cpar{COL}{Plotting line colour (1, 2, 3, \dots 10).}
   \cpar{COLROT}{Whether line colour is changed after next plot.}
   \cpar{HIST}{Whether lines are drawn as histograms.}
   \cpar{QUAL}{Whether data quality information is plotted.}
   \cpar{XL}{$x$-axis plotting limits, undefined or [0, 0] means auto-scale.}
   \cpar{YL}{$y$-axis plotting limits, undefined or [0, 0] means auto-scale.}
}
\sstdescription{
   This command plots the calibrated flux spectrum from \verb+DATASET+
   on the graphics device and zone specified by the \verb+DEVICE+ and
   \verb+ZONE+ parameters respectively.

   In the case of a LORES spectrum, if there is more than a single
   aperture available, then the \verb+APERTURE+ parameter needs to be specified.

   In the case of a HIRES spectrum, if there is more than a single
   \'{e}chelle order, then the \verb+ORDER+ parameter needs to be specified.

   The \verb+RS+ parameter specifies whether a new plot is started, or whether
   the data can be plotted over an existing plot.

   The \verb+HIST+ parameter determines whether the line is drawn as a
   histogram rather than a continuous polyline.

   The \verb+LINE+ and \verb+LINEROT+ parameters determine the line style
   which will be used for plotting.

   The \verb+COL+ and \verb+COLROT+ parameters determine the line colour which
   will be used for plotting if the \verb+DEVICE+ supports colour graphics.

   The diagram limits are specified by the \verb+XL+ and \verb+YL+ parameter
   values.
   If \verb+XL+ and \verb+YL+ have values

   \begin {quote}
      \verb+XL=[0,0], YL=[0,0]+
   \end {quote}
   then the plot limits along each axis are determined so that the whole
   spectrum is visible.
   If the values of \verb+XL+ or \verb+YL+ are specified in decreasing order,
   then the coordinates will be reversed along the appropriate axis.

   The \verb+QUAL+ parameter indicates whether faulty points are flagged with
   their data quality codes (see Section~\ref{se:introduction}).

   If a point is affected by more than one of the above faults, then
   the highest code is plotted.
   Points marked bad by user edits are only indicated if they are otherwise
   fault-free.
}
}

\sstroutine{PLGRS}
{
   Plot pseudo-gross and background from spectrum extraction.
}{
   \sstparameters{
   \cpar{DATASET}{Dataset name.}
   \cpar{ORDER}{\'{E}chelle order number.}
   \cpar{APERTURE}{Aperture name (\verb+SAP+ or \verb+LAP+).}
   \cpar{RS}{Whether display is reset before plotting.}
   \cpar{DEVICE}{GKS/SGS graphics device name.}
   \cpar{ZONE}{Zone to be used for plotting.}
   \cpar{LINE}{Plotting line style (\verb+SOLID+, \verb+DASH+, \verb+DOTDASH+
               or \verb+DOT+).}
   \cpar{LINEROT}{Whether line style is changed after next plot}
   \cpar{COL}{Plotting line colour (1, 2, 3, \ldots 10).}
   \cpar{COLROT}{Whether line colour is changed after next plot.}
   \cpar{HIST}{Whether lines are drawn as histograms.}
   \cpar{QUAL}{Whether data quality information is plotted.}
   \cpar{XL}{$x$-axis plotting limits, undefined or [0, 0] means auto-scale.}
   \cpar{YL}{$y$-axis plotting limits, undefined or [0, 0] means auto-scale.}
}
\sstdescription{
   This command plots the pseudo-gross and smooth background produced during
   the most recent spectrum extraction from \verb+DATASET+
   on the graphics device and zone specified by the \verb+DEVICE+ and
   \verb+ZONE+ parameters respectively.

   The pseudo-gross is constructed by taking the net spectrum and adding
   the smooth background multiplied by the width of the object channel.
   The smooth background plotted is also for the object channel width.

   In the case of a LORES spectrum, if there is more than a single
   aperture available, then the \verb+APERTURE+ parameter needs to be specified.

   In the case of a HIRES spectrum, if there is more than a single
   \'{e}chelle order, then the \verb+ORDER+ parameter needs to be specified.

   The \verb+RS+ parameter specifies whether a new plot is started, or whether
   the data can be plotted over an existing plot.

   The \verb+HIST+ parameter determines whether the line is drawn as a histogram
   rather than a continuous polyline.

   The \verb+LINE+ and \verb+LINEROT+ parameters determine the line style
   which will be used for plotting.

   The \verb+COL+ and \verb+COLROT+ parameters determine the line colour which
   will be used for plotting if the \verb+DEVICE+ supports colour graphics.

   The diagram limits are specified by the \verb+XL+ and \verb+YL+ parameter
   values.
   If \verb+XL+ and \verb+YL+ have values

   \begin {quote}
      \verb+XL=[0,0], YL=[0,0]+
   \end {quote}
   then the plot limits along each axis are determined so that the whole
   spectrum is visible.
   If the values of \verb+XL+ or \verb+YL+ are specified in decreasing order,
   then the coordinates will be reversed along the appropriate axis.

   The \verb+QUAL+ parameter indicates whether faulty points are flagged with
   their data quality codes (see Section~\ref{se:introduction}).

   If a point is affected by more than one of the above faults, then
   the highest code is plotted.
   Points marked bad by user edits are only indicated if they are otherwise
   fault-free.
}
}

\sstroutine{PLMEAN}
{
   Plot mean spectrum.
}{
   \sstparameters{
   \cpar{DATASET}{Dataset name.}
   \cpar{RS}{Whether display is reset before plotting.}
   \cpar{DEVICE}{GKS/SGS graphics device name.}
   \cpar{ZONE}{Zone to be used for plotting.}
   \cpar{LINE}{Plotting line style (\verb+SOLID+, \verb+DASH+, \verb+DOTDASH+
               or \verb+DOT+).}
   \cpar{LINEROT}{Whether line style is changed after next plot}
   \cpar{COL}{Plotting line colour (1, 2, 3, \ldots 10).}
   \cpar{COLROT}{Whether line colour is changed after next plot.}
   \cpar{HIST}{Whether lines are drawn as histograms.}
   \cpar{QUAL}{Whether data quality information is plotted.}
   \cpar{XL}{$x$-axis plotting limits, undefined or [0, 0] means auto-scale.}
   \cpar{YL}{$y$-axis plotting limits, undefined or [0, 0] means auto-scale.}
}
\sstdescription{
   This command plots the mean spectrum associated with \verb+DATASET+ on the
   graphics device and zone specified by the \verb+DEVICE+ and \verb+ZONE+
   parameters respectively.

   The \verb+RS+ parameter specifies whether a new plot is started, or whether
   the data can be plotted over an existing plot.

   The \verb+HIST+ parameter determines whether the line is drawn as a histogram
   rather than a continuous polyline.

   The \verb+LINE+ and \verb+LINEROT+ parameters determine the line style which
   will be used for plotting.

   The \verb+COL+ and \verb+COLROT+ parameters determine the line colour which
   will be used for plotting if the \verb+DEVICE+ supports colour graphics.

   The diagram limits are specified by the \verb+XL+ and \verb+YL+ parameter
   values.
   If \verb+XL+ and \verb+YL+ have values

   \begin {quote}
      \verb+XL=[0,0], YL=[0,0]+
   \end {quote}
   then the plot limits along each axis are determined so that the whole
   spectrum is visible.
   If the values of \verb+XL+ or \verb+YL+ are specified in decreasing order,
   then the coordinates will be reversed along the appropriate axis.

   The \verb+QUAL+ parameter indicates whether faulty points are flagged with
   their data quality codes (see Section~\ref{se:introduction}).

   If a point is affected by more than one of the above faults, then
   the highest code is plotted.
   Points marked bad by user edits are only indicated if they are otherwise
   fault-free.
}
}

\sstroutine{PLNET}
{
   Plot uncalibrated net spectrum.
}{
   \sstparameters{
   \cpar{DATASET}{Dataset name.}
   \cpar{ORDER}{\'{E}chelle order number.}
   \cpar{APERTURE}{Aperture name (\verb+SAP+ or \verb+LAP+).}
   \cpar{RS}{Whether display is reset before plotting.}
   \cpar{DEVICE}{GKS/SGS graphics device name.}
   \cpar{ZONE}{Zone to be used for plotting.}
   \cpar{LINE}{Plotting line style (\verb+SOLID+, \verb+DASH+, \verb+DOTDASH+
               or \verb+DOT+).}
   \cpar{LINEROT}{Whether line style is changed after next plot}
   \cpar{COL}{Plotting line colour (1, 2, 3, \ldots 10).}
   \cpar{COLROT}{Whether line colour is changed after next plot.}
   \cpar{HIST}{Whether lines are drawn as histograms.}
   \cpar{QUAL}{Whether data quality information is plotted.}
   \cpar{XL}{$x$-axis plotting limits, undefined or [0, 0] means auto-scale.}
   \cpar{YL}{$y$-axis plotting limits, undefined or [0, 0] means auto-scale.}
}
\sstdescription{
   This command plots the uncalibrated net spectrum specified by the
   \verb+DATASET+
   parameter on the graphics device and zone specified by the
   \verb+DEVICE+ and \verb+ZONE+ parameters respectively.

   In the case of a LORES spectrum, if there is more than a single
   aperture available, then the \verb+APERTURE+ parameter needs to be specified.

   In the case of a HIRES spectrum, if there is more than a single
   \'{e}chelle order, then the \verb+ORDER+ parameter needs to be specified.

   The \verb+RS+ parameter specifies whether a new plot is started, or whether
   the data can be plotted over an existing plot.

   The \verb+HIST+ parameter determines whether the line is drawn as a histogram
   rather than a continuous polyline.

   The \verb+LINE+ and \verb+LINEROT+ parameters determine the line style which
   will be used for plotting.

   The \verb+COL+ and \verb+COLROT+ parameters determine the line colour which
   will be used for plotting if the \verb+DEVICE+ supports colour graphics.

   The diagram limits are specified by the \verb+XL+ and \verb+YL+ parameter
   values.
   If \verb+XL+ and \verb+YL+ have values

   \begin {quote}
      \verb+XL=[0,0], YL=[0,0]+
   \end {quote}
   then the plot limits along each axis are determined so that the whole
   spectrum is visible.
   If the values of \verb+XL+ or \verb+YL+ are specified in decreasing order,
   then the coordinates will be reversed along the appropriate axis.

   The \verb+QUAL+ parameter indicates whether faulty points are flagged with
   their data quality codes (see Section~\ref{se:introduction}).

   If a point is affected by more than one of the above faults, then
   the highest code is plotted.
   Points marked bad by user edits are only indicated if they are otherwise
   fault-free.
}
}

\sstroutine{PLSCAN}
{
   Plot scan perpendicular to dispersion.
}{
   \sstparameters{
   \cpar{DATASET}{Dataset name.}
   \cpar{RS}{Whether display is reset before plotting.}
   \cpar{DEVICE}{GKS/SGS graphics device name.}
   \cpar{ZONE}{Zone to be used for plotting.}
   \cpar{LINE}{Plotting line style (\verb+SOLID+, \verb+DASH+, \verb+DOTDASH+
               or \verb+DOT+).}
   \cpar{LINEROT}{Whether line style is changed after next plot.}
   \cpar{COL}{Plotting line colour (1, 2, 3, \ldots 10).}
   \cpar{COLROT}{Whether line colour is changed after next plot.}
   \cpar{QUAL}{Whether data quality information is plotted.}
   \cpar{XL}{$x$-axis plotting limits, undefined or [0, 0] means auto-scale.}
   \cpar{YL}{$y$-axis plotting limits, undefined or [0, 0] means auto-scale.}
}
\sstdescription{
   This command plots the most recent scan perpendicular to dispersion
   associated with \verb+DATASET+ on the graphics device and zone specified by
   the \verb+DEVICE+ and \verb+ZONE+ parameters respectively.

   The \verb+RS+ parameter specifies whether a new plot is started, or whether
   the data can be plotted over an existing plot.

   The \verb+LINE+ and \verb+LINEROT+ parameters determine the line style which
   will be used for plotting.

   The \verb+COL+ and \verb+COLROT+ parameters determine the line colour which
   will be used for plotting if the \verb+DEVICE+ supports colour graphics.

   The diagram limits are specified by the \verb+XL+ and \verb+YL+ parameter
   values.
   If \verb+XL+ and \verb+YL+ have values

   \begin {quote}
      \verb+XL=[0,0], YL=[0,0]+
   \end {quote}
   then the plot limits along each axis are determined so that the whole
   spectrum is visible.
   If the values of \verb+XL+ or \verb+YL+ are specified in decreasing order,
   then the coordinates will be reversed along the appropriate axis.

   The \verb+QUAL+ parameter indicates whether faulty points are flagged with
   their data quality codes (see Section~\ref{se:introduction}).

   If a point is affected by more than one of the above faults, then
   the highest code is plotted.
   Points marked bad by user edits are only indicated if they are otherwise
   fault-free.
}
}

\newpage
\sstroutine{PRGRS}
{
   Print the current extracted aperture or order spectrum in tabular
   form.
}{
   \sstparameters{
   \cpar{DATASET}{Dataset name.}
   \cpar{APERTURE}{Aperture name (\verb+SAP+ or \verb+LAP+).}
}
\sstdescription{
   This command prints the recently extracted spectrum associated
   with \verb+ORDER+ or \verb+APERTURE+
   and \verb+DATASET+ in tabular form.
   The table consists of wavelengths, ``gross'', smooth background, net
   and calibrated fluxes, along
   with any data quality information.
   The ``gross'' and smooth background correspond to an image sample
   with width specified by the adopted extraction slit.

   The output from this command is likely to be too voluminous to read at the
   terminal, refering to the \verb+session.lis+ file may be easier.
}
}

\sstroutine{PRLBLS}
{
   Print the current LBLS array in tabular form.
}{
   \sstparameters{
   \cpar{DATASET}{Dataset name.}
}
\sstdescription{
   This command prints the current LBLS array in tabular form.

   Each line of the main table consists of a wavelength and a set of
   mapped image intensities (FN), corresponding to cells at distances, R
   (pixels), from spectrum centre.

   Any array cells which are affected by ``bad'' image pixels
   ({\it{e.g.,}}\ reseaux, saturation, etc.) have data quality values printed
   below them, the meaning of which is given at the start of the output.

   The output from this command is likely to be too voluminous to read at the
   terminal, refering to the \verb+session.lis+ file may be easier.
}
}

\sstroutine{PRMEAN}
{
   Print the current mean spectrum in tabular form.
}{
   \sstparameters{
   \cpar{DATASET}{Dataset name.}
}
\sstdescription{
   This command prints the mean spectrum associated with DATASET in tabular form.

   The table consists of wavelengths and calibrated fluxes, along
   with any data quality information.

   The output from this command is likely to be too voluminous to read at the
   terminal, refering to the \verb+session.lis+ file may be easier.
}
}

\sstroutine{PRSCAN}
{
   Print the intensities of the current image scan in tabular form.
}{
   \sstparameters{
   \cpar{DATASET}{Dataset name.}
}
\sstdescription{
   This command prints the scan associated with \verb+DATASET+ in tabular form.
   The table consists of wavelengths and net fluxes, along
   with any data quality information.

   The output from this command should be diverted to a file, since
   it is likely to be too voluminous to read at the terminal.
}
}

\sstroutine{PRSPEC}
{
   Print the current aperture or order spectrum in tabular form.
}{
   \sstparameters{
   \cpar{DATASET}{Dataset name.}
   \cpar{APERTURE}{Aperture name (\verb+SAP+ or \verb+LAP+).}
   \cpar{ORDER}{\'{E}chelle order number.}
}
\sstdescription{
   This command prints the spectrum associated with \verb+DATASET+ and either
   \verb+ORDER+ or \verb+APERTURE+ in tabular form.

   The table consists of wavelengths, net and calibrated fluxes, along
   with any data quality information.

   The output from this command is likely to be too voluminous to read at the
   terminal, refering to the \verb+session.lis+ file may be easier.
}
}

\sstroutine{QUIT}
{
   Quit IUEDR.
}{
   \sstparameters{
   \npar{None.}
}
\sstdescription{
   This command quits IUEDR.
   Any files that require new versions will be written by this
   command.
   This command is a synonym for the \verb+EXIT+ command.
}
}

\newpage
\sstroutine{READIUE}
{
   Read a RAW, GPHOT or PHOT IUE image from GO format tape or file.
}{
   \sstparameters{
   \cpar{DRIVE}{Tape drive or file name.}
   \cpar{FILE}{Tape file number.}
   \cpar{NLINE}{Number of IUE header lines printed.}
   \cpar{DATASET}{Dataset name.}
   \cpar{TYPE}{Dataset type (\verb+RAW+, \verb+PHOT+ or \verb+GPHOT+).}
   \cpar{OBJECT}{Object identification text.}
   \cpar{CAMERA}{Camera name (\verb+LWP+, \verb+LWR+ or \verb+SWP+).}
   \cpar{IMAGE}{Image number.}
   \cpar{APERTURES}{Aperture name.}
   \cpar{RESOLUTION}{Spectrograph resolution mode (\verb+LORES+ or
                     \verb+HIRES+).}
   \cpar{EXPOSURES}{Spectrum exposure time(s) (seconds).}
   \cpar{THDA}{IUE camera temperature (C).}
   \cpar{ITFMAX}{Pixel value on tape for ITF saturation.}
   \cpar{BADITF}{Whether bad LORES SWP ITF requires correction.}
   \cpar{YEAR}{Year number (A.D.).}
   \cpar{MONTH}{Month number (1-12).}
   \cpar{DAY}{Day number in Month.}
   \cpar{NGEOM}{Number of Chebyshev terms used to represent geometry.}
   \cpar{ITF}{This is the ITF generation used in the image calibration.}
}
\sstdescription{
   This command reads an IUE dataset (\verb+RAW+, \verb+GPHOT+ or \verb+PHOT+)
   from tape or from a GO format disk file.
   The \verb+DATASET+ parameter determines the names of files that will contain
   the various data components ({\it{e.g.,}}\ Calibration, Image \& Image Quality
   etc.)

   The \verb+ITFMAX+ and \verb+NGEOM+ parameters are only prompted for if the
   image data are geometrically and photometrically calibrated.
}
}


\newpage
\sstroutine{READSIPS}
{
   Read MELO/MEHI from IUESIPS\#1 or IUESIPS\#2 tape or file.
}{
   \sstparameters{
   \cpar{DRIVE}{Tape drive or file name.}
   \cpar{FILE}{Tape file number.}
   \cpar{NLINE}{Number of IUE header lines printed.}
   \cpar{DATASET}{Dataset name.}
   \cpar{OBJECT}{Object identification text.}
   \cpar{CAMERA}{Camera name (\verb+LWP+, \verb+LWR+ or \verb+SWP+).}
   \cpar{IMAGE}{Image number.}
   \cpar{APERTURES}{Aperture name.}
   \cpar{EXPOSURES}{Spectrum exposure time(s) (seconds).}
   \cpar{THDA}{IUE camera temperature (C).}
   \cpar{YEAR}{Year number (A.D.).}
   \cpar{MONTH}{Month number (1-12).}
   \cpar{DAY}{Day number in Month.}
   \cpar{ITF}{This is the ITF generation used in the image calibration.}
}
\sstdescription{
   This command reads the MELO/MEHI product from IUESIPS\#1 or IUESIPS\#2
   tape or file.  Operation is much like \verb+READIUE+, except that some
   parameters and associated information are not needed. Only calibration
   ({\tt .UEC}) and spectrum ({\tt \_UES.sdf}) files are created.
   The values for certain parameters may be obtained from the tape or file,
   in which case you will not be prompted for them.
}
}

\sstroutine{SAVE}
{
   Write new versions for any files that have had their contents
   changed during the current session.
}{
   \sstparameters{
   \npar{None.}
}
\sstdescription{
   This command overwrites dataset files that have had their contents
   changed during the current session.
   If there are no outstanding files then this command does nothing.
}
}

\newpage
\sstroutine{SCAN}
{
   This command performs a scan perpendicular to spectrograph
   dispersion.
}{
   \sstparameters{
   \cpar{DATASET}{Dataset name.}
   \cpar{ORDERS}{This delineates a range of \'{e}chelle orders.}
   \cpar{APERTURE}{Aperture name (\verb+SAP+ or \verb+LAP+).}
   \cpar{SCANDIST}{HIRES scan distance from camera faceplate centre}
   \cparc{(geometric pixels.)}
   \cpar{SCANAV}{Averaging filter HWHM for image scan}
   \cparc{(geometric pixels).}
   \cpar{SCANWV}{Central wavelength for LORES image scan (\AA).}
}
\sstdescription{
   This command performs a scan perpendicular to spectrograph dispersion.
   The scan is performed by folding pixels with a triangle function
   with HWHM of \verb+SCANAV+ geometric pixels along the dispersion direction.

   In the case of HIRES, the \verb+SCANDIST+ parameter determines the distance
   of the scan from the faceplate centre.

   In the case of LORES, the \verb+SCANWV+ parameter determines the central
   wavelength of the scan in Angstroms.

   The algorithm used to produce scan intensities is not very good
   and so quantitative results should not be sought from this command.
   Its sole intention lies in providing data for aligning the spectrum.
}
}

\sstroutine{SETA}
{
   Set dataset parameters that are APERTURE specific.
}{
   \sstparameters{
   \cpar{DATASET}{Dataset name.}
   \cpar{APERTURE}{Aperture name (\verb+SAP+ or \verb+LAP+).}
   \cpar{EXPOSURE}{Spectrum exposure time (seconds).}
   \cpar{FSCALE}{Flux scale factor.}
   \cpar{WSHIFT}{Constant wavelength shift (\AA).}
   \cpar{VSHIFT}{Velocity shift of detector relative to Sun (km/s).}
   \cpar{ESHIFT}{Global \'{e}chelle wavelength shift.}
   \cpar{GSHIFT}{Global shift of spectrum on image (geometric pixels).}
}
\sstdescription{
   This command allows changes to be made to dataset values which are
   specific to the specified \verb+APERTURE+\@.
   Items for which parameters are not specified retain their current
   values.
}
}

\sstroutine{SETD}
{
   Set dataset parameters which are independent of ORDER/APERTURE.
}{
   \sstparameters{
   \cpar{DATASET}{Dataset name.}
   \cpar{OBJECT}{Object identification text.}
   \cpar{THDA}{IUE camera temperature (C).}
   \cpar{FIDSIZE}{Half width of fiducials (pixels).}
   \cpar{BADITF}{Whether bad LORES ITF requires correction.}
   \cpar{NGEOM}{Number of Chebyshev terms used to represent geometry.}
   \cpar{RIPK}{\'{E}chelle ripple constant (\AA).}
   \cpar{RIPA}{Ripple function scale factor.}
   \cpar{XCUT}{Global \'{e}chelle wavelength clipping.}
   \cpar{HALTYPE}{The type of halation (order-overlap) correction used.}
   \cpar{HALC}{Halation correction constant (fraction of continuum).}
   \cpar{HALWC}{Wavelength for which the halation correction \verb+HALC+}
   \cparc{is defined in angstroms.}
   \cpar{HALW0}{Wavelength at which halation correction is zero (\AA).}
   \cpar{HALAV}{Averaging FWHM for halation correction (gometric pixels).}
}
\sstdescription{
   This command allows changes to be made to dataset values which are
   independent of any specific \verb+APERTURE+/\verb+ORDER+\@.
   Items for which parameters are not specified retain their current
   values.
}
}

\sstroutine{SETM}
{
   Set dataset parameters that are ORDER specific.
}{
   \sstparameters{
   \cpar{DATASET}{Dataset name.}
   \cpar{ORDER}{\'{E}chelle order number.}
   \cpar{RIPK}{\'{E}chelle ripple constant (\AA).}
   \cpar{RIPA}{Ripple function scale factor.}
   \cpar{RIPC}{Ripple function correction polynomial.}
   \cpar{WCUT}{Wavelength limits for \'{e}chelle order (\AA).}
}
\sstdescription{
   This command allows changes to be made to dataset values which are
   specific to the specified \verb+ORDER+\@.
   Items for which parameters are not specified retain their current
   values.
}
}

\newpage
\sstroutine{SGS}
{
   Write names of available SGS devices at the terminal.
}{
   \sstparameters{
   \npar{None.}
}
\sstdescription{
   This commands writes a list of available SGS device names at the terminal.
   See \xref{SUN/85}{sun85}{} for details of the SGS graphics system.
}
}

\sstroutine{SHOW}
{
   Print dataset values.
}{
   \sstparameters{
   \cpar{DATASET}{Dataset name.}
   \cpar{V}{List of items to be printed.}
}
\sstdescription{
   This command shows the values of parameters in the dataset specified
   by the \verb+DATASET+ parameter. The items to be printed are specified by
   the \verb+V+ parameter, which is a string containing any of the following
   characters:

   \begin {description}
      \item H --- Header and file information
      \item I --- Image details
      \item F --- Fiducials
      \item G --- Geometry
      \item D --- Dispersion
      \item C --- Centroid templates
      \item R --- \'{E}chelle Ripple and halation
      \item A --- Absolute calibration
      \item S --- Raw Spectrum
      \item M --- Mean spectrum
      \item * --- All of the above
      \item Q --- Image data Quality summary
   \end {description}

   The \verb+*+ character needs to be placed within inverted commas.

   The \verb+V+ parameter is cancelled afterwards.
}
}

\sstroutine{TRAK}
{
   Extract spectrum from image.
}{
   \sstparameters{
   \cpar{DATASET}{Dataset name.}
   \cpar{ORDER}{\'{E}chelle order number.}
   \cpar{APERTURE}{Aperture name (\verb+SAP+ or \verb+LAP+).}
   \cpar{NORDER}{Number of \'{e}chelle orders to be processed.}
   \cpar{AUTOSLIT}{Whether \verb+GSLIT+, \verb+BDIST+ and \verb+BSLIT+
                   parameters are}
   \cparc{determined automatically.}
   \cpar{GSLIT}{Object channel limits (geometric pixels).}
   \cpar{BSLIT}{Background channel half widths (geometric pixels).}
   \cpar{BDIST}{Distances of background channels from object channel}
   \cparc{centre (geometric pixels).}
   \cpar{GSAMP}{Spectrum grid sampling rate (geometric pixels).}
   \cpar{CUTWV}{Whether wavelength cutoff data used for extraction grid.}
   \cpar{BKGIT}{Number of background smoothing iterations.}
   \cpar{BKGAV}{Background averaging filter FWHM (geometric pixels).}
   \cpar{BKGSD}{Discrimination level for background pixels (s.d.).}
   \cpar{CENTM}{Whether pre-existing centroid template is used.}
   \cpar{CENSH}{Whether the spectrum template is just shifted linearly.}
   \cpar{CENSV}{Whether the spectrum template is saved in the dataset.}
   \cpar{CENIT}{Number of centroid tracking iterations.}
   \cpar{CENAV}{Centroid averaging filter FWHM (geometric pixels).}
   \cpar{CENSD}{Significance level for signal to be used for centroids
                (s.d.).}
   \cpar{EXTENDED}{Whether the object is not a point source.}
   \cpar{CONTINUUM}{Whether the object spectrum is expected to have a}
   \cparc{``continuum''.}
}
\sstdescription{
   This command extracts a spectrum from an image.
   It does this by defining an evenly spaced wavelength grid along the
   spectrum, and mapping pixel intensities onto this grid in object
   and background channels.
   The background pixels are used to form a smooth background spectrum.
   The object pixels (less smooth background) are used to form the
   integrated net signal for the object.

   In the LORES case, the spectrum specified by the \verb+APERTURE+ parameter
   is extracted.

   In the HIRES case, the first \'{e}chelle order to be extracted is
   specified by \verb+ORDER+\@.
   Up to \verb+NORDER+ orders are extracted, with \verb+ORDER+ being
   decremented each time.

   The wavelength grid is defined by the region of the dispersion line
   contained in the image subset (faceplate).
   The grid spacing is set by the \verb+GSAMP+ parameter value which is
   the sample step in geometric pixels.
   The wavelength limits can optionally be constrained within the
   \'{e}chelle cutoff values by specifying \verb+CUTWV=TRUE+\@.

   The object and background channel widths and positions are determined
   automatically if \verb+AUTOSLIT=TRUE+\@.
   Otherwise, the object channel is specified by the values of the
   \verb+GSLIT+ parameter, whilst the background channel positions and
   widths are determined by the \verb+BDIST+ and \verb+BSLIT+ parameter values
   respectively.

   The \verb+EXTENDED+ and \verb+CONTINUUM+ parameters allow more precise
   control over slit determinations (see the IUEDR User Guide (MUD/45) for
   details).

   The background spectrum is smoothed with a triangle function with a
   FWHM given in geometric pixels by the \verb+BKGAV+ parameter.
   Once the background channel spectra have been obtained, they are
   extracted a further \verb+BKGIT+ times.
   Prior to each additional background extraction pixels which are
   outside \verb+BKGSD+ local standard deviations are rejected.

   The object spectrum is obtained by integrating pixel intensities
   (less smooth background) within the object channel.
   Once the object spectrum has been obtained it is extracted an
   additional \verb+CENIT+ times, the centroid positions
   from the previous extraction being used to ``follow'' the
   spectrum each time.
   The centroid spectrum (template) is smoothed by folding with a
   triangle function, FWHM given in geometric
   pixels by the \verb+CENAV+ parameter.
   Wavelengths with flux levels below \verb+CENSD+ standard deviations
   above background are not used in determining the centroid spectrum.

   By default, the initial spectrum template is given by the dispersion
   relations and the geometric shifts determined using the \verb+CGSHIFT+\@.
   However, if \verb+CENTM=TRUE+, then a pre-defined template associated with
   the dataset may be used as a start guess.
   If \verb+CENSH=TRUE+, then this template can be shifted linearly to match the
   image ({\it{i.e.}}\ without changing its shape).
   If \verb+CENSV=TRUE+, then the final centroid spectrum is used to update the
   spectrum template in the dataset.

   The net flux associated with a wavelength point in the final extracted
   spectrum is defined as the integral of pixel intensities over a rectangle
   with dimensions given by the object channel width and the wavelength
   interval.
   These fluxes are scaled so that they correspond to an interval
   along the wavelength direction of 1.414 geometric pixels.
   This is so that the standard IUESIPS calibrations can be applied
   regardless of what actual sample rate has been employed.
   The integral is performed by using linear interpolation of pixel intensities.
}
}

\newpage
%------------------------------------------------------------------------------

\section{\xlabel{paramaters}\label{se:parameters}Parameters}
\markboth{Parameters}{\stardocname}

There follows a detailed description for each of the parameters used by
IUEDR commands.
The description for a particular parameter applies in any
command which uses it.
Some parameters have default values which are initialised on invoking IUEDR\@.
Default parameter behaviour is described in
Appendix~\ref{se:parameter_defaults}\@.

This release of IUEDR uses the ADAM parameter system. In this system
the parameters and their usage are described in an interface file
(See \xref{SG/4}{sg4}{} for more detail). It is possible to override the
default
interface file with your own personal version. This permits you to tailor
the precise behavior of each parameter according to your requirements.

In the following descriptions the required parameter value is one of:
\begin{description}
   \item [\_CHAR] A character string.
   \item [\_DOUBLE] A floating point number.  The decimal point need not be
                    included if the value is integer only.
   \item [\_INTEGER] Integer number.
   \item [\_LOGICAL] A logical value: {\tt YES, TRUE, NO} or {\tt FALSE.}
\end{description}

Where a parameter value is of the form:

\begin{quote}
{\bf
   [\_TYPE\{,\_TYPE\}]
}
\end{quote}

At least one value of type {\bf \_TYPE} is required, the second is optional.

\rule{\textwidth}{0.5mm}

\iueparlist{

\iueparameter{ABSFILE}
{
   \_CHAR
}{
   This is the name of a file containing an absolute flux calibration.
   A file type of \verb+.abs+ is assumed and need not be specified
   explicitly.
   If the file name contains a directory specification, then it should be
   enclosed in quotes.
}

\iueparameter{APERTURE}
{
   \_CHAR
}{
   This is the name of an individual aperture.
   The following names have defined meanings:

   \begin {description}
      \item {\tt SAP} --- IUE small aperture.
      \item {\tt LAP} --- IUE large aperture.
   \end {description}

   Other apertures may also be defined.
}

\iueparameter{APERTURES}
{
   \_CHAR
}{
   This specifies an aperture or group of apertures.
   The following three names have defined meanings:

   \begin {description}
      \item {\tt SAP} --- IUE small aperture.
      \item {\tt LAP} --- IUE large aperture.
      \item {\tt BAP} --- IUE both apertures ({\it{i.e.}}\ SAP and LAP together).
   \end {description}
}

\iueparameter{AUTOSLIT}
{
   \_LOGICAL
}{
   This determines whether the extraction slit is determined automatically
   by the command.
   When \verb+AUTOSLIT=TRUE+ the \verb+GSLIT+, \verb+BDIST+ and \verb+BSLIT+
   parameter values are
   determined automatically, based on the IUE camera, resolution, aperture,
   and on the values of the \verb+EXTENDED+ and \verb+CONTINUUM+ parameters.
   This mode of operation is probably the best for point source objects.

   This parameter has a default value of \verb+TRUE+\@.
}

\iueparameter{BADITF}
{
   \_LOGICAL
}{
   This parameter determines whether a correction is made to the pixel
   intensities to account for errors during Ground Station ITF
   calibration.
   (Note that the best scientific results would be obtained
   from reprocessed data which can be obtained on request.)
   The following case is handled:

   \begin {description}
      \item SWP, LORES --- correction of 2nd (faulty) ITF.
   \end {description}
}

\iueparameter{BDIST}
{
   [\_DOUBLE\{,\_DOUBLE\}]
}{
   This is a pair of numbers delineating the background spectrum channel
   positions during spectrum extraction.
   The distances are measured in geometric pixels from the spectrum centre.

   Negative distances mean ``to the left of centre'', and positive distances
   mean ``to the right of centre''.

   If only one value is defined, then this is taken as meaning
   that the channels are positioned symmetrically
   about centre.

   The spectrum ``centre'' is determined by the dispersion relations, and
   modified by any prevailing centroid shifts.
}

\iueparameter{BKGAV}
{
   \_DOUBLE
}{
   This is the FWHM of a triangle function filter used in folding the
   pixel intensities to form the smooth background spectrum.
   It is measured in geometric pixels.

   This parameter has a default value of 30.0.
}

\iueparameter{BKGIT}
{
   \_DOUBLE
}{
   This is the number of background smoothing iterations performed during
   spectrum extraction.

   \begin {description}
      \item \verb+BKGIT=0+ means that the background is taken as the result of
      the first pass of a triangle function filter with a FWHM defined by
      the \verb+BKGAV+ parameter.

      \item \verb+BKGIT=1+ means that, after producing the initial estimate for
      the smooth background, pixels discrepant by more that \verb+BKGSD+
      standard deviations are marked as ``spikes''.
      The smooth background is then re-evaluated, missing out these marked
      pixels.
   \end {description}

   Higher values of \verb+BKGIT+ are possible, but seldom necessary.

   This parameter has a default value of 1.
}

\iueparameter{BKGSD}
{
   \_DOUBLE
}{
   This is the discrimination level, measured in standard deviations,
   beyond which background pixels are marked as ``spikes''.
   It is not used for \verb+BKGIT=0+.

   This parameter has a default value of 2.0.
}

\iueparameter{BSLIT}
{
   [\_DOUBLE\{,\_DOUBLE\}]
}{
   This defines the half width of each background channel, measured
   in geometric pixels.
   A single value means that both channels have the same width.
}

\iueparameter{CAMERA}
{
   \_CHAR
}{
   This is the camera name.
   The following are defined:

   \begin {quote}
   \begin {description}
      \item {\tt LWP} --- IUE long wavelength prime.
      \item {\tt LWR} --- IUE long wavelength redundant.
      \item {\tt SWP} --- IUE short wavelength prime.
   \end {description}
   \end {quote}
}

\iueparameter{CENAV}
{
   \_DOUBLE
}{
   This is the FWHM of a triangle function filter used in folding the
   pixel intensities to form the smooth spectrum centroid positions.
   It is measured in geometric pixels.

   This parameter has a default value of 30.0.
}

\iueparameter{CENIT}
{
   \_INTEGER
}{
   This is the number of spectrum centroid tracking iterations performed during
   spectrum extraction.

   \begin {description}
      \item \verb+CENIT=0+ means that the spectrum position is taken directly
      from the dispersion relations.

      \item \verb+CENIT=1+ means that the spectrum position is first taken
      from the dispersion relations, but is modified to force it to
      follow the spectrum centroid.
   \end {description}

   Higher values of \verb+CENIT+ are possible, but seldom necessary:  it either
   works or fails.

   This parameter has a default value of 1.
}

\iueparameter{CENSD}
{
   \_DOUBLE
}{
   This is the discrimination level, measured in standard deviations,
   below which object signal is not considered significant enough
   to be used to determine the centroid position.
   It is not used for \verb+CENIT=0+\@.

   This parameter has a default value of 4.0.
}

\iueparameter{CENSH}
{
   \_LOGICAL
}{
   This indicates whether the spectrum signal produces a single linear
   shift to the initial template.

   This can be used in cases where the object signal is too weak
   to provide a detailed centroid determination by moving a pre-existing
   template shape into the right position.

   This parameter has a default value of \verb+FALSE+\@.
}

\iueparameter{CENSV}
{
   \_LOGICAL
}{
   This indicates whether the spectrum template, as refined by the
   object centroid during spectrum extraction, is saved in the calibration
   dataset.

   The primary use of this facility is in determining templates from,
   say, the whole spectrum using \verb+TRAK+, and subsequently using these
   with \verb+LBLS+, or another spectrum.

   This parameter has a default value of \verb+FALSE+\@.
}

\iueparameter{CENTM}
{
   \_LOGICAL
}{
   This indicates whether a centroid template from the calibration dataset
   is used as a start in defining the precise position of the spectrum
   signal on the image.

   This parameter has a default value of \verb+FALSE+\@.
}

\iueparameter{CENTREWAVE}
{
   [\_DOUBLE[,\_DOUBLE\ldots ]]
}{
   These are the laboratory wavelengths of a set of absorption features in
   the spectrum to be used to estimate a value for the \verb+ESHIFT+
   parameter.
}

\iueparameter{COL}
{
   \_INTEGER
}{
   This specifies the line colour to be used for the next curve to be
   plotted.
   It can be an integer in the range 1 to 10, and the corresponding
   colours are as follows:

   \begin {quote}
   \begin {description}
      \item {\tt 1} --- Yellow.
      \item {\tt 2} --- Green.
      \item {\tt 3} --- Red.
      \item {\tt 4} --- Blue.
      \item {\tt 5} --- Pink.
      \item {\tt 6} --- Violet.
      \item {\tt 7} --- Turquoise.
      \item {\tt 8} --- Orange.
      \item {\tt 9} --- Light green.
      \item {\tt 10} --- Olive.
   \end {description}
   \end {quote}

   Lines will only appear with different colours it the device supports colour
   graphics, on other devices \verb+COL+ is ignored.
}

\iueparameter{COLOUR}
{
   \_LOGICAL
}{
   Whether a spectrum-style false colour look-up table is used by
   \verb+DRIMAGE+\@.
   If \verb+FALSE+ a greyscale is used.

   The default is to use a greyscale.
}

\iueparameter{COLROT}
{
   \_LOGICAL
}{
   This indicates whether the line colour is to be changed after the next plot.
   The initial line has colour index 1 (YELLOW), unless specified explicitly
   using the \verb+COL+ parameter.
   The sequence of colour indices goes (1, 2, 3, \ldots 10, 1, 2, \ldots).

   In commands where more than one line is plotted, \verb+COLROT+ determines
   whether these lines have different colours.

   Lines will only appear with different colours if the device supports
   colour graphics; on other devices \verb+COLROT+ is harmless.

   This parameter has a default value of \verb+TRUE+\@.
}

\iueparameter{CONTINUUM}
{
   \_LOGICAL
}{
   This indicates whether the object spectrum is expected to contain a
   significant continuum.
   It is used in conjunction with the \verb+EXTENDED+ parameter in determining
   the positions and widths of object and background channels for
   spectrum extraction from HIRES datasets.

   This parameter has a default value of \verb+TRUE+\@.
}

\iueparameter{COVERGAP}
{
   \_LOGICAL
}{
   If after mapping an order/aperture, a grid point is marked as unusable,
   then this parameter determines whether other orders/apertures
   can be allowed to contribute to this grid point.

   This parameter has a default value of \verb+FALSE+\@.
}

\iueparameter{CUTFILE}
{
   \_CHAR
}{
   This is the name of a file containing an \'{e}chelle order
   wavelength limits.
   A file type of \verb+.cut+ is assumed and need not be specified
   explicitly.
   If the file name contains a directory specification, then it should be
   enclosed in quotes.
}

\iueparameter{CUTWV}
{
   \_LOGICAL
}{
   This indicates whether any available \'{e}chelle order wavelength cutoff
   limits are to be used for the spectrum extraction wavelength grid
   limits.
   Highly recommended, provided that you are happy with these wavelength limits.

   This parameter has a default value of \verb+TRUE+\@.
}

\iueparameter{DATASET}
{
   \_CHAR
}{
   This is the root name of the files containing the dataset.
   The file type ({\it{e.g.,}}\ \_\verb+UED.sdf+) should {\bf not} be given in
   the \verb+DATASET+ name.
   If the file name contains a directory specification, then it
   should be enclosed in quotes.

   Note that the actual filenames contain additional characters
   to define their contents ({\it{e.g.,}}\ \verb+LWP12345+\_\verb+UES.sdf+,
   contains spectral data).
}

\iueparameter{DAY}
{
   \_INTEGER
}{
   This is the day number, measured from the start of the month, used
   for constructing dates.
   The \verb+DAY+, \verb+MONTH+ and \verb+YEAR+ parameters refer to the date
   the IUE observations were made and are important to the calibration of the
   data.
}

\iueparameter{DELTAWAVE}
{
   [\_DOUBLE[,\_DOUBLE\ldots]\,]
}{
   The half-width of the window used to search for an absorption line feature
   for wavelength calibration in Angstroms.  If more than one line is being
   used then each may be given a different search window width.
}

\iueparameter{DEVICE}
{
   \_CHAR
}{
   This is the GKS/SGS graphics device.
   A list of possible GKS workstations may be found in
   \xref{SUN/83}{sun83}{}.
   A list of SGS workstation names available at your
   site may be obtained either by a null response to the \verb+DEVICE+ parameter
   prompt, {\it{i.e.}}\ \verb+!+, or by using the IUEDR Command \verb+SGS+\@.
}

\iueparameter{DISPFILE}
{
   \_CHAR
}{
   This is the name of a file containing dispersion data.
   A file type of \verb+.dsp+ is assumed and need not be specified
   explicitly.
   If the file name contains a directory specification, then it should be
   enclosed in quotes.
}

\iueparameter{DRIVE}
{
   \_CHAR
}{
   This is the name of the tape drive. Feasible value
   are \verb+/dev/nrmt0h+ on a UNIX machine and \verb+MSA0+ on VMS.

   This version of IUEDR supports the direct reading of
   IUEDR data from disk files which have the same format as those on
   GO format tapes.

   In order to read directly from such a file (probably grabbed from
   an on-line archive such as NDADSA), you specify its name directly in
   response to the \verb+DRIVE+ parameter prompt.

   If the file is not in the current directory then you must provide
   the full pathname.
}

\iueparameter{ESHIFT}
{
   \_DOUBLE
}{
   This is a global wavelength shift applied to the wavelengths in
   \'{e}chelle spectral orders.
   It is measured in Angstroms, and affects the spectrum wavelengths as follows:

   \begin {equation}
      \lambda _{new} = \lambda _{old} + \frac{\rm ESHIFT}{\rm ORDER}
   \end {equation}

   This is designed to account for wavelength errors that result
   from a global linear shift of the spectrum format on the
   image.
}

\iueparameter{EXPOSURE}
{
   \_DOUBLE
}{
   This is the exposure time associated with the spectrum, measured in seconds.
   If there is more than one aperture, then this time applies
   to that specified by the \verb+APERTURE+ parameter.
}

\iueparameter{EXPOSURES}
{
   [\_DOUBLE\{,\_DOUBLE\}]
}{
   This is one or more exposure times associated with the
   spectrum, measured in seconds.
   There is an exposure time for each aperture defined.
}

\iueparameter{EXTENDED}
{
   \_LOGICAL
}{
   This indicates whether the object spectrum is expected to be extended,
   rather than a point source.
   It is used in conjunction with the \verb+CONTINUUM+ parameter in determining
   the positions and widths of the object and background channels used
   for spectrum extraction from HIRES datasets.

   This parameter has a default value of \verb+FALSE+\@.
}

\iueparameter{FIDFILE}
{
   \_CHAR
}{
   This is the name of a file containing fiducial positions.
   A file type of \verb+.fid+ is assumed and need not be specified
   explicitly.
   If the file name contains a directory specification, then it should be
   enclosed in quotes.
}

\iueparameter{FIDSIZE}
{
   \_DOUBLE
}{
   This is the half width of a fiducial measured in pixel units. The fiducials
   are considered to be square.
}

\iueparameter{FILE}
{
   \_INTEGER
}{
   This is the tape file number.
   The first file on a tape would be \verb+FILE=1+\@.
   One case which may present some problems is
   that of a tape with a an end-of-volume (EOV) mark in the middle
   and with valuable data beyond.
   An EOV is two consecutive tape marks (sometimes called ``file marks'').
   A file is defined here as the information between two tape marks.
   So if the number for a real file before EOV is FILEN, then
   the number of the next real file following the EOV is (FILEN+2).
}

\iueparameter{FILLGAP}
{
   \_LOGICAL
}{
   If a grid point in the mean spectrum would have had a contribution
   from a bad data point, this parameter determines whether that
   grid point is marked as unusable within the context
   of the order or aperture being mapped.
   If the grid point is marked as unusable in this way then other
   good points cannot contribute to it.

   This parameter has a default value of \verb+FALSE+\@.
}

\iueparameter{FLAG}
{
   \_LOGICAL
}{
   This specifies whether the data quality information is displayed
   along with the image.
   If so, then faulty pixels will be marked with a colour according to
   the following scheme:

   \begin {quote}
   \begin {description}
      \item Green --- pixels affected by reseau marks.
      \item Red --- pixels which are saturated (DN = 255).
      \item Orange --- pixels affected by ITF truncation.
      \item Yellow --- pixels marked bad by the user.
   \end {description}
   \end {quote}

   If a pixel is affected by more than one of the above faults, then
   the first in the list is adopted for display.
   Hence, user edits are only shown where no other fault is present.

   This option would normally only be used when assessing the quality
   of faulty pixels, possibly with a view to using them, {\it{i.e.}}\ marking
   them ``good'' with a cursor editor.

   This parameter has a default value of \verb+TRUE+\@.
}

\iueparameter{FN}
{
   \_DOUBLE
}{
   This parameter is the replacement Flux Number for a pixel changed
   explicitly by the user.
}

\iueparameter{FSCALE}
{
   \_DOUBLE
}{
   This is an arbitrary scale factor applied to spectrum fluxes.
   It affects fluxes as follows:

   \begin {equation}
      {\cal F}_{new} = {\cal F}_{old} \times {\rm FSCALE}
   \end {equation}

   It finds application in accounting for grey attenuation, or obscuration
   of object signal through a narrow aperture.
}

\iueparameter{GSAMP}
{
   \_DOUBLE
}{
   This is the sampling rate used for spectrum extraction.
   It is measured in geometric pixels.
   \verb+GSAMP=1.414+ corresponds to the IUESIPS\#1 sampling
   rate, while \verb+GSAMP=0.707+ corresponds to the IUESIPS\#2 sampling
   rate.
   Other values can be chosen.

   This parameter has a default value of 1.414.
}

\iueparameter{GSHIFT}
{
   [\_DOUBLE,\_DOUBLE]
}{
   This is a global constant shift of the spectrum format on the image,
   $(dx,dy)$, where the geometric coordinates, $(x,y)$ of a spectrum position
   are

   \begin {equation}
      x_{new} = x_{old} + dx
   \end {equation}

   and

   \begin {equation}
      y_{new} = y_{old} + dy
   \end {equation}
}

\iueparameter{GSLIT}
{
   [\_DOUBLE\{,\_DOUBLE\}]
}{
   This is a pair of numbers delineating the object spectrum channel
   during spectrum extraction.
   The distances are measured in geometric pixels.

   Negative distances mean ``to the left of centre'', and positive distances
   mean ``to the right of centre''.

   Object channels that do not cover the actual object signal on the
   image will not be meaningful when centroid tracking is employed.

   If only one value is defined, then this is taken as representing
   a channel that is symmetrical about the spectrum centre.

   The spectrum ``centre'' is determined by the dispersion relations
   modified by any prevailing centroid shifts.
}

\iueparameter{HALAV}
{
   \_DOUBLE
}{
   This is the FWHM of a triangle function used for smoothing the
   net spectrum for the \verb+HALTYPE=POWER+ halation correction technique.
}

\iueparameter{HALC}
{
   \_DOUBLE
}{
   This is the Halation correction constant used for \verb+HALTYPE=POWER+
   cases, and defined at wavelength \verb+HALWC+\@.
   The value of the correction constant
   corresponds roughly to the measured depression of a broad
   zero intensity absorption below zero, in units
   of the continuum in adjacent orders.
   The ``constant'', $C$, varies with wavelength as follows:

   \begin {equation}
      C_\lambda = \frac{{\rm HALC} \times (\lambda - {\rm HALW0})}
                       {({\rm HALWC} - {\rm HALW0})}
   \end {equation}

   See the \verb+HALTYPE+, \verb+HALWC+, \verb+HALW0+ and \verb+HALAV+
   parameters.
}

\iueparameter{HALTYPE}
{
   \_CHAR
}{
   This is the type of Halation or order-overlap correction applied to the
   flux spectrum.
   It can take the value

   \begin {quote}
   \begin {description}
      \item {\tt POWER} --- correction based on power-law PSF decay.
   \end {description}
   \end {quote}
}

\iueparameter{HALW0}
{
   \_DOUBLE
}{
   This is the wavelength, measured in Angstroms, at which the
   halation correction is zero.

   See the \verb+HALTYPE+, \verb+HALC+ and \verb+HALWC+ parameters.
}

\iueparameter{HALWC}
{
   \_DOUBLE
}{
   This is the wavelength, measured in Angstroms, at which the
   halation correction is \verb+HALC+\@.

   See the \verb+HALTYPE+, \verb+HALC+ and \verb+HALW0+ parameters.
}

\iueparameter{HIST}
{
   \_LOGICAL
}{
   This determines whether lines are plotted as histograms rather than
   continuous lines (polylines).

   This parameter has a default value of \verb+TRUE+\@.
}

\iueparameter{IMAGE}
{
   \_INTEGER
}{
   This is the Image Sequence Number.
}

\iueparameter{ITF}
{
   \_INTEGER
}{
   This is the ITF generation used in the photometric calibration of the
   image.  This information is needed for the correct absolute flux
   calibration of the resulting spectra.  Possible values for each camera
   are as follows:

   \begin {quote}
   \begin {description}
      \item SWP --- 2
      \item LWR --- 1 and 2
      \item LWP --- 1 and 2
   \end {description}
   \end {quote}

   The appropriate value can be determined from inspection of the
   IUE header text for the GPHOT/PHOT file using the table
   of numbers following the line beginning \verb+PCF C/**+\@.
   Here are the \verb+ITF+ values associated with various forms of this
   table:

   \begin {quote}
   \begin {tabbing}
   ITFMAXxxx\= 0xx\= 1800xx\= 3700xx\= 5600xx\= ...xx\= 30000xxx\= (Corrected, 3rd SWP ITF)\kill
   {\bf ITF}\> \> \>{\bf TABLE}\> \> \> \>{\bf IDENTIFICATION}\\
   \\
   ITF 0\>0\>1800\>3700\>5600\>...\>30000\>Preliminary LWR ITF\\
   ITF 1\>0\>2303\>4069\>8008\>...\>42032\>2nd LWR ITF\\
   ITF 1\>0\>2300\>3969\>6062\>...\>32973\>1st LWP ITF\\
   ITF 2\>0\>2723\>5429\>8145\>...\>38389\>2nd LWP ITF\\
   ITF 0\>0\>1800\>3600\>5500\>...\>\>Preliminary SWP ITF\\
   ITF 1\>0\>1753\>3461\>6936\>...\>28674\>Faulty, 2nd SWP ITF\\
   ITF 2\>0\>1684\>3374\>6873\>...\>28500\>Corrected, 3rd SWP ITF\\
   \end {tabbing}
   \end {quote}

   If the ITF table used has no corresponding absolute flux calibration within
   IUEDR, {\it{e.g.,}}\ LWR ITF0 or SWP ITF0, you are advised to contact the IUE
   Project.
   Although the \verb+BADITF+ parameter is available for data calibrated using
   SWP ITF1, it is advisable to have these data reprocessed by the IUE Project.
}

\iueparameter{ITFMAX}
{
   \_INTEGER
}{
   This is the pixel value on tape corresponding to ITF saturation.
   Its value is fixed for a given ITF table.
   The value of \verb+ITFMAX+ is only needed for IUE images of type GPHOT.
   The appropriate value can be determined from inspection of the
   IUE header text for the GPHOT file using the table
   of numbers following the line beginning \verb+PCF C/**+.

   \begin {quote}
   \begin {tabbing}
   ITFMAXxxx\= 0xx\= 1800xx\= 3700xx\= 5600xx\= ...xx\= 30000xxx\= (Corrected, 3rd SWP ITF)\kill
   {\bf ITFMAX}\>\>\>{\bf TABLE}\>\>\>\>{\bf IDENTIFICATION}\\
   \\
   20000\>0\>1800\>3700\>5600\>...\>30000\>Preliminary LWR ITF\\
   27220\>0\>2303\>4069\>8008\>...\>42032\>2nd LWR ITF\\
   19983\>0\>1800\>3600\>5500\>...\>\>Preliminary SWP ITF\\
   19740\>0\>1753\>3461\>6936\>...\>28674\>Faulty, 2nd SWP ITF\\
   19632\>0\>1684\>3374\>6873\>...\>28500\>Corrected, 3rd SWP ITF\\
   \end {tabbing}
   \end {quote}
}

\iueparameter{LINE}
{
   \_CHAR
}{
   This specifies the line style to be used for the next curve to be
   plotted.
   It can be one of the following:

   \begin {quote}
   \begin {description}
      \item {\tt SOLID} --- solid (continuous) line.
      \item {\tt DASH} --- dashed line.
      \item {\tt DOTDASH} --- dot-dash line.
      \item {\tt DOT} --- dotted line.
   \end {description}
   \end {quote}

   The order of these is that invoked when automatic line style rotation
   is in effect (see the \verb+LINEROT+ parameter).
}

\iueparameter{LINEROT}
{
   \_LOGICAL
}{
   This indicates whether the line style is to be changed after the
   next plot.
   The initial line style is \verb+SOLID+, unless specified explicitly
   using the \verb+LINE+ parameter.
   The sequence of line styles goes (\verb+SOLID+, \verb+DASH+, \verb+DOTDASH+,
   \verb+DOT+, \verb+SOLID+, \verb+DASH+\dots).

   In commands where more than one line is plotted, \verb+LINEROT+ determines
   whether these lines have different styles.

   This parameter has a default value of \verb+FALSE+\@.
}

\iueparameter{ML}
{
   [\_DOUBLE,\_DOUBLE]
}{
   This is a pair of wavelength values defining the start and end of
   the mean spectrum grid.
   The grid will consist of evenly spaced vacuum wavelengths between these
   values.
}

\iueparameter{MONTH}
{
   \_INTEGER
}{
   This is the month number, measured from the start of the Year,
   used in constructing dates.
}

\iueparameter{MSAMP}
{
   \_DOUBLE
}{
   This is the vacuum wavelength sampling rate for the mean spectrum grid.
   If it does not fit an integral number of times into the grid limits,
   then the latter are adjusted to fit.
}

\iueparameter{NFILE}
{
   \_INTEGER
}{
   This is the number of tape files to be processed.
   A value of \verb+NFILE=-1+ means all files until the end.

   This parameter has a default value of 1.
}

\iueparameter{NGEOM}
{
   \_INTEGER
}{
   This is the number of Chebyshev terms used to represent the
   geometrical distortion.
   The same value is used for each axis direction.
}

\iueparameter{NLINE}
{
   \_INTEGER
}{
   This is the number of IUE header lines printed.
   A value of \verb+NLINE=-1+ means the entire header is printed.

   This parameter has a default value of 10.
}

\iueparameter{NORDER}
{
   \_INTEGER
}{
   This is the number of \'{e}chelle orders to be processed by a command.

   This parameter has a default value of 0.
}

\iueparameter{NSKIP}
{
   \_INTEGER
}{
   This is the number of tape marks to be skipped.

   This parameter has a default value of 1.
}

\iueparameter{OBJECT}
{
   \_CHAR
}{
   This is a string containing text to identify the object of the
   observation.
   It can also contain information about the observation
   ({\it{e.g.,}}\ camera, image\ldots ) if required.
   The maximum allowed length of the string is 40 characters.
}

\iueparameter{ORDER}
{
   \_INTEGER
}{
   This is the \'{e}chelle order number.
}

\iueparameter{ORDERS}
{
   [\_INTEGER\{,\_INTEGER\}]
}{
   This is a pair of \'{e}chelle order numbers delineating a range.
   If only a single value is specified, then the range consists of that
   order only.
   The sequence of the two numbers is not significant.
   The useful maximum range for each camera is as follows:

   \begin {quote}
   \begin {description}
      \item SWP --- orders 66 to 125.
      \item LWR --- orders 72 to 125.
      \item LWP --- orders 72 to 125.
   \end {description}
   \end {quote}
}

\iueparameter{OUTFILE}
{
   \_CHAR
}{
   This is the name of a file to receive the output spectrum.
   This release of IUEDR uses the STARLINK NDF format for all output
   spectra. This means that all standard STARLINK packages can be used
   to plot/display/analyse the spectra, in particular some of the
   facilities of KAPPA and FIGARO may prove useful to the general user.
}

\iueparameter{QUAL}
{
   \_LOGICAL
}{
   This specifies whether the data quality information is plotted
   along with the data.
   If so, then faulty points will be marked with their data quality
   severity code, which is one from:

   \begin {quote}
   \begin {description}
      \item 1 --- affected by extrapolated ITF.
      \item 2 --- affected by microphonics.
      \item 3 --- affected by spike.
      \item 4 --- affected by bright point (or user).
      \item 5 --- affected by reseau mark.
      \item 6 --- affected by ITF truncation.
      \item 7 --- affected by saturation.
      \item U --- affected by user edit.
   \end {description}
   \end {quote}

   User edits are only shown where no other fault is present.

   This option would normally only be used when assessing the quality
   of faulty points, possibly with a view to using them, {\it{i.e.}}\ marking
   them ``good'' with a cursor editor.

   This parameter has a default value of \verb+TRUE+\@.
}

\iueparameter{RESOLUTION}
{
   \_CHAR
}{
   This is the spectrograph resolution mode.
   The following modes are defined:

   \begin {quote}
   \begin {description}
      \item {\tt LORES} --- IUE Low Resolution.
      \item {\tt HIRES} --- IUE High Resolution (\'{e}chelle mode).
   \end {description}
   \end {quote}
}

\iueparameter{RIPA}
{
   \_DOUBLE
}{
   This is an empirical scale factor that can be used to modify the
   \'{e}chelle ripple function.
   The normal value is 1.0.
   The primary component of the ripple function is

   \begin {equation}
      {\rm SCALE} = (\frac{\sin x}{x})^2
   \end {equation}

   where

   \begin {equation}
      x = \frac{\pi \times {\rm RIPA} \times (\lambda - \lambda_c)
                \times {\rm ORDER}}
               {\lambda_c}
   \end {equation}

   and

   \begin {equation}
      \lambda_c = \frac{\rm RIPK}{\rm ORDER}
   \end {equation}

   The net spectrum is divided by SCALE above.
   Empirical values of \verb+RIPA+ can be used to optimise the ripple
   correction.

   See the \verb+RIPC+ and \verb+RIPK+ parameter descriptions.
}

\iueparameter{RIPC}
{
   [\_DOUBLE\{,\_DOUBLE\}]
}{
   This is a polynomial in $x$ used to modify the standard \'{e}chelle
   ripple calibration function.
   The calibration is given by

   \begin {equation}
      {\rm SCALE} = (\frac{\sin x}{x})^2 \times (
                {\rm RIPC}(1) + {\rm RIPC}(2) \times x +
                {\rm RIPC}(3) \times x^2 \ldots)
   \end {equation}

   where

   \begin {equation}
      x = \frac {\pi \times {\rm RIPA} \times (\lambda - \lambda_c) \times
                 {\rm ORDER}}
                {\lambda_c}
   \end {equation}

   and

   \begin {equation}
      \lambda_c = \frac {\rm RIPK}{\rm ORDER}
   \end {equation}

   The net spectrum is divided by SCALE above.

   See the \verb+RIPA+ and \verb+RIPK+ parameter descriptions.
}

\iueparameter{RIPFILE}
{
   \_CHAR
}{
   This is the name of a file containing an \'{e}chelle ripple calibration.
   A file type of \verb+.rip+ is assumed and need not be specified
   explicitly.
   If the file name contains a directory specification, then it should be
   enclosed in quotes.
}

\iueparameter{RIPK}
{
   [\_DOUBLE\{,\_DOUBLE\}]
}{
   This is the \'{e}chelle ripple constant measured in Angstroms.
   It corresponds to the central wavelength of \'{e}chelle order number 1.
   The central wavelength of an arbitrary ORDER is

   \begin {equation}
      \lambda_c = \frac {\rm RIPK}{\rm ORDER}
   \end {equation}

   Where this parameter is used for an entire HIRES dataset, the
   parameter can have more than one value, and represent a polynomial
   in ORDER

   \begin {equation}
      \lambda_c =  \frac{({\rm RIPK}(1) + {\rm RIPK}(2)
                         \times {\rm ORDER} + {\rm RIPK}(3)
                         \times {\rm ORDER}^2 + \ldots)}
                        {\rm ORDER}
   \end {equation}
}

\iueparameter{RL}
{
   [\_DOUBLE,\_DOUBLE]
}{
   This is a pair of radial coordinate values defining the start and end of
   the radial grid in an LBLS array.
   These radial coordinates are measured in geometric pixels, and run
   perpendicular to the dispersion direction.
   A coordinate value of 0.0 corresponds to the centre of the spectrum.

   Values \verb+RL=[0.0, 0.0]+ indicate that internal defaults are to be
   adopted.
   A single value is reflected symmetrically about 0.0.
}

\iueparameter{RM}
{
   \_LOGICAL
}{
   This determines whether the mean spectrum is reset before a mapping
   takes place.
   If the spectrum is not reset, then the spectra being mapped will be
   averaged with the existing mean spectrum.

   This parameter has a default value of \verb+TRUE+\@.
}

\iueparameter{RS}
{
   \_LOGICAL
}{
   This determines whether the display screen is reset before plotting.

   This parameter has a default value of \verb+TRUE+\@.
}

\iueparameter{RSAMP}
{
   \_DOUBLE
}{
   This is the sample spacing used for the radial grid in the LBLS array.
   If it does not fit an integral number of times into the grid limits,
   \verb+RL+, then the latter are adjusted to fit.

   Suggested values range from 0.707 to 1.414 pixels, the latter
   corresponding to the IUESIPS LBLS grid.

   This parameter has a default value of 1.414.
}

\iueparameter{SCANAV}
{
   \_DOUBLE
}{
   This is the HWHM of a triangle function with which pixels are folded
   during the generation of a scan across the image perpendicular to
   spectrograph dispersion.
   It is measured in geometric pixels.

   This parameter has a default value of 5.
}

\iueparameter{SCANDIST}
{
   \_DOUBLE
}{
   This is the distance of a scan across a HIRES image from the faceplate
   centre.
   It is measured in geometric pixels.
}

\iueparameter{SCANWV}
{
   \_DOUBLE
}{
   This is the central wavelength for a scan of a LORES image
   perpendicular to spectrograph dispersion.
   It is measured in Angstroms in vacuo.
}

\iueparameter{SKIPNEXT}
{
   \_LOGICAL
}{
   This determines whether the tape is positioned at the start of the
   next file after processing.
   If only the start of a file is being processed, then by setting
   \verb+SKIPNEXT=FALSE+ time can be saved.

   This parameter has a default value of \verb+FALSE+\@.
}

\iueparameter{SPECTYPE}
{
   \_INTEGER
}{
   This is the type of file, in the DIPSO SP format terminology, to be
   created. The following values are allowed:

   \begin {quote}
   \begin {description}
      \item {\tt 0} --- Starlink NDF format file.
      \item {\tt 1} --- SP1, fixed format text file.
      \item {\tt 2} --- SP2, free format text file.
   \end {description}
   \end {quote}

   It is recommended that datasets with many points be written with
   \verb+SPECTYPE=0+, which is more efficient in disk space and time spent
   reading and writing.
   A description of the format of each of these file types can be found
   in Section~\ref{se:spectrum}\@.

   This parameter has a default value of 0.
}

\iueparameter{TEMFILE}
{
   \_CHAR
}{
   This is the name of a file containing the standard spectrum template
   data.
   A  file type of \verb+.tem+ is assumed and need not be specified
   explicitly.
   If the file name contains a directory specification, then it should be
   enclosed in quotes.
}

\iueparameter{THDA}
{
   \_DOUBLE
}{
   This is the IUE camera temperature, measured in degrees Centigrade.
   It is used for such things as adjustments to fiducial positions
   and spectrograph dispersion relations.
   A value of 0.0 implies that no \verb+THDA+ value is available, the program
   will then  use a suitable mean \verb+THDA+ for the camera being used.
   Values for the \verb+THDA+ can be found in the IUE header text of the final
   spectrum file on the Guest Observer tape for IUESIPS\#2 ---
   \verb+THDA+ values derived from spectrum motion are best.
}

\iueparameter{THRESH}
{
   \_DOUBLE
}{
   This is the  minimum value considered to be good when using \verb+CLEAN+\@.
   All pixels with values below this threshold will be marked BAD.
}

\iueparameter{TYPE}
{
   \_CHAR
}{
   This is the type of dataset.
   Defined values are as follows:

   \begin {quote}
   \begin {description}
      \item {\tt RAW} --- IUE raw image.
      \item {\tt GPHOT} --- IUE GPHOT image (geometric and photometric).
      \item {\tt PHOT} --- IUE PHOT image (photometric only).
   \end {description}
   \end {quote}

   Types \verb+PHOT+ and \verb+GPHOT+ are not automatically distinguishable
   from IUE Guest Observer tape contents.
}

\iueparameter{V}
{
   \_CHAR
}{
   This is a string defining a list of items and includes
   any of the following characters:

   \begin {quote}
   \begin {description}
      \item {\tt H} --- header and file information.
      \item {\tt I} --- image details.
      \item {\tt F} --- fiducials.
      \item {\tt G} --- geometry.
      \item {\tt D} --- dispersion.
      \item {\tt C} --- centroid templates.
      \item {\tt R} --- \'{e}chelle ripple and halation.
      \item {\tt A} --- absolute calibration.
      \item {\tt S} --- raw spectrum.
      \item {\tt M} --- mean spectrum.
      \item {\tt *} --- all of the above.
      \item {\tt Q} --- image data quality summary.
   \end {description}
   \end {quote}
}

\iueparameter{VSHIFT}
{
   \_DOUBLE
}{
   This is the radial velocity of the detector relative to the Sun.
   It is measured in km/s and affects the calibrated wavelength
   scale as follows:

   \begin {equation}
      \lambda_{true} = \frac{\lambda_{obs}}
                            {(1 + \frac{\rm VSHIFT}{c})}
   \end {equation}

   where $c$ is the velocity of light in km/s.
}

\iueparameter{WCUT}
{
   [\_DOUBLE,\_DOUBLE]
}{
   This is one of the mechanisms that can be used to delimit the
   parts of \'{e}chelle orders that are calibrated for ripple response.
   The two values of this parameter are the start and end wavelengths
   for a specific \verb+ORDER+\@.

   Apart from poor ripple calibration, the order ends can also be affected
   by the parts of the camera faceplate that are retained in the image.
}

\iueparameter{WSHIFT}
{
   \_DOUBLE
}{
   This is a constant wavelength shift applied to spectrum wavelengths.
   It is measured in Angstroms and affects the spectrum wavelengths as follows:

   \begin {equation}
      \lambda_{new} = \lambda_{old} + {\rm WSHIFT}
   \end {equation}

   This is only used for LORES spectra.
}

\iueparameter{XCUT}
{
   [\_DOUBLE,\_DOUBLE]
}{
   This is one of the mechanisms that can be used to delimit the
   ends of \'{e}chelle orders that are calibrated for ripple response.
   The two values of this parameter are the start and end $x$ coordinates
   of the order, where

   \begin {equation}
      x = \frac{\pi \times {\rm RIPA} \times (\lambda - \lambda_c)
                \times {\rm ORDER}}
               {\lambda_c}
   \end {equation}

   and

   \begin {equation}
      \lambda_c = \frac {\rm RIPK}{\rm ORDER}
   \end {equation}

   The nature of the standard ripple function is that $x$ is only
   formally meaningful in the range ($-\pi$, $+\pi$).

   See the \verb+RIPA+, \verb+RIPC+ and \verb+RIPK+ parameter descriptions.
}

\iueparameter{XL}
{
   [\_DOUBLE,\_DOUBLE]
}{
   This specifies the data limits used for plotting in the $x$-direction.
   This parameter is only read if the display has been reset, and
   the axes are being redrawn.
   If both values are the same ({\it{e.g.,}}\ \verb+[0, 0]+),
   then the data limits in the $x$-direction will be determined from the
   data being plotted.
}

\iueparameter{XP}
{
   [\_INTEGER,\_INTEGER]
}{
   This specifies the pixel limits along the $x$-direction used for
   image display.
   Values in decreasing order will cause the image to be inverted
   along the $x$-direction.
   If the values are undefined, the pixel limits will default to include
   the whole extent of the image along the $x$-direction.
}

\iueparameter{YEAR}
{
   \_INTEGER
}{
   This is the year (A.D.) used in constructing dates.
}

\iueparameter{YL}
{
   [\_DOUBLE,\_DOUBLE]
}{
   This specifies the data limits used for plotting in the $y$-direction.
   This parameter is only read if the display has been reset, and
   the axes are being redrawn.
   If both values are the same ({\it{e.g.,}}\ \verb+[0, 0]+),
   then the data limits in the $y$-direction will be determined from the
   data being plotted.
}

\iueparameter{YP}
{
   [\_INTEGER,\_INTEGER]
}{
   This specifies the pixel limits along the $y$-direction used for
   image display.
   Values in decreasing order will cause the image to be inverted
   along the $y$-direction.
   If the values are undefined, the pixel limits will default to include
   the whole extent of the image along the $y$-direction.
}

\iueparameter{ZL}
{
   [\_DOUBLE,\_DOUBLE]
}{
   This specifies the data limits used for display of images.
   If the values are given in decreasing order, then high data
   values will be represented by low (dark) display intensities,
   and vice-versa.
   If the values are undefined, then the full intensity range of the
   image will be used.

   Data values which fall at or below the lowest display intensity are drawn
   {\bf black,} those which are at the highest display intensity are drawn
   {\bf white} and those which are above the highest display intensity are
   drawn {\bf blue.}
}

\iueparameter{ZONE}
{
   \_INTEGER
}{
   This specifies the zone to be used for plotting.
   The zone numbers range
   from 0 to 8 and correspond to those defined by the TZONE command in
   DIPSO (see \xref{SUN/50}{sun50}{}), {\it{e.g.,}}

   \begin {quote}
   \begin {description}
      \item {\tt 0} --- entire display surface.
      \item {\tt 1} --- top left quarter.
      \item {\tt 2} --- top right quarter.
      \item {\tt 3} --- bottom left quarter.
      \item {\tt 4} --- bottom right quarter.
      \item {\tt 5} --- top half.
      \item {\tt 6} --- bottom half.
      \item {\tt 7} --- left half.
      \item {\tt 8} --- right half.
   \end {description}
   \end {quote}

   This parameter has a default value of 0.
}
}

\newpage
%------------------------------------------------------------------------------

\appendix
\section{\xlabel{parameter_defaults}\label{se:parameter_defaults}Parameter
          defaults}
\markboth{Parameter defaults}{\stardocname}

Some IUEDR parameters have default values.  Some have no default value and one
{\bf must} be provided.  Other parameters values are either calculated or
simply set by the program.  The default values and/or behaviour of each
parameter are listed here.

\begin {description}

\item [\htmlref{ABSFILE}{ABSFILE}] \lmbox
   No default value exists, a file name must be provided.
   The parameter is cancelled each time it is used.
\item [\htmlref{APERTURE}{APERTURE}] \lmbox
   Has no automatic default value.  An \verb+APERTURE+ must be selected.
   If only one is present in an image then this is taken as \verb+APERTURE+ by
   default.
\item [\htmlref{APERTURES}{APERTURES}] \lmbox
   Used only by \verb+READIUE+ and \verb+READSIPS+\@.  The value must be
   supplied by reading the IUE GO header `by eye'.
\item [\htmlref{AUTOSLIT}{AUTOSLIT} = TRUE] \lmbox
   Whether \verb+GSLIT+, \verb+BDIST+ and \verb+BSLIT+ are determined
   automatically.
\item [\htmlref{BADITF}{BADITF} (= TRUE)] \lmbox
   Has no default value, however it is recommended to be set \verb+TRUE+ as
   this will ensure any ITF error correction available for the particular
   \verb+ITF+ will be used.
   This may seem counter-intuitive, however, data using error-free ITF
   information will not be affected by setting \verb+BADITF=TRUE+\@.
\item [\htmlref{BDIST}{BDIST}] \lmbox
   No automatic default.  The value is calculated if \verb+AUTOSLIT=TRUE+,
   otherwise it should be estimated by looking at \verb+SCAN+ plots.
\item [\htmlref{BKGAV}{BKGAV} = 30.0] \lmbox
   Background averaging filter FWHM (geometric pixels).
\item [\htmlref{BKGIT}{BKGIT} = 1] \lmbox
   Number of background smoothing iterations.
\item [\htmlref{BKGSD}{BKGSD} = 2.0] \lmbox
   Discrimination level for background pixels (standard deviations).
\item [\htmlref{BSLIT}{BSLIT}] \lmbox
   No automatic default.  The value is calculated if \verb+AUTOSLIT=TRUE+,
   otherwise it should be estimated by looking at \verb+SCAN+ plots.
\item [\htmlref{CAMERA}{CAMERA}] \lmbox
   The value is read by the program from an IUE GO header.  The default value
   presented in a prompt will be the value found in the header.
\item [\htmlref{CENAV}{CENAV} = 30.0] \lmbox
   Centroid averaging filter FWHM (geometric pixels).
\item [\htmlref{CENIT}{CENIT} = 1] \lmbox
   Number of centroid tracking iterations.
\item [\htmlref{CENSD}{CENSD} = 4.0] \lmbox
   Discrimination level for signal to be used for centroids
   (standard deviations).
\item [\htmlref{CENSH}{CENSH} = FALSE] \lmbox
   Whether the spectrum template is just shifted linearly.
\item [\htmlref{CENSV}{CENSV} = FALSE] \lmbox
   Whether the spectrum template is saved in the dataset.
\item [\htmlref{CENTM}{CENTM} = FALSE] \lmbox
   Whether an existing centroid template is used.
\item [\htmlref{COL}{COL} (= 1)] \lmbox
   The value of \verb+COL+ will be calculated for each command requiring it.
   If \verb+COLROT=FALSE+ then \verb+COL+ will take the value 1, otherwise it
   will be incremented for each plotting command.
\item [\htmlref{COLOUR}{COLOUR} = FALSE] \lmbox
   Whether a spectrum-style false colour look-up table is used by
   \verb+DRIMAGE+\@.
\item [\htmlref{COLROT}{COLROT} = TRUE] \lmbox
   Whether the line colour is changed after the next plot.
\item [\htmlref{CONTINUUM}{CONTINUUM} = TRUE] \lmbox
   Whether the object spectrum is expected to have a ``continuum''.
\item [\htmlref{COVERGAP}{COVERGAP} = FALSE] \lmbox
   Whether gaps can be filled by covering orders.
\item [\htmlref{CUTFILE}{CUTFILE}] \lmbox
   No default value exists, a file name must be provided.
   The parameter is cancelled each time it is used.
\item [\htmlref{CUTWV}{CUTWV} = TRUE] \lmbox
   Whether wavelength cutoff data is to be used for the extraction grid.
\item [\htmlref{DATASET}{DATASET}] \lmbox
   No automatic default value.  The last value of \verb+DATASET+ used will be
   taken as the default.  The exception to this rule is when using
   \verb+READIUE+ or \verb+READSIPS+ when a suggested default value will be
   presented by the program in the parameter prompt.
\item [\htmlref{DAY}{DAY}] \lmbox
   No default value.  The program will attempt to extract the \verb+DAY+ from
   the IUE GO tape/file header and present this as the default value for that
   particular command.
\item [\htmlref{DEVICE}{DEVICE}] \lmbox
   No automatic default value.  The last value of \verb+DEVICE+ used will be
   taken as the default.
\item [\htmlref{DISPFILE}{DISPFILE}] \lmbox
   No default value exists, a file name must be provided.
   The parameter is cancelled each time it is used.
\item [\htmlref{DRIVE}{DRIVE}] \lmbox
   No automatic default value.  The last value of \verb+DRIVE+ used will be
   taken as the default.
\item [\htmlref{ESHIFT}{ESHIFT}] \lmbox
   No automatic default value.  If an \verb+ESHIFT+ has previously been set for
   the current \verb+DATASET+, this will be presented as the default value in
   the prompt.
\item [\htmlref{EXPOSURE}{EXPOSURE}] \lmbox
   No automatic default value.  The value of \verb+EXPOSURE+ previously set when
   using \verb+READIUE+ or \verb+READSIPS+ for the current \verb+APERTURE+ will
   be presented as the default value in the prompt.
\item [\htmlref{EXPOSURES}{EXPOSURES}] \lmbox
   No default value.  The program will attempt to extract the \verb+EXPOSURES+
   from the IUE GO tape/file header and present this as the default value for
   that particular command.
\item [\htmlref{EXTENDED}{EXTENDED} = FALSE] \lmbox
   Whether the object spectrum is expected to be extended.
\item [\htmlref{FIDFILE}{FIDFILE}] \lmbox
   No default value exists, a file name must be provided.
   The parameter is cancelled each time it is used.
\item [\htmlref{FIDSIZE}{FIDSIZE}] \lmbox
   The default \verb+FIDSIZE+ is read from the appropriate file in the
   \verb+$IUEDR_DATA+ directory.  This is presented as the default in parameter
   prompts.
\item [\htmlref{FILE}{FILE}] \lmbox
   No automatic default.  Takes the value 1 when reading a new tape or file.
\item [\htmlref{FILLGAP}{FILLGAP} = FALSE] \lmbox
   Whether gaps can be filled within an order.
\item [\htmlref{FLAG}{FLAG} = TRUE] \lmbox
   Whether data quality for faulty pixels is displayed.
\item [\htmlref{FN}{FN}] \lmbox
   No default.  A value must be supplied.
\item [\htmlref{FSCALE}{FSCALE} = 1] \lmbox
   No scale factor is applied by default.
\item [\htmlref{GSAMP}{GSAMP} = 1.414] \lmbox
   Spectrum grid sampling rate (geometric pixels).
\item [\htmlref{GSHIFT}{GSHIFT}] \lmbox
   No automatic default value.  The value of \verb+GSHIFT+ previously set when
   using \verb+CGSHIFT+ will be presented as the default value in the prompt.
\item [\htmlref{GSLIT}{GSLIT}] \lmbox
   No automatic default.  The value is calculated if \verb+AUTOSLIT=TRUE+,
   otherwise it should be estimated by looking at \verb+SCAN+ plots.
\item [\htmlref{HALAV}{HALAV} = 30.0] \lmbox
   FWHM of triangle function used for spectrum smoothing.
\item [\htmlref{HALC}{HALC}] \lmbox
   No default value.  Values must be positive or zero.
\item [\htmlref{HALTYPE}{HALTYPE} = POWER] \lmbox
   Currently this parameter can only take the value \verb+POWER+\@.
\item [\htmlref{HALW0}{HALW0} = 1400.0 or 1800.0] \lmbox
   Takes the value \verb+1400.0+ for the SWP camera, \verb+1800.0+ for
   LWP and LWR cameras.
\item [\htmlref{HALWC}{HALWC} = 1200.0 or 2400.0] \lmbox
   Takes the value \verb+1200.0+ for the SWP camera, \verb+2400.0+ for
   LWP and LWR cameras.
\item [\htmlref{HIST}{HIST} = TRUE] \lmbox
   Whether lines are to be drawn as histograms during plotting.
\item [\htmlref{IMAGE}{IMAGE}] \lmbox
   No default value.  The program will attempt to extract the \verb+IMAGE+
   number from the IUE GO tape/file header and present this as the default
   value for that particular command.
\item [\htmlref{ITF}{ITF}] \lmbox
   No default value.  Refer to page~\pageref{ITF} for details of
   working out which transfer function to use.
\item [\htmlref{ITFMAX}{ITFMAX}] \lmbox
   No default value.  Refer to page~\pageref{ITFMAX} for details of
   working out which transfer function saturation value to use.
\item [\htmlref{LINE}{LINE} (=SOLID)] \lmbox
   The value of LINE will be calculated for each command requiring it.
   If \verb+LINEROT=FALSE+ then \verb+LINE+ will take the value \verb+SOLID+,
   otherwise it will be rotated for each plotting command.
\item [\htmlref{LINEROT}{LINEROT} = FALSE] \lmbox
   Whether line style is to be changed after the next plot.
\item [\htmlref{ML}{ML}] \lmbox
   No default value.  Limits {\bf must} be specified.
\item [\htmlref{MONTH}{MONTH}] \lmbox
   No default value.  The program will attempt to extract the \verb+MONTH+ from
   the IUE GO tape/file header and present this as the default value for that
   particular command.
\item [\htmlref{MSAMP}{MSAMP}] \lmbox
   No default value exists.
\item [\htmlref{NFILE}{NFILE} = 1] \lmbox
   Number of tape files to be processed.
\item [\htmlref{NGEOM}{NGEOM} = 5] \lmbox
   When reading IUE GO tapes/files this is the value suggested.
\item [\htmlref{NLINE}{NLINE} = 10] \lmbox
   Number of IUE header lines to be printed.
\item [\htmlref{NORDER}{NORDER} = 0] \lmbox
   Number of \'{e}chelle orders to be processed.
\item [\htmlref{NSKIP}{NSKIP} = 1] \lmbox
   Number of tape marks to be skipped over.
\item [\htmlref{OBJECT}{OBJECT}] \lmbox
   The program will attempt to extract the \verb+OBJECT+ description text from
   the IUE GO header.  This is presented as the default in parameter prompts.
\item [\htmlref{ORDER}{ORDER}] \lmbox
   No default value. A valid \verb+ORDER+ {\bf must} be specified.
   Use \verb+SHOW V=S+ to find out which orders have been extracted from a
   HIRES image.
\item [{\htmlref{ORDERS}{ORDERS} (=[125,66])}] \lmbox
   Takes the values given by default for a new dataset and otherwise in
   response to a parameter cancel.
\item [\htmlref{OUTFILE}{OUTFILE}] \lmbox
   No automatic default.  The program will construct a suggested file name
   based on the \verb+CAMERA+, \verb+RESOLUTION+, \verb+APERTURE+ and
   \verb+ORDER+ as appropriate.
\item [\htmlref{QUAL}{QUAL} = TRUE] \lmbox
   Whether data quality information is plotted.
\item [\htmlref{RESOLUTION}{RESOLUTION}] \lmbox
   The program will attempt to extract the spectrograph \verb+RESOLUTION+ from
   the IUE GO header.  This is presented as the default in parameter prompts.
\item [\htmlref{RIPA}{RIPA} (=1)] \lmbox
   This parameter takes the value 1 when a new dataset is created.
\item [{\htmlref{RIPC}{RIPC} (=[1,0,0,0,0,0])}] \lmbox
   Takes the default values given above when a new dataset is created.
\item [\htmlref{RIPFILE}{RIPFILE}] \lmbox
   No default value exists, a file name must be provided.
   The parameter is cancelled each time it is used.
\item [\htmlref{RIPK}{RIPK}] \lmbox
   Takes the central wavelength value of an order by default.
\item [\htmlref{RL}{RL}] \lmbox
   No default.  \verb+RL=[0.0, 0.0]+ indicates the program should calculate
   values.
\item [\htmlref{RM}{RM} = TRUE] \lmbox
   Whether the mean spectrum is reset before averaging.
\item [\htmlref{RS}{RS} = TRUE] \lmbox
   Whether the display is reset before plotting.
\item [\htmlref{RSAMP}{RSAMP} = 1.414] \lmbox
   Radial coordinate sampling rate for \verb+LBLS+ grid (pixels).
\item [\htmlref{SCANAV}{SCANAV} = 5] \lmbox
   Averaging filter HWHM for image scan (geometric pixels).
\item [\htmlref{SCANDIST}{SCANDIST} (=0)] \lmbox
   Takes the last value used.  The first time \verb+SCAN+ is used the
   program will suggest a value of zero.
\item [\htmlref{SCANWV}{SCANWV}] \lmbox
   The value of \verb+SCANWV+ taken by default is calculated as the centre
   of the wavelength scale for the current \verb+APERTURE+\@.
\item [\htmlref{SKIPNEXT}{SKIPNEXT} = FALSE] \lmbox
   Whether skip to next tape file.
\item [\htmlref{SPECTYPE}{SPECTYPE} = 0] \lmbox
   DIPSO SP format file type (0, 1 or 2). Type 0 is a Starlink NDF.
\item [\htmlref{TEMFILE}{TEMFILE}] \lmbox
   No default value exists, a file name must be provided.
   The parameter is cancelled each time it is used.
\item [\htmlref{THDA}{THDA} (=0.0)] \lmbox
   No default, however, \verb+THDA=0.0+ implies that no value is available
   and the program will select a suitable mean \verb+THDA+ for the camera
   being used.
\item [\htmlref{THRESH}{THRESH}] \lmbox
   No default value.  The last value used will be presented as the default
   in parameter prompts.
\item [\htmlref{TYPE}{TYPE}] \lmbox
   No default.  This must be evaluated from the IUE GO file header contents.
\item [\htmlref{V}{V} = H] \lmbox
   This parameter is cancelled each time it is used.
\item [\htmlref{VSHIFT}{VSHIFT}] \lmbox
   No automatic default value.  If an VSHIFT has previously been set for
   the current \verb+DATASET+, this will be presented as the default value in
   the prompt.
\item [\htmlref{WCUT}{WCUT}] \lmbox
   The wavelength cut-off values are normally read from a \verb+.cut+ file and
   these are used for \verb+WCUT+ prompt values by default.
\item [\htmlref{WSHIFT}{WSHIFT} = 0] \lmbox
   No automatic default value.  If a \verb+WSHIFT+ has previously been set for
   the current \verb+DATASET+, this will be presented as the default value in
   the parameter prompt.
\item [{\htmlref{XCUT}{XCUT} =[-3.0,3.0]}] \lmbox
   Takes the value \verb+[-3.0,3.0]+ for a new dataset.
\item [{\htmlref{XL}{XL} (=[0,0])}] \lmbox
   No default value.  The limits are taken as the full x-range in the data
   to be plotted if no value of \verb+YL+ is set.
\item [{\htmlref{XP}{XP} (=[0,0])}] \lmbox
   No default value.  The limits are taken as the full image width if \verb+YP+
   is not set.
\item [\htmlref{YEAR}{YEAR}] \lmbox
   No default value.  The program will attempt to extract the \verb+YEAR+ from
   the IUE GO tape/file header and present this as the default value for that
   particular command.
\item [{\htmlref{YL}{YL} (=[0,0])}] \lmbox
   No default value.  The limits are taken as the full y-range in the data
   to be plotted if no value of \verb+YL+ is set.
\item [{\htmlref{YP}{YP} (=[0,0])}] \lmbox
   No default value.  The limits are taken as the full image height if
   \verb+YP+ is not set.
\item [{\htmlref{ZL}{ZL} (=[0,0])}] \lmbox
   No default value.  The limits are taken as the full intensity range for the
   current \verb+DATASET+ if \verb+ZL+ is not set.
\item [\htmlref{ZONE}{ZONE} = 0] \lmbox
   Graphics zone to be used for plotting.

\end {description}

\newpage
\section{\xlabel{vms_data}\label{se:vmsunix}Handling of VMS IUEDR data files}
\markboth{Handling of VMS IUEDR data files}{\stardocname}

IUEDR data files have changed format in order to allow
inter-machine operation. However, VMS IUEDR will still read the old format
files if they are present (this is only useful on the VAX as all old
format files will have been created on VAXen). If you have old format files
then you should use \verb+iuecnv+ to convert them to the new format by
following the procedure described below.
The resulting files can then be used with UNIX IUEDR.

During the conversion of an IUE dataset \verb+iuecnv+ will create a number of
binary data files. Their filenames are as follows:

\begin {description}
   \item \verb+<dataset>.UEC+ --- calibration file.
   \item \verb+<dataset>_UED.SDF+ --- image data and quality file.
   \item \verb+<dataset>_UES.SDF+ --- uncalibrated spectrum file.
   \item \verb+<dataset>_UEM.SDF+ --- calibrated mean spectrum file.
\end {description}

where \verb+<dataset>+ refers to the IUEDR DATASET parameter.

The {\tt .SDF} files are STARLINK NDF format files and can be read and
processed by any of the standard  packages ({\it{e.g.,}}\ KAPPA,FIGARO).
These files are in a machine independent format and can be freely
copied between any of the platforms which STARLINK supports.

\subsection{Moving IUEDR files to UNIX systems}

The file formats used by UNIX IUEDR are based on the STARLINK standard
NDF library. This makes the files transportable between all supported
systems. If you have old VMS IUEDR files ({\it{i.e.}}\ {\tt .UEC}, {\tt
.UED}, {\tt .UES}, {\tt .UEM} files) then these will need to be
converted into the new format {\bf before} they are transferred to a UNIX
system.

There is a VMS executable provided for this purpose, and a command file
to use it. To use the executable you must first copy:

\begin{verbatim}
   /star/bin/iuedr/iuecnv.exe
   /star/bin/iuedr/iuecnv.com
\end{verbatim}

onto your VMS system (use binary transfer for {\tt iuecnv.exe}).

When {\tt iuecnv.exe} is installed, you can then move to a directory
where your IUEDR datasets are stored and type:

\begin{verbatim}
   $ @somedisk:[somedir]iuecnv  dataset
\end{verbatim}

where {\tt somedisk:[somedir]} is wherever you copied {\tt iuecnv} to, and
{\tt dataset} is the name of an IUE dataset ({\it{e.g.,}}\ SWP23456).

The program will then locate and convert all the IUEDR datafiles
associated with the named dataset. Note that the {\tt .UEC} file is also
converted (although its name stays the same).

When conversion is complete you may copy the files ({\tt .UEC} and {\tt
*\_UE*.SDF}) to your UNIX system. The {\tt .UEC} files MUST be
transferred in ascii mode, and the {\tt .SDF} files MUST be transferred
in binary mode.

UNIX NDF expects that NDF container files end in the extension \verb+.sdf+ and
does not yet recognise \verb+.SDF+ files. Thus you may need to rename
files to have the lowercase \verb+.sdf+ extension (depending upon how
you do the transfer).

\subsection{VAX-UNIX IUEDR image file exchange}

An IUEDR image file is one of \verb+RAW+, \verb+PHOT+, or \verb+GPHOT+ type
and consist of 768 $\times$ 768 pixels each stored in a 1 or 2-byte integer.

The transfer of files between VAX and UNIX systems is complicated by
the sophistication of the VAX file system. Under VMS the system
records a complex description of the precise format of all the  files
(and stores it in  the directory entry). Under UNIX this information
has to be provided by the user of the file when it is  opened. Because
of this difference it is sometimes necessary to use the following
format conversion utilities.

\subsubsection{VAX to UNIX}

If you wish to transfer IUE image data from a VAX onto a UNIX machine
in order to use the UNIX IUEDR then the transfer should be done using
FTP (in {\bf binary} mode).

If you intend to copy the file using some other method ({\it{e.g.,}}\ via NFS)
then you should first use the command:

\begin{verbatim}
   UNIX_FORMAT image-name
\end{verbatim}

to ensure the file is properly transferred.

Note that this also applies if you wish to just access the file
via NFS without explicitly transferring it.

\subsubsection{UNIX to VAX}

If you wish to transfer IUE image data from a UNIX machine onto a VAX
in order to use the VAX IUEDR then the transfer should be done using
FTP (in {\bf binary} mode) and the command:

\begin{verbatim}
   VAX_FORMAT  image-name
\end{verbatim}

should then be used to ensure the file has the correct format.

If you use some other method of transferring the file ({\it{e.g.,}}\ NFS) then
the above command is {\bf still} required.

\subsubsection{What will work?}

In general the following two commands will allow you to use any disk
based IUE image with any machine running IUEDR:

\begin{itemize}
   \item {\tt VAX\_FORMAT} sets the file format as required by VAX IUEDR
   \item {\tt UNIX\_FORMAT} sets the file format as required by UNIX IUEDR
\end{itemize}

Both commands operate only on the VAX.

\subsection{\label{se:nfs}Accessing data via NFS}

UNIX machines commonly provide disk sharing amongst remote machines
using the NFS protocol.

For example your data frame may reside on a DECstation local  disk
called \verb+iuedata+ in the Rutherford cluster on machine \verb+adam4+\@. In
order to get IUEDR to read it directly you could enter the following
in response to the DRIVE prompt:

\begin{verbatim}
   DRIVE> /adam4/iuedata/swp12345.raw
\end{verbatim}

To see which disks you have NFS access to you should use the {\tt
\%~df} command. In general any disks whose entry does not start with
\verb+/dev+ are being served by a remote machine.

{\bf IMPORTANT NOTE\\}
IUEDR allows you to use this method of data access with the following proviso.

If the data resides on a VAX served disk then you must first  convert
its directory entry (on the VAX) using the following command:

\begin{verbatim}
   $ UNIX_FORMAT image-name
\end{verbatim}

This command does not change the data in any way.  It merely alters the
description of the file format as stored in the VAX directory.

If at some later stage you wish to use the VAX version of IUEDR on the
same data file it first be necessary to use the command:

\begin{verbatim}
   $ VAX_FORMAT image-name
\end{verbatim}

to convert back.

\twocolumn[
\section{\label{se:index}Command and parameter index}
]
\markboth{Index}{\stardocname}
\findexentry{A}{ABSFILE}{46}
\indexentry{AESHIFT}{14}
\indexentry{AGSHIFT}{14}
\indexentry{APERTURES}{46}
\indexentry{APERTURE}{46}
\indexentry{AUTOSLIT}{47}
\findexentry{B}{BADITF}{47}
\indexentry{BARKER}{15}
\indexentry{BDIST}{47}
\indexentry{BKGAV}{47}
\indexentry{BKGIT}{47}
\indexentry{BKGSD}{48}
\indexentry{BSLIT}{48}
\findexentry{C}{CAMERA}{48}
\indexentry{CENAV}{48}
\indexentry{CENIT}{48}
\indexentry{CENSD}{48}
\indexentry{CENSH}{48}
\indexentry{CENSV}{48}
\indexentry{CENTM}{49}
\indexentry{CENTREWAVE}{49}
\indexentry{CGSHIFT}{15}
\indexentry{CLEAN}{16}
\indexentry{COLOUR}{49}
\indexentry{COLROT}{49}
\indexentry{COL}{49}
\indexentry{CONTINUUM}{50}
\indexentry{COVERGAP}{50}
\indexentry{CULIMITS}{16}
\indexentry{CURSOR}{17}
\indexentry{CUTFILE}{50}
\indexentry{CUTWV}{50}
\findexentry{D}{DATASET}{50}
\indexentry{DAY}{50}
\indexentry{DELTAWAVE}{50}
\indexentry{DEVICE}{50}
\indexentry{DISPFILE}{50}
\indexentry{DRIMAGE}{17}
\indexentry{DRIVE}{51}
\indexentry{EDIMAGE}{18}
\findexentry{E}{EDMEAN}{19}
\indexentry{EDSPEC}{20}
\indexentry{ERASE}{20}
\indexentry{ESHIFT}{51}
\indexentry{EXIT}{21}
\indexentry{EXPOSURES}{51}
\indexentry{EXPOSURE}{51}
\indexentry{EXTENDED}{51}
\findexentry{F}{FIDFILE}{51}
\indexentry{FIDSIZE}{51}
\indexentry{FILE}{51}
\indexentry{FILLGAP}{52}
\indexentry{FLAG}{52}
\indexentry{FN}{52}
\indexentry{FSCALE}{52}
\findexentry{G}{GSAMP}{52}
\indexentry{GSHIFT}{52}
\indexentry{GSLIT}{53}
\findexentry{H}{HALAV}{53}
\indexentry{HALC}{53}
\indexentry{HALTYPE}{53}
\indexentry{HALW0}{53}
\indexentry{HALWC}{53}
\indexentry{HELP}{21}
\indexentry{HIST}{53}
\findexentry{I}{IMAGE}{53}
\indexentry{ITFMAX}{54}
\indexentry{ITF}{54}
\findexentry{L}{LBLS}{21}
\indexentry{LINEROT}{55}
\indexentry{LINE}{54}
\indexentry{LISTIUE}{22}
\findexentry{M}{MAP}{23}
\indexentry{ML}{55}
\indexentry{MODIMAGE}{23}
\indexentry{MONTH}{55}
\indexentry{MSAMP}{55}
\indexentry{MTMOVE}{24}
\indexentry{MTREW}{24}
\indexentry{MTSHOW}{24}
\indexentry{MTSKIPEOV}{25}
\indexentry{MTSKIPF}{25}
\findexentry{N}{NEWABS}{25}
\indexentry{NEWCUT}{26}
\indexentry{NEWDISP}{26}
\indexentry{NEWFID}{26}
\indexentry{NEWRIP}{27}
\indexentry{NEWTEM}{27}
\indexentry{NFILE}{55}
\indexentry{NGEOM}{55}
\indexentry{NLINE}{55}
\indexentry{NORDER}{55}
\indexentry{NSKIP}{55}
\findexentry{O}{OBJECT}{56}
\indexentry{ORDERS}{56}
\indexentry{ORDER}{56}
\indexentry{OUTEM}{27}
\indexentry{OUTFILE}{56}
\indexentry{OUTLBLS}{28}
\indexentry{OUTMEAN}{28}
\indexentry{OUTNET}{29}
\indexentry{OUTRAK}{29}
\indexentry{OUTSCAN}{30}
\indexentry{OUTSPEC}{30}
\findexentry{P}{PLCEN}{31}
\indexentry{PLFLUX}{32}
\indexentry{PLGRS}{33}
\indexentry{PLMEAN}{34}
\indexentry{PLNET}{35}
\indexentry{PLSCAN}{36}
\indexentry{PRGRS}{37}
\indexentry{PRLBLS}{37}
\indexentry{PRMEAN}{37}
\indexentry{PRSCAN}{38}
\indexentry{PRSPEC}{38}
\findexentry{Q}{QUAL}{56}
\indexentry{QUIT}{38}
\findexentry{R}{READIUE}{39}
\indexentry{READSIPS}{40}
\indexentry{RESOLUTION}{56}
\indexentry{RIPA}{57}
\indexentry{RIPC}{57}
\indexentry{RIPFILE}{57}
\indexentry{RIPK}{57}
\indexentry{RL}{58}
\indexentry{RM}{58}
\indexentry{RSAMP}{58}
\indexentry{RS}{58}
\findexentry{S}{SAVE}{40}
\indexentry{SCANAV}{58}
\indexentry{SCANDIST}{58}
\indexentry{SCANWV}{58}
\indexentry{SCAN}{41}
\indexentry{SETA}{41}
\indexentry{SETD}{42}
\indexentry{SETM}{42}
\indexentry{SGS}{43}
\indexentry{SHOW}{43}
\indexentry{SKIPNEXT}{58}
\indexentry{SPECTYPE}{59}
\findexentry{T}{TEMFILE}{59}
\indexentry{THDA}{59}
\indexentry{THRESH}{59}
\indexentry{TRAK}{44}
\indexentry{TYPE}{59}
\findexentry{V}{VSHIFT}{60}
\indexentry{V}{59}
\findexentry{W}{WCUT}{60}
\indexentry{WSHIFT}{60}
\findexentry{X}{XCUT}{60}
\indexentry{XL}{61}
\indexentry{XP}{61}
\findexentry{Y}{YEAR}{61}
\indexentry{YL}{61}
\indexentry{YP}{61}
\findexentry{Z}{ZL}{61}
\indexentry{ZONE}{61}

%\documentstyle[11pt,twoside]{article}
\pagestyle{myheadings}
\makeindex

%------------------------------------------------------------------------------
\newcommand{\stardoccategory}  {Starlink Guide}
\newcommand{\stardocinitials}  {SG}
\newcommand{\stardocsource}    {sg3.5}
\newcommand{\stardocnumber}    {3.5}
\newcommand{\stardocauthors}   {Paul Rees, Jack Giddings, Dave Mills \& Martin Clayton}
\newcommand{\stardocdate}      {12 March 1996}
\newcommand{\stardoctitle}     {IUEDR---Reference Manual}
%------------------------------------------------------------------------------


\newcommand{\stardocname}{\stardocinitials /\stardocnumber}
\newcommand{\numcir}[1]{\mbox{\hspace{3ex}$\bigcirc$\hspace{-1.7ex}{\small #1}}}
\newcommand{\lsk}{\raisebox{-0.4ex}{\rm *}}
%\renewcommand{\_}{{\tt\char'137}}     % re-centres the underscore - DONE LATER
\markright{\stardocname}
\setlength{\textwidth}{160mm}
\setlength{\textheight}{230mm}
\setlength{\topmargin}{-2mm}
\setlength{\oddsidemargin}{0mm}
\setlength{\evensidemargin}{0mm}
\setlength{\parindent}{0mm}
\setlength{\parskip}{\medskipamount}
\setlength{\unitlength}{1mm}


% -----------------------------------------------------------------------------
%  Hypertext definitions.
%  ======================
%  These are used by the LaTeX2HTML translator in conjuction with star2html.

%  Comment.sty: version 2.0, 19 June 1992
%  Selectively in/exclude pieces of text.
%
%  Author
%    Victor Eijkhout                                      <eijkhout@cs.utk.edu>
%    Department of Computer Science
%    University Tennessee at Knoxville
%    104 Ayres Hall
%    Knoxville, TN 37996
%    USA

%  Do not remove the %\begin{rawtex} and %\end{rawtex} lines (used by
%  star2html to signify raw TeX that latex2html cannot process).
%\begin{rawtex}
\makeatletter
\def\makeinnocent#1{\catcode`#1=12 }
\def\csarg#1#2{\expandafter#1\csname#2\endcsname}

\def\ThrowAwayComment#1{\begingroup
    \def\CurrentComment{#1}%
    \let\do\makeinnocent \dospecials
    \makeinnocent\^^L% and whatever other special cases
    \endlinechar`\^^M \catcode`\^^M=12 \xComment}
{\catcode`\^^M=12 \endlinechar=-1 %
 \gdef\xComment#1^^M{\def\test{#1}
      \csarg\ifx{PlainEnd\CurrentComment Test}\test
          \let\html@next\endgroup
      \else \csarg\ifx{LaLaEnd\CurrentComment Test}\test
            \edef\html@next{\endgroup\noexpand\end{\CurrentComment}}
      \else \let\html@next\xComment
      \fi \fi \html@next}
}
\makeatother

\def\includecomment
 #1{\expandafter\def\csname#1\endcsname{}%
    \expandafter\def\csname end#1\endcsname{}}
\def\excludecomment
 #1{\expandafter\def\csname#1\endcsname{\ThrowAwayComment{#1}}%
    {\escapechar=-1\relax
     \csarg\xdef{PlainEnd#1Test}{\string\\end#1}%
     \csarg\xdef{LaLaEnd#1Test}{\string\\end\string\{#1\string\}}%
    }}

%  Define environments that ignore their contents.
\excludecomment{comment}
\excludecomment{rawhtml}
\excludecomment{htmlonly}
%\end{rawtex}

%  Hypertext commands etc. This is a condensed version of the html.sty
%  file supplied with LaTeX2HTML by: Nikos Drakos <nikos@cbl.leeds.ac.uk> &
%  Jelle van Zeijl <jvzeijl@isou17.estec.esa.nl>. The LaTeX2HTML documentation
%  should be consulted about all commands (and the environments defined above)
%  except \xref and \xlabel which are Starlink specific.

\newcommand{\htmladdnormallinkfoot}[2]{#1\footnote{#2}}
\newcommand{\htmladdnormallink}[2]{#1}
\newcommand{\htmladdimg}[1]{}
\newenvironment{latexonly}{}{}
\newcommand{\hyperref}[4]{#2\ref{#4}#3}
\newcommand{\htmlref}[2]{#1}
\newcommand{\htmlimage}[1]{}
\newcommand{\htmladdtonavigation}[1]{}

%  Starlink cross-references and labels.
\newcommand{\xref}[3]{#1}
\newcommand{\xlabel}[1]{}

%  LaTeX2HTML symbol.
\newcommand{\latextohtml}{{\bf LaTeX}{2}{\tt{HTML}}}

%  Define command to recentre underscore for Latex and leave as normal
%  for HTML (severe problems with \_ in tabbing environments and \_\_
%  generally otherwise).
\newcommand{\latex}[1]{#1}
\newcommand{\setunderscore}{\renewcommand{\_}{{\tt\symbol{95}}}}
\latex{\setunderscore}

%  Redefine the \tableofcontents command. This procrastination is necessary
%  to stop the automatic creation of a second table of contents page
%  by latex2html.
\newcommand{\latexonlytoc}[0]{\tableofcontents}

% -----------------------------------------------------------------------------
%  Debugging.
%  =========
%  Un-comment the following to debug links in the HTML version using Latex.

% \newcommand{\hotlink}[2]{\fbox{\begin{tabular}[t]{@{}c@{}}#1\\\hline{\footnotesize #2}\end{tabular}}}
% \renewcommand{\htmladdnormallinkfoot}[2]{\hotlink{#1}{#2}}
% \renewcommand{\htmladdnormallink}[2]{\hotlink{#1}{#2}}
% \renewcommand{\hyperref}[4]{\hotlink{#1}{\S\ref{#4}}}
% \renewcommand{\htmlref}[2]{\hotlink{#1}{\S\ref{#2}}}
% \renewcommand{\xref}[3]{\hotlink{#1}{#2 -- #3}}
% -----------------------------------------------------------------------------
%  Add any document specific \newcommand or \newenvironment commands here

\newcommand{\lmbox}
{
    \mbox{} \\
}

\newcommand{\cpar}[2]
{
    \makebox[30mm][l]{\bf #1} & #2 (p~\pageref{#1}.)\\
}

\newcommand{\cparc}[1]
{
    \makebox[30mm][l]{ } & #1\\
}

\newcommand{\npar}[1]
{
    \makebox[30mm][l]{\bf #1} & \\
}

\newcommand{\iueparlist}[1]{
   \begin{description}
      #1
   \end{description}
}

\newcommand{\iueparameter}[3]
{
   \item [\label{#1}\index{#1}#1 = #2] \mbox{}\\
   #3
}

\newcommand{\indexentry}[2]
{
{\bf #1}\dotfill #2 \hspace*{15mm}\\
}

\newcommand{\findexentry}[3]
{
   \hspace*{\fill}\vspace*{3mm}\\
   \hspace*{\fill}{\large\bf --- #1 ---}\hspace*{\fill} \hspace*{15mm}\\
   \hspace*{\fill}\vspace*{-3mm}\\
   {\bf #2}\dotfill #3 \hspace*{15mm}\\
}

\newcommand{\comdescenv}[1]
{
\begin {tabular}{ll}
  #1
\end {tabular}
}

\newcommand{\comdesc}[2]
{
   \makebox[27mm][l]{\bf #1} & #2 \\
}

\newcommand{\comdescc}[1]
{
   \makebox[27mm][l]{ } & #1 \\
}


%% Redefine commands for hypertext version.

\begin{htmlonly}

\renewcommand{\lmbox}
{ }

\renewcommand{\cpar}[2]
{
    \item [\htmlref{#1}{#1}] #2
}

\renewcommand{\cparc}[1]
{
  #1
}

\rerenewcommand{\npar}[1]
{
    \item [#1]
}

\renewcommand{\indexentry}[2]
{
    {\bf \htmlref{#1}{#1}}\\
}

\renewcommand{\findexentry}[3]
{
    {\bf \htmlref{#2}{#2}}\\
}

\renewcommand{\comdescenv}[1]
{
   \begin{description}
       #1
   \end{description}
}

\renewcommand{\comdesc}[2]
{
   \item [{\bf \htmlref{#1}{#1}}] #2
}

\newcommand{\comdescc}[1]
{
  #1
}

\renewcommand{\iueparlist}[1]
{
      #1
}

\renewcommand{\iueparameter}[3]
{
\subsection{\xlabel{#1}\label{#1}#1}
   \begin{description}
   \item [{\bf Type:}] #2
   \item [{\bf Description:}] #3
   \end{description}
}

\end{htmlonly}

%+
%  Name:
%     SST.TEX

%  Purpose:
%     Define LaTeX commands for laying out Starlink routine descriptions.

%  Language:
%     LaTeX

%  Type of Module:
%     LaTeX data file.

%  Description:
%     This file defines LaTeX commands which allow routine documentation
%     produced by the SST application PROLAT to be processed by LaTeX and
%     by LaTeX2html. The contents of this file should be included in the
%     source prior to any statements that make of the sst commnds.

%  Notes:
%     The commands defined in the style file html.sty provided with LaTeX2html
%     are used. These should either be made available by using the appropriate
%     sun.tex (with hypertext extensions) or by putting the file html.sty
%     on your TEXINPUTS path (and including the name as part of the
%     documentstyle declaration).

%  Authors:
%     RFWS: R.F. Warren-Smith (STARLINK)
%     PDRAPER: P.W. Draper (Starlink - Durham University)

%  History:
%     10-SEP-1990 (RFWS):
%        Original version.
%     10-SEP-1990 (RFWS):
%        Added the implementation status section.
%     12-SEP-1990 (RFWS):
%        Added support for the usage section and adjusted various spacings.
%     8-DEC-1994 (PDRAPER):
%        Added support for simplified formatting using LaTeX2html.
%     {enter_further_changes_here}

%  Bugs:
%     {note_any_bugs_here}

% -

%  Define length variables.
\newlength{\sstbannerlength}
\newlength{\sstcaptionlength}
\newlength{\sstexampleslength}
\newlength{\sstexampleswidth}

%  Define a \tt font of the required size.
\newfont{\ssttt}{cmtt10 scaled 1095}

%  Define a command to produce a routine header, including its name,
%  a purpose description and the rest of the routine's documentation.
\newcommand{\sstroutine}[3]{
   \goodbreak
   \rule{\textwidth}{0.5mm}
   \vspace{-7ex}
   \newline
   \settowidth{\sstbannerlength}{{\Large {\bf #1}}}
   \setlength{\sstcaptionlength}{\textwidth}
   \setlength{\sstexampleslength}{\textwidth}
   \addtolength{\sstbannerlength}{0.5em}
   \addtolength{\sstcaptionlength}{-2.0\sstbannerlength}
   \addtolength{\sstcaptionlength}{-5.0pt}
   \settowidth{\sstexampleswidth}{{\bf Examples:}}
   \addtolength{\sstexampleslength}{-\sstexampleswidth}
   \parbox[t]{\sstbannerlength}{\flushleft{\Large {\bf #1}}}
   \parbox[t]{\sstcaptionlength}{\center{\Large #2}}
   \parbox[t]{\sstbannerlength}{\flushright{\Large {\bf #1}}}
   \label{#1}\index{#1}
   \begin{description}
      #3
   \end{description}
}

%  Format the description section.
\newcommand{\sstdescription}[1]{\item {\bf Description:}\vspace*{6pt}\\ #1}

%  Format the usage section.
\newcommand{\sstusage}[1]{\item[Usage:] \mbox{} \\[1.3ex] {\ssttt #1}}

%  Format the invocation section.
\newcommand{\sstinvocation}[1]{\item[Invocation:]\hspace{0.4em}{\tt #1}}

%  Format the arguments section.
\newcommand{\sstarguments}[1]
{
   \item[Arguments:] \mbox{} \\
   \vspace{-3.5ex}
   \begin{description}
      #1
   \end{description}
}

%  Format the returned value section (for a function).
\newcommand{\sstreturnedvalue}[1]{
   \item[Returned Value:] \mbox{} \\
   \vspace{-3.5ex}
   \begin{description}
      #1
   \end{description}
}

%  Format the parameters section (for an application).
\newcommand{\sstparameters}[1]{
\item {\bf Parameters:\vspace*{6pt}\\}
    \begin{tabular}{ll}
    #1
    \end{tabular}
}

%  Format the examples section.
\newcommand{\sstexamples}[1]{
   \item[Examples:] \mbox{} \\
   \vspace{-3.5ex}
   \begin{description}
      #1
   \end{description}
}

%  Define the format of a subsection in a normal section.
\newcommand{\sstsubsection}[1]{ \item[{#1}] \mbox{} \\}

%  Define the format of a subsection in the examples section.
\newcommand{\sstexamplesubsection}[2]{\sloppy
\item[\parbox{\sstexampleslength}{\ssttt #1}] \mbox{} \\ #2 }

%  Format the notes section.
\newcommand{\sstnotes}[1]{\item[Notes:] \mbox{} \\[1.3ex] #1}

%  Provide a general-purpose format for additional (DIY) sections.
\newcommand{\sstdiytopic}[2]{\item[{\hspace{-0.35em}#1\hspace{-0.35em}:}] \mbox{} \\[1.3ex] #2}

%  Format the implementation status section.
\newcommand{\sstimplementationstatus}[1]{
   \item[{Implementation Status:}] \mbox{} \\[1.3ex] #1}

%  Format the bugs section.
\newcommand{\sstbugs}[1]{\item[Bugs:] #1}

%  Format a list of items while in paragraph mode.
\newcommand{\sstitemlist}[1]{
  \mbox{} \\
  \vspace{-3.5ex}
  \begin{itemize}
     #1
  \end{itemize}
}

%  Define the format of an item.
\newcommand{\sstitem}{\item}

%% Now define html equivalents of those already set. These are used by
%  latex2html and are defined in the html.sty files.

\begin{htmlonly}

%  Re-define \ssttt.
   \newcommand{\ssttt}{\tt}

%  sstroutine.
   \renewcommand{\sstroutine}[3]{
\subsection{\xlabel{#1}\label{#1}#1}
      \begin{description}
         \item[{\bf Purpose:}] #2
         #3
      \end{description}
   }

%  sstdescription
   \renewcommand{\sstdescription}[1]{
      \item[{\bf Description:}]
      \begin{description}
         #1
      \end{description}
   }

%  sstusage
   \renewcommand{\sstusage}[1]{\item[Usage:]
      \begin{description}
         {\ssttt #1}
      \end{description}
   }

%  sstinvocation
   \renewcommand{\sstinvocation}[1]{\item[Invocation:]
      \begin{description}
         {\ssttt #1}
      \end{description}
   }

%  sstarguments
   \renewcommand{\sstarguments}[1]{
      \item[Arguments:]
      \begin{description}
         #1
      \end{description}
   }

%  sstreturnedvalue
   \renewcommand{\sstreturnedvalue}[1]{
      \item[Returned Value:]
      \begin{description}
         #1
      \end{description}
   }

%  sstparameters
   \renewcommand{\sstparameters}[1]{
      \item[{\bf Parameters:}]
      \begin{description}
         #1
      \end{description}
   }

%  sstexamples
   \renewcommand{\sstexamples}[1]{
      \item[Examples:]
      \begin{description}
         #1
      \end{description}
   }

%  sstsubsection
   \renewcommand{\sstsubsection}[1]{\item[{#1}]}

%  sstexamplesubsection
   \renewcommand{\sstexamplesubsection}[2]{\item[{\ssttt #1}] \\ #2}

%  sstnotes
   \renewcommand{\sstnotes}[1]{\item[Notes:]
      \begin{description}
         #1
      \end{description}
   }

%  sstdiytopic
   \renewcommand{\sstdiytopic}[2]{\item[{#1}]
      \begin{description}
         #2
      \end{description}
   }

%  sstimplementationstatus
   \renewcommand{\sstimplementationstatus}[1]{\item[Implementation Status:]
      \begin{description}
         #1
      \end{description}
   }

%  sstitemlist
   \newcommand{\sstitemlist}[1]{
      \begin{itemize}
         #1
      \end{itemize}
   }
\end{htmlonly}

%  End of "sst.tex" layout definitions.

% -----------------------------------------------------------------------------
%  Title Page.
%  ===========
\renewcommand{\thepage}{\roman{page}}
\begin{document}
\thispagestyle{empty}
%  Latex document header.
%  ======================
\begin{latexonly}
   CCLRC / {\sc Rutherford Appleton Laboratory} \hfill {\bf \stardocname}\\
   {\large Particle Physics \& Astronomy Research Council}\\
   {\large Starlink Project\\}
   {\large \stardoccategory\ \stardocnumber}
   \begin{flushright}
   \stardocauthors\\
   \stardocdate
   \end{flushright}
   \vspace{-4mm}
   \rule{\textwidth}{0.5mm}
   \vspace{5mm}
   \begin{center}
   {\Large\bf \stardoctitle}
   \end{center}
   \vspace{5mm}

%  Add heading for abstract if used.
%   \vspace{10mm}
%   \begin{center}
%      {\Large\bf Description}
%   \end{center}
\end{latexonly}

%  HTML documentation header.
%  ==========================
\begin{htmlonly}
   \xlabel{}
   \begin{rawhtml} <H1> \end{rawhtml}
      \stardoctitle
   \begin{rawhtml} </H1> \end{rawhtml}

%  Add picture here if required.

   \begin{rawhtml} <P> <I> \end{rawhtml}
   \stardoccategory \stardocnumber \\
   \stardocauthors \\
   \stardocdate
   \begin{rawhtml} </I> </P> <H3> \end{rawhtml}
      \htmladdnormallink{CCLRC}{http://www.cclrc.ac.uk} /
      \htmladdnormallink{Rutherford Appleton Laboratory}
                        {http://www.cclrc.ac.uk/ral} \\
      Particle Physics \& Astronomy Research Council \\
   \begin{rawhtml} </H3> <H2> \end{rawhtml}
      \htmladdnormallink{Starlink Project}{http://star-www.rl.ac.uk/}
   \begin{rawhtml} </H2> \end{rawhtml}
   \htmladdnormallink{\htmladdimg{source.gif} Retrieve hardcopy}
      {http://star-www.rl.ac.uk/cgi-bin/hcserver?\stardocsource}\\

%  HTML document table of contents.
%  ================================
%  Add table of contents header and a navigation button to return to this
%  point in the document (this should always go before the abstract \section).
  \label{stardoccontents}
  \begin{rawhtml}
    <HR>
    <H2>Contents</H2>
  \end{rawhtml}
  \renewcommand{\latexonlytoc}[0]{}
  \htmladdtonavigation{\htmlref{\htmladdimg{contents_motif.gif}}
        {stardoccontents}}

%  Start new section for abstract if used.
%  \section{\xlabel{abstract}Abstract}

\end{htmlonly}

% -----------------------------------------------------------------------------
%  Document Abstract. (if used)
%  ==================
% -----------------------------------------------------------------------------
%  Latex document Table of Contents (if used).
%  ===========================================
\begin{latexonly}
   \setlength{\parskip}{0mm}
   \latexonlytoc
   \setlength{\parskip}{\medskipamount}
   \markright{\stardocname}
\end{latexonly}
% -----------------------------------------------------------------------------

%%%%%%%%%%%%%%%%%%%%%%%%%%%%%%%%%%%%%%%%%%%%%%%%%%%%%%%%%%%%%%%%%%%%%%%%%%%
\newpage
\renewcommand{\thepage}{\arabic{page}}
\setcounter{page}{1}
\section{\xlabel{introduction}\label{se:introduction}Introduction }
\markboth{Introduction}{\stardocname}

This manual describes the commands and parameters used by IUEDR\@.
It is intended as a reference aid for people using IUEDR\@.

If you are new to IUE data reduction, you may like to read
\xref{{\sl IUE Analysis
a Tutorial}}{sg7}{} (SG/7) and the
\xref{{\sl IUEDR User Guide}}{mud45}{} (MUD/45) before proceeding
any further.
The Starlink User Note \xref{SUN/37}{sun37}{} contains a general description
of IUEDR which
overlaps with the early sections of this manual and also contains any notes on
the most recent release of the program.

Commands are described in Section~\ref{se:commands}, with parameters described
in more detail in Section~\ref{se:parameters}\@.
A list of default parameter behaviour and values is given in
Appendix~\ref{se:parameter_defaults}\@.
Details of transferring old-style IUEDR files from VMS systems to UNIX systems
and the conversion process are given in Appendix~\ref{se:vmsunix}\@.

\begin{latexonly}
An index of both commands and parameters is given in Appendix~\ref{se:index}\@.
\end{latexonly}

IUEDR functions fall into a number of specific categories:

\begin {itemize}
   \item IUE GO tape or file inspection and reading.
   \item Data display and manipulation.
   \item Spectrum extraction and calibration.
   \item Extraction product inspection and manipulation.
   \item Extraction product output.
   \item General operational commands.
\end {itemize}

These functions are controlled by over fifty commands, with nearly one hundred
global parameters within IUEDR\@.
There follows a summary of the commands available in each of the categories
listed above.

\subsection {IUE GO tape or file inspection and reading}

\comdescenv{
   \comdesc{LISTIUE}{Analyse the contents of one or more IUE tape files.}
   \comdesc{MTMOVE}{Move to the start of a tape file.}
   \comdesc{MTREW}{Rewind to the start of the tape.}
   \comdesc{MTSHOW}{Show the current tape position.}
   \comdesc{MTSKIPEOV}{Skip over the end-of-volume mark.}
   \comdesc{MTSKIPF}{Skip over NSKIP tape marks.}
   \comdesc{READIUE}{Read a RAW, GPHOT or PHOT IUE image from the tape/file.}
   \comdesc{READSIPS}{Read the MELO or MEHI IUESIPS product from the tape/file.}
}

\subsection {Data display and manipulation}

\comdescenv{
   \comdesc{CULIMITS}{Delineate the graphical display limits using the
                      graphics cursor.}
   \comdesc{CURSOR}{Determine display coordinates using the graphics cursor}
           \comdescc{and print them at the terminal.}
   \comdesc{DRIMAGE}{Display an IUE image on a suitable graphics workstation.}
   \comdesc{EDIMAGE}{Edit the image data quality using the graphics cursor.}
   \comdesc{MODIMAGE}{Modify image pixel intensities interactively.}
   \comdesc{CLEAN}{Mark as `bad' pixels with value below a given threshold.}
   \comdesc{SHOW}{Print information relating to the current dataset at the
                  terminal.}
   \comdesc{ERASE}{Erase the display screen of the current graphics
                   workstation.}
}

Image displays are colour coded to provide data quality information.
The colour codes used by IUEDR are as follows:

\begin{latexonly}
\begin {quote}
\begin {description}
   \item [Green] pixels affected by reseau marks
   \item [Red] pixels which are saturated (DN=255)
   \item [Orange] pixels affected by ITF truncation
   \item [Yellow] pixels marked bad by the user
\end {description}
\end {quote}
\end{latexonly}

\begin{htmlonly}
\begin{rawhtml}
<PRE>
   <B>Green</B>  - pixels affected by reseau marks
   <B>Red</B>    - pixels which are saturated (DN=255)
   <B>Orange</B> - pixels affected by ITF truncation
   <B>Yellow</B> - pixels marked bad by the user
</PRE>
\end{rawhtml}
\end{htmlonly}

The colour {\bf Blue} is used to indicate a pixel which has a value above the
maximum that can be displayed using the linear greyscale image display colour
look-up table.

When using a mouse or tracker-ball with the graphics cursor, the cursor hit
buttons are normally numbered in increasing order from left to right.
For example the left mouse button corresponds to cursor key hit 1, middle
button to cursor key hit 2 and so on.
Many terminals allow left-handed users to reverse the mouse button order.

\subsection {Spectrum extraction and calibration}

\comdescenv{
   \comdesc{AESHIFT}{Determine (HIRES) spectrum ESHIFT automatically.}
   \comdesc{AGSHIFT}{Determine spectrum template shift automatically.}
   \comdesc{BARKER}{Correct the extracted data for \'{e}chelle ripple}
           \comdescc{using a method based upon that of Barker (1984).}
   \comdesc{CGSHIFT}{Determine spectrum template shift using the cursor
                     on a SCAN plot.}
   \comdesc{LBLS}{Extract a line-by-line-spectrum array from the image.}
   \comdesc{NEWABS}{Associate a new absolute flux calibration with the
                  current  dataset.}
   \comdesc{NEWCUT}{Associate new \'{e}chelle order wavelength limits with the
                  current  dataset.}
   \comdesc{NEWDISP}{Associate new spectrograph dispersion data with the
                   current  dataset.}
   \comdesc{NEWFID}{Associate new fiducial positions with the current
                  dataset.}
   \comdesc{NEWRIP}{Associate new ripple calibration data with the current
                  dataset.}
   \comdesc{NEWTEM}{Associate new spectrum centroid template data with the
                    current dataset.}
   \comdesc{SCAN}{Perform a scan of the image data perpendicular}
           \comdescc{to the spectrograph dispersion.}
   \comdesc{SETA}{Set dataset parameters which are aperture specific.}
   \comdesc{SETD}{Set dataset parameters which are independent of order and
                aperture.}
   \comdesc{SETM}{Set dataset parameters which are order specific.}
   \comdesc{TRAK}{Extract a spectrum from the image.}
}

\subsection {Extraction product inspection and manipulation}

\comdescenv{
   \comdesc{EDMEAN}{Edit the mean extracted spectrum using the graphics
                  cursor.}
   \comdesc{EDSPEC}{Edit the net extracted spectrum using the graphics
                  cursor.}
   \comdesc{MAP}{Map and merge extracted spectrum components to produce}
           \comdescc{a mean spectrum.}
   \comdesc{PLCEN}{Plot the smoothed spectrum centroid shifts.}
   \comdesc{PLFLUX}{Plot the calibrated flux spectrum.}
   \comdesc{PLGRS}{Plot the pseudo-gross and background resulting from the}
           \comdescc{spectrum extraction.}
   \comdesc{PLMEAN}{Plot the mean spectrum.}
   \comdesc{PLNET}{Plot the uncalibrated net spectrum.}
   \comdesc{PLSCAN}{Plot the image scan perpendicular to the dispersion.}
   \comdesc{SGS}{Print the names of the available SGS graphics devices at
               the  terminal.}
}

Plots of extracted IUE spectra and image scans include data quality information
flags for bad data.
The data quality codes used by IUEDR are as follows:

\begin{latexonly}
\begin {quote}
\begin {description}
   \item [1] affected by extrapolated ITF
   \item [2] affected by microphonics
   \item [3] affected by noise spike
   \item [4] affected by bright point (or user)
   \item [5] affected by reseau mark
   \item [6] affected by ITF truncation
   \item [7] affected by saturation
   \item [U] affected by user edit
\end {description}
\end {quote}
\end{latexonly}

\begin{htmlonly}
\begin{rawhtml}
<PRE>
   <B>1</B> - affected by extrapolated ITF
   <B>2</B> - affected by microphonics
   <B>3</B> - affected by noise spike
   <B>4</B> - affected by bright point (or user)
   <B>5</B> - affected by reseau mark
   <B>6</B> - affected by ITF truncation
   <B>7</B> - affected by saturation
   <B>U</B> - affected by user edit
</PRE>
\end{rawhtml}
\end{htmlonly}

\subsection {Extraction product output}

\comdescenv{
   \comdesc{OUTEM}{Output the current spectrum template data to a formatted
                 data  file.}
   \comdesc{OUTLBLS}{Output the current LBLS array to a binary data file.}
   \comdesc{OUTMEAN}{Output the current mean spectrum to a DIPSO SP
                   format  data file.}
   \comdesc{OUTNET}{Output the current net spectrum to a DIPSO SP
                  format data file.}
   \comdesc{OUTRAK}{Output the current uncalibrated spectrum to a}
           \comdescc{``TRAK'' formatted data file.}
   \comdesc{OUTSCAN}{Output the current scan data to a DIPSO SP format
                   data  file.}
   \comdesc{OUTSPEC}{Output the current aperture (LORES) or order (HIRES)}
           \comdescc{spectrum to a DIPSO SP format data file.}
   \comdesc{PRGRS}{Print the current extracted aperture or order spectrum
                 in tabular form.}
   \comdesc{PRLBLS}{Print the current LBLS array in tabular form.}
   \comdesc{PRMEAN}{Print the current mean spectrum in tabular form.}
   \comdesc{PRSCAN}{Print the intensities of the current image scan in
                  tabular  form.}
   \comdesc{PRSPEC}{Print the current aperture or order spectrum in tabular
                  form.}
}

\subsection {General operational commands}

\comdescenv{
   \comdesc{EXIT}{Leave IUEDR and update any files altered during the}
           \comdescc{current IUEDR session.}
   \comdesc{QUIT}{Leave IUEDR and update any files altered during the}
           \comdescc{current IUEDR session.}
   \comdesc{SAVE}{Overwrite any files that have had their contents}
           \comdescc{updated during the current IUEDR session.}
}

A log of all commands typed during an IUEDR session and program output at the
terminal can be found in the file {\tt session.lis}.
This is particularly useful when investigating the contents of IUE tapes.

\newpage
\section{\xlabel{user_interface}User interface}
\markboth{User interface}{\stardocname}

The IUEDR user interface uses the Starlink ADAM parameter system.
The interface will be familiar to users of the VMS IUEDR, with a few changes
to reach a level of consistency with other Starlink packages.

\subsection {Starting IUEDR}

To initialise for IUEDR type
\begin{verbatim}
   % iuedr
\end{verbatim}

at the shell prompt \verb+%+.
The first time you type the command, IUEDR environment variables are set up in
your session.
You can now start the program by typing
\begin{verbatim}
   % iuedr
\end{verbatim}
again.
Edit your \verb+.login+ file if you want to avoid having to type the command
that extra time.

The command line interface prompt is \verb+>+\@.
This should appear after a welcome message and you can then type commands as
you would in a typical command shell.

\subsection {Response to command prompts}

Instructions to IUEDR are given as command lines.

Command lines begin with a command and an optional list of parameter
assignments. For example:
\begin{verbatim}
   > DRIMAGE DATASET=SWP14931 DEVICE=xw
\end{verbatim}

Usually IUEDR will only prompt for parameters required by commands if
they have no currently defined value.
However, some parameters are either cancelled during the execution of a
command or are set so that the user is always prompted for a value.
A command can be forced to prompt for all required parameter values thus:

\begin{verbatim}
   > READIUE PROMPT
\end{verbatim}

The \verb+PROMPT+ may be abbreviated to \verb+PR+\@.

In a similar way commands which always prompt for parameter values can be made
to accept default values thus:

\begin{verbatim}
   > READIUE ACCEPT
\end{verbatim}

Command input and output printed at the terminal is also copied to the
file \verb+session.lis+ in the working directory.
Note that this file is rewritten each time IUEDR is run.

\subsection {Response to parameter prompts}

Help about a parameter can be obtained by responding to the parameter prompt
with a question mark, {\it{e.g.,}}

\begin{verbatim}
   DATASET - Dataset Name. > ?
\end{verbatim}

Help information will then be printed at the terminal and the prompt repeated.

Sometimes an undefined parameter value is interpreted by a command in
a specific way ({\it{e.g.,}}\ auto-scaling within plotting commands)\@.
A parameter can be set undefined by responding to the prompt with an
exclamation mark, {\it{e.g.,}}

\begin{verbatim}
   XL - X-axis plotting limits, [0,0] means auto-scale. /[1150,1950]/ > !
\end{verbatim}

A command may be aborted by responding to a parameter prompt with a
double exclamation mark, {\it{e.g.,}}

\begin{verbatim}
   XL - X-axis plotting limits, [0,0] means auto-scale. /[1150,1950]/ > !!
\end{verbatim}

To prevent a command from prompting for further parameter values (once in
prompt mode) type a backslash, {\it{e.g.,}}

\begin{verbatim}
   XL - X-axis plotting limits, [0,0] means auto-scale. /[1150,1950]/ > \
\end{verbatim}

The command will then only prompt for parameters for which it cannot generate
a suitable value.

\subsection {Getting HELP}

\begin{latexonly}
Type HELP at the IUEDR command line prompt.
You may optionally append a detailed description of the topic on which help
is required. See page~\pageref{com: HELP } for further details.
\end{latexonly}

\begin{htmlonly}
Type HELP at the IUEDR command line prompt.
You may optionally append a detailed description of the topic on which help
is required. See HELP for further details.
\end{htmlonly}

\subsection {IUEDR in script and batch modes}

It is possible to run IUEDR in script mode, where command and
parameter input originates from a file instead of the terminal.
The file from which the command input is to be taken is piped into IUEDR:

\begin{verbatim}
   % iuedr < script_file
\end{verbatim}

This will result in the command input being taken from the file
\verb+script_file+ and the text output being written to the \verb+session.lis+
file.

Alternatively the output can be directed to a specific file:

\begin{verbatim}
   % iuedr < script_file > script_log
\end{verbatim}

In this case \verb+script_log+ contains the output log \verb+session.lis+ is
{\bf not} over-written.

\newpage
\section{\xlabel{IUEDR_data_files}IUEDR data files}
\markboth{IUEDR data files}{\stardocname}

Once an IUE GO format file has been read by IUEDR two files are created.  One
is a text file containing information about the data set calibration data.
This file has a name constructed
\begin{verbatim}
   <dataset>.UEC
\end{verbatim}
where \verb+<dataset>+ corresponds to the value of the IUEDR \verb+DATASET+
parameter.

when a GPHOT, PHOT or RAW image is read the data produced are stored in an
Image and Data Quality file
\begin{verbatim}
   <dataset>_UED.sdf
\end{verbatim}
After the spectrum extraction process the uncalibrated spectral data are stored
in a file
\begin{verbatim}
   <dataset>_UES.sdf
\end{verbatim}
This file is also produced when an IUESIPS MELO or MEHI file is read with
\verb+READSIPS+\@.  No \verb+_UED.sdf+ being created in this case.

Calibrated spectra produced by the \verb+MAP+ command are stored in a file
\begin{verbatim}
   <dataset>_UEM.sdf
\end{verbatim}

All these \verb+.sdf+ files are Starlink NDF format files (See
\xref{SUN/33}{sun33}{}
for details of access to NDFs) which can be read by any of the standard
packages (KAPPA, FIGARO etc.).  The contents of these files can be examined
outside of IUEDR using the \verb+hdstrace+ command.

These files are in addition to the IUEDR log file \verb+session.lis+ and the
files generated by output commands.  They should {\bf not} be deleted until the
data reduction is complete and the output spectra obtained.

The \verb+OUT*+ family of IUEDR output commands also generate NDFs which can
be read by programs such as DIPSO.

In summary:
\begin {description}
   \item \verb+<dataset>.UEC+ --- calibration file.
   \item \verb+<dataset>_UED.SDF+ --- image data and quality file.
   \item \verb+<dataset>_UES.SDF+ --- uncalibrated spectrum file.
   \item \verb+<dataset>_UEM.SDF+ --- calibrated mean spectrum file.
\end {description}

{\bf Refer to Appendix~\ref{se:vmsunix} for VMS to UNIX file conversion.}

\subsection{\label{subap:ndf}NDF Components in IUEDR files}

This section gives a summary of the NDF components present in IUEDR files for
those who may wish to access the files from their own programs.
The structure and content of an NDF can be inspected using the {\tt hdstrace}
utility (See \xref{SUN/102}{sun102}{})\@.  Values for components have been
given where they are constant for all files of the particular type.

\subsubsection{Image data and quality file {\tt \_UED}}

\begin{latexonly}
A simple NDF, each point in the $768\times 768$ image is described by a datum
and a quality flag.  The image is given the generic title `IUE image'\@.
\end{latexonly}

\begin{htmlonly}
A simple NDF, each point in the 768x768 image is described by a datum
and a quality flag.  The image is given the generic title `IUE image'\@.
\end{htmlonly}

\begin{verbatim}
IUEDR  <NDF>

   DATA_ARRAY(768,768)  <_WORD>

   QUALITY        <QUALITY>       {structure}
      QUALITY(768,768)  <_UBYTE>

   TITLE          <_CHAR*9>       'IUE image'
\end{verbatim}

\subsubsection{Uncalibrated spectrum file {\tt \_UES}}

This NDF contains IUEDR specific extensions, which are written and read by the
program when processing spectra.  In the description below {\tt no} is
the number of orders processed with the \verb+TRAK+ command.  {\tt mo} is
the number of data points in the longest order processed.
{\tt WAVES} contains the wavelengths of each of the flux data in each order.
{\tt ORDERS} holds a list of the order numbers processed.
{\tt NWAVS} stores the actual number of points in each of the {\tt no} orders.

\begin{verbatim}
IUEDR  <NDF>

   DATA_ARRAY(mo,no)   <_REAL>

   QUALITY        <QUALITY>       {structure}
      QUALITY(mo,no)   <_UBYTE>

   MORE           <EXT>           {structure}
      IUEDR_EXTRA    <EXTENSION>     {structure}
         WAVES(mo,no)   <_REAL>
         ORDERS(no)     <_INTEGER>
         NWAVS(no)      <_INTEGER>

   TITLE          <_CHAR*80>
   LABEL          <_CHAR*4>       'Flux'
\end{verbatim}

\subsubsection{Calibrated mean spectrum file {\tt \_UEM}}

The mean spectrum is basically an array of flux values against a wavelength
scale.  Quality information is included in the data file.  The axes units are
also included.
This NDF type is again IUEDR specific, the unusual components are as follows.
{\tt WAVES} are wavelengths of flux data points.
{\tt WEIGHTS} are weights applied to the flux.
{\tt XCOMB1} is the start wavelength.
{\tt DXCOMB} is the wavelength step from point to point.



\begin{verbatim}
IUEDR  <NDF>

   DATA_ARRAY(17001)  <_REAL>

   QUALITY        <QUALITY>       {structure}
      QUALITY(17001)  <_UBYTE>

   MORE           <EXT>           {structure}
      IUEDR_EXTRA    <EXTENSION>     {structure}
         WAVES(17001)   <_REAL>
         WEIGHTS(17001)  <_REAL>
         XCOMB1         <_DOUBLE>
         DXCOMB         <_DOUBLE>

   AXIS(1)        <AXIS>          {structure}
      DATA_ARRAY(17001)  <_REAL>
      UNITS          <_CHAR*40>      '(A)'
      LABEL          <_CHAR*40>      'Wavelength'

   TITLE          <_CHAR*80>
   UNITS          <_CHAR*40>      '(FN/s)'
   LABEL          <_CHAR*40>      'Flux'
\end{verbatim}

\subsubsection{\label{se:spectrum}SPECTRUM format output files}

SPECTRUM is a data analysis programme written by Steve Adams at UCL\@.
Although the programme is no longer used (I guess it might be in use
somewhere\ldots) the file formats it introduced were adopted by the popular
spectrum analysis programme DIPSO (described in \xref{SUN/50}{sun50}{})\@.

The basic input to a SPECTRUM file is a single spectrum (wavelength, flux)\@.
The wavelengths should be in increasing order, and evenly spaced.

There are three variants of the SPECTRUM file format.  Using its terminology:

\begin{latexonly}
\begin{tabular}{ll}
Format number & File characteristics\\
0             & Unformatted (Binary)\\
1             & Fixed Format Text\\
2             & Free-field Format Text\\
\end{tabular}
\end{latexonly}

\begin{htmlonly}
\begin{rawhtml}
<PRE>
<B>Format number     File characteristics</B>
      0           Unformatted (Binary)
      1           Fixed Format Text
      2           Free-field Format Text
</PRE>
\end{rawhtml}
\end{htmlonly}

IUEDR {\bf no longer} produces output of the SP0 type.  Instead NDFs are used.
In practice this is invisible to the user as the DIPSO SP0RD command (read
SPECTRUM format 0 file) now reads NDFs!  The other two formats are still
available.  A Description of the old SP0 format is included here in case
anyone needs to read an existing file in this format (DIPSO can still read
SP0 format via the SP0RD command)\@.

Using a FORTRAN77 notation, the contents of a SPECTRUM file can be
expressed as:

\begin{verbatim}
   PARAMETER(MAXWAV=8000) ! maximum number of wavelengths
   CHARACTER*79 CLINE1    ! first line of text
   CHARACTER*79 CLINE2    ! second line of text
   INTEGER NWAV           ! number of wavelengths
   REAL WAV(MAXWAV)       ! wavelengths
   REAL FLUX(MAXWAV)      ! fluxes
\end{verbatim}

Both \verb+CLINE1+ and \verb+CLINE2+ are totally unstructured text strings,
and are used to describe the spectrum.  The convention is that
\verb+FLUX(I)=0.0+ when its value is undefined.

Here, briefly, is the code needed to read the SPECTRUM formats:

Format number 0 is an unformatted (binary) file read by:

\begin{verbatim}
   OPEN(UNIT=1, ACCESS='SEQUENTIAL', FORM='UNFORMATTED')
   READ(1) CLINE1(1:79)
   READ(1) CLINE2(1:79)
   READ(1) NWAV
   READ(1) (WAV(I),FLUX(I),I=1,NWAV)
   CLOSE(UNIT=1)
\end{verbatim}

Format number 1 is a fixed format text file read by:

\begin{verbatim}
   OPEN(UNIT=1, ACCESS='SEQUENTIAL')
   READ(1,'(A79)') CLINE1(1:79)
   READ(1,'(A79)') CLINE2(1:79)
   READ(1,'(20X,I6)') NWAV
   READ(1,'(4(F8.3,E10.3))') (WAV(I),FLUX(I),I=1,NWAV)
   CLOSE(UNIT=1)
\end{verbatim}

Format number 2 is a free-field text file read by:

\begin{verbatim}
   OPEN(UNIT=1, ACCESS='SEQUENTIAL')
   READ(1,'(A79)') CLINE1(1:79)
   READ(1,'(A79)') CLINE2(1:79)
   READ(1,*) NWAV
   READ(1,*) (WAV(I),FLUX(I),I=1,NWAV)
   CLOSE(UNIT=1)
\end{verbatim}

The sections of FORTRAN 77 code shown above are not intended to be serious
attempts to write a SPECTRUM file reading programme.  Instead they are designed
to define the contents as succinctly as possible.

\newpage
\section{\xlabel{porting_changes}Changes during the port to UNIX}
\markboth{Changes during the port to UNIX}{\stardocname}

This section describes the main changes that have been made to IUEDR
during its conversion to an ADAM based application which runs on
all Starlink supported platforms.  \xref{SUN/37}{sun37}{} gives notes on the
very latest version of IUEDR.

If you are a seasoned IUEDR user then you should study this section
especially carefully.

The most significant change from the scientific point of view is that
the precision of all floating point calculations has been upgraded to
DOUBLE PRECISION. This was done after it was noticed that for high
resolution extraction the output spectra were subject to rounding
noise at the 1\% level.

The format of the calibration file ({\tt .UEC}) created by IUEDR has
been  changed to make it more readable. A VMS program to convert
IUEDR datasets to the new format is available, see Appendix~\ref{se:vmsunix}
for details.

The functionality of the package has been enhanced to allow image data
to be read directly from disk.

The general operation of IUEDR, and all the command and parameter
names, are identical to those used in previous versions.

\subsection{IUEDR command files}

The \verb+.CMD+ style of VMS IUEDR command files is not directly supported by
UNIX IUEDR, and neither is the associated input/output redirection using
\verb+<+ and \verb+>+\@.

It is very easy to convert a {\tt .CMD} file into a UNIX IUEDR command
script.
{\it{e.g.,}}\ a {\tt DEMO.CMD} procedure:

\begin{verbatim}
   DATASET=SWP03196
   SHOW
   SCAN ORDERS=(125,66)
   TRAK APERTURE=LAP
   SHOW V=S
\end{verbatim}

would become a UNIX IUEDR command script \verb+demo.cmd+, thus:

\begin{verbatim}
   SHOW DATASET=SWP03196
   SCAN ORDERS=[125,66]
   TRAK APERTURE=LAP
   SHOW V=S
\end{verbatim}

The only changes which need be made are to move any parameter
specifications ({\it{e.g.,}}\ \verb+DATASET=+) onto the same line as the command
they apply to, and to change vector parameter specifications to use square
\verb+[]+ brackets instead of the old-style round \verb+()+ brackets.

This script can then be read into IUEDR by
\begin{verbatim}
   % iuedr < demo.cmd
\end{verbatim}

\subsection{Interaction with DIPSO}

As part of the port to UNIX the
format of the default DIPSO spectrum format SP0 files has been
changed to use the STARLINK NDF data format. This means that these
files can be read by any standard STARLINK package.

The IUEDR/DIPSO user should notice no difference, as both IUEDR and
DIPSO understand the new format.

\subsection{DRIVE parameter options}

The use of the DRIVE parameter has been enhanced to allow
specification of disk files containing IUE datasets. This is intended
for use with  files obtained from online archives (RAL and NASA).

The syntax is to provide the full filename and extension in response to
the DRIVE prompt:

\begin{verbatim}
   DRIVE> SWP12345.RAW
\end{verbatim}

\subsection{Specifying vector parameters}

Some IUEDR parameters ({\it{e.g.,}}\ \verb+XP+, \verb+YP+) require the
specification of a pair of numbers defining the limits of a range of values
({\it{e.g.,}}\ pixels).

The method of setting such values has changed to the ADAM style:

\begin{verbatim}
   XP=[100,300]
\end{verbatim}

Note that the square brackets are only necessary when vector
parameters are specified on the command line. They are not required
when IUEDR prompts the user for a vector parameter.

\subsection{Calibration files and the NEW* family of commands}

The missing Calibration file for SWP camera HIRES data has been added to the
IUEDR package.

The action of the \verb+NEW*+ family of commands for updating IUEDR
calibrations, geometry and so on have been altered  to improve functionality.
Users will find that the previous style of file name entry still works,
and that the following features have been added:
\begin{itemize}
   \item Both Logical Name and Environment Variable style file specifications
   may be given as parameter values, for example
   \begin{verbatim}
      > NEWABS ABSFILE=$IUEDR_DATA/swphi
   \end{verbatim}
   and
   \begin{verbatim}
      > NEWABS ABSFILE=IUEDR_DATA:swphi
   \end{verbatim}
   are equivalent and allowed on all platforms.
   \item The default file name extension need not be given, however if the
   file to be read has a different extension this {\bf should} be given.
   For example,
   \begin{verbatim}
      > NEWABS ABSFILE=$IUEDR_DATA/swphi
   \end{verbatim}
   and
   \begin{verbatim}
      > NEWABS ABSFILE=$IUEDR_DATA/swphi.abs
   \end{verbatim}
   are {\bf both} valid.

   This behaviour is a change to the VMS-only IUEDR where the extension had
   to be omitted and the default value for the appropriate command was always
   taken.
   \item For case-sensitive file systems (like UNIX) if the Enviroment Variable
   \verb+$IUEDR_DATA+ is used as part of the file specification then the case
   of the file name itself is always converted to {\bf lower case}.  All the
   files available in the \verb+$IUEDR_DATA+ directory have lower case names
   so this is not a problem, rather it allows default calibration file names
   to be upper- or lower-case in IUEDR command scripts.
\end{itemize}

\subsection{Documentation}

The excellent introduction to IUEDR
\xref{{\sl IUE Analysis---A Tutorial}}{sg7}{}
(SG/7) by Richard Tweedy has been updated for UNIX and included as a standard
part of IUEDR\@.  Some special calibration corrections described in this
document have also been added to the IUEDR package.

\newpage
\section{\xlabel{commands}\label{se:commands}Commands}
\markboth{Commands}{\stardocname}

This section contains a detailed description of each of the commands
available in IUEDR.  A list of the parameters used by each command is given,
along with a brief description of each.  The pages on which you will find
full parameter descriptions are given at the end of each line in the parameter
list.

\sstroutine{AESHIFT}
{
   Determine (HIRES) spectrum ESHIFT automatically.
}{
   \sstparameters{
   \cpar{DATASET}{Dataset name.}
   \cpar{CENTREWAVE}{Line central wavelengths (A).}
   \cpar{DELTAWAVE}{Half-width of line search windows (A).}
}
\sstdescription{
   \verb+AESHIFT+ can be used to measure the global \'{e}chelle shift for a
   HIRES
   spectrum.  A set of laboratory wavelengths of absorption features which
   should be present in the spectrum are located in the spectrum and the
   \verb+ESHIFT+ for each is calculated.  The median of these \verb+ESHIFT+s
   is then applied to the whole dataset.
}
}

\sstroutine{AGSHIFT}
{
   Determine spectrum shift for HIRES automatically.
}{
   \sstparameters{
   \cpar{DATASET}{Dataset name.}
   \cpar{ORDERS}{This delineates a range of \'{e}chelle orders.}
}
\sstdescription{
   \verb+AGSHIFT+ can be used to measure the global geometric shift for a HIRES
   spectrum.  A scan of the \verb+DATASET+ must be made available using the
   \verb+SCAN+ command.  The scan data is traversed starting at the lowest
   numbered order and a probable site for the peak of each order is found.
   The central position of each order is then estimated using a centroiding
   algorithm.  An estimate of the geometric shift is made for each order and
   these are recorded and displayed.

   A weighted mean in which the shifts determined for orders 100 to 110
   inclusive are given greater weight than other shifts is calculated.
   The individual order shifts are compared to the mean and any shift
   greater than 3 pixels from the mean position is rejected and a new value
   for the mean shift calculated from the remaining orders.
   The mean shift is displayed and the \verb+GSHIFT+ parameter is set.

   Using \verb+AGSHIFT+ it is possible to automate the spectrum extraction
   process.  It should be noted that objects with no continuum may break
   the \verb+AGSHIFT+ mechanism, giving poor shift values, in these cases
   the interactive \verb+CGSHIFT+ command should be used.
}
}

\sstroutine{BARKER}
{
   Correct spectrum data for \'{e}chelle ripple using a method based
   upon that of Barker (1984).
}{
   \sstparameters{
   \cpar{DATASET}{Dataset name.}
   \cpar{ORDERS}{This delineates a range of \'{e}chelle orders.}
}
\sstdescription{
   The spectrum data in \verb+DATASET+ are corrected for residual \'{e}chelle
   ripple using  the method described by Barker (1984. Astronomical Journal,
   \underline{89},  899). Orders in the range \verb+ORDERS+ are used in the
   ripple correction optimisation.  Note that this optimisation method is
   only applicable for SWP spectra.
}
}


\sstroutine{CGSHIFT}
{
   Determine spectrum template shift using the cursor on a scan plot.
}{
   \sstparameters{
   \cpar{DATASET}{Dataset name.}
   \cpar{APERTURE}{Aperture name (\verb+SAP+ or \verb+LAP+).}
   \cpar{ORDERS}{This delineates a range of \'{e}chelle orders.}
   \cpar{DEVICE}{GKS/SGS graphics device name.}
}
\sstdescription{
   This command allows the graphics cursor to be used to provide
   information about spectrum template registration shifts.

   A plot of the current spectrum scan must be available on the graphics
   \verb+DEVICE+\@.

   A cycle consisting of any  number of left or middle mouse button hits is
   used to mark the position of the spectrum. Each hit is used to calculate
   a linear geometric shift of the spectrum template relative to the image.
   The cycle is terminated by pressing the right mouse button.

   Keyboard keys 1, 2 and 3 can be used  in place of left, middle and right
   mouse buttons respectively.

   For LORES, when the cycle is complete the last geometric shift
   determined is adopted and the scan is revoked.

   For HIRES, each cursor hit is automatically associated with an \'{e}chelle
   order in the range defined by the \verb+ORDERS+ parameter. The last shift is
   again adopted, but the scan is available for further display or
   measurement.
}
}

\newpage
\sstroutine{CLEAN}
{
   Mark pixels with values below a selected threshold as BAD.
}{
   \sstparameters{
   \cpar{DATASET}{Dataset name.}
   \cpar{THRESH}{Smallest pixel value to be accepted as GOOD.}
}
\sstdescription{
   Some IUE  datasets are effected  by horizontal  bars of low pixel values.
   These are caused by a weak  signal from the IUE craft at the time of data
   download.  When there are  a few bars  they can be marked as bad with the
   \verb+EDIMAGE+ command.  In the case of many bars a quicker solution is to
   mark all pixels in  the image below  a user selected  threshold value as BAD.

   Successive \verb+CLEAN+ and \verb+DRIMAGE+ commands starting with a value of
   \verb+THRESH=-1000+ and increasing \verb+THRESH+ towards zero will  allow
   the user to chose a value suitable for the problem image.
}
}

\sstroutine{CULIMITS}
{
   Set display limits with the cursor.
}{
   \sstparameters{
   \cpar{DEVICE}{GKS/SGS graphics device name.}
   \cpar{XL}{$x$-axis plotting limits, undefined or [0, 0] means auto-scale.}
   \cpar{YL}{$y$-axis plotting limits, undefined or [0, 0] means auto-scale.}
   \cpar{XP}{$x$-axis pixel limits, undefined or [0, 0] means full extent.}
   \cpar{YP}{$y$-axis pixel limits, undefined or [0, 0] means full extent.}
}
\sstdescription{
   This command uses the cursor to delineate part of a current display,
   graph or image, to be displayed in some subsequent command
   ({\it{e.g.,}}\ \verb+PLFLUX+, \verb+DRIMAGE+\ldots ).

   The two cursor positions should be at the corners of the required
   rectangular subset. The relation between cursor position sequences and
   axis reversals for graphs is:

   \begin{tabular}{llll}
   {\bf Position 1} & {\bf Position 2} & {\bf x-reversed} & {\bf y-reversed}\\
   bottom/left  & top/right    & NO  & NO \\
   bottom/right & top/left     & YES & NO \\
   top/left     & bottom/right & NO  & YES \\
   top/right    & bottom/left  & YES & YES
   \end{tabular}

   The \verb+XL+ and \verb+YL+ values are changed accordingly.

   In the case of an image display, the \verb+XP+ and \verb+YP+ parameter values
   are changed.
   The image will {\bf always} be drawn without axis reversals.
}
}

\sstroutine{CURSOR}
{
   Find display coordinates using the cursor and print them at the terminal.
}{
   \sstparameters{
   \npar{None.}
}
\sstdescription{
   This command uses the graphics cursor to find coordinates on a
   displayed graph or image.

   Pressing the left or middle  mouse button displays information about the
   pixel being  pointed to.  Pressing the right mouse button terminates the
   \verb+CURSOR+ cycle.  The coordinates for each hit are printed on the
   terminal; they correspond to the unit scale of the axes prevailing on the
   current diagram,  ({\it{e.g.,}}\ (wavelength,  flux)).
   If meaningful,  additional coordinate information is also printed.
   Keyboard keys 1, 2 and 3 may be used in place of left, middle and right
   mouse buttons respectively.
}
}

\sstroutine{DRIMAGE}
{
   Display an IUE image on an suitable graphics workstation.
}{
   \sstparameters{
   \cpar{DATASET}{Dataset name.}
   \cpar{DEVICE}{GKS/SGS graphics device name.}
   \cpar{XP}{$x$-axis pixel limits, undefined or [0, 0] means full extent.}
   \cpar{YP}{$y$-axis pixel limits, undefined or [0, 0] means full extent.}
   \cpar{ZL}{Data limits for image display, undefined means full range.}
   \cpar{COLOUR}{Whether a false colour look-up table is used.}
   \cpar{ZONE}{Zone to be used for plotting.}
   \cpar{FLAG}{Whether data quality for faulty pixels are displayed.}
}
\sstdescription{
   This command displays the image specified by the \verb+DATASET+
   parameter on the device specified by the \verb+DEVICE+ parameter.

   The part of the image displayed is specified by the \verb+XP+ and \verb+YP+
   parameter values.
   If unspecified, \verb+XP+ and \verb+YP+ default to the entire image extent,
   {\it{i.e.}}

   \begin {quote}
      \verb+XP = [1,768], YP = [1,768]+
   \end {quote}

   If the values of \verb+XP+ or \verb+YP+ are specified in decreasing order,
   the image will {\bf not} be reversed along the appropriate axis.

   The range of data values displayed as a grey scale is limited
   by the two values of the \verb+ZL+ parameter.
   Data values at or below \verb+ZL[1]+ will appear {\bf black},
   those at \verb+ZL[2]+ will appear {\bf white} and those above \verb+ZL[2]+
   will appear {\bf blue}.
   If the \verb+ZL+ values are given in decreasing order, then high data
   values will be represented by low (dark) display intensities,
   and vice-versa.
   If the values are undefined, then the full intensity range of the
   image will be used.
   The full intensity range of the image can be found using the command

   \begin {quote}
      \verb+> SHOW V=I+
   \end {quote}

   The \verb+FLAG+ parameter specifies whether faulty pixels are flagged using
   the following colour scheme:

   \begin{description}
      \item GREEN --- pixels affected by reseau marks
      \item RED --- pixels which are saturated (DN=255)
      \item ORANGE --- pixels affected by ITF truncation
      \item YELLOW --- pixels marked bad by the user
   \end{description}

   If a pixel is affected by more than one of the above faults, then
   the first in the list is adopted for display.

   The \verb+ZONE+ parameter is accepted by \verb+DRIMAGE+ but is ignored, the
   display always using \verb+ZONE=0+\@.
}
}

\sstroutine{EDIMAGE}
{
   Edit the image data quality using the graphics cursor.
}{
   \sstparameters{
   \cpar{DATASET}{Dataset name.}
   \cpar{DEVICE}{GKS/SGS graphics device name.}
}
\sstdescription{
   This command uses the image display cursor to mark pixels and
   regions of the current image that are ``bad'' or ``good''.
   The image should have previously been displayed using the
   \verb+DRIMAGE+ command.
   So that faulty pixels can be seen, the \verb+FLAG=TRUE+ option in
   \verb+DRIMAGE+ should be used.

   The image display is specified by the \verb+DEVICE+ parameter and the
   associated dataset by the \verb+DATASET+ parameter.

   The following cursor hit sequences can be used in a cycle:

   \begin {description}
      \item 1 then 1 --- marks all pixels in the rectangle GOOD.
      \item 2 then 2 --- marks all points in the rectangle BAD.
      \item 1 --- marks the nearest pixel GOOD.
      \item 2 --- marks the nearest pixel BAD.
      \item 3 --- causes the cursor cycle to terminate.
   \end {description}

   Mouse buttons can be used for cursor hits where:

   \begin {description}
      \item {\bf left} mouse button     is  hit 1.
      \item {\bf middle} mouse button   is  hit 2.
      \item {\bf right} mouse button    is  hit 3.
   \end{description}

   Alternatively, keyboard keys 1, 2 and 3 can be used to mark hits.

   The pixels or ranges changed are printed on the terminal.
   The term ``rectangle'' is used above to indicate a rectangular
   set of pixels delineated by the two cursor positions.
   Thus, for the first hit, the cursor can be positioned at the
   bottom left corner, and for the second at the top right corner.

   Only the user-defined data quality bit can be changed by this
   command.
   Initially, all faulty pixels have this bit set BAD, so that
   spectrum extraction (say) can ignore these where appropriate.
   However, the user-defined data quality can also be set GOOD.

   See the IUEDR User Guide (MUD/45) for further information on data quality.
}
}

\sstroutine{EDMEAN}
{
   Edit the mean extracted spectrum using the graphics cursor.
}{
   \sstparameters{
   \cpar{DATASET}{Dataset name.}
   \cpar{DEVICE}{GKS/SGS graphics device name.}
}
\sstdescription{
   This command uses the graphics cursor to mark points and
   regions of the mean spectrum that are ``bad'' or ``good''.

   The following cursor hit sequences can be used in a cycle:

   \begin {description}
      \item 1 then 1 --- marks all points in the $x$-range GOOD.
      \item 2 then 2 --- marks all points in the $x$-range BAD.
      \item 1 --- marks the nearest point in $x$-direction GOOD.
      \item 2 --- marks the nearest point in $x$-direction BAD.
      \item 3 --- causes the cursor cycle to terminate.
   \end {description}

   The points or ranges changed are printed on the terminal.

   Mouse buttons can be used for cursor hits where:

   \begin {description}
      \item {\bf left} mouse button     is  hit 1.
      \item {\bf middle} mouse button   is  hit 2.
      \item {\bf right} mouse button    is  hit 3.
   \end{description}

   Alternatively, keyboard keys 1, 2 and 3 can be used to mark hits.

   See the IUEDR User Guide (MUD/45) for further information on data quality.
}
}

\sstroutine{EDSPEC}
{
   Edit the net extracted spectrum using the graphics cursor.
}{
   \sstparameters{
   \cpar{DATASET}{Dataset name.}
   \cpar{ORDER}{\'{E}chelle order number.}
   \cpar{APERTURE}{Aperture name (\verb+SAP+ or \verb+LAP+).}
   \cpar{DEVICE}{GKS/SGS graphics device name.}
}
\sstdescription{
   This command uses the graphics cursor to mark points and
   regions of the current net spectrum that are ``bad'' or ``good''.
   A plot of the \verb+APERTURE+ or \verb+ORDER+ spectrum is required before
   this command can be used.

   The following cursor hit sequences can be used in a cycle:

   \begin {description}
      \item 1 then 1 --- marks all points in the $x$-range GOOD.
      \item 2 then 2 --- marks all points in the $x$-range BAD.
      \item 1 --- marks the nearest point in $x$-direction GOOD.
      \item 2 --- marks the nearest point in $x$-direction BAD.
      \item 3 --- causes the cursor cycle to terminate.
   \end {description}

   The points or ranges changed are printed on the terminal.

   Mouse buttons can be used for cursor hits where:

   \begin {description}
      \item {\bf left} mouse button     is  hit 1.
      \item {\bf middle} mouse button   is  hit 2.
      \item {\bf right} mouse button    is  hit 3.
   \end{description}

   Alternatively, keyboard keys 1, 2 and 3 can be used to mark hits.

   Only the user-defined data quality bit can be changed by this
   command.
   Initially, all faulty points have this bit set BAD ({\it{e.g.,}}\ by
   \verb+TRAK+)\@. However, whether they are considered bad ({\it{e.g.,}}\ when
   plotting or creating output files) is determined by the user-defined
   bit, which can be changed at will.

   See the IUEDR User Guide (MUD/45) for further information on data quality.
}
}

\sstroutine{ERASE}
{
   Erase the display screen of the graphics device.
}{
   \sstparameters{
   \cpar{DEVICE}{GKS/SGS graphics device name.}
}
\sstdescription{
   The display screen of the specified graphics device is erased.
}
}

\sstroutine{EXIT}
{
   Quit IUEDR.
}{
   \sstparameters{
   \npar{None.}
}
\sstdescription{
   This command quits IUEDR\@.
   Any files that require new versions will be written by this command.
   This command is a synonym for the \verb+QUIT+ command.
}
}

\sstroutine{HELP}
{
   Find out about IUEDR commands and parameters.
}{
   \sstparameters{
   \npar{None.}
}
\sstdescription{
   By simply typing \verb+HELP+ \label{com: HELP }the user is presented with
   a brief introduction to IUEDR, a list of the commands available and some
   general information topics. Users familiar with he VMS help system will
   find this facility very essentially the same to use.

   The \verb+HELP+ system provides a list of topics which can be
   selected from by typing enough characters of a topic name to uniquely
   identify it and pressing return.  Pressing the return key with no topic
   chosen takes the \verb+HELP+ system back one topic-level.
   At any time, pressing the return key a few times will return you to the
   IUEDR prompt.

   You may optionally give a specific topic to the \verb+HELP+ command at the
   IUEDR prompt, for example
   \begin{quote}
      \verb+> HELP DRIMAGE+
   \end{quote}
   or even
   \begin{quote}
      \verb+> HELP DRIMAGE COLOUR+
   \end{quote}
}
}

\sstroutine{LBLS}
{
   Extracts a line-by-line-spectrum array from the image.
}{
   \sstparameters{
   \cpar{DATASET}{Dataset name.}
   \cpar{ORDER}{\'{E}chelle order number.}
   \cpar{APERTURE}{Aperture name (\verb+SAP+ or \verb+LAP+).}
   \cpar{GSAMP}{Spectrum grid sampling rate (geometric pixels).}
   \cpar{CUTWV}{Whether wavelength cutoff data used for extraction grid.}
   \cpar{CENTM}{Whether pre-existing centroid template is used.}
   \cpar{RL}{Limits across spectrum for LBLS array (pixels).}
   \cpar{RSAMP}{Radial coordinate sampling rate for LBLS grid (pixels).}
}
\sstdescription{
   This command creates a line-by-line-spectrum (LBLS) array from the
   image defined by \verb+DATASET+\@.
   The array consists of intensities $F(IR, I\lambda )$ for a grid of
   wavelengths, $W(I\lambda)$, and radial coordinates, $R(IR)$\@.
   The wavelength grid, $\lambda$, is determined in a similar way to the
   \verb+TRAK+ command, using the \verb+CUTWV+ (HIRES) and \verb+GSAMP+
   (HIRES/LORES) parameters.

   The radial coordinates are distances from the centre of the spectrum,
   derived from the template data,
   along a line perpendicular to the dispersion direction and
   measured in geometric pixels.
   The radial grid, $R$, is determined by the \verb+RL+ and \verb+RSAMP+
   parameters.

   The value of each pixel in the array corresponds to the surface
   over the image of a rectangle centred on its $(R, \lambda )$ coordinates,
   and extents

   \begin {equation}
      (R(IR) - dR / 2, R(IR) + dR / 2)
   \end {equation}
   and

   \begin {equation}
      (W(I\lambda ) - d\lambda / 2, W(I\lambda ) + d\lambda / 2)
   \end {equation}
   $dR$ is the distance between $R$ values, and $d\lambda$ is the wavelength
   step between $\lambda$ values.

   This surface integral is scaled along the $\lambda$ direction to
   correspond to an interval of 1.414 geometric pixels.
   The reason for this is to make LBLS intensities consistent with
   those produced by the \verb+TRAK+ command.
   For a particular wavelength, $W(I\lambda )$, the sum of LBLS intensities
   after removal of background should correspond to the net
   flux as measured by \verb+TRAK+\@.
}
}

\sstroutine{LISTIUE}
{
   Analyse the contents of IUE tapes or files.
}{
   \sstparameters{
   \cpar{DRIVE}{Tape drive or file name.}
   \cpar{FILE}{Tape file number.}
   \cpar{NFILE}{Number of tape files to be processed.}
   \cpar{NLINE}{Number of IUE header lines printed.}
   \cpar{SKIPNEXT}{Whether skip to next tape file.}
}
\sstdescription{
   This performs an analysis of \verb+NFILE+ IUE tape files, starting at
   the file specified by the \verb+FILE+ parameter.
   \verb+NFILE=-1+ means list all files until the end of the tape.
   \verb+NLINE=-1+ means print all lines in file header.

   \verb+LISTIUE+ can also be used to list the header of a GO format disk file.
}
}

\newpage
\sstroutine{MAP}
{
   Map and merge the extracted spectrum components to produce a mean spectrum.
}{
   \sstparameters{
   \cpar{DATASET}{Dataset name.}
   \cpar{ORDERS}{This delineates a range of \'{e}chelle orders.}
   \cpar{APERTURE}{Aperture name (\verb+SAP+ or \verb+LAP+).}
   \cpar{RM}{Whether mean spectrum is reset before averaging.}
   \cpar{ML}{Wavelength grid limits for mean spectrum.}
   \cpar{MSAMP}{Wavelength sampling rate for mean spectrum grid.}
   \cpar{FILLGAP}{Whether gaps can be filled within order.}
   \cpar{COVERGAP}{Whether gaps can be filled by covering orders.}
}
\sstdescription{
   This command can be used to produce a mean spectrum with contributions
   from several \'{e}chelle orders (HIRES), or from several apertures (LORES).

   If \verb+RM=TRUE+, or if there is no existing mean spectrum, then an
   evenly spaced wavelength grid is constructed between the
   limits specified by the \verb+ML+ parameter using the sampling rate
   specified by the \verb+MSAMP+ parameter.

   If \verb+RM=FALSE+ and there {\bf is}
   an existing mean spectrum, then the
   wavelength grid {\bf and contents} are retained.
   New components will be averaged with what is already there.

   In the case of HIRES, the \verb+ORDERS+ parameter is used to delimit the
   range of \'{e}chelle orders that are allowed to contribute to the mean.

   In the case of LORES, only a single aperture specified by the
   \verb+APERTURE+ parameter is mapped at a given time.
}
}

\sstroutine{MODIMAGE}
{
   Modifies image pixel intensities interactively.
}{
   \sstparameters{
   \cpar{DATASET}{Dataset name.}
   \cpar{DEVICE}{GKS/SGS graphics device name.}
   \cpar{FN}{Replacement Flux Number for pixel.}
}
\sstdescription{
   This command uses the image display cursor to modify image data.
   The image should already have been displayed using the \verb+DRIMAGE+
   command.

   The following cursor sequences are adopted:

   \begin {description}
      \item 1 then 2 --- copy intensity of first picked pixel to the second.
      \item 2 --- prompt for replacement pixel intensity.
      \item 3 --- finish.
   \end {description}

   Mouse buttons can be used for cursor hits where:

   \begin {description}
      \item {\bf left} mouse button     is  hit 1.
      \item {\bf middle} mouse button   is  hit 2.
      \item {\bf right} mouse button    is  hit 3.
   \end{description}

   Alternatively, keyboard keys 1, 2 and 3 can be used to mark hits.

   If the data or data qualities change after a session, then the file is
   saved on disk.

   The assumption is made that the current image displayed corresponds
   to the current dataset!
}
}

\sstroutine{MTMOVE}
{
   Move to the start of a tape file.
}{
   \sstparameters{
   \cpar{DRIVE}{Tape drive.}
   \cpar{FILE}{Tape file number.}
}
\sstdescription{
   Move to the start of the file specified by the \verb+FILE+ parameter on the
   tape specified by the \verb+DRIVE+ parameter.
}
}

\sstroutine{MTREW}
{
   Rewind to the start of the tape.
}{
   \sstparameters{
   \cpar{DRIVE}{Tape drive.}
}
\sstdescription{
   This command rewinds the tape specified by the \verb+DRIVE+ parameter.
   The \verb+FILE+ parameter is also set to 1 by this command.
}
}

\sstroutine{MTSHOW}
{
   Show the current tape position.
}{
   \sstparameters{
   \cpar{DRIVE}{Tape drive.}
}
\sstdescription{
   This command displays the current tape position.
   This includes the file number and the block position relative to either
   the start or the end of the file.

   Note that the actual file position may differ from the
   value of the \verb+FILE+ parameter.
}
}

\sstroutine{MTSKIPEOV}
{
   Skip over end-of-volume (EOV) mark.
}{
   \sstparameters{
   \cpar{DRIVE}{Tape drive.}
}
\sstdescription{
   This command skips over an end-of-volume (EOV) mark on the tape specified
   by the DRIVE parameter.
   An EOV condition is where there are two consecutive tape marks.
   When attempting to skip across an EOV, an error will be reported
   and the tape left positioned between the two marks.
   Subsequent attempts to skip forward will fail and
   only this command can be used to move forward beyond the
   second tape mark.
}
}

\sstroutine{MTSKIPF}
{
   Skip over NSKIP tape marks.
}{
   \sstparameters{
   \cpar{DRIVE}{Tape drive.}
   \cpar{NSKIP}{Number of tape marks to be skipped over.}
}
\sstdescription{
   This command skips over \verb+NSKIP+ tape marks on the tape specified
   by the \verb+DRIVE+ parameter.
   If \verb+NSKIP+ is negative this means that tape marks are skipped in the
   reverse direction, {\it{i.e.}}\ towards the start of the tape.
}
}

\sstroutine{NEWABS}
{
   Associate a new absolute flux calibration with the current dataset.
}{
   \sstparameters{
   \cpar{DATASET}{Dataset name.}
   \cpar{ABSFILE}{Name of file containing absolute flux calibration.}
}
\sstdescription{
   This command reads the absolute flux calibration from a text file
   specified by the \verb+ABSFILE+ parameter and stores it in the dataset
   specified by \verb+DATASET+\@.

   The file type is assumed to be \verb+.abs+ and need not be
   specified as part of the \verb+ABSFILE+ parameter.

   The calibration of any current spectrum is automatically updated.
}
}

\newpage
\sstroutine{NEWCUT}
{
   Associate new \'{e}chelle order wavelength limits with the current
   dataset.
}{
   \sstparameters{
   \cpar{DATASET}{Dataset name.}
   \cpar{CUTFILE}{Name of file containing \'{e}chelle order wavelength limits.}
}
\sstdescription{
   This command reads the \'{e}chelle order wavelength limits from a text file
   specified by the \verb+CUTFILE+ parameter and stores them in the dataset
   specified by \verb+DATASET+\@.

   The file type is assumed to be \verb+.cut+ and need not be
   specified as part of the \verb+CUTFILE+ parameter.

   The calibration of any current spectrum is automatically updated.
}
}

\sstroutine{NEWDISP}
{
   Associate new spectrograph dispersion data with the current dataset.
}{
   \sstparameters{
   \cpar{DATASET}{Dataset name.}
   \cpar{DISPFILE}{Name of file containing dispersion data.}
}
\sstdescription{
   This command reads spectrograph dispersion data from the text file
   specified by the \verb+DISPFILE+ parameter and stores them in the dataset
   specified by \verb+DATASET+\@.

   The file type is assumed to be \verb+.dsp+ and need not be specified
   as part of the \verb+DISPFILE+ parameter.
}
}

\sstroutine{NEWFID}
{
   Read IUE fiducial positions from text file.
}{
   \sstparameters{
   \cpar{DATASET}{Dataset name.}
   \cpar{FIDFILE}{Name of file containing fiducial positions.}
   \cpar{NGEOM}{Number of Chebyshev terms used to represent geometry.}
}
\sstdescription{
   This command reads IUE fiducial positions from a text file
   specified by the \verb+FIDFILE+ parameter and stores them in the dataset
   specified by \verb+DATASET+\@.

   The file type is assumed to be \verb+.fid+ and need not be specified
   as part of the FIDFILE parameter.

   The image data quality and geometry representation are updated to
   account for any changes that these fiducial positions imply.
   In the case of datasets containing image distortion, the \verb+NGEOM+
   parameter is used to specify the number of terms used for the Chebyshev
   representation along each axis.
}
}

\sstroutine{NEWRIP}
{
   Read \'{e}chelle ripple calibration from text file.
}{
   \sstparameters{
   \cpar{DATASET}{Dataset name.}
   \cpar{RIPFILE}{Name of file containing \'{e}chelle ripple calibration.}
}
\sstdescription{
   This command reads an \'{e}chelle ripple calibration from a text file
   specified by the \verb+RIPFILE+ parameter and stores it in the dataset
   specified by \verb+DATASET+\@.

   The file type is assumed to be \verb+.rip+ and need not be specified
   as part of the \verb+RIPFILE+ parameter.

   The calibration of any current spectrum is automatically updated.
}
}

\sstroutine{NEWTEM}
{
   Read spectrum centroid template data from text file.
}{
   \sstparameters{
   \cpar{DATASET}{Dataset name.}
   \cpar{TEMFILE}{Name of file containing spectrum template data.}
}
\sstdescription{
   This command reads the spectrum centroid template data into \verb+DATASET+
   from a text file with name specified by \verb+TEMFILE+\@.

   The file type is assumed to be \verb+.tem+ and need not be specified
   as part of the \verb+TEMFILE+\@.
}
}

\sstroutine{OUTEM}
{
   Output the current spectrum template data to a formatted data file.
}{
   \sstparameters{
   \cpar{DATASET}{Dataset name.}
   \cpar{TEMFILE}{Name of file containing spectrum template data.}
}
\sstdescription{
   This command outputs the templates stored with the current dataset to a
   text file.
   If not specified, the file name is constructed as:

   \begin {quote}
      \verb+<CAMERA>HI<APERTURE>.TEM+
   \end {quote}
   or

   \begin {quote}
      \verb+<CAMERA>LO.TEM+
   \end {quote}

   for the HIRES and LORES cases respectively.
}
}

\sstroutine{OUTLBLS}
{
   Output the current LBLS array to a binary data file.
}{
   \sstparameters{
   \cpar{DATASET}{Dataset name.}
   \cpar{OUTFILE}{Name of output file.}
}
\sstdescription{
   This command outputs the current LBLS array to a file.
   If not specified by the \verb+OUTFILE+ parameter, the file name is
   constructed as:

   \begin {quote}
      \verb+<CAMERA><IMAGE>R.DAT+
   \end {quote}
   The format of this file is described by the Fortran 77 routine, RDLBLS, which
   can be found in the file:

   \begin {quote}
      {\tt \$IUEDR\_USER/rdlbls.for}
   \end {quote}
   The directory {\tt \$IUEDR\_USER} also contains a test
   program for using RDLBLS and other helpful items.
}
}

\sstroutine{OUTMEAN}
{
   Output current mean spectrum to a DIPSO SP format file.
}{
   \sstparameters{
   \cpar{DATASET}{Dataset name.}
   \cpar{OUTFILE}{Name of output file.}
   \cpar{SPECTYPE}{DIPSO SP file type (0, 1 or 2).}
}
\sstdescription{
   This command outputs the mean spectrum associated with
   \verb+DATASET+ to a file that can be read into DIPSO
   (see \xref{SUN/50}{sun50}{}).

   This file is created with type specified by the \verb+SPECTYPE+ parameter
   (see Section~\ref{se:spectrum}
   for SP options).
   If not specified, the file name is constructed as:

   \begin {quote}
      \verb+<CAMERA><IMAGE>M.sdf+ \hspace*{8mm} for SP0\\
      \verb+<CAMERA><IMAGE>M.DAT+ \hspace*{8mm} for SP1 or SP2
   \end {quote}
   In DIPSO SP format, bad points are indicated by having zero intensities.
   In determining which points in the output file are to be marked
   ``bad'', the user-defined data quality bit is used.
   Since this bit can be arbitrarily edited,
   faulty data values can be written to the output file
   without subsequent information being retained.
}
}

\sstroutine{OUTNET}
{
   Output the current net spectrum to a DIPSO  SP format data file.
}{
   \sstparameters{
   \cpar{DATASET}{Dataset name.}
   \cpar{APERTURE}{Aperture name (\verb+SAP+ or \verb+LAP+).}
   \cpar{ORDER}{\'{E}chelle order number.}
   \cpar{OUTFILE}{Name of output file.}
   \cpar{SPECTYPE}{DIPSO SP file type (0, 1 or 2).}
}
\sstdescription{
   This command outputs the net spectrum associated with \verb+ORDER+ or
   \verb+APERTURE+ and \verb+DATASET+ to a file that can be read into DIPSO
   (see \xref{SUN/50}{sun50}{}).

   The file is created with type specified
   by the \verb+SPECTYPE+ parameter (see Section~\ref{se:spectrum}
   for SP
   options).
   If not specified, the file name is constructed as:

   \begin {quote}
      \verb+<CAMERA><IMAGE>+\_\verb+<APERTURE>.sdf+ \hspace*{8mm} for SP0\\
      \verb+<CAMERA><IMAGE>.<APERTURE>+ \hspace*{8mm} for SP1 or SP2
   \end {quote}
   in the case of LORES and

   \begin {quote}
      \verb+<CAMERA><IMAGE>+\_\verb+<ORDER>.sdf+ \hspace*{8mm} for SP0\\
      \verb+<CAMERA><IMAGE>.<ORDER>+ \hspace*{8mm} for SP1 or SP2
   \end {quote}
   in the case of HIRES.
   Here, \verb+<APERTURE>+ is the aperture name (\verb+SAP+ or \verb+LAP+),
   or index, and \verb+<ORDER>+ is the \'{e}chelle order number.

   In DIPSO SP format, bad points are indicated by having zero intensities.
   In determining which points in the output file are to be marked
   ``bad'', the user-defined data quality bit is used.
   Since this bit can be arbitrarily edited,
   faulty data values can be written to the output file
   without subsequent information being retained.
}
}

\sstroutine{OUTRAK}
{
   Output the current uncalibrated spectrum to a ``TRAK'' formatted data file.
}{
   \sstparameters{
   \cpar{DATASET}{Dataset name.}
   \cpar{OUTFILE}{Name of output file.}
}
\sstdescription{
   This command outputs the uncalibrated spectrum associated with
   \verb+DATASET+ to a formatted file that is compatible with output
   from the old ``TRAK'' program.
   The default file name is of the form:

   \begin {quote}
      \verb+<CAMERA><IMAGE>.TRK+
   \end {quote}
   The main difference from an actual ``TRAK'' file is that the background
   level is uniformly zero, so that GROSS=NET.
}
}

\sstroutine{OUTSCAN}
{
   Output the current scan data to a DIPSO  SP format data file.
}{
   \sstparameters{
   \cpar{DATASET}{Dataset name.}
   \cpar{OUTFILE}{Name of output file.}
   \cpar{SPECTYPE}{DIPSO SP file type (0, 1 or 2).}
}
\sstdescription{
   This command outputs the current scan associated with
   \verb+DATASET+ to a file which can be read into DIPSO
   (see \xref{SUN/50}{sun50}{}).

   The file is created with type specified
   by the \verb+SPECTYPE+ parameter
   (see Section~\ref{se:spectrum}
   for SP options).
   If not specified, the file name is constructed as:

   \begin {quote}
      \verb+<CAMERA><IMAGE>P.sdf+ \hspace*{8mm} for SP0\\
      \verb+<CAMERA><IMAGE>P.DAT+ \hspace*{8mm} for SP1 or SP2
   \end {quote}
   In DIPSO SP format, bad points are indicated by having zero intensities.
   In determining which points in the output file are to be marked
   ``bad'', the user-defined data quality bit is used.
   Since this bit can be arbitrarily edited,
   faulty data values can be written to the output file
   without subsequent information being retained.
}
}

\sstroutine{OUTSPEC}
{
   Output the current aperture or order spectrum to a DIPSO SP format data file.
}{
   \sstparameters{
   \cpar{DATASET}{Dataset name.}
   \cpar{APERTURE}{Aperture name (\verb+SAP+ or \verb+LAP+).}
   \cpar{ORDER}{\'{E}chelle order number.}
   \cpar{OUTFILE}{Name of output file.}
   \cpar{SPECTYPE}{DIPSO SP file type (0, 1 or 2).}
}
\sstdescription{
   This command outputs the spectrum associated with the \verb+ORDER+ or
   \verb+APERTURE+ and \verb+DATASET+ to a file which can be read into DIPSO
   (see \xref{SUN/50}{sun50}{}).

   The file is created with type specified
   by the \verb+SPECTYPE+ parameter
   (see Section~\ref{se:spectrum}
   for SP options).
   If not specified, the file name is constructed as:

   \begin {quote}
      \verb+<CAMERA><IMAGE>+\_\verb+<APERTURE>.sdf+ \hspace*{8mm} for SP0\\
      \verb+<CAMERA><IMAGE>.<APERTURE>+ \hspace*{8mm} for SP1 or SP2
   \end {quote}
   in the case of LORES and

   \begin {quote}
      \verb+<CAMERA><IMAGE>+\_\verb+<ORDER>.sdf+ \hspace*{8mm} for SP0\\
      \verb+<CAMERA><IMAGE>.<ORDER>+ \hspace*{8mm} for SP1 or SP2
   \end {quote}
   in the case of HIRES.
   Here, \verb+<APERTURE>+ is the aperture name (\verb+SAP+ or \verb+LAP+), or
   index, and \verb+<ORDER>+ is the \'{e}chelle order number.

   In DIPSO SP format, bad points are indicated by having zero intensities.
   In determining which points in the output file are to be marked
   ``bad'', the user-defined data quality bit is used.
   Since this bit can be arbitrarily edited,
   faulty data values can be written to the output file
   without subsequent information being retained.
}
}

\sstroutine{PLCEN}
{
   Plot smoothed centroid shifts.
}{
   \sstparameters{
   \cpar{DATASET}{Dataset name.}
   \cpar{ORDER}{\'{E}chelle order number.}
   \cpar{APERTURE}{Aperture name (\verb+SAP+ or \verb+LAP+).}
   \cpar{RS}{Whether display is reset before plotting.}
   \cpar{DEVICE}{GKS/SGS graphics device name.}
   \cpar{ZONE}{Zone to be used for plotting.}
   \cpar{LINE}{Plotting line style (\verb+SOLID+, \verb+DASH+, \verb+DOTDASH+
               or \verb+DOT+).}
   \cpar{LINEROT}{Whether line style is changed after next plot.}
   \cpar{COL}{Plotting line colour (1, 2, 3, \dots 10).}
   \cpar{COLROT}{Whether line colour is changed after next plot.}
   \cpar{XL}{$x$-axis plotting limits, undefined or [0, 0] means auto-scale.}
   \cpar{YL}{$y$-axis plotting limits, undefined or [0, 0] means auto-scale.}
}
\sstdescription{
   This command plots the smoothed centroid shifts produced during
   the most recent spectrum extraction from \verb+DATASET+
   on the graphics device and zone specified by the \verb+DEVICE+ and
   \verb+ZONE+ parameters respectively.

   In the case of a LORES spectrum, if there is more than a single
   aperture available, then the \verb+APERTURE+ parameter needs to be specified.

   In the case of a HIRES spectrum, if there is more than a single
   \'{e}chelle order, then the \verb+ORDER+ parameter needs to be specified.

   The \verb+RS+ parameter specifies whether a new plot is started, or whether
   the data can be plotted over an existing plot.

   The \verb+LINE+ and \verb+LINEROT+ parameters determine the line style
   which will be used for plotting.

   The \verb+COL+ and \verb+COLROT+ parameters determine the line colour which
   will be used for plotting if the \verb+DEVICE+ supports colour graphics.

   The diagram limits are specified by the \verb+XL+ and \verb+YL+ parameter
   values.
   If \verb+XL+ and \verb+YL+ have values

   \begin {quote}
      \verb+XL=[0,0], YL=[0,0]+
   \end {quote}
   then the plot limits along each axis are determined so that the whole
   spectrum is visible.
   If the values of \verb+XL+ or \verb+YL+ are specified in decreasing order,
   then the coordinates will be reversed along the appropriate axis.
}
}

\sstroutine{PLFLUX}
{
   Plot calibrated flux spectrum.
}{
   \sstparameters{
   \cpar{DATASET}{Dataset name.}
   \cpar{ORDER}{\'{E}chelle order number.}
   \cpar{APERTURE}{Aperture name (\verb+SAP+ or \verb+LAP+).}
   \cpar{RS}{Whether display is reset before plotting.}
   \cpar{DEVICE}{GKS/SGS graphics device name.}
   \cpar{ZONE}{Zone to be used for plotting.}
   \cpar{LINE}{Plotting line style (\verb+SOLID+, \verb+DASH+, \verb+DOTDASH+
               or \verb+DOT+).}
   \cpar{LINEROT}{Whether line style is changed after next plot.}
   \cpar{COL}{Plotting line colour (1, 2, 3, \dots 10).}
   \cpar{COLROT}{Whether line colour is changed after next plot.}
   \cpar{HIST}{Whether lines are drawn as histograms.}
   \cpar{QUAL}{Whether data quality information is plotted.}
   \cpar{XL}{$x$-axis plotting limits, undefined or [0, 0] means auto-scale.}
   \cpar{YL}{$y$-axis plotting limits, undefined or [0, 0] means auto-scale.}
}
\sstdescription{
   This command plots the calibrated flux spectrum from \verb+DATASET+
   on the graphics device and zone specified by the \verb+DEVICE+ and
   \verb+ZONE+ parameters respectively.

   In the case of a LORES spectrum, if there is more than a single
   aperture available, then the \verb+APERTURE+ parameter needs to be specified.

   In the case of a HIRES spectrum, if there is more than a single
   \'{e}chelle order, then the \verb+ORDER+ parameter needs to be specified.

   The \verb+RS+ parameter specifies whether a new plot is started, or whether
   the data can be plotted over an existing plot.

   The \verb+HIST+ parameter determines whether the line is drawn as a
   histogram rather than a continuous polyline.

   The \verb+LINE+ and \verb+LINEROT+ parameters determine the line style
   which will be used for plotting.

   The \verb+COL+ and \verb+COLROT+ parameters determine the line colour which
   will be used for plotting if the \verb+DEVICE+ supports colour graphics.

   The diagram limits are specified by the \verb+XL+ and \verb+YL+ parameter
   values.
   If \verb+XL+ and \verb+YL+ have values

   \begin {quote}
      \verb+XL=[0,0], YL=[0,0]+
   \end {quote}
   then the plot limits along each axis are determined so that the whole
   spectrum is visible.
   If the values of \verb+XL+ or \verb+YL+ are specified in decreasing order,
   then the coordinates will be reversed along the appropriate axis.

   The \verb+QUAL+ parameter indicates whether faulty points are flagged with
   their data quality codes (see Section~\ref{se:introduction}).

   If a point is affected by more than one of the above faults, then
   the highest code is plotted.
   Points marked bad by user edits are only indicated if they are otherwise
   fault-free.
}
}

\sstroutine{PLGRS}
{
   Plot pseudo-gross and background from spectrum extraction.
}{
   \sstparameters{
   \cpar{DATASET}{Dataset name.}
   \cpar{ORDER}{\'{E}chelle order number.}
   \cpar{APERTURE}{Aperture name (\verb+SAP+ or \verb+LAP+).}
   \cpar{RS}{Whether display is reset before plotting.}
   \cpar{DEVICE}{GKS/SGS graphics device name.}
   \cpar{ZONE}{Zone to be used for plotting.}
   \cpar{LINE}{Plotting line style (\verb+SOLID+, \verb+DASH+, \verb+DOTDASH+
               or \verb+DOT+).}
   \cpar{LINEROT}{Whether line style is changed after next plot}
   \cpar{COL}{Plotting line colour (1, 2, 3, \ldots 10).}
   \cpar{COLROT}{Whether line colour is changed after next plot.}
   \cpar{HIST}{Whether lines are drawn as histograms.}
   \cpar{QUAL}{Whether data quality information is plotted.}
   \cpar{XL}{$x$-axis plotting limits, undefined or [0, 0] means auto-scale.}
   \cpar{YL}{$y$-axis plotting limits, undefined or [0, 0] means auto-scale.}
}
\sstdescription{
   This command plots the pseudo-gross and smooth background produced during
   the most recent spectrum extraction from \verb+DATASET+
   on the graphics device and zone specified by the \verb+DEVICE+ and
   \verb+ZONE+ parameters respectively.

   The pseudo-gross is constructed by taking the net spectrum and adding
   the smooth background multiplied by the width of the object channel.
   The smooth background plotted is also for the object channel width.

   In the case of a LORES spectrum, if there is more than a single
   aperture available, then the \verb+APERTURE+ parameter needs to be specified.

   In the case of a HIRES spectrum, if there is more than a single
   \'{e}chelle order, then the \verb+ORDER+ parameter needs to be specified.

   The \verb+RS+ parameter specifies whether a new plot is started, or whether
   the data can be plotted over an existing plot.

   The \verb+HIST+ parameter determines whether the line is drawn as a histogram
   rather than a continuous polyline.

   The \verb+LINE+ and \verb+LINEROT+ parameters determine the line style
   which will be used for plotting.

   The \verb+COL+ and \verb+COLROT+ parameters determine the line colour which
   will be used for plotting if the \verb+DEVICE+ supports colour graphics.

   The diagram limits are specified by the \verb+XL+ and \verb+YL+ parameter
   values.
   If \verb+XL+ and \verb+YL+ have values

   \begin {quote}
      \verb+XL=[0,0], YL=[0,0]+
   \end {quote}
   then the plot limits along each axis are determined so that the whole
   spectrum is visible.
   If the values of \verb+XL+ or \verb+YL+ are specified in decreasing order,
   then the coordinates will be reversed along the appropriate axis.

   The \verb+QUAL+ parameter indicates whether faulty points are flagged with
   their data quality codes (see Section~\ref{se:introduction}).

   If a point is affected by more than one of the above faults, then
   the highest code is plotted.
   Points marked bad by user edits are only indicated if they are otherwise
   fault-free.
}
}

\sstroutine{PLMEAN}
{
   Plot mean spectrum.
}{
   \sstparameters{
   \cpar{DATASET}{Dataset name.}
   \cpar{RS}{Whether display is reset before plotting.}
   \cpar{DEVICE}{GKS/SGS graphics device name.}
   \cpar{ZONE}{Zone to be used for plotting.}
   \cpar{LINE}{Plotting line style (\verb+SOLID+, \verb+DASH+, \verb+DOTDASH+
               or \verb+DOT+).}
   \cpar{LINEROT}{Whether line style is changed after next plot}
   \cpar{COL}{Plotting line colour (1, 2, 3, \ldots 10).}
   \cpar{COLROT}{Whether line colour is changed after next plot.}
   \cpar{HIST}{Whether lines are drawn as histograms.}
   \cpar{QUAL}{Whether data quality information is plotted.}
   \cpar{XL}{$x$-axis plotting limits, undefined or [0, 0] means auto-scale.}
   \cpar{YL}{$y$-axis plotting limits, undefined or [0, 0] means auto-scale.}
}
\sstdescription{
   This command plots the mean spectrum associated with \verb+DATASET+ on the
   graphics device and zone specified by the \verb+DEVICE+ and \verb+ZONE+
   parameters respectively.

   The \verb+RS+ parameter specifies whether a new plot is started, or whether
   the data can be plotted over an existing plot.

   The \verb+HIST+ parameter determines whether the line is drawn as a histogram
   rather than a continuous polyline.

   The \verb+LINE+ and \verb+LINEROT+ parameters determine the line style which
   will be used for plotting.

   The \verb+COL+ and \verb+COLROT+ parameters determine the line colour which
   will be used for plotting if the \verb+DEVICE+ supports colour graphics.

   The diagram limits are specified by the \verb+XL+ and \verb+YL+ parameter
   values.
   If \verb+XL+ and \verb+YL+ have values

   \begin {quote}
      \verb+XL=[0,0], YL=[0,0]+
   \end {quote}
   then the plot limits along each axis are determined so that the whole
   spectrum is visible.
   If the values of \verb+XL+ or \verb+YL+ are specified in decreasing order,
   then the coordinates will be reversed along the appropriate axis.

   The \verb+QUAL+ parameter indicates whether faulty points are flagged with
   their data quality codes (see Section~\ref{se:introduction}).

   If a point is affected by more than one of the above faults, then
   the highest code is plotted.
   Points marked bad by user edits are only indicated if they are otherwise
   fault-free.
}
}

\sstroutine{PLNET}
{
   Plot uncalibrated net spectrum.
}{
   \sstparameters{
   \cpar{DATASET}{Dataset name.}
   \cpar{ORDER}{\'{E}chelle order number.}
   \cpar{APERTURE}{Aperture name (\verb+SAP+ or \verb+LAP+).}
   \cpar{RS}{Whether display is reset before plotting.}
   \cpar{DEVICE}{GKS/SGS graphics device name.}
   \cpar{ZONE}{Zone to be used for plotting.}
   \cpar{LINE}{Plotting line style (\verb+SOLID+, \verb+DASH+, \verb+DOTDASH+
               or \verb+DOT+).}
   \cpar{LINEROT}{Whether line style is changed after next plot}
   \cpar{COL}{Plotting line colour (1, 2, 3, \ldots 10).}
   \cpar{COLROT}{Whether line colour is changed after next plot.}
   \cpar{HIST}{Whether lines are drawn as histograms.}
   \cpar{QUAL}{Whether data quality information is plotted.}
   \cpar{XL}{$x$-axis plotting limits, undefined or [0, 0] means auto-scale.}
   \cpar{YL}{$y$-axis plotting limits, undefined or [0, 0] means auto-scale.}
}
\sstdescription{
   This command plots the uncalibrated net spectrum specified by the
   \verb+DATASET+
   parameter on the graphics device and zone specified by the
   \verb+DEVICE+ and \verb+ZONE+ parameters respectively.

   In the case of a LORES spectrum, if there is more than a single
   aperture available, then the \verb+APERTURE+ parameter needs to be specified.

   In the case of a HIRES spectrum, if there is more than a single
   \'{e}chelle order, then the \verb+ORDER+ parameter needs to be specified.

   The \verb+RS+ parameter specifies whether a new plot is started, or whether
   the data can be plotted over an existing plot.

   The \verb+HIST+ parameter determines whether the line is drawn as a histogram
   rather than a continuous polyline.

   The \verb+LINE+ and \verb+LINEROT+ parameters determine the line style which
   will be used for plotting.

   The \verb+COL+ and \verb+COLROT+ parameters determine the line colour which
   will be used for plotting if the \verb+DEVICE+ supports colour graphics.

   The diagram limits are specified by the \verb+XL+ and \verb+YL+ parameter
   values.
   If \verb+XL+ and \verb+YL+ have values

   \begin {quote}
      \verb+XL=[0,0], YL=[0,0]+
   \end {quote}
   then the plot limits along each axis are determined so that the whole
   spectrum is visible.
   If the values of \verb+XL+ or \verb+YL+ are specified in decreasing order,
   then the coordinates will be reversed along the appropriate axis.

   The \verb+QUAL+ parameter indicates whether faulty points are flagged with
   their data quality codes (see Section~\ref{se:introduction}).

   If a point is affected by more than one of the above faults, then
   the highest code is plotted.
   Points marked bad by user edits are only indicated if they are otherwise
   fault-free.
}
}

\sstroutine{PLSCAN}
{
   Plot scan perpendicular to dispersion.
}{
   \sstparameters{
   \cpar{DATASET}{Dataset name.}
   \cpar{RS}{Whether display is reset before plotting.}
   \cpar{DEVICE}{GKS/SGS graphics device name.}
   \cpar{ZONE}{Zone to be used for plotting.}
   \cpar{LINE}{Plotting line style (\verb+SOLID+, \verb+DASH+, \verb+DOTDASH+
               or \verb+DOT+).}
   \cpar{LINEROT}{Whether line style is changed after next plot.}
   \cpar{COL}{Plotting line colour (1, 2, 3, \ldots 10).}
   \cpar{COLROT}{Whether line colour is changed after next plot.}
   \cpar{QUAL}{Whether data quality information is plotted.}
   \cpar{XL}{$x$-axis plotting limits, undefined or [0, 0] means auto-scale.}
   \cpar{YL}{$y$-axis plotting limits, undefined or [0, 0] means auto-scale.}
}
\sstdescription{
   This command plots the most recent scan perpendicular to dispersion
   associated with \verb+DATASET+ on the graphics device and zone specified by
   the \verb+DEVICE+ and \verb+ZONE+ parameters respectively.

   The \verb+RS+ parameter specifies whether a new plot is started, or whether
   the data can be plotted over an existing plot.

   The \verb+LINE+ and \verb+LINEROT+ parameters determine the line style which
   will be used for plotting.

   The \verb+COL+ and \verb+COLROT+ parameters determine the line colour which
   will be used for plotting if the \verb+DEVICE+ supports colour graphics.

   The diagram limits are specified by the \verb+XL+ and \verb+YL+ parameter
   values.
   If \verb+XL+ and \verb+YL+ have values

   \begin {quote}
      \verb+XL=[0,0], YL=[0,0]+
   \end {quote}
   then the plot limits along each axis are determined so that the whole
   spectrum is visible.
   If the values of \verb+XL+ or \verb+YL+ are specified in decreasing order,
   then the coordinates will be reversed along the appropriate axis.

   The \verb+QUAL+ parameter indicates whether faulty points are flagged with
   their data quality codes (see Section~\ref{se:introduction}).

   If a point is affected by more than one of the above faults, then
   the highest code is plotted.
   Points marked bad by user edits are only indicated if they are otherwise
   fault-free.
}
}

\newpage
\sstroutine{PRGRS}
{
   Print the current extracted aperture or order spectrum in tabular
   form.
}{
   \sstparameters{
   \cpar{DATASET}{Dataset name.}
   \cpar{APERTURE}{Aperture name (\verb+SAP+ or \verb+LAP+).}
}
\sstdescription{
   This command prints the recently extracted spectrum associated
   with \verb+ORDER+ or \verb+APERTURE+
   and \verb+DATASET+ in tabular form.
   The table consists of wavelengths, ``gross'', smooth background, net
   and calibrated fluxes, along
   with any data quality information.
   The ``gross'' and smooth background correspond to an image sample
   with width specified by the adopted extraction slit.

   The output from this command is likely to be too voluminous to read at the
   terminal, refering to the \verb+session.lis+ file may be easier.
}
}

\sstroutine{PRLBLS}
{
   Print the current LBLS array in tabular form.
}{
   \sstparameters{
   \cpar{DATASET}{Dataset name.}
}
\sstdescription{
   This command prints the current LBLS array in tabular form.

   Each line of the main table consists of a wavelength and a set of
   mapped image intensities (FN), corresponding to cells at distances, R
   (pixels), from spectrum centre.

   Any array cells which are affected by ``bad'' image pixels
   ({\it{e.g.,}}\ reseaux, saturation, etc.) have data quality values printed
   below them, the meaning of which is given at the start of the output.

   The output from this command is likely to be too voluminous to read at the
   terminal, refering to the \verb+session.lis+ file may be easier.
}
}

\sstroutine{PRMEAN}
{
   Print the current mean spectrum in tabular form.
}{
   \sstparameters{
   \cpar{DATASET}{Dataset name.}
}
\sstdescription{
   This command prints the mean spectrum associated with DATASET in tabular form.

   The table consists of wavelengths and calibrated fluxes, along
   with any data quality information.

   The output from this command is likely to be too voluminous to read at the
   terminal, refering to the \verb+session.lis+ file may be easier.
}
}

\sstroutine{PRSCAN}
{
   Print the intensities of the current image scan in tabular form.
}{
   \sstparameters{
   \cpar{DATASET}{Dataset name.}
}
\sstdescription{
   This command prints the scan associated with \verb+DATASET+ in tabular form.
   The table consists of wavelengths and net fluxes, along
   with any data quality information.

   The output from this command should be diverted to a file, since
   it is likely to be too voluminous to read at the terminal.
}
}

\sstroutine{PRSPEC}
{
   Print the current aperture or order spectrum in tabular form.
}{
   \sstparameters{
   \cpar{DATASET}{Dataset name.}
   \cpar{APERTURE}{Aperture name (\verb+SAP+ or \verb+LAP+).}
   \cpar{ORDER}{\'{E}chelle order number.}
}
\sstdescription{
   This command prints the spectrum associated with \verb+DATASET+ and either
   \verb+ORDER+ or \verb+APERTURE+ in tabular form.

   The table consists of wavelengths, net and calibrated fluxes, along
   with any data quality information.

   The output from this command is likely to be too voluminous to read at the
   terminal, refering to the \verb+session.lis+ file may be easier.
}
}

\sstroutine{QUIT}
{
   Quit IUEDR.
}{
   \sstparameters{
   \npar{None.}
}
\sstdescription{
   This command quits IUEDR.
   Any files that require new versions will be written by this
   command.
   This command is a synonym for the \verb+EXIT+ command.
}
}

\newpage
\sstroutine{READIUE}
{
   Read a RAW, GPHOT or PHOT IUE image from GO format tape or file.
}{
   \sstparameters{
   \cpar{DRIVE}{Tape drive or file name.}
   \cpar{FILE}{Tape file number.}
   \cpar{NLINE}{Number of IUE header lines printed.}
   \cpar{DATASET}{Dataset name.}
   \cpar{TYPE}{Dataset type (\verb+RAW+, \verb+PHOT+ or \verb+GPHOT+).}
   \cpar{OBJECT}{Object identification text.}
   \cpar{CAMERA}{Camera name (\verb+LWP+, \verb+LWR+ or \verb+SWP+).}
   \cpar{IMAGE}{Image number.}
   \cpar{APERTURES}{Aperture name.}
   \cpar{RESOLUTION}{Spectrograph resolution mode (\verb+LORES+ or
                     \verb+HIRES+).}
   \cpar{EXPOSURES}{Spectrum exposure time(s) (seconds).}
   \cpar{THDA}{IUE camera temperature (C).}
   \cpar{ITFMAX}{Pixel value on tape for ITF saturation.}
   \cpar{BADITF}{Whether bad LORES SWP ITF requires correction.}
   \cpar{YEAR}{Year number (A.D.).}
   \cpar{MONTH}{Month number (1-12).}
   \cpar{DAY}{Day number in Month.}
   \cpar{NGEOM}{Number of Chebyshev terms used to represent geometry.}
   \cpar{ITF}{This is the ITF generation used in the image calibration.}
}
\sstdescription{
   This command reads an IUE dataset (\verb+RAW+, \verb+GPHOT+ or \verb+PHOT+)
   from tape or from a GO format disk file.
   The \verb+DATASET+ parameter determines the names of files that will contain
   the various data components ({\it{e.g.,}}\ Calibration, Image \& Image Quality
   etc.)

   The \verb+ITFMAX+ and \verb+NGEOM+ parameters are only prompted for if the
   image data are geometrically and photometrically calibrated.
}
}


\newpage
\sstroutine{READSIPS}
{
   Read MELO/MEHI from IUESIPS\#1 or IUESIPS\#2 tape or file.
}{
   \sstparameters{
   \cpar{DRIVE}{Tape drive or file name.}
   \cpar{FILE}{Tape file number.}
   \cpar{NLINE}{Number of IUE header lines printed.}
   \cpar{DATASET}{Dataset name.}
   \cpar{OBJECT}{Object identification text.}
   \cpar{CAMERA}{Camera name (\verb+LWP+, \verb+LWR+ or \verb+SWP+).}
   \cpar{IMAGE}{Image number.}
   \cpar{APERTURES}{Aperture name.}
   \cpar{EXPOSURES}{Spectrum exposure time(s) (seconds).}
   \cpar{THDA}{IUE camera temperature (C).}
   \cpar{YEAR}{Year number (A.D.).}
   \cpar{MONTH}{Month number (1-12).}
   \cpar{DAY}{Day number in Month.}
   \cpar{ITF}{This is the ITF generation used in the image calibration.}
}
\sstdescription{
   This command reads the MELO/MEHI product from IUESIPS\#1 or IUESIPS\#2
   tape or file.  Operation is much like \verb+READIUE+, except that some
   parameters and associated information are not needed. Only calibration
   ({\tt .UEC}) and spectrum ({\tt \_UES.sdf}) files are created.
   The values for certain parameters may be obtained from the tape or file,
   in which case you will not be prompted for them.
}
}

\sstroutine{SAVE}
{
   Write new versions for any files that have had their contents
   changed during the current session.
}{
   \sstparameters{
   \npar{None.}
}
\sstdescription{
   This command overwrites dataset files that have had their contents
   changed during the current session.
   If there are no outstanding files then this command does nothing.
}
}

\newpage
\sstroutine{SCAN}
{
   This command performs a scan perpendicular to spectrograph
   dispersion.
}{
   \sstparameters{
   \cpar{DATASET}{Dataset name.}
   \cpar{ORDERS}{This delineates a range of \'{e}chelle orders.}
   \cpar{APERTURE}{Aperture name (\verb+SAP+ or \verb+LAP+).}
   \cpar{SCANDIST}{HIRES scan distance from camera faceplate centre}
   \cparc{(geometric pixels.)}
   \cpar{SCANAV}{Averaging filter HWHM for image scan}
   \cparc{(geometric pixels).}
   \cpar{SCANWV}{Central wavelength for LORES image scan (\AA).}
}
\sstdescription{
   This command performs a scan perpendicular to spectrograph dispersion.
   The scan is performed by folding pixels with a triangle function
   with HWHM of \verb+SCANAV+ geometric pixels along the dispersion direction.

   In the case of HIRES, the \verb+SCANDIST+ parameter determines the distance
   of the scan from the faceplate centre.

   In the case of LORES, the \verb+SCANWV+ parameter determines the central
   wavelength of the scan in Angstroms.

   The algorithm used to produce scan intensities is not very good
   and so quantitative results should not be sought from this command.
   Its sole intention lies in providing data for aligning the spectrum.
}
}

\sstroutine{SETA}
{
   Set dataset parameters that are APERTURE specific.
}{
   \sstparameters{
   \cpar{DATASET}{Dataset name.}
   \cpar{APERTURE}{Aperture name (\verb+SAP+ or \verb+LAP+).}
   \cpar{EXPOSURE}{Spectrum exposure time (seconds).}
   \cpar{FSCALE}{Flux scale factor.}
   \cpar{WSHIFT}{Constant wavelength shift (\AA).}
   \cpar{VSHIFT}{Velocity shift of detector relative to Sun (km/s).}
   \cpar{ESHIFT}{Global \'{e}chelle wavelength shift.}
   \cpar{GSHIFT}{Global shift of spectrum on image (geometric pixels).}
}
\sstdescription{
   This command allows changes to be made to dataset values which are
   specific to the specified \verb+APERTURE+\@.
   Items for which parameters are not specified retain their current
   values.
}
}

\sstroutine{SETD}
{
   Set dataset parameters which are independent of ORDER/APERTURE.
}{
   \sstparameters{
   \cpar{DATASET}{Dataset name.}
   \cpar{OBJECT}{Object identification text.}
   \cpar{THDA}{IUE camera temperature (C).}
   \cpar{FIDSIZE}{Half width of fiducials (pixels).}
   \cpar{BADITF}{Whether bad LORES ITF requires correction.}
   \cpar{NGEOM}{Number of Chebyshev terms used to represent geometry.}
   \cpar{RIPK}{\'{E}chelle ripple constant (\AA).}
   \cpar{RIPA}{Ripple function scale factor.}
   \cpar{XCUT}{Global \'{e}chelle wavelength clipping.}
   \cpar{HALTYPE}{The type of halation (order-overlap) correction used.}
   \cpar{HALC}{Halation correction constant (fraction of continuum).}
   \cpar{HALWC}{Wavelength for which the halation correction \verb+HALC+}
   \cparc{is defined in angstroms.}
   \cpar{HALW0}{Wavelength at which halation correction is zero (\AA).}
   \cpar{HALAV}{Averaging FWHM for halation correction (gometric pixels).}
}
\sstdescription{
   This command allows changes to be made to dataset values which are
   independent of any specific \verb+APERTURE+/\verb+ORDER+\@.
   Items for which parameters are not specified retain their current
   values.
}
}

\sstroutine{SETM}
{
   Set dataset parameters that are ORDER specific.
}{
   \sstparameters{
   \cpar{DATASET}{Dataset name.}
   \cpar{ORDER}{\'{E}chelle order number.}
   \cpar{RIPK}{\'{E}chelle ripple constant (\AA).}
   \cpar{RIPA}{Ripple function scale factor.}
   \cpar{RIPC}{Ripple function correction polynomial.}
   \cpar{WCUT}{Wavelength limits for \'{e}chelle order (\AA).}
}
\sstdescription{
   This command allows changes to be made to dataset values which are
   specific to the specified \verb+ORDER+\@.
   Items for which parameters are not specified retain their current
   values.
}
}

\newpage
\sstroutine{SGS}
{
   Write names of available SGS devices at the terminal.
}{
   \sstparameters{
   \npar{None.}
}
\sstdescription{
   This commands writes a list of available SGS device names at the terminal.
   See \xref{SUN/85}{sun85}{} for details of the SGS graphics system.
}
}

\sstroutine{SHOW}
{
   Print dataset values.
}{
   \sstparameters{
   \cpar{DATASET}{Dataset name.}
   \cpar{V}{List of items to be printed.}
}
\sstdescription{
   This command shows the values of parameters in the dataset specified
   by the \verb+DATASET+ parameter. The items to be printed are specified by
   the \verb+V+ parameter, which is a string containing any of the following
   characters:

   \begin {description}
      \item H --- Header and file information
      \item I --- Image details
      \item F --- Fiducials
      \item G --- Geometry
      \item D --- Dispersion
      \item C --- Centroid templates
      \item R --- \'{E}chelle Ripple and halation
      \item A --- Absolute calibration
      \item S --- Raw Spectrum
      \item M --- Mean spectrum
      \item * --- All of the above
      \item Q --- Image data Quality summary
   \end {description}

   The \verb+*+ character needs to be placed within inverted commas.

   The \verb+V+ parameter is cancelled afterwards.
}
}

\sstroutine{TRAK}
{
   Extract spectrum from image.
}{
   \sstparameters{
   \cpar{DATASET}{Dataset name.}
   \cpar{ORDER}{\'{E}chelle order number.}
   \cpar{APERTURE}{Aperture name (\verb+SAP+ or \verb+LAP+).}
   \cpar{NORDER}{Number of \'{e}chelle orders to be processed.}
   \cpar{AUTOSLIT}{Whether \verb+GSLIT+, \verb+BDIST+ and \verb+BSLIT+
                   parameters are}
   \cparc{determined automatically.}
   \cpar{GSLIT}{Object channel limits (geometric pixels).}
   \cpar{BSLIT}{Background channel half widths (geometric pixels).}
   \cpar{BDIST}{Distances of background channels from object channel}
   \cparc{centre (geometric pixels).}
   \cpar{GSAMP}{Spectrum grid sampling rate (geometric pixels).}
   \cpar{CUTWV}{Whether wavelength cutoff data used for extraction grid.}
   \cpar{BKGIT}{Number of background smoothing iterations.}
   \cpar{BKGAV}{Background averaging filter FWHM (geometric pixels).}
   \cpar{BKGSD}{Discrimination level for background pixels (s.d.).}
   \cpar{CENTM}{Whether pre-existing centroid template is used.}
   \cpar{CENSH}{Whether the spectrum template is just shifted linearly.}
   \cpar{CENSV}{Whether the spectrum template is saved in the dataset.}
   \cpar{CENIT}{Number of centroid tracking iterations.}
   \cpar{CENAV}{Centroid averaging filter FWHM (geometric pixels).}
   \cpar{CENSD}{Significance level for signal to be used for centroids
                (s.d.).}
   \cpar{EXTENDED}{Whether the object is not a point source.}
   \cpar{CONTINUUM}{Whether the object spectrum is expected to have a}
   \cparc{``continuum''.}
}
\sstdescription{
   This command extracts a spectrum from an image.
   It does this by defining an evenly spaced wavelength grid along the
   spectrum, and mapping pixel intensities onto this grid in object
   and background channels.
   The background pixels are used to form a smooth background spectrum.
   The object pixels (less smooth background) are used to form the
   integrated net signal for the object.

   In the LORES case, the spectrum specified by the \verb+APERTURE+ parameter
   is extracted.

   In the HIRES case, the first \'{e}chelle order to be extracted is
   specified by \verb+ORDER+\@.
   Up to \verb+NORDER+ orders are extracted, with \verb+ORDER+ being
   decremented each time.

   The wavelength grid is defined by the region of the dispersion line
   contained in the image subset (faceplate).
   The grid spacing is set by the \verb+GSAMP+ parameter value which is
   the sample step in geometric pixels.
   The wavelength limits can optionally be constrained within the
   \'{e}chelle cutoff values by specifying \verb+CUTWV=TRUE+\@.

   The object and background channel widths and positions are determined
   automatically if \verb+AUTOSLIT=TRUE+\@.
   Otherwise, the object channel is specified by the values of the
   \verb+GSLIT+ parameter, whilst the background channel positions and
   widths are determined by the \verb+BDIST+ and \verb+BSLIT+ parameter values
   respectively.

   The \verb+EXTENDED+ and \verb+CONTINUUM+ parameters allow more precise
   control over slit determinations (see the IUEDR User Guide (MUD/45) for
   details).

   The background spectrum is smoothed with a triangle function with a
   FWHM given in geometric pixels by the \verb+BKGAV+ parameter.
   Once the background channel spectra have been obtained, they are
   extracted a further \verb+BKGIT+ times.
   Prior to each additional background extraction pixels which are
   outside \verb+BKGSD+ local standard deviations are rejected.

   The object spectrum is obtained by integrating pixel intensities
   (less smooth background) within the object channel.
   Once the object spectrum has been obtained it is extracted an
   additional \verb+CENIT+ times, the centroid positions
   from the previous extraction being used to ``follow'' the
   spectrum each time.
   The centroid spectrum (template) is smoothed by folding with a
   triangle function, FWHM given in geometric
   pixels by the \verb+CENAV+ parameter.
   Wavelengths with flux levels below \verb+CENSD+ standard deviations
   above background are not used in determining the centroid spectrum.

   By default, the initial spectrum template is given by the dispersion
   relations and the geometric shifts determined using the \verb+CGSHIFT+\@.
   However, if \verb+CENTM=TRUE+, then a pre-defined template associated with
   the dataset may be used as a start guess.
   If \verb+CENSH=TRUE+, then this template can be shifted linearly to match the
   image ({\it{i.e.}}\ without changing its shape).
   If \verb+CENSV=TRUE+, then the final centroid spectrum is used to update the
   spectrum template in the dataset.

   The net flux associated with a wavelength point in the final extracted
   spectrum is defined as the integral of pixel intensities over a rectangle
   with dimensions given by the object channel width and the wavelength
   interval.
   These fluxes are scaled so that they correspond to an interval
   along the wavelength direction of 1.414 geometric pixels.
   This is so that the standard IUESIPS calibrations can be applied
   regardless of what actual sample rate has been employed.
   The integral is performed by using linear interpolation of pixel intensities.
}
}

\newpage
%------------------------------------------------------------------------------

\section{\xlabel{paramaters}\label{se:parameters}Parameters}
\markboth{Parameters}{\stardocname}

There follows a detailed description for each of the parameters used by
IUEDR commands.
The description for a particular parameter applies in any
command which uses it.
Some parameters have default values which are initialised on invoking IUEDR\@.
Default parameter behaviour is described in
Appendix~\ref{se:parameter_defaults}\@.

This release of IUEDR uses the ADAM parameter system. In this system
the parameters and their usage are described in an interface file
(See \xref{SG/4}{sg4}{} for more detail). It is possible to override the
default
interface file with your own personal version. This permits you to tailor
the precise behavior of each parameter according to your requirements.

In the following descriptions the required parameter value is one of:
\begin{description}
   \item [\_CHAR] A character string.
   \item [\_DOUBLE] A floating point number.  The decimal point need not be
                    included if the value is integer only.
   \item [\_INTEGER] Integer number.
   \item [\_LOGICAL] A logical value: {\tt YES, TRUE, NO} or {\tt FALSE.}
\end{description}

Where a parameter value is of the form:

\begin{quote}
{\bf
   [\_TYPE\{,\_TYPE\}]
}
\end{quote}

At least one value of type {\bf \_TYPE} is required, the second is optional.

\rule{\textwidth}{0.5mm}

\iueparlist{

\iueparameter{ABSFILE}
{
   \_CHAR
}{
   This is the name of a file containing an absolute flux calibration.
   A file type of \verb+.abs+ is assumed and need not be specified
   explicitly.
   If the file name contains a directory specification, then it should be
   enclosed in quotes.
}

\iueparameter{APERTURE}
{
   \_CHAR
}{
   This is the name of an individual aperture.
   The following names have defined meanings:

   \begin {description}
      \item {\tt SAP} --- IUE small aperture.
      \item {\tt LAP} --- IUE large aperture.
   \end {description}

   Other apertures may also be defined.
}

\iueparameter{APERTURES}
{
   \_CHAR
}{
   This specifies an aperture or group of apertures.
   The following three names have defined meanings:

   \begin {description}
      \item {\tt SAP} --- IUE small aperture.
      \item {\tt LAP} --- IUE large aperture.
      \item {\tt BAP} --- IUE both apertures ({\it{i.e.}}\ SAP and LAP together).
   \end {description}
}

\iueparameter{AUTOSLIT}
{
   \_LOGICAL
}{
   This determines whether the extraction slit is determined automatically
   by the command.
   When \verb+AUTOSLIT=TRUE+ the \verb+GSLIT+, \verb+BDIST+ and \verb+BSLIT+
   parameter values are
   determined automatically, based on the IUE camera, resolution, aperture,
   and on the values of the \verb+EXTENDED+ and \verb+CONTINUUM+ parameters.
   This mode of operation is probably the best for point source objects.

   This parameter has a default value of \verb+TRUE+\@.
}

\iueparameter{BADITF}
{
   \_LOGICAL
}{
   This parameter determines whether a correction is made to the pixel
   intensities to account for errors during Ground Station ITF
   calibration.
   (Note that the best scientific results would be obtained
   from reprocessed data which can be obtained on request.)
   The following case is handled:

   \begin {description}
      \item SWP, LORES --- correction of 2nd (faulty) ITF.
   \end {description}
}

\iueparameter{BDIST}
{
   [\_DOUBLE\{,\_DOUBLE\}]
}{
   This is a pair of numbers delineating the background spectrum channel
   positions during spectrum extraction.
   The distances are measured in geometric pixels from the spectrum centre.

   Negative distances mean ``to the left of centre'', and positive distances
   mean ``to the right of centre''.

   If only one value is defined, then this is taken as meaning
   that the channels are positioned symmetrically
   about centre.

   The spectrum ``centre'' is determined by the dispersion relations, and
   modified by any prevailing centroid shifts.
}

\iueparameter{BKGAV}
{
   \_DOUBLE
}{
   This is the FWHM of a triangle function filter used in folding the
   pixel intensities to form the smooth background spectrum.
   It is measured in geometric pixels.

   This parameter has a default value of 30.0.
}

\iueparameter{BKGIT}
{
   \_DOUBLE
}{
   This is the number of background smoothing iterations performed during
   spectrum extraction.

   \begin {description}
      \item \verb+BKGIT=0+ means that the background is taken as the result of
      the first pass of a triangle function filter with a FWHM defined by
      the \verb+BKGAV+ parameter.

      \item \verb+BKGIT=1+ means that, after producing the initial estimate for
      the smooth background, pixels discrepant by more that \verb+BKGSD+
      standard deviations are marked as ``spikes''.
      The smooth background is then re-evaluated, missing out these marked
      pixels.
   \end {description}

   Higher values of \verb+BKGIT+ are possible, but seldom necessary.

   This parameter has a default value of 1.
}

\iueparameter{BKGSD}
{
   \_DOUBLE
}{
   This is the discrimination level, measured in standard deviations,
   beyond which background pixels are marked as ``spikes''.
   It is not used for \verb+BKGIT=0+.

   This parameter has a default value of 2.0.
}

\iueparameter{BSLIT}
{
   [\_DOUBLE\{,\_DOUBLE\}]
}{
   This defines the half width of each background channel, measured
   in geometric pixels.
   A single value means that both channels have the same width.
}

\iueparameter{CAMERA}
{
   \_CHAR
}{
   This is the camera name.
   The following are defined:

   \begin {quote}
   \begin {description}
      \item {\tt LWP} --- IUE long wavelength prime.
      \item {\tt LWR} --- IUE long wavelength redundant.
      \item {\tt SWP} --- IUE short wavelength prime.
   \end {description}
   \end {quote}
}

\iueparameter{CENAV}
{
   \_DOUBLE
}{
   This is the FWHM of a triangle function filter used in folding the
   pixel intensities to form the smooth spectrum centroid positions.
   It is measured in geometric pixels.

   This parameter has a default value of 30.0.
}

\iueparameter{CENIT}
{
   \_INTEGER
}{
   This is the number of spectrum centroid tracking iterations performed during
   spectrum extraction.

   \begin {description}
      \item \verb+CENIT=0+ means that the spectrum position is taken directly
      from the dispersion relations.

      \item \verb+CENIT=1+ means that the spectrum position is first taken
      from the dispersion relations, but is modified to force it to
      follow the spectrum centroid.
   \end {description}

   Higher values of \verb+CENIT+ are possible, but seldom necessary:  it either
   works or fails.

   This parameter has a default value of 1.
}

\iueparameter{CENSD}
{
   \_DOUBLE
}{
   This is the discrimination level, measured in standard deviations,
   below which object signal is not considered significant enough
   to be used to determine the centroid position.
   It is not used for \verb+CENIT=0+\@.

   This parameter has a default value of 4.0.
}

\iueparameter{CENSH}
{
   \_LOGICAL
}{
   This indicates whether the spectrum signal produces a single linear
   shift to the initial template.

   This can be used in cases where the object signal is too weak
   to provide a detailed centroid determination by moving a pre-existing
   template shape into the right position.

   This parameter has a default value of \verb+FALSE+\@.
}

\iueparameter{CENSV}
{
   \_LOGICAL
}{
   This indicates whether the spectrum template, as refined by the
   object centroid during spectrum extraction, is saved in the calibration
   dataset.

   The primary use of this facility is in determining templates from,
   say, the whole spectrum using \verb+TRAK+, and subsequently using these
   with \verb+LBLS+, or another spectrum.

   This parameter has a default value of \verb+FALSE+\@.
}

\iueparameter{CENTM}
{
   \_LOGICAL
}{
   This indicates whether a centroid template from the calibration dataset
   is used as a start in defining the precise position of the spectrum
   signal on the image.

   This parameter has a default value of \verb+FALSE+\@.
}

\iueparameter{CENTREWAVE}
{
   [\_DOUBLE[,\_DOUBLE\ldots ]]
}{
   These are the laboratory wavelengths of a set of absorption features in
   the spectrum to be used to estimate a value for the \verb+ESHIFT+
   parameter.
}

\iueparameter{COL}
{
   \_INTEGER
}{
   This specifies the line colour to be used for the next curve to be
   plotted.
   It can be an integer in the range 1 to 10, and the corresponding
   colours are as follows:

   \begin {quote}
   \begin {description}
      \item {\tt 1} --- Yellow.
      \item {\tt 2} --- Green.
      \item {\tt 3} --- Red.
      \item {\tt 4} --- Blue.
      \item {\tt 5} --- Pink.
      \item {\tt 6} --- Violet.
      \item {\tt 7} --- Turquoise.
      \item {\tt 8} --- Orange.
      \item {\tt 9} --- Light green.
      \item {\tt 10} --- Olive.
   \end {description}
   \end {quote}

   Lines will only appear with different colours it the device supports colour
   graphics, on other devices \verb+COL+ is ignored.
}

\iueparameter{COLOUR}
{
   \_LOGICAL
}{
   Whether a spectrum-style false colour look-up table is used by
   \verb+DRIMAGE+\@.
   If \verb+FALSE+ a greyscale is used.

   The default is to use a greyscale.
}

\iueparameter{COLROT}
{
   \_LOGICAL
}{
   This indicates whether the line colour is to be changed after the next plot.
   The initial line has colour index 1 (YELLOW), unless specified explicitly
   using the \verb+COL+ parameter.
   The sequence of colour indices goes (1, 2, 3, \ldots 10, 1, 2, \ldots).

   In commands where more than one line is plotted, \verb+COLROT+ determines
   whether these lines have different colours.

   Lines will only appear with different colours if the device supports
   colour graphics; on other devices \verb+COLROT+ is harmless.

   This parameter has a default value of \verb+TRUE+\@.
}

\iueparameter{CONTINUUM}
{
   \_LOGICAL
}{
   This indicates whether the object spectrum is expected to contain a
   significant continuum.
   It is used in conjunction with the \verb+EXTENDED+ parameter in determining
   the positions and widths of object and background channels for
   spectrum extraction from HIRES datasets.

   This parameter has a default value of \verb+TRUE+\@.
}

\iueparameter{COVERGAP}
{
   \_LOGICAL
}{
   If after mapping an order/aperture, a grid point is marked as unusable,
   then this parameter determines whether other orders/apertures
   can be allowed to contribute to this grid point.

   This parameter has a default value of \verb+FALSE+\@.
}

\iueparameter{CUTFILE}
{
   \_CHAR
}{
   This is the name of a file containing an \'{e}chelle order
   wavelength limits.
   A file type of \verb+.cut+ is assumed and need not be specified
   explicitly.
   If the file name contains a directory specification, then it should be
   enclosed in quotes.
}

\iueparameter{CUTWV}
{
   \_LOGICAL
}{
   This indicates whether any available \'{e}chelle order wavelength cutoff
   limits are to be used for the spectrum extraction wavelength grid
   limits.
   Highly recommended, provided that you are happy with these wavelength limits.

   This parameter has a default value of \verb+TRUE+\@.
}

\iueparameter{DATASET}
{
   \_CHAR
}{
   This is the root name of the files containing the dataset.
   The file type ({\it{e.g.,}}\ \_\verb+UED.sdf+) should {\bf not} be given in
   the \verb+DATASET+ name.
   If the file name contains a directory specification, then it
   should be enclosed in quotes.

   Note that the actual filenames contain additional characters
   to define their contents ({\it{e.g.,}}\ \verb+LWP12345+\_\verb+UES.sdf+,
   contains spectral data).
}

\iueparameter{DAY}
{
   \_INTEGER
}{
   This is the day number, measured from the start of the month, used
   for constructing dates.
   The \verb+DAY+, \verb+MONTH+ and \verb+YEAR+ parameters refer to the date
   the IUE observations were made and are important to the calibration of the
   data.
}

\iueparameter{DELTAWAVE}
{
   [\_DOUBLE[,\_DOUBLE\ldots]\,]
}{
   The half-width of the window used to search for an absorption line feature
   for wavelength calibration in Angstroms.  If more than one line is being
   used then each may be given a different search window width.
}

\iueparameter{DEVICE}
{
   \_CHAR
}{
   This is the GKS/SGS graphics device.
   A list of possible GKS workstations may be found in
   \xref{SUN/83}{sun83}{}.
   A list of SGS workstation names available at your
   site may be obtained either by a null response to the \verb+DEVICE+ parameter
   prompt, {\it{i.e.}}\ \verb+!+, or by using the IUEDR Command \verb+SGS+\@.
}

\iueparameter{DISPFILE}
{
   \_CHAR
}{
   This is the name of a file containing dispersion data.
   A file type of \verb+.dsp+ is assumed and need not be specified
   explicitly.
   If the file name contains a directory specification, then it should be
   enclosed in quotes.
}

\iueparameter{DRIVE}
{
   \_CHAR
}{
   This is the name of the tape drive. Feasible value
   are \verb+/dev/nrmt0h+ on a UNIX machine and \verb+MSA0+ on VMS.

   This version of IUEDR supports the direct reading of
   IUEDR data from disk files which have the same format as those on
   GO format tapes.

   In order to read directly from such a file (probably grabbed from
   an on-line archive such as NDADSA), you specify its name directly in
   response to the \verb+DRIVE+ parameter prompt.

   If the file is not in the current directory then you must provide
   the full pathname.
}

\iueparameter{ESHIFT}
{
   \_DOUBLE
}{
   This is a global wavelength shift applied to the wavelengths in
   \'{e}chelle spectral orders.
   It is measured in Angstroms, and affects the spectrum wavelengths as follows:

   \begin {equation}
      \lambda _{new} = \lambda _{old} + \frac{\rm ESHIFT}{\rm ORDER}
   \end {equation}

   This is designed to account for wavelength errors that result
   from a global linear shift of the spectrum format on the
   image.
}

\iueparameter{EXPOSURE}
{
   \_DOUBLE
}{
   This is the exposure time associated with the spectrum, measured in seconds.
   If there is more than one aperture, then this time applies
   to that specified by the \verb+APERTURE+ parameter.
}

\iueparameter{EXPOSURES}
{
   [\_DOUBLE\{,\_DOUBLE\}]
}{
   This is one or more exposure times associated with the
   spectrum, measured in seconds.
   There is an exposure time for each aperture defined.
}

\iueparameter{EXTENDED}
{
   \_LOGICAL
}{
   This indicates whether the object spectrum is expected to be extended,
   rather than a point source.
   It is used in conjunction with the \verb+CONTINUUM+ parameter in determining
   the positions and widths of the object and background channels used
   for spectrum extraction from HIRES datasets.

   This parameter has a default value of \verb+FALSE+\@.
}

\iueparameter{FIDFILE}
{
   \_CHAR
}{
   This is the name of a file containing fiducial positions.
   A file type of \verb+.fid+ is assumed and need not be specified
   explicitly.
   If the file name contains a directory specification, then it should be
   enclosed in quotes.
}

\iueparameter{FIDSIZE}
{
   \_DOUBLE
}{
   This is the half width of a fiducial measured in pixel units. The fiducials
   are considered to be square.
}

\iueparameter{FILE}
{
   \_INTEGER
}{
   This is the tape file number.
   The first file on a tape would be \verb+FILE=1+\@.
   One case which may present some problems is
   that of a tape with a an end-of-volume (EOV) mark in the middle
   and with valuable data beyond.
   An EOV is two consecutive tape marks (sometimes called ``file marks'').
   A file is defined here as the information between two tape marks.
   So if the number for a real file before EOV is FILEN, then
   the number of the next real file following the EOV is (FILEN+2).
}

\iueparameter{FILLGAP}
{
   \_LOGICAL
}{
   If a grid point in the mean spectrum would have had a contribution
   from a bad data point, this parameter determines whether that
   grid point is marked as unusable within the context
   of the order or aperture being mapped.
   If the grid point is marked as unusable in this way then other
   good points cannot contribute to it.

   This parameter has a default value of \verb+FALSE+\@.
}

\iueparameter{FLAG}
{
   \_LOGICAL
}{
   This specifies whether the data quality information is displayed
   along with the image.
   If so, then faulty pixels will be marked with a colour according to
   the following scheme:

   \begin {quote}
   \begin {description}
      \item Green --- pixels affected by reseau marks.
      \item Red --- pixels which are saturated (DN = 255).
      \item Orange --- pixels affected by ITF truncation.
      \item Yellow --- pixels marked bad by the user.
   \end {description}
   \end {quote}

   If a pixel is affected by more than one of the above faults, then
   the first in the list is adopted for display.
   Hence, user edits are only shown where no other fault is present.

   This option would normally only be used when assessing the quality
   of faulty pixels, possibly with a view to using them, {\it{i.e.}}\ marking
   them ``good'' with a cursor editor.

   This parameter has a default value of \verb+TRUE+\@.
}

\iueparameter{FN}
{
   \_DOUBLE
}{
   This parameter is the replacement Flux Number for a pixel changed
   explicitly by the user.
}

\iueparameter{FSCALE}
{
   \_DOUBLE
}{
   This is an arbitrary scale factor applied to spectrum fluxes.
   It affects fluxes as follows:

   \begin {equation}
      {\cal F}_{new} = {\cal F}_{old} \times {\rm FSCALE}
   \end {equation}

   It finds application in accounting for grey attenuation, or obscuration
   of object signal through a narrow aperture.
}

\iueparameter{GSAMP}
{
   \_DOUBLE
}{
   This is the sampling rate used for spectrum extraction.
   It is measured in geometric pixels.
   \verb+GSAMP=1.414+ corresponds to the IUESIPS\#1 sampling
   rate, while \verb+GSAMP=0.707+ corresponds to the IUESIPS\#2 sampling
   rate.
   Other values can be chosen.

   This parameter has a default value of 1.414.
}

\iueparameter{GSHIFT}
{
   [\_DOUBLE,\_DOUBLE]
}{
   This is a global constant shift of the spectrum format on the image,
   $(dx,dy)$, where the geometric coordinates, $(x,y)$ of a spectrum position
   are

   \begin {equation}
      x_{new} = x_{old} + dx
   \end {equation}

   and

   \begin {equation}
      y_{new} = y_{old} + dy
   \end {equation}
}

\iueparameter{GSLIT}
{
   [\_DOUBLE\{,\_DOUBLE\}]
}{
   This is a pair of numbers delineating the object spectrum channel
   during spectrum extraction.
   The distances are measured in geometric pixels.

   Negative distances mean ``to the left of centre'', and positive distances
   mean ``to the right of centre''.

   Object channels that do not cover the actual object signal on the
   image will not be meaningful when centroid tracking is employed.

   If only one value is defined, then this is taken as representing
   a channel that is symmetrical about the spectrum centre.

   The spectrum ``centre'' is determined by the dispersion relations
   modified by any prevailing centroid shifts.
}

\iueparameter{HALAV}
{
   \_DOUBLE
}{
   This is the FWHM of a triangle function used for smoothing the
   net spectrum for the \verb+HALTYPE=POWER+ halation correction technique.
}

\iueparameter{HALC}
{
   \_DOUBLE
}{
   This is the Halation correction constant used for \verb+HALTYPE=POWER+
   cases, and defined at wavelength \verb+HALWC+\@.
   The value of the correction constant
   corresponds roughly to the measured depression of a broad
   zero intensity absorption below zero, in units
   of the continuum in adjacent orders.
   The ``constant'', $C$, varies with wavelength as follows:

   \begin {equation}
      C_\lambda = \frac{{\rm HALC} \times (\lambda - {\rm HALW0})}
                       {({\rm HALWC} - {\rm HALW0})}
   \end {equation}

   See the \verb+HALTYPE+, \verb+HALWC+, \verb+HALW0+ and \verb+HALAV+
   parameters.
}

\iueparameter{HALTYPE}
{
   \_CHAR
}{
   This is the type of Halation or order-overlap correction applied to the
   flux spectrum.
   It can take the value

   \begin {quote}
   \begin {description}
      \item {\tt POWER} --- correction based on power-law PSF decay.
   \end {description}
   \end {quote}
}

\iueparameter{HALW0}
{
   \_DOUBLE
}{
   This is the wavelength, measured in Angstroms, at which the
   halation correction is zero.

   See the \verb+HALTYPE+, \verb+HALC+ and \verb+HALWC+ parameters.
}

\iueparameter{HALWC}
{
   \_DOUBLE
}{
   This is the wavelength, measured in Angstroms, at which the
   halation correction is \verb+HALC+\@.

   See the \verb+HALTYPE+, \verb+HALC+ and \verb+HALW0+ parameters.
}

\iueparameter{HIST}
{
   \_LOGICAL
}{
   This determines whether lines are plotted as histograms rather than
   continuous lines (polylines).

   This parameter has a default value of \verb+TRUE+\@.
}

\iueparameter{IMAGE}
{
   \_INTEGER
}{
   This is the Image Sequence Number.
}

\iueparameter{ITF}
{
   \_INTEGER
}{
   This is the ITF generation used in the photometric calibration of the
   image.  This information is needed for the correct absolute flux
   calibration of the resulting spectra.  Possible values for each camera
   are as follows:

   \begin {quote}
   \begin {description}
      \item SWP --- 2
      \item LWR --- 1 and 2
      \item LWP --- 1 and 2
   \end {description}
   \end {quote}

   The appropriate value can be determined from inspection of the
   IUE header text for the GPHOT/PHOT file using the table
   of numbers following the line beginning \verb+PCF C/**+\@.
   Here are the \verb+ITF+ values associated with various forms of this
   table:

   \begin {quote}
   \begin {tabbing}
   ITFMAXxxx\= 0xx\= 1800xx\= 3700xx\= 5600xx\= ...xx\= 30000xxx\= (Corrected, 3rd SWP ITF)\kill
   {\bf ITF}\> \> \>{\bf TABLE}\> \> \> \>{\bf IDENTIFICATION}\\
   \\
   ITF 0\>0\>1800\>3700\>5600\>...\>30000\>Preliminary LWR ITF\\
   ITF 1\>0\>2303\>4069\>8008\>...\>42032\>2nd LWR ITF\\
   ITF 1\>0\>2300\>3969\>6062\>...\>32973\>1st LWP ITF\\
   ITF 2\>0\>2723\>5429\>8145\>...\>38389\>2nd LWP ITF\\
   ITF 0\>0\>1800\>3600\>5500\>...\>\>Preliminary SWP ITF\\
   ITF 1\>0\>1753\>3461\>6936\>...\>28674\>Faulty, 2nd SWP ITF\\
   ITF 2\>0\>1684\>3374\>6873\>...\>28500\>Corrected, 3rd SWP ITF\\
   \end {tabbing}
   \end {quote}

   If the ITF table used has no corresponding absolute flux calibration within
   IUEDR, {\it{e.g.,}}\ LWR ITF0 or SWP ITF0, you are advised to contact the IUE
   Project.
   Although the \verb+BADITF+ parameter is available for data calibrated using
   SWP ITF1, it is advisable to have these data reprocessed by the IUE Project.
}

\iueparameter{ITFMAX}
{
   \_INTEGER
}{
   This is the pixel value on tape corresponding to ITF saturation.
   Its value is fixed for a given ITF table.
   The value of \verb+ITFMAX+ is only needed for IUE images of type GPHOT.
   The appropriate value can be determined from inspection of the
   IUE header text for the GPHOT file using the table
   of numbers following the line beginning \verb+PCF C/**+.

   \begin {quote}
   \begin {tabbing}
   ITFMAXxxx\= 0xx\= 1800xx\= 3700xx\= 5600xx\= ...xx\= 30000xxx\= (Corrected, 3rd SWP ITF)\kill
   {\bf ITFMAX}\>\>\>{\bf TABLE}\>\>\>\>{\bf IDENTIFICATION}\\
   \\
   20000\>0\>1800\>3700\>5600\>...\>30000\>Preliminary LWR ITF\\
   27220\>0\>2303\>4069\>8008\>...\>42032\>2nd LWR ITF\\
   19983\>0\>1800\>3600\>5500\>...\>\>Preliminary SWP ITF\\
   19740\>0\>1753\>3461\>6936\>...\>28674\>Faulty, 2nd SWP ITF\\
   19632\>0\>1684\>3374\>6873\>...\>28500\>Corrected, 3rd SWP ITF\\
   \end {tabbing}
   \end {quote}
}

\iueparameter{LINE}
{
   \_CHAR
}{
   This specifies the line style to be used for the next curve to be
   plotted.
   It can be one of the following:

   \begin {quote}
   \begin {description}
      \item {\tt SOLID} --- solid (continuous) line.
      \item {\tt DASH} --- dashed line.
      \item {\tt DOTDASH} --- dot-dash line.
      \item {\tt DOT} --- dotted line.
   \end {description}
   \end {quote}

   The order of these is that invoked when automatic line style rotation
   is in effect (see the \verb+LINEROT+ parameter).
}

\iueparameter{LINEROT}
{
   \_LOGICAL
}{
   This indicates whether the line style is to be changed after the
   next plot.
   The initial line style is \verb+SOLID+, unless specified explicitly
   using the \verb+LINE+ parameter.
   The sequence of line styles goes (\verb+SOLID+, \verb+DASH+, \verb+DOTDASH+,
   \verb+DOT+, \verb+SOLID+, \verb+DASH+\dots).

   In commands where more than one line is plotted, \verb+LINEROT+ determines
   whether these lines have different styles.

   This parameter has a default value of \verb+FALSE+\@.
}

\iueparameter{ML}
{
   [\_DOUBLE,\_DOUBLE]
}{
   This is a pair of wavelength values defining the start and end of
   the mean spectrum grid.
   The grid will consist of evenly spaced vacuum wavelengths between these
   values.
}

\iueparameter{MONTH}
{
   \_INTEGER
}{
   This is the month number, measured from the start of the Year,
   used in constructing dates.
}

\iueparameter{MSAMP}
{
   \_DOUBLE
}{
   This is the vacuum wavelength sampling rate for the mean spectrum grid.
   If it does not fit an integral number of times into the grid limits,
   then the latter are adjusted to fit.
}

\iueparameter{NFILE}
{
   \_INTEGER
}{
   This is the number of tape files to be processed.
   A value of \verb+NFILE=-1+ means all files until the end.

   This parameter has a default value of 1.
}

\iueparameter{NGEOM}
{
   \_INTEGER
}{
   This is the number of Chebyshev terms used to represent the
   geometrical distortion.
   The same value is used for each axis direction.
}

\iueparameter{NLINE}
{
   \_INTEGER
}{
   This is the number of IUE header lines printed.
   A value of \verb+NLINE=-1+ means the entire header is printed.

   This parameter has a default value of 10.
}

\iueparameter{NORDER}
{
   \_INTEGER
}{
   This is the number of \'{e}chelle orders to be processed by a command.

   This parameter has a default value of 0.
}

\iueparameter{NSKIP}
{
   \_INTEGER
}{
   This is the number of tape marks to be skipped.

   This parameter has a default value of 1.
}

\iueparameter{OBJECT}
{
   \_CHAR
}{
   This is a string containing text to identify the object of the
   observation.
   It can also contain information about the observation
   ({\it{e.g.,}}\ camera, image\ldots ) if required.
   The maximum allowed length of the string is 40 characters.
}

\iueparameter{ORDER}
{
   \_INTEGER
}{
   This is the \'{e}chelle order number.
}

\iueparameter{ORDERS}
{
   [\_INTEGER\{,\_INTEGER\}]
}{
   This is a pair of \'{e}chelle order numbers delineating a range.
   If only a single value is specified, then the range consists of that
   order only.
   The sequence of the two numbers is not significant.
   The useful maximum range for each camera is as follows:

   \begin {quote}
   \begin {description}
      \item SWP --- orders 66 to 125.
      \item LWR --- orders 72 to 125.
      \item LWP --- orders 72 to 125.
   \end {description}
   \end {quote}
}

\iueparameter{OUTFILE}
{
   \_CHAR
}{
   This is the name of a file to receive the output spectrum.
   This release of IUEDR uses the STARLINK NDF format for all output
   spectra. This means that all standard STARLINK packages can be used
   to plot/display/analyse the spectra, in particular some of the
   facilities of KAPPA and FIGARO may prove useful to the general user.
}

\iueparameter{QUAL}
{
   \_LOGICAL
}{
   This specifies whether the data quality information is plotted
   along with the data.
   If so, then faulty points will be marked with their data quality
   severity code, which is one from:

   \begin {quote}
   \begin {description}
      \item 1 --- affected by extrapolated ITF.
      \item 2 --- affected by microphonics.
      \item 3 --- affected by spike.
      \item 4 --- affected by bright point (or user).
      \item 5 --- affected by reseau mark.
      \item 6 --- affected by ITF truncation.
      \item 7 --- affected by saturation.
      \item U --- affected by user edit.
   \end {description}
   \end {quote}

   User edits are only shown where no other fault is present.

   This option would normally only be used when assessing the quality
   of faulty points, possibly with a view to using them, {\it{i.e.}}\ marking
   them ``good'' with a cursor editor.

   This parameter has a default value of \verb+TRUE+\@.
}

\iueparameter{RESOLUTION}
{
   \_CHAR
}{
   This is the spectrograph resolution mode.
   The following modes are defined:

   \begin {quote}
   \begin {description}
      \item {\tt LORES} --- IUE Low Resolution.
      \item {\tt HIRES} --- IUE High Resolution (\'{e}chelle mode).
   \end {description}
   \end {quote}
}

\iueparameter{RIPA}
{
   \_DOUBLE
}{
   This is an empirical scale factor that can be used to modify the
   \'{e}chelle ripple function.
   The normal value is 1.0.
   The primary component of the ripple function is

   \begin {equation}
      {\rm SCALE} = (\frac{\sin x}{x})^2
   \end {equation}

   where

   \begin {equation}
      x = \frac{\pi \times {\rm RIPA} \times (\lambda - \lambda_c)
                \times {\rm ORDER}}
               {\lambda_c}
   \end {equation}

   and

   \begin {equation}
      \lambda_c = \frac{\rm RIPK}{\rm ORDER}
   \end {equation}

   The net spectrum is divided by SCALE above.
   Empirical values of \verb+RIPA+ can be used to optimise the ripple
   correction.

   See the \verb+RIPC+ and \verb+RIPK+ parameter descriptions.
}

\iueparameter{RIPC}
{
   [\_DOUBLE\{,\_DOUBLE\}]
}{
   This is a polynomial in $x$ used to modify the standard \'{e}chelle
   ripple calibration function.
   The calibration is given by

   \begin {equation}
      {\rm SCALE} = (\frac{\sin x}{x})^2 \times (
                {\rm RIPC}(1) + {\rm RIPC}(2) \times x +
                {\rm RIPC}(3) \times x^2 \ldots)
   \end {equation}

   where

   \begin {equation}
      x = \frac {\pi \times {\rm RIPA} \times (\lambda - \lambda_c) \times
                 {\rm ORDER}}
                {\lambda_c}
   \end {equation}

   and

   \begin {equation}
      \lambda_c = \frac {\rm RIPK}{\rm ORDER}
   \end {equation}

   The net spectrum is divided by SCALE above.

   See the \verb+RIPA+ and \verb+RIPK+ parameter descriptions.
}

\iueparameter{RIPFILE}
{
   \_CHAR
}{
   This is the name of a file containing an \'{e}chelle ripple calibration.
   A file type of \verb+.rip+ is assumed and need not be specified
   explicitly.
   If the file name contains a directory specification, then it should be
   enclosed in quotes.
}

\iueparameter{RIPK}
{
   [\_DOUBLE\{,\_DOUBLE\}]
}{
   This is the \'{e}chelle ripple constant measured in Angstroms.
   It corresponds to the central wavelength of \'{e}chelle order number 1.
   The central wavelength of an arbitrary ORDER is

   \begin {equation}
      \lambda_c = \frac {\rm RIPK}{\rm ORDER}
   \end {equation}

   Where this parameter is used for an entire HIRES dataset, the
   parameter can have more than one value, and represent a polynomial
   in ORDER

   \begin {equation}
      \lambda_c =  \frac{({\rm RIPK}(1) + {\rm RIPK}(2)
                         \times {\rm ORDER} + {\rm RIPK}(3)
                         \times {\rm ORDER}^2 + \ldots)}
                        {\rm ORDER}
   \end {equation}
}

\iueparameter{RL}
{
   [\_DOUBLE,\_DOUBLE]
}{
   This is a pair of radial coordinate values defining the start and end of
   the radial grid in an LBLS array.
   These radial coordinates are measured in geometric pixels, and run
   perpendicular to the dispersion direction.
   A coordinate value of 0.0 corresponds to the centre of the spectrum.

   Values \verb+RL=[0.0, 0.0]+ indicate that internal defaults are to be
   adopted.
   A single value is reflected symmetrically about 0.0.
}

\iueparameter{RM}
{
   \_LOGICAL
}{
   This determines whether the mean spectrum is reset before a mapping
   takes place.
   If the spectrum is not reset, then the spectra being mapped will be
   averaged with the existing mean spectrum.

   This parameter has a default value of \verb+TRUE+\@.
}

\iueparameter{RS}
{
   \_LOGICAL
}{
   This determines whether the display screen is reset before plotting.

   This parameter has a default value of \verb+TRUE+\@.
}

\iueparameter{RSAMP}
{
   \_DOUBLE
}{
   This is the sample spacing used for the radial grid in the LBLS array.
   If it does not fit an integral number of times into the grid limits,
   \verb+RL+, then the latter are adjusted to fit.

   Suggested values range from 0.707 to 1.414 pixels, the latter
   corresponding to the IUESIPS LBLS grid.

   This parameter has a default value of 1.414.
}

\iueparameter{SCANAV}
{
   \_DOUBLE
}{
   This is the HWHM of a triangle function with which pixels are folded
   during the generation of a scan across the image perpendicular to
   spectrograph dispersion.
   It is measured in geometric pixels.

   This parameter has a default value of 5.
}

\iueparameter{SCANDIST}
{
   \_DOUBLE
}{
   This is the distance of a scan across a HIRES image from the faceplate
   centre.
   It is measured in geometric pixels.
}

\iueparameter{SCANWV}
{
   \_DOUBLE
}{
   This is the central wavelength for a scan of a LORES image
   perpendicular to spectrograph dispersion.
   It is measured in Angstroms in vacuo.
}

\iueparameter{SKIPNEXT}
{
   \_LOGICAL
}{
   This determines whether the tape is positioned at the start of the
   next file after processing.
   If only the start of a file is being processed, then by setting
   \verb+SKIPNEXT=FALSE+ time can be saved.

   This parameter has a default value of \verb+FALSE+\@.
}

\iueparameter{SPECTYPE}
{
   \_INTEGER
}{
   This is the type of file, in the DIPSO SP format terminology, to be
   created. The following values are allowed:

   \begin {quote}
   \begin {description}
      \item {\tt 0} --- Starlink NDF format file.
      \item {\tt 1} --- SP1, fixed format text file.
      \item {\tt 2} --- SP2, free format text file.
   \end {description}
   \end {quote}

   It is recommended that datasets with many points be written with
   \verb+SPECTYPE=0+, which is more efficient in disk space and time spent
   reading and writing.
   A description of the format of each of these file types can be found
   in Section~\ref{se:spectrum}\@.

   This parameter has a default value of 0.
}

\iueparameter{TEMFILE}
{
   \_CHAR
}{
   This is the name of a file containing the standard spectrum template
   data.
   A  file type of \verb+.tem+ is assumed and need not be specified
   explicitly.
   If the file name contains a directory specification, then it should be
   enclosed in quotes.
}

\iueparameter{THDA}
{
   \_DOUBLE
}{
   This is the IUE camera temperature, measured in degrees Centigrade.
   It is used for such things as adjustments to fiducial positions
   and spectrograph dispersion relations.
   A value of 0.0 implies that no \verb+THDA+ value is available, the program
   will then  use a suitable mean \verb+THDA+ for the camera being used.
   Values for the \verb+THDA+ can be found in the IUE header text of the final
   spectrum file on the Guest Observer tape for IUESIPS\#2 ---
   \verb+THDA+ values derived from spectrum motion are best.
}

\iueparameter{THRESH}
{
   \_DOUBLE
}{
   This is the  minimum value considered to be good when using \verb+CLEAN+\@.
   All pixels with values below this threshold will be marked BAD.
}

\iueparameter{TYPE}
{
   \_CHAR
}{
   This is the type of dataset.
   Defined values are as follows:

   \begin {quote}
   \begin {description}
      \item {\tt RAW} --- IUE raw image.
      \item {\tt GPHOT} --- IUE GPHOT image (geometric and photometric).
      \item {\tt PHOT} --- IUE PHOT image (photometric only).
   \end {description}
   \end {quote}

   Types \verb+PHOT+ and \verb+GPHOT+ are not automatically distinguishable
   from IUE Guest Observer tape contents.
}

\iueparameter{V}
{
   \_CHAR
}{
   This is a string defining a list of items and includes
   any of the following characters:

   \begin {quote}
   \begin {description}
      \item {\tt H} --- header and file information.
      \item {\tt I} --- image details.
      \item {\tt F} --- fiducials.
      \item {\tt G} --- geometry.
      \item {\tt D} --- dispersion.
      \item {\tt C} --- centroid templates.
      \item {\tt R} --- \'{e}chelle ripple and halation.
      \item {\tt A} --- absolute calibration.
      \item {\tt S} --- raw spectrum.
      \item {\tt M} --- mean spectrum.
      \item {\tt *} --- all of the above.
      \item {\tt Q} --- image data quality summary.
   \end {description}
   \end {quote}
}

\iueparameter{VSHIFT}
{
   \_DOUBLE
}{
   This is the radial velocity of the detector relative to the Sun.
   It is measured in km/s and affects the calibrated wavelength
   scale as follows:

   \begin {equation}
      \lambda_{true} = \frac{\lambda_{obs}}
                            {(1 + \frac{\rm VSHIFT}{c})}
   \end {equation}

   where $c$ is the velocity of light in km/s.
}

\iueparameter{WCUT}
{
   [\_DOUBLE,\_DOUBLE]
}{
   This is one of the mechanisms that can be used to delimit the
   parts of \'{e}chelle orders that are calibrated for ripple response.
   The two values of this parameter are the start and end wavelengths
   for a specific \verb+ORDER+\@.

   Apart from poor ripple calibration, the order ends can also be affected
   by the parts of the camera faceplate that are retained in the image.
}

\iueparameter{WSHIFT}
{
   \_DOUBLE
}{
   This is a constant wavelength shift applied to spectrum wavelengths.
   It is measured in Angstroms and affects the spectrum wavelengths as follows:

   \begin {equation}
      \lambda_{new} = \lambda_{old} + {\rm WSHIFT}
   \end {equation}

   This is only used for LORES spectra.
}

\iueparameter{XCUT}
{
   [\_DOUBLE,\_DOUBLE]
}{
   This is one of the mechanisms that can be used to delimit the
   ends of \'{e}chelle orders that are calibrated for ripple response.
   The two values of this parameter are the start and end $x$ coordinates
   of the order, where

   \begin {equation}
      x = \frac{\pi \times {\rm RIPA} \times (\lambda - \lambda_c)
                \times {\rm ORDER}}
               {\lambda_c}
   \end {equation}

   and

   \begin {equation}
      \lambda_c = \frac {\rm RIPK}{\rm ORDER}
   \end {equation}

   The nature of the standard ripple function is that $x$ is only
   formally meaningful in the range ($-\pi$, $+\pi$).

   See the \verb+RIPA+, \verb+RIPC+ and \verb+RIPK+ parameter descriptions.
}

\iueparameter{XL}
{
   [\_DOUBLE,\_DOUBLE]
}{
   This specifies the data limits used for plotting in the $x$-direction.
   This parameter is only read if the display has been reset, and
   the axes are being redrawn.
   If both values are the same ({\it{e.g.,}}\ \verb+[0, 0]+),
   then the data limits in the $x$-direction will be determined from the
   data being plotted.
}

\iueparameter{XP}
{
   [\_INTEGER,\_INTEGER]
}{
   This specifies the pixel limits along the $x$-direction used for
   image display.
   Values in decreasing order will cause the image to be inverted
   along the $x$-direction.
   If the values are undefined, the pixel limits will default to include
   the whole extent of the image along the $x$-direction.
}

\iueparameter{YEAR}
{
   \_INTEGER
}{
   This is the year (A.D.) used in constructing dates.
}

\iueparameter{YL}
{
   [\_DOUBLE,\_DOUBLE]
}{
   This specifies the data limits used for plotting in the $y$-direction.
   This parameter is only read if the display has been reset, and
   the axes are being redrawn.
   If both values are the same ({\it{e.g.,}}\ \verb+[0, 0]+),
   then the data limits in the $y$-direction will be determined from the
   data being plotted.
}

\iueparameter{YP}
{
   [\_INTEGER,\_INTEGER]
}{
   This specifies the pixel limits along the $y$-direction used for
   image display.
   Values in decreasing order will cause the image to be inverted
   along the $y$-direction.
   If the values are undefined, the pixel limits will default to include
   the whole extent of the image along the $y$-direction.
}

\iueparameter{ZL}
{
   [\_DOUBLE,\_DOUBLE]
}{
   This specifies the data limits used for display of images.
   If the values are given in decreasing order, then high data
   values will be represented by low (dark) display intensities,
   and vice-versa.
   If the values are undefined, then the full intensity range of the
   image will be used.

   Data values which fall at or below the lowest display intensity are drawn
   {\bf black,} those which are at the highest display intensity are drawn
   {\bf white} and those which are above the highest display intensity are
   drawn {\bf blue.}
}

\iueparameter{ZONE}
{
   \_INTEGER
}{
   This specifies the zone to be used for plotting.
   The zone numbers range
   from 0 to 8 and correspond to those defined by the TZONE command in
   DIPSO (see \xref{SUN/50}{sun50}{}), {\it{e.g.,}}

   \begin {quote}
   \begin {description}
      \item {\tt 0} --- entire display surface.
      \item {\tt 1} --- top left quarter.
      \item {\tt 2} --- top right quarter.
      \item {\tt 3} --- bottom left quarter.
      \item {\tt 4} --- bottom right quarter.
      \item {\tt 5} --- top half.
      \item {\tt 6} --- bottom half.
      \item {\tt 7} --- left half.
      \item {\tt 8} --- right half.
   \end {description}
   \end {quote}

   This parameter has a default value of 0.
}
}

\newpage
%------------------------------------------------------------------------------

\appendix
\section{\xlabel{parameter_defaults}\label{se:parameter_defaults}Parameter
          defaults}
\markboth{Parameter defaults}{\stardocname}

Some IUEDR parameters have default values.  Some have no default value and one
{\bf must} be provided.  Other parameters values are either calculated or
simply set by the program.  The default values and/or behaviour of each
parameter are listed here.

\begin {description}

\item [\htmlref{ABSFILE}{ABSFILE}] \lmbox
   No default value exists, a file name must be provided.
   The parameter is cancelled each time it is used.
\item [\htmlref{APERTURE}{APERTURE}] \lmbox
   Has no automatic default value.  An \verb+APERTURE+ must be selected.
   If only one is present in an image then this is taken as \verb+APERTURE+ by
   default.
\item [\htmlref{APERTURES}{APERTURES}] \lmbox
   Used only by \verb+READIUE+ and \verb+READSIPS+\@.  The value must be
   supplied by reading the IUE GO header `by eye'.
\item [\htmlref{AUTOSLIT}{AUTOSLIT} = TRUE] \lmbox
   Whether \verb+GSLIT+, \verb+BDIST+ and \verb+BSLIT+ are determined
   automatically.
\item [\htmlref{BADITF}{BADITF} (= TRUE)] \lmbox
   Has no default value, however it is recommended to be set \verb+TRUE+ as
   this will ensure any ITF error correction available for the particular
   \verb+ITF+ will be used.
   This may seem counter-intuitive, however, data using error-free ITF
   information will not be affected by setting \verb+BADITF=TRUE+\@.
\item [\htmlref{BDIST}{BDIST}] \lmbox
   No automatic default.  The value is calculated if \verb+AUTOSLIT=TRUE+,
   otherwise it should be estimated by looking at \verb+SCAN+ plots.
\item [\htmlref{BKGAV}{BKGAV} = 30.0] \lmbox
   Background averaging filter FWHM (geometric pixels).
\item [\htmlref{BKGIT}{BKGIT} = 1] \lmbox
   Number of background smoothing iterations.
\item [\htmlref{BKGSD}{BKGSD} = 2.0] \lmbox
   Discrimination level for background pixels (standard deviations).
\item [\htmlref{BSLIT}{BSLIT}] \lmbox
   No automatic default.  The value is calculated if \verb+AUTOSLIT=TRUE+,
   otherwise it should be estimated by looking at \verb+SCAN+ plots.
\item [\htmlref{CAMERA}{CAMERA}] \lmbox
   The value is read by the program from an IUE GO header.  The default value
   presented in a prompt will be the value found in the header.
\item [\htmlref{CENAV}{CENAV} = 30.0] \lmbox
   Centroid averaging filter FWHM (geometric pixels).
\item [\htmlref{CENIT}{CENIT} = 1] \lmbox
   Number of centroid tracking iterations.
\item [\htmlref{CENSD}{CENSD} = 4.0] \lmbox
   Discrimination level for signal to be used for centroids
   (standard deviations).
\item [\htmlref{CENSH}{CENSH} = FALSE] \lmbox
   Whether the spectrum template is just shifted linearly.
\item [\htmlref{CENSV}{CENSV} = FALSE] \lmbox
   Whether the spectrum template is saved in the dataset.
\item [\htmlref{CENTM}{CENTM} = FALSE] \lmbox
   Whether an existing centroid template is used.
\item [\htmlref{COL}{COL} (= 1)] \lmbox
   The value of \verb+COL+ will be calculated for each command requiring it.
   If \verb+COLROT=FALSE+ then \verb+COL+ will take the value 1, otherwise it
   will be incremented for each plotting command.
\item [\htmlref{COLOUR}{COLOUR} = FALSE] \lmbox
   Whether a spectrum-style false colour look-up table is used by
   \verb+DRIMAGE+\@.
\item [\htmlref{COLROT}{COLROT} = TRUE] \lmbox
   Whether the line colour is changed after the next plot.
\item [\htmlref{CONTINUUM}{CONTINUUM} = TRUE] \lmbox
   Whether the object spectrum is expected to have a ``continuum''.
\item [\htmlref{COVERGAP}{COVERGAP} = FALSE] \lmbox
   Whether gaps can be filled by covering orders.
\item [\htmlref{CUTFILE}{CUTFILE}] \lmbox
   No default value exists, a file name must be provided.
   The parameter is cancelled each time it is used.
\item [\htmlref{CUTWV}{CUTWV} = TRUE] \lmbox
   Whether wavelength cutoff data is to be used for the extraction grid.
\item [\htmlref{DATASET}{DATASET}] \lmbox
   No automatic default value.  The last value of \verb+DATASET+ used will be
   taken as the default.  The exception to this rule is when using
   \verb+READIUE+ or \verb+READSIPS+ when a suggested default value will be
   presented by the program in the parameter prompt.
\item [\htmlref{DAY}{DAY}] \lmbox
   No default value.  The program will attempt to extract the \verb+DAY+ from
   the IUE GO tape/file header and present this as the default value for that
   particular command.
\item [\htmlref{DEVICE}{DEVICE}] \lmbox
   No automatic default value.  The last value of \verb+DEVICE+ used will be
   taken as the default.
\item [\htmlref{DISPFILE}{DISPFILE}] \lmbox
   No default value exists, a file name must be provided.
   The parameter is cancelled each time it is used.
\item [\htmlref{DRIVE}{DRIVE}] \lmbox
   No automatic default value.  The last value of \verb+DRIVE+ used will be
   taken as the default.
\item [\htmlref{ESHIFT}{ESHIFT}] \lmbox
   No automatic default value.  If an \verb+ESHIFT+ has previously been set for
   the current \verb+DATASET+, this will be presented as the default value in
   the prompt.
\item [\htmlref{EXPOSURE}{EXPOSURE}] \lmbox
   No automatic default value.  The value of \verb+EXPOSURE+ previously set when
   using \verb+READIUE+ or \verb+READSIPS+ for the current \verb+APERTURE+ will
   be presented as the default value in the prompt.
\item [\htmlref{EXPOSURES}{EXPOSURES}] \lmbox
   No default value.  The program will attempt to extract the \verb+EXPOSURES+
   from the IUE GO tape/file header and present this as the default value for
   that particular command.
\item [\htmlref{EXTENDED}{EXTENDED} = FALSE] \lmbox
   Whether the object spectrum is expected to be extended.
\item [\htmlref{FIDFILE}{FIDFILE}] \lmbox
   No default value exists, a file name must be provided.
   The parameter is cancelled each time it is used.
\item [\htmlref{FIDSIZE}{FIDSIZE}] \lmbox
   The default \verb+FIDSIZE+ is read from the appropriate file in the
   \verb+$IUEDR_DATA+ directory.  This is presented as the default in parameter
   prompts.
\item [\htmlref{FILE}{FILE}] \lmbox
   No automatic default.  Takes the value 1 when reading a new tape or file.
\item [\htmlref{FILLGAP}{FILLGAP} = FALSE] \lmbox
   Whether gaps can be filled within an order.
\item [\htmlref{FLAG}{FLAG} = TRUE] \lmbox
   Whether data quality for faulty pixels is displayed.
\item [\htmlref{FN}{FN}] \lmbox
   No default.  A value must be supplied.
\item [\htmlref{FSCALE}{FSCALE} = 1] \lmbox
   No scale factor is applied by default.
\item [\htmlref{GSAMP}{GSAMP} = 1.414] \lmbox
   Spectrum grid sampling rate (geometric pixels).
\item [\htmlref{GSHIFT}{GSHIFT}] \lmbox
   No automatic default value.  The value of \verb+GSHIFT+ previously set when
   using \verb+CGSHIFT+ will be presented as the default value in the prompt.
\item [\htmlref{GSLIT}{GSLIT}] \lmbox
   No automatic default.  The value is calculated if \verb+AUTOSLIT=TRUE+,
   otherwise it should be estimated by looking at \verb+SCAN+ plots.
\item [\htmlref{HALAV}{HALAV} = 30.0] \lmbox
   FWHM of triangle function used for spectrum smoothing.
\item [\htmlref{HALC}{HALC}] \lmbox
   No default value.  Values must be positive or zero.
\item [\htmlref{HALTYPE}{HALTYPE} = POWER] \lmbox
   Currently this parameter can only take the value \verb+POWER+\@.
\item [\htmlref{HALW0}{HALW0} = 1400.0 or 1800.0] \lmbox
   Takes the value \verb+1400.0+ for the SWP camera, \verb+1800.0+ for
   LWP and LWR cameras.
\item [\htmlref{HALWC}{HALWC} = 1200.0 or 2400.0] \lmbox
   Takes the value \verb+1200.0+ for the SWP camera, \verb+2400.0+ for
   LWP and LWR cameras.
\item [\htmlref{HIST}{HIST} = TRUE] \lmbox
   Whether lines are to be drawn as histograms during plotting.
\item [\htmlref{IMAGE}{IMAGE}] \lmbox
   No default value.  The program will attempt to extract the \verb+IMAGE+
   number from the IUE GO tape/file header and present this as the default
   value for that particular command.
\item [\htmlref{ITF}{ITF}] \lmbox
   No default value.  Refer to page~\pageref{ITF} for details of
   working out which transfer function to use.
\item [\htmlref{ITFMAX}{ITFMAX}] \lmbox
   No default value.  Refer to page~\pageref{ITFMAX} for details of
   working out which transfer function saturation value to use.
\item [\htmlref{LINE}{LINE} (=SOLID)] \lmbox
   The value of LINE will be calculated for each command requiring it.
   If \verb+LINEROT=FALSE+ then \verb+LINE+ will take the value \verb+SOLID+,
   otherwise it will be rotated for each plotting command.
\item [\htmlref{LINEROT}{LINEROT} = FALSE] \lmbox
   Whether line style is to be changed after the next plot.
\item [\htmlref{ML}{ML}] \lmbox
   No default value.  Limits {\bf must} be specified.
\item [\htmlref{MONTH}{MONTH}] \lmbox
   No default value.  The program will attempt to extract the \verb+MONTH+ from
   the IUE GO tape/file header and present this as the default value for that
   particular command.
\item [\htmlref{MSAMP}{MSAMP}] \lmbox
   No default value exists.
\item [\htmlref{NFILE}{NFILE} = 1] \lmbox
   Number of tape files to be processed.
\item [\htmlref{NGEOM}{NGEOM} = 5] \lmbox
   When reading IUE GO tapes/files this is the value suggested.
\item [\htmlref{NLINE}{NLINE} = 10] \lmbox
   Number of IUE header lines to be printed.
\item [\htmlref{NORDER}{NORDER} = 0] \lmbox
   Number of \'{e}chelle orders to be processed.
\item [\htmlref{NSKIP}{NSKIP} = 1] \lmbox
   Number of tape marks to be skipped over.
\item [\htmlref{OBJECT}{OBJECT}] \lmbox
   The program will attempt to extract the \verb+OBJECT+ description text from
   the IUE GO header.  This is presented as the default in parameter prompts.
\item [\htmlref{ORDER}{ORDER}] \lmbox
   No default value. A valid \verb+ORDER+ {\bf must} be specified.
   Use \verb+SHOW V=S+ to find out which orders have been extracted from a
   HIRES image.
\item [{\htmlref{ORDERS}{ORDERS} (=[125,66])}] \lmbox
   Takes the values given by default for a new dataset and otherwise in
   response to a parameter cancel.
\item [\htmlref{OUTFILE}{OUTFILE}] \lmbox
   No automatic default.  The program will construct a suggested file name
   based on the \verb+CAMERA+, \verb+RESOLUTION+, \verb+APERTURE+ and
   \verb+ORDER+ as appropriate.
\item [\htmlref{QUAL}{QUAL} = TRUE] \lmbox
   Whether data quality information is plotted.
\item [\htmlref{RESOLUTION}{RESOLUTION}] \lmbox
   The program will attempt to extract the spectrograph \verb+RESOLUTION+ from
   the IUE GO header.  This is presented as the default in parameter prompts.
\item [\htmlref{RIPA}{RIPA} (=1)] \lmbox
   This parameter takes the value 1 when a new dataset is created.
\item [{\htmlref{RIPC}{RIPC} (=[1,0,0,0,0,0])}] \lmbox
   Takes the default values given above when a new dataset is created.
\item [\htmlref{RIPFILE}{RIPFILE}] \lmbox
   No default value exists, a file name must be provided.
   The parameter is cancelled each time it is used.
\item [\htmlref{RIPK}{RIPK}] \lmbox
   Takes the central wavelength value of an order by default.
\item [\htmlref{RL}{RL}] \lmbox
   No default.  \verb+RL=[0.0, 0.0]+ indicates the program should calculate
   values.
\item [\htmlref{RM}{RM} = TRUE] \lmbox
   Whether the mean spectrum is reset before averaging.
\item [\htmlref{RS}{RS} = TRUE] \lmbox
   Whether the display is reset before plotting.
\item [\htmlref{RSAMP}{RSAMP} = 1.414] \lmbox
   Radial coordinate sampling rate for \verb+LBLS+ grid (pixels).
\item [\htmlref{SCANAV}{SCANAV} = 5] \lmbox
   Averaging filter HWHM for image scan (geometric pixels).
\item [\htmlref{SCANDIST}{SCANDIST} (=0)] \lmbox
   Takes the last value used.  The first time \verb+SCAN+ is used the
   program will suggest a value of zero.
\item [\htmlref{SCANWV}{SCANWV}] \lmbox
   The value of \verb+SCANWV+ taken by default is calculated as the centre
   of the wavelength scale for the current \verb+APERTURE+\@.
\item [\htmlref{SKIPNEXT}{SKIPNEXT} = FALSE] \lmbox
   Whether skip to next tape file.
\item [\htmlref{SPECTYPE}{SPECTYPE} = 0] \lmbox
   DIPSO SP format file type (0, 1 or 2). Type 0 is a Starlink NDF.
\item [\htmlref{TEMFILE}{TEMFILE}] \lmbox
   No default value exists, a file name must be provided.
   The parameter is cancelled each time it is used.
\item [\htmlref{THDA}{THDA} (=0.0)] \lmbox
   No default, however, \verb+THDA=0.0+ implies that no value is available
   and the program will select a suitable mean \verb+THDA+ for the camera
   being used.
\item [\htmlref{THRESH}{THRESH}] \lmbox
   No default value.  The last value used will be presented as the default
   in parameter prompts.
\item [\htmlref{TYPE}{TYPE}] \lmbox
   No default.  This must be evaluated from the IUE GO file header contents.
\item [\htmlref{V}{V} = H] \lmbox
   This parameter is cancelled each time it is used.
\item [\htmlref{VSHIFT}{VSHIFT}] \lmbox
   No automatic default value.  If an VSHIFT has previously been set for
   the current \verb+DATASET+, this will be presented as the default value in
   the prompt.
\item [\htmlref{WCUT}{WCUT}] \lmbox
   The wavelength cut-off values are normally read from a \verb+.cut+ file and
   these are used for \verb+WCUT+ prompt values by default.
\item [\htmlref{WSHIFT}{WSHIFT} = 0] \lmbox
   No automatic default value.  If a \verb+WSHIFT+ has previously been set for
   the current \verb+DATASET+, this will be presented as the default value in
   the parameter prompt.
\item [{\htmlref{XCUT}{XCUT} =[-3.0,3.0]}] \lmbox
   Takes the value \verb+[-3.0,3.0]+ for a new dataset.
\item [{\htmlref{XL}{XL} (=[0,0])}] \lmbox
   No default value.  The limits are taken as the full x-range in the data
   to be plotted if no value of \verb+YL+ is set.
\item [{\htmlref{XP}{XP} (=[0,0])}] \lmbox
   No default value.  The limits are taken as the full image width if \verb+YP+
   is not set.
\item [\htmlref{YEAR}{YEAR}] \lmbox
   No default value.  The program will attempt to extract the \verb+YEAR+ from
   the IUE GO tape/file header and present this as the default value for that
   particular command.
\item [{\htmlref{YL}{YL} (=[0,0])}] \lmbox
   No default value.  The limits are taken as the full y-range in the data
   to be plotted if no value of \verb+YL+ is set.
\item [{\htmlref{YP}{YP} (=[0,0])}] \lmbox
   No default value.  The limits are taken as the full image height if
   \verb+YP+ is not set.
\item [{\htmlref{ZL}{ZL} (=[0,0])}] \lmbox
   No default value.  The limits are taken as the full intensity range for the
   current \verb+DATASET+ if \verb+ZL+ is not set.
\item [\htmlref{ZONE}{ZONE} = 0] \lmbox
   Graphics zone to be used for plotting.

\end {description}

\newpage
\section{\xlabel{vms_data}\label{se:vmsunix}Handling of VMS IUEDR data files}
\markboth{Handling of VMS IUEDR data files}{\stardocname}

IUEDR data files have changed format in order to allow
inter-machine operation. However, VMS IUEDR will still read the old format
files if they are present (this is only useful on the VAX as all old
format files will have been created on VAXen). If you have old format files
then you should use \verb+iuecnv+ to convert them to the new format by
following the procedure described below.
The resulting files can then be used with UNIX IUEDR.

During the conversion of an IUE dataset \verb+iuecnv+ will create a number of
binary data files. Their filenames are as follows:

\begin {description}
   \item \verb+<dataset>.UEC+ --- calibration file.
   \item \verb+<dataset>_UED.SDF+ --- image data and quality file.
   \item \verb+<dataset>_UES.SDF+ --- uncalibrated spectrum file.
   \item \verb+<dataset>_UEM.SDF+ --- calibrated mean spectrum file.
\end {description}

where \verb+<dataset>+ refers to the IUEDR DATASET parameter.

The {\tt .SDF} files are STARLINK NDF format files and can be read and
processed by any of the standard  packages ({\it{e.g.,}}\ KAPPA,FIGARO).
These files are in a machine independent format and can be freely
copied between any of the platforms which STARLINK supports.

\subsection{Moving IUEDR files to UNIX systems}

The file formats used by UNIX IUEDR are based on the STARLINK standard
NDF library. This makes the files transportable between all supported
systems. If you have old VMS IUEDR files ({\it{i.e.}}\ {\tt .UEC}, {\tt
.UED}, {\tt .UES}, {\tt .UEM} files) then these will need to be
converted into the new format {\bf before} they are transferred to a UNIX
system.

There is a VMS executable provided for this purpose, and a command file
to use it. To use the executable you must first copy:

\begin{verbatim}
   /star/bin/iuedr/iuecnv.exe
   /star/bin/iuedr/iuecnv.com
\end{verbatim}

onto your VMS system (use binary transfer for {\tt iuecnv.exe}).

When {\tt iuecnv.exe} is installed, you can then move to a directory
where your IUEDR datasets are stored and type:

\begin{verbatim}
   $ @somedisk:[somedir]iuecnv  dataset
\end{verbatim}

where {\tt somedisk:[somedir]} is wherever you copied {\tt iuecnv} to, and
{\tt dataset} is the name of an IUE dataset ({\it{e.g.,}}\ SWP23456).

The program will then locate and convert all the IUEDR datafiles
associated with the named dataset. Note that the {\tt .UEC} file is also
converted (although its name stays the same).

When conversion is complete you may copy the files ({\tt .UEC} and {\tt
*\_UE*.SDF}) to your UNIX system. The {\tt .UEC} files MUST be
transferred in ascii mode, and the {\tt .SDF} files MUST be transferred
in binary mode.

UNIX NDF expects that NDF container files end in the extension \verb+.sdf+ and
does not yet recognise \verb+.SDF+ files. Thus you may need to rename
files to have the lowercase \verb+.sdf+ extension (depending upon how
you do the transfer).

\subsection{VAX-UNIX IUEDR image file exchange}

An IUEDR image file is one of \verb+RAW+, \verb+PHOT+, or \verb+GPHOT+ type
and consist of 768 $\times$ 768 pixels each stored in a 1 or 2-byte integer.

The transfer of files between VAX and UNIX systems is complicated by
the sophistication of the VAX file system. Under VMS the system
records a complex description of the precise format of all the  files
(and stores it in  the directory entry). Under UNIX this information
has to be provided by the user of the file when it is  opened. Because
of this difference it is sometimes necessary to use the following
format conversion utilities.

\subsubsection{VAX to UNIX}

If you wish to transfer IUE image data from a VAX onto a UNIX machine
in order to use the UNIX IUEDR then the transfer should be done using
FTP (in {\bf binary} mode).

If you intend to copy the file using some other method ({\it{e.g.,}}\ via NFS)
then you should first use the command:

\begin{verbatim}
   UNIX_FORMAT image-name
\end{verbatim}

to ensure the file is properly transferred.

Note that this also applies if you wish to just access the file
via NFS without explicitly transferring it.

\subsubsection{UNIX to VAX}

If you wish to transfer IUE image data from a UNIX machine onto a VAX
in order to use the VAX IUEDR then the transfer should be done using
FTP (in {\bf binary} mode) and the command:

\begin{verbatim}
   VAX_FORMAT  image-name
\end{verbatim}

should then be used to ensure the file has the correct format.

If you use some other method of transferring the file ({\it{e.g.,}}\ NFS) then
the above command is {\bf still} required.

\subsubsection{What will work?}

In general the following two commands will allow you to use any disk
based IUE image with any machine running IUEDR:

\begin{itemize}
   \item {\tt VAX\_FORMAT} sets the file format as required by VAX IUEDR
   \item {\tt UNIX\_FORMAT} sets the file format as required by UNIX IUEDR
\end{itemize}

Both commands operate only on the VAX.

\subsection{\label{se:nfs}Accessing data via NFS}

UNIX machines commonly provide disk sharing amongst remote machines
using the NFS protocol.

For example your data frame may reside on a DECstation local  disk
called \verb+iuedata+ in the Rutherford cluster on machine \verb+adam4+\@. In
order to get IUEDR to read it directly you could enter the following
in response to the DRIVE prompt:

\begin{verbatim}
   DRIVE> /adam4/iuedata/swp12345.raw
\end{verbatim}

To see which disks you have NFS access to you should use the {\tt
\%~df} command. In general any disks whose entry does not start with
\verb+/dev+ are being served by a remote machine.

{\bf IMPORTANT NOTE\\}
IUEDR allows you to use this method of data access with the following proviso.

If the data resides on a VAX served disk then you must first  convert
its directory entry (on the VAX) using the following command:

\begin{verbatim}
   $ UNIX_FORMAT image-name
\end{verbatim}

This command does not change the data in any way.  It merely alters the
description of the file format as stored in the VAX directory.

If at some later stage you wish to use the VAX version of IUEDR on the
same data file it first be necessary to use the command:

\begin{verbatim}
   $ VAX_FORMAT image-name
\end{verbatim}

to convert back.

\twocolumn[
\section{\label{se:index}Command and parameter index}
]
\markboth{Index}{\stardocname}
\findexentry{A}{ABSFILE}{46}
\indexentry{AESHIFT}{14}
\indexentry{AGSHIFT}{14}
\indexentry{APERTURES}{46}
\indexentry{APERTURE}{46}
\indexentry{AUTOSLIT}{47}
\findexentry{B}{BADITF}{47}
\indexentry{BARKER}{15}
\indexentry{BDIST}{47}
\indexentry{BKGAV}{47}
\indexentry{BKGIT}{47}
\indexentry{BKGSD}{48}
\indexentry{BSLIT}{48}
\findexentry{C}{CAMERA}{48}
\indexentry{CENAV}{48}
\indexentry{CENIT}{48}
\indexentry{CENSD}{48}
\indexentry{CENSH}{48}
\indexentry{CENSV}{48}
\indexentry{CENTM}{49}
\indexentry{CENTREWAVE}{49}
\indexentry{CGSHIFT}{15}
\indexentry{CLEAN}{16}
\indexentry{COLOUR}{49}
\indexentry{COLROT}{49}
\indexentry{COL}{49}
\indexentry{CONTINUUM}{50}
\indexentry{COVERGAP}{50}
\indexentry{CULIMITS}{16}
\indexentry{CURSOR}{17}
\indexentry{CUTFILE}{50}
\indexentry{CUTWV}{50}
\findexentry{D}{DATASET}{50}
\indexentry{DAY}{50}
\indexentry{DELTAWAVE}{50}
\indexentry{DEVICE}{50}
\indexentry{DISPFILE}{50}
\indexentry{DRIMAGE}{17}
\indexentry{DRIVE}{51}
\indexentry{EDIMAGE}{18}
\findexentry{E}{EDMEAN}{19}
\indexentry{EDSPEC}{20}
\indexentry{ERASE}{20}
\indexentry{ESHIFT}{51}
\indexentry{EXIT}{21}
\indexentry{EXPOSURES}{51}
\indexentry{EXPOSURE}{51}
\indexentry{EXTENDED}{51}
\findexentry{F}{FIDFILE}{51}
\indexentry{FIDSIZE}{51}
\indexentry{FILE}{51}
\indexentry{FILLGAP}{52}
\indexentry{FLAG}{52}
\indexentry{FN}{52}
\indexentry{FSCALE}{52}
\findexentry{G}{GSAMP}{52}
\indexentry{GSHIFT}{52}
\indexentry{GSLIT}{53}
\findexentry{H}{HALAV}{53}
\indexentry{HALC}{53}
\indexentry{HALTYPE}{53}
\indexentry{HALW0}{53}
\indexentry{HALWC}{53}
\indexentry{HELP}{21}
\indexentry{HIST}{53}
\findexentry{I}{IMAGE}{53}
\indexentry{ITFMAX}{54}
\indexentry{ITF}{54}
\findexentry{L}{LBLS}{21}
\indexentry{LINEROT}{55}
\indexentry{LINE}{54}
\indexentry{LISTIUE}{22}
\findexentry{M}{MAP}{23}
\indexentry{ML}{55}
\indexentry{MODIMAGE}{23}
\indexentry{MONTH}{55}
\indexentry{MSAMP}{55}
\indexentry{MTMOVE}{24}
\indexentry{MTREW}{24}
\indexentry{MTSHOW}{24}
\indexentry{MTSKIPEOV}{25}
\indexentry{MTSKIPF}{25}
\findexentry{N}{NEWABS}{25}
\indexentry{NEWCUT}{26}
\indexentry{NEWDISP}{26}
\indexentry{NEWFID}{26}
\indexentry{NEWRIP}{27}
\indexentry{NEWTEM}{27}
\indexentry{NFILE}{55}
\indexentry{NGEOM}{55}
\indexentry{NLINE}{55}
\indexentry{NORDER}{55}
\indexentry{NSKIP}{55}
\findexentry{O}{OBJECT}{56}
\indexentry{ORDERS}{56}
\indexentry{ORDER}{56}
\indexentry{OUTEM}{27}
\indexentry{OUTFILE}{56}
\indexentry{OUTLBLS}{28}
\indexentry{OUTMEAN}{28}
\indexentry{OUTNET}{29}
\indexentry{OUTRAK}{29}
\indexentry{OUTSCAN}{30}
\indexentry{OUTSPEC}{30}
\findexentry{P}{PLCEN}{31}
\indexentry{PLFLUX}{32}
\indexentry{PLGRS}{33}
\indexentry{PLMEAN}{34}
\indexentry{PLNET}{35}
\indexentry{PLSCAN}{36}
\indexentry{PRGRS}{37}
\indexentry{PRLBLS}{37}
\indexentry{PRMEAN}{37}
\indexentry{PRSCAN}{38}
\indexentry{PRSPEC}{38}
\findexentry{Q}{QUAL}{56}
\indexentry{QUIT}{38}
\findexentry{R}{READIUE}{39}
\indexentry{READSIPS}{40}
\indexentry{RESOLUTION}{56}
\indexentry{RIPA}{57}
\indexentry{RIPC}{57}
\indexentry{RIPFILE}{57}
\indexentry{RIPK}{57}
\indexentry{RL}{58}
\indexentry{RM}{58}
\indexentry{RSAMP}{58}
\indexentry{RS}{58}
\findexentry{S}{SAVE}{40}
\indexentry{SCANAV}{58}
\indexentry{SCANDIST}{58}
\indexentry{SCANWV}{58}
\indexentry{SCAN}{41}
\indexentry{SETA}{41}
\indexentry{SETD}{42}
\indexentry{SETM}{42}
\indexentry{SGS}{43}
\indexentry{SHOW}{43}
\indexentry{SKIPNEXT}{58}
\indexentry{SPECTYPE}{59}
\findexentry{T}{TEMFILE}{59}
\indexentry{THDA}{59}
\indexentry{THRESH}{59}
\indexentry{TRAK}{44}
\indexentry{TYPE}{59}
\findexentry{V}{VSHIFT}{60}
\indexentry{V}{59}
\findexentry{W}{WCUT}{60}
\indexentry{WSHIFT}{60}
\findexentry{X}{XCUT}{60}
\indexentry{XL}{61}
\indexentry{XP}{61}
\findexentry{Y}{YEAR}{61}
\indexentry{YL}{61}
\indexentry{YP}{61}
\findexentry{Z}{ZL}{61}
\indexentry{ZONE}{61}

%\documentstyle[11pt,twoside]{article}
\pagestyle{myheadings}
\makeindex

%------------------------------------------------------------------------------
\newcommand{\stardoccategory}  {Starlink Guide}
\newcommand{\stardocinitials}  {SG}
\newcommand{\stardocsource}    {sg3.5}
\newcommand{\stardocnumber}    {3.5}
\newcommand{\stardocauthors}   {Paul Rees, Jack Giddings, Dave Mills \& Martin Clayton}
\newcommand{\stardocdate}      {12 March 1996}
\newcommand{\stardoctitle}     {IUEDR---Reference Manual}
%------------------------------------------------------------------------------


\newcommand{\stardocname}{\stardocinitials /\stardocnumber}
\newcommand{\numcir}[1]{\mbox{\hspace{3ex}$\bigcirc$\hspace{-1.7ex}{\small #1}}}
\newcommand{\lsk}{\raisebox{-0.4ex}{\rm *}}
%\renewcommand{\_}{{\tt\char'137}}     % re-centres the underscore - DONE LATER
\markright{\stardocname}
\setlength{\textwidth}{160mm}
\setlength{\textheight}{230mm}
\setlength{\topmargin}{-2mm}
\setlength{\oddsidemargin}{0mm}
\setlength{\evensidemargin}{0mm}
\setlength{\parindent}{0mm}
\setlength{\parskip}{\medskipamount}
\setlength{\unitlength}{1mm}


% -----------------------------------------------------------------------------
%  Hypertext definitions.
%  ======================
%  These are used by the LaTeX2HTML translator in conjuction with star2html.

%  Comment.sty: version 2.0, 19 June 1992
%  Selectively in/exclude pieces of text.
%
%  Author
%    Victor Eijkhout                                      <eijkhout@cs.utk.edu>
%    Department of Computer Science
%    University Tennessee at Knoxville
%    104 Ayres Hall
%    Knoxville, TN 37996
%    USA

%  Do not remove the %\begin{rawtex} and %\end{rawtex} lines (used by
%  star2html to signify raw TeX that latex2html cannot process).
%\begin{rawtex}
\makeatletter
\def\makeinnocent#1{\catcode`#1=12 }
\def\csarg#1#2{\expandafter#1\csname#2\endcsname}

\def\ThrowAwayComment#1{\begingroup
    \def\CurrentComment{#1}%
    \let\do\makeinnocent \dospecials
    \makeinnocent\^^L% and whatever other special cases
    \endlinechar`\^^M \catcode`\^^M=12 \xComment}
{\catcode`\^^M=12 \endlinechar=-1 %
 \gdef\xComment#1^^M{\def\test{#1}
      \csarg\ifx{PlainEnd\CurrentComment Test}\test
          \let\html@next\endgroup
      \else \csarg\ifx{LaLaEnd\CurrentComment Test}\test
            \edef\html@next{\endgroup\noexpand\end{\CurrentComment}}
      \else \let\html@next\xComment
      \fi \fi \html@next}
}
\makeatother

\def\includecomment
 #1{\expandafter\def\csname#1\endcsname{}%
    \expandafter\def\csname end#1\endcsname{}}
\def\excludecomment
 #1{\expandafter\def\csname#1\endcsname{\ThrowAwayComment{#1}}%
    {\escapechar=-1\relax
     \csarg\xdef{PlainEnd#1Test}{\string\\end#1}%
     \csarg\xdef{LaLaEnd#1Test}{\string\\end\string\{#1\string\}}%
    }}

%  Define environments that ignore their contents.
\excludecomment{comment}
\excludecomment{rawhtml}
\excludecomment{htmlonly}
%\end{rawtex}

%  Hypertext commands etc. This is a condensed version of the html.sty
%  file supplied with LaTeX2HTML by: Nikos Drakos <nikos@cbl.leeds.ac.uk> &
%  Jelle van Zeijl <jvzeijl@isou17.estec.esa.nl>. The LaTeX2HTML documentation
%  should be consulted about all commands (and the environments defined above)
%  except \xref and \xlabel which are Starlink specific.

\newcommand{\htmladdnormallinkfoot}[2]{#1\footnote{#2}}
\newcommand{\htmladdnormallink}[2]{#1}
\newcommand{\htmladdimg}[1]{}
\newenvironment{latexonly}{}{}
\newcommand{\hyperref}[4]{#2\ref{#4}#3}
\newcommand{\htmlref}[2]{#1}
\newcommand{\htmlimage}[1]{}
\newcommand{\htmladdtonavigation}[1]{}

%  Starlink cross-references and labels.
\newcommand{\xref}[3]{#1}
\newcommand{\xlabel}[1]{}

%  LaTeX2HTML symbol.
\newcommand{\latextohtml}{{\bf LaTeX}{2}{\tt{HTML}}}

%  Define command to recentre underscore for Latex and leave as normal
%  for HTML (severe problems with \_ in tabbing environments and \_\_
%  generally otherwise).
\newcommand{\latex}[1]{#1}
\newcommand{\setunderscore}{\renewcommand{\_}{{\tt\symbol{95}}}}
\latex{\setunderscore}

%  Redefine the \tableofcontents command. This procrastination is necessary
%  to stop the automatic creation of a second table of contents page
%  by latex2html.
\newcommand{\latexonlytoc}[0]{\tableofcontents}

% -----------------------------------------------------------------------------
%  Debugging.
%  =========
%  Un-comment the following to debug links in the HTML version using Latex.

% \newcommand{\hotlink}[2]{\fbox{\begin{tabular}[t]{@{}c@{}}#1\\\hline{\footnotesize #2}\end{tabular}}}
% \renewcommand{\htmladdnormallinkfoot}[2]{\hotlink{#1}{#2}}
% \renewcommand{\htmladdnormallink}[2]{\hotlink{#1}{#2}}
% \renewcommand{\hyperref}[4]{\hotlink{#1}{\S\ref{#4}}}
% \renewcommand{\htmlref}[2]{\hotlink{#1}{\S\ref{#2}}}
% \renewcommand{\xref}[3]{\hotlink{#1}{#2 -- #3}}
% -----------------------------------------------------------------------------
%  Add any document specific \newcommand or \newenvironment commands here

\newcommand{\lmbox}
{
    \mbox{} \\
}

\newcommand{\cpar}[2]
{
    \makebox[30mm][l]{\bf #1} & #2 (p~\pageref{#1}.)\\
}

\newcommand{\cparc}[1]
{
    \makebox[30mm][l]{ } & #1\\
}

\newcommand{\npar}[1]
{
    \makebox[30mm][l]{\bf #1} & \\
}

\newcommand{\iueparlist}[1]{
   \begin{description}
      #1
   \end{description}
}

\newcommand{\iueparameter}[3]
{
   \item [\label{#1}\index{#1}#1 = #2] \mbox{}\\
   #3
}

\newcommand{\indexentry}[2]
{
{\bf #1}\dotfill #2 \hspace*{15mm}\\
}

\newcommand{\findexentry}[3]
{
   \hspace*{\fill}\vspace*{3mm}\\
   \hspace*{\fill}{\large\bf --- #1 ---}\hspace*{\fill} \hspace*{15mm}\\
   \hspace*{\fill}\vspace*{-3mm}\\
   {\bf #2}\dotfill #3 \hspace*{15mm}\\
}

\newcommand{\comdescenv}[1]
{
\begin {tabular}{ll}
  #1
\end {tabular}
}

\newcommand{\comdesc}[2]
{
   \makebox[27mm][l]{\bf #1} & #2 \\
}

\newcommand{\comdescc}[1]
{
   \makebox[27mm][l]{ } & #1 \\
}


%% Redefine commands for hypertext version.

\begin{htmlonly}

\renewcommand{\lmbox}
{ }

\renewcommand{\cpar}[2]
{
    \item [\htmlref{#1}{#1}] #2
}

\renewcommand{\cparc}[1]
{
  #1
}

\rerenewcommand{\npar}[1]
{
    \item [#1]
}

\renewcommand{\indexentry}[2]
{
    {\bf \htmlref{#1}{#1}}\\
}

\renewcommand{\findexentry}[3]
{
    {\bf \htmlref{#2}{#2}}\\
}

\renewcommand{\comdescenv}[1]
{
   \begin{description}
       #1
   \end{description}
}

\renewcommand{\comdesc}[2]
{
   \item [{\bf \htmlref{#1}{#1}}] #2
}

\newcommand{\comdescc}[1]
{
  #1
}

\renewcommand{\iueparlist}[1]
{
      #1
}

\renewcommand{\iueparameter}[3]
{
\subsection{\xlabel{#1}\label{#1}#1}
   \begin{description}
   \item [{\bf Type:}] #2
   \item [{\bf Description:}] #3
   \end{description}
}

\end{htmlonly}

%+
%  Name:
%     SST.TEX

%  Purpose:
%     Define LaTeX commands for laying out Starlink routine descriptions.

%  Language:
%     LaTeX

%  Type of Module:
%     LaTeX data file.

%  Description:
%     This file defines LaTeX commands which allow routine documentation
%     produced by the SST application PROLAT to be processed by LaTeX and
%     by LaTeX2html. The contents of this file should be included in the
%     source prior to any statements that make of the sst commnds.

%  Notes:
%     The commands defined in the style file html.sty provided with LaTeX2html
%     are used. These should either be made available by using the appropriate
%     sun.tex (with hypertext extensions) or by putting the file html.sty
%     on your TEXINPUTS path (and including the name as part of the
%     documentstyle declaration).

%  Authors:
%     RFWS: R.F. Warren-Smith (STARLINK)
%     PDRAPER: P.W. Draper (Starlink - Durham University)

%  History:
%     10-SEP-1990 (RFWS):
%        Original version.
%     10-SEP-1990 (RFWS):
%        Added the implementation status section.
%     12-SEP-1990 (RFWS):
%        Added support for the usage section and adjusted various spacings.
%     8-DEC-1994 (PDRAPER):
%        Added support for simplified formatting using LaTeX2html.
%     {enter_further_changes_here}

%  Bugs:
%     {note_any_bugs_here}

% -

%  Define length variables.
\newlength{\sstbannerlength}
\newlength{\sstcaptionlength}
\newlength{\sstexampleslength}
\newlength{\sstexampleswidth}

%  Define a \tt font of the required size.
\newfont{\ssttt}{cmtt10 scaled 1095}

%  Define a command to produce a routine header, including its name,
%  a purpose description and the rest of the routine's documentation.
\newcommand{\sstroutine}[3]{
   \goodbreak
   \rule{\textwidth}{0.5mm}
   \vspace{-7ex}
   \newline
   \settowidth{\sstbannerlength}{{\Large {\bf #1}}}
   \setlength{\sstcaptionlength}{\textwidth}
   \setlength{\sstexampleslength}{\textwidth}
   \addtolength{\sstbannerlength}{0.5em}
   \addtolength{\sstcaptionlength}{-2.0\sstbannerlength}
   \addtolength{\sstcaptionlength}{-5.0pt}
   \settowidth{\sstexampleswidth}{{\bf Examples:}}
   \addtolength{\sstexampleslength}{-\sstexampleswidth}
   \parbox[t]{\sstbannerlength}{\flushleft{\Large {\bf #1}}}
   \parbox[t]{\sstcaptionlength}{\center{\Large #2}}
   \parbox[t]{\sstbannerlength}{\flushright{\Large {\bf #1}}}
   \label{#1}\index{#1}
   \begin{description}
      #3
   \end{description}
}

%  Format the description section.
\newcommand{\sstdescription}[1]{\item {\bf Description:}\vspace*{6pt}\\ #1}

%  Format the usage section.
\newcommand{\sstusage}[1]{\item[Usage:] \mbox{} \\[1.3ex] {\ssttt #1}}

%  Format the invocation section.
\newcommand{\sstinvocation}[1]{\item[Invocation:]\hspace{0.4em}{\tt #1}}

%  Format the arguments section.
\newcommand{\sstarguments}[1]
{
   \item[Arguments:] \mbox{} \\
   \vspace{-3.5ex}
   \begin{description}
      #1
   \end{description}
}

%  Format the returned value section (for a function).
\newcommand{\sstreturnedvalue}[1]{
   \item[Returned Value:] \mbox{} \\
   \vspace{-3.5ex}
   \begin{description}
      #1
   \end{description}
}

%  Format the parameters section (for an application).
\newcommand{\sstparameters}[1]{
\item {\bf Parameters:\vspace*{6pt}\\}
    \begin{tabular}{ll}
    #1
    \end{tabular}
}

%  Format the examples section.
\newcommand{\sstexamples}[1]{
   \item[Examples:] \mbox{} \\
   \vspace{-3.5ex}
   \begin{description}
      #1
   \end{description}
}

%  Define the format of a subsection in a normal section.
\newcommand{\sstsubsection}[1]{ \item[{#1}] \mbox{} \\}

%  Define the format of a subsection in the examples section.
\newcommand{\sstexamplesubsection}[2]{\sloppy
\item[\parbox{\sstexampleslength}{\ssttt #1}] \mbox{} \\ #2 }

%  Format the notes section.
\newcommand{\sstnotes}[1]{\item[Notes:] \mbox{} \\[1.3ex] #1}

%  Provide a general-purpose format for additional (DIY) sections.
\newcommand{\sstdiytopic}[2]{\item[{\hspace{-0.35em}#1\hspace{-0.35em}:}] \mbox{} \\[1.3ex] #2}

%  Format the implementation status section.
\newcommand{\sstimplementationstatus}[1]{
   \item[{Implementation Status:}] \mbox{} \\[1.3ex] #1}

%  Format the bugs section.
\newcommand{\sstbugs}[1]{\item[Bugs:] #1}

%  Format a list of items while in paragraph mode.
\newcommand{\sstitemlist}[1]{
  \mbox{} \\
  \vspace{-3.5ex}
  \begin{itemize}
     #1
  \end{itemize}
}

%  Define the format of an item.
\newcommand{\sstitem}{\item}

%% Now define html equivalents of those already set. These are used by
%  latex2html and are defined in the html.sty files.

\begin{htmlonly}

%  Re-define \ssttt.
   \newcommand{\ssttt}{\tt}

%  sstroutine.
   \renewcommand{\sstroutine}[3]{
\subsection{\xlabel{#1}\label{#1}#1}
      \begin{description}
         \item[{\bf Purpose:}] #2
         #3
      \end{description}
   }

%  sstdescription
   \renewcommand{\sstdescription}[1]{
      \item[{\bf Description:}]
      \begin{description}
         #1
      \end{description}
   }

%  sstusage
   \renewcommand{\sstusage}[1]{\item[Usage:]
      \begin{description}
         {\ssttt #1}
      \end{description}
   }

%  sstinvocation
   \renewcommand{\sstinvocation}[1]{\item[Invocation:]
      \begin{description}
         {\ssttt #1}
      \end{description}
   }

%  sstarguments
   \renewcommand{\sstarguments}[1]{
      \item[Arguments:]
      \begin{description}
         #1
      \end{description}
   }

%  sstreturnedvalue
   \renewcommand{\sstreturnedvalue}[1]{
      \item[Returned Value:]
      \begin{description}
         #1
      \end{description}
   }

%  sstparameters
   \renewcommand{\sstparameters}[1]{
      \item[{\bf Parameters:}]
      \begin{description}
         #1
      \end{description}
   }

%  sstexamples
   \renewcommand{\sstexamples}[1]{
      \item[Examples:]
      \begin{description}
         #1
      \end{description}
   }

%  sstsubsection
   \renewcommand{\sstsubsection}[1]{\item[{#1}]}

%  sstexamplesubsection
   \renewcommand{\sstexamplesubsection}[2]{\item[{\ssttt #1}] \\ #2}

%  sstnotes
   \renewcommand{\sstnotes}[1]{\item[Notes:]
      \begin{description}
         #1
      \end{description}
   }

%  sstdiytopic
   \renewcommand{\sstdiytopic}[2]{\item[{#1}]
      \begin{description}
         #2
      \end{description}
   }

%  sstimplementationstatus
   \renewcommand{\sstimplementationstatus}[1]{\item[Implementation Status:]
      \begin{description}
         #1
      \end{description}
   }

%  sstitemlist
   \newcommand{\sstitemlist}[1]{
      \begin{itemize}
         #1
      \end{itemize}
   }
\end{htmlonly}

%  End of "sst.tex" layout definitions.

% -----------------------------------------------------------------------------
%  Title Page.
%  ===========
\renewcommand{\thepage}{\roman{page}}
\begin{document}
\thispagestyle{empty}
%  Latex document header.
%  ======================
\begin{latexonly}
   CCLRC / {\sc Rutherford Appleton Laboratory} \hfill {\bf \stardocname}\\
   {\large Particle Physics \& Astronomy Research Council}\\
   {\large Starlink Project\\}
   {\large \stardoccategory\ \stardocnumber}
   \begin{flushright}
   \stardocauthors\\
   \stardocdate
   \end{flushright}
   \vspace{-4mm}
   \rule{\textwidth}{0.5mm}
   \vspace{5mm}
   \begin{center}
   {\Large\bf \stardoctitle}
   \end{center}
   \vspace{5mm}

%  Add heading for abstract if used.
%   \vspace{10mm}
%   \begin{center}
%      {\Large\bf Description}
%   \end{center}
\end{latexonly}

%  HTML documentation header.
%  ==========================
\begin{htmlonly}
   \xlabel{}
   \begin{rawhtml} <H1> \end{rawhtml}
      \stardoctitle
   \begin{rawhtml} </H1> \end{rawhtml}

%  Add picture here if required.

   \begin{rawhtml} <P> <I> \end{rawhtml}
   \stardoccategory \stardocnumber \\
   \stardocauthors \\
   \stardocdate
   \begin{rawhtml} </I> </P> <H3> \end{rawhtml}
      \htmladdnormallink{CCLRC}{http://www.cclrc.ac.uk} /
      \htmladdnormallink{Rutherford Appleton Laboratory}
                        {http://www.cclrc.ac.uk/ral} \\
      Particle Physics \& Astronomy Research Council \\
   \begin{rawhtml} </H3> <H2> \end{rawhtml}
      \htmladdnormallink{Starlink Project}{http://star-www.rl.ac.uk/}
   \begin{rawhtml} </H2> \end{rawhtml}
   \htmladdnormallink{\htmladdimg{source.gif} Retrieve hardcopy}
      {http://star-www.rl.ac.uk/cgi-bin/hcserver?\stardocsource}\\

%  HTML document table of contents.
%  ================================
%  Add table of contents header and a navigation button to return to this
%  point in the document (this should always go before the abstract \section).
  \label{stardoccontents}
  \begin{rawhtml}
    <HR>
    <H2>Contents</H2>
  \end{rawhtml}
  \renewcommand{\latexonlytoc}[0]{}
  \htmladdtonavigation{\htmlref{\htmladdimg{contents_motif.gif}}
        {stardoccontents}}

%  Start new section for abstract if used.
%  \section{\xlabel{abstract}Abstract}

\end{htmlonly}

% -----------------------------------------------------------------------------
%  Document Abstract. (if used)
%  ==================
% -----------------------------------------------------------------------------
%  Latex document Table of Contents (if used).
%  ===========================================
\begin{latexonly}
   \setlength{\parskip}{0mm}
   \latexonlytoc
   \setlength{\parskip}{\medskipamount}
   \markright{\stardocname}
\end{latexonly}
% -----------------------------------------------------------------------------

%%%%%%%%%%%%%%%%%%%%%%%%%%%%%%%%%%%%%%%%%%%%%%%%%%%%%%%%%%%%%%%%%%%%%%%%%%%
\newpage
\renewcommand{\thepage}{\arabic{page}}
\setcounter{page}{1}
\section{\xlabel{introduction}\label{se:introduction}Introduction }
\markboth{Introduction}{\stardocname}

This manual describes the commands and parameters used by IUEDR\@.
It is intended as a reference aid for people using IUEDR\@.

If you are new to IUE data reduction, you may like to read
\xref{{\sl IUE Analysis
a Tutorial}}{sg7}{} (SG/7) and the
\xref{{\sl IUEDR User Guide}}{mud45}{} (MUD/45) before proceeding
any further.
The Starlink User Note \xref{SUN/37}{sun37}{} contains a general description
of IUEDR which
overlaps with the early sections of this manual and also contains any notes on
the most recent release of the program.

Commands are described in Section~\ref{se:commands}, with parameters described
in more detail in Section~\ref{se:parameters}\@.
A list of default parameter behaviour and values is given in
Appendix~\ref{se:parameter_defaults}\@.
Details of transferring old-style IUEDR files from VMS systems to UNIX systems
and the conversion process are given in Appendix~\ref{se:vmsunix}\@.

\begin{latexonly}
An index of both commands and parameters is given in Appendix~\ref{se:index}\@.
\end{latexonly}

IUEDR functions fall into a number of specific categories:

\begin {itemize}
   \item IUE GO tape or file inspection and reading.
   \item Data display and manipulation.
   \item Spectrum extraction and calibration.
   \item Extraction product inspection and manipulation.
   \item Extraction product output.
   \item General operational commands.
\end {itemize}

These functions are controlled by over fifty commands, with nearly one hundred
global parameters within IUEDR\@.
There follows a summary of the commands available in each of the categories
listed above.

\subsection {IUE GO tape or file inspection and reading}

\comdescenv{
   \comdesc{LISTIUE}{Analyse the contents of one or more IUE tape files.}
   \comdesc{MTMOVE}{Move to the start of a tape file.}
   \comdesc{MTREW}{Rewind to the start of the tape.}
   \comdesc{MTSHOW}{Show the current tape position.}
   \comdesc{MTSKIPEOV}{Skip over the end-of-volume mark.}
   \comdesc{MTSKIPF}{Skip over NSKIP tape marks.}
   \comdesc{READIUE}{Read a RAW, GPHOT or PHOT IUE image from the tape/file.}
   \comdesc{READSIPS}{Read the MELO or MEHI IUESIPS product from the tape/file.}
}

\subsection {Data display and manipulation}

\comdescenv{
   \comdesc{CULIMITS}{Delineate the graphical display limits using the
                      graphics cursor.}
   \comdesc{CURSOR}{Determine display coordinates using the graphics cursor}
           \comdescc{and print them at the terminal.}
   \comdesc{DRIMAGE}{Display an IUE image on a suitable graphics workstation.}
   \comdesc{EDIMAGE}{Edit the image data quality using the graphics cursor.}
   \comdesc{MODIMAGE}{Modify image pixel intensities interactively.}
   \comdesc{CLEAN}{Mark as `bad' pixels with value below a given threshold.}
   \comdesc{SHOW}{Print information relating to the current dataset at the
                  terminal.}
   \comdesc{ERASE}{Erase the display screen of the current graphics
                   workstation.}
}

Image displays are colour coded to provide data quality information.
The colour codes used by IUEDR are as follows:

\begin{latexonly}
\begin {quote}
\begin {description}
   \item [Green] pixels affected by reseau marks
   \item [Red] pixels which are saturated (DN=255)
   \item [Orange] pixels affected by ITF truncation
   \item [Yellow] pixels marked bad by the user
\end {description}
\end {quote}
\end{latexonly}

\begin{htmlonly}
\begin{rawhtml}
<PRE>
   <B>Green</B>  - pixels affected by reseau marks
   <B>Red</B>    - pixels which are saturated (DN=255)
   <B>Orange</B> - pixels affected by ITF truncation
   <B>Yellow</B> - pixels marked bad by the user
</PRE>
\end{rawhtml}
\end{htmlonly}

The colour {\bf Blue} is used to indicate a pixel which has a value above the
maximum that can be displayed using the linear greyscale image display colour
look-up table.

When using a mouse or tracker-ball with the graphics cursor, the cursor hit
buttons are normally numbered in increasing order from left to right.
For example the left mouse button corresponds to cursor key hit 1, middle
button to cursor key hit 2 and so on.
Many terminals allow left-handed users to reverse the mouse button order.

\subsection {Spectrum extraction and calibration}

\comdescenv{
   \comdesc{AESHIFT}{Determine (HIRES) spectrum ESHIFT automatically.}
   \comdesc{AGSHIFT}{Determine spectrum template shift automatically.}
   \comdesc{BARKER}{Correct the extracted data for \'{e}chelle ripple}
           \comdescc{using a method based upon that of Barker (1984).}
   \comdesc{CGSHIFT}{Determine spectrum template shift using the cursor
                     on a SCAN plot.}
   \comdesc{LBLS}{Extract a line-by-line-spectrum array from the image.}
   \comdesc{NEWABS}{Associate a new absolute flux calibration with the
                  current  dataset.}
   \comdesc{NEWCUT}{Associate new \'{e}chelle order wavelength limits with the
                  current  dataset.}
   \comdesc{NEWDISP}{Associate new spectrograph dispersion data with the
                   current  dataset.}
   \comdesc{NEWFID}{Associate new fiducial positions with the current
                  dataset.}
   \comdesc{NEWRIP}{Associate new ripple calibration data with the current
                  dataset.}
   \comdesc{NEWTEM}{Associate new spectrum centroid template data with the
                    current dataset.}
   \comdesc{SCAN}{Perform a scan of the image data perpendicular}
           \comdescc{to the spectrograph dispersion.}
   \comdesc{SETA}{Set dataset parameters which are aperture specific.}
   \comdesc{SETD}{Set dataset parameters which are independent of order and
                aperture.}
   \comdesc{SETM}{Set dataset parameters which are order specific.}
   \comdesc{TRAK}{Extract a spectrum from the image.}
}

\subsection {Extraction product inspection and manipulation}

\comdescenv{
   \comdesc{EDMEAN}{Edit the mean extracted spectrum using the graphics
                  cursor.}
   \comdesc{EDSPEC}{Edit the net extracted spectrum using the graphics
                  cursor.}
   \comdesc{MAP}{Map and merge extracted spectrum components to produce}
           \comdescc{a mean spectrum.}
   \comdesc{PLCEN}{Plot the smoothed spectrum centroid shifts.}
   \comdesc{PLFLUX}{Plot the calibrated flux spectrum.}
   \comdesc{PLGRS}{Plot the pseudo-gross and background resulting from the}
           \comdescc{spectrum extraction.}
   \comdesc{PLMEAN}{Plot the mean spectrum.}
   \comdesc{PLNET}{Plot the uncalibrated net spectrum.}
   \comdesc{PLSCAN}{Plot the image scan perpendicular to the dispersion.}
   \comdesc{SGS}{Print the names of the available SGS graphics devices at
               the  terminal.}
}

Plots of extracted IUE spectra and image scans include data quality information
flags for bad data.
The data quality codes used by IUEDR are as follows:

\begin{latexonly}
\begin {quote}
\begin {description}
   \item [1] affected by extrapolated ITF
   \item [2] affected by microphonics
   \item [3] affected by noise spike
   \item [4] affected by bright point (or user)
   \item [5] affected by reseau mark
   \item [6] affected by ITF truncation
   \item [7] affected by saturation
   \item [U] affected by user edit
\end {description}
\end {quote}
\end{latexonly}

\begin{htmlonly}
\begin{rawhtml}
<PRE>
   <B>1</B> - affected by extrapolated ITF
   <B>2</B> - affected by microphonics
   <B>3</B> - affected by noise spike
   <B>4</B> - affected by bright point (or user)
   <B>5</B> - affected by reseau mark
   <B>6</B> - affected by ITF truncation
   <B>7</B> - affected by saturation
   <B>U</B> - affected by user edit
</PRE>
\end{rawhtml}
\end{htmlonly}

\subsection {Extraction product output}

\comdescenv{
   \comdesc{OUTEM}{Output the current spectrum template data to a formatted
                 data  file.}
   \comdesc{OUTLBLS}{Output the current LBLS array to a binary data file.}
   \comdesc{OUTMEAN}{Output the current mean spectrum to a DIPSO SP
                   format  data file.}
   \comdesc{OUTNET}{Output the current net spectrum to a DIPSO SP
                  format data file.}
   \comdesc{OUTRAK}{Output the current uncalibrated spectrum to a}
           \comdescc{``TRAK'' formatted data file.}
   \comdesc{OUTSCAN}{Output the current scan data to a DIPSO SP format
                   data  file.}
   \comdesc{OUTSPEC}{Output the current aperture (LORES) or order (HIRES)}
           \comdescc{spectrum to a DIPSO SP format data file.}
   \comdesc{PRGRS}{Print the current extracted aperture or order spectrum
                 in tabular form.}
   \comdesc{PRLBLS}{Print the current LBLS array in tabular form.}
   \comdesc{PRMEAN}{Print the current mean spectrum in tabular form.}
   \comdesc{PRSCAN}{Print the intensities of the current image scan in
                  tabular  form.}
   \comdesc{PRSPEC}{Print the current aperture or order spectrum in tabular
                  form.}
}

\subsection {General operational commands}

\comdescenv{
   \comdesc{EXIT}{Leave IUEDR and update any files altered during the}
           \comdescc{current IUEDR session.}
   \comdesc{QUIT}{Leave IUEDR and update any files altered during the}
           \comdescc{current IUEDR session.}
   \comdesc{SAVE}{Overwrite any files that have had their contents}
           \comdescc{updated during the current IUEDR session.}
}

A log of all commands typed during an IUEDR session and program output at the
terminal can be found in the file {\tt session.lis}.
This is particularly useful when investigating the contents of IUE tapes.

\newpage
\section{\xlabel{user_interface}User interface}
\markboth{User interface}{\stardocname}

The IUEDR user interface uses the Starlink ADAM parameter system.
The interface will be familiar to users of the VMS IUEDR, with a few changes
to reach a level of consistency with other Starlink packages.

\subsection {Starting IUEDR}

To initialise for IUEDR type
\begin{verbatim}
   % iuedr
\end{verbatim}

at the shell prompt \verb+%+.
The first time you type the command, IUEDR environment variables are set up in
your session.
You can now start the program by typing
\begin{verbatim}
   % iuedr
\end{verbatim}
again.
Edit your \verb+.login+ file if you want to avoid having to type the command
that extra time.

The command line interface prompt is \verb+>+\@.
This should appear after a welcome message and you can then type commands as
you would in a typical command shell.

\subsection {Response to command prompts}

Instructions to IUEDR are given as command lines.

Command lines begin with a command and an optional list of parameter
assignments. For example:
\begin{verbatim}
   > DRIMAGE DATASET=SWP14931 DEVICE=xw
\end{verbatim}

Usually IUEDR will only prompt for parameters required by commands if
they have no currently defined value.
However, some parameters are either cancelled during the execution of a
command or are set so that the user is always prompted for a value.
A command can be forced to prompt for all required parameter values thus:

\begin{verbatim}
   > READIUE PROMPT
\end{verbatim}

The \verb+PROMPT+ may be abbreviated to \verb+PR+\@.

In a similar way commands which always prompt for parameter values can be made
to accept default values thus:

\begin{verbatim}
   > READIUE ACCEPT
\end{verbatim}

Command input and output printed at the terminal is also copied to the
file \verb+session.lis+ in the working directory.
Note that this file is rewritten each time IUEDR is run.

\subsection {Response to parameter prompts}

Help about a parameter can be obtained by responding to the parameter prompt
with a question mark, {\it{e.g.,}}

\begin{verbatim}
   DATASET - Dataset Name. > ?
\end{verbatim}

Help information will then be printed at the terminal and the prompt repeated.

Sometimes an undefined parameter value is interpreted by a command in
a specific way ({\it{e.g.,}}\ auto-scaling within plotting commands)\@.
A parameter can be set undefined by responding to the prompt with an
exclamation mark, {\it{e.g.,}}

\begin{verbatim}
   XL - X-axis plotting limits, [0,0] means auto-scale. /[1150,1950]/ > !
\end{verbatim}

A command may be aborted by responding to a parameter prompt with a
double exclamation mark, {\it{e.g.,}}

\begin{verbatim}
   XL - X-axis plotting limits, [0,0] means auto-scale. /[1150,1950]/ > !!
\end{verbatim}

To prevent a command from prompting for further parameter values (once in
prompt mode) type a backslash, {\it{e.g.,}}

\begin{verbatim}
   XL - X-axis plotting limits, [0,0] means auto-scale. /[1150,1950]/ > \
\end{verbatim}

The command will then only prompt for parameters for which it cannot generate
a suitable value.

\subsection {Getting HELP}

\begin{latexonly}
Type HELP at the IUEDR command line prompt.
You may optionally append a detailed description of the topic on which help
is required. See page~\pageref{com: HELP } for further details.
\end{latexonly}

\begin{htmlonly}
Type HELP at the IUEDR command line prompt.
You may optionally append a detailed description of the topic on which help
is required. See HELP for further details.
\end{htmlonly}

\subsection {IUEDR in script and batch modes}

It is possible to run IUEDR in script mode, where command and
parameter input originates from a file instead of the terminal.
The file from which the command input is to be taken is piped into IUEDR:

\begin{verbatim}
   % iuedr < script_file
\end{verbatim}

This will result in the command input being taken from the file
\verb+script_file+ and the text output being written to the \verb+session.lis+
file.

Alternatively the output can be directed to a specific file:

\begin{verbatim}
   % iuedr < script_file > script_log
\end{verbatim}

In this case \verb+script_log+ contains the output log \verb+session.lis+ is
{\bf not} over-written.

\newpage
\section{\xlabel{IUEDR_data_files}IUEDR data files}
\markboth{IUEDR data files}{\stardocname}

Once an IUE GO format file has been read by IUEDR two files are created.  One
is a text file containing information about the data set calibration data.
This file has a name constructed
\begin{verbatim}
   <dataset>.UEC
\end{verbatim}
where \verb+<dataset>+ corresponds to the value of the IUEDR \verb+DATASET+
parameter.

when a GPHOT, PHOT or RAW image is read the data produced are stored in an
Image and Data Quality file
\begin{verbatim}
   <dataset>_UED.sdf
\end{verbatim}
After the spectrum extraction process the uncalibrated spectral data are stored
in a file
\begin{verbatim}
   <dataset>_UES.sdf
\end{verbatim}
This file is also produced when an IUESIPS MELO or MEHI file is read with
\verb+READSIPS+\@.  No \verb+_UED.sdf+ being created in this case.

Calibrated spectra produced by the \verb+MAP+ command are stored in a file
\begin{verbatim}
   <dataset>_UEM.sdf
\end{verbatim}

All these \verb+.sdf+ files are Starlink NDF format files (See
\xref{SUN/33}{sun33}{}
for details of access to NDFs) which can be read by any of the standard
packages (KAPPA, FIGARO etc.).  The contents of these files can be examined
outside of IUEDR using the \verb+hdstrace+ command.

These files are in addition to the IUEDR log file \verb+session.lis+ and the
files generated by output commands.  They should {\bf not} be deleted until the
data reduction is complete and the output spectra obtained.

The \verb+OUT*+ family of IUEDR output commands also generate NDFs which can
be read by programs such as DIPSO.

In summary:
\begin {description}
   \item \verb+<dataset>.UEC+ --- calibration file.
   \item \verb+<dataset>_UED.SDF+ --- image data and quality file.
   \item \verb+<dataset>_UES.SDF+ --- uncalibrated spectrum file.
   \item \verb+<dataset>_UEM.SDF+ --- calibrated mean spectrum file.
\end {description}

{\bf Refer to Appendix~\ref{se:vmsunix} for VMS to UNIX file conversion.}

\subsection{\label{subap:ndf}NDF Components in IUEDR files}

This section gives a summary of the NDF components present in IUEDR files for
those who may wish to access the files from their own programs.
The structure and content of an NDF can be inspected using the {\tt hdstrace}
utility (See \xref{SUN/102}{sun102}{})\@.  Values for components have been
given where they are constant for all files of the particular type.

\subsubsection{Image data and quality file {\tt \_UED}}

\begin{latexonly}
A simple NDF, each point in the $768\times 768$ image is described by a datum
and a quality flag.  The image is given the generic title `IUE image'\@.
\end{latexonly}

\begin{htmlonly}
A simple NDF, each point in the 768x768 image is described by a datum
and a quality flag.  The image is given the generic title `IUE image'\@.
\end{htmlonly}

\begin{verbatim}
IUEDR  <NDF>

   DATA_ARRAY(768,768)  <_WORD>

   QUALITY        <QUALITY>       {structure}
      QUALITY(768,768)  <_UBYTE>

   TITLE          <_CHAR*9>       'IUE image'
\end{verbatim}

\subsubsection{Uncalibrated spectrum file {\tt \_UES}}

This NDF contains IUEDR specific extensions, which are written and read by the
program when processing spectra.  In the description below {\tt no} is
the number of orders processed with the \verb+TRAK+ command.  {\tt mo} is
the number of data points in the longest order processed.
{\tt WAVES} contains the wavelengths of each of the flux data in each order.
{\tt ORDERS} holds a list of the order numbers processed.
{\tt NWAVS} stores the actual number of points in each of the {\tt no} orders.

\begin{verbatim}
IUEDR  <NDF>

   DATA_ARRAY(mo,no)   <_REAL>

   QUALITY        <QUALITY>       {structure}
      QUALITY(mo,no)   <_UBYTE>

   MORE           <EXT>           {structure}
      IUEDR_EXTRA    <EXTENSION>     {structure}
         WAVES(mo,no)   <_REAL>
         ORDERS(no)     <_INTEGER>
         NWAVS(no)      <_INTEGER>

   TITLE          <_CHAR*80>
   LABEL          <_CHAR*4>       'Flux'
\end{verbatim}

\subsubsection{Calibrated mean spectrum file {\tt \_UEM}}

The mean spectrum is basically an array of flux values against a wavelength
scale.  Quality information is included in the data file.  The axes units are
also included.
This NDF type is again IUEDR specific, the unusual components are as follows.
{\tt WAVES} are wavelengths of flux data points.
{\tt WEIGHTS} are weights applied to the flux.
{\tt XCOMB1} is the start wavelength.
{\tt DXCOMB} is the wavelength step from point to point.



\begin{verbatim}
IUEDR  <NDF>

   DATA_ARRAY(17001)  <_REAL>

   QUALITY        <QUALITY>       {structure}
      QUALITY(17001)  <_UBYTE>

   MORE           <EXT>           {structure}
      IUEDR_EXTRA    <EXTENSION>     {structure}
         WAVES(17001)   <_REAL>
         WEIGHTS(17001)  <_REAL>
         XCOMB1         <_DOUBLE>
         DXCOMB         <_DOUBLE>

   AXIS(1)        <AXIS>          {structure}
      DATA_ARRAY(17001)  <_REAL>
      UNITS          <_CHAR*40>      '(A)'
      LABEL          <_CHAR*40>      'Wavelength'

   TITLE          <_CHAR*80>
   UNITS          <_CHAR*40>      '(FN/s)'
   LABEL          <_CHAR*40>      'Flux'
\end{verbatim}

\subsubsection{\label{se:spectrum}SPECTRUM format output files}

SPECTRUM is a data analysis programme written by Steve Adams at UCL\@.
Although the programme is no longer used (I guess it might be in use
somewhere\ldots) the file formats it introduced were adopted by the popular
spectrum analysis programme DIPSO (described in \xref{SUN/50}{sun50}{})\@.

The basic input to a SPECTRUM file is a single spectrum (wavelength, flux)\@.
The wavelengths should be in increasing order, and evenly spaced.

There are three variants of the SPECTRUM file format.  Using its terminology:

\begin{latexonly}
\begin{tabular}{ll}
Format number & File characteristics\\
0             & Unformatted (Binary)\\
1             & Fixed Format Text\\
2             & Free-field Format Text\\
\end{tabular}
\end{latexonly}

\begin{htmlonly}
\begin{rawhtml}
<PRE>
<B>Format number     File characteristics</B>
      0           Unformatted (Binary)
      1           Fixed Format Text
      2           Free-field Format Text
</PRE>
\end{rawhtml}
\end{htmlonly}

IUEDR {\bf no longer} produces output of the SP0 type.  Instead NDFs are used.
In practice this is invisible to the user as the DIPSO SP0RD command (read
SPECTRUM format 0 file) now reads NDFs!  The other two formats are still
available.  A Description of the old SP0 format is included here in case
anyone needs to read an existing file in this format (DIPSO can still read
SP0 format via the SP0RD command)\@.

Using a FORTRAN77 notation, the contents of a SPECTRUM file can be
expressed as:

\begin{verbatim}
   PARAMETER(MAXWAV=8000) ! maximum number of wavelengths
   CHARACTER*79 CLINE1    ! first line of text
   CHARACTER*79 CLINE2    ! second line of text
   INTEGER NWAV           ! number of wavelengths
   REAL WAV(MAXWAV)       ! wavelengths
   REAL FLUX(MAXWAV)      ! fluxes
\end{verbatim}

Both \verb+CLINE1+ and \verb+CLINE2+ are totally unstructured text strings,
and are used to describe the spectrum.  The convention is that
\verb+FLUX(I)=0.0+ when its value is undefined.

Here, briefly, is the code needed to read the SPECTRUM formats:

Format number 0 is an unformatted (binary) file read by:

\begin{verbatim}
   OPEN(UNIT=1, ACCESS='SEQUENTIAL', FORM='UNFORMATTED')
   READ(1) CLINE1(1:79)
   READ(1) CLINE2(1:79)
   READ(1) NWAV
   READ(1) (WAV(I),FLUX(I),I=1,NWAV)
   CLOSE(UNIT=1)
\end{verbatim}

Format number 1 is a fixed format text file read by:

\begin{verbatim}
   OPEN(UNIT=1, ACCESS='SEQUENTIAL')
   READ(1,'(A79)') CLINE1(1:79)
   READ(1,'(A79)') CLINE2(1:79)
   READ(1,'(20X,I6)') NWAV
   READ(1,'(4(F8.3,E10.3))') (WAV(I),FLUX(I),I=1,NWAV)
   CLOSE(UNIT=1)
\end{verbatim}

Format number 2 is a free-field text file read by:

\begin{verbatim}
   OPEN(UNIT=1, ACCESS='SEQUENTIAL')
   READ(1,'(A79)') CLINE1(1:79)
   READ(1,'(A79)') CLINE2(1:79)
   READ(1,*) NWAV
   READ(1,*) (WAV(I),FLUX(I),I=1,NWAV)
   CLOSE(UNIT=1)
\end{verbatim}

The sections of FORTRAN 77 code shown above are not intended to be serious
attempts to write a SPECTRUM file reading programme.  Instead they are designed
to define the contents as succinctly as possible.

\newpage
\section{\xlabel{porting_changes}Changes during the port to UNIX}
\markboth{Changes during the port to UNIX}{\stardocname}

This section describes the main changes that have been made to IUEDR
during its conversion to an ADAM based application which runs on
all Starlink supported platforms.  \xref{SUN/37}{sun37}{} gives notes on the
very latest version of IUEDR.

If you are a seasoned IUEDR user then you should study this section
especially carefully.

The most significant change from the scientific point of view is that
the precision of all floating point calculations has been upgraded to
DOUBLE PRECISION. This was done after it was noticed that for high
resolution extraction the output spectra were subject to rounding
noise at the 1\% level.

The format of the calibration file ({\tt .UEC}) created by IUEDR has
been  changed to make it more readable. A VMS program to convert
IUEDR datasets to the new format is available, see Appendix~\ref{se:vmsunix}
for details.

The functionality of the package has been enhanced to allow image data
to be read directly from disk.

The general operation of IUEDR, and all the command and parameter
names, are identical to those used in previous versions.

\subsection{IUEDR command files}

The \verb+.CMD+ style of VMS IUEDR command files is not directly supported by
UNIX IUEDR, and neither is the associated input/output redirection using
\verb+<+ and \verb+>+\@.

It is very easy to convert a {\tt .CMD} file into a UNIX IUEDR command
script.
{\it{e.g.,}}\ a {\tt DEMO.CMD} procedure:

\begin{verbatim}
   DATASET=SWP03196
   SHOW
   SCAN ORDERS=(125,66)
   TRAK APERTURE=LAP
   SHOW V=S
\end{verbatim}

would become a UNIX IUEDR command script \verb+demo.cmd+, thus:

\begin{verbatim}
   SHOW DATASET=SWP03196
   SCAN ORDERS=[125,66]
   TRAK APERTURE=LAP
   SHOW V=S
\end{verbatim}

The only changes which need be made are to move any parameter
specifications ({\it{e.g.,}}\ \verb+DATASET=+) onto the same line as the command
they apply to, and to change vector parameter specifications to use square
\verb+[]+ brackets instead of the old-style round \verb+()+ brackets.

This script can then be read into IUEDR by
\begin{verbatim}
   % iuedr < demo.cmd
\end{verbatim}

\subsection{Interaction with DIPSO}

As part of the port to UNIX the
format of the default DIPSO spectrum format SP0 files has been
changed to use the STARLINK NDF data format. This means that these
files can be read by any standard STARLINK package.

The IUEDR/DIPSO user should notice no difference, as both IUEDR and
DIPSO understand the new format.

\subsection{DRIVE parameter options}

The use of the DRIVE parameter has been enhanced to allow
specification of disk files containing IUE datasets. This is intended
for use with  files obtained from online archives (RAL and NASA).

The syntax is to provide the full filename and extension in response to
the DRIVE prompt:

\begin{verbatim}
   DRIVE> SWP12345.RAW
\end{verbatim}

\subsection{Specifying vector parameters}

Some IUEDR parameters ({\it{e.g.,}}\ \verb+XP+, \verb+YP+) require the
specification of a pair of numbers defining the limits of a range of values
({\it{e.g.,}}\ pixels).

The method of setting such values has changed to the ADAM style:

\begin{verbatim}
   XP=[100,300]
\end{verbatim}

Note that the square brackets are only necessary when vector
parameters are specified on the command line. They are not required
when IUEDR prompts the user for a vector parameter.

\subsection{Calibration files and the NEW* family of commands}

The missing Calibration file for SWP camera HIRES data has been added to the
IUEDR package.

The action of the \verb+NEW*+ family of commands for updating IUEDR
calibrations, geometry and so on have been altered  to improve functionality.
Users will find that the previous style of file name entry still works,
and that the following features have been added:
\begin{itemize}
   \item Both Logical Name and Environment Variable style file specifications
   may be given as parameter values, for example
   \begin{verbatim}
      > NEWABS ABSFILE=$IUEDR_DATA/swphi
   \end{verbatim}
   and
   \begin{verbatim}
      > NEWABS ABSFILE=IUEDR_DATA:swphi
   \end{verbatim}
   are equivalent and allowed on all platforms.
   \item The default file name extension need not be given, however if the
   file to be read has a different extension this {\bf should} be given.
   For example,
   \begin{verbatim}
      > NEWABS ABSFILE=$IUEDR_DATA/swphi
   \end{verbatim}
   and
   \begin{verbatim}
      > NEWABS ABSFILE=$IUEDR_DATA/swphi.abs
   \end{verbatim}
   are {\bf both} valid.

   This behaviour is a change to the VMS-only IUEDR where the extension had
   to be omitted and the default value for the appropriate command was always
   taken.
   \item For case-sensitive file systems (like UNIX) if the Enviroment Variable
   \verb+$IUEDR_DATA+ is used as part of the file specification then the case
   of the file name itself is always converted to {\bf lower case}.  All the
   files available in the \verb+$IUEDR_DATA+ directory have lower case names
   so this is not a problem, rather it allows default calibration file names
   to be upper- or lower-case in IUEDR command scripts.
\end{itemize}

\subsection{Documentation}

The excellent introduction to IUEDR
\xref{{\sl IUE Analysis---A Tutorial}}{sg7}{}
(SG/7) by Richard Tweedy has been updated for UNIX and included as a standard
part of IUEDR\@.  Some special calibration corrections described in this
document have also been added to the IUEDR package.

\newpage
\section{\xlabel{commands}\label{se:commands}Commands}
\markboth{Commands}{\stardocname}

This section contains a detailed description of each of the commands
available in IUEDR.  A list of the parameters used by each command is given,
along with a brief description of each.  The pages on which you will find
full parameter descriptions are given at the end of each line in the parameter
list.

\sstroutine{AESHIFT}
{
   Determine (HIRES) spectrum ESHIFT automatically.
}{
   \sstparameters{
   \cpar{DATASET}{Dataset name.}
   \cpar{CENTREWAVE}{Line central wavelengths (A).}
   \cpar{DELTAWAVE}{Half-width of line search windows (A).}
}
\sstdescription{
   \verb+AESHIFT+ can be used to measure the global \'{e}chelle shift for a
   HIRES
   spectrum.  A set of laboratory wavelengths of absorption features which
   should be present in the spectrum are located in the spectrum and the
   \verb+ESHIFT+ for each is calculated.  The median of these \verb+ESHIFT+s
   is then applied to the whole dataset.
}
}

\sstroutine{AGSHIFT}
{
   Determine spectrum shift for HIRES automatically.
}{
   \sstparameters{
   \cpar{DATASET}{Dataset name.}
   \cpar{ORDERS}{This delineates a range of \'{e}chelle orders.}
}
\sstdescription{
   \verb+AGSHIFT+ can be used to measure the global geometric shift for a HIRES
   spectrum.  A scan of the \verb+DATASET+ must be made available using the
   \verb+SCAN+ command.  The scan data is traversed starting at the lowest
   numbered order and a probable site for the peak of each order is found.
   The central position of each order is then estimated using a centroiding
   algorithm.  An estimate of the geometric shift is made for each order and
   these are recorded and displayed.

   A weighted mean in which the shifts determined for orders 100 to 110
   inclusive are given greater weight than other shifts is calculated.
   The individual order shifts are compared to the mean and any shift
   greater than 3 pixels from the mean position is rejected and a new value
   for the mean shift calculated from the remaining orders.
   The mean shift is displayed and the \verb+GSHIFT+ parameter is set.

   Using \verb+AGSHIFT+ it is possible to automate the spectrum extraction
   process.  It should be noted that objects with no continuum may break
   the \verb+AGSHIFT+ mechanism, giving poor shift values, in these cases
   the interactive \verb+CGSHIFT+ command should be used.
}
}

\sstroutine{BARKER}
{
   Correct spectrum data for \'{e}chelle ripple using a method based
   upon that of Barker (1984).
}{
   \sstparameters{
   \cpar{DATASET}{Dataset name.}
   \cpar{ORDERS}{This delineates a range of \'{e}chelle orders.}
}
\sstdescription{
   The spectrum data in \verb+DATASET+ are corrected for residual \'{e}chelle
   ripple using  the method described by Barker (1984. Astronomical Journal,
   \underline{89},  899). Orders in the range \verb+ORDERS+ are used in the
   ripple correction optimisation.  Note that this optimisation method is
   only applicable for SWP spectra.
}
}


\sstroutine{CGSHIFT}
{
   Determine spectrum template shift using the cursor on a scan plot.
}{
   \sstparameters{
   \cpar{DATASET}{Dataset name.}
   \cpar{APERTURE}{Aperture name (\verb+SAP+ or \verb+LAP+).}
   \cpar{ORDERS}{This delineates a range of \'{e}chelle orders.}
   \cpar{DEVICE}{GKS/SGS graphics device name.}
}
\sstdescription{
   This command allows the graphics cursor to be used to provide
   information about spectrum template registration shifts.

   A plot of the current spectrum scan must be available on the graphics
   \verb+DEVICE+\@.

   A cycle consisting of any  number of left or middle mouse button hits is
   used to mark the position of the spectrum. Each hit is used to calculate
   a linear geometric shift of the spectrum template relative to the image.
   The cycle is terminated by pressing the right mouse button.

   Keyboard keys 1, 2 and 3 can be used  in place of left, middle and right
   mouse buttons respectively.

   For LORES, when the cycle is complete the last geometric shift
   determined is adopted and the scan is revoked.

   For HIRES, each cursor hit is automatically associated with an \'{e}chelle
   order in the range defined by the \verb+ORDERS+ parameter. The last shift is
   again adopted, but the scan is available for further display or
   measurement.
}
}

\newpage
\sstroutine{CLEAN}
{
   Mark pixels with values below a selected threshold as BAD.
}{
   \sstparameters{
   \cpar{DATASET}{Dataset name.}
   \cpar{THRESH}{Smallest pixel value to be accepted as GOOD.}
}
\sstdescription{
   Some IUE  datasets are effected  by horizontal  bars of low pixel values.
   These are caused by a weak  signal from the IUE craft at the time of data
   download.  When there are  a few bars  they can be marked as bad with the
   \verb+EDIMAGE+ command.  In the case of many bars a quicker solution is to
   mark all pixels in  the image below  a user selected  threshold value as BAD.

   Successive \verb+CLEAN+ and \verb+DRIMAGE+ commands starting with a value of
   \verb+THRESH=-1000+ and increasing \verb+THRESH+ towards zero will  allow
   the user to chose a value suitable for the problem image.
}
}

\sstroutine{CULIMITS}
{
   Set display limits with the cursor.
}{
   \sstparameters{
   \cpar{DEVICE}{GKS/SGS graphics device name.}
   \cpar{XL}{$x$-axis plotting limits, undefined or [0, 0] means auto-scale.}
   \cpar{YL}{$y$-axis plotting limits, undefined or [0, 0] means auto-scale.}
   \cpar{XP}{$x$-axis pixel limits, undefined or [0, 0] means full extent.}
   \cpar{YP}{$y$-axis pixel limits, undefined or [0, 0] means full extent.}
}
\sstdescription{
   This command uses the cursor to delineate part of a current display,
   graph or image, to be displayed in some subsequent command
   ({\it{e.g.,}}\ \verb+PLFLUX+, \verb+DRIMAGE+\ldots ).

   The two cursor positions should be at the corners of the required
   rectangular subset. The relation between cursor position sequences and
   axis reversals for graphs is:

   \begin{tabular}{llll}
   {\bf Position 1} & {\bf Position 2} & {\bf x-reversed} & {\bf y-reversed}\\
   bottom/left  & top/right    & NO  & NO \\
   bottom/right & top/left     & YES & NO \\
   top/left     & bottom/right & NO  & YES \\
   top/right    & bottom/left  & YES & YES
   \end{tabular}

   The \verb+XL+ and \verb+YL+ values are changed accordingly.

   In the case of an image display, the \verb+XP+ and \verb+YP+ parameter values
   are changed.
   The image will {\bf always} be drawn without axis reversals.
}
}

\sstroutine{CURSOR}
{
   Find display coordinates using the cursor and print them at the terminal.
}{
   \sstparameters{
   \npar{None.}
}
\sstdescription{
   This command uses the graphics cursor to find coordinates on a
   displayed graph or image.

   Pressing the left or middle  mouse button displays information about the
   pixel being  pointed to.  Pressing the right mouse button terminates the
   \verb+CURSOR+ cycle.  The coordinates for each hit are printed on the
   terminal; they correspond to the unit scale of the axes prevailing on the
   current diagram,  ({\it{e.g.,}}\ (wavelength,  flux)).
   If meaningful,  additional coordinate information is also printed.
   Keyboard keys 1, 2 and 3 may be used in place of left, middle and right
   mouse buttons respectively.
}
}

\sstroutine{DRIMAGE}
{
   Display an IUE image on an suitable graphics workstation.
}{
   \sstparameters{
   \cpar{DATASET}{Dataset name.}
   \cpar{DEVICE}{GKS/SGS graphics device name.}
   \cpar{XP}{$x$-axis pixel limits, undefined or [0, 0] means full extent.}
   \cpar{YP}{$y$-axis pixel limits, undefined or [0, 0] means full extent.}
   \cpar{ZL}{Data limits for image display, undefined means full range.}
   \cpar{COLOUR}{Whether a false colour look-up table is used.}
   \cpar{ZONE}{Zone to be used for plotting.}
   \cpar{FLAG}{Whether data quality for faulty pixels are displayed.}
}
\sstdescription{
   This command displays the image specified by the \verb+DATASET+
   parameter on the device specified by the \verb+DEVICE+ parameter.

   The part of the image displayed is specified by the \verb+XP+ and \verb+YP+
   parameter values.
   If unspecified, \verb+XP+ and \verb+YP+ default to the entire image extent,
   {\it{i.e.}}

   \begin {quote}
      \verb+XP = [1,768], YP = [1,768]+
   \end {quote}

   If the values of \verb+XP+ or \verb+YP+ are specified in decreasing order,
   the image will {\bf not} be reversed along the appropriate axis.

   The range of data values displayed as a grey scale is limited
   by the two values of the \verb+ZL+ parameter.
   Data values at or below \verb+ZL[1]+ will appear {\bf black},
   those at \verb+ZL[2]+ will appear {\bf white} and those above \verb+ZL[2]+
   will appear {\bf blue}.
   If the \verb+ZL+ values are given in decreasing order, then high data
   values will be represented by low (dark) display intensities,
   and vice-versa.
   If the values are undefined, then the full intensity range of the
   image will be used.
   The full intensity range of the image can be found using the command

   \begin {quote}
      \verb+> SHOW V=I+
   \end {quote}

   The \verb+FLAG+ parameter specifies whether faulty pixels are flagged using
   the following colour scheme:

   \begin{description}
      \item GREEN --- pixels affected by reseau marks
      \item RED --- pixels which are saturated (DN=255)
      \item ORANGE --- pixels affected by ITF truncation
      \item YELLOW --- pixels marked bad by the user
   \end{description}

   If a pixel is affected by more than one of the above faults, then
   the first in the list is adopted for display.

   The \verb+ZONE+ parameter is accepted by \verb+DRIMAGE+ but is ignored, the
   display always using \verb+ZONE=0+\@.
}
}

\sstroutine{EDIMAGE}
{
   Edit the image data quality using the graphics cursor.
}{
   \sstparameters{
   \cpar{DATASET}{Dataset name.}
   \cpar{DEVICE}{GKS/SGS graphics device name.}
}
\sstdescription{
   This command uses the image display cursor to mark pixels and
   regions of the current image that are ``bad'' or ``good''.
   The image should have previously been displayed using the
   \verb+DRIMAGE+ command.
   So that faulty pixels can be seen, the \verb+FLAG=TRUE+ option in
   \verb+DRIMAGE+ should be used.

   The image display is specified by the \verb+DEVICE+ parameter and the
   associated dataset by the \verb+DATASET+ parameter.

   The following cursor hit sequences can be used in a cycle:

   \begin {description}
      \item 1 then 1 --- marks all pixels in the rectangle GOOD.
      \item 2 then 2 --- marks all points in the rectangle BAD.
      \item 1 --- marks the nearest pixel GOOD.
      \item 2 --- marks the nearest pixel BAD.
      \item 3 --- causes the cursor cycle to terminate.
   \end {description}

   Mouse buttons can be used for cursor hits where:

   \begin {description}
      \item {\bf left} mouse button     is  hit 1.
      \item {\bf middle} mouse button   is  hit 2.
      \item {\bf right} mouse button    is  hit 3.
   \end{description}

   Alternatively, keyboard keys 1, 2 and 3 can be used to mark hits.

   The pixels or ranges changed are printed on the terminal.
   The term ``rectangle'' is used above to indicate a rectangular
   set of pixels delineated by the two cursor positions.
   Thus, for the first hit, the cursor can be positioned at the
   bottom left corner, and for the second at the top right corner.

   Only the user-defined data quality bit can be changed by this
   command.
   Initially, all faulty pixels have this bit set BAD, so that
   spectrum extraction (say) can ignore these where appropriate.
   However, the user-defined data quality can also be set GOOD.

   See the IUEDR User Guide (MUD/45) for further information on data quality.
}
}

\sstroutine{EDMEAN}
{
   Edit the mean extracted spectrum using the graphics cursor.
}{
   \sstparameters{
   \cpar{DATASET}{Dataset name.}
   \cpar{DEVICE}{GKS/SGS graphics device name.}
}
\sstdescription{
   This command uses the graphics cursor to mark points and
   regions of the mean spectrum that are ``bad'' or ``good''.

   The following cursor hit sequences can be used in a cycle:

   \begin {description}
      \item 1 then 1 --- marks all points in the $x$-range GOOD.
      \item 2 then 2 --- marks all points in the $x$-range BAD.
      \item 1 --- marks the nearest point in $x$-direction GOOD.
      \item 2 --- marks the nearest point in $x$-direction BAD.
      \item 3 --- causes the cursor cycle to terminate.
   \end {description}

   The points or ranges changed are printed on the terminal.

   Mouse buttons can be used for cursor hits where:

   \begin {description}
      \item {\bf left} mouse button     is  hit 1.
      \item {\bf middle} mouse button   is  hit 2.
      \item {\bf right} mouse button    is  hit 3.
   \end{description}

   Alternatively, keyboard keys 1, 2 and 3 can be used to mark hits.

   See the IUEDR User Guide (MUD/45) for further information on data quality.
}
}

\sstroutine{EDSPEC}
{
   Edit the net extracted spectrum using the graphics cursor.
}{
   \sstparameters{
   \cpar{DATASET}{Dataset name.}
   \cpar{ORDER}{\'{E}chelle order number.}
   \cpar{APERTURE}{Aperture name (\verb+SAP+ or \verb+LAP+).}
   \cpar{DEVICE}{GKS/SGS graphics device name.}
}
\sstdescription{
   This command uses the graphics cursor to mark points and
   regions of the current net spectrum that are ``bad'' or ``good''.
   A plot of the \verb+APERTURE+ or \verb+ORDER+ spectrum is required before
   this command can be used.

   The following cursor hit sequences can be used in a cycle:

   \begin {description}
      \item 1 then 1 --- marks all points in the $x$-range GOOD.
      \item 2 then 2 --- marks all points in the $x$-range BAD.
      \item 1 --- marks the nearest point in $x$-direction GOOD.
      \item 2 --- marks the nearest point in $x$-direction BAD.
      \item 3 --- causes the cursor cycle to terminate.
   \end {description}

   The points or ranges changed are printed on the terminal.

   Mouse buttons can be used for cursor hits where:

   \begin {description}
      \item {\bf left} mouse button     is  hit 1.
      \item {\bf middle} mouse button   is  hit 2.
      \item {\bf right} mouse button    is  hit 3.
   \end{description}

   Alternatively, keyboard keys 1, 2 and 3 can be used to mark hits.

   Only the user-defined data quality bit can be changed by this
   command.
   Initially, all faulty points have this bit set BAD ({\it{e.g.,}}\ by
   \verb+TRAK+)\@. However, whether they are considered bad ({\it{e.g.,}}\ when
   plotting or creating output files) is determined by the user-defined
   bit, which can be changed at will.

   See the IUEDR User Guide (MUD/45) for further information on data quality.
}
}

\sstroutine{ERASE}
{
   Erase the display screen of the graphics device.
}{
   \sstparameters{
   \cpar{DEVICE}{GKS/SGS graphics device name.}
}
\sstdescription{
   The display screen of the specified graphics device is erased.
}
}

\sstroutine{EXIT}
{
   Quit IUEDR.
}{
   \sstparameters{
   \npar{None.}
}
\sstdescription{
   This command quits IUEDR\@.
   Any files that require new versions will be written by this command.
   This command is a synonym for the \verb+QUIT+ command.
}
}

\sstroutine{HELP}
{
   Find out about IUEDR commands and parameters.
}{
   \sstparameters{
   \npar{None.}
}
\sstdescription{
   By simply typing \verb+HELP+ \label{com: HELP }the user is presented with
   a brief introduction to IUEDR, a list of the commands available and some
   general information topics. Users familiar with he VMS help system will
   find this facility very essentially the same to use.

   The \verb+HELP+ system provides a list of topics which can be
   selected from by typing enough characters of a topic name to uniquely
   identify it and pressing return.  Pressing the return key with no topic
   chosen takes the \verb+HELP+ system back one topic-level.
   At any time, pressing the return key a few times will return you to the
   IUEDR prompt.

   You may optionally give a specific topic to the \verb+HELP+ command at the
   IUEDR prompt, for example
   \begin{quote}
      \verb+> HELP DRIMAGE+
   \end{quote}
   or even
   \begin{quote}
      \verb+> HELP DRIMAGE COLOUR+
   \end{quote}
}
}

\sstroutine{LBLS}
{
   Extracts a line-by-line-spectrum array from the image.
}{
   \sstparameters{
   \cpar{DATASET}{Dataset name.}
   \cpar{ORDER}{\'{E}chelle order number.}
   \cpar{APERTURE}{Aperture name (\verb+SAP+ or \verb+LAP+).}
   \cpar{GSAMP}{Spectrum grid sampling rate (geometric pixels).}
   \cpar{CUTWV}{Whether wavelength cutoff data used for extraction grid.}
   \cpar{CENTM}{Whether pre-existing centroid template is used.}
   \cpar{RL}{Limits across spectrum for LBLS array (pixels).}
   \cpar{RSAMP}{Radial coordinate sampling rate for LBLS grid (pixels).}
}
\sstdescription{
   This command creates a line-by-line-spectrum (LBLS) array from the
   image defined by \verb+DATASET+\@.
   The array consists of intensities $F(IR, I\lambda )$ for a grid of
   wavelengths, $W(I\lambda)$, and radial coordinates, $R(IR)$\@.
   The wavelength grid, $\lambda$, is determined in a similar way to the
   \verb+TRAK+ command, using the \verb+CUTWV+ (HIRES) and \verb+GSAMP+
   (HIRES/LORES) parameters.

   The radial coordinates are distances from the centre of the spectrum,
   derived from the template data,
   along a line perpendicular to the dispersion direction and
   measured in geometric pixels.
   The radial grid, $R$, is determined by the \verb+RL+ and \verb+RSAMP+
   parameters.

   The value of each pixel in the array corresponds to the surface
   over the image of a rectangle centred on its $(R, \lambda )$ coordinates,
   and extents

   \begin {equation}
      (R(IR) - dR / 2, R(IR) + dR / 2)
   \end {equation}
   and

   \begin {equation}
      (W(I\lambda ) - d\lambda / 2, W(I\lambda ) + d\lambda / 2)
   \end {equation}
   $dR$ is the distance between $R$ values, and $d\lambda$ is the wavelength
   step between $\lambda$ values.

   This surface integral is scaled along the $\lambda$ direction to
   correspond to an interval of 1.414 geometric pixels.
   The reason for this is to make LBLS intensities consistent with
   those produced by the \verb+TRAK+ command.
   For a particular wavelength, $W(I\lambda )$, the sum of LBLS intensities
   after removal of background should correspond to the net
   flux as measured by \verb+TRAK+\@.
}
}

\sstroutine{LISTIUE}
{
   Analyse the contents of IUE tapes or files.
}{
   \sstparameters{
   \cpar{DRIVE}{Tape drive or file name.}
   \cpar{FILE}{Tape file number.}
   \cpar{NFILE}{Number of tape files to be processed.}
   \cpar{NLINE}{Number of IUE header lines printed.}
   \cpar{SKIPNEXT}{Whether skip to next tape file.}
}
\sstdescription{
   This performs an analysis of \verb+NFILE+ IUE tape files, starting at
   the file specified by the \verb+FILE+ parameter.
   \verb+NFILE=-1+ means list all files until the end of the tape.
   \verb+NLINE=-1+ means print all lines in file header.

   \verb+LISTIUE+ can also be used to list the header of a GO format disk file.
}
}

\newpage
\sstroutine{MAP}
{
   Map and merge the extracted spectrum components to produce a mean spectrum.
}{
   \sstparameters{
   \cpar{DATASET}{Dataset name.}
   \cpar{ORDERS}{This delineates a range of \'{e}chelle orders.}
   \cpar{APERTURE}{Aperture name (\verb+SAP+ or \verb+LAP+).}
   \cpar{RM}{Whether mean spectrum is reset before averaging.}
   \cpar{ML}{Wavelength grid limits for mean spectrum.}
   \cpar{MSAMP}{Wavelength sampling rate for mean spectrum grid.}
   \cpar{FILLGAP}{Whether gaps can be filled within order.}
   \cpar{COVERGAP}{Whether gaps can be filled by covering orders.}
}
\sstdescription{
   This command can be used to produce a mean spectrum with contributions
   from several \'{e}chelle orders (HIRES), or from several apertures (LORES).

   If \verb+RM=TRUE+, or if there is no existing mean spectrum, then an
   evenly spaced wavelength grid is constructed between the
   limits specified by the \verb+ML+ parameter using the sampling rate
   specified by the \verb+MSAMP+ parameter.

   If \verb+RM=FALSE+ and there {\bf is}
   an existing mean spectrum, then the
   wavelength grid {\bf and contents} are retained.
   New components will be averaged with what is already there.

   In the case of HIRES, the \verb+ORDERS+ parameter is used to delimit the
   range of \'{e}chelle orders that are allowed to contribute to the mean.

   In the case of LORES, only a single aperture specified by the
   \verb+APERTURE+ parameter is mapped at a given time.
}
}

\sstroutine{MODIMAGE}
{
   Modifies image pixel intensities interactively.
}{
   \sstparameters{
   \cpar{DATASET}{Dataset name.}
   \cpar{DEVICE}{GKS/SGS graphics device name.}
   \cpar{FN}{Replacement Flux Number for pixel.}
}
\sstdescription{
   This command uses the image display cursor to modify image data.
   The image should already have been displayed using the \verb+DRIMAGE+
   command.

   The following cursor sequences are adopted:

   \begin {description}
      \item 1 then 2 --- copy intensity of first picked pixel to the second.
      \item 2 --- prompt for replacement pixel intensity.
      \item 3 --- finish.
   \end {description}

   Mouse buttons can be used for cursor hits where:

   \begin {description}
      \item {\bf left} mouse button     is  hit 1.
      \item {\bf middle} mouse button   is  hit 2.
      \item {\bf right} mouse button    is  hit 3.
   \end{description}

   Alternatively, keyboard keys 1, 2 and 3 can be used to mark hits.

   If the data or data qualities change after a session, then the file is
   saved on disk.

   The assumption is made that the current image displayed corresponds
   to the current dataset!
}
}

\sstroutine{MTMOVE}
{
   Move to the start of a tape file.
}{
   \sstparameters{
   \cpar{DRIVE}{Tape drive.}
   \cpar{FILE}{Tape file number.}
}
\sstdescription{
   Move to the start of the file specified by the \verb+FILE+ parameter on the
   tape specified by the \verb+DRIVE+ parameter.
}
}

\sstroutine{MTREW}
{
   Rewind to the start of the tape.
}{
   \sstparameters{
   \cpar{DRIVE}{Tape drive.}
}
\sstdescription{
   This command rewinds the tape specified by the \verb+DRIVE+ parameter.
   The \verb+FILE+ parameter is also set to 1 by this command.
}
}

\sstroutine{MTSHOW}
{
   Show the current tape position.
}{
   \sstparameters{
   \cpar{DRIVE}{Tape drive.}
}
\sstdescription{
   This command displays the current tape position.
   This includes the file number and the block position relative to either
   the start or the end of the file.

   Note that the actual file position may differ from the
   value of the \verb+FILE+ parameter.
}
}

\sstroutine{MTSKIPEOV}
{
   Skip over end-of-volume (EOV) mark.
}{
   \sstparameters{
   \cpar{DRIVE}{Tape drive.}
}
\sstdescription{
   This command skips over an end-of-volume (EOV) mark on the tape specified
   by the DRIVE parameter.
   An EOV condition is where there are two consecutive tape marks.
   When attempting to skip across an EOV, an error will be reported
   and the tape left positioned between the two marks.
   Subsequent attempts to skip forward will fail and
   only this command can be used to move forward beyond the
   second tape mark.
}
}

\sstroutine{MTSKIPF}
{
   Skip over NSKIP tape marks.
}{
   \sstparameters{
   \cpar{DRIVE}{Tape drive.}
   \cpar{NSKIP}{Number of tape marks to be skipped over.}
}
\sstdescription{
   This command skips over \verb+NSKIP+ tape marks on the tape specified
   by the \verb+DRIVE+ parameter.
   If \verb+NSKIP+ is negative this means that tape marks are skipped in the
   reverse direction, {\it{i.e.}}\ towards the start of the tape.
}
}

\sstroutine{NEWABS}
{
   Associate a new absolute flux calibration with the current dataset.
}{
   \sstparameters{
   \cpar{DATASET}{Dataset name.}
   \cpar{ABSFILE}{Name of file containing absolute flux calibration.}
}
\sstdescription{
   This command reads the absolute flux calibration from a text file
   specified by the \verb+ABSFILE+ parameter and stores it in the dataset
   specified by \verb+DATASET+\@.

   The file type is assumed to be \verb+.abs+ and need not be
   specified as part of the \verb+ABSFILE+ parameter.

   The calibration of any current spectrum is automatically updated.
}
}

\newpage
\sstroutine{NEWCUT}
{
   Associate new \'{e}chelle order wavelength limits with the current
   dataset.
}{
   \sstparameters{
   \cpar{DATASET}{Dataset name.}
   \cpar{CUTFILE}{Name of file containing \'{e}chelle order wavelength limits.}
}
\sstdescription{
   This command reads the \'{e}chelle order wavelength limits from a text file
   specified by the \verb+CUTFILE+ parameter and stores them in the dataset
   specified by \verb+DATASET+\@.

   The file type is assumed to be \verb+.cut+ and need not be
   specified as part of the \verb+CUTFILE+ parameter.

   The calibration of any current spectrum is automatically updated.
}
}

\sstroutine{NEWDISP}
{
   Associate new spectrograph dispersion data with the current dataset.
}{
   \sstparameters{
   \cpar{DATASET}{Dataset name.}
   \cpar{DISPFILE}{Name of file containing dispersion data.}
}
\sstdescription{
   This command reads spectrograph dispersion data from the text file
   specified by the \verb+DISPFILE+ parameter and stores them in the dataset
   specified by \verb+DATASET+\@.

   The file type is assumed to be \verb+.dsp+ and need not be specified
   as part of the \verb+DISPFILE+ parameter.
}
}

\sstroutine{NEWFID}
{
   Read IUE fiducial positions from text file.
}{
   \sstparameters{
   \cpar{DATASET}{Dataset name.}
   \cpar{FIDFILE}{Name of file containing fiducial positions.}
   \cpar{NGEOM}{Number of Chebyshev terms used to represent geometry.}
}
\sstdescription{
   This command reads IUE fiducial positions from a text file
   specified by the \verb+FIDFILE+ parameter and stores them in the dataset
   specified by \verb+DATASET+\@.

   The file type is assumed to be \verb+.fid+ and need not be specified
   as part of the FIDFILE parameter.

   The image data quality and geometry representation are updated to
   account for any changes that these fiducial positions imply.
   In the case of datasets containing image distortion, the \verb+NGEOM+
   parameter is used to specify the number of terms used for the Chebyshev
   representation along each axis.
}
}

\sstroutine{NEWRIP}
{
   Read \'{e}chelle ripple calibration from text file.
}{
   \sstparameters{
   \cpar{DATASET}{Dataset name.}
   \cpar{RIPFILE}{Name of file containing \'{e}chelle ripple calibration.}
}
\sstdescription{
   This command reads an \'{e}chelle ripple calibration from a text file
   specified by the \verb+RIPFILE+ parameter and stores it in the dataset
   specified by \verb+DATASET+\@.

   The file type is assumed to be \verb+.rip+ and need not be specified
   as part of the \verb+RIPFILE+ parameter.

   The calibration of any current spectrum is automatically updated.
}
}

\sstroutine{NEWTEM}
{
   Read spectrum centroid template data from text file.
}{
   \sstparameters{
   \cpar{DATASET}{Dataset name.}
   \cpar{TEMFILE}{Name of file containing spectrum template data.}
}
\sstdescription{
   This command reads the spectrum centroid template data into \verb+DATASET+
   from a text file with name specified by \verb+TEMFILE+\@.

   The file type is assumed to be \verb+.tem+ and need not be specified
   as part of the \verb+TEMFILE+\@.
}
}

\sstroutine{OUTEM}
{
   Output the current spectrum template data to a formatted data file.
}{
   \sstparameters{
   \cpar{DATASET}{Dataset name.}
   \cpar{TEMFILE}{Name of file containing spectrum template data.}
}
\sstdescription{
   This command outputs the templates stored with the current dataset to a
   text file.
   If not specified, the file name is constructed as:

   \begin {quote}
      \verb+<CAMERA>HI<APERTURE>.TEM+
   \end {quote}
   or

   \begin {quote}
      \verb+<CAMERA>LO.TEM+
   \end {quote}

   for the HIRES and LORES cases respectively.
}
}

\sstroutine{OUTLBLS}
{
   Output the current LBLS array to a binary data file.
}{
   \sstparameters{
   \cpar{DATASET}{Dataset name.}
   \cpar{OUTFILE}{Name of output file.}
}
\sstdescription{
   This command outputs the current LBLS array to a file.
   If not specified by the \verb+OUTFILE+ parameter, the file name is
   constructed as:

   \begin {quote}
      \verb+<CAMERA><IMAGE>R.DAT+
   \end {quote}
   The format of this file is described by the Fortran 77 routine, RDLBLS, which
   can be found in the file:

   \begin {quote}
      {\tt \$IUEDR\_USER/rdlbls.for}
   \end {quote}
   The directory {\tt \$IUEDR\_USER} also contains a test
   program for using RDLBLS and other helpful items.
}
}

\sstroutine{OUTMEAN}
{
   Output current mean spectrum to a DIPSO SP format file.
}{
   \sstparameters{
   \cpar{DATASET}{Dataset name.}
   \cpar{OUTFILE}{Name of output file.}
   \cpar{SPECTYPE}{DIPSO SP file type (0, 1 or 2).}
}
\sstdescription{
   This command outputs the mean spectrum associated with
   \verb+DATASET+ to a file that can be read into DIPSO
   (see \xref{SUN/50}{sun50}{}).

   This file is created with type specified by the \verb+SPECTYPE+ parameter
   (see Section~\ref{se:spectrum}
   for SP options).
   If not specified, the file name is constructed as:

   \begin {quote}
      \verb+<CAMERA><IMAGE>M.sdf+ \hspace*{8mm} for SP0\\
      \verb+<CAMERA><IMAGE>M.DAT+ \hspace*{8mm} for SP1 or SP2
   \end {quote}
   In DIPSO SP format, bad points are indicated by having zero intensities.
   In determining which points in the output file are to be marked
   ``bad'', the user-defined data quality bit is used.
   Since this bit can be arbitrarily edited,
   faulty data values can be written to the output file
   without subsequent information being retained.
}
}

\sstroutine{OUTNET}
{
   Output the current net spectrum to a DIPSO  SP format data file.
}{
   \sstparameters{
   \cpar{DATASET}{Dataset name.}
   \cpar{APERTURE}{Aperture name (\verb+SAP+ or \verb+LAP+).}
   \cpar{ORDER}{\'{E}chelle order number.}
   \cpar{OUTFILE}{Name of output file.}
   \cpar{SPECTYPE}{DIPSO SP file type (0, 1 or 2).}
}
\sstdescription{
   This command outputs the net spectrum associated with \verb+ORDER+ or
   \verb+APERTURE+ and \verb+DATASET+ to a file that can be read into DIPSO
   (see \xref{SUN/50}{sun50}{}).

   The file is created with type specified
   by the \verb+SPECTYPE+ parameter (see Section~\ref{se:spectrum}
   for SP
   options).
   If not specified, the file name is constructed as:

   \begin {quote}
      \verb+<CAMERA><IMAGE>+\_\verb+<APERTURE>.sdf+ \hspace*{8mm} for SP0\\
      \verb+<CAMERA><IMAGE>.<APERTURE>+ \hspace*{8mm} for SP1 or SP2
   \end {quote}
   in the case of LORES and

   \begin {quote}
      \verb+<CAMERA><IMAGE>+\_\verb+<ORDER>.sdf+ \hspace*{8mm} for SP0\\
      \verb+<CAMERA><IMAGE>.<ORDER>+ \hspace*{8mm} for SP1 or SP2
   \end {quote}
   in the case of HIRES.
   Here, \verb+<APERTURE>+ is the aperture name (\verb+SAP+ or \verb+LAP+),
   or index, and \verb+<ORDER>+ is the \'{e}chelle order number.

   In DIPSO SP format, bad points are indicated by having zero intensities.
   In determining which points in the output file are to be marked
   ``bad'', the user-defined data quality bit is used.
   Since this bit can be arbitrarily edited,
   faulty data values can be written to the output file
   without subsequent information being retained.
}
}

\sstroutine{OUTRAK}
{
   Output the current uncalibrated spectrum to a ``TRAK'' formatted data file.
}{
   \sstparameters{
   \cpar{DATASET}{Dataset name.}
   \cpar{OUTFILE}{Name of output file.}
}
\sstdescription{
   This command outputs the uncalibrated spectrum associated with
   \verb+DATASET+ to a formatted file that is compatible with output
   from the old ``TRAK'' program.
   The default file name is of the form:

   \begin {quote}
      \verb+<CAMERA><IMAGE>.TRK+
   \end {quote}
   The main difference from an actual ``TRAK'' file is that the background
   level is uniformly zero, so that GROSS=NET.
}
}

\sstroutine{OUTSCAN}
{
   Output the current scan data to a DIPSO  SP format data file.
}{
   \sstparameters{
   \cpar{DATASET}{Dataset name.}
   \cpar{OUTFILE}{Name of output file.}
   \cpar{SPECTYPE}{DIPSO SP file type (0, 1 or 2).}
}
\sstdescription{
   This command outputs the current scan associated with
   \verb+DATASET+ to a file which can be read into DIPSO
   (see \xref{SUN/50}{sun50}{}).

   The file is created with type specified
   by the \verb+SPECTYPE+ parameter
   (see Section~\ref{se:spectrum}
   for SP options).
   If not specified, the file name is constructed as:

   \begin {quote}
      \verb+<CAMERA><IMAGE>P.sdf+ \hspace*{8mm} for SP0\\
      \verb+<CAMERA><IMAGE>P.DAT+ \hspace*{8mm} for SP1 or SP2
   \end {quote}
   In DIPSO SP format, bad points are indicated by having zero intensities.
   In determining which points in the output file are to be marked
   ``bad'', the user-defined data quality bit is used.
   Since this bit can be arbitrarily edited,
   faulty data values can be written to the output file
   without subsequent information being retained.
}
}

\sstroutine{OUTSPEC}
{
   Output the current aperture or order spectrum to a DIPSO SP format data file.
}{
   \sstparameters{
   \cpar{DATASET}{Dataset name.}
   \cpar{APERTURE}{Aperture name (\verb+SAP+ or \verb+LAP+).}
   \cpar{ORDER}{\'{E}chelle order number.}
   \cpar{OUTFILE}{Name of output file.}
   \cpar{SPECTYPE}{DIPSO SP file type (0, 1 or 2).}
}
\sstdescription{
   This command outputs the spectrum associated with the \verb+ORDER+ or
   \verb+APERTURE+ and \verb+DATASET+ to a file which can be read into DIPSO
   (see \xref{SUN/50}{sun50}{}).

   The file is created with type specified
   by the \verb+SPECTYPE+ parameter
   (see Section~\ref{se:spectrum}
   for SP options).
   If not specified, the file name is constructed as:

   \begin {quote}
      \verb+<CAMERA><IMAGE>+\_\verb+<APERTURE>.sdf+ \hspace*{8mm} for SP0\\
      \verb+<CAMERA><IMAGE>.<APERTURE>+ \hspace*{8mm} for SP1 or SP2
   \end {quote}
   in the case of LORES and

   \begin {quote}
      \verb+<CAMERA><IMAGE>+\_\verb+<ORDER>.sdf+ \hspace*{8mm} for SP0\\
      \verb+<CAMERA><IMAGE>.<ORDER>+ \hspace*{8mm} for SP1 or SP2
   \end {quote}
   in the case of HIRES.
   Here, \verb+<APERTURE>+ is the aperture name (\verb+SAP+ or \verb+LAP+), or
   index, and \verb+<ORDER>+ is the \'{e}chelle order number.

   In DIPSO SP format, bad points are indicated by having zero intensities.
   In determining which points in the output file are to be marked
   ``bad'', the user-defined data quality bit is used.
   Since this bit can be arbitrarily edited,
   faulty data values can be written to the output file
   without subsequent information being retained.
}
}

\sstroutine{PLCEN}
{
   Plot smoothed centroid shifts.
}{
   \sstparameters{
   \cpar{DATASET}{Dataset name.}
   \cpar{ORDER}{\'{E}chelle order number.}
   \cpar{APERTURE}{Aperture name (\verb+SAP+ or \verb+LAP+).}
   \cpar{RS}{Whether display is reset before plotting.}
   \cpar{DEVICE}{GKS/SGS graphics device name.}
   \cpar{ZONE}{Zone to be used for plotting.}
   \cpar{LINE}{Plotting line style (\verb+SOLID+, \verb+DASH+, \verb+DOTDASH+
               or \verb+DOT+).}
   \cpar{LINEROT}{Whether line style is changed after next plot.}
   \cpar{COL}{Plotting line colour (1, 2, 3, \dots 10).}
   \cpar{COLROT}{Whether line colour is changed after next plot.}
   \cpar{XL}{$x$-axis plotting limits, undefined or [0, 0] means auto-scale.}
   \cpar{YL}{$y$-axis plotting limits, undefined or [0, 0] means auto-scale.}
}
\sstdescription{
   This command plots the smoothed centroid shifts produced during
   the most recent spectrum extraction from \verb+DATASET+
   on the graphics device and zone specified by the \verb+DEVICE+ and
   \verb+ZONE+ parameters respectively.

   In the case of a LORES spectrum, if there is more than a single
   aperture available, then the \verb+APERTURE+ parameter needs to be specified.

   In the case of a HIRES spectrum, if there is more than a single
   \'{e}chelle order, then the \verb+ORDER+ parameter needs to be specified.

   The \verb+RS+ parameter specifies whether a new plot is started, or whether
   the data can be plotted over an existing plot.

   The \verb+LINE+ and \verb+LINEROT+ parameters determine the line style
   which will be used for plotting.

   The \verb+COL+ and \verb+COLROT+ parameters determine the line colour which
   will be used for plotting if the \verb+DEVICE+ supports colour graphics.

   The diagram limits are specified by the \verb+XL+ and \verb+YL+ parameter
   values.
   If \verb+XL+ and \verb+YL+ have values

   \begin {quote}
      \verb+XL=[0,0], YL=[0,0]+
   \end {quote}
   then the plot limits along each axis are determined so that the whole
   spectrum is visible.
   If the values of \verb+XL+ or \verb+YL+ are specified in decreasing order,
   then the coordinates will be reversed along the appropriate axis.
}
}

\sstroutine{PLFLUX}
{
   Plot calibrated flux spectrum.
}{
   \sstparameters{
   \cpar{DATASET}{Dataset name.}
   \cpar{ORDER}{\'{E}chelle order number.}
   \cpar{APERTURE}{Aperture name (\verb+SAP+ or \verb+LAP+).}
   \cpar{RS}{Whether display is reset before plotting.}
   \cpar{DEVICE}{GKS/SGS graphics device name.}
   \cpar{ZONE}{Zone to be used for plotting.}
   \cpar{LINE}{Plotting line style (\verb+SOLID+, \verb+DASH+, \verb+DOTDASH+
               or \verb+DOT+).}
   \cpar{LINEROT}{Whether line style is changed after next plot.}
   \cpar{COL}{Plotting line colour (1, 2, 3, \dots 10).}
   \cpar{COLROT}{Whether line colour is changed after next plot.}
   \cpar{HIST}{Whether lines are drawn as histograms.}
   \cpar{QUAL}{Whether data quality information is plotted.}
   \cpar{XL}{$x$-axis plotting limits, undefined or [0, 0] means auto-scale.}
   \cpar{YL}{$y$-axis plotting limits, undefined or [0, 0] means auto-scale.}
}
\sstdescription{
   This command plots the calibrated flux spectrum from \verb+DATASET+
   on the graphics device and zone specified by the \verb+DEVICE+ and
   \verb+ZONE+ parameters respectively.

   In the case of a LORES spectrum, if there is more than a single
   aperture available, then the \verb+APERTURE+ parameter needs to be specified.

   In the case of a HIRES spectrum, if there is more than a single
   \'{e}chelle order, then the \verb+ORDER+ parameter needs to be specified.

   The \verb+RS+ parameter specifies whether a new plot is started, or whether
   the data can be plotted over an existing plot.

   The \verb+HIST+ parameter determines whether the line is drawn as a
   histogram rather than a continuous polyline.

   The \verb+LINE+ and \verb+LINEROT+ parameters determine the line style
   which will be used for plotting.

   The \verb+COL+ and \verb+COLROT+ parameters determine the line colour which
   will be used for plotting if the \verb+DEVICE+ supports colour graphics.

   The diagram limits are specified by the \verb+XL+ and \verb+YL+ parameter
   values.
   If \verb+XL+ and \verb+YL+ have values

   \begin {quote}
      \verb+XL=[0,0], YL=[0,0]+
   \end {quote}
   then the plot limits along each axis are determined so that the whole
   spectrum is visible.
   If the values of \verb+XL+ or \verb+YL+ are specified in decreasing order,
   then the coordinates will be reversed along the appropriate axis.

   The \verb+QUAL+ parameter indicates whether faulty points are flagged with
   their data quality codes (see Section~\ref{se:introduction}).

   If a point is affected by more than one of the above faults, then
   the highest code is plotted.
   Points marked bad by user edits are only indicated if they are otherwise
   fault-free.
}
}

\sstroutine{PLGRS}
{
   Plot pseudo-gross and background from spectrum extraction.
}{
   \sstparameters{
   \cpar{DATASET}{Dataset name.}
   \cpar{ORDER}{\'{E}chelle order number.}
   \cpar{APERTURE}{Aperture name (\verb+SAP+ or \verb+LAP+).}
   \cpar{RS}{Whether display is reset before plotting.}
   \cpar{DEVICE}{GKS/SGS graphics device name.}
   \cpar{ZONE}{Zone to be used for plotting.}
   \cpar{LINE}{Plotting line style (\verb+SOLID+, \verb+DASH+, \verb+DOTDASH+
               or \verb+DOT+).}
   \cpar{LINEROT}{Whether line style is changed after next plot}
   \cpar{COL}{Plotting line colour (1, 2, 3, \ldots 10).}
   \cpar{COLROT}{Whether line colour is changed after next plot.}
   \cpar{HIST}{Whether lines are drawn as histograms.}
   \cpar{QUAL}{Whether data quality information is plotted.}
   \cpar{XL}{$x$-axis plotting limits, undefined or [0, 0] means auto-scale.}
   \cpar{YL}{$y$-axis plotting limits, undefined or [0, 0] means auto-scale.}
}
\sstdescription{
   This command plots the pseudo-gross and smooth background produced during
   the most recent spectrum extraction from \verb+DATASET+
   on the graphics device and zone specified by the \verb+DEVICE+ and
   \verb+ZONE+ parameters respectively.

   The pseudo-gross is constructed by taking the net spectrum and adding
   the smooth background multiplied by the width of the object channel.
   The smooth background plotted is also for the object channel width.

   In the case of a LORES spectrum, if there is more than a single
   aperture available, then the \verb+APERTURE+ parameter needs to be specified.

   In the case of a HIRES spectrum, if there is more than a single
   \'{e}chelle order, then the \verb+ORDER+ parameter needs to be specified.

   The \verb+RS+ parameter specifies whether a new plot is started, or whether
   the data can be plotted over an existing plot.

   The \verb+HIST+ parameter determines whether the line is drawn as a histogram
   rather than a continuous polyline.

   The \verb+LINE+ and \verb+LINEROT+ parameters determine the line style
   which will be used for plotting.

   The \verb+COL+ and \verb+COLROT+ parameters determine the line colour which
   will be used for plotting if the \verb+DEVICE+ supports colour graphics.

   The diagram limits are specified by the \verb+XL+ and \verb+YL+ parameter
   values.
   If \verb+XL+ and \verb+YL+ have values

   \begin {quote}
      \verb+XL=[0,0], YL=[0,0]+
   \end {quote}
   then the plot limits along each axis are determined so that the whole
   spectrum is visible.
   If the values of \verb+XL+ or \verb+YL+ are specified in decreasing order,
   then the coordinates will be reversed along the appropriate axis.

   The \verb+QUAL+ parameter indicates whether faulty points are flagged with
   their data quality codes (see Section~\ref{se:introduction}).

   If a point is affected by more than one of the above faults, then
   the highest code is plotted.
   Points marked bad by user edits are only indicated if they are otherwise
   fault-free.
}
}

\sstroutine{PLMEAN}
{
   Plot mean spectrum.
}{
   \sstparameters{
   \cpar{DATASET}{Dataset name.}
   \cpar{RS}{Whether display is reset before plotting.}
   \cpar{DEVICE}{GKS/SGS graphics device name.}
   \cpar{ZONE}{Zone to be used for plotting.}
   \cpar{LINE}{Plotting line style (\verb+SOLID+, \verb+DASH+, \verb+DOTDASH+
               or \verb+DOT+).}
   \cpar{LINEROT}{Whether line style is changed after next plot}
   \cpar{COL}{Plotting line colour (1, 2, 3, \ldots 10).}
   \cpar{COLROT}{Whether line colour is changed after next plot.}
   \cpar{HIST}{Whether lines are drawn as histograms.}
   \cpar{QUAL}{Whether data quality information is plotted.}
   \cpar{XL}{$x$-axis plotting limits, undefined or [0, 0] means auto-scale.}
   \cpar{YL}{$y$-axis plotting limits, undefined or [0, 0] means auto-scale.}
}
\sstdescription{
   This command plots the mean spectrum associated with \verb+DATASET+ on the
   graphics device and zone specified by the \verb+DEVICE+ and \verb+ZONE+
   parameters respectively.

   The \verb+RS+ parameter specifies whether a new plot is started, or whether
   the data can be plotted over an existing plot.

   The \verb+HIST+ parameter determines whether the line is drawn as a histogram
   rather than a continuous polyline.

   The \verb+LINE+ and \verb+LINEROT+ parameters determine the line style which
   will be used for plotting.

   The \verb+COL+ and \verb+COLROT+ parameters determine the line colour which
   will be used for plotting if the \verb+DEVICE+ supports colour graphics.

   The diagram limits are specified by the \verb+XL+ and \verb+YL+ parameter
   values.
   If \verb+XL+ and \verb+YL+ have values

   \begin {quote}
      \verb+XL=[0,0], YL=[0,0]+
   \end {quote}
   then the plot limits along each axis are determined so that the whole
   spectrum is visible.
   If the values of \verb+XL+ or \verb+YL+ are specified in decreasing order,
   then the coordinates will be reversed along the appropriate axis.

   The \verb+QUAL+ parameter indicates whether faulty points are flagged with
   their data quality codes (see Section~\ref{se:introduction}).

   If a point is affected by more than one of the above faults, then
   the highest code is plotted.
   Points marked bad by user edits are only indicated if they are otherwise
   fault-free.
}
}

\sstroutine{PLNET}
{
   Plot uncalibrated net spectrum.
}{
   \sstparameters{
   \cpar{DATASET}{Dataset name.}
   \cpar{ORDER}{\'{E}chelle order number.}
   \cpar{APERTURE}{Aperture name (\verb+SAP+ or \verb+LAP+).}
   \cpar{RS}{Whether display is reset before plotting.}
   \cpar{DEVICE}{GKS/SGS graphics device name.}
   \cpar{ZONE}{Zone to be used for plotting.}
   \cpar{LINE}{Plotting line style (\verb+SOLID+, \verb+DASH+, \verb+DOTDASH+
               or \verb+DOT+).}
   \cpar{LINEROT}{Whether line style is changed after next plot}
   \cpar{COL}{Plotting line colour (1, 2, 3, \ldots 10).}
   \cpar{COLROT}{Whether line colour is changed after next plot.}
   \cpar{HIST}{Whether lines are drawn as histograms.}
   \cpar{QUAL}{Whether data quality information is plotted.}
   \cpar{XL}{$x$-axis plotting limits, undefined or [0, 0] means auto-scale.}
   \cpar{YL}{$y$-axis plotting limits, undefined or [0, 0] means auto-scale.}
}
\sstdescription{
   This command plots the uncalibrated net spectrum specified by the
   \verb+DATASET+
   parameter on the graphics device and zone specified by the
   \verb+DEVICE+ and \verb+ZONE+ parameters respectively.

   In the case of a LORES spectrum, if there is more than a single
   aperture available, then the \verb+APERTURE+ parameter needs to be specified.

   In the case of a HIRES spectrum, if there is more than a single
   \'{e}chelle order, then the \verb+ORDER+ parameter needs to be specified.

   The \verb+RS+ parameter specifies whether a new plot is started, or whether
   the data can be plotted over an existing plot.

   The \verb+HIST+ parameter determines whether the line is drawn as a histogram
   rather than a continuous polyline.

   The \verb+LINE+ and \verb+LINEROT+ parameters determine the line style which
   will be used for plotting.

   The \verb+COL+ and \verb+COLROT+ parameters determine the line colour which
   will be used for plotting if the \verb+DEVICE+ supports colour graphics.

   The diagram limits are specified by the \verb+XL+ and \verb+YL+ parameter
   values.
   If \verb+XL+ and \verb+YL+ have values

   \begin {quote}
      \verb+XL=[0,0], YL=[0,0]+
   \end {quote}
   then the plot limits along each axis are determined so that the whole
   spectrum is visible.
   If the values of \verb+XL+ or \verb+YL+ are specified in decreasing order,
   then the coordinates will be reversed along the appropriate axis.

   The \verb+QUAL+ parameter indicates whether faulty points are flagged with
   their data quality codes (see Section~\ref{se:introduction}).

   If a point is affected by more than one of the above faults, then
   the highest code is plotted.
   Points marked bad by user edits are only indicated if they are otherwise
   fault-free.
}
}

\sstroutine{PLSCAN}
{
   Plot scan perpendicular to dispersion.
}{
   \sstparameters{
   \cpar{DATASET}{Dataset name.}
   \cpar{RS}{Whether display is reset before plotting.}
   \cpar{DEVICE}{GKS/SGS graphics device name.}
   \cpar{ZONE}{Zone to be used for plotting.}
   \cpar{LINE}{Plotting line style (\verb+SOLID+, \verb+DASH+, \verb+DOTDASH+
               or \verb+DOT+).}
   \cpar{LINEROT}{Whether line style is changed after next plot.}
   \cpar{COL}{Plotting line colour (1, 2, 3, \ldots 10).}
   \cpar{COLROT}{Whether line colour is changed after next plot.}
   \cpar{QUAL}{Whether data quality information is plotted.}
   \cpar{XL}{$x$-axis plotting limits, undefined or [0, 0] means auto-scale.}
   \cpar{YL}{$y$-axis plotting limits, undefined or [0, 0] means auto-scale.}
}
\sstdescription{
   This command plots the most recent scan perpendicular to dispersion
   associated with \verb+DATASET+ on the graphics device and zone specified by
   the \verb+DEVICE+ and \verb+ZONE+ parameters respectively.

   The \verb+RS+ parameter specifies whether a new plot is started, or whether
   the data can be plotted over an existing plot.

   The \verb+LINE+ and \verb+LINEROT+ parameters determine the line style which
   will be used for plotting.

   The \verb+COL+ and \verb+COLROT+ parameters determine the line colour which
   will be used for plotting if the \verb+DEVICE+ supports colour graphics.

   The diagram limits are specified by the \verb+XL+ and \verb+YL+ parameter
   values.
   If \verb+XL+ and \verb+YL+ have values

   \begin {quote}
      \verb+XL=[0,0], YL=[0,0]+
   \end {quote}
   then the plot limits along each axis are determined so that the whole
   spectrum is visible.
   If the values of \verb+XL+ or \verb+YL+ are specified in decreasing order,
   then the coordinates will be reversed along the appropriate axis.

   The \verb+QUAL+ parameter indicates whether faulty points are flagged with
   their data quality codes (see Section~\ref{se:introduction}).

   If a point is affected by more than one of the above faults, then
   the highest code is plotted.
   Points marked bad by user edits are only indicated if they are otherwise
   fault-free.
}
}

\newpage
\sstroutine{PRGRS}
{
   Print the current extracted aperture or order spectrum in tabular
   form.
}{
   \sstparameters{
   \cpar{DATASET}{Dataset name.}
   \cpar{APERTURE}{Aperture name (\verb+SAP+ or \verb+LAP+).}
}
\sstdescription{
   This command prints the recently extracted spectrum associated
   with \verb+ORDER+ or \verb+APERTURE+
   and \verb+DATASET+ in tabular form.
   The table consists of wavelengths, ``gross'', smooth background, net
   and calibrated fluxes, along
   with any data quality information.
   The ``gross'' and smooth background correspond to an image sample
   with width specified by the adopted extraction slit.

   The output from this command is likely to be too voluminous to read at the
   terminal, refering to the \verb+session.lis+ file may be easier.
}
}

\sstroutine{PRLBLS}
{
   Print the current LBLS array in tabular form.
}{
   \sstparameters{
   \cpar{DATASET}{Dataset name.}
}
\sstdescription{
   This command prints the current LBLS array in tabular form.

   Each line of the main table consists of a wavelength and a set of
   mapped image intensities (FN), corresponding to cells at distances, R
   (pixels), from spectrum centre.

   Any array cells which are affected by ``bad'' image pixels
   ({\it{e.g.,}}\ reseaux, saturation, etc.) have data quality values printed
   below them, the meaning of which is given at the start of the output.

   The output from this command is likely to be too voluminous to read at the
   terminal, refering to the \verb+session.lis+ file may be easier.
}
}

\sstroutine{PRMEAN}
{
   Print the current mean spectrum in tabular form.
}{
   \sstparameters{
   \cpar{DATASET}{Dataset name.}
}
\sstdescription{
   This command prints the mean spectrum associated with DATASET in tabular form.

   The table consists of wavelengths and calibrated fluxes, along
   with any data quality information.

   The output from this command is likely to be too voluminous to read at the
   terminal, refering to the \verb+session.lis+ file may be easier.
}
}

\sstroutine{PRSCAN}
{
   Print the intensities of the current image scan in tabular form.
}{
   \sstparameters{
   \cpar{DATASET}{Dataset name.}
}
\sstdescription{
   This command prints the scan associated with \verb+DATASET+ in tabular form.
   The table consists of wavelengths and net fluxes, along
   with any data quality information.

   The output from this command should be diverted to a file, since
   it is likely to be too voluminous to read at the terminal.
}
}

\sstroutine{PRSPEC}
{
   Print the current aperture or order spectrum in tabular form.
}{
   \sstparameters{
   \cpar{DATASET}{Dataset name.}
   \cpar{APERTURE}{Aperture name (\verb+SAP+ or \verb+LAP+).}
   \cpar{ORDER}{\'{E}chelle order number.}
}
\sstdescription{
   This command prints the spectrum associated with \verb+DATASET+ and either
   \verb+ORDER+ or \verb+APERTURE+ in tabular form.

   The table consists of wavelengths, net and calibrated fluxes, along
   with any data quality information.

   The output from this command is likely to be too voluminous to read at the
   terminal, refering to the \verb+session.lis+ file may be easier.
}
}

\sstroutine{QUIT}
{
   Quit IUEDR.
}{
   \sstparameters{
   \npar{None.}
}
\sstdescription{
   This command quits IUEDR.
   Any files that require new versions will be written by this
   command.
   This command is a synonym for the \verb+EXIT+ command.
}
}

\newpage
\sstroutine{READIUE}
{
   Read a RAW, GPHOT or PHOT IUE image from GO format tape or file.
}{
   \sstparameters{
   \cpar{DRIVE}{Tape drive or file name.}
   \cpar{FILE}{Tape file number.}
   \cpar{NLINE}{Number of IUE header lines printed.}
   \cpar{DATASET}{Dataset name.}
   \cpar{TYPE}{Dataset type (\verb+RAW+, \verb+PHOT+ or \verb+GPHOT+).}
   \cpar{OBJECT}{Object identification text.}
   \cpar{CAMERA}{Camera name (\verb+LWP+, \verb+LWR+ or \verb+SWP+).}
   \cpar{IMAGE}{Image number.}
   \cpar{APERTURES}{Aperture name.}
   \cpar{RESOLUTION}{Spectrograph resolution mode (\verb+LORES+ or
                     \verb+HIRES+).}
   \cpar{EXPOSURES}{Spectrum exposure time(s) (seconds).}
   \cpar{THDA}{IUE camera temperature (C).}
   \cpar{ITFMAX}{Pixel value on tape for ITF saturation.}
   \cpar{BADITF}{Whether bad LORES SWP ITF requires correction.}
   \cpar{YEAR}{Year number (A.D.).}
   \cpar{MONTH}{Month number (1-12).}
   \cpar{DAY}{Day number in Month.}
   \cpar{NGEOM}{Number of Chebyshev terms used to represent geometry.}
   \cpar{ITF}{This is the ITF generation used in the image calibration.}
}
\sstdescription{
   This command reads an IUE dataset (\verb+RAW+, \verb+GPHOT+ or \verb+PHOT+)
   from tape or from a GO format disk file.
   The \verb+DATASET+ parameter determines the names of files that will contain
   the various data components ({\it{e.g.,}}\ Calibration, Image \& Image Quality
   etc.)

   The \verb+ITFMAX+ and \verb+NGEOM+ parameters are only prompted for if the
   image data are geometrically and photometrically calibrated.
}
}


\newpage
\sstroutine{READSIPS}
{
   Read MELO/MEHI from IUESIPS\#1 or IUESIPS\#2 tape or file.
}{
   \sstparameters{
   \cpar{DRIVE}{Tape drive or file name.}
   \cpar{FILE}{Tape file number.}
   \cpar{NLINE}{Number of IUE header lines printed.}
   \cpar{DATASET}{Dataset name.}
   \cpar{OBJECT}{Object identification text.}
   \cpar{CAMERA}{Camera name (\verb+LWP+, \verb+LWR+ or \verb+SWP+).}
   \cpar{IMAGE}{Image number.}
   \cpar{APERTURES}{Aperture name.}
   \cpar{EXPOSURES}{Spectrum exposure time(s) (seconds).}
   \cpar{THDA}{IUE camera temperature (C).}
   \cpar{YEAR}{Year number (A.D.).}
   \cpar{MONTH}{Month number (1-12).}
   \cpar{DAY}{Day number in Month.}
   \cpar{ITF}{This is the ITF generation used in the image calibration.}
}
\sstdescription{
   This command reads the MELO/MEHI product from IUESIPS\#1 or IUESIPS\#2
   tape or file.  Operation is much like \verb+READIUE+, except that some
   parameters and associated information are not needed. Only calibration
   ({\tt .UEC}) and spectrum ({\tt \_UES.sdf}) files are created.
   The values for certain parameters may be obtained from the tape or file,
   in which case you will not be prompted for them.
}
}

\sstroutine{SAVE}
{
   Write new versions for any files that have had their contents
   changed during the current session.
}{
   \sstparameters{
   \npar{None.}
}
\sstdescription{
   This command overwrites dataset files that have had their contents
   changed during the current session.
   If there are no outstanding files then this command does nothing.
}
}

\newpage
\sstroutine{SCAN}
{
   This command performs a scan perpendicular to spectrograph
   dispersion.
}{
   \sstparameters{
   \cpar{DATASET}{Dataset name.}
   \cpar{ORDERS}{This delineates a range of \'{e}chelle orders.}
   \cpar{APERTURE}{Aperture name (\verb+SAP+ or \verb+LAP+).}
   \cpar{SCANDIST}{HIRES scan distance from camera faceplate centre}
   \cparc{(geometric pixels.)}
   \cpar{SCANAV}{Averaging filter HWHM for image scan}
   \cparc{(geometric pixels).}
   \cpar{SCANWV}{Central wavelength for LORES image scan (\AA).}
}
\sstdescription{
   This command performs a scan perpendicular to spectrograph dispersion.
   The scan is performed by folding pixels with a triangle function
   with HWHM of \verb+SCANAV+ geometric pixels along the dispersion direction.

   In the case of HIRES, the \verb+SCANDIST+ parameter determines the distance
   of the scan from the faceplate centre.

   In the case of LORES, the \verb+SCANWV+ parameter determines the central
   wavelength of the scan in Angstroms.

   The algorithm used to produce scan intensities is not very good
   and so quantitative results should not be sought from this command.
   Its sole intention lies in providing data for aligning the spectrum.
}
}

\sstroutine{SETA}
{
   Set dataset parameters that are APERTURE specific.
}{
   \sstparameters{
   \cpar{DATASET}{Dataset name.}
   \cpar{APERTURE}{Aperture name (\verb+SAP+ or \verb+LAP+).}
   \cpar{EXPOSURE}{Spectrum exposure time (seconds).}
   \cpar{FSCALE}{Flux scale factor.}
   \cpar{WSHIFT}{Constant wavelength shift (\AA).}
   \cpar{VSHIFT}{Velocity shift of detector relative to Sun (km/s).}
   \cpar{ESHIFT}{Global \'{e}chelle wavelength shift.}
   \cpar{GSHIFT}{Global shift of spectrum on image (geometric pixels).}
}
\sstdescription{
   This command allows changes to be made to dataset values which are
   specific to the specified \verb+APERTURE+\@.
   Items for which parameters are not specified retain their current
   values.
}
}

\sstroutine{SETD}
{
   Set dataset parameters which are independent of ORDER/APERTURE.
}{
   \sstparameters{
   \cpar{DATASET}{Dataset name.}
   \cpar{OBJECT}{Object identification text.}
   \cpar{THDA}{IUE camera temperature (C).}
   \cpar{FIDSIZE}{Half width of fiducials (pixels).}
   \cpar{BADITF}{Whether bad LORES ITF requires correction.}
   \cpar{NGEOM}{Number of Chebyshev terms used to represent geometry.}
   \cpar{RIPK}{\'{E}chelle ripple constant (\AA).}
   \cpar{RIPA}{Ripple function scale factor.}
   \cpar{XCUT}{Global \'{e}chelle wavelength clipping.}
   \cpar{HALTYPE}{The type of halation (order-overlap) correction used.}
   \cpar{HALC}{Halation correction constant (fraction of continuum).}
   \cpar{HALWC}{Wavelength for which the halation correction \verb+HALC+}
   \cparc{is defined in angstroms.}
   \cpar{HALW0}{Wavelength at which halation correction is zero (\AA).}
   \cpar{HALAV}{Averaging FWHM for halation correction (gometric pixels).}
}
\sstdescription{
   This command allows changes to be made to dataset values which are
   independent of any specific \verb+APERTURE+/\verb+ORDER+\@.
   Items for which parameters are not specified retain their current
   values.
}
}

\sstroutine{SETM}
{
   Set dataset parameters that are ORDER specific.
}{
   \sstparameters{
   \cpar{DATASET}{Dataset name.}
   \cpar{ORDER}{\'{E}chelle order number.}
   \cpar{RIPK}{\'{E}chelle ripple constant (\AA).}
   \cpar{RIPA}{Ripple function scale factor.}
   \cpar{RIPC}{Ripple function correction polynomial.}
   \cpar{WCUT}{Wavelength limits for \'{e}chelle order (\AA).}
}
\sstdescription{
   This command allows changes to be made to dataset values which are
   specific to the specified \verb+ORDER+\@.
   Items for which parameters are not specified retain their current
   values.
}
}

\newpage
\sstroutine{SGS}
{
   Write names of available SGS devices at the terminal.
}{
   \sstparameters{
   \npar{None.}
}
\sstdescription{
   This commands writes a list of available SGS device names at the terminal.
   See \xref{SUN/85}{sun85}{} for details of the SGS graphics system.
}
}

\sstroutine{SHOW}
{
   Print dataset values.
}{
   \sstparameters{
   \cpar{DATASET}{Dataset name.}
   \cpar{V}{List of items to be printed.}
}
\sstdescription{
   This command shows the values of parameters in the dataset specified
   by the \verb+DATASET+ parameter. The items to be printed are specified by
   the \verb+V+ parameter, which is a string containing any of the following
   characters:

   \begin {description}
      \item H --- Header and file information
      \item I --- Image details
      \item F --- Fiducials
      \item G --- Geometry
      \item D --- Dispersion
      \item C --- Centroid templates
      \item R --- \'{E}chelle Ripple and halation
      \item A --- Absolute calibration
      \item S --- Raw Spectrum
      \item M --- Mean spectrum
      \item * --- All of the above
      \item Q --- Image data Quality summary
   \end {description}

   The \verb+*+ character needs to be placed within inverted commas.

   The \verb+V+ parameter is cancelled afterwards.
}
}

\sstroutine{TRAK}
{
   Extract spectrum from image.
}{
   \sstparameters{
   \cpar{DATASET}{Dataset name.}
   \cpar{ORDER}{\'{E}chelle order number.}
   \cpar{APERTURE}{Aperture name (\verb+SAP+ or \verb+LAP+).}
   \cpar{NORDER}{Number of \'{e}chelle orders to be processed.}
   \cpar{AUTOSLIT}{Whether \verb+GSLIT+, \verb+BDIST+ and \verb+BSLIT+
                   parameters are}
   \cparc{determined automatically.}
   \cpar{GSLIT}{Object channel limits (geometric pixels).}
   \cpar{BSLIT}{Background channel half widths (geometric pixels).}
   \cpar{BDIST}{Distances of background channels from object channel}
   \cparc{centre (geometric pixels).}
   \cpar{GSAMP}{Spectrum grid sampling rate (geometric pixels).}
   \cpar{CUTWV}{Whether wavelength cutoff data used for extraction grid.}
   \cpar{BKGIT}{Number of background smoothing iterations.}
   \cpar{BKGAV}{Background averaging filter FWHM (geometric pixels).}
   \cpar{BKGSD}{Discrimination level for background pixels (s.d.).}
   \cpar{CENTM}{Whether pre-existing centroid template is used.}
   \cpar{CENSH}{Whether the spectrum template is just shifted linearly.}
   \cpar{CENSV}{Whether the spectrum template is saved in the dataset.}
   \cpar{CENIT}{Number of centroid tracking iterations.}
   \cpar{CENAV}{Centroid averaging filter FWHM (geometric pixels).}
   \cpar{CENSD}{Significance level for signal to be used for centroids
                (s.d.).}
   \cpar{EXTENDED}{Whether the object is not a point source.}
   \cpar{CONTINUUM}{Whether the object spectrum is expected to have a}
   \cparc{``continuum''.}
}
\sstdescription{
   This command extracts a spectrum from an image.
   It does this by defining an evenly spaced wavelength grid along the
   spectrum, and mapping pixel intensities onto this grid in object
   and background channels.
   The background pixels are used to form a smooth background spectrum.
   The object pixels (less smooth background) are used to form the
   integrated net signal for the object.

   In the LORES case, the spectrum specified by the \verb+APERTURE+ parameter
   is extracted.

   In the HIRES case, the first \'{e}chelle order to be extracted is
   specified by \verb+ORDER+\@.
   Up to \verb+NORDER+ orders are extracted, with \verb+ORDER+ being
   decremented each time.

   The wavelength grid is defined by the region of the dispersion line
   contained in the image subset (faceplate).
   The grid spacing is set by the \verb+GSAMP+ parameter value which is
   the sample step in geometric pixels.
   The wavelength limits can optionally be constrained within the
   \'{e}chelle cutoff values by specifying \verb+CUTWV=TRUE+\@.

   The object and background channel widths and positions are determined
   automatically if \verb+AUTOSLIT=TRUE+\@.
   Otherwise, the object channel is specified by the values of the
   \verb+GSLIT+ parameter, whilst the background channel positions and
   widths are determined by the \verb+BDIST+ and \verb+BSLIT+ parameter values
   respectively.

   The \verb+EXTENDED+ and \verb+CONTINUUM+ parameters allow more precise
   control over slit determinations (see the IUEDR User Guide (MUD/45) for
   details).

   The background spectrum is smoothed with a triangle function with a
   FWHM given in geometric pixels by the \verb+BKGAV+ parameter.
   Once the background channel spectra have been obtained, they are
   extracted a further \verb+BKGIT+ times.
   Prior to each additional background extraction pixels which are
   outside \verb+BKGSD+ local standard deviations are rejected.

   The object spectrum is obtained by integrating pixel intensities
   (less smooth background) within the object channel.
   Once the object spectrum has been obtained it is extracted an
   additional \verb+CENIT+ times, the centroid positions
   from the previous extraction being used to ``follow'' the
   spectrum each time.
   The centroid spectrum (template) is smoothed by folding with a
   triangle function, FWHM given in geometric
   pixels by the \verb+CENAV+ parameter.
   Wavelengths with flux levels below \verb+CENSD+ standard deviations
   above background are not used in determining the centroid spectrum.

   By default, the initial spectrum template is given by the dispersion
   relations and the geometric shifts determined using the \verb+CGSHIFT+\@.
   However, if \verb+CENTM=TRUE+, then a pre-defined template associated with
   the dataset may be used as a start guess.
   If \verb+CENSH=TRUE+, then this template can be shifted linearly to match the
   image ({\it{i.e.}}\ without changing its shape).
   If \verb+CENSV=TRUE+, then the final centroid spectrum is used to update the
   spectrum template in the dataset.

   The net flux associated with a wavelength point in the final extracted
   spectrum is defined as the integral of pixel intensities over a rectangle
   with dimensions given by the object channel width and the wavelength
   interval.
   These fluxes are scaled so that they correspond to an interval
   along the wavelength direction of 1.414 geometric pixels.
   This is so that the standard IUESIPS calibrations can be applied
   regardless of what actual sample rate has been employed.
   The integral is performed by using linear interpolation of pixel intensities.
}
}

\newpage
%------------------------------------------------------------------------------

\section{\xlabel{paramaters}\label{se:parameters}Parameters}
\markboth{Parameters}{\stardocname}

There follows a detailed description for each of the parameters used by
IUEDR commands.
The description for a particular parameter applies in any
command which uses it.
Some parameters have default values which are initialised on invoking IUEDR\@.
Default parameter behaviour is described in
Appendix~\ref{se:parameter_defaults}\@.

This release of IUEDR uses the ADAM parameter system. In this system
the parameters and their usage are described in an interface file
(See \xref{SG/4}{sg4}{} for more detail). It is possible to override the
default
interface file with your own personal version. This permits you to tailor
the precise behavior of each parameter according to your requirements.

In the following descriptions the required parameter value is one of:
\begin{description}
   \item [\_CHAR] A character string.
   \item [\_DOUBLE] A floating point number.  The decimal point need not be
                    included if the value is integer only.
   \item [\_INTEGER] Integer number.
   \item [\_LOGICAL] A logical value: {\tt YES, TRUE, NO} or {\tt FALSE.}
\end{description}

Where a parameter value is of the form:

\begin{quote}
{\bf
   [\_TYPE\{,\_TYPE\}]
}
\end{quote}

At least one value of type {\bf \_TYPE} is required, the second is optional.

\rule{\textwidth}{0.5mm}

\iueparlist{

\iueparameter{ABSFILE}
{
   \_CHAR
}{
   This is the name of a file containing an absolute flux calibration.
   A file type of \verb+.abs+ is assumed and need not be specified
   explicitly.
   If the file name contains a directory specification, then it should be
   enclosed in quotes.
}

\iueparameter{APERTURE}
{
   \_CHAR
}{
   This is the name of an individual aperture.
   The following names have defined meanings:

   \begin {description}
      \item {\tt SAP} --- IUE small aperture.
      \item {\tt LAP} --- IUE large aperture.
   \end {description}

   Other apertures may also be defined.
}

\iueparameter{APERTURES}
{
   \_CHAR
}{
   This specifies an aperture or group of apertures.
   The following three names have defined meanings:

   \begin {description}
      \item {\tt SAP} --- IUE small aperture.
      \item {\tt LAP} --- IUE large aperture.
      \item {\tt BAP} --- IUE both apertures ({\it{i.e.}}\ SAP and LAP together).
   \end {description}
}

\iueparameter{AUTOSLIT}
{
   \_LOGICAL
}{
   This determines whether the extraction slit is determined automatically
   by the command.
   When \verb+AUTOSLIT=TRUE+ the \verb+GSLIT+, \verb+BDIST+ and \verb+BSLIT+
   parameter values are
   determined automatically, based on the IUE camera, resolution, aperture,
   and on the values of the \verb+EXTENDED+ and \verb+CONTINUUM+ parameters.
   This mode of operation is probably the best for point source objects.

   This parameter has a default value of \verb+TRUE+\@.
}

\iueparameter{BADITF}
{
   \_LOGICAL
}{
   This parameter determines whether a correction is made to the pixel
   intensities to account for errors during Ground Station ITF
   calibration.
   (Note that the best scientific results would be obtained
   from reprocessed data which can be obtained on request.)
   The following case is handled:

   \begin {description}
      \item SWP, LORES --- correction of 2nd (faulty) ITF.
   \end {description}
}

\iueparameter{BDIST}
{
   [\_DOUBLE\{,\_DOUBLE\}]
}{
   This is a pair of numbers delineating the background spectrum channel
   positions during spectrum extraction.
   The distances are measured in geometric pixels from the spectrum centre.

   Negative distances mean ``to the left of centre'', and positive distances
   mean ``to the right of centre''.

   If only one value is defined, then this is taken as meaning
   that the channels are positioned symmetrically
   about centre.

   The spectrum ``centre'' is determined by the dispersion relations, and
   modified by any prevailing centroid shifts.
}

\iueparameter{BKGAV}
{
   \_DOUBLE
}{
   This is the FWHM of a triangle function filter used in folding the
   pixel intensities to form the smooth background spectrum.
   It is measured in geometric pixels.

   This parameter has a default value of 30.0.
}

\iueparameter{BKGIT}
{
   \_DOUBLE
}{
   This is the number of background smoothing iterations performed during
   spectrum extraction.

   \begin {description}
      \item \verb+BKGIT=0+ means that the background is taken as the result of
      the first pass of a triangle function filter with a FWHM defined by
      the \verb+BKGAV+ parameter.

      \item \verb+BKGIT=1+ means that, after producing the initial estimate for
      the smooth background, pixels discrepant by more that \verb+BKGSD+
      standard deviations are marked as ``spikes''.
      The smooth background is then re-evaluated, missing out these marked
      pixels.
   \end {description}

   Higher values of \verb+BKGIT+ are possible, but seldom necessary.

   This parameter has a default value of 1.
}

\iueparameter{BKGSD}
{
   \_DOUBLE
}{
   This is the discrimination level, measured in standard deviations,
   beyond which background pixels are marked as ``spikes''.
   It is not used for \verb+BKGIT=0+.

   This parameter has a default value of 2.0.
}

\iueparameter{BSLIT}
{
   [\_DOUBLE\{,\_DOUBLE\}]
}{
   This defines the half width of each background channel, measured
   in geometric pixels.
   A single value means that both channels have the same width.
}

\iueparameter{CAMERA}
{
   \_CHAR
}{
   This is the camera name.
   The following are defined:

   \begin {quote}
   \begin {description}
      \item {\tt LWP} --- IUE long wavelength prime.
      \item {\tt LWR} --- IUE long wavelength redundant.
      \item {\tt SWP} --- IUE short wavelength prime.
   \end {description}
   \end {quote}
}

\iueparameter{CENAV}
{
   \_DOUBLE
}{
   This is the FWHM of a triangle function filter used in folding the
   pixel intensities to form the smooth spectrum centroid positions.
   It is measured in geometric pixels.

   This parameter has a default value of 30.0.
}

\iueparameter{CENIT}
{
   \_INTEGER
}{
   This is the number of spectrum centroid tracking iterations performed during
   spectrum extraction.

   \begin {description}
      \item \verb+CENIT=0+ means that the spectrum position is taken directly
      from the dispersion relations.

      \item \verb+CENIT=1+ means that the spectrum position is first taken
      from the dispersion relations, but is modified to force it to
      follow the spectrum centroid.
   \end {description}

   Higher values of \verb+CENIT+ are possible, but seldom necessary:  it either
   works or fails.

   This parameter has a default value of 1.
}

\iueparameter{CENSD}
{
   \_DOUBLE
}{
   This is the discrimination level, measured in standard deviations,
   below which object signal is not considered significant enough
   to be used to determine the centroid position.
   It is not used for \verb+CENIT=0+\@.

   This parameter has a default value of 4.0.
}

\iueparameter{CENSH}
{
   \_LOGICAL
}{
   This indicates whether the spectrum signal produces a single linear
   shift to the initial template.

   This can be used in cases where the object signal is too weak
   to provide a detailed centroid determination by moving a pre-existing
   template shape into the right position.

   This parameter has a default value of \verb+FALSE+\@.
}

\iueparameter{CENSV}
{
   \_LOGICAL
}{
   This indicates whether the spectrum template, as refined by the
   object centroid during spectrum extraction, is saved in the calibration
   dataset.

   The primary use of this facility is in determining templates from,
   say, the whole spectrum using \verb+TRAK+, and subsequently using these
   with \verb+LBLS+, or another spectrum.

   This parameter has a default value of \verb+FALSE+\@.
}

\iueparameter{CENTM}
{
   \_LOGICAL
}{
   This indicates whether a centroid template from the calibration dataset
   is used as a start in defining the precise position of the spectrum
   signal on the image.

   This parameter has a default value of \verb+FALSE+\@.
}

\iueparameter{CENTREWAVE}
{
   [\_DOUBLE[,\_DOUBLE\ldots ]]
}{
   These are the laboratory wavelengths of a set of absorption features in
   the spectrum to be used to estimate a value for the \verb+ESHIFT+
   parameter.
}

\iueparameter{COL}
{
   \_INTEGER
}{
   This specifies the line colour to be used for the next curve to be
   plotted.
   It can be an integer in the range 1 to 10, and the corresponding
   colours are as follows:

   \begin {quote}
   \begin {description}
      \item {\tt 1} --- Yellow.
      \item {\tt 2} --- Green.
      \item {\tt 3} --- Red.
      \item {\tt 4} --- Blue.
      \item {\tt 5} --- Pink.
      \item {\tt 6} --- Violet.
      \item {\tt 7} --- Turquoise.
      \item {\tt 8} --- Orange.
      \item {\tt 9} --- Light green.
      \item {\tt 10} --- Olive.
   \end {description}
   \end {quote}

   Lines will only appear with different colours it the device supports colour
   graphics, on other devices \verb+COL+ is ignored.
}

\iueparameter{COLOUR}
{
   \_LOGICAL
}{
   Whether a spectrum-style false colour look-up table is used by
   \verb+DRIMAGE+\@.
   If \verb+FALSE+ a greyscale is used.

   The default is to use a greyscale.
}

\iueparameter{COLROT}
{
   \_LOGICAL
}{
   This indicates whether the line colour is to be changed after the next plot.
   The initial line has colour index 1 (YELLOW), unless specified explicitly
   using the \verb+COL+ parameter.
   The sequence of colour indices goes (1, 2, 3, \ldots 10, 1, 2, \ldots).

   In commands where more than one line is plotted, \verb+COLROT+ determines
   whether these lines have different colours.

   Lines will only appear with different colours if the device supports
   colour graphics; on other devices \verb+COLROT+ is harmless.

   This parameter has a default value of \verb+TRUE+\@.
}

\iueparameter{CONTINUUM}
{
   \_LOGICAL
}{
   This indicates whether the object spectrum is expected to contain a
   significant continuum.
   It is used in conjunction with the \verb+EXTENDED+ parameter in determining
   the positions and widths of object and background channels for
   spectrum extraction from HIRES datasets.

   This parameter has a default value of \verb+TRUE+\@.
}

\iueparameter{COVERGAP}
{
   \_LOGICAL
}{
   If after mapping an order/aperture, a grid point is marked as unusable,
   then this parameter determines whether other orders/apertures
   can be allowed to contribute to this grid point.

   This parameter has a default value of \verb+FALSE+\@.
}

\iueparameter{CUTFILE}
{
   \_CHAR
}{
   This is the name of a file containing an \'{e}chelle order
   wavelength limits.
   A file type of \verb+.cut+ is assumed and need not be specified
   explicitly.
   If the file name contains a directory specification, then it should be
   enclosed in quotes.
}

\iueparameter{CUTWV}
{
   \_LOGICAL
}{
   This indicates whether any available \'{e}chelle order wavelength cutoff
   limits are to be used for the spectrum extraction wavelength grid
   limits.
   Highly recommended, provided that you are happy with these wavelength limits.

   This parameter has a default value of \verb+TRUE+\@.
}

\iueparameter{DATASET}
{
   \_CHAR
}{
   This is the root name of the files containing the dataset.
   The file type ({\it{e.g.,}}\ \_\verb+UED.sdf+) should {\bf not} be given in
   the \verb+DATASET+ name.
   If the file name contains a directory specification, then it
   should be enclosed in quotes.

   Note that the actual filenames contain additional characters
   to define their contents ({\it{e.g.,}}\ \verb+LWP12345+\_\verb+UES.sdf+,
   contains spectral data).
}

\iueparameter{DAY}
{
   \_INTEGER
}{
   This is the day number, measured from the start of the month, used
   for constructing dates.
   The \verb+DAY+, \verb+MONTH+ and \verb+YEAR+ parameters refer to the date
   the IUE observations were made and are important to the calibration of the
   data.
}

\iueparameter{DELTAWAVE}
{
   [\_DOUBLE[,\_DOUBLE\ldots]\,]
}{
   The half-width of the window used to search for an absorption line feature
   for wavelength calibration in Angstroms.  If more than one line is being
   used then each may be given a different search window width.
}

\iueparameter{DEVICE}
{
   \_CHAR
}{
   This is the GKS/SGS graphics device.
   A list of possible GKS workstations may be found in
   \xref{SUN/83}{sun83}{}.
   A list of SGS workstation names available at your
   site may be obtained either by a null response to the \verb+DEVICE+ parameter
   prompt, {\it{i.e.}}\ \verb+!+, or by using the IUEDR Command \verb+SGS+\@.
}

\iueparameter{DISPFILE}
{
   \_CHAR
}{
   This is the name of a file containing dispersion data.
   A file type of \verb+.dsp+ is assumed and need not be specified
   explicitly.
   If the file name contains a directory specification, then it should be
   enclosed in quotes.
}

\iueparameter{DRIVE}
{
   \_CHAR
}{
   This is the name of the tape drive. Feasible value
   are \verb+/dev/nrmt0h+ on a UNIX machine and \verb+MSA0+ on VMS.

   This version of IUEDR supports the direct reading of
   IUEDR data from disk files which have the same format as those on
   GO format tapes.

   In order to read directly from such a file (probably grabbed from
   an on-line archive such as NDADSA), you specify its name directly in
   response to the \verb+DRIVE+ parameter prompt.

   If the file is not in the current directory then you must provide
   the full pathname.
}

\iueparameter{ESHIFT}
{
   \_DOUBLE
}{
   This is a global wavelength shift applied to the wavelengths in
   \'{e}chelle spectral orders.
   It is measured in Angstroms, and affects the spectrum wavelengths as follows:

   \begin {equation}
      \lambda _{new} = \lambda _{old} + \frac{\rm ESHIFT}{\rm ORDER}
   \end {equation}

   This is designed to account for wavelength errors that result
   from a global linear shift of the spectrum format on the
   image.
}

\iueparameter{EXPOSURE}
{
   \_DOUBLE
}{
   This is the exposure time associated with the spectrum, measured in seconds.
   If there is more than one aperture, then this time applies
   to that specified by the \verb+APERTURE+ parameter.
}

\iueparameter{EXPOSURES}
{
   [\_DOUBLE\{,\_DOUBLE\}]
}{
   This is one or more exposure times associated with the
   spectrum, measured in seconds.
   There is an exposure time for each aperture defined.
}

\iueparameter{EXTENDED}
{
   \_LOGICAL
}{
   This indicates whether the object spectrum is expected to be extended,
   rather than a point source.
   It is used in conjunction with the \verb+CONTINUUM+ parameter in determining
   the positions and widths of the object and background channels used
   for spectrum extraction from HIRES datasets.

   This parameter has a default value of \verb+FALSE+\@.
}

\iueparameter{FIDFILE}
{
   \_CHAR
}{
   This is the name of a file containing fiducial positions.
   A file type of \verb+.fid+ is assumed and need not be specified
   explicitly.
   If the file name contains a directory specification, then it should be
   enclosed in quotes.
}

\iueparameter{FIDSIZE}
{
   \_DOUBLE
}{
   This is the half width of a fiducial measured in pixel units. The fiducials
   are considered to be square.
}

\iueparameter{FILE}
{
   \_INTEGER
}{
   This is the tape file number.
   The first file on a tape would be \verb+FILE=1+\@.
   One case which may present some problems is
   that of a tape with a an end-of-volume (EOV) mark in the middle
   and with valuable data beyond.
   An EOV is two consecutive tape marks (sometimes called ``file marks'').
   A file is defined here as the information between two tape marks.
   So if the number for a real file before EOV is FILEN, then
   the number of the next real file following the EOV is (FILEN+2).
}

\iueparameter{FILLGAP}
{
   \_LOGICAL
}{
   If a grid point in the mean spectrum would have had a contribution
   from a bad data point, this parameter determines whether that
   grid point is marked as unusable within the context
   of the order or aperture being mapped.
   If the grid point is marked as unusable in this way then other
   good points cannot contribute to it.

   This parameter has a default value of \verb+FALSE+\@.
}

\iueparameter{FLAG}
{
   \_LOGICAL
}{
   This specifies whether the data quality information is displayed
   along with the image.
   If so, then faulty pixels will be marked with a colour according to
   the following scheme:

   \begin {quote}
   \begin {description}
      \item Green --- pixels affected by reseau marks.
      \item Red --- pixels which are saturated (DN = 255).
      \item Orange --- pixels affected by ITF truncation.
      \item Yellow --- pixels marked bad by the user.
   \end {description}
   \end {quote}

   If a pixel is affected by more than one of the above faults, then
   the first in the list is adopted for display.
   Hence, user edits are only shown where no other fault is present.

   This option would normally only be used when assessing the quality
   of faulty pixels, possibly with a view to using them, {\it{i.e.}}\ marking
   them ``good'' with a cursor editor.

   This parameter has a default value of \verb+TRUE+\@.
}

\iueparameter{FN}
{
   \_DOUBLE
}{
   This parameter is the replacement Flux Number for a pixel changed
   explicitly by the user.
}

\iueparameter{FSCALE}
{
   \_DOUBLE
}{
   This is an arbitrary scale factor applied to spectrum fluxes.
   It affects fluxes as follows:

   \begin {equation}
      {\cal F}_{new} = {\cal F}_{old} \times {\rm FSCALE}
   \end {equation}

   It finds application in accounting for grey attenuation, or obscuration
   of object signal through a narrow aperture.
}

\iueparameter{GSAMP}
{
   \_DOUBLE
}{
   This is the sampling rate used for spectrum extraction.
   It is measured in geometric pixels.
   \verb+GSAMP=1.414+ corresponds to the IUESIPS\#1 sampling
   rate, while \verb+GSAMP=0.707+ corresponds to the IUESIPS\#2 sampling
   rate.
   Other values can be chosen.

   This parameter has a default value of 1.414.
}

\iueparameter{GSHIFT}
{
   [\_DOUBLE,\_DOUBLE]
}{
   This is a global constant shift of the spectrum format on the image,
   $(dx,dy)$, where the geometric coordinates, $(x,y)$ of a spectrum position
   are

   \begin {equation}
      x_{new} = x_{old} + dx
   \end {equation}

   and

   \begin {equation}
      y_{new} = y_{old} + dy
   \end {equation}
}

\iueparameter{GSLIT}
{
   [\_DOUBLE\{,\_DOUBLE\}]
}{
   This is a pair of numbers delineating the object spectrum channel
   during spectrum extraction.
   The distances are measured in geometric pixels.

   Negative distances mean ``to the left of centre'', and positive distances
   mean ``to the right of centre''.

   Object channels that do not cover the actual object signal on the
   image will not be meaningful when centroid tracking is employed.

   If only one value is defined, then this is taken as representing
   a channel that is symmetrical about the spectrum centre.

   The spectrum ``centre'' is determined by the dispersion relations
   modified by any prevailing centroid shifts.
}

\iueparameter{HALAV}
{
   \_DOUBLE
}{
   This is the FWHM of a triangle function used for smoothing the
   net spectrum for the \verb+HALTYPE=POWER+ halation correction technique.
}

\iueparameter{HALC}
{
   \_DOUBLE
}{
   This is the Halation correction constant used for \verb+HALTYPE=POWER+
   cases, and defined at wavelength \verb+HALWC+\@.
   The value of the correction constant
   corresponds roughly to the measured depression of a broad
   zero intensity absorption below zero, in units
   of the continuum in adjacent orders.
   The ``constant'', $C$, varies with wavelength as follows:

   \begin {equation}
      C_\lambda = \frac{{\rm HALC} \times (\lambda - {\rm HALW0})}
                       {({\rm HALWC} - {\rm HALW0})}
   \end {equation}

   See the \verb+HALTYPE+, \verb+HALWC+, \verb+HALW0+ and \verb+HALAV+
   parameters.
}

\iueparameter{HALTYPE}
{
   \_CHAR
}{
   This is the type of Halation or order-overlap correction applied to the
   flux spectrum.
   It can take the value

   \begin {quote}
   \begin {description}
      \item {\tt POWER} --- correction based on power-law PSF decay.
   \end {description}
   \end {quote}
}

\iueparameter{HALW0}
{
   \_DOUBLE
}{
   This is the wavelength, measured in Angstroms, at which the
   halation correction is zero.

   See the \verb+HALTYPE+, \verb+HALC+ and \verb+HALWC+ parameters.
}

\iueparameter{HALWC}
{
   \_DOUBLE
}{
   This is the wavelength, measured in Angstroms, at which the
   halation correction is \verb+HALC+\@.

   See the \verb+HALTYPE+, \verb+HALC+ and \verb+HALW0+ parameters.
}

\iueparameter{HIST}
{
   \_LOGICAL
}{
   This determines whether lines are plotted as histograms rather than
   continuous lines (polylines).

   This parameter has a default value of \verb+TRUE+\@.
}

\iueparameter{IMAGE}
{
   \_INTEGER
}{
   This is the Image Sequence Number.
}

\iueparameter{ITF}
{
   \_INTEGER
}{
   This is the ITF generation used in the photometric calibration of the
   image.  This information is needed for the correct absolute flux
   calibration of the resulting spectra.  Possible values for each camera
   are as follows:

   \begin {quote}
   \begin {description}
      \item SWP --- 2
      \item LWR --- 1 and 2
      \item LWP --- 1 and 2
   \end {description}
   \end {quote}

   The appropriate value can be determined from inspection of the
   IUE header text for the GPHOT/PHOT file using the table
   of numbers following the line beginning \verb+PCF C/**+\@.
   Here are the \verb+ITF+ values associated with various forms of this
   table:

   \begin {quote}
   \begin {tabbing}
   ITFMAXxxx\= 0xx\= 1800xx\= 3700xx\= 5600xx\= ...xx\= 30000xxx\= (Corrected, 3rd SWP ITF)\kill
   {\bf ITF}\> \> \>{\bf TABLE}\> \> \> \>{\bf IDENTIFICATION}\\
   \\
   ITF 0\>0\>1800\>3700\>5600\>...\>30000\>Preliminary LWR ITF\\
   ITF 1\>0\>2303\>4069\>8008\>...\>42032\>2nd LWR ITF\\
   ITF 1\>0\>2300\>3969\>6062\>...\>32973\>1st LWP ITF\\
   ITF 2\>0\>2723\>5429\>8145\>...\>38389\>2nd LWP ITF\\
   ITF 0\>0\>1800\>3600\>5500\>...\>\>Preliminary SWP ITF\\
   ITF 1\>0\>1753\>3461\>6936\>...\>28674\>Faulty, 2nd SWP ITF\\
   ITF 2\>0\>1684\>3374\>6873\>...\>28500\>Corrected, 3rd SWP ITF\\
   \end {tabbing}
   \end {quote}

   If the ITF table used has no corresponding absolute flux calibration within
   IUEDR, {\it{e.g.,}}\ LWR ITF0 or SWP ITF0, you are advised to contact the IUE
   Project.
   Although the \verb+BADITF+ parameter is available for data calibrated using
   SWP ITF1, it is advisable to have these data reprocessed by the IUE Project.
}

\iueparameter{ITFMAX}
{
   \_INTEGER
}{
   This is the pixel value on tape corresponding to ITF saturation.
   Its value is fixed for a given ITF table.
   The value of \verb+ITFMAX+ is only needed for IUE images of type GPHOT.
   The appropriate value can be determined from inspection of the
   IUE header text for the GPHOT file using the table
   of numbers following the line beginning \verb+PCF C/**+.

   \begin {quote}
   \begin {tabbing}
   ITFMAXxxx\= 0xx\= 1800xx\= 3700xx\= 5600xx\= ...xx\= 30000xxx\= (Corrected, 3rd SWP ITF)\kill
   {\bf ITFMAX}\>\>\>{\bf TABLE}\>\>\>\>{\bf IDENTIFICATION}\\
   \\
   20000\>0\>1800\>3700\>5600\>...\>30000\>Preliminary LWR ITF\\
   27220\>0\>2303\>4069\>8008\>...\>42032\>2nd LWR ITF\\
   19983\>0\>1800\>3600\>5500\>...\>\>Preliminary SWP ITF\\
   19740\>0\>1753\>3461\>6936\>...\>28674\>Faulty, 2nd SWP ITF\\
   19632\>0\>1684\>3374\>6873\>...\>28500\>Corrected, 3rd SWP ITF\\
   \end {tabbing}
   \end {quote}
}

\iueparameter{LINE}
{
   \_CHAR
}{
   This specifies the line style to be used for the next curve to be
   plotted.
   It can be one of the following:

   \begin {quote}
   \begin {description}
      \item {\tt SOLID} --- solid (continuous) line.
      \item {\tt DASH} --- dashed line.
      \item {\tt DOTDASH} --- dot-dash line.
      \item {\tt DOT} --- dotted line.
   \end {description}
   \end {quote}

   The order of these is that invoked when automatic line style rotation
   is in effect (see the \verb+LINEROT+ parameter).
}

\iueparameter{LINEROT}
{
   \_LOGICAL
}{
   This indicates whether the line style is to be changed after the
   next plot.
   The initial line style is \verb+SOLID+, unless specified explicitly
   using the \verb+LINE+ parameter.
   The sequence of line styles goes (\verb+SOLID+, \verb+DASH+, \verb+DOTDASH+,
   \verb+DOT+, \verb+SOLID+, \verb+DASH+\dots).

   In commands where more than one line is plotted, \verb+LINEROT+ determines
   whether these lines have different styles.

   This parameter has a default value of \verb+FALSE+\@.
}

\iueparameter{ML}
{
   [\_DOUBLE,\_DOUBLE]
}{
   This is a pair of wavelength values defining the start and end of
   the mean spectrum grid.
   The grid will consist of evenly spaced vacuum wavelengths between these
   values.
}

\iueparameter{MONTH}
{
   \_INTEGER
}{
   This is the month number, measured from the start of the Year,
   used in constructing dates.
}

\iueparameter{MSAMP}
{
   \_DOUBLE
}{
   This is the vacuum wavelength sampling rate for the mean spectrum grid.
   If it does not fit an integral number of times into the grid limits,
   then the latter are adjusted to fit.
}

\iueparameter{NFILE}
{
   \_INTEGER
}{
   This is the number of tape files to be processed.
   A value of \verb+NFILE=-1+ means all files until the end.

   This parameter has a default value of 1.
}

\iueparameter{NGEOM}
{
   \_INTEGER
}{
   This is the number of Chebyshev terms used to represent the
   geometrical distortion.
   The same value is used for each axis direction.
}

\iueparameter{NLINE}
{
   \_INTEGER
}{
   This is the number of IUE header lines printed.
   A value of \verb+NLINE=-1+ means the entire header is printed.

   This parameter has a default value of 10.
}

\iueparameter{NORDER}
{
   \_INTEGER
}{
   This is the number of \'{e}chelle orders to be processed by a command.

   This parameter has a default value of 0.
}

\iueparameter{NSKIP}
{
   \_INTEGER
}{
   This is the number of tape marks to be skipped.

   This parameter has a default value of 1.
}

\iueparameter{OBJECT}
{
   \_CHAR
}{
   This is a string containing text to identify the object of the
   observation.
   It can also contain information about the observation
   ({\it{e.g.,}}\ camera, image\ldots ) if required.
   The maximum allowed length of the string is 40 characters.
}

\iueparameter{ORDER}
{
   \_INTEGER
}{
   This is the \'{e}chelle order number.
}

\iueparameter{ORDERS}
{
   [\_INTEGER\{,\_INTEGER\}]
}{
   This is a pair of \'{e}chelle order numbers delineating a range.
   If only a single value is specified, then the range consists of that
   order only.
   The sequence of the two numbers is not significant.
   The useful maximum range for each camera is as follows:

   \begin {quote}
   \begin {description}
      \item SWP --- orders 66 to 125.
      \item LWR --- orders 72 to 125.
      \item LWP --- orders 72 to 125.
   \end {description}
   \end {quote}
}

\iueparameter{OUTFILE}
{
   \_CHAR
}{
   This is the name of a file to receive the output spectrum.
   This release of IUEDR uses the STARLINK NDF format for all output
   spectra. This means that all standard STARLINK packages can be used
   to plot/display/analyse the spectra, in particular some of the
   facilities of KAPPA and FIGARO may prove useful to the general user.
}

\iueparameter{QUAL}
{
   \_LOGICAL
}{
   This specifies whether the data quality information is plotted
   along with the data.
   If so, then faulty points will be marked with their data quality
   severity code, which is one from:

   \begin {quote}
   \begin {description}
      \item 1 --- affected by extrapolated ITF.
      \item 2 --- affected by microphonics.
      \item 3 --- affected by spike.
      \item 4 --- affected by bright point (or user).
      \item 5 --- affected by reseau mark.
      \item 6 --- affected by ITF truncation.
      \item 7 --- affected by saturation.
      \item U --- affected by user edit.
   \end {description}
   \end {quote}

   User edits are only shown where no other fault is present.

   This option would normally only be used when assessing the quality
   of faulty points, possibly with a view to using them, {\it{i.e.}}\ marking
   them ``good'' with a cursor editor.

   This parameter has a default value of \verb+TRUE+\@.
}

\iueparameter{RESOLUTION}
{
   \_CHAR
}{
   This is the spectrograph resolution mode.
   The following modes are defined:

   \begin {quote}
   \begin {description}
      \item {\tt LORES} --- IUE Low Resolution.
      \item {\tt HIRES} --- IUE High Resolution (\'{e}chelle mode).
   \end {description}
   \end {quote}
}

\iueparameter{RIPA}
{
   \_DOUBLE
}{
   This is an empirical scale factor that can be used to modify the
   \'{e}chelle ripple function.
   The normal value is 1.0.
   The primary component of the ripple function is

   \begin {equation}
      {\rm SCALE} = (\frac{\sin x}{x})^2
   \end {equation}

   where

   \begin {equation}
      x = \frac{\pi \times {\rm RIPA} \times (\lambda - \lambda_c)
                \times {\rm ORDER}}
               {\lambda_c}
   \end {equation}

   and

   \begin {equation}
      \lambda_c = \frac{\rm RIPK}{\rm ORDER}
   \end {equation}

   The net spectrum is divided by SCALE above.
   Empirical values of \verb+RIPA+ can be used to optimise the ripple
   correction.

   See the \verb+RIPC+ and \verb+RIPK+ parameter descriptions.
}

\iueparameter{RIPC}
{
   [\_DOUBLE\{,\_DOUBLE\}]
}{
   This is a polynomial in $x$ used to modify the standard \'{e}chelle
   ripple calibration function.
   The calibration is given by

   \begin {equation}
      {\rm SCALE} = (\frac{\sin x}{x})^2 \times (
                {\rm RIPC}(1) + {\rm RIPC}(2) \times x +
                {\rm RIPC}(3) \times x^2 \ldots)
   \end {equation}

   where

   \begin {equation}
      x = \frac {\pi \times {\rm RIPA} \times (\lambda - \lambda_c) \times
                 {\rm ORDER}}
                {\lambda_c}
   \end {equation}

   and

   \begin {equation}
      \lambda_c = \frac {\rm RIPK}{\rm ORDER}
   \end {equation}

   The net spectrum is divided by SCALE above.

   See the \verb+RIPA+ and \verb+RIPK+ parameter descriptions.
}

\iueparameter{RIPFILE}
{
   \_CHAR
}{
   This is the name of a file containing an \'{e}chelle ripple calibration.
   A file type of \verb+.rip+ is assumed and need not be specified
   explicitly.
   If the file name contains a directory specification, then it should be
   enclosed in quotes.
}

\iueparameter{RIPK}
{
   [\_DOUBLE\{,\_DOUBLE\}]
}{
   This is the \'{e}chelle ripple constant measured in Angstroms.
   It corresponds to the central wavelength of \'{e}chelle order number 1.
   The central wavelength of an arbitrary ORDER is

   \begin {equation}
      \lambda_c = \frac {\rm RIPK}{\rm ORDER}
   \end {equation}

   Where this parameter is used for an entire HIRES dataset, the
   parameter can have more than one value, and represent a polynomial
   in ORDER

   \begin {equation}
      \lambda_c =  \frac{({\rm RIPK}(1) + {\rm RIPK}(2)
                         \times {\rm ORDER} + {\rm RIPK}(3)
                         \times {\rm ORDER}^2 + \ldots)}
                        {\rm ORDER}
   \end {equation}
}

\iueparameter{RL}
{
   [\_DOUBLE,\_DOUBLE]
}{
   This is a pair of radial coordinate values defining the start and end of
   the radial grid in an LBLS array.
   These radial coordinates are measured in geometric pixels, and run
   perpendicular to the dispersion direction.
   A coordinate value of 0.0 corresponds to the centre of the spectrum.

   Values \verb+RL=[0.0, 0.0]+ indicate that internal defaults are to be
   adopted.
   A single value is reflected symmetrically about 0.0.
}

\iueparameter{RM}
{
   \_LOGICAL
}{
   This determines whether the mean spectrum is reset before a mapping
   takes place.
   If the spectrum is not reset, then the spectra being mapped will be
   averaged with the existing mean spectrum.

   This parameter has a default value of \verb+TRUE+\@.
}

\iueparameter{RS}
{
   \_LOGICAL
}{
   This determines whether the display screen is reset before plotting.

   This parameter has a default value of \verb+TRUE+\@.
}

\iueparameter{RSAMP}
{
   \_DOUBLE
}{
   This is the sample spacing used for the radial grid in the LBLS array.
   If it does not fit an integral number of times into the grid limits,
   \verb+RL+, then the latter are adjusted to fit.

   Suggested values range from 0.707 to 1.414 pixels, the latter
   corresponding to the IUESIPS LBLS grid.

   This parameter has a default value of 1.414.
}

\iueparameter{SCANAV}
{
   \_DOUBLE
}{
   This is the HWHM of a triangle function with which pixels are folded
   during the generation of a scan across the image perpendicular to
   spectrograph dispersion.
   It is measured in geometric pixels.

   This parameter has a default value of 5.
}

\iueparameter{SCANDIST}
{
   \_DOUBLE
}{
   This is the distance of a scan across a HIRES image from the faceplate
   centre.
   It is measured in geometric pixels.
}

\iueparameter{SCANWV}
{
   \_DOUBLE
}{
   This is the central wavelength for a scan of a LORES image
   perpendicular to spectrograph dispersion.
   It is measured in Angstroms in vacuo.
}

\iueparameter{SKIPNEXT}
{
   \_LOGICAL
}{
   This determines whether the tape is positioned at the start of the
   next file after processing.
   If only the start of a file is being processed, then by setting
   \verb+SKIPNEXT=FALSE+ time can be saved.

   This parameter has a default value of \verb+FALSE+\@.
}

\iueparameter{SPECTYPE}
{
   \_INTEGER
}{
   This is the type of file, in the DIPSO SP format terminology, to be
   created. The following values are allowed:

   \begin {quote}
   \begin {description}
      \item {\tt 0} --- Starlink NDF format file.
      \item {\tt 1} --- SP1, fixed format text file.
      \item {\tt 2} --- SP2, free format text file.
   \end {description}
   \end {quote}

   It is recommended that datasets with many points be written with
   \verb+SPECTYPE=0+, which is more efficient in disk space and time spent
   reading and writing.
   A description of the format of each of these file types can be found
   in Section~\ref{se:spectrum}\@.

   This parameter has a default value of 0.
}

\iueparameter{TEMFILE}
{
   \_CHAR
}{
   This is the name of a file containing the standard spectrum template
   data.
   A  file type of \verb+.tem+ is assumed and need not be specified
   explicitly.
   If the file name contains a directory specification, then it should be
   enclosed in quotes.
}

\iueparameter{THDA}
{
   \_DOUBLE
}{
   This is the IUE camera temperature, measured in degrees Centigrade.
   It is used for such things as adjustments to fiducial positions
   and spectrograph dispersion relations.
   A value of 0.0 implies that no \verb+THDA+ value is available, the program
   will then  use a suitable mean \verb+THDA+ for the camera being used.
   Values for the \verb+THDA+ can be found in the IUE header text of the final
   spectrum file on the Guest Observer tape for IUESIPS\#2 ---
   \verb+THDA+ values derived from spectrum motion are best.
}

\iueparameter{THRESH}
{
   \_DOUBLE
}{
   This is the  minimum value considered to be good when using \verb+CLEAN+\@.
   All pixels with values below this threshold will be marked BAD.
}

\iueparameter{TYPE}
{
   \_CHAR
}{
   This is the type of dataset.
   Defined values are as follows:

   \begin {quote}
   \begin {description}
      \item {\tt RAW} --- IUE raw image.
      \item {\tt GPHOT} --- IUE GPHOT image (geometric and photometric).
      \item {\tt PHOT} --- IUE PHOT image (photometric only).
   \end {description}
   \end {quote}

   Types \verb+PHOT+ and \verb+GPHOT+ are not automatically distinguishable
   from IUE Guest Observer tape contents.
}

\iueparameter{V}
{
   \_CHAR
}{
   This is a string defining a list of items and includes
   any of the following characters:

   \begin {quote}
   \begin {description}
      \item {\tt H} --- header and file information.
      \item {\tt I} --- image details.
      \item {\tt F} --- fiducials.
      \item {\tt G} --- geometry.
      \item {\tt D} --- dispersion.
      \item {\tt C} --- centroid templates.
      \item {\tt R} --- \'{e}chelle ripple and halation.
      \item {\tt A} --- absolute calibration.
      \item {\tt S} --- raw spectrum.
      \item {\tt M} --- mean spectrum.
      \item {\tt *} --- all of the above.
      \item {\tt Q} --- image data quality summary.
   \end {description}
   \end {quote}
}

\iueparameter{VSHIFT}
{
   \_DOUBLE
}{
   This is the radial velocity of the detector relative to the Sun.
   It is measured in km/s and affects the calibrated wavelength
   scale as follows:

   \begin {equation}
      \lambda_{true} = \frac{\lambda_{obs}}
                            {(1 + \frac{\rm VSHIFT}{c})}
   \end {equation}

   where $c$ is the velocity of light in km/s.
}

\iueparameter{WCUT}
{
   [\_DOUBLE,\_DOUBLE]
}{
   This is one of the mechanisms that can be used to delimit the
   parts of \'{e}chelle orders that are calibrated for ripple response.
   The two values of this parameter are the start and end wavelengths
   for a specific \verb+ORDER+\@.

   Apart from poor ripple calibration, the order ends can also be affected
   by the parts of the camera faceplate that are retained in the image.
}

\iueparameter{WSHIFT}
{
   \_DOUBLE
}{
   This is a constant wavelength shift applied to spectrum wavelengths.
   It is measured in Angstroms and affects the spectrum wavelengths as follows:

   \begin {equation}
      \lambda_{new} = \lambda_{old} + {\rm WSHIFT}
   \end {equation}

   This is only used for LORES spectra.
}

\iueparameter{XCUT}
{
   [\_DOUBLE,\_DOUBLE]
}{
   This is one of the mechanisms that can be used to delimit the
   ends of \'{e}chelle orders that are calibrated for ripple response.
   The two values of this parameter are the start and end $x$ coordinates
   of the order, where

   \begin {equation}
      x = \frac{\pi \times {\rm RIPA} \times (\lambda - \lambda_c)
                \times {\rm ORDER}}
               {\lambda_c}
   \end {equation}

   and

   \begin {equation}
      \lambda_c = \frac {\rm RIPK}{\rm ORDER}
   \end {equation}

   The nature of the standard ripple function is that $x$ is only
   formally meaningful in the range ($-\pi$, $+\pi$).

   See the \verb+RIPA+, \verb+RIPC+ and \verb+RIPK+ parameter descriptions.
}

\iueparameter{XL}
{
   [\_DOUBLE,\_DOUBLE]
}{
   This specifies the data limits used for plotting in the $x$-direction.
   This parameter is only read if the display has been reset, and
   the axes are being redrawn.
   If both values are the same ({\it{e.g.,}}\ \verb+[0, 0]+),
   then the data limits in the $x$-direction will be determined from the
   data being plotted.
}

\iueparameter{XP}
{
   [\_INTEGER,\_INTEGER]
}{
   This specifies the pixel limits along the $x$-direction used for
   image display.
   Values in decreasing order will cause the image to be inverted
   along the $x$-direction.
   If the values are undefined, the pixel limits will default to include
   the whole extent of the image along the $x$-direction.
}

\iueparameter{YEAR}
{
   \_INTEGER
}{
   This is the year (A.D.) used in constructing dates.
}

\iueparameter{YL}
{
   [\_DOUBLE,\_DOUBLE]
}{
   This specifies the data limits used for plotting in the $y$-direction.
   This parameter is only read if the display has been reset, and
   the axes are being redrawn.
   If both values are the same ({\it{e.g.,}}\ \verb+[0, 0]+),
   then the data limits in the $y$-direction will be determined from the
   data being plotted.
}

\iueparameter{YP}
{
   [\_INTEGER,\_INTEGER]
}{
   This specifies the pixel limits along the $y$-direction used for
   image display.
   Values in decreasing order will cause the image to be inverted
   along the $y$-direction.
   If the values are undefined, the pixel limits will default to include
   the whole extent of the image along the $y$-direction.
}

\iueparameter{ZL}
{
   [\_DOUBLE,\_DOUBLE]
}{
   This specifies the data limits used for display of images.
   If the values are given in decreasing order, then high data
   values will be represented by low (dark) display intensities,
   and vice-versa.
   If the values are undefined, then the full intensity range of the
   image will be used.

   Data values which fall at or below the lowest display intensity are drawn
   {\bf black,} those which are at the highest display intensity are drawn
   {\bf white} and those which are above the highest display intensity are
   drawn {\bf blue.}
}

\iueparameter{ZONE}
{
   \_INTEGER
}{
   This specifies the zone to be used for plotting.
   The zone numbers range
   from 0 to 8 and correspond to those defined by the TZONE command in
   DIPSO (see \xref{SUN/50}{sun50}{}), {\it{e.g.,}}

   \begin {quote}
   \begin {description}
      \item {\tt 0} --- entire display surface.
      \item {\tt 1} --- top left quarter.
      \item {\tt 2} --- top right quarter.
      \item {\tt 3} --- bottom left quarter.
      \item {\tt 4} --- bottom right quarter.
      \item {\tt 5} --- top half.
      \item {\tt 6} --- bottom half.
      \item {\tt 7} --- left half.
      \item {\tt 8} --- right half.
   \end {description}
   \end {quote}

   This parameter has a default value of 0.
}
}

\newpage
%------------------------------------------------------------------------------

\appendix
\section{\xlabel{parameter_defaults}\label{se:parameter_defaults}Parameter
          defaults}
\markboth{Parameter defaults}{\stardocname}

Some IUEDR parameters have default values.  Some have no default value and one
{\bf must} be provided.  Other parameters values are either calculated or
simply set by the program.  The default values and/or behaviour of each
parameter are listed here.

\begin {description}

\item [\htmlref{ABSFILE}{ABSFILE}] \lmbox
   No default value exists, a file name must be provided.
   The parameter is cancelled each time it is used.
\item [\htmlref{APERTURE}{APERTURE}] \lmbox
   Has no automatic default value.  An \verb+APERTURE+ must be selected.
   If only one is present in an image then this is taken as \verb+APERTURE+ by
   default.
\item [\htmlref{APERTURES}{APERTURES}] \lmbox
   Used only by \verb+READIUE+ and \verb+READSIPS+\@.  The value must be
   supplied by reading the IUE GO header `by eye'.
\item [\htmlref{AUTOSLIT}{AUTOSLIT} = TRUE] \lmbox
   Whether \verb+GSLIT+, \verb+BDIST+ and \verb+BSLIT+ are determined
   automatically.
\item [\htmlref{BADITF}{BADITF} (= TRUE)] \lmbox
   Has no default value, however it is recommended to be set \verb+TRUE+ as
   this will ensure any ITF error correction available for the particular
   \verb+ITF+ will be used.
   This may seem counter-intuitive, however, data using error-free ITF
   information will not be affected by setting \verb+BADITF=TRUE+\@.
\item [\htmlref{BDIST}{BDIST}] \lmbox
   No automatic default.  The value is calculated if \verb+AUTOSLIT=TRUE+,
   otherwise it should be estimated by looking at \verb+SCAN+ plots.
\item [\htmlref{BKGAV}{BKGAV} = 30.0] \lmbox
   Background averaging filter FWHM (geometric pixels).
\item [\htmlref{BKGIT}{BKGIT} = 1] \lmbox
   Number of background smoothing iterations.
\item [\htmlref{BKGSD}{BKGSD} = 2.0] \lmbox
   Discrimination level for background pixels (standard deviations).
\item [\htmlref{BSLIT}{BSLIT}] \lmbox
   No automatic default.  The value is calculated if \verb+AUTOSLIT=TRUE+,
   otherwise it should be estimated by looking at \verb+SCAN+ plots.
\item [\htmlref{CAMERA}{CAMERA}] \lmbox
   The value is read by the program from an IUE GO header.  The default value
   presented in a prompt will be the value found in the header.
\item [\htmlref{CENAV}{CENAV} = 30.0] \lmbox
   Centroid averaging filter FWHM (geometric pixels).
\item [\htmlref{CENIT}{CENIT} = 1] \lmbox
   Number of centroid tracking iterations.
\item [\htmlref{CENSD}{CENSD} = 4.0] \lmbox
   Discrimination level for signal to be used for centroids
   (standard deviations).
\item [\htmlref{CENSH}{CENSH} = FALSE] \lmbox
   Whether the spectrum template is just shifted linearly.
\item [\htmlref{CENSV}{CENSV} = FALSE] \lmbox
   Whether the spectrum template is saved in the dataset.
\item [\htmlref{CENTM}{CENTM} = FALSE] \lmbox
   Whether an existing centroid template is used.
\item [\htmlref{COL}{COL} (= 1)] \lmbox
   The value of \verb+COL+ will be calculated for each command requiring it.
   If \verb+COLROT=FALSE+ then \verb+COL+ will take the value 1, otherwise it
   will be incremented for each plotting command.
\item [\htmlref{COLOUR}{COLOUR} = FALSE] \lmbox
   Whether a spectrum-style false colour look-up table is used by
   \verb+DRIMAGE+\@.
\item [\htmlref{COLROT}{COLROT} = TRUE] \lmbox
   Whether the line colour is changed after the next plot.
\item [\htmlref{CONTINUUM}{CONTINUUM} = TRUE] \lmbox
   Whether the object spectrum is expected to have a ``continuum''.
\item [\htmlref{COVERGAP}{COVERGAP} = FALSE] \lmbox
   Whether gaps can be filled by covering orders.
\item [\htmlref{CUTFILE}{CUTFILE}] \lmbox
   No default value exists, a file name must be provided.
   The parameter is cancelled each time it is used.
\item [\htmlref{CUTWV}{CUTWV} = TRUE] \lmbox
   Whether wavelength cutoff data is to be used for the extraction grid.
\item [\htmlref{DATASET}{DATASET}] \lmbox
   No automatic default value.  The last value of \verb+DATASET+ used will be
   taken as the default.  The exception to this rule is when using
   \verb+READIUE+ or \verb+READSIPS+ when a suggested default value will be
   presented by the program in the parameter prompt.
\item [\htmlref{DAY}{DAY}] \lmbox
   No default value.  The program will attempt to extract the \verb+DAY+ from
   the IUE GO tape/file header and present this as the default value for that
   particular command.
\item [\htmlref{DEVICE}{DEVICE}] \lmbox
   No automatic default value.  The last value of \verb+DEVICE+ used will be
   taken as the default.
\item [\htmlref{DISPFILE}{DISPFILE}] \lmbox
   No default value exists, a file name must be provided.
   The parameter is cancelled each time it is used.
\item [\htmlref{DRIVE}{DRIVE}] \lmbox
   No automatic default value.  The last value of \verb+DRIVE+ used will be
   taken as the default.
\item [\htmlref{ESHIFT}{ESHIFT}] \lmbox
   No automatic default value.  If an \verb+ESHIFT+ has previously been set for
   the current \verb+DATASET+, this will be presented as the default value in
   the prompt.
\item [\htmlref{EXPOSURE}{EXPOSURE}] \lmbox
   No automatic default value.  The value of \verb+EXPOSURE+ previously set when
   using \verb+READIUE+ or \verb+READSIPS+ for the current \verb+APERTURE+ will
   be presented as the default value in the prompt.
\item [\htmlref{EXPOSURES}{EXPOSURES}] \lmbox
   No default value.  The program will attempt to extract the \verb+EXPOSURES+
   from the IUE GO tape/file header and present this as the default value for
   that particular command.
\item [\htmlref{EXTENDED}{EXTENDED} = FALSE] \lmbox
   Whether the object spectrum is expected to be extended.
\item [\htmlref{FIDFILE}{FIDFILE}] \lmbox
   No default value exists, a file name must be provided.
   The parameter is cancelled each time it is used.
\item [\htmlref{FIDSIZE}{FIDSIZE}] \lmbox
   The default \verb+FIDSIZE+ is read from the appropriate file in the
   \verb+$IUEDR_DATA+ directory.  This is presented as the default in parameter
   prompts.
\item [\htmlref{FILE}{FILE}] \lmbox
   No automatic default.  Takes the value 1 when reading a new tape or file.
\item [\htmlref{FILLGAP}{FILLGAP} = FALSE] \lmbox
   Whether gaps can be filled within an order.
\item [\htmlref{FLAG}{FLAG} = TRUE] \lmbox
   Whether data quality for faulty pixels is displayed.
\item [\htmlref{FN}{FN}] \lmbox
   No default.  A value must be supplied.
\item [\htmlref{FSCALE}{FSCALE} = 1] \lmbox
   No scale factor is applied by default.
\item [\htmlref{GSAMP}{GSAMP} = 1.414] \lmbox
   Spectrum grid sampling rate (geometric pixels).
\item [\htmlref{GSHIFT}{GSHIFT}] \lmbox
   No automatic default value.  The value of \verb+GSHIFT+ previously set when
   using \verb+CGSHIFT+ will be presented as the default value in the prompt.
\item [\htmlref{GSLIT}{GSLIT}] \lmbox
   No automatic default.  The value is calculated if \verb+AUTOSLIT=TRUE+,
   otherwise it should be estimated by looking at \verb+SCAN+ plots.
\item [\htmlref{HALAV}{HALAV} = 30.0] \lmbox
   FWHM of triangle function used for spectrum smoothing.
\item [\htmlref{HALC}{HALC}] \lmbox
   No default value.  Values must be positive or zero.
\item [\htmlref{HALTYPE}{HALTYPE} = POWER] \lmbox
   Currently this parameter can only take the value \verb+POWER+\@.
\item [\htmlref{HALW0}{HALW0} = 1400.0 or 1800.0] \lmbox
   Takes the value \verb+1400.0+ for the SWP camera, \verb+1800.0+ for
   LWP and LWR cameras.
\item [\htmlref{HALWC}{HALWC} = 1200.0 or 2400.0] \lmbox
   Takes the value \verb+1200.0+ for the SWP camera, \verb+2400.0+ for
   LWP and LWR cameras.
\item [\htmlref{HIST}{HIST} = TRUE] \lmbox
   Whether lines are to be drawn as histograms during plotting.
\item [\htmlref{IMAGE}{IMAGE}] \lmbox
   No default value.  The program will attempt to extract the \verb+IMAGE+
   number from the IUE GO tape/file header and present this as the default
   value for that particular command.
\item [\htmlref{ITF}{ITF}] \lmbox
   No default value.  Refer to page~\pageref{ITF} for details of
   working out which transfer function to use.
\item [\htmlref{ITFMAX}{ITFMAX}] \lmbox
   No default value.  Refer to page~\pageref{ITFMAX} for details of
   working out which transfer function saturation value to use.
\item [\htmlref{LINE}{LINE} (=SOLID)] \lmbox
   The value of LINE will be calculated for each command requiring it.
   If \verb+LINEROT=FALSE+ then \verb+LINE+ will take the value \verb+SOLID+,
   otherwise it will be rotated for each plotting command.
\item [\htmlref{LINEROT}{LINEROT} = FALSE] \lmbox
   Whether line style is to be changed after the next plot.
\item [\htmlref{ML}{ML}] \lmbox
   No default value.  Limits {\bf must} be specified.
\item [\htmlref{MONTH}{MONTH}] \lmbox
   No default value.  The program will attempt to extract the \verb+MONTH+ from
   the IUE GO tape/file header and present this as the default value for that
   particular command.
\item [\htmlref{MSAMP}{MSAMP}] \lmbox
   No default value exists.
\item [\htmlref{NFILE}{NFILE} = 1] \lmbox
   Number of tape files to be processed.
\item [\htmlref{NGEOM}{NGEOM} = 5] \lmbox
   When reading IUE GO tapes/files this is the value suggested.
\item [\htmlref{NLINE}{NLINE} = 10] \lmbox
   Number of IUE header lines to be printed.
\item [\htmlref{NORDER}{NORDER} = 0] \lmbox
   Number of \'{e}chelle orders to be processed.
\item [\htmlref{NSKIP}{NSKIP} = 1] \lmbox
   Number of tape marks to be skipped over.
\item [\htmlref{OBJECT}{OBJECT}] \lmbox
   The program will attempt to extract the \verb+OBJECT+ description text from
   the IUE GO header.  This is presented as the default in parameter prompts.
\item [\htmlref{ORDER}{ORDER}] \lmbox
   No default value. A valid \verb+ORDER+ {\bf must} be specified.
   Use \verb+SHOW V=S+ to find out which orders have been extracted from a
   HIRES image.
\item [{\htmlref{ORDERS}{ORDERS} (=[125,66])}] \lmbox
   Takes the values given by default for a new dataset and otherwise in
   response to a parameter cancel.
\item [\htmlref{OUTFILE}{OUTFILE}] \lmbox
   No automatic default.  The program will construct a suggested file name
   based on the \verb+CAMERA+, \verb+RESOLUTION+, \verb+APERTURE+ and
   \verb+ORDER+ as appropriate.
\item [\htmlref{QUAL}{QUAL} = TRUE] \lmbox
   Whether data quality information is plotted.
\item [\htmlref{RESOLUTION}{RESOLUTION}] \lmbox
   The program will attempt to extract the spectrograph \verb+RESOLUTION+ from
   the IUE GO header.  This is presented as the default in parameter prompts.
\item [\htmlref{RIPA}{RIPA} (=1)] \lmbox
   This parameter takes the value 1 when a new dataset is created.
\item [{\htmlref{RIPC}{RIPC} (=[1,0,0,0,0,0])}] \lmbox
   Takes the default values given above when a new dataset is created.
\item [\htmlref{RIPFILE}{RIPFILE}] \lmbox
   No default value exists, a file name must be provided.
   The parameter is cancelled each time it is used.
\item [\htmlref{RIPK}{RIPK}] \lmbox
   Takes the central wavelength value of an order by default.
\item [\htmlref{RL}{RL}] \lmbox
   No default.  \verb+RL=[0.0, 0.0]+ indicates the program should calculate
   values.
\item [\htmlref{RM}{RM} = TRUE] \lmbox
   Whether the mean spectrum is reset before averaging.
\item [\htmlref{RS}{RS} = TRUE] \lmbox
   Whether the display is reset before plotting.
\item [\htmlref{RSAMP}{RSAMP} = 1.414] \lmbox
   Radial coordinate sampling rate for \verb+LBLS+ grid (pixels).
\item [\htmlref{SCANAV}{SCANAV} = 5] \lmbox
   Averaging filter HWHM for image scan (geometric pixels).
\item [\htmlref{SCANDIST}{SCANDIST} (=0)] \lmbox
   Takes the last value used.  The first time \verb+SCAN+ is used the
   program will suggest a value of zero.
\item [\htmlref{SCANWV}{SCANWV}] \lmbox
   The value of \verb+SCANWV+ taken by default is calculated as the centre
   of the wavelength scale for the current \verb+APERTURE+\@.
\item [\htmlref{SKIPNEXT}{SKIPNEXT} = FALSE] \lmbox
   Whether skip to next tape file.
\item [\htmlref{SPECTYPE}{SPECTYPE} = 0] \lmbox
   DIPSO SP format file type (0, 1 or 2). Type 0 is a Starlink NDF.
\item [\htmlref{TEMFILE}{TEMFILE}] \lmbox
   No default value exists, a file name must be provided.
   The parameter is cancelled each time it is used.
\item [\htmlref{THDA}{THDA} (=0.0)] \lmbox
   No default, however, \verb+THDA=0.0+ implies that no value is available
   and the program will select a suitable mean \verb+THDA+ for the camera
   being used.
\item [\htmlref{THRESH}{THRESH}] \lmbox
   No default value.  The last value used will be presented as the default
   in parameter prompts.
\item [\htmlref{TYPE}{TYPE}] \lmbox
   No default.  This must be evaluated from the IUE GO file header contents.
\item [\htmlref{V}{V} = H] \lmbox
   This parameter is cancelled each time it is used.
\item [\htmlref{VSHIFT}{VSHIFT}] \lmbox
   No automatic default value.  If an VSHIFT has previously been set for
   the current \verb+DATASET+, this will be presented as the default value in
   the prompt.
\item [\htmlref{WCUT}{WCUT}] \lmbox
   The wavelength cut-off values are normally read from a \verb+.cut+ file and
   these are used for \verb+WCUT+ prompt values by default.
\item [\htmlref{WSHIFT}{WSHIFT} = 0] \lmbox
   No automatic default value.  If a \verb+WSHIFT+ has previously been set for
   the current \verb+DATASET+, this will be presented as the default value in
   the parameter prompt.
\item [{\htmlref{XCUT}{XCUT} =[-3.0,3.0]}] \lmbox
   Takes the value \verb+[-3.0,3.0]+ for a new dataset.
\item [{\htmlref{XL}{XL} (=[0,0])}] \lmbox
   No default value.  The limits are taken as the full x-range in the data
   to be plotted if no value of \verb+YL+ is set.
\item [{\htmlref{XP}{XP} (=[0,0])}] \lmbox
   No default value.  The limits are taken as the full image width if \verb+YP+
   is not set.
\item [\htmlref{YEAR}{YEAR}] \lmbox
   No default value.  The program will attempt to extract the \verb+YEAR+ from
   the IUE GO tape/file header and present this as the default value for that
   particular command.
\item [{\htmlref{YL}{YL} (=[0,0])}] \lmbox
   No default value.  The limits are taken as the full y-range in the data
   to be plotted if no value of \verb+YL+ is set.
\item [{\htmlref{YP}{YP} (=[0,0])}] \lmbox
   No default value.  The limits are taken as the full image height if
   \verb+YP+ is not set.
\item [{\htmlref{ZL}{ZL} (=[0,0])}] \lmbox
   No default value.  The limits are taken as the full intensity range for the
   current \verb+DATASET+ if \verb+ZL+ is not set.
\item [\htmlref{ZONE}{ZONE} = 0] \lmbox
   Graphics zone to be used for plotting.

\end {description}

\newpage
\section{\xlabel{vms_data}\label{se:vmsunix}Handling of VMS IUEDR data files}
\markboth{Handling of VMS IUEDR data files}{\stardocname}

IUEDR data files have changed format in order to allow
inter-machine operation. However, VMS IUEDR will still read the old format
files if they are present (this is only useful on the VAX as all old
format files will have been created on VAXen). If you have old format files
then you should use \verb+iuecnv+ to convert them to the new format by
following the procedure described below.
The resulting files can then be used with UNIX IUEDR.

During the conversion of an IUE dataset \verb+iuecnv+ will create a number of
binary data files. Their filenames are as follows:

\begin {description}
   \item \verb+<dataset>.UEC+ --- calibration file.
   \item \verb+<dataset>_UED.SDF+ --- image data and quality file.
   \item \verb+<dataset>_UES.SDF+ --- uncalibrated spectrum file.
   \item \verb+<dataset>_UEM.SDF+ --- calibrated mean spectrum file.
\end {description}

where \verb+<dataset>+ refers to the IUEDR DATASET parameter.

The {\tt .SDF} files are STARLINK NDF format files and can be read and
processed by any of the standard  packages ({\it{e.g.,}}\ KAPPA,FIGARO).
These files are in a machine independent format and can be freely
copied between any of the platforms which STARLINK supports.

\subsection{Moving IUEDR files to UNIX systems}

The file formats used by UNIX IUEDR are based on the STARLINK standard
NDF library. This makes the files transportable between all supported
systems. If you have old VMS IUEDR files ({\it{i.e.}}\ {\tt .UEC}, {\tt
.UED}, {\tt .UES}, {\tt .UEM} files) then these will need to be
converted into the new format {\bf before} they are transferred to a UNIX
system.

There is a VMS executable provided for this purpose, and a command file
to use it. To use the executable you must first copy:

\begin{verbatim}
   /star/bin/iuedr/iuecnv.exe
   /star/bin/iuedr/iuecnv.com
\end{verbatim}

onto your VMS system (use binary transfer for {\tt iuecnv.exe}).

When {\tt iuecnv.exe} is installed, you can then move to a directory
where your IUEDR datasets are stored and type:

\begin{verbatim}
   $ @somedisk:[somedir]iuecnv  dataset
\end{verbatim}

where {\tt somedisk:[somedir]} is wherever you copied {\tt iuecnv} to, and
{\tt dataset} is the name of an IUE dataset ({\it{e.g.,}}\ SWP23456).

The program will then locate and convert all the IUEDR datafiles
associated with the named dataset. Note that the {\tt .UEC} file is also
converted (although its name stays the same).

When conversion is complete you may copy the files ({\tt .UEC} and {\tt
*\_UE*.SDF}) to your UNIX system. The {\tt .UEC} files MUST be
transferred in ascii mode, and the {\tt .SDF} files MUST be transferred
in binary mode.

UNIX NDF expects that NDF container files end in the extension \verb+.sdf+ and
does not yet recognise \verb+.SDF+ files. Thus you may need to rename
files to have the lowercase \verb+.sdf+ extension (depending upon how
you do the transfer).

\subsection{VAX-UNIX IUEDR image file exchange}

An IUEDR image file is one of \verb+RAW+, \verb+PHOT+, or \verb+GPHOT+ type
and consist of 768 $\times$ 768 pixels each stored in a 1 or 2-byte integer.

The transfer of files between VAX and UNIX systems is complicated by
the sophistication of the VAX file system. Under VMS the system
records a complex description of the precise format of all the  files
(and stores it in  the directory entry). Under UNIX this information
has to be provided by the user of the file when it is  opened. Because
of this difference it is sometimes necessary to use the following
format conversion utilities.

\subsubsection{VAX to UNIX}

If you wish to transfer IUE image data from a VAX onto a UNIX machine
in order to use the UNIX IUEDR then the transfer should be done using
FTP (in {\bf binary} mode).

If you intend to copy the file using some other method ({\it{e.g.,}}\ via NFS)
then you should first use the command:

\begin{verbatim}
   UNIX_FORMAT image-name
\end{verbatim}

to ensure the file is properly transferred.

Note that this also applies if you wish to just access the file
via NFS without explicitly transferring it.

\subsubsection{UNIX to VAX}

If you wish to transfer IUE image data from a UNIX machine onto a VAX
in order to use the VAX IUEDR then the transfer should be done using
FTP (in {\bf binary} mode) and the command:

\begin{verbatim}
   VAX_FORMAT  image-name
\end{verbatim}

should then be used to ensure the file has the correct format.

If you use some other method of transferring the file ({\it{e.g.,}}\ NFS) then
the above command is {\bf still} required.

\subsubsection{What will work?}

In general the following two commands will allow you to use any disk
based IUE image with any machine running IUEDR:

\begin{itemize}
   \item {\tt VAX\_FORMAT} sets the file format as required by VAX IUEDR
   \item {\tt UNIX\_FORMAT} sets the file format as required by UNIX IUEDR
\end{itemize}

Both commands operate only on the VAX.

\subsection{\label{se:nfs}Accessing data via NFS}

UNIX machines commonly provide disk sharing amongst remote machines
using the NFS protocol.

For example your data frame may reside on a DECstation local  disk
called \verb+iuedata+ in the Rutherford cluster on machine \verb+adam4+\@. In
order to get IUEDR to read it directly you could enter the following
in response to the DRIVE prompt:

\begin{verbatim}
   DRIVE> /adam4/iuedata/swp12345.raw
\end{verbatim}

To see which disks you have NFS access to you should use the {\tt
\%~df} command. In general any disks whose entry does not start with
\verb+/dev+ are being served by a remote machine.

{\bf IMPORTANT NOTE\\}
IUEDR allows you to use this method of data access with the following proviso.

If the data resides on a VAX served disk then you must first  convert
its directory entry (on the VAX) using the following command:

\begin{verbatim}
   $ UNIX_FORMAT image-name
\end{verbatim}

This command does not change the data in any way.  It merely alters the
description of the file format as stored in the VAX directory.

If at some later stage you wish to use the VAX version of IUEDR on the
same data file it first be necessary to use the command:

\begin{verbatim}
   $ VAX_FORMAT image-name
\end{verbatim}

to convert back.

\twocolumn[
\section{\label{se:index}Command and parameter index}
]
\markboth{Index}{\stardocname}
\findexentry{A}{ABSFILE}{46}
\indexentry{AESHIFT}{14}
\indexentry{AGSHIFT}{14}
\indexentry{APERTURES}{46}
\indexentry{APERTURE}{46}
\indexentry{AUTOSLIT}{47}
\findexentry{B}{BADITF}{47}
\indexentry{BARKER}{15}
\indexentry{BDIST}{47}
\indexentry{BKGAV}{47}
\indexentry{BKGIT}{47}
\indexentry{BKGSD}{48}
\indexentry{BSLIT}{48}
\findexentry{C}{CAMERA}{48}
\indexentry{CENAV}{48}
\indexentry{CENIT}{48}
\indexentry{CENSD}{48}
\indexentry{CENSH}{48}
\indexentry{CENSV}{48}
\indexentry{CENTM}{49}
\indexentry{CENTREWAVE}{49}
\indexentry{CGSHIFT}{15}
\indexentry{CLEAN}{16}
\indexentry{COLOUR}{49}
\indexentry{COLROT}{49}
\indexentry{COL}{49}
\indexentry{CONTINUUM}{50}
\indexentry{COVERGAP}{50}
\indexentry{CULIMITS}{16}
\indexentry{CURSOR}{17}
\indexentry{CUTFILE}{50}
\indexentry{CUTWV}{50}
\findexentry{D}{DATASET}{50}
\indexentry{DAY}{50}
\indexentry{DELTAWAVE}{50}
\indexentry{DEVICE}{50}
\indexentry{DISPFILE}{50}
\indexentry{DRIMAGE}{17}
\indexentry{DRIVE}{51}
\indexentry{EDIMAGE}{18}
\findexentry{E}{EDMEAN}{19}
\indexentry{EDSPEC}{20}
\indexentry{ERASE}{20}
\indexentry{ESHIFT}{51}
\indexentry{EXIT}{21}
\indexentry{EXPOSURES}{51}
\indexentry{EXPOSURE}{51}
\indexentry{EXTENDED}{51}
\findexentry{F}{FIDFILE}{51}
\indexentry{FIDSIZE}{51}
\indexentry{FILE}{51}
\indexentry{FILLGAP}{52}
\indexentry{FLAG}{52}
\indexentry{FN}{52}
\indexentry{FSCALE}{52}
\findexentry{G}{GSAMP}{52}
\indexentry{GSHIFT}{52}
\indexentry{GSLIT}{53}
\findexentry{H}{HALAV}{53}
\indexentry{HALC}{53}
\indexentry{HALTYPE}{53}
\indexentry{HALW0}{53}
\indexentry{HALWC}{53}
\indexentry{HELP}{21}
\indexentry{HIST}{53}
\findexentry{I}{IMAGE}{53}
\indexentry{ITFMAX}{54}
\indexentry{ITF}{54}
\findexentry{L}{LBLS}{21}
\indexentry{LINEROT}{55}
\indexentry{LINE}{54}
\indexentry{LISTIUE}{22}
\findexentry{M}{MAP}{23}
\indexentry{ML}{55}
\indexentry{MODIMAGE}{23}
\indexentry{MONTH}{55}
\indexentry{MSAMP}{55}
\indexentry{MTMOVE}{24}
\indexentry{MTREW}{24}
\indexentry{MTSHOW}{24}
\indexentry{MTSKIPEOV}{25}
\indexentry{MTSKIPF}{25}
\findexentry{N}{NEWABS}{25}
\indexentry{NEWCUT}{26}
\indexentry{NEWDISP}{26}
\indexentry{NEWFID}{26}
\indexentry{NEWRIP}{27}
\indexentry{NEWTEM}{27}
\indexentry{NFILE}{55}
\indexentry{NGEOM}{55}
\indexentry{NLINE}{55}
\indexentry{NORDER}{55}
\indexentry{NSKIP}{55}
\findexentry{O}{OBJECT}{56}
\indexentry{ORDERS}{56}
\indexentry{ORDER}{56}
\indexentry{OUTEM}{27}
\indexentry{OUTFILE}{56}
\indexentry{OUTLBLS}{28}
\indexentry{OUTMEAN}{28}
\indexentry{OUTNET}{29}
\indexentry{OUTRAK}{29}
\indexentry{OUTSCAN}{30}
\indexentry{OUTSPEC}{30}
\findexentry{P}{PLCEN}{31}
\indexentry{PLFLUX}{32}
\indexentry{PLGRS}{33}
\indexentry{PLMEAN}{34}
\indexentry{PLNET}{35}
\indexentry{PLSCAN}{36}
\indexentry{PRGRS}{37}
\indexentry{PRLBLS}{37}
\indexentry{PRMEAN}{37}
\indexentry{PRSCAN}{38}
\indexentry{PRSPEC}{38}
\findexentry{Q}{QUAL}{56}
\indexentry{QUIT}{38}
\findexentry{R}{READIUE}{39}
\indexentry{READSIPS}{40}
\indexentry{RESOLUTION}{56}
\indexentry{RIPA}{57}
\indexentry{RIPC}{57}
\indexentry{RIPFILE}{57}
\indexentry{RIPK}{57}
\indexentry{RL}{58}
\indexentry{RM}{58}
\indexentry{RSAMP}{58}
\indexentry{RS}{58}
\findexentry{S}{SAVE}{40}
\indexentry{SCANAV}{58}
\indexentry{SCANDIST}{58}
\indexentry{SCANWV}{58}
\indexentry{SCAN}{41}
\indexentry{SETA}{41}
\indexentry{SETD}{42}
\indexentry{SETM}{42}
\indexentry{SGS}{43}
\indexentry{SHOW}{43}
\indexentry{SKIPNEXT}{58}
\indexentry{SPECTYPE}{59}
\findexentry{T}{TEMFILE}{59}
\indexentry{THDA}{59}
\indexentry{THRESH}{59}
\indexentry{TRAK}{44}
\indexentry{TYPE}{59}
\findexentry{V}{VSHIFT}{60}
\indexentry{V}{59}
\findexentry{W}{WCUT}{60}
\indexentry{WSHIFT}{60}
\findexentry{X}{XCUT}{60}
\indexentry{XL}{61}
\indexentry{XP}{61}
\findexentry{Y}{YEAR}{61}
\indexentry{YL}{61}
\indexentry{YP}{61}
\findexentry{Z}{ZL}{61}
\indexentry{ZONE}{61}

%\documentstyle[11pt,twoside]{article}
\pagestyle{myheadings}
\makeindex

%------------------------------------------------------------------------------
\newcommand{\stardoccategory}  {Starlink Guide}
\newcommand{\stardocinitials}  {SG}
\newcommand{\stardocsource}    {sg3.5}
\newcommand{\stardocnumber}    {3.5}
\newcommand{\stardocauthors}   {Paul Rees, Jack Giddings, Dave Mills \& Martin Clayton}
\newcommand{\stardocdate}      {12 March 1996}
\newcommand{\stardoctitle}     {IUEDR---Reference Manual}
%------------------------------------------------------------------------------


\newcommand{\stardocname}{\stardocinitials /\stardocnumber}
\newcommand{\numcir}[1]{\mbox{\hspace{3ex}$\bigcirc$\hspace{-1.7ex}{\small #1}}}
\newcommand{\lsk}{\raisebox{-0.4ex}{\rm *}}
%\renewcommand{\_}{{\tt\char'137}}     % re-centres the underscore - DONE LATER
\markright{\stardocname}
\setlength{\textwidth}{160mm}
\setlength{\textheight}{230mm}
\setlength{\topmargin}{-2mm}
\setlength{\oddsidemargin}{0mm}
\setlength{\evensidemargin}{0mm}
\setlength{\parindent}{0mm}
\setlength{\parskip}{\medskipamount}
\setlength{\unitlength}{1mm}


% -----------------------------------------------------------------------------
%  Hypertext definitions.
%  ======================
%  These are used by the LaTeX2HTML translator in conjuction with star2html.

%  Comment.sty: version 2.0, 19 June 1992
%  Selectively in/exclude pieces of text.
%
%  Author
%    Victor Eijkhout                                      <eijkhout@cs.utk.edu>
%    Department of Computer Science
%    University Tennessee at Knoxville
%    104 Ayres Hall
%    Knoxville, TN 37996
%    USA

%  Do not remove the %\begin{rawtex} and %\end{rawtex} lines (used by
%  star2html to signify raw TeX that latex2html cannot process).
%\begin{rawtex}
\makeatletter
\def\makeinnocent#1{\catcode`#1=12 }
\def\csarg#1#2{\expandafter#1\csname#2\endcsname}

\def\ThrowAwayComment#1{\begingroup
    \def\CurrentComment{#1}%
    \let\do\makeinnocent \dospecials
    \makeinnocent\^^L% and whatever other special cases
    \endlinechar`\^^M \catcode`\^^M=12 \xComment}
{\catcode`\^^M=12 \endlinechar=-1 %
 \gdef\xComment#1^^M{\def\test{#1}
      \csarg\ifx{PlainEnd\CurrentComment Test}\test
          \let\html@next\endgroup
      \else \csarg\ifx{LaLaEnd\CurrentComment Test}\test
            \edef\html@next{\endgroup\noexpand\end{\CurrentComment}}
      \else \let\html@next\xComment
      \fi \fi \html@next}
}
\makeatother

\def\includecomment
 #1{\expandafter\def\csname#1\endcsname{}%
    \expandafter\def\csname end#1\endcsname{}}
\def\excludecomment
 #1{\expandafter\def\csname#1\endcsname{\ThrowAwayComment{#1}}%
    {\escapechar=-1\relax
     \csarg\xdef{PlainEnd#1Test}{\string\\end#1}%
     \csarg\xdef{LaLaEnd#1Test}{\string\\end\string\{#1\string\}}%
    }}

%  Define environments that ignore their contents.
\excludecomment{comment}
\excludecomment{rawhtml}
\excludecomment{htmlonly}
%\end{rawtex}

%  Hypertext commands etc. This is a condensed version of the html.sty
%  file supplied with LaTeX2HTML by: Nikos Drakos <nikos@cbl.leeds.ac.uk> &
%  Jelle van Zeijl <jvzeijl@isou17.estec.esa.nl>. The LaTeX2HTML documentation
%  should be consulted about all commands (and the environments defined above)
%  except \xref and \xlabel which are Starlink specific.

\newcommand{\htmladdnormallinkfoot}[2]{#1\footnote{#2}}
\newcommand{\htmladdnormallink}[2]{#1}
\newcommand{\htmladdimg}[1]{}
\newenvironment{latexonly}{}{}
\newcommand{\hyperref}[4]{#2\ref{#4}#3}
\newcommand{\htmlref}[2]{#1}
\newcommand{\htmlimage}[1]{}
\newcommand{\htmladdtonavigation}[1]{}

%  Starlink cross-references and labels.
\newcommand{\xref}[3]{#1}
\newcommand{\xlabel}[1]{}

%  LaTeX2HTML symbol.
\newcommand{\latextohtml}{{\bf LaTeX}{2}{\tt{HTML}}}

%  Define command to recentre underscore for Latex and leave as normal
%  for HTML (severe problems with \_ in tabbing environments and \_\_
%  generally otherwise).
\newcommand{\latex}[1]{#1}
\newcommand{\setunderscore}{\renewcommand{\_}{{\tt\symbol{95}}}}
\latex{\setunderscore}

%  Redefine the \tableofcontents command. This procrastination is necessary
%  to stop the automatic creation of a second table of contents page
%  by latex2html.
\newcommand{\latexonlytoc}[0]{\tableofcontents}

% -----------------------------------------------------------------------------
%  Debugging.
%  =========
%  Un-comment the following to debug links in the HTML version using Latex.

% \newcommand{\hotlink}[2]{\fbox{\begin{tabular}[t]{@{}c@{}}#1\\\hline{\footnotesize #2}\end{tabular}}}
% \renewcommand{\htmladdnormallinkfoot}[2]{\hotlink{#1}{#2}}
% \renewcommand{\htmladdnormallink}[2]{\hotlink{#1}{#2}}
% \renewcommand{\hyperref}[4]{\hotlink{#1}{\S\ref{#4}}}
% \renewcommand{\htmlref}[2]{\hotlink{#1}{\S\ref{#2}}}
% \renewcommand{\xref}[3]{\hotlink{#1}{#2 -- #3}}
% -----------------------------------------------------------------------------
%  Add any document specific \newcommand or \newenvironment commands here

\newcommand{\lmbox}
{
    \mbox{} \\
}

\newcommand{\cpar}[2]
{
    \makebox[30mm][l]{\bf #1} & #2 (p~\pageref{#1}.)\\
}

\newcommand{\cparc}[1]
{
    \makebox[30mm][l]{ } & #1\\
}

\newcommand{\npar}[1]
{
    \makebox[30mm][l]{\bf #1} & \\
}

\newcommand{\iueparlist}[1]{
   \begin{description}
      #1
   \end{description}
}

\newcommand{\iueparameter}[3]
{
   \item [\label{#1}\index{#1}#1 = #2] \mbox{}\\
   #3
}

\newcommand{\indexentry}[2]
{
{\bf #1}\dotfill #2 \hspace*{15mm}\\
}

\newcommand{\findexentry}[3]
{
   \hspace*{\fill}\vspace*{3mm}\\
   \hspace*{\fill}{\large\bf --- #1 ---}\hspace*{\fill} \hspace*{15mm}\\
   \hspace*{\fill}\vspace*{-3mm}\\
   {\bf #2}\dotfill #3 \hspace*{15mm}\\
}

\newcommand{\comdescenv}[1]
{
\begin {tabular}{ll}
  #1
\end {tabular}
}

\newcommand{\comdesc}[2]
{
   \makebox[27mm][l]{\bf #1} & #2 \\
}

\newcommand{\comdescc}[1]
{
   \makebox[27mm][l]{ } & #1 \\
}


%% Redefine commands for hypertext version.

\begin{htmlonly}

\renewcommand{\lmbox}
{ }

\renewcommand{\cpar}[2]
{
    \item [\htmlref{#1}{#1}] #2
}

\renewcommand{\cparc}[1]
{
  #1
}

\rerenewcommand{\npar}[1]
{
    \item [#1]
}

\renewcommand{\indexentry}[2]
{
    {\bf \htmlref{#1}{#1}}\\
}

\renewcommand{\findexentry}[3]
{
    {\bf \htmlref{#2}{#2}}\\
}

\renewcommand{\comdescenv}[1]
{
   \begin{description}
       #1
   \end{description}
}

\renewcommand{\comdesc}[2]
{
   \item [{\bf \htmlref{#1}{#1}}] #2
}

\newcommand{\comdescc}[1]
{
  #1
}

\renewcommand{\iueparlist}[1]
{
      #1
}

\renewcommand{\iueparameter}[3]
{
\subsection{\xlabel{#1}\label{#1}#1}
   \begin{description}
   \item [{\bf Type:}] #2
   \item [{\bf Description:}] #3
   \end{description}
}

\end{htmlonly}

%+
%  Name:
%     SST.TEX

%  Purpose:
%     Define LaTeX commands for laying out Starlink routine descriptions.

%  Language:
%     LaTeX

%  Type of Module:
%     LaTeX data file.

%  Description:
%     This file defines LaTeX commands which allow routine documentation
%     produced by the SST application PROLAT to be processed by LaTeX and
%     by LaTeX2html. The contents of this file should be included in the
%     source prior to any statements that make of the sst commnds.

%  Notes:
%     The commands defined in the style file html.sty provided with LaTeX2html
%     are used. These should either be made available by using the appropriate
%     sun.tex (with hypertext extensions) or by putting the file html.sty
%     on your TEXINPUTS path (and including the name as part of the
%     documentstyle declaration).

%  Authors:
%     RFWS: R.F. Warren-Smith (STARLINK)
%     PDRAPER: P.W. Draper (Starlink - Durham University)

%  History:
%     10-SEP-1990 (RFWS):
%        Original version.
%     10-SEP-1990 (RFWS):
%        Added the implementation status section.
%     12-SEP-1990 (RFWS):
%        Added support for the usage section and adjusted various spacings.
%     8-DEC-1994 (PDRAPER):
%        Added support for simplified formatting using LaTeX2html.
%     {enter_further_changes_here}

%  Bugs:
%     {note_any_bugs_here}

% -

%  Define length variables.
\newlength{\sstbannerlength}
\newlength{\sstcaptionlength}
\newlength{\sstexampleslength}
\newlength{\sstexampleswidth}

%  Define a \tt font of the required size.
\newfont{\ssttt}{cmtt10 scaled 1095}

%  Define a command to produce a routine header, including its name,
%  a purpose description and the rest of the routine's documentation.
\newcommand{\sstroutine}[3]{
   \goodbreak
   \rule{\textwidth}{0.5mm}
   \vspace{-7ex}
   \newline
   \settowidth{\sstbannerlength}{{\Large {\bf #1}}}
   \setlength{\sstcaptionlength}{\textwidth}
   \setlength{\sstexampleslength}{\textwidth}
   \addtolength{\sstbannerlength}{0.5em}
   \addtolength{\sstcaptionlength}{-2.0\sstbannerlength}
   \addtolength{\sstcaptionlength}{-5.0pt}
   \settowidth{\sstexampleswidth}{{\bf Examples:}}
   \addtolength{\sstexampleslength}{-\sstexampleswidth}
   \parbox[t]{\sstbannerlength}{\flushleft{\Large {\bf #1}}}
   \parbox[t]{\sstcaptionlength}{\center{\Large #2}}
   \parbox[t]{\sstbannerlength}{\flushright{\Large {\bf #1}}}
   \label{#1}\index{#1}
   \begin{description}
      #3
   \end{description}
}

%  Format the description section.
\newcommand{\sstdescription}[1]{\item {\bf Description:}\vspace*{6pt}\\ #1}

%  Format the usage section.
\newcommand{\sstusage}[1]{\item[Usage:] \mbox{} \\[1.3ex] {\ssttt #1}}

%  Format the invocation section.
\newcommand{\sstinvocation}[1]{\item[Invocation:]\hspace{0.4em}{\tt #1}}

%  Format the arguments section.
\newcommand{\sstarguments}[1]
{
   \item[Arguments:] \mbox{} \\
   \vspace{-3.5ex}
   \begin{description}
      #1
   \end{description}
}

%  Format the returned value section (for a function).
\newcommand{\sstreturnedvalue}[1]{
   \item[Returned Value:] \mbox{} \\
   \vspace{-3.5ex}
   \begin{description}
      #1
   \end{description}
}

%  Format the parameters section (for an application).
\newcommand{\sstparameters}[1]{
\item {\bf Parameters:\vspace*{6pt}\\}
    \begin{tabular}{ll}
    #1
    \end{tabular}
}

%  Format the examples section.
\newcommand{\sstexamples}[1]{
   \item[Examples:] \mbox{} \\
   \vspace{-3.5ex}
   \begin{description}
      #1
   \end{description}
}

%  Define the format of a subsection in a normal section.
\newcommand{\sstsubsection}[1]{ \item[{#1}] \mbox{} \\}

%  Define the format of a subsection in the examples section.
\newcommand{\sstexamplesubsection}[2]{\sloppy
\item[\parbox{\sstexampleslength}{\ssttt #1}] \mbox{} \\ #2 }

%  Format the notes section.
\newcommand{\sstnotes}[1]{\item[Notes:] \mbox{} \\[1.3ex] #1}

%  Provide a general-purpose format for additional (DIY) sections.
\newcommand{\sstdiytopic}[2]{\item[{\hspace{-0.35em}#1\hspace{-0.35em}:}] \mbox{} \\[1.3ex] #2}

%  Format the implementation status section.
\newcommand{\sstimplementationstatus}[1]{
   \item[{Implementation Status:}] \mbox{} \\[1.3ex] #1}

%  Format the bugs section.
\newcommand{\sstbugs}[1]{\item[Bugs:] #1}

%  Format a list of items while in paragraph mode.
\newcommand{\sstitemlist}[1]{
  \mbox{} \\
  \vspace{-3.5ex}
  \begin{itemize}
     #1
  \end{itemize}
}

%  Define the format of an item.
\newcommand{\sstitem}{\item}

%% Now define html equivalents of those already set. These are used by
%  latex2html and are defined in the html.sty files.

\begin{htmlonly}

%  Re-define \ssttt.
   \newcommand{\ssttt}{\tt}

%  sstroutine.
   \renewcommand{\sstroutine}[3]{
\subsection{\xlabel{#1}\label{#1}#1}
      \begin{description}
         \item[{\bf Purpose:}] #2
         #3
      \end{description}
   }

%  sstdescription
   \renewcommand{\sstdescription}[1]{
      \item[{\bf Description:}]
      \begin{description}
         #1
      \end{description}
   }

%  sstusage
   \renewcommand{\sstusage}[1]{\item[Usage:]
      \begin{description}
         {\ssttt #1}
      \end{description}
   }

%  sstinvocation
   \renewcommand{\sstinvocation}[1]{\item[Invocation:]
      \begin{description}
         {\ssttt #1}
      \end{description}
   }

%  sstarguments
   \renewcommand{\sstarguments}[1]{
      \item[Arguments:]
      \begin{description}
         #1
      \end{description}
   }

%  sstreturnedvalue
   \renewcommand{\sstreturnedvalue}[1]{
      \item[Returned Value:]
      \begin{description}
         #1
      \end{description}
   }

%  sstparameters
   \renewcommand{\sstparameters}[1]{
      \item[{\bf Parameters:}]
      \begin{description}
         #1
      \end{description}
   }

%  sstexamples
   \renewcommand{\sstexamples}[1]{
      \item[Examples:]
      \begin{description}
         #1
      \end{description}
   }

%  sstsubsection
   \renewcommand{\sstsubsection}[1]{\item[{#1}]}

%  sstexamplesubsection
   \renewcommand{\sstexamplesubsection}[2]{\item[{\ssttt #1}] \\ #2}

%  sstnotes
   \renewcommand{\sstnotes}[1]{\item[Notes:]
      \begin{description}
         #1
      \end{description}
   }

%  sstdiytopic
   \renewcommand{\sstdiytopic}[2]{\item[{#1}]
      \begin{description}
         #2
      \end{description}
   }

%  sstimplementationstatus
   \renewcommand{\sstimplementationstatus}[1]{\item[Implementation Status:]
      \begin{description}
         #1
      \end{description}
   }

%  sstitemlist
   \newcommand{\sstitemlist}[1]{
      \begin{itemize}
         #1
      \end{itemize}
   }
\end{htmlonly}

%  End of "sst.tex" layout definitions.

% -----------------------------------------------------------------------------
%  Title Page.
%  ===========
\renewcommand{\thepage}{\roman{page}}
\begin{document}
\thispagestyle{empty}
%  Latex document header.
%  ======================
\begin{latexonly}
   CCLRC / {\sc Rutherford Appleton Laboratory} \hfill {\bf \stardocname}\\
   {\large Particle Physics \& Astronomy Research Council}\\
   {\large Starlink Project\\}
   {\large \stardoccategory\ \stardocnumber}
   \begin{flushright}
   \stardocauthors\\
   \stardocdate
   \end{flushright}
   \vspace{-4mm}
   \rule{\textwidth}{0.5mm}
   \vspace{5mm}
   \begin{center}
   {\Large\bf \stardoctitle}
   \end{center}
   \vspace{5mm}

%  Add heading for abstract if used.
%   \vspace{10mm}
%   \begin{center}
%      {\Large\bf Description}
%   \end{center}
\end{latexonly}

%  HTML documentation header.
%  ==========================
\begin{htmlonly}
   \xlabel{}
   \begin{rawhtml} <H1> \end{rawhtml}
      \stardoctitle
   \begin{rawhtml} </H1> \end{rawhtml}

%  Add picture here if required.

   \begin{rawhtml} <P> <I> \end{rawhtml}
   \stardoccategory \stardocnumber \\
   \stardocauthors \\
   \stardocdate
   \begin{rawhtml} </I> </P> <H3> \end{rawhtml}
      \htmladdnormallink{CCLRC}{http://www.cclrc.ac.uk} /
      \htmladdnormallink{Rutherford Appleton Laboratory}
                        {http://www.cclrc.ac.uk/ral} \\
      Particle Physics \& Astronomy Research Council \\
   \begin{rawhtml} </H3> <H2> \end{rawhtml}
      \htmladdnormallink{Starlink Project}{http://star-www.rl.ac.uk/}
   \begin{rawhtml} </H2> \end{rawhtml}
   \htmladdnormallink{\htmladdimg{source.gif} Retrieve hardcopy}
      {http://star-www.rl.ac.uk/cgi-bin/hcserver?\stardocsource}\\

%  HTML document table of contents.
%  ================================
%  Add table of contents header and a navigation button to return to this
%  point in the document (this should always go before the abstract \section).
  \label{stardoccontents}
  \begin{rawhtml}
    <HR>
    <H2>Contents</H2>
  \end{rawhtml}
  \renewcommand{\latexonlytoc}[0]{}
  \htmladdtonavigation{\htmlref{\htmladdimg{contents_motif.gif}}
        {stardoccontents}}

%  Start new section for abstract if used.
%  \section{\xlabel{abstract}Abstract}

\end{htmlonly}

% -----------------------------------------------------------------------------
%  Document Abstract. (if used)
%  ==================
% -----------------------------------------------------------------------------
%  Latex document Table of Contents (if used).
%  ===========================================
\begin{latexonly}
   \setlength{\parskip}{0mm}
   \latexonlytoc
   \setlength{\parskip}{\medskipamount}
   \markright{\stardocname}
\end{latexonly}
% -----------------------------------------------------------------------------

%%%%%%%%%%%%%%%%%%%%%%%%%%%%%%%%%%%%%%%%%%%%%%%%%%%%%%%%%%%%%%%%%%%%%%%%%%%
\newpage
\renewcommand{\thepage}{\arabic{page}}
\setcounter{page}{1}
\section{\xlabel{introduction}\label{se:introduction}Introduction }
\markboth{Introduction}{\stardocname}

This manual describes the commands and parameters used by IUEDR\@.
It is intended as a reference aid for people using IUEDR\@.

If you are new to IUE data reduction, you may like to read
\xref{{\sl IUE Analysis
a Tutorial}}{sg7}{} (SG/7) and the
\xref{{\sl IUEDR User Guide}}{mud45}{} (MUD/45) before proceeding
any further.
The Starlink User Note \xref{SUN/37}{sun37}{} contains a general description
of IUEDR which
overlaps with the early sections of this manual and also contains any notes on
the most recent release of the program.

Commands are described in Section~\ref{se:commands}, with parameters described
in more detail in Section~\ref{se:parameters}\@.
A list of default parameter behaviour and values is given in
Appendix~\ref{se:parameter_defaults}\@.
Details of transferring old-style IUEDR files from VMS systems to UNIX systems
and the conversion process are given in Appendix~\ref{se:vmsunix}\@.

\begin{latexonly}
An index of both commands and parameters is given in Appendix~\ref{se:index}\@.
\end{latexonly}

IUEDR functions fall into a number of specific categories:

\begin {itemize}
   \item IUE GO tape or file inspection and reading.
   \item Data display and manipulation.
   \item Spectrum extraction and calibration.
   \item Extraction product inspection and manipulation.
   \item Extraction product output.
   \item General operational commands.
\end {itemize}

These functions are controlled by over fifty commands, with nearly one hundred
global parameters within IUEDR\@.
There follows a summary of the commands available in each of the categories
listed above.

\subsection {IUE GO tape or file inspection and reading}

\comdescenv{
   \comdesc{LISTIUE}{Analyse the contents of one or more IUE tape files.}
   \comdesc{MTMOVE}{Move to the start of a tape file.}
   \comdesc{MTREW}{Rewind to the start of the tape.}
   \comdesc{MTSHOW}{Show the current tape position.}
   \comdesc{MTSKIPEOV}{Skip over the end-of-volume mark.}
   \comdesc{MTSKIPF}{Skip over NSKIP tape marks.}
   \comdesc{READIUE}{Read a RAW, GPHOT or PHOT IUE image from the tape/file.}
   \comdesc{READSIPS}{Read the MELO or MEHI IUESIPS product from the tape/file.}
}

\subsection {Data display and manipulation}

\comdescenv{
   \comdesc{CULIMITS}{Delineate the graphical display limits using the
                      graphics cursor.}
   \comdesc{CURSOR}{Determine display coordinates using the graphics cursor}
           \comdescc{and print them at the terminal.}
   \comdesc{DRIMAGE}{Display an IUE image on a suitable graphics workstation.}
   \comdesc{EDIMAGE}{Edit the image data quality using the graphics cursor.}
   \comdesc{MODIMAGE}{Modify image pixel intensities interactively.}
   \comdesc{CLEAN}{Mark as `bad' pixels with value below a given threshold.}
   \comdesc{SHOW}{Print information relating to the current dataset at the
                  terminal.}
   \comdesc{ERASE}{Erase the display screen of the current graphics
                   workstation.}
}

Image displays are colour coded to provide data quality information.
The colour codes used by IUEDR are as follows:

\begin{latexonly}
\begin {quote}
\begin {description}
   \item [Green] pixels affected by reseau marks
   \item [Red] pixels which are saturated (DN=255)
   \item [Orange] pixels affected by ITF truncation
   \item [Yellow] pixels marked bad by the user
\end {description}
\end {quote}
\end{latexonly}

\begin{htmlonly}
\begin{rawhtml}
<PRE>
   <B>Green</B>  - pixels affected by reseau marks
   <B>Red</B>    - pixels which are saturated (DN=255)
   <B>Orange</B> - pixels affected by ITF truncation
   <B>Yellow</B> - pixels marked bad by the user
</PRE>
\end{rawhtml}
\end{htmlonly}

The colour {\bf Blue} is used to indicate a pixel which has a value above the
maximum that can be displayed using the linear greyscale image display colour
look-up table.

When using a mouse or tracker-ball with the graphics cursor, the cursor hit
buttons are normally numbered in increasing order from left to right.
For example the left mouse button corresponds to cursor key hit 1, middle
button to cursor key hit 2 and so on.
Many terminals allow left-handed users to reverse the mouse button order.

\subsection {Spectrum extraction and calibration}

\comdescenv{
   \comdesc{AESHIFT}{Determine (HIRES) spectrum ESHIFT automatically.}
   \comdesc{AGSHIFT}{Determine spectrum template shift automatically.}
   \comdesc{BARKER}{Correct the extracted data for \'{e}chelle ripple}
           \comdescc{using a method based upon that of Barker (1984).}
   \comdesc{CGSHIFT}{Determine spectrum template shift using the cursor
                     on a SCAN plot.}
   \comdesc{LBLS}{Extract a line-by-line-spectrum array from the image.}
   \comdesc{NEWABS}{Associate a new absolute flux calibration with the
                  current  dataset.}
   \comdesc{NEWCUT}{Associate new \'{e}chelle order wavelength limits with the
                  current  dataset.}
   \comdesc{NEWDISP}{Associate new spectrograph dispersion data with the
                   current  dataset.}
   \comdesc{NEWFID}{Associate new fiducial positions with the current
                  dataset.}
   \comdesc{NEWRIP}{Associate new ripple calibration data with the current
                  dataset.}
   \comdesc{NEWTEM}{Associate new spectrum centroid template data with the
                    current dataset.}
   \comdesc{SCAN}{Perform a scan of the image data perpendicular}
           \comdescc{to the spectrograph dispersion.}
   \comdesc{SETA}{Set dataset parameters which are aperture specific.}
   \comdesc{SETD}{Set dataset parameters which are independent of order and
                aperture.}
   \comdesc{SETM}{Set dataset parameters which are order specific.}
   \comdesc{TRAK}{Extract a spectrum from the image.}
}

\subsection {Extraction product inspection and manipulation}

\comdescenv{
   \comdesc{EDMEAN}{Edit the mean extracted spectrum using the graphics
                  cursor.}
   \comdesc{EDSPEC}{Edit the net extracted spectrum using the graphics
                  cursor.}
   \comdesc{MAP}{Map and merge extracted spectrum components to produce}
           \comdescc{a mean spectrum.}
   \comdesc{PLCEN}{Plot the smoothed spectrum centroid shifts.}
   \comdesc{PLFLUX}{Plot the calibrated flux spectrum.}
   \comdesc{PLGRS}{Plot the pseudo-gross and background resulting from the}
           \comdescc{spectrum extraction.}
   \comdesc{PLMEAN}{Plot the mean spectrum.}
   \comdesc{PLNET}{Plot the uncalibrated net spectrum.}
   \comdesc{PLSCAN}{Plot the image scan perpendicular to the dispersion.}
   \comdesc{SGS}{Print the names of the available SGS graphics devices at
               the  terminal.}
}

Plots of extracted IUE spectra and image scans include data quality information
flags for bad data.
The data quality codes used by IUEDR are as follows:

\begin{latexonly}
\begin {quote}
\begin {description}
   \item [1] affected by extrapolated ITF
   \item [2] affected by microphonics
   \item [3] affected by noise spike
   \item [4] affected by bright point (or user)
   \item [5] affected by reseau mark
   \item [6] affected by ITF truncation
   \item [7] affected by saturation
   \item [U] affected by user edit
\end {description}
\end {quote}
\end{latexonly}

\begin{htmlonly}
\begin{rawhtml}
<PRE>
   <B>1</B> - affected by extrapolated ITF
   <B>2</B> - affected by microphonics
   <B>3</B> - affected by noise spike
   <B>4</B> - affected by bright point (or user)
   <B>5</B> - affected by reseau mark
   <B>6</B> - affected by ITF truncation
   <B>7</B> - affected by saturation
   <B>U</B> - affected by user edit
</PRE>
\end{rawhtml}
\end{htmlonly}

\subsection {Extraction product output}

\comdescenv{
   \comdesc{OUTEM}{Output the current spectrum template data to a formatted
                 data  file.}
   \comdesc{OUTLBLS}{Output the current LBLS array to a binary data file.}
   \comdesc{OUTMEAN}{Output the current mean spectrum to a DIPSO SP
                   format  data file.}
   \comdesc{OUTNET}{Output the current net spectrum to a DIPSO SP
                  format data file.}
   \comdesc{OUTRAK}{Output the current uncalibrated spectrum to a}
           \comdescc{``TRAK'' formatted data file.}
   \comdesc{OUTSCAN}{Output the current scan data to a DIPSO SP format
                   data  file.}
   \comdesc{OUTSPEC}{Output the current aperture (LORES) or order (HIRES)}
           \comdescc{spectrum to a DIPSO SP format data file.}
   \comdesc{PRGRS}{Print the current extracted aperture or order spectrum
                 in tabular form.}
   \comdesc{PRLBLS}{Print the current LBLS array in tabular form.}
   \comdesc{PRMEAN}{Print the current mean spectrum in tabular form.}
   \comdesc{PRSCAN}{Print the intensities of the current image scan in
                  tabular  form.}
   \comdesc{PRSPEC}{Print the current aperture or order spectrum in tabular
                  form.}
}

\subsection {General operational commands}

\comdescenv{
   \comdesc{EXIT}{Leave IUEDR and update any files altered during the}
           \comdescc{current IUEDR session.}
   \comdesc{QUIT}{Leave IUEDR and update any files altered during the}
           \comdescc{current IUEDR session.}
   \comdesc{SAVE}{Overwrite any files that have had their contents}
           \comdescc{updated during the current IUEDR session.}
}

A log of all commands typed during an IUEDR session and program output at the
terminal can be found in the file {\tt session.lis}.
This is particularly useful when investigating the contents of IUE tapes.

\newpage
\section{\xlabel{user_interface}User interface}
\markboth{User interface}{\stardocname}

The IUEDR user interface uses the Starlink ADAM parameter system.
The interface will be familiar to users of the VMS IUEDR, with a few changes
to reach a level of consistency with other Starlink packages.

\subsection {Starting IUEDR}

To initialise for IUEDR type
\begin{verbatim}
   % iuedr
\end{verbatim}

at the shell prompt \verb+%+.
The first time you type the command, IUEDR environment variables are set up in
your session.
You can now start the program by typing
\begin{verbatim}
   % iuedr
\end{verbatim}
again.
Edit your \verb+.login+ file if you want to avoid having to type the command
that extra time.

The command line interface prompt is \verb+>+\@.
This should appear after a welcome message and you can then type commands as
you would in a typical command shell.

\subsection {Response to command prompts}

Instructions to IUEDR are given as command lines.

Command lines begin with a command and an optional list of parameter
assignments. For example:
\begin{verbatim}
   > DRIMAGE DATASET=SWP14931 DEVICE=xw
\end{verbatim}

Usually IUEDR will only prompt for parameters required by commands if
they have no currently defined value.
However, some parameters are either cancelled during the execution of a
command or are set so that the user is always prompted for a value.
A command can be forced to prompt for all required parameter values thus:

\begin{verbatim}
   > READIUE PROMPT
\end{verbatim}

The \verb+PROMPT+ may be abbreviated to \verb+PR+\@.

In a similar way commands which always prompt for parameter values can be made
to accept default values thus:

\begin{verbatim}
   > READIUE ACCEPT
\end{verbatim}

Command input and output printed at the terminal is also copied to the
file \verb+session.lis+ in the working directory.
Note that this file is rewritten each time IUEDR is run.

\subsection {Response to parameter prompts}

Help about a parameter can be obtained by responding to the parameter prompt
with a question mark, {\it{e.g.,}}

\begin{verbatim}
   DATASET - Dataset Name. > ?
\end{verbatim}

Help information will then be printed at the terminal and the prompt repeated.

Sometimes an undefined parameter value is interpreted by a command in
a specific way ({\it{e.g.,}}\ auto-scaling within plotting commands)\@.
A parameter can be set undefined by responding to the prompt with an
exclamation mark, {\it{e.g.,}}

\begin{verbatim}
   XL - X-axis plotting limits, [0,0] means auto-scale. /[1150,1950]/ > !
\end{verbatim}

A command may be aborted by responding to a parameter prompt with a
double exclamation mark, {\it{e.g.,}}

\begin{verbatim}
   XL - X-axis plotting limits, [0,0] means auto-scale. /[1150,1950]/ > !!
\end{verbatim}

To prevent a command from prompting for further parameter values (once in
prompt mode) type a backslash, {\it{e.g.,}}

\begin{verbatim}
   XL - X-axis plotting limits, [0,0] means auto-scale. /[1150,1950]/ > \
\end{verbatim}

The command will then only prompt for parameters for which it cannot generate
a suitable value.

\subsection {Getting HELP}

\begin{latexonly}
Type HELP at the IUEDR command line prompt.
You may optionally append a detailed description of the topic on which help
is required. See page~\pageref{com: HELP } for further details.
\end{latexonly}

\begin{htmlonly}
Type HELP at the IUEDR command line prompt.
You may optionally append a detailed description of the topic on which help
is required. See HELP for further details.
\end{htmlonly}

\subsection {IUEDR in script and batch modes}

It is possible to run IUEDR in script mode, where command and
parameter input originates from a file instead of the terminal.
The file from which the command input is to be taken is piped into IUEDR:

\begin{verbatim}
   % iuedr < script_file
\end{verbatim}

This will result in the command input being taken from the file
\verb+script_file+ and the text output being written to the \verb+session.lis+
file.

Alternatively the output can be directed to a specific file:

\begin{verbatim}
   % iuedr < script_file > script_log
\end{verbatim}

In this case \verb+script_log+ contains the output log \verb+session.lis+ is
{\bf not} over-written.

\newpage
\section{\xlabel{IUEDR_data_files}IUEDR data files}
\markboth{IUEDR data files}{\stardocname}

Once an IUE GO format file has been read by IUEDR two files are created.  One
is a text file containing information about the data set calibration data.
This file has a name constructed
\begin{verbatim}
   <dataset>.UEC
\end{verbatim}
where \verb+<dataset>+ corresponds to the value of the IUEDR \verb+DATASET+
parameter.

when a GPHOT, PHOT or RAW image is read the data produced are stored in an
Image and Data Quality file
\begin{verbatim}
   <dataset>_UED.sdf
\end{verbatim}
After the spectrum extraction process the uncalibrated spectral data are stored
in a file
\begin{verbatim}
   <dataset>_UES.sdf
\end{verbatim}
This file is also produced when an IUESIPS MELO or MEHI file is read with
\verb+READSIPS+\@.  No \verb+_UED.sdf+ being created in this case.

Calibrated spectra produced by the \verb+MAP+ command are stored in a file
\begin{verbatim}
   <dataset>_UEM.sdf
\end{verbatim}

All these \verb+.sdf+ files are Starlink NDF format files (See
\xref{SUN/33}{sun33}{}
for details of access to NDFs) which can be read by any of the standard
packages (KAPPA, FIGARO etc.).  The contents of these files can be examined
outside of IUEDR using the \verb+hdstrace+ command.

These files are in addition to the IUEDR log file \verb+session.lis+ and the
files generated by output commands.  They should {\bf not} be deleted until the
data reduction is complete and the output spectra obtained.

The \verb+OUT*+ family of IUEDR output commands also generate NDFs which can
be read by programs such as DIPSO.

In summary:
\begin {description}
   \item \verb+<dataset>.UEC+ --- calibration file.
   \item \verb+<dataset>_UED.SDF+ --- image data and quality file.
   \item \verb+<dataset>_UES.SDF+ --- uncalibrated spectrum file.
   \item \verb+<dataset>_UEM.SDF+ --- calibrated mean spectrum file.
\end {description}

{\bf Refer to Appendix~\ref{se:vmsunix} for VMS to UNIX file conversion.}

\subsection{\label{subap:ndf}NDF Components in IUEDR files}

This section gives a summary of the NDF components present in IUEDR files for
those who may wish to access the files from their own programs.
The structure and content of an NDF can be inspected using the {\tt hdstrace}
utility (See \xref{SUN/102}{sun102}{})\@.  Values for components have been
given where they are constant for all files of the particular type.

\subsubsection{Image data and quality file {\tt \_UED}}

\begin{latexonly}
A simple NDF, each point in the $768\times 768$ image is described by a datum
and a quality flag.  The image is given the generic title `IUE image'\@.
\end{latexonly}

\begin{htmlonly}
A simple NDF, each point in the 768x768 image is described by a datum
and a quality flag.  The image is given the generic title `IUE image'\@.
\end{htmlonly}

\begin{verbatim}
IUEDR  <NDF>

   DATA_ARRAY(768,768)  <_WORD>

   QUALITY        <QUALITY>       {structure}
      QUALITY(768,768)  <_UBYTE>

   TITLE          <_CHAR*9>       'IUE image'
\end{verbatim}

\subsubsection{Uncalibrated spectrum file {\tt \_UES}}

This NDF contains IUEDR specific extensions, which are written and read by the
program when processing spectra.  In the description below {\tt no} is
the number of orders processed with the \verb+TRAK+ command.  {\tt mo} is
the number of data points in the longest order processed.
{\tt WAVES} contains the wavelengths of each of the flux data in each order.
{\tt ORDERS} holds a list of the order numbers processed.
{\tt NWAVS} stores the actual number of points in each of the {\tt no} orders.

\begin{verbatim}
IUEDR  <NDF>

   DATA_ARRAY(mo,no)   <_REAL>

   QUALITY        <QUALITY>       {structure}
      QUALITY(mo,no)   <_UBYTE>

   MORE           <EXT>           {structure}
      IUEDR_EXTRA    <EXTENSION>     {structure}
         WAVES(mo,no)   <_REAL>
         ORDERS(no)     <_INTEGER>
         NWAVS(no)      <_INTEGER>

   TITLE          <_CHAR*80>
   LABEL          <_CHAR*4>       'Flux'
\end{verbatim}

\subsubsection{Calibrated mean spectrum file {\tt \_UEM}}

The mean spectrum is basically an array of flux values against a wavelength
scale.  Quality information is included in the data file.  The axes units are
also included.
This NDF type is again IUEDR specific, the unusual components are as follows.
{\tt WAVES} are wavelengths of flux data points.
{\tt WEIGHTS} are weights applied to the flux.
{\tt XCOMB1} is the start wavelength.
{\tt DXCOMB} is the wavelength step from point to point.



\begin{verbatim}
IUEDR  <NDF>

   DATA_ARRAY(17001)  <_REAL>

   QUALITY        <QUALITY>       {structure}
      QUALITY(17001)  <_UBYTE>

   MORE           <EXT>           {structure}
      IUEDR_EXTRA    <EXTENSION>     {structure}
         WAVES(17001)   <_REAL>
         WEIGHTS(17001)  <_REAL>
         XCOMB1         <_DOUBLE>
         DXCOMB         <_DOUBLE>

   AXIS(1)        <AXIS>          {structure}
      DATA_ARRAY(17001)  <_REAL>
      UNITS          <_CHAR*40>      '(A)'
      LABEL          <_CHAR*40>      'Wavelength'

   TITLE          <_CHAR*80>
   UNITS          <_CHAR*40>      '(FN/s)'
   LABEL          <_CHAR*40>      'Flux'
\end{verbatim}

\subsubsection{\label{se:spectrum}SPECTRUM format output files}

SPECTRUM is a data analysis programme written by Steve Adams at UCL\@.
Although the programme is no longer used (I guess it might be in use
somewhere\ldots) the file formats it introduced were adopted by the popular
spectrum analysis programme DIPSO (described in \xref{SUN/50}{sun50}{})\@.

The basic input to a SPECTRUM file is a single spectrum (wavelength, flux)\@.
The wavelengths should be in increasing order, and evenly spaced.

There are three variants of the SPECTRUM file format.  Using its terminology:

\begin{latexonly}
\begin{tabular}{ll}
Format number & File characteristics\\
0             & Unformatted (Binary)\\
1             & Fixed Format Text\\
2             & Free-field Format Text\\
\end{tabular}
\end{latexonly}

\begin{htmlonly}
\begin{rawhtml}
<PRE>
<B>Format number     File characteristics</B>
      0           Unformatted (Binary)
      1           Fixed Format Text
      2           Free-field Format Text
</PRE>
\end{rawhtml}
\end{htmlonly}

IUEDR {\bf no longer} produces output of the SP0 type.  Instead NDFs are used.
In practice this is invisible to the user as the DIPSO SP0RD command (read
SPECTRUM format 0 file) now reads NDFs!  The other two formats are still
available.  A Description of the old SP0 format is included here in case
anyone needs to read an existing file in this format (DIPSO can still read
SP0 format via the SP0RD command)\@.

Using a FORTRAN77 notation, the contents of a SPECTRUM file can be
expressed as:

\begin{verbatim}
   PARAMETER(MAXWAV=8000) ! maximum number of wavelengths
   CHARACTER*79 CLINE1    ! first line of text
   CHARACTER*79 CLINE2    ! second line of text
   INTEGER NWAV           ! number of wavelengths
   REAL WAV(MAXWAV)       ! wavelengths
   REAL FLUX(MAXWAV)      ! fluxes
\end{verbatim}

Both \verb+CLINE1+ and \verb+CLINE2+ are totally unstructured text strings,
and are used to describe the spectrum.  The convention is that
\verb+FLUX(I)=0.0+ when its value is undefined.

Here, briefly, is the code needed to read the SPECTRUM formats:

Format number 0 is an unformatted (binary) file read by:

\begin{verbatim}
   OPEN(UNIT=1, ACCESS='SEQUENTIAL', FORM='UNFORMATTED')
   READ(1) CLINE1(1:79)
   READ(1) CLINE2(1:79)
   READ(1) NWAV
   READ(1) (WAV(I),FLUX(I),I=1,NWAV)
   CLOSE(UNIT=1)
\end{verbatim}

Format number 1 is a fixed format text file read by:

\begin{verbatim}
   OPEN(UNIT=1, ACCESS='SEQUENTIAL')
   READ(1,'(A79)') CLINE1(1:79)
   READ(1,'(A79)') CLINE2(1:79)
   READ(1,'(20X,I6)') NWAV
   READ(1,'(4(F8.3,E10.3))') (WAV(I),FLUX(I),I=1,NWAV)
   CLOSE(UNIT=1)
\end{verbatim}

Format number 2 is a free-field text file read by:

\begin{verbatim}
   OPEN(UNIT=1, ACCESS='SEQUENTIAL')
   READ(1,'(A79)') CLINE1(1:79)
   READ(1,'(A79)') CLINE2(1:79)
   READ(1,*) NWAV
   READ(1,*) (WAV(I),FLUX(I),I=1,NWAV)
   CLOSE(UNIT=1)
\end{verbatim}

The sections of FORTRAN 77 code shown above are not intended to be serious
attempts to write a SPECTRUM file reading programme.  Instead they are designed
to define the contents as succinctly as possible.

\newpage
\section{\xlabel{porting_changes}Changes during the port to UNIX}
\markboth{Changes during the port to UNIX}{\stardocname}

This section describes the main changes that have been made to IUEDR
during its conversion to an ADAM based application which runs on
all Starlink supported platforms.  \xref{SUN/37}{sun37}{} gives notes on the
very latest version of IUEDR.

If you are a seasoned IUEDR user then you should study this section
especially carefully.

The most significant change from the scientific point of view is that
the precision of all floating point calculations has been upgraded to
DOUBLE PRECISION. This was done after it was noticed that for high
resolution extraction the output spectra were subject to rounding
noise at the 1\% level.

The format of the calibration file ({\tt .UEC}) created by IUEDR has
been  changed to make it more readable. A VMS program to convert
IUEDR datasets to the new format is available, see Appendix~\ref{se:vmsunix}
for details.

The functionality of the package has been enhanced to allow image data
to be read directly from disk.

The general operation of IUEDR, and all the command and parameter
names, are identical to those used in previous versions.

\subsection{IUEDR command files}

The \verb+.CMD+ style of VMS IUEDR command files is not directly supported by
UNIX IUEDR, and neither is the associated input/output redirection using
\verb+<+ and \verb+>+\@.

It is very easy to convert a {\tt .CMD} file into a UNIX IUEDR command
script.
{\it{e.g.,}}\ a {\tt DEMO.CMD} procedure:

\begin{verbatim}
   DATASET=SWP03196
   SHOW
   SCAN ORDERS=(125,66)
   TRAK APERTURE=LAP
   SHOW V=S
\end{verbatim}

would become a UNIX IUEDR command script \verb+demo.cmd+, thus:

\begin{verbatim}
   SHOW DATASET=SWP03196
   SCAN ORDERS=[125,66]
   TRAK APERTURE=LAP
   SHOW V=S
\end{verbatim}

The only changes which need be made are to move any parameter
specifications ({\it{e.g.,}}\ \verb+DATASET=+) onto the same line as the command
they apply to, and to change vector parameter specifications to use square
\verb+[]+ brackets instead of the old-style round \verb+()+ brackets.

This script can then be read into IUEDR by
\begin{verbatim}
   % iuedr < demo.cmd
\end{verbatim}

\subsection{Interaction with DIPSO}

As part of the port to UNIX the
format of the default DIPSO spectrum format SP0 files has been
changed to use the STARLINK NDF data format. This means that these
files can be read by any standard STARLINK package.

The IUEDR/DIPSO user should notice no difference, as both IUEDR and
DIPSO understand the new format.

\subsection{DRIVE parameter options}

The use of the DRIVE parameter has been enhanced to allow
specification of disk files containing IUE datasets. This is intended
for use with  files obtained from online archives (RAL and NASA).

The syntax is to provide the full filename and extension in response to
the DRIVE prompt:

\begin{verbatim}
   DRIVE> SWP12345.RAW
\end{verbatim}

\subsection{Specifying vector parameters}

Some IUEDR parameters ({\it{e.g.,}}\ \verb+XP+, \verb+YP+) require the
specification of a pair of numbers defining the limits of a range of values
({\it{e.g.,}}\ pixels).

The method of setting such values has changed to the ADAM style:

\begin{verbatim}
   XP=[100,300]
\end{verbatim}

Note that the square brackets are only necessary when vector
parameters are specified on the command line. They are not required
when IUEDR prompts the user for a vector parameter.

\subsection{Calibration files and the NEW* family of commands}

The missing Calibration file for SWP camera HIRES data has been added to the
IUEDR package.

The action of the \verb+NEW*+ family of commands for updating IUEDR
calibrations, geometry and so on have been altered  to improve functionality.
Users will find that the previous style of file name entry still works,
and that the following features have been added:
\begin{itemize}
   \item Both Logical Name and Environment Variable style file specifications
   may be given as parameter values, for example
   \begin{verbatim}
      > NEWABS ABSFILE=$IUEDR_DATA/swphi
   \end{verbatim}
   and
   \begin{verbatim}
      > NEWABS ABSFILE=IUEDR_DATA:swphi
   \end{verbatim}
   are equivalent and allowed on all platforms.
   \item The default file name extension need not be given, however if the
   file to be read has a different extension this {\bf should} be given.
   For example,
   \begin{verbatim}
      > NEWABS ABSFILE=$IUEDR_DATA/swphi
   \end{verbatim}
   and
   \begin{verbatim}
      > NEWABS ABSFILE=$IUEDR_DATA/swphi.abs
   \end{verbatim}
   are {\bf both} valid.

   This behaviour is a change to the VMS-only IUEDR where the extension had
   to be omitted and the default value for the appropriate command was always
   taken.
   \item For case-sensitive file systems (like UNIX) if the Enviroment Variable
   \verb+$IUEDR_DATA+ is used as part of the file specification then the case
   of the file name itself is always converted to {\bf lower case}.  All the
   files available in the \verb+$IUEDR_DATA+ directory have lower case names
   so this is not a problem, rather it allows default calibration file names
   to be upper- or lower-case in IUEDR command scripts.
\end{itemize}

\subsection{Documentation}

The excellent introduction to IUEDR
\xref{{\sl IUE Analysis---A Tutorial}}{sg7}{}
(SG/7) by Richard Tweedy has been updated for UNIX and included as a standard
part of IUEDR\@.  Some special calibration corrections described in this
document have also been added to the IUEDR package.

\newpage
\section{\xlabel{commands}\label{se:commands}Commands}
\markboth{Commands}{\stardocname}

This section contains a detailed description of each of the commands
available in IUEDR.  A list of the parameters used by each command is given,
along with a brief description of each.  The pages on which you will find
full parameter descriptions are given at the end of each line in the parameter
list.

\sstroutine{AESHIFT}
{
   Determine (HIRES) spectrum ESHIFT automatically.
}{
   \sstparameters{
   \cpar{DATASET}{Dataset name.}
   \cpar{CENTREWAVE}{Line central wavelengths (A).}
   \cpar{DELTAWAVE}{Half-width of line search windows (A).}
}
\sstdescription{
   \verb+AESHIFT+ can be used to measure the global \'{e}chelle shift for a
   HIRES
   spectrum.  A set of laboratory wavelengths of absorption features which
   should be present in the spectrum are located in the spectrum and the
   \verb+ESHIFT+ for each is calculated.  The median of these \verb+ESHIFT+s
   is then applied to the whole dataset.
}
}

\sstroutine{AGSHIFT}
{
   Determine spectrum shift for HIRES automatically.
}{
   \sstparameters{
   \cpar{DATASET}{Dataset name.}
   \cpar{ORDERS}{This delineates a range of \'{e}chelle orders.}
}
\sstdescription{
   \verb+AGSHIFT+ can be used to measure the global geometric shift for a HIRES
   spectrum.  A scan of the \verb+DATASET+ must be made available using the
   \verb+SCAN+ command.  The scan data is traversed starting at the lowest
   numbered order and a probable site for the peak of each order is found.
   The central position of each order is then estimated using a centroiding
   algorithm.  An estimate of the geometric shift is made for each order and
   these are recorded and displayed.

   A weighted mean in which the shifts determined for orders 100 to 110
   inclusive are given greater weight than other shifts is calculated.
   The individual order shifts are compared to the mean and any shift
   greater than 3 pixels from the mean position is rejected and a new value
   for the mean shift calculated from the remaining orders.
   The mean shift is displayed and the \verb+GSHIFT+ parameter is set.

   Using \verb+AGSHIFT+ it is possible to automate the spectrum extraction
   process.  It should be noted that objects with no continuum may break
   the \verb+AGSHIFT+ mechanism, giving poor shift values, in these cases
   the interactive \verb+CGSHIFT+ command should be used.
}
}

\sstroutine{BARKER}
{
   Correct spectrum data for \'{e}chelle ripple using a method based
   upon that of Barker (1984).
}{
   \sstparameters{
   \cpar{DATASET}{Dataset name.}
   \cpar{ORDERS}{This delineates a range of \'{e}chelle orders.}
}
\sstdescription{
   The spectrum data in \verb+DATASET+ are corrected for residual \'{e}chelle
   ripple using  the method described by Barker (1984. Astronomical Journal,
   \underline{89},  899). Orders in the range \verb+ORDERS+ are used in the
   ripple correction optimisation.  Note that this optimisation method is
   only applicable for SWP spectra.
}
}


\sstroutine{CGSHIFT}
{
   Determine spectrum template shift using the cursor on a scan plot.
}{
   \sstparameters{
   \cpar{DATASET}{Dataset name.}
   \cpar{APERTURE}{Aperture name (\verb+SAP+ or \verb+LAP+).}
   \cpar{ORDERS}{This delineates a range of \'{e}chelle orders.}
   \cpar{DEVICE}{GKS/SGS graphics device name.}
}
\sstdescription{
   This command allows the graphics cursor to be used to provide
   information about spectrum template registration shifts.

   A plot of the current spectrum scan must be available on the graphics
   \verb+DEVICE+\@.

   A cycle consisting of any  number of left or middle mouse button hits is
   used to mark the position of the spectrum. Each hit is used to calculate
   a linear geometric shift of the spectrum template relative to the image.
   The cycle is terminated by pressing the right mouse button.

   Keyboard keys 1, 2 and 3 can be used  in place of left, middle and right
   mouse buttons respectively.

   For LORES, when the cycle is complete the last geometric shift
   determined is adopted and the scan is revoked.

   For HIRES, each cursor hit is automatically associated with an \'{e}chelle
   order in the range defined by the \verb+ORDERS+ parameter. The last shift is
   again adopted, but the scan is available for further display or
   measurement.
}
}

\newpage
\sstroutine{CLEAN}
{
   Mark pixels with values below a selected threshold as BAD.
}{
   \sstparameters{
   \cpar{DATASET}{Dataset name.}
   \cpar{THRESH}{Smallest pixel value to be accepted as GOOD.}
}
\sstdescription{
   Some IUE  datasets are effected  by horizontal  bars of low pixel values.
   These are caused by a weak  signal from the IUE craft at the time of data
   download.  When there are  a few bars  they can be marked as bad with the
   \verb+EDIMAGE+ command.  In the case of many bars a quicker solution is to
   mark all pixels in  the image below  a user selected  threshold value as BAD.

   Successive \verb+CLEAN+ and \verb+DRIMAGE+ commands starting with a value of
   \verb+THRESH=-1000+ and increasing \verb+THRESH+ towards zero will  allow
   the user to chose a value suitable for the problem image.
}
}

\sstroutine{CULIMITS}
{
   Set display limits with the cursor.
}{
   \sstparameters{
   \cpar{DEVICE}{GKS/SGS graphics device name.}
   \cpar{XL}{$x$-axis plotting limits, undefined or [0, 0] means auto-scale.}
   \cpar{YL}{$y$-axis plotting limits, undefined or [0, 0] means auto-scale.}
   \cpar{XP}{$x$-axis pixel limits, undefined or [0, 0] means full extent.}
   \cpar{YP}{$y$-axis pixel limits, undefined or [0, 0] means full extent.}
}
\sstdescription{
   This command uses the cursor to delineate part of a current display,
   graph or image, to be displayed in some subsequent command
   ({\it{e.g.,}}\ \verb+PLFLUX+, \verb+DRIMAGE+\ldots ).

   The two cursor positions should be at the corners of the required
   rectangular subset. The relation between cursor position sequences and
   axis reversals for graphs is:

   \begin{tabular}{llll}
   {\bf Position 1} & {\bf Position 2} & {\bf x-reversed} & {\bf y-reversed}\\
   bottom/left  & top/right    & NO  & NO \\
   bottom/right & top/left     & YES & NO \\
   top/left     & bottom/right & NO  & YES \\
   top/right    & bottom/left  & YES & YES
   \end{tabular}

   The \verb+XL+ and \verb+YL+ values are changed accordingly.

   In the case of an image display, the \verb+XP+ and \verb+YP+ parameter values
   are changed.
   The image will {\bf always} be drawn without axis reversals.
}
}

\sstroutine{CURSOR}
{
   Find display coordinates using the cursor and print them at the terminal.
}{
   \sstparameters{
   \npar{None.}
}
\sstdescription{
   This command uses the graphics cursor to find coordinates on a
   displayed graph or image.

   Pressing the left or middle  mouse button displays information about the
   pixel being  pointed to.  Pressing the right mouse button terminates the
   \verb+CURSOR+ cycle.  The coordinates for each hit are printed on the
   terminal; they correspond to the unit scale of the axes prevailing on the
   current diagram,  ({\it{e.g.,}}\ (wavelength,  flux)).
   If meaningful,  additional coordinate information is also printed.
   Keyboard keys 1, 2 and 3 may be used in place of left, middle and right
   mouse buttons respectively.
}
}

\sstroutine{DRIMAGE}
{
   Display an IUE image on an suitable graphics workstation.
}{
   \sstparameters{
   \cpar{DATASET}{Dataset name.}
   \cpar{DEVICE}{GKS/SGS graphics device name.}
   \cpar{XP}{$x$-axis pixel limits, undefined or [0, 0] means full extent.}
   \cpar{YP}{$y$-axis pixel limits, undefined or [0, 0] means full extent.}
   \cpar{ZL}{Data limits for image display, undefined means full range.}
   \cpar{COLOUR}{Whether a false colour look-up table is used.}
   \cpar{ZONE}{Zone to be used for plotting.}
   \cpar{FLAG}{Whether data quality for faulty pixels are displayed.}
}
\sstdescription{
   This command displays the image specified by the \verb+DATASET+
   parameter on the device specified by the \verb+DEVICE+ parameter.

   The part of the image displayed is specified by the \verb+XP+ and \verb+YP+
   parameter values.
   If unspecified, \verb+XP+ and \verb+YP+ default to the entire image extent,
   {\it{i.e.}}

   \begin {quote}
      \verb+XP = [1,768], YP = [1,768]+
   \end {quote}

   If the values of \verb+XP+ or \verb+YP+ are specified in decreasing order,
   the image will {\bf not} be reversed along the appropriate axis.

   The range of data values displayed as a grey scale is limited
   by the two values of the \verb+ZL+ parameter.
   Data values at or below \verb+ZL[1]+ will appear {\bf black},
   those at \verb+ZL[2]+ will appear {\bf white} and those above \verb+ZL[2]+
   will appear {\bf blue}.
   If the \verb+ZL+ values are given in decreasing order, then high data
   values will be represented by low (dark) display intensities,
   and vice-versa.
   If the values are undefined, then the full intensity range of the
   image will be used.
   The full intensity range of the image can be found using the command

   \begin {quote}
      \verb+> SHOW V=I+
   \end {quote}

   The \verb+FLAG+ parameter specifies whether faulty pixels are flagged using
   the following colour scheme:

   \begin{description}
      \item GREEN --- pixels affected by reseau marks
      \item RED --- pixels which are saturated (DN=255)
      \item ORANGE --- pixels affected by ITF truncation
      \item YELLOW --- pixels marked bad by the user
   \end{description}

   If a pixel is affected by more than one of the above faults, then
   the first in the list is adopted for display.

   The \verb+ZONE+ parameter is accepted by \verb+DRIMAGE+ but is ignored, the
   display always using \verb+ZONE=0+\@.
}
}

\sstroutine{EDIMAGE}
{
   Edit the image data quality using the graphics cursor.
}{
   \sstparameters{
   \cpar{DATASET}{Dataset name.}
   \cpar{DEVICE}{GKS/SGS graphics device name.}
}
\sstdescription{
   This command uses the image display cursor to mark pixels and
   regions of the current image that are ``bad'' or ``good''.
   The image should have previously been displayed using the
   \verb+DRIMAGE+ command.
   So that faulty pixels can be seen, the \verb+FLAG=TRUE+ option in
   \verb+DRIMAGE+ should be used.

   The image display is specified by the \verb+DEVICE+ parameter and the
   associated dataset by the \verb+DATASET+ parameter.

   The following cursor hit sequences can be used in a cycle:

   \begin {description}
      \item 1 then 1 --- marks all pixels in the rectangle GOOD.
      \item 2 then 2 --- marks all points in the rectangle BAD.
      \item 1 --- marks the nearest pixel GOOD.
      \item 2 --- marks the nearest pixel BAD.
      \item 3 --- causes the cursor cycle to terminate.
   \end {description}

   Mouse buttons can be used for cursor hits where:

   \begin {description}
      \item {\bf left} mouse button     is  hit 1.
      \item {\bf middle} mouse button   is  hit 2.
      \item {\bf right} mouse button    is  hit 3.
   \end{description}

   Alternatively, keyboard keys 1, 2 and 3 can be used to mark hits.

   The pixels or ranges changed are printed on the terminal.
   The term ``rectangle'' is used above to indicate a rectangular
   set of pixels delineated by the two cursor positions.
   Thus, for the first hit, the cursor can be positioned at the
   bottom left corner, and for the second at the top right corner.

   Only the user-defined data quality bit can be changed by this
   command.
   Initially, all faulty pixels have this bit set BAD, so that
   spectrum extraction (say) can ignore these where appropriate.
   However, the user-defined data quality can also be set GOOD.

   See the IUEDR User Guide (MUD/45) for further information on data quality.
}
}

\sstroutine{EDMEAN}
{
   Edit the mean extracted spectrum using the graphics cursor.
}{
   \sstparameters{
   \cpar{DATASET}{Dataset name.}
   \cpar{DEVICE}{GKS/SGS graphics device name.}
}
\sstdescription{
   This command uses the graphics cursor to mark points and
   regions of the mean spectrum that are ``bad'' or ``good''.

   The following cursor hit sequences can be used in a cycle:

   \begin {description}
      \item 1 then 1 --- marks all points in the $x$-range GOOD.
      \item 2 then 2 --- marks all points in the $x$-range BAD.
      \item 1 --- marks the nearest point in $x$-direction GOOD.
      \item 2 --- marks the nearest point in $x$-direction BAD.
      \item 3 --- causes the cursor cycle to terminate.
   \end {description}

   The points or ranges changed are printed on the terminal.

   Mouse buttons can be used for cursor hits where:

   \begin {description}
      \item {\bf left} mouse button     is  hit 1.
      \item {\bf middle} mouse button   is  hit 2.
      \item {\bf right} mouse button    is  hit 3.
   \end{description}

   Alternatively, keyboard keys 1, 2 and 3 can be used to mark hits.

   See the IUEDR User Guide (MUD/45) for further information on data quality.
}
}

\sstroutine{EDSPEC}
{
   Edit the net extracted spectrum using the graphics cursor.
}{
   \sstparameters{
   \cpar{DATASET}{Dataset name.}
   \cpar{ORDER}{\'{E}chelle order number.}
   \cpar{APERTURE}{Aperture name (\verb+SAP+ or \verb+LAP+).}
   \cpar{DEVICE}{GKS/SGS graphics device name.}
}
\sstdescription{
   This command uses the graphics cursor to mark points and
   regions of the current net spectrum that are ``bad'' or ``good''.
   A plot of the \verb+APERTURE+ or \verb+ORDER+ spectrum is required before
   this command can be used.

   The following cursor hit sequences can be used in a cycle:

   \begin {description}
      \item 1 then 1 --- marks all points in the $x$-range GOOD.
      \item 2 then 2 --- marks all points in the $x$-range BAD.
      \item 1 --- marks the nearest point in $x$-direction GOOD.
      \item 2 --- marks the nearest point in $x$-direction BAD.
      \item 3 --- causes the cursor cycle to terminate.
   \end {description}

   The points or ranges changed are printed on the terminal.

   Mouse buttons can be used for cursor hits where:

   \begin {description}
      \item {\bf left} mouse button     is  hit 1.
      \item {\bf middle} mouse button   is  hit 2.
      \item {\bf right} mouse button    is  hit 3.
   \end{description}

   Alternatively, keyboard keys 1, 2 and 3 can be used to mark hits.

   Only the user-defined data quality bit can be changed by this
   command.
   Initially, all faulty points have this bit set BAD ({\it{e.g.,}}\ by
   \verb+TRAK+)\@. However, whether they are considered bad ({\it{e.g.,}}\ when
   plotting or creating output files) is determined by the user-defined
   bit, which can be changed at will.

   See the IUEDR User Guide (MUD/45) for further information on data quality.
}
}

\sstroutine{ERASE}
{
   Erase the display screen of the graphics device.
}{
   \sstparameters{
   \cpar{DEVICE}{GKS/SGS graphics device name.}
}
\sstdescription{
   The display screen of the specified graphics device is erased.
}
}

\sstroutine{EXIT}
{
   Quit IUEDR.
}{
   \sstparameters{
   \npar{None.}
}
\sstdescription{
   This command quits IUEDR\@.
   Any files that require new versions will be written by this command.
   This command is a synonym for the \verb+QUIT+ command.
}
}

\sstroutine{HELP}
{
   Find out about IUEDR commands and parameters.
}{
   \sstparameters{
   \npar{None.}
}
\sstdescription{
   By simply typing \verb+HELP+ \label{com: HELP }the user is presented with
   a brief introduction to IUEDR, a list of the commands available and some
   general information topics. Users familiar with he VMS help system will
   find this facility very essentially the same to use.

   The \verb+HELP+ system provides a list of topics which can be
   selected from by typing enough characters of a topic name to uniquely
   identify it and pressing return.  Pressing the return key with no topic
   chosen takes the \verb+HELP+ system back one topic-level.
   At any time, pressing the return key a few times will return you to the
   IUEDR prompt.

   You may optionally give a specific topic to the \verb+HELP+ command at the
   IUEDR prompt, for example
   \begin{quote}
      \verb+> HELP DRIMAGE+
   \end{quote}
   or even
   \begin{quote}
      \verb+> HELP DRIMAGE COLOUR+
   \end{quote}
}
}

\sstroutine{LBLS}
{
   Extracts a line-by-line-spectrum array from the image.
}{
   \sstparameters{
   \cpar{DATASET}{Dataset name.}
   \cpar{ORDER}{\'{E}chelle order number.}
   \cpar{APERTURE}{Aperture name (\verb+SAP+ or \verb+LAP+).}
   \cpar{GSAMP}{Spectrum grid sampling rate (geometric pixels).}
   \cpar{CUTWV}{Whether wavelength cutoff data used for extraction grid.}
   \cpar{CENTM}{Whether pre-existing centroid template is used.}
   \cpar{RL}{Limits across spectrum for LBLS array (pixels).}
   \cpar{RSAMP}{Radial coordinate sampling rate for LBLS grid (pixels).}
}
\sstdescription{
   This command creates a line-by-line-spectrum (LBLS) array from the
   image defined by \verb+DATASET+\@.
   The array consists of intensities $F(IR, I\lambda )$ for a grid of
   wavelengths, $W(I\lambda)$, and radial coordinates, $R(IR)$\@.
   The wavelength grid, $\lambda$, is determined in a similar way to the
   \verb+TRAK+ command, using the \verb+CUTWV+ (HIRES) and \verb+GSAMP+
   (HIRES/LORES) parameters.

   The radial coordinates are distances from the centre of the spectrum,
   derived from the template data,
   along a line perpendicular to the dispersion direction and
   measured in geometric pixels.
   The radial grid, $R$, is determined by the \verb+RL+ and \verb+RSAMP+
   parameters.

   The value of each pixel in the array corresponds to the surface
   over the image of a rectangle centred on its $(R, \lambda )$ coordinates,
   and extents

   \begin {equation}
      (R(IR) - dR / 2, R(IR) + dR / 2)
   \end {equation}
   and

   \begin {equation}
      (W(I\lambda ) - d\lambda / 2, W(I\lambda ) + d\lambda / 2)
   \end {equation}
   $dR$ is the distance between $R$ values, and $d\lambda$ is the wavelength
   step between $\lambda$ values.

   This surface integral is scaled along the $\lambda$ direction to
   correspond to an interval of 1.414 geometric pixels.
   The reason for this is to make LBLS intensities consistent with
   those produced by the \verb+TRAK+ command.
   For a particular wavelength, $W(I\lambda )$, the sum of LBLS intensities
   after removal of background should correspond to the net
   flux as measured by \verb+TRAK+\@.
}
}

\sstroutine{LISTIUE}
{
   Analyse the contents of IUE tapes or files.
}{
   \sstparameters{
   \cpar{DRIVE}{Tape drive or file name.}
   \cpar{FILE}{Tape file number.}
   \cpar{NFILE}{Number of tape files to be processed.}
   \cpar{NLINE}{Number of IUE header lines printed.}
   \cpar{SKIPNEXT}{Whether skip to next tape file.}
}
\sstdescription{
   This performs an analysis of \verb+NFILE+ IUE tape files, starting at
   the file specified by the \verb+FILE+ parameter.
   \verb+NFILE=-1+ means list all files until the end of the tape.
   \verb+NLINE=-1+ means print all lines in file header.

   \verb+LISTIUE+ can also be used to list the header of a GO format disk file.
}
}

\newpage
\sstroutine{MAP}
{
   Map and merge the extracted spectrum components to produce a mean spectrum.
}{
   \sstparameters{
   \cpar{DATASET}{Dataset name.}
   \cpar{ORDERS}{This delineates a range of \'{e}chelle orders.}
   \cpar{APERTURE}{Aperture name (\verb+SAP+ or \verb+LAP+).}
   \cpar{RM}{Whether mean spectrum is reset before averaging.}
   \cpar{ML}{Wavelength grid limits for mean spectrum.}
   \cpar{MSAMP}{Wavelength sampling rate for mean spectrum grid.}
   \cpar{FILLGAP}{Whether gaps can be filled within order.}
   \cpar{COVERGAP}{Whether gaps can be filled by covering orders.}
}
\sstdescription{
   This command can be used to produce a mean spectrum with contributions
   from several \'{e}chelle orders (HIRES), or from several apertures (LORES).

   If \verb+RM=TRUE+, or if there is no existing mean spectrum, then an
   evenly spaced wavelength grid is constructed between the
   limits specified by the \verb+ML+ parameter using the sampling rate
   specified by the \verb+MSAMP+ parameter.

   If \verb+RM=FALSE+ and there {\bf is}
   an existing mean spectrum, then the
   wavelength grid {\bf and contents} are retained.
   New components will be averaged with what is already there.

   In the case of HIRES, the \verb+ORDERS+ parameter is used to delimit the
   range of \'{e}chelle orders that are allowed to contribute to the mean.

   In the case of LORES, only a single aperture specified by the
   \verb+APERTURE+ parameter is mapped at a given time.
}
}

\sstroutine{MODIMAGE}
{
   Modifies image pixel intensities interactively.
}{
   \sstparameters{
   \cpar{DATASET}{Dataset name.}
   \cpar{DEVICE}{GKS/SGS graphics device name.}
   \cpar{FN}{Replacement Flux Number for pixel.}
}
\sstdescription{
   This command uses the image display cursor to modify image data.
   The image should already have been displayed using the \verb+DRIMAGE+
   command.

   The following cursor sequences are adopted:

   \begin {description}
      \item 1 then 2 --- copy intensity of first picked pixel to the second.
      \item 2 --- prompt for replacement pixel intensity.
      \item 3 --- finish.
   \end {description}

   Mouse buttons can be used for cursor hits where:

   \begin {description}
      \item {\bf left} mouse button     is  hit 1.
      \item {\bf middle} mouse button   is  hit 2.
      \item {\bf right} mouse button    is  hit 3.
   \end{description}

   Alternatively, keyboard keys 1, 2 and 3 can be used to mark hits.

   If the data or data qualities change after a session, then the file is
   saved on disk.

   The assumption is made that the current image displayed corresponds
   to the current dataset!
}
}

\sstroutine{MTMOVE}
{
   Move to the start of a tape file.
}{
   \sstparameters{
   \cpar{DRIVE}{Tape drive.}
   \cpar{FILE}{Tape file number.}
}
\sstdescription{
   Move to the start of the file specified by the \verb+FILE+ parameter on the
   tape specified by the \verb+DRIVE+ parameter.
}
}

\sstroutine{MTREW}
{
   Rewind to the start of the tape.
}{
   \sstparameters{
   \cpar{DRIVE}{Tape drive.}
}
\sstdescription{
   This command rewinds the tape specified by the \verb+DRIVE+ parameter.
   The \verb+FILE+ parameter is also set to 1 by this command.
}
}

\sstroutine{MTSHOW}
{
   Show the current tape position.
}{
   \sstparameters{
   \cpar{DRIVE}{Tape drive.}
}
\sstdescription{
   This command displays the current tape position.
   This includes the file number and the block position relative to either
   the start or the end of the file.

   Note that the actual file position may differ from the
   value of the \verb+FILE+ parameter.
}
}

\sstroutine{MTSKIPEOV}
{
   Skip over end-of-volume (EOV) mark.
}{
   \sstparameters{
   \cpar{DRIVE}{Tape drive.}
}
\sstdescription{
   This command skips over an end-of-volume (EOV) mark on the tape specified
   by the DRIVE parameter.
   An EOV condition is where there are two consecutive tape marks.
   When attempting to skip across an EOV, an error will be reported
   and the tape left positioned between the two marks.
   Subsequent attempts to skip forward will fail and
   only this command can be used to move forward beyond the
   second tape mark.
}
}

\sstroutine{MTSKIPF}
{
   Skip over NSKIP tape marks.
}{
   \sstparameters{
   \cpar{DRIVE}{Tape drive.}
   \cpar{NSKIP}{Number of tape marks to be skipped over.}
}
\sstdescription{
   This command skips over \verb+NSKIP+ tape marks on the tape specified
   by the \verb+DRIVE+ parameter.
   If \verb+NSKIP+ is negative this means that tape marks are skipped in the
   reverse direction, {\it{i.e.}}\ towards the start of the tape.
}
}

\sstroutine{NEWABS}
{
   Associate a new absolute flux calibration with the current dataset.
}{
   \sstparameters{
   \cpar{DATASET}{Dataset name.}
   \cpar{ABSFILE}{Name of file containing absolute flux calibration.}
}
\sstdescription{
   This command reads the absolute flux calibration from a text file
   specified by the \verb+ABSFILE+ parameter and stores it in the dataset
   specified by \verb+DATASET+\@.

   The file type is assumed to be \verb+.abs+ and need not be
   specified as part of the \verb+ABSFILE+ parameter.

   The calibration of any current spectrum is automatically updated.
}
}

\newpage
\sstroutine{NEWCUT}
{
   Associate new \'{e}chelle order wavelength limits with the current
   dataset.
}{
   \sstparameters{
   \cpar{DATASET}{Dataset name.}
   \cpar{CUTFILE}{Name of file containing \'{e}chelle order wavelength limits.}
}
\sstdescription{
   This command reads the \'{e}chelle order wavelength limits from a text file
   specified by the \verb+CUTFILE+ parameter and stores them in the dataset
   specified by \verb+DATASET+\@.

   The file type is assumed to be \verb+.cut+ and need not be
   specified as part of the \verb+CUTFILE+ parameter.

   The calibration of any current spectrum is automatically updated.
}
}

\sstroutine{NEWDISP}
{
   Associate new spectrograph dispersion data with the current dataset.
}{
   \sstparameters{
   \cpar{DATASET}{Dataset name.}
   \cpar{DISPFILE}{Name of file containing dispersion data.}
}
\sstdescription{
   This command reads spectrograph dispersion data from the text file
   specified by the \verb+DISPFILE+ parameter and stores them in the dataset
   specified by \verb+DATASET+\@.

   The file type is assumed to be \verb+.dsp+ and need not be specified
   as part of the \verb+DISPFILE+ parameter.
}
}

\sstroutine{NEWFID}
{
   Read IUE fiducial positions from text file.
}{
   \sstparameters{
   \cpar{DATASET}{Dataset name.}
   \cpar{FIDFILE}{Name of file containing fiducial positions.}
   \cpar{NGEOM}{Number of Chebyshev terms used to represent geometry.}
}
\sstdescription{
   This command reads IUE fiducial positions from a text file
   specified by the \verb+FIDFILE+ parameter and stores them in the dataset
   specified by \verb+DATASET+\@.

   The file type is assumed to be \verb+.fid+ and need not be specified
   as part of the FIDFILE parameter.

   The image data quality and geometry representation are updated to
   account for any changes that these fiducial positions imply.
   In the case of datasets containing image distortion, the \verb+NGEOM+
   parameter is used to specify the number of terms used for the Chebyshev
   representation along each axis.
}
}

\sstroutine{NEWRIP}
{
   Read \'{e}chelle ripple calibration from text file.
}{
   \sstparameters{
   \cpar{DATASET}{Dataset name.}
   \cpar{RIPFILE}{Name of file containing \'{e}chelle ripple calibration.}
}
\sstdescription{
   This command reads an \'{e}chelle ripple calibration from a text file
   specified by the \verb+RIPFILE+ parameter and stores it in the dataset
   specified by \verb+DATASET+\@.

   The file type is assumed to be \verb+.rip+ and need not be specified
   as part of the \verb+RIPFILE+ parameter.

   The calibration of any current spectrum is automatically updated.
}
}

\sstroutine{NEWTEM}
{
   Read spectrum centroid template data from text file.
}{
   \sstparameters{
   \cpar{DATASET}{Dataset name.}
   \cpar{TEMFILE}{Name of file containing spectrum template data.}
}
\sstdescription{
   This command reads the spectrum centroid template data into \verb+DATASET+
   from a text file with name specified by \verb+TEMFILE+\@.

   The file type is assumed to be \verb+.tem+ and need not be specified
   as part of the \verb+TEMFILE+\@.
}
}

\sstroutine{OUTEM}
{
   Output the current spectrum template data to a formatted data file.
}{
   \sstparameters{
   \cpar{DATASET}{Dataset name.}
   \cpar{TEMFILE}{Name of file containing spectrum template data.}
}
\sstdescription{
   This command outputs the templates stored with the current dataset to a
   text file.
   If not specified, the file name is constructed as:

   \begin {quote}
      \verb+<CAMERA>HI<APERTURE>.TEM+
   \end {quote}
   or

   \begin {quote}
      \verb+<CAMERA>LO.TEM+
   \end {quote}

   for the HIRES and LORES cases respectively.
}
}

\sstroutine{OUTLBLS}
{
   Output the current LBLS array to a binary data file.
}{
   \sstparameters{
   \cpar{DATASET}{Dataset name.}
   \cpar{OUTFILE}{Name of output file.}
}
\sstdescription{
   This command outputs the current LBLS array to a file.
   If not specified by the \verb+OUTFILE+ parameter, the file name is
   constructed as:

   \begin {quote}
      \verb+<CAMERA><IMAGE>R.DAT+
   \end {quote}
   The format of this file is described by the Fortran 77 routine, RDLBLS, which
   can be found in the file:

   \begin {quote}
      {\tt \$IUEDR\_USER/rdlbls.for}
   \end {quote}
   The directory {\tt \$IUEDR\_USER} also contains a test
   program for using RDLBLS and other helpful items.
}
}

\sstroutine{OUTMEAN}
{
   Output current mean spectrum to a DIPSO SP format file.
}{
   \sstparameters{
   \cpar{DATASET}{Dataset name.}
   \cpar{OUTFILE}{Name of output file.}
   \cpar{SPECTYPE}{DIPSO SP file type (0, 1 or 2).}
}
\sstdescription{
   This command outputs the mean spectrum associated with
   \verb+DATASET+ to a file that can be read into DIPSO
   (see \xref{SUN/50}{sun50}{}).

   This file is created with type specified by the \verb+SPECTYPE+ parameter
   (see Section~\ref{se:spectrum}
   for SP options).
   If not specified, the file name is constructed as:

   \begin {quote}
      \verb+<CAMERA><IMAGE>M.sdf+ \hspace*{8mm} for SP0\\
      \verb+<CAMERA><IMAGE>M.DAT+ \hspace*{8mm} for SP1 or SP2
   \end {quote}
   In DIPSO SP format, bad points are indicated by having zero intensities.
   In determining which points in the output file are to be marked
   ``bad'', the user-defined data quality bit is used.
   Since this bit can be arbitrarily edited,
   faulty data values can be written to the output file
   without subsequent information being retained.
}
}

\sstroutine{OUTNET}
{
   Output the current net spectrum to a DIPSO  SP format data file.
}{
   \sstparameters{
   \cpar{DATASET}{Dataset name.}
   \cpar{APERTURE}{Aperture name (\verb+SAP+ or \verb+LAP+).}
   \cpar{ORDER}{\'{E}chelle order number.}
   \cpar{OUTFILE}{Name of output file.}
   \cpar{SPECTYPE}{DIPSO SP file type (0, 1 or 2).}
}
\sstdescription{
   This command outputs the net spectrum associated with \verb+ORDER+ or
   \verb+APERTURE+ and \verb+DATASET+ to a file that can be read into DIPSO
   (see \xref{SUN/50}{sun50}{}).

   The file is created with type specified
   by the \verb+SPECTYPE+ parameter (see Section~\ref{se:spectrum}
   for SP
   options).
   If not specified, the file name is constructed as:

   \begin {quote}
      \verb+<CAMERA><IMAGE>+\_\verb+<APERTURE>.sdf+ \hspace*{8mm} for SP0\\
      \verb+<CAMERA><IMAGE>.<APERTURE>+ \hspace*{8mm} for SP1 or SP2
   \end {quote}
   in the case of LORES and

   \begin {quote}
      \verb+<CAMERA><IMAGE>+\_\verb+<ORDER>.sdf+ \hspace*{8mm} for SP0\\
      \verb+<CAMERA><IMAGE>.<ORDER>+ \hspace*{8mm} for SP1 or SP2
   \end {quote}
   in the case of HIRES.
   Here, \verb+<APERTURE>+ is the aperture name (\verb+SAP+ or \verb+LAP+),
   or index, and \verb+<ORDER>+ is the \'{e}chelle order number.

   In DIPSO SP format, bad points are indicated by having zero intensities.
   In determining which points in the output file are to be marked
   ``bad'', the user-defined data quality bit is used.
   Since this bit can be arbitrarily edited,
   faulty data values can be written to the output file
   without subsequent information being retained.
}
}

\sstroutine{OUTRAK}
{
   Output the current uncalibrated spectrum to a ``TRAK'' formatted data file.
}{
   \sstparameters{
   \cpar{DATASET}{Dataset name.}
   \cpar{OUTFILE}{Name of output file.}
}
\sstdescription{
   This command outputs the uncalibrated spectrum associated with
   \verb+DATASET+ to a formatted file that is compatible with output
   from the old ``TRAK'' program.
   The default file name is of the form:

   \begin {quote}
      \verb+<CAMERA><IMAGE>.TRK+
   \end {quote}
   The main difference from an actual ``TRAK'' file is that the background
   level is uniformly zero, so that GROSS=NET.
}
}

\sstroutine{OUTSCAN}
{
   Output the current scan data to a DIPSO  SP format data file.
}{
   \sstparameters{
   \cpar{DATASET}{Dataset name.}
   \cpar{OUTFILE}{Name of output file.}
   \cpar{SPECTYPE}{DIPSO SP file type (0, 1 or 2).}
}
\sstdescription{
   This command outputs the current scan associated with
   \verb+DATASET+ to a file which can be read into DIPSO
   (see \xref{SUN/50}{sun50}{}).

   The file is created with type specified
   by the \verb+SPECTYPE+ parameter
   (see Section~\ref{se:spectrum}
   for SP options).
   If not specified, the file name is constructed as:

   \begin {quote}
      \verb+<CAMERA><IMAGE>P.sdf+ \hspace*{8mm} for SP0\\
      \verb+<CAMERA><IMAGE>P.DAT+ \hspace*{8mm} for SP1 or SP2
   \end {quote}
   In DIPSO SP format, bad points are indicated by having zero intensities.
   In determining which points in the output file are to be marked
   ``bad'', the user-defined data quality bit is used.
   Since this bit can be arbitrarily edited,
   faulty data values can be written to the output file
   without subsequent information being retained.
}
}

\sstroutine{OUTSPEC}
{
   Output the current aperture or order spectrum to a DIPSO SP format data file.
}{
   \sstparameters{
   \cpar{DATASET}{Dataset name.}
   \cpar{APERTURE}{Aperture name (\verb+SAP+ or \verb+LAP+).}
   \cpar{ORDER}{\'{E}chelle order number.}
   \cpar{OUTFILE}{Name of output file.}
   \cpar{SPECTYPE}{DIPSO SP file type (0, 1 or 2).}
}
\sstdescription{
   This command outputs the spectrum associated with the \verb+ORDER+ or
   \verb+APERTURE+ and \verb+DATASET+ to a file which can be read into DIPSO
   (see \xref{SUN/50}{sun50}{}).

   The file is created with type specified
   by the \verb+SPECTYPE+ parameter
   (see Section~\ref{se:spectrum}
   for SP options).
   If not specified, the file name is constructed as:

   \begin {quote}
      \verb+<CAMERA><IMAGE>+\_\verb+<APERTURE>.sdf+ \hspace*{8mm} for SP0\\
      \verb+<CAMERA><IMAGE>.<APERTURE>+ \hspace*{8mm} for SP1 or SP2
   \end {quote}
   in the case of LORES and

   \begin {quote}
      \verb+<CAMERA><IMAGE>+\_\verb+<ORDER>.sdf+ \hspace*{8mm} for SP0\\
      \verb+<CAMERA><IMAGE>.<ORDER>+ \hspace*{8mm} for SP1 or SP2
   \end {quote}
   in the case of HIRES.
   Here, \verb+<APERTURE>+ is the aperture name (\verb+SAP+ or \verb+LAP+), or
   index, and \verb+<ORDER>+ is the \'{e}chelle order number.

   In DIPSO SP format, bad points are indicated by having zero intensities.
   In determining which points in the output file are to be marked
   ``bad'', the user-defined data quality bit is used.
   Since this bit can be arbitrarily edited,
   faulty data values can be written to the output file
   without subsequent information being retained.
}
}

\sstroutine{PLCEN}
{
   Plot smoothed centroid shifts.
}{
   \sstparameters{
   \cpar{DATASET}{Dataset name.}
   \cpar{ORDER}{\'{E}chelle order number.}
   \cpar{APERTURE}{Aperture name (\verb+SAP+ or \verb+LAP+).}
   \cpar{RS}{Whether display is reset before plotting.}
   \cpar{DEVICE}{GKS/SGS graphics device name.}
   \cpar{ZONE}{Zone to be used for plotting.}
   \cpar{LINE}{Plotting line style (\verb+SOLID+, \verb+DASH+, \verb+DOTDASH+
               or \verb+DOT+).}
   \cpar{LINEROT}{Whether line style is changed after next plot.}
   \cpar{COL}{Plotting line colour (1, 2, 3, \dots 10).}
   \cpar{COLROT}{Whether line colour is changed after next plot.}
   \cpar{XL}{$x$-axis plotting limits, undefined or [0, 0] means auto-scale.}
   \cpar{YL}{$y$-axis plotting limits, undefined or [0, 0] means auto-scale.}
}
\sstdescription{
   This command plots the smoothed centroid shifts produced during
   the most recent spectrum extraction from \verb+DATASET+
   on the graphics device and zone specified by the \verb+DEVICE+ and
   \verb+ZONE+ parameters respectively.

   In the case of a LORES spectrum, if there is more than a single
   aperture available, then the \verb+APERTURE+ parameter needs to be specified.

   In the case of a HIRES spectrum, if there is more than a single
   \'{e}chelle order, then the \verb+ORDER+ parameter needs to be specified.

   The \verb+RS+ parameter specifies whether a new plot is started, or whether
   the data can be plotted over an existing plot.

   The \verb+LINE+ and \verb+LINEROT+ parameters determine the line style
   which will be used for plotting.

   The \verb+COL+ and \verb+COLROT+ parameters determine the line colour which
   will be used for plotting if the \verb+DEVICE+ supports colour graphics.

   The diagram limits are specified by the \verb+XL+ and \verb+YL+ parameter
   values.
   If \verb+XL+ and \verb+YL+ have values

   \begin {quote}
      \verb+XL=[0,0], YL=[0,0]+
   \end {quote}
   then the plot limits along each axis are determined so that the whole
   spectrum is visible.
   If the values of \verb+XL+ or \verb+YL+ are specified in decreasing order,
   then the coordinates will be reversed along the appropriate axis.
}
}

\sstroutine{PLFLUX}
{
   Plot calibrated flux spectrum.
}{
   \sstparameters{
   \cpar{DATASET}{Dataset name.}
   \cpar{ORDER}{\'{E}chelle order number.}
   \cpar{APERTURE}{Aperture name (\verb+SAP+ or \verb+LAP+).}
   \cpar{RS}{Whether display is reset before plotting.}
   \cpar{DEVICE}{GKS/SGS graphics device name.}
   \cpar{ZONE}{Zone to be used for plotting.}
   \cpar{LINE}{Plotting line style (\verb+SOLID+, \verb+DASH+, \verb+DOTDASH+
               or \verb+DOT+).}
   \cpar{LINEROT}{Whether line style is changed after next plot.}
   \cpar{COL}{Plotting line colour (1, 2, 3, \dots 10).}
   \cpar{COLROT}{Whether line colour is changed after next plot.}
   \cpar{HIST}{Whether lines are drawn as histograms.}
   \cpar{QUAL}{Whether data quality information is plotted.}
   \cpar{XL}{$x$-axis plotting limits, undefined or [0, 0] means auto-scale.}
   \cpar{YL}{$y$-axis plotting limits, undefined or [0, 0] means auto-scale.}
}
\sstdescription{
   This command plots the calibrated flux spectrum from \verb+DATASET+
   on the graphics device and zone specified by the \verb+DEVICE+ and
   \verb+ZONE+ parameters respectively.

   In the case of a LORES spectrum, if there is more than a single
   aperture available, then the \verb+APERTURE+ parameter needs to be specified.

   In the case of a HIRES spectrum, if there is more than a single
   \'{e}chelle order, then the \verb+ORDER+ parameter needs to be specified.

   The \verb+RS+ parameter specifies whether a new plot is started, or whether
   the data can be plotted over an existing plot.

   The \verb+HIST+ parameter determines whether the line is drawn as a
   histogram rather than a continuous polyline.

   The \verb+LINE+ and \verb+LINEROT+ parameters determine the line style
   which will be used for plotting.

   The \verb+COL+ and \verb+COLROT+ parameters determine the line colour which
   will be used for plotting if the \verb+DEVICE+ supports colour graphics.

   The diagram limits are specified by the \verb+XL+ and \verb+YL+ parameter
   values.
   If \verb+XL+ and \verb+YL+ have values

   \begin {quote}
      \verb+XL=[0,0], YL=[0,0]+
   \end {quote}
   then the plot limits along each axis are determined so that the whole
   spectrum is visible.
   If the values of \verb+XL+ or \verb+YL+ are specified in decreasing order,
   then the coordinates will be reversed along the appropriate axis.

   The \verb+QUAL+ parameter indicates whether faulty points are flagged with
   their data quality codes (see Section~\ref{se:introduction}).

   If a point is affected by more than one of the above faults, then
   the highest code is plotted.
   Points marked bad by user edits are only indicated if they are otherwise
   fault-free.
}
}

\sstroutine{PLGRS}
{
   Plot pseudo-gross and background from spectrum extraction.
}{
   \sstparameters{
   \cpar{DATASET}{Dataset name.}
   \cpar{ORDER}{\'{E}chelle order number.}
   \cpar{APERTURE}{Aperture name (\verb+SAP+ or \verb+LAP+).}
   \cpar{RS}{Whether display is reset before plotting.}
   \cpar{DEVICE}{GKS/SGS graphics device name.}
   \cpar{ZONE}{Zone to be used for plotting.}
   \cpar{LINE}{Plotting line style (\verb+SOLID+, \verb+DASH+, \verb+DOTDASH+
               or \verb+DOT+).}
   \cpar{LINEROT}{Whether line style is changed after next plot}
   \cpar{COL}{Plotting line colour (1, 2, 3, \ldots 10).}
   \cpar{COLROT}{Whether line colour is changed after next plot.}
   \cpar{HIST}{Whether lines are drawn as histograms.}
   \cpar{QUAL}{Whether data quality information is plotted.}
   \cpar{XL}{$x$-axis plotting limits, undefined or [0, 0] means auto-scale.}
   \cpar{YL}{$y$-axis plotting limits, undefined or [0, 0] means auto-scale.}
}
\sstdescription{
   This command plots the pseudo-gross and smooth background produced during
   the most recent spectrum extraction from \verb+DATASET+
   on the graphics device and zone specified by the \verb+DEVICE+ and
   \verb+ZONE+ parameters respectively.

   The pseudo-gross is constructed by taking the net spectrum and adding
   the smooth background multiplied by the width of the object channel.
   The smooth background plotted is also for the object channel width.

   In the case of a LORES spectrum, if there is more than a single
   aperture available, then the \verb+APERTURE+ parameter needs to be specified.

   In the case of a HIRES spectrum, if there is more than a single
   \'{e}chelle order, then the \verb+ORDER+ parameter needs to be specified.

   The \verb+RS+ parameter specifies whether a new plot is started, or whether
   the data can be plotted over an existing plot.

   The \verb+HIST+ parameter determines whether the line is drawn as a histogram
   rather than a continuous polyline.

   The \verb+LINE+ and \verb+LINEROT+ parameters determine the line style
   which will be used for plotting.

   The \verb+COL+ and \verb+COLROT+ parameters determine the line colour which
   will be used for plotting if the \verb+DEVICE+ supports colour graphics.

   The diagram limits are specified by the \verb+XL+ and \verb+YL+ parameter
   values.
   If \verb+XL+ and \verb+YL+ have values

   \begin {quote}
      \verb+XL=[0,0], YL=[0,0]+
   \end {quote}
   then the plot limits along each axis are determined so that the whole
   spectrum is visible.
   If the values of \verb+XL+ or \verb+YL+ are specified in decreasing order,
   then the coordinates will be reversed along the appropriate axis.

   The \verb+QUAL+ parameter indicates whether faulty points are flagged with
   their data quality codes (see Section~\ref{se:introduction}).

   If a point is affected by more than one of the above faults, then
   the highest code is plotted.
   Points marked bad by user edits are only indicated if they are otherwise
   fault-free.
}
}

\sstroutine{PLMEAN}
{
   Plot mean spectrum.
}{
   \sstparameters{
   \cpar{DATASET}{Dataset name.}
   \cpar{RS}{Whether display is reset before plotting.}
   \cpar{DEVICE}{GKS/SGS graphics device name.}
   \cpar{ZONE}{Zone to be used for plotting.}
   \cpar{LINE}{Plotting line style (\verb+SOLID+, \verb+DASH+, \verb+DOTDASH+
               or \verb+DOT+).}
   \cpar{LINEROT}{Whether line style is changed after next plot}
   \cpar{COL}{Plotting line colour (1, 2, 3, \ldots 10).}
   \cpar{COLROT}{Whether line colour is changed after next plot.}
   \cpar{HIST}{Whether lines are drawn as histograms.}
   \cpar{QUAL}{Whether data quality information is plotted.}
   \cpar{XL}{$x$-axis plotting limits, undefined or [0, 0] means auto-scale.}
   \cpar{YL}{$y$-axis plotting limits, undefined or [0, 0] means auto-scale.}
}
\sstdescription{
   This command plots the mean spectrum associated with \verb+DATASET+ on the
   graphics device and zone specified by the \verb+DEVICE+ and \verb+ZONE+
   parameters respectively.

   The \verb+RS+ parameter specifies whether a new plot is started, or whether
   the data can be plotted over an existing plot.

   The \verb+HIST+ parameter determines whether the line is drawn as a histogram
   rather than a continuous polyline.

   The \verb+LINE+ and \verb+LINEROT+ parameters determine the line style which
   will be used for plotting.

   The \verb+COL+ and \verb+COLROT+ parameters determine the line colour which
   will be used for plotting if the \verb+DEVICE+ supports colour graphics.

   The diagram limits are specified by the \verb+XL+ and \verb+YL+ parameter
   values.
   If \verb+XL+ and \verb+YL+ have values

   \begin {quote}
      \verb+XL=[0,0], YL=[0,0]+
   \end {quote}
   then the plot limits along each axis are determined so that the whole
   spectrum is visible.
   If the values of \verb+XL+ or \verb+YL+ are specified in decreasing order,
   then the coordinates will be reversed along the appropriate axis.

   The \verb+QUAL+ parameter indicates whether faulty points are flagged with
   their data quality codes (see Section~\ref{se:introduction}).

   If a point is affected by more than one of the above faults, then
   the highest code is plotted.
   Points marked bad by user edits are only indicated if they are otherwise
   fault-free.
}
}

\sstroutine{PLNET}
{
   Plot uncalibrated net spectrum.
}{
   \sstparameters{
   \cpar{DATASET}{Dataset name.}
   \cpar{ORDER}{\'{E}chelle order number.}
   \cpar{APERTURE}{Aperture name (\verb+SAP+ or \verb+LAP+).}
   \cpar{RS}{Whether display is reset before plotting.}
   \cpar{DEVICE}{GKS/SGS graphics device name.}
   \cpar{ZONE}{Zone to be used for plotting.}
   \cpar{LINE}{Plotting line style (\verb+SOLID+, \verb+DASH+, \verb+DOTDASH+
               or \verb+DOT+).}
   \cpar{LINEROT}{Whether line style is changed after next plot}
   \cpar{COL}{Plotting line colour (1, 2, 3, \ldots 10).}
   \cpar{COLROT}{Whether line colour is changed after next plot.}
   \cpar{HIST}{Whether lines are drawn as histograms.}
   \cpar{QUAL}{Whether data quality information is plotted.}
   \cpar{XL}{$x$-axis plotting limits, undefined or [0, 0] means auto-scale.}
   \cpar{YL}{$y$-axis plotting limits, undefined or [0, 0] means auto-scale.}
}
\sstdescription{
   This command plots the uncalibrated net spectrum specified by the
   \verb+DATASET+
   parameter on the graphics device and zone specified by the
   \verb+DEVICE+ and \verb+ZONE+ parameters respectively.

   In the case of a LORES spectrum, if there is more than a single
   aperture available, then the \verb+APERTURE+ parameter needs to be specified.

   In the case of a HIRES spectrum, if there is more than a single
   \'{e}chelle order, then the \verb+ORDER+ parameter needs to be specified.

   The \verb+RS+ parameter specifies whether a new plot is started, or whether
   the data can be plotted over an existing plot.

   The \verb+HIST+ parameter determines whether the line is drawn as a histogram
   rather than a continuous polyline.

   The \verb+LINE+ and \verb+LINEROT+ parameters determine the line style which
   will be used for plotting.

   The \verb+COL+ and \verb+COLROT+ parameters determine the line colour which
   will be used for plotting if the \verb+DEVICE+ supports colour graphics.

   The diagram limits are specified by the \verb+XL+ and \verb+YL+ parameter
   values.
   If \verb+XL+ and \verb+YL+ have values

   \begin {quote}
      \verb+XL=[0,0], YL=[0,0]+
   \end {quote}
   then the plot limits along each axis are determined so that the whole
   spectrum is visible.
   If the values of \verb+XL+ or \verb+YL+ are specified in decreasing order,
   then the coordinates will be reversed along the appropriate axis.

   The \verb+QUAL+ parameter indicates whether faulty points are flagged with
   their data quality codes (see Section~\ref{se:introduction}).

   If a point is affected by more than one of the above faults, then
   the highest code is plotted.
   Points marked bad by user edits are only indicated if they are otherwise
   fault-free.
}
}

\sstroutine{PLSCAN}
{
   Plot scan perpendicular to dispersion.
}{
   \sstparameters{
   \cpar{DATASET}{Dataset name.}
   \cpar{RS}{Whether display is reset before plotting.}
   \cpar{DEVICE}{GKS/SGS graphics device name.}
   \cpar{ZONE}{Zone to be used for plotting.}
   \cpar{LINE}{Plotting line style (\verb+SOLID+, \verb+DASH+, \verb+DOTDASH+
               or \verb+DOT+).}
   \cpar{LINEROT}{Whether line style is changed after next plot.}
   \cpar{COL}{Plotting line colour (1, 2, 3, \ldots 10).}
   \cpar{COLROT}{Whether line colour is changed after next plot.}
   \cpar{QUAL}{Whether data quality information is plotted.}
   \cpar{XL}{$x$-axis plotting limits, undefined or [0, 0] means auto-scale.}
   \cpar{YL}{$y$-axis plotting limits, undefined or [0, 0] means auto-scale.}
}
\sstdescription{
   This command plots the most recent scan perpendicular to dispersion
   associated with \verb+DATASET+ on the graphics device and zone specified by
   the \verb+DEVICE+ and \verb+ZONE+ parameters respectively.

   The \verb+RS+ parameter specifies whether a new plot is started, or whether
   the data can be plotted over an existing plot.

   The \verb+LINE+ and \verb+LINEROT+ parameters determine the line style which
   will be used for plotting.

   The \verb+COL+ and \verb+COLROT+ parameters determine the line colour which
   will be used for plotting if the \verb+DEVICE+ supports colour graphics.

   The diagram limits are specified by the \verb+XL+ and \verb+YL+ parameter
   values.
   If \verb+XL+ and \verb+YL+ have values

   \begin {quote}
      \verb+XL=[0,0], YL=[0,0]+
   \end {quote}
   then the plot limits along each axis are determined so that the whole
   spectrum is visible.
   If the values of \verb+XL+ or \verb+YL+ are specified in decreasing order,
   then the coordinates will be reversed along the appropriate axis.

   The \verb+QUAL+ parameter indicates whether faulty points are flagged with
   their data quality codes (see Section~\ref{se:introduction}).

   If a point is affected by more than one of the above faults, then
   the highest code is plotted.
   Points marked bad by user edits are only indicated if they are otherwise
   fault-free.
}
}

\newpage
\sstroutine{PRGRS}
{
   Print the current extracted aperture or order spectrum in tabular
   form.
}{
   \sstparameters{
   \cpar{DATASET}{Dataset name.}
   \cpar{APERTURE}{Aperture name (\verb+SAP+ or \verb+LAP+).}
}
\sstdescription{
   This command prints the recently extracted spectrum associated
   with \verb+ORDER+ or \verb+APERTURE+
   and \verb+DATASET+ in tabular form.
   The table consists of wavelengths, ``gross'', smooth background, net
   and calibrated fluxes, along
   with any data quality information.
   The ``gross'' and smooth background correspond to an image sample
   with width specified by the adopted extraction slit.

   The output from this command is likely to be too voluminous to read at the
   terminal, refering to the \verb+session.lis+ file may be easier.
}
}

\sstroutine{PRLBLS}
{
   Print the current LBLS array in tabular form.
}{
   \sstparameters{
   \cpar{DATASET}{Dataset name.}
}
\sstdescription{
   This command prints the current LBLS array in tabular form.

   Each line of the main table consists of a wavelength and a set of
   mapped image intensities (FN), corresponding to cells at distances, R
   (pixels), from spectrum centre.

   Any array cells which are affected by ``bad'' image pixels
   ({\it{e.g.,}}\ reseaux, saturation, etc.) have data quality values printed
   below them, the meaning of which is given at the start of the output.

   The output from this command is likely to be too voluminous to read at the
   terminal, refering to the \verb+session.lis+ file may be easier.
}
}

\sstroutine{PRMEAN}
{
   Print the current mean spectrum in tabular form.
}{
   \sstparameters{
   \cpar{DATASET}{Dataset name.}
}
\sstdescription{
   This command prints the mean spectrum associated with DATASET in tabular form.

   The table consists of wavelengths and calibrated fluxes, along
   with any data quality information.

   The output from this command is likely to be too voluminous to read at the
   terminal, refering to the \verb+session.lis+ file may be easier.
}
}

\sstroutine{PRSCAN}
{
   Print the intensities of the current image scan in tabular form.
}{
   \sstparameters{
   \cpar{DATASET}{Dataset name.}
}
\sstdescription{
   This command prints the scan associated with \verb+DATASET+ in tabular form.
   The table consists of wavelengths and net fluxes, along
   with any data quality information.

   The output from this command should be diverted to a file, since
   it is likely to be too voluminous to read at the terminal.
}
}

\sstroutine{PRSPEC}
{
   Print the current aperture or order spectrum in tabular form.
}{
   \sstparameters{
   \cpar{DATASET}{Dataset name.}
   \cpar{APERTURE}{Aperture name (\verb+SAP+ or \verb+LAP+).}
   \cpar{ORDER}{\'{E}chelle order number.}
}
\sstdescription{
   This command prints the spectrum associated with \verb+DATASET+ and either
   \verb+ORDER+ or \verb+APERTURE+ in tabular form.

   The table consists of wavelengths, net and calibrated fluxes, along
   with any data quality information.

   The output from this command is likely to be too voluminous to read at the
   terminal, refering to the \verb+session.lis+ file may be easier.
}
}

\sstroutine{QUIT}
{
   Quit IUEDR.
}{
   \sstparameters{
   \npar{None.}
}
\sstdescription{
   This command quits IUEDR.
   Any files that require new versions will be written by this
   command.
   This command is a synonym for the \verb+EXIT+ command.
}
}

\newpage
\sstroutine{READIUE}
{
   Read a RAW, GPHOT or PHOT IUE image from GO format tape or file.
}{
   \sstparameters{
   \cpar{DRIVE}{Tape drive or file name.}
   \cpar{FILE}{Tape file number.}
   \cpar{NLINE}{Number of IUE header lines printed.}
   \cpar{DATASET}{Dataset name.}
   \cpar{TYPE}{Dataset type (\verb+RAW+, \verb+PHOT+ or \verb+GPHOT+).}
   \cpar{OBJECT}{Object identification text.}
   \cpar{CAMERA}{Camera name (\verb+LWP+, \verb+LWR+ or \verb+SWP+).}
   \cpar{IMAGE}{Image number.}
   \cpar{APERTURES}{Aperture name.}
   \cpar{RESOLUTION}{Spectrograph resolution mode (\verb+LORES+ or
                     \verb+HIRES+).}
   \cpar{EXPOSURES}{Spectrum exposure time(s) (seconds).}
   \cpar{THDA}{IUE camera temperature (C).}
   \cpar{ITFMAX}{Pixel value on tape for ITF saturation.}
   \cpar{BADITF}{Whether bad LORES SWP ITF requires correction.}
   \cpar{YEAR}{Year number (A.D.).}
   \cpar{MONTH}{Month number (1-12).}
   \cpar{DAY}{Day number in Month.}
   \cpar{NGEOM}{Number of Chebyshev terms used to represent geometry.}
   \cpar{ITF}{This is the ITF generation used in the image calibration.}
}
\sstdescription{
   This command reads an IUE dataset (\verb+RAW+, \verb+GPHOT+ or \verb+PHOT+)
   from tape or from a GO format disk file.
   The \verb+DATASET+ parameter determines the names of files that will contain
   the various data components ({\it{e.g.,}}\ Calibration, Image \& Image Quality
   etc.)

   The \verb+ITFMAX+ and \verb+NGEOM+ parameters are only prompted for if the
   image data are geometrically and photometrically calibrated.
}
}


\newpage
\sstroutine{READSIPS}
{
   Read MELO/MEHI from IUESIPS\#1 or IUESIPS\#2 tape or file.
}{
   \sstparameters{
   \cpar{DRIVE}{Tape drive or file name.}
   \cpar{FILE}{Tape file number.}
   \cpar{NLINE}{Number of IUE header lines printed.}
   \cpar{DATASET}{Dataset name.}
   \cpar{OBJECT}{Object identification text.}
   \cpar{CAMERA}{Camera name (\verb+LWP+, \verb+LWR+ or \verb+SWP+).}
   \cpar{IMAGE}{Image number.}
   \cpar{APERTURES}{Aperture name.}
   \cpar{EXPOSURES}{Spectrum exposure time(s) (seconds).}
   \cpar{THDA}{IUE camera temperature (C).}
   \cpar{YEAR}{Year number (A.D.).}
   \cpar{MONTH}{Month number (1-12).}
   \cpar{DAY}{Day number in Month.}
   \cpar{ITF}{This is the ITF generation used in the image calibration.}
}
\sstdescription{
   This command reads the MELO/MEHI product from IUESIPS\#1 or IUESIPS\#2
   tape or file.  Operation is much like \verb+READIUE+, except that some
   parameters and associated information are not needed. Only calibration
   ({\tt .UEC}) and spectrum ({\tt \_UES.sdf}) files are created.
   The values for certain parameters may be obtained from the tape or file,
   in which case you will not be prompted for them.
}
}

\sstroutine{SAVE}
{
   Write new versions for any files that have had their contents
   changed during the current session.
}{
   \sstparameters{
   \npar{None.}
}
\sstdescription{
   This command overwrites dataset files that have had their contents
   changed during the current session.
   If there are no outstanding files then this command does nothing.
}
}

\newpage
\sstroutine{SCAN}
{
   This command performs a scan perpendicular to spectrograph
   dispersion.
}{
   \sstparameters{
   \cpar{DATASET}{Dataset name.}
   \cpar{ORDERS}{This delineates a range of \'{e}chelle orders.}
   \cpar{APERTURE}{Aperture name (\verb+SAP+ or \verb+LAP+).}
   \cpar{SCANDIST}{HIRES scan distance from camera faceplate centre}
   \cparc{(geometric pixels.)}
   \cpar{SCANAV}{Averaging filter HWHM for image scan}
   \cparc{(geometric pixels).}
   \cpar{SCANWV}{Central wavelength for LORES image scan (\AA).}
}
\sstdescription{
   This command performs a scan perpendicular to spectrograph dispersion.
   The scan is performed by folding pixels with a triangle function
   with HWHM of \verb+SCANAV+ geometric pixels along the dispersion direction.

   In the case of HIRES, the \verb+SCANDIST+ parameter determines the distance
   of the scan from the faceplate centre.

   In the case of LORES, the \verb+SCANWV+ parameter determines the central
   wavelength of the scan in Angstroms.

   The algorithm used to produce scan intensities is not very good
   and so quantitative results should not be sought from this command.
   Its sole intention lies in providing data for aligning the spectrum.
}
}

\sstroutine{SETA}
{
   Set dataset parameters that are APERTURE specific.
}{
   \sstparameters{
   \cpar{DATASET}{Dataset name.}
   \cpar{APERTURE}{Aperture name (\verb+SAP+ or \verb+LAP+).}
   \cpar{EXPOSURE}{Spectrum exposure time (seconds).}
   \cpar{FSCALE}{Flux scale factor.}
   \cpar{WSHIFT}{Constant wavelength shift (\AA).}
   \cpar{VSHIFT}{Velocity shift of detector relative to Sun (km/s).}
   \cpar{ESHIFT}{Global \'{e}chelle wavelength shift.}
   \cpar{GSHIFT}{Global shift of spectrum on image (geometric pixels).}
}
\sstdescription{
   This command allows changes to be made to dataset values which are
   specific to the specified \verb+APERTURE+\@.
   Items for which parameters are not specified retain their current
   values.
}
}

\sstroutine{SETD}
{
   Set dataset parameters which are independent of ORDER/APERTURE.
}{
   \sstparameters{
   \cpar{DATASET}{Dataset name.}
   \cpar{OBJECT}{Object identification text.}
   \cpar{THDA}{IUE camera temperature (C).}
   \cpar{FIDSIZE}{Half width of fiducials (pixels).}
   \cpar{BADITF}{Whether bad LORES ITF requires correction.}
   \cpar{NGEOM}{Number of Chebyshev terms used to represent geometry.}
   \cpar{RIPK}{\'{E}chelle ripple constant (\AA).}
   \cpar{RIPA}{Ripple function scale factor.}
   \cpar{XCUT}{Global \'{e}chelle wavelength clipping.}
   \cpar{HALTYPE}{The type of halation (order-overlap) correction used.}
   \cpar{HALC}{Halation correction constant (fraction of continuum).}
   \cpar{HALWC}{Wavelength for which the halation correction \verb+HALC+}
   \cparc{is defined in angstroms.}
   \cpar{HALW0}{Wavelength at which halation correction is zero (\AA).}
   \cpar{HALAV}{Averaging FWHM for halation correction (gometric pixels).}
}
\sstdescription{
   This command allows changes to be made to dataset values which are
   independent of any specific \verb+APERTURE+/\verb+ORDER+\@.
   Items for which parameters are not specified retain their current
   values.
}
}

\sstroutine{SETM}
{
   Set dataset parameters that are ORDER specific.
}{
   \sstparameters{
   \cpar{DATASET}{Dataset name.}
   \cpar{ORDER}{\'{E}chelle order number.}
   \cpar{RIPK}{\'{E}chelle ripple constant (\AA).}
   \cpar{RIPA}{Ripple function scale factor.}
   \cpar{RIPC}{Ripple function correction polynomial.}
   \cpar{WCUT}{Wavelength limits for \'{e}chelle order (\AA).}
}
\sstdescription{
   This command allows changes to be made to dataset values which are
   specific to the specified \verb+ORDER+\@.
   Items for which parameters are not specified retain their current
   values.
}
}

\newpage
\sstroutine{SGS}
{
   Write names of available SGS devices at the terminal.
}{
   \sstparameters{
   \npar{None.}
}
\sstdescription{
   This commands writes a list of available SGS device names at the terminal.
   See \xref{SUN/85}{sun85}{} for details of the SGS graphics system.
}
}

\sstroutine{SHOW}
{
   Print dataset values.
}{
   \sstparameters{
   \cpar{DATASET}{Dataset name.}
   \cpar{V}{List of items to be printed.}
}
\sstdescription{
   This command shows the values of parameters in the dataset specified
   by the \verb+DATASET+ parameter. The items to be printed are specified by
   the \verb+V+ parameter, which is a string containing any of the following
   characters:

   \begin {description}
      \item H --- Header and file information
      \item I --- Image details
      \item F --- Fiducials
      \item G --- Geometry
      \item D --- Dispersion
      \item C --- Centroid templates
      \item R --- \'{E}chelle Ripple and halation
      \item A --- Absolute calibration
      \item S --- Raw Spectrum
      \item M --- Mean spectrum
      \item * --- All of the above
      \item Q --- Image data Quality summary
   \end {description}

   The \verb+*+ character needs to be placed within inverted commas.

   The \verb+V+ parameter is cancelled afterwards.
}
}

\sstroutine{TRAK}
{
   Extract spectrum from image.
}{
   \sstparameters{
   \cpar{DATASET}{Dataset name.}
   \cpar{ORDER}{\'{E}chelle order number.}
   \cpar{APERTURE}{Aperture name (\verb+SAP+ or \verb+LAP+).}
   \cpar{NORDER}{Number of \'{e}chelle orders to be processed.}
   \cpar{AUTOSLIT}{Whether \verb+GSLIT+, \verb+BDIST+ and \verb+BSLIT+
                   parameters are}
   \cparc{determined automatically.}
   \cpar{GSLIT}{Object channel limits (geometric pixels).}
   \cpar{BSLIT}{Background channel half widths (geometric pixels).}
   \cpar{BDIST}{Distances of background channels from object channel}
   \cparc{centre (geometric pixels).}
   \cpar{GSAMP}{Spectrum grid sampling rate (geometric pixels).}
   \cpar{CUTWV}{Whether wavelength cutoff data used for extraction grid.}
   \cpar{BKGIT}{Number of background smoothing iterations.}
   \cpar{BKGAV}{Background averaging filter FWHM (geometric pixels).}
   \cpar{BKGSD}{Discrimination level for background pixels (s.d.).}
   \cpar{CENTM}{Whether pre-existing centroid template is used.}
   \cpar{CENSH}{Whether the spectrum template is just shifted linearly.}
   \cpar{CENSV}{Whether the spectrum template is saved in the dataset.}
   \cpar{CENIT}{Number of centroid tracking iterations.}
   \cpar{CENAV}{Centroid averaging filter FWHM (geometric pixels).}
   \cpar{CENSD}{Significance level for signal to be used for centroids
                (s.d.).}
   \cpar{EXTENDED}{Whether the object is not a point source.}
   \cpar{CONTINUUM}{Whether the object spectrum is expected to have a}
   \cparc{``continuum''.}
}
\sstdescription{
   This command extracts a spectrum from an image.
   It does this by defining an evenly spaced wavelength grid along the
   spectrum, and mapping pixel intensities onto this grid in object
   and background channels.
   The background pixels are used to form a smooth background spectrum.
   The object pixels (less smooth background) are used to form the
   integrated net signal for the object.

   In the LORES case, the spectrum specified by the \verb+APERTURE+ parameter
   is extracted.

   In the HIRES case, the first \'{e}chelle order to be extracted is
   specified by \verb+ORDER+\@.
   Up to \verb+NORDER+ orders are extracted, with \verb+ORDER+ being
   decremented each time.

   The wavelength grid is defined by the region of the dispersion line
   contained in the image subset (faceplate).
   The grid spacing is set by the \verb+GSAMP+ parameter value which is
   the sample step in geometric pixels.
   The wavelength limits can optionally be constrained within the
   \'{e}chelle cutoff values by specifying \verb+CUTWV=TRUE+\@.

   The object and background channel widths and positions are determined
   automatically if \verb+AUTOSLIT=TRUE+\@.
   Otherwise, the object channel is specified by the values of the
   \verb+GSLIT+ parameter, whilst the background channel positions and
   widths are determined by the \verb+BDIST+ and \verb+BSLIT+ parameter values
   respectively.

   The \verb+EXTENDED+ and \verb+CONTINUUM+ parameters allow more precise
   control over slit determinations (see the IUEDR User Guide (MUD/45) for
   details).

   The background spectrum is smoothed with a triangle function with a
   FWHM given in geometric pixels by the \verb+BKGAV+ parameter.
   Once the background channel spectra have been obtained, they are
   extracted a further \verb+BKGIT+ times.
   Prior to each additional background extraction pixels which are
   outside \verb+BKGSD+ local standard deviations are rejected.

   The object spectrum is obtained by integrating pixel intensities
   (less smooth background) within the object channel.
   Once the object spectrum has been obtained it is extracted an
   additional \verb+CENIT+ times, the centroid positions
   from the previous extraction being used to ``follow'' the
   spectrum each time.
   The centroid spectrum (template) is smoothed by folding with a
   triangle function, FWHM given in geometric
   pixels by the \verb+CENAV+ parameter.
   Wavelengths with flux levels below \verb+CENSD+ standard deviations
   above background are not used in determining the centroid spectrum.

   By default, the initial spectrum template is given by the dispersion
   relations and the geometric shifts determined using the \verb+CGSHIFT+\@.
   However, if \verb+CENTM=TRUE+, then a pre-defined template associated with
   the dataset may be used as a start guess.
   If \verb+CENSH=TRUE+, then this template can be shifted linearly to match the
   image ({\it{i.e.}}\ without changing its shape).
   If \verb+CENSV=TRUE+, then the final centroid spectrum is used to update the
   spectrum template in the dataset.

   The net flux associated with a wavelength point in the final extracted
   spectrum is defined as the integral of pixel intensities over a rectangle
   with dimensions given by the object channel width and the wavelength
   interval.
   These fluxes are scaled so that they correspond to an interval
   along the wavelength direction of 1.414 geometric pixels.
   This is so that the standard IUESIPS calibrations can be applied
   regardless of what actual sample rate has been employed.
   The integral is performed by using linear interpolation of pixel intensities.
}
}

\newpage
%------------------------------------------------------------------------------

\section{\xlabel{paramaters}\label{se:parameters}Parameters}
\markboth{Parameters}{\stardocname}

There follows a detailed description for each of the parameters used by
IUEDR commands.
The description for a particular parameter applies in any
command which uses it.
Some parameters have default values which are initialised on invoking IUEDR\@.
Default parameter behaviour is described in
Appendix~\ref{se:parameter_defaults}\@.

This release of IUEDR uses the ADAM parameter system. In this system
the parameters and their usage are described in an interface file
(See \xref{SG/4}{sg4}{} for more detail). It is possible to override the
default
interface file with your own personal version. This permits you to tailor
the precise behavior of each parameter according to your requirements.

In the following descriptions the required parameter value is one of:
\begin{description}
   \item [\_CHAR] A character string.
   \item [\_DOUBLE] A floating point number.  The decimal point need not be
                    included if the value is integer only.
   \item [\_INTEGER] Integer number.
   \item [\_LOGICAL] A logical value: {\tt YES, TRUE, NO} or {\tt FALSE.}
\end{description}

Where a parameter value is of the form:

\begin{quote}
{\bf
   [\_TYPE\{,\_TYPE\}]
}
\end{quote}

At least one value of type {\bf \_TYPE} is required, the second is optional.

\rule{\textwidth}{0.5mm}

\iueparlist{

\iueparameter{ABSFILE}
{
   \_CHAR
}{
   This is the name of a file containing an absolute flux calibration.
   A file type of \verb+.abs+ is assumed and need not be specified
   explicitly.
   If the file name contains a directory specification, then it should be
   enclosed in quotes.
}

\iueparameter{APERTURE}
{
   \_CHAR
}{
   This is the name of an individual aperture.
   The following names have defined meanings:

   \begin {description}
      \item {\tt SAP} --- IUE small aperture.
      \item {\tt LAP} --- IUE large aperture.
   \end {description}

   Other apertures may also be defined.
}

\iueparameter{APERTURES}
{
   \_CHAR
}{
   This specifies an aperture or group of apertures.
   The following three names have defined meanings:

   \begin {description}
      \item {\tt SAP} --- IUE small aperture.
      \item {\tt LAP} --- IUE large aperture.
      \item {\tt BAP} --- IUE both apertures ({\it{i.e.}}\ SAP and LAP together).
   \end {description}
}

\iueparameter{AUTOSLIT}
{
   \_LOGICAL
}{
   This determines whether the extraction slit is determined automatically
   by the command.
   When \verb+AUTOSLIT=TRUE+ the \verb+GSLIT+, \verb+BDIST+ and \verb+BSLIT+
   parameter values are
   determined automatically, based on the IUE camera, resolution, aperture,
   and on the values of the \verb+EXTENDED+ and \verb+CONTINUUM+ parameters.
   This mode of operation is probably the best for point source objects.

   This parameter has a default value of \verb+TRUE+\@.
}

\iueparameter{BADITF}
{
   \_LOGICAL
}{
   This parameter determines whether a correction is made to the pixel
   intensities to account for errors during Ground Station ITF
   calibration.
   (Note that the best scientific results would be obtained
   from reprocessed data which can be obtained on request.)
   The following case is handled:

   \begin {description}
      \item SWP, LORES --- correction of 2nd (faulty) ITF.
   \end {description}
}

\iueparameter{BDIST}
{
   [\_DOUBLE\{,\_DOUBLE\}]
}{
   This is a pair of numbers delineating the background spectrum channel
   positions during spectrum extraction.
   The distances are measured in geometric pixels from the spectrum centre.

   Negative distances mean ``to the left of centre'', and positive distances
   mean ``to the right of centre''.

   If only one value is defined, then this is taken as meaning
   that the channels are positioned symmetrically
   about centre.

   The spectrum ``centre'' is determined by the dispersion relations, and
   modified by any prevailing centroid shifts.
}

\iueparameter{BKGAV}
{
   \_DOUBLE
}{
   This is the FWHM of a triangle function filter used in folding the
   pixel intensities to form the smooth background spectrum.
   It is measured in geometric pixels.

   This parameter has a default value of 30.0.
}

\iueparameter{BKGIT}
{
   \_DOUBLE
}{
   This is the number of background smoothing iterations performed during
   spectrum extraction.

   \begin {description}
      \item \verb+BKGIT=0+ means that the background is taken as the result of
      the first pass of a triangle function filter with a FWHM defined by
      the \verb+BKGAV+ parameter.

      \item \verb+BKGIT=1+ means that, after producing the initial estimate for
      the smooth background, pixels discrepant by more that \verb+BKGSD+
      standard deviations are marked as ``spikes''.
      The smooth background is then re-evaluated, missing out these marked
      pixels.
   \end {description}

   Higher values of \verb+BKGIT+ are possible, but seldom necessary.

   This parameter has a default value of 1.
}

\iueparameter{BKGSD}
{
   \_DOUBLE
}{
   This is the discrimination level, measured in standard deviations,
   beyond which background pixels are marked as ``spikes''.
   It is not used for \verb+BKGIT=0+.

   This parameter has a default value of 2.0.
}

\iueparameter{BSLIT}
{
   [\_DOUBLE\{,\_DOUBLE\}]
}{
   This defines the half width of each background channel, measured
   in geometric pixels.
   A single value means that both channels have the same width.
}

\iueparameter{CAMERA}
{
   \_CHAR
}{
   This is the camera name.
   The following are defined:

   \begin {quote}
   \begin {description}
      \item {\tt LWP} --- IUE long wavelength prime.
      \item {\tt LWR} --- IUE long wavelength redundant.
      \item {\tt SWP} --- IUE short wavelength prime.
   \end {description}
   \end {quote}
}

\iueparameter{CENAV}
{
   \_DOUBLE
}{
   This is the FWHM of a triangle function filter used in folding the
   pixel intensities to form the smooth spectrum centroid positions.
   It is measured in geometric pixels.

   This parameter has a default value of 30.0.
}

\iueparameter{CENIT}
{
   \_INTEGER
}{
   This is the number of spectrum centroid tracking iterations performed during
   spectrum extraction.

   \begin {description}
      \item \verb+CENIT=0+ means that the spectrum position is taken directly
      from the dispersion relations.

      \item \verb+CENIT=1+ means that the spectrum position is first taken
      from the dispersion relations, but is modified to force it to
      follow the spectrum centroid.
   \end {description}

   Higher values of \verb+CENIT+ are possible, but seldom necessary:  it either
   works or fails.

   This parameter has a default value of 1.
}

\iueparameter{CENSD}
{
   \_DOUBLE
}{
   This is the discrimination level, measured in standard deviations,
   below which object signal is not considered significant enough
   to be used to determine the centroid position.
   It is not used for \verb+CENIT=0+\@.

   This parameter has a default value of 4.0.
}

\iueparameter{CENSH}
{
   \_LOGICAL
}{
   This indicates whether the spectrum signal produces a single linear
   shift to the initial template.

   This can be used in cases where the object signal is too weak
   to provide a detailed centroid determination by moving a pre-existing
   template shape into the right position.

   This parameter has a default value of \verb+FALSE+\@.
}

\iueparameter{CENSV}
{
   \_LOGICAL
}{
   This indicates whether the spectrum template, as refined by the
   object centroid during spectrum extraction, is saved in the calibration
   dataset.

   The primary use of this facility is in determining templates from,
   say, the whole spectrum using \verb+TRAK+, and subsequently using these
   with \verb+LBLS+, or another spectrum.

   This parameter has a default value of \verb+FALSE+\@.
}

\iueparameter{CENTM}
{
   \_LOGICAL
}{
   This indicates whether a centroid template from the calibration dataset
   is used as a start in defining the precise position of the spectrum
   signal on the image.

   This parameter has a default value of \verb+FALSE+\@.
}

\iueparameter{CENTREWAVE}
{
   [\_DOUBLE[,\_DOUBLE\ldots ]]
}{
   These are the laboratory wavelengths of a set of absorption features in
   the spectrum to be used to estimate a value for the \verb+ESHIFT+
   parameter.
}

\iueparameter{COL}
{
   \_INTEGER
}{
   This specifies the line colour to be used for the next curve to be
   plotted.
   It can be an integer in the range 1 to 10, and the corresponding
   colours are as follows:

   \begin {quote}
   \begin {description}
      \item {\tt 1} --- Yellow.
      \item {\tt 2} --- Green.
      \item {\tt 3} --- Red.
      \item {\tt 4} --- Blue.
      \item {\tt 5} --- Pink.
      \item {\tt 6} --- Violet.
      \item {\tt 7} --- Turquoise.
      \item {\tt 8} --- Orange.
      \item {\tt 9} --- Light green.
      \item {\tt 10} --- Olive.
   \end {description}
   \end {quote}

   Lines will only appear with different colours it the device supports colour
   graphics, on other devices \verb+COL+ is ignored.
}

\iueparameter{COLOUR}
{
   \_LOGICAL
}{
   Whether a spectrum-style false colour look-up table is used by
   \verb+DRIMAGE+\@.
   If \verb+FALSE+ a greyscale is used.

   The default is to use a greyscale.
}

\iueparameter{COLROT}
{
   \_LOGICAL
}{
   This indicates whether the line colour is to be changed after the next plot.
   The initial line has colour index 1 (YELLOW), unless specified explicitly
   using the \verb+COL+ parameter.
   The sequence of colour indices goes (1, 2, 3, \ldots 10, 1, 2, \ldots).

   In commands where more than one line is plotted, \verb+COLROT+ determines
   whether these lines have different colours.

   Lines will only appear with different colours if the device supports
   colour graphics; on other devices \verb+COLROT+ is harmless.

   This parameter has a default value of \verb+TRUE+\@.
}

\iueparameter{CONTINUUM}
{
   \_LOGICAL
}{
   This indicates whether the object spectrum is expected to contain a
   significant continuum.
   It is used in conjunction with the \verb+EXTENDED+ parameter in determining
   the positions and widths of object and background channels for
   spectrum extraction from HIRES datasets.

   This parameter has a default value of \verb+TRUE+\@.
}

\iueparameter{COVERGAP}
{
   \_LOGICAL
}{
   If after mapping an order/aperture, a grid point is marked as unusable,
   then this parameter determines whether other orders/apertures
   can be allowed to contribute to this grid point.

   This parameter has a default value of \verb+FALSE+\@.
}

\iueparameter{CUTFILE}
{
   \_CHAR
}{
   This is the name of a file containing an \'{e}chelle order
   wavelength limits.
   A file type of \verb+.cut+ is assumed and need not be specified
   explicitly.
   If the file name contains a directory specification, then it should be
   enclosed in quotes.
}

\iueparameter{CUTWV}
{
   \_LOGICAL
}{
   This indicates whether any available \'{e}chelle order wavelength cutoff
   limits are to be used for the spectrum extraction wavelength grid
   limits.
   Highly recommended, provided that you are happy with these wavelength limits.

   This parameter has a default value of \verb+TRUE+\@.
}

\iueparameter{DATASET}
{
   \_CHAR
}{
   This is the root name of the files containing the dataset.
   The file type ({\it{e.g.,}}\ \_\verb+UED.sdf+) should {\bf not} be given in
   the \verb+DATASET+ name.
   If the file name contains a directory specification, then it
   should be enclosed in quotes.

   Note that the actual filenames contain additional characters
   to define their contents ({\it{e.g.,}}\ \verb+LWP12345+\_\verb+UES.sdf+,
   contains spectral data).
}

\iueparameter{DAY}
{
   \_INTEGER
}{
   This is the day number, measured from the start of the month, used
   for constructing dates.
   The \verb+DAY+, \verb+MONTH+ and \verb+YEAR+ parameters refer to the date
   the IUE observations were made and are important to the calibration of the
   data.
}

\iueparameter{DELTAWAVE}
{
   [\_DOUBLE[,\_DOUBLE\ldots]\,]
}{
   The half-width of the window used to search for an absorption line feature
   for wavelength calibration in Angstroms.  If more than one line is being
   used then each may be given a different search window width.
}

\iueparameter{DEVICE}
{
   \_CHAR
}{
   This is the GKS/SGS graphics device.
   A list of possible GKS workstations may be found in
   \xref{SUN/83}{sun83}{}.
   A list of SGS workstation names available at your
   site may be obtained either by a null response to the \verb+DEVICE+ parameter
   prompt, {\it{i.e.}}\ \verb+!+, or by using the IUEDR Command \verb+SGS+\@.
}

\iueparameter{DISPFILE}
{
   \_CHAR
}{
   This is the name of a file containing dispersion data.
   A file type of \verb+.dsp+ is assumed and need not be specified
   explicitly.
   If the file name contains a directory specification, then it should be
   enclosed in quotes.
}

\iueparameter{DRIVE}
{
   \_CHAR
}{
   This is the name of the tape drive. Feasible value
   are \verb+/dev/nrmt0h+ on a UNIX machine and \verb+MSA0+ on VMS.

   This version of IUEDR supports the direct reading of
   IUEDR data from disk files which have the same format as those on
   GO format tapes.

   In order to read directly from such a file (probably grabbed from
   an on-line archive such as NDADSA), you specify its name directly in
   response to the \verb+DRIVE+ parameter prompt.

   If the file is not in the current directory then you must provide
   the full pathname.
}

\iueparameter{ESHIFT}
{
   \_DOUBLE
}{
   This is a global wavelength shift applied to the wavelengths in
   \'{e}chelle spectral orders.
   It is measured in Angstroms, and affects the spectrum wavelengths as follows:

   \begin {equation}
      \lambda _{new} = \lambda _{old} + \frac{\rm ESHIFT}{\rm ORDER}
   \end {equation}

   This is designed to account for wavelength errors that result
   from a global linear shift of the spectrum format on the
   image.
}

\iueparameter{EXPOSURE}
{
   \_DOUBLE
}{
   This is the exposure time associated with the spectrum, measured in seconds.
   If there is more than one aperture, then this time applies
   to that specified by the \verb+APERTURE+ parameter.
}

\iueparameter{EXPOSURES}
{
   [\_DOUBLE\{,\_DOUBLE\}]
}{
   This is one or more exposure times associated with the
   spectrum, measured in seconds.
   There is an exposure time for each aperture defined.
}

\iueparameter{EXTENDED}
{
   \_LOGICAL
}{
   This indicates whether the object spectrum is expected to be extended,
   rather than a point source.
   It is used in conjunction with the \verb+CONTINUUM+ parameter in determining
   the positions and widths of the object and background channels used
   for spectrum extraction from HIRES datasets.

   This parameter has a default value of \verb+FALSE+\@.
}

\iueparameter{FIDFILE}
{
   \_CHAR
}{
   This is the name of a file containing fiducial positions.
   A file type of \verb+.fid+ is assumed and need not be specified
   explicitly.
   If the file name contains a directory specification, then it should be
   enclosed in quotes.
}

\iueparameter{FIDSIZE}
{
   \_DOUBLE
}{
   This is the half width of a fiducial measured in pixel units. The fiducials
   are considered to be square.
}

\iueparameter{FILE}
{
   \_INTEGER
}{
   This is the tape file number.
   The first file on a tape would be \verb+FILE=1+\@.
   One case which may present some problems is
   that of a tape with a an end-of-volume (EOV) mark in the middle
   and with valuable data beyond.
   An EOV is two consecutive tape marks (sometimes called ``file marks'').
   A file is defined here as the information between two tape marks.
   So if the number for a real file before EOV is FILEN, then
   the number of the next real file following the EOV is (FILEN+2).
}

\iueparameter{FILLGAP}
{
   \_LOGICAL
}{
   If a grid point in the mean spectrum would have had a contribution
   from a bad data point, this parameter determines whether that
   grid point is marked as unusable within the context
   of the order or aperture being mapped.
   If the grid point is marked as unusable in this way then other
   good points cannot contribute to it.

   This parameter has a default value of \verb+FALSE+\@.
}

\iueparameter{FLAG}
{
   \_LOGICAL
}{
   This specifies whether the data quality information is displayed
   along with the image.
   If so, then faulty pixels will be marked with a colour according to
   the following scheme:

   \begin {quote}
   \begin {description}
      \item Green --- pixels affected by reseau marks.
      \item Red --- pixels which are saturated (DN = 255).
      \item Orange --- pixels affected by ITF truncation.
      \item Yellow --- pixels marked bad by the user.
   \end {description}
   \end {quote}

   If a pixel is affected by more than one of the above faults, then
   the first in the list is adopted for display.
   Hence, user edits are only shown where no other fault is present.

   This option would normally only be used when assessing the quality
   of faulty pixels, possibly with a view to using them, {\it{i.e.}}\ marking
   them ``good'' with a cursor editor.

   This parameter has a default value of \verb+TRUE+\@.
}

\iueparameter{FN}
{
   \_DOUBLE
}{
   This parameter is the replacement Flux Number for a pixel changed
   explicitly by the user.
}

\iueparameter{FSCALE}
{
   \_DOUBLE
}{
   This is an arbitrary scale factor applied to spectrum fluxes.
   It affects fluxes as follows:

   \begin {equation}
      {\cal F}_{new} = {\cal F}_{old} \times {\rm FSCALE}
   \end {equation}

   It finds application in accounting for grey attenuation, or obscuration
   of object signal through a narrow aperture.
}

\iueparameter{GSAMP}
{
   \_DOUBLE
}{
   This is the sampling rate used for spectrum extraction.
   It is measured in geometric pixels.
   \verb+GSAMP=1.414+ corresponds to the IUESIPS\#1 sampling
   rate, while \verb+GSAMP=0.707+ corresponds to the IUESIPS\#2 sampling
   rate.
   Other values can be chosen.

   This parameter has a default value of 1.414.
}

\iueparameter{GSHIFT}
{
   [\_DOUBLE,\_DOUBLE]
}{
   This is a global constant shift of the spectrum format on the image,
   $(dx,dy)$, where the geometric coordinates, $(x,y)$ of a spectrum position
   are

   \begin {equation}
      x_{new} = x_{old} + dx
   \end {equation}

   and

   \begin {equation}
      y_{new} = y_{old} + dy
   \end {equation}
}

\iueparameter{GSLIT}
{
   [\_DOUBLE\{,\_DOUBLE\}]
}{
   This is a pair of numbers delineating the object spectrum channel
   during spectrum extraction.
   The distances are measured in geometric pixels.

   Negative distances mean ``to the left of centre'', and positive distances
   mean ``to the right of centre''.

   Object channels that do not cover the actual object signal on the
   image will not be meaningful when centroid tracking is employed.

   If only one value is defined, then this is taken as representing
   a channel that is symmetrical about the spectrum centre.

   The spectrum ``centre'' is determined by the dispersion relations
   modified by any prevailing centroid shifts.
}

\iueparameter{HALAV}
{
   \_DOUBLE
}{
   This is the FWHM of a triangle function used for smoothing the
   net spectrum for the \verb+HALTYPE=POWER+ halation correction technique.
}

\iueparameter{HALC}
{
   \_DOUBLE
}{
   This is the Halation correction constant used for \verb+HALTYPE=POWER+
   cases, and defined at wavelength \verb+HALWC+\@.
   The value of the correction constant
   corresponds roughly to the measured depression of a broad
   zero intensity absorption below zero, in units
   of the continuum in adjacent orders.
   The ``constant'', $C$, varies with wavelength as follows:

   \begin {equation}
      C_\lambda = \frac{{\rm HALC} \times (\lambda - {\rm HALW0})}
                       {({\rm HALWC} - {\rm HALW0})}
   \end {equation}

   See the \verb+HALTYPE+, \verb+HALWC+, \verb+HALW0+ and \verb+HALAV+
   parameters.
}

\iueparameter{HALTYPE}
{
   \_CHAR
}{
   This is the type of Halation or order-overlap correction applied to the
   flux spectrum.
   It can take the value

   \begin {quote}
   \begin {description}
      \item {\tt POWER} --- correction based on power-law PSF decay.
   \end {description}
   \end {quote}
}

\iueparameter{HALW0}
{
   \_DOUBLE
}{
   This is the wavelength, measured in Angstroms, at which the
   halation correction is zero.

   See the \verb+HALTYPE+, \verb+HALC+ and \verb+HALWC+ parameters.
}

\iueparameter{HALWC}
{
   \_DOUBLE
}{
   This is the wavelength, measured in Angstroms, at which the
   halation correction is \verb+HALC+\@.

   See the \verb+HALTYPE+, \verb+HALC+ and \verb+HALW0+ parameters.
}

\iueparameter{HIST}
{
   \_LOGICAL
}{
   This determines whether lines are plotted as histograms rather than
   continuous lines (polylines).

   This parameter has a default value of \verb+TRUE+\@.
}

\iueparameter{IMAGE}
{
   \_INTEGER
}{
   This is the Image Sequence Number.
}

\iueparameter{ITF}
{
   \_INTEGER
}{
   This is the ITF generation used in the photometric calibration of the
   image.  This information is needed for the correct absolute flux
   calibration of the resulting spectra.  Possible values for each camera
   are as follows:

   \begin {quote}
   \begin {description}
      \item SWP --- 2
      \item LWR --- 1 and 2
      \item LWP --- 1 and 2
   \end {description}
   \end {quote}

   The appropriate value can be determined from inspection of the
   IUE header text for the GPHOT/PHOT file using the table
   of numbers following the line beginning \verb+PCF C/**+\@.
   Here are the \verb+ITF+ values associated with various forms of this
   table:

   \begin {quote}
   \begin {tabbing}
   ITFMAXxxx\= 0xx\= 1800xx\= 3700xx\= 5600xx\= ...xx\= 30000xxx\= (Corrected, 3rd SWP ITF)\kill
   {\bf ITF}\> \> \>{\bf TABLE}\> \> \> \>{\bf IDENTIFICATION}\\
   \\
   ITF 0\>0\>1800\>3700\>5600\>...\>30000\>Preliminary LWR ITF\\
   ITF 1\>0\>2303\>4069\>8008\>...\>42032\>2nd LWR ITF\\
   ITF 1\>0\>2300\>3969\>6062\>...\>32973\>1st LWP ITF\\
   ITF 2\>0\>2723\>5429\>8145\>...\>38389\>2nd LWP ITF\\
   ITF 0\>0\>1800\>3600\>5500\>...\>\>Preliminary SWP ITF\\
   ITF 1\>0\>1753\>3461\>6936\>...\>28674\>Faulty, 2nd SWP ITF\\
   ITF 2\>0\>1684\>3374\>6873\>...\>28500\>Corrected, 3rd SWP ITF\\
   \end {tabbing}
   \end {quote}

   If the ITF table used has no corresponding absolute flux calibration within
   IUEDR, {\it{e.g.,}}\ LWR ITF0 or SWP ITF0, you are advised to contact the IUE
   Project.
   Although the \verb+BADITF+ parameter is available for data calibrated using
   SWP ITF1, it is advisable to have these data reprocessed by the IUE Project.
}

\iueparameter{ITFMAX}
{
   \_INTEGER
}{
   This is the pixel value on tape corresponding to ITF saturation.
   Its value is fixed for a given ITF table.
   The value of \verb+ITFMAX+ is only needed for IUE images of type GPHOT.
   The appropriate value can be determined from inspection of the
   IUE header text for the GPHOT file using the table
   of numbers following the line beginning \verb+PCF C/**+.

   \begin {quote}
   \begin {tabbing}
   ITFMAXxxx\= 0xx\= 1800xx\= 3700xx\= 5600xx\= ...xx\= 30000xxx\= (Corrected, 3rd SWP ITF)\kill
   {\bf ITFMAX}\>\>\>{\bf TABLE}\>\>\>\>{\bf IDENTIFICATION}\\
   \\
   20000\>0\>1800\>3700\>5600\>...\>30000\>Preliminary LWR ITF\\
   27220\>0\>2303\>4069\>8008\>...\>42032\>2nd LWR ITF\\
   19983\>0\>1800\>3600\>5500\>...\>\>Preliminary SWP ITF\\
   19740\>0\>1753\>3461\>6936\>...\>28674\>Faulty, 2nd SWP ITF\\
   19632\>0\>1684\>3374\>6873\>...\>28500\>Corrected, 3rd SWP ITF\\
   \end {tabbing}
   \end {quote}
}

\iueparameter{LINE}
{
   \_CHAR
}{
   This specifies the line style to be used for the next curve to be
   plotted.
   It can be one of the following:

   \begin {quote}
   \begin {description}
      \item {\tt SOLID} --- solid (continuous) line.
      \item {\tt DASH} --- dashed line.
      \item {\tt DOTDASH} --- dot-dash line.
      \item {\tt DOT} --- dotted line.
   \end {description}
   \end {quote}

   The order of these is that invoked when automatic line style rotation
   is in effect (see the \verb+LINEROT+ parameter).
}

\iueparameter{LINEROT}
{
   \_LOGICAL
}{
   This indicates whether the line style is to be changed after the
   next plot.
   The initial line style is \verb+SOLID+, unless specified explicitly
   using the \verb+LINE+ parameter.
   The sequence of line styles goes (\verb+SOLID+, \verb+DASH+, \verb+DOTDASH+,
   \verb+DOT+, \verb+SOLID+, \verb+DASH+\dots).

   In commands where more than one line is plotted, \verb+LINEROT+ determines
   whether these lines have different styles.

   This parameter has a default value of \verb+FALSE+\@.
}

\iueparameter{ML}
{
   [\_DOUBLE,\_DOUBLE]
}{
   This is a pair of wavelength values defining the start and end of
   the mean spectrum grid.
   The grid will consist of evenly spaced vacuum wavelengths between these
   values.
}

\iueparameter{MONTH}
{
   \_INTEGER
}{
   This is the month number, measured from the start of the Year,
   used in constructing dates.
}

\iueparameter{MSAMP}
{
   \_DOUBLE
}{
   This is the vacuum wavelength sampling rate for the mean spectrum grid.
   If it does not fit an integral number of times into the grid limits,
   then the latter are adjusted to fit.
}

\iueparameter{NFILE}
{
   \_INTEGER
}{
   This is the number of tape files to be processed.
   A value of \verb+NFILE=-1+ means all files until the end.

   This parameter has a default value of 1.
}

\iueparameter{NGEOM}
{
   \_INTEGER
}{
   This is the number of Chebyshev terms used to represent the
   geometrical distortion.
   The same value is used for each axis direction.
}

\iueparameter{NLINE}
{
   \_INTEGER
}{
   This is the number of IUE header lines printed.
   A value of \verb+NLINE=-1+ means the entire header is printed.

   This parameter has a default value of 10.
}

\iueparameter{NORDER}
{
   \_INTEGER
}{
   This is the number of \'{e}chelle orders to be processed by a command.

   This parameter has a default value of 0.
}

\iueparameter{NSKIP}
{
   \_INTEGER
}{
   This is the number of tape marks to be skipped.

   This parameter has a default value of 1.
}

\iueparameter{OBJECT}
{
   \_CHAR
}{
   This is a string containing text to identify the object of the
   observation.
   It can also contain information about the observation
   ({\it{e.g.,}}\ camera, image\ldots ) if required.
   The maximum allowed length of the string is 40 characters.
}

\iueparameter{ORDER}
{
   \_INTEGER
}{
   This is the \'{e}chelle order number.
}

\iueparameter{ORDERS}
{
   [\_INTEGER\{,\_INTEGER\}]
}{
   This is a pair of \'{e}chelle order numbers delineating a range.
   If only a single value is specified, then the range consists of that
   order only.
   The sequence of the two numbers is not significant.
   The useful maximum range for each camera is as follows:

   \begin {quote}
   \begin {description}
      \item SWP --- orders 66 to 125.
      \item LWR --- orders 72 to 125.
      \item LWP --- orders 72 to 125.
   \end {description}
   \end {quote}
}

\iueparameter{OUTFILE}
{
   \_CHAR
}{
   This is the name of a file to receive the output spectrum.
   This release of IUEDR uses the STARLINK NDF format for all output
   spectra. This means that all standard STARLINK packages can be used
   to plot/display/analyse the spectra, in particular some of the
   facilities of KAPPA and FIGARO may prove useful to the general user.
}

\iueparameter{QUAL}
{
   \_LOGICAL
}{
   This specifies whether the data quality information is plotted
   along with the data.
   If so, then faulty points will be marked with their data quality
   severity code, which is one from:

   \begin {quote}
   \begin {description}
      \item 1 --- affected by extrapolated ITF.
      \item 2 --- affected by microphonics.
      \item 3 --- affected by spike.
      \item 4 --- affected by bright point (or user).
      \item 5 --- affected by reseau mark.
      \item 6 --- affected by ITF truncation.
      \item 7 --- affected by saturation.
      \item U --- affected by user edit.
   \end {description}
   \end {quote}

   User edits are only shown where no other fault is present.

   This option would normally only be used when assessing the quality
   of faulty points, possibly with a view to using them, {\it{i.e.}}\ marking
   them ``good'' with a cursor editor.

   This parameter has a default value of \verb+TRUE+\@.
}

\iueparameter{RESOLUTION}
{
   \_CHAR
}{
   This is the spectrograph resolution mode.
   The following modes are defined:

   \begin {quote}
   \begin {description}
      \item {\tt LORES} --- IUE Low Resolution.
      \item {\tt HIRES} --- IUE High Resolution (\'{e}chelle mode).
   \end {description}
   \end {quote}
}

\iueparameter{RIPA}
{
   \_DOUBLE
}{
   This is an empirical scale factor that can be used to modify the
   \'{e}chelle ripple function.
   The normal value is 1.0.
   The primary component of the ripple function is

   \begin {equation}
      {\rm SCALE} = (\frac{\sin x}{x})^2
   \end {equation}

   where

   \begin {equation}
      x = \frac{\pi \times {\rm RIPA} \times (\lambda - \lambda_c)
                \times {\rm ORDER}}
               {\lambda_c}
   \end {equation}

   and

   \begin {equation}
      \lambda_c = \frac{\rm RIPK}{\rm ORDER}
   \end {equation}

   The net spectrum is divided by SCALE above.
   Empirical values of \verb+RIPA+ can be used to optimise the ripple
   correction.

   See the \verb+RIPC+ and \verb+RIPK+ parameter descriptions.
}

\iueparameter{RIPC}
{
   [\_DOUBLE\{,\_DOUBLE\}]
}{
   This is a polynomial in $x$ used to modify the standard \'{e}chelle
   ripple calibration function.
   The calibration is given by

   \begin {equation}
      {\rm SCALE} = (\frac{\sin x}{x})^2 \times (
                {\rm RIPC}(1) + {\rm RIPC}(2) \times x +
                {\rm RIPC}(3) \times x^2 \ldots)
   \end {equation}

   where

   \begin {equation}
      x = \frac {\pi \times {\rm RIPA} \times (\lambda - \lambda_c) \times
                 {\rm ORDER}}
                {\lambda_c}
   \end {equation}

   and

   \begin {equation}
      \lambda_c = \frac {\rm RIPK}{\rm ORDER}
   \end {equation}

   The net spectrum is divided by SCALE above.

   See the \verb+RIPA+ and \verb+RIPK+ parameter descriptions.
}

\iueparameter{RIPFILE}
{
   \_CHAR
}{
   This is the name of a file containing an \'{e}chelle ripple calibration.
   A file type of \verb+.rip+ is assumed and need not be specified
   explicitly.
   If the file name contains a directory specification, then it should be
   enclosed in quotes.
}

\iueparameter{RIPK}
{
   [\_DOUBLE\{,\_DOUBLE\}]
}{
   This is the \'{e}chelle ripple constant measured in Angstroms.
   It corresponds to the central wavelength of \'{e}chelle order number 1.
   The central wavelength of an arbitrary ORDER is

   \begin {equation}
      \lambda_c = \frac {\rm RIPK}{\rm ORDER}
   \end {equation}

   Where this parameter is used for an entire HIRES dataset, the
   parameter can have more than one value, and represent a polynomial
   in ORDER

   \begin {equation}
      \lambda_c =  \frac{({\rm RIPK}(1) + {\rm RIPK}(2)
                         \times {\rm ORDER} + {\rm RIPK}(3)
                         \times {\rm ORDER}^2 + \ldots)}
                        {\rm ORDER}
   \end {equation}
}

\iueparameter{RL}
{
   [\_DOUBLE,\_DOUBLE]
}{
   This is a pair of radial coordinate values defining the start and end of
   the radial grid in an LBLS array.
   These radial coordinates are measured in geometric pixels, and run
   perpendicular to the dispersion direction.
   A coordinate value of 0.0 corresponds to the centre of the spectrum.

   Values \verb+RL=[0.0, 0.0]+ indicate that internal defaults are to be
   adopted.
   A single value is reflected symmetrically about 0.0.
}

\iueparameter{RM}
{
   \_LOGICAL
}{
   This determines whether the mean spectrum is reset before a mapping
   takes place.
   If the spectrum is not reset, then the spectra being mapped will be
   averaged with the existing mean spectrum.

   This parameter has a default value of \verb+TRUE+\@.
}

\iueparameter{RS}
{
   \_LOGICAL
}{
   This determines whether the display screen is reset before plotting.

   This parameter has a default value of \verb+TRUE+\@.
}

\iueparameter{RSAMP}
{
   \_DOUBLE
}{
   This is the sample spacing used for the radial grid in the LBLS array.
   If it does not fit an integral number of times into the grid limits,
   \verb+RL+, then the latter are adjusted to fit.

   Suggested values range from 0.707 to 1.414 pixels, the latter
   corresponding to the IUESIPS LBLS grid.

   This parameter has a default value of 1.414.
}

\iueparameter{SCANAV}
{
   \_DOUBLE
}{
   This is the HWHM of a triangle function with which pixels are folded
   during the generation of a scan across the image perpendicular to
   spectrograph dispersion.
   It is measured in geometric pixels.

   This parameter has a default value of 5.
}

\iueparameter{SCANDIST}
{
   \_DOUBLE
}{
   This is the distance of a scan across a HIRES image from the faceplate
   centre.
   It is measured in geometric pixels.
}

\iueparameter{SCANWV}
{
   \_DOUBLE
}{
   This is the central wavelength for a scan of a LORES image
   perpendicular to spectrograph dispersion.
   It is measured in Angstroms in vacuo.
}

\iueparameter{SKIPNEXT}
{
   \_LOGICAL
}{
   This determines whether the tape is positioned at the start of the
   next file after processing.
   If only the start of a file is being processed, then by setting
   \verb+SKIPNEXT=FALSE+ time can be saved.

   This parameter has a default value of \verb+FALSE+\@.
}

\iueparameter{SPECTYPE}
{
   \_INTEGER
}{
   This is the type of file, in the DIPSO SP format terminology, to be
   created. The following values are allowed:

   \begin {quote}
   \begin {description}
      \item {\tt 0} --- Starlink NDF format file.
      \item {\tt 1} --- SP1, fixed format text file.
      \item {\tt 2} --- SP2, free format text file.
   \end {description}
   \end {quote}

   It is recommended that datasets with many points be written with
   \verb+SPECTYPE=0+, which is more efficient in disk space and time spent
   reading and writing.
   A description of the format of each of these file types can be found
   in Section~\ref{se:spectrum}\@.

   This parameter has a default value of 0.
}

\iueparameter{TEMFILE}
{
   \_CHAR
}{
   This is the name of a file containing the standard spectrum template
   data.
   A  file type of \verb+.tem+ is assumed and need not be specified
   explicitly.
   If the file name contains a directory specification, then it should be
   enclosed in quotes.
}

\iueparameter{THDA}
{
   \_DOUBLE
}{
   This is the IUE camera temperature, measured in degrees Centigrade.
   It is used for such things as adjustments to fiducial positions
   and spectrograph dispersion relations.
   A value of 0.0 implies that no \verb+THDA+ value is available, the program
   will then  use a suitable mean \verb+THDA+ for the camera being used.
   Values for the \verb+THDA+ can be found in the IUE header text of the final
   spectrum file on the Guest Observer tape for IUESIPS\#2 ---
   \verb+THDA+ values derived from spectrum motion are best.
}

\iueparameter{THRESH}
{
   \_DOUBLE
}{
   This is the  minimum value considered to be good when using \verb+CLEAN+\@.
   All pixels with values below this threshold will be marked BAD.
}

\iueparameter{TYPE}
{
   \_CHAR
}{
   This is the type of dataset.
   Defined values are as follows:

   \begin {quote}
   \begin {description}
      \item {\tt RAW} --- IUE raw image.
      \item {\tt GPHOT} --- IUE GPHOT image (geometric and photometric).
      \item {\tt PHOT} --- IUE PHOT image (photometric only).
   \end {description}
   \end {quote}

   Types \verb+PHOT+ and \verb+GPHOT+ are not automatically distinguishable
   from IUE Guest Observer tape contents.
}

\iueparameter{V}
{
   \_CHAR
}{
   This is a string defining a list of items and includes
   any of the following characters:

   \begin {quote}
   \begin {description}
      \item {\tt H} --- header and file information.
      \item {\tt I} --- image details.
      \item {\tt F} --- fiducials.
      \item {\tt G} --- geometry.
      \item {\tt D} --- dispersion.
      \item {\tt C} --- centroid templates.
      \item {\tt R} --- \'{e}chelle ripple and halation.
      \item {\tt A} --- absolute calibration.
      \item {\tt S} --- raw spectrum.
      \item {\tt M} --- mean spectrum.
      \item {\tt *} --- all of the above.
      \item {\tt Q} --- image data quality summary.
   \end {description}
   \end {quote}
}

\iueparameter{VSHIFT}
{
   \_DOUBLE
}{
   This is the radial velocity of the detector relative to the Sun.
   It is measured in km/s and affects the calibrated wavelength
   scale as follows:

   \begin {equation}
      \lambda_{true} = \frac{\lambda_{obs}}
                            {(1 + \frac{\rm VSHIFT}{c})}
   \end {equation}

   where $c$ is the velocity of light in km/s.
}

\iueparameter{WCUT}
{
   [\_DOUBLE,\_DOUBLE]
}{
   This is one of the mechanisms that can be used to delimit the
   parts of \'{e}chelle orders that are calibrated for ripple response.
   The two values of this parameter are the start and end wavelengths
   for a specific \verb+ORDER+\@.

   Apart from poor ripple calibration, the order ends can also be affected
   by the parts of the camera faceplate that are retained in the image.
}

\iueparameter{WSHIFT}
{
   \_DOUBLE
}{
   This is a constant wavelength shift applied to spectrum wavelengths.
   It is measured in Angstroms and affects the spectrum wavelengths as follows:

   \begin {equation}
      \lambda_{new} = \lambda_{old} + {\rm WSHIFT}
   \end {equation}

   This is only used for LORES spectra.
}

\iueparameter{XCUT}
{
   [\_DOUBLE,\_DOUBLE]
}{
   This is one of the mechanisms that can be used to delimit the
   ends of \'{e}chelle orders that are calibrated for ripple response.
   The two values of this parameter are the start and end $x$ coordinates
   of the order, where

   \begin {equation}
      x = \frac{\pi \times {\rm RIPA} \times (\lambda - \lambda_c)
                \times {\rm ORDER}}
               {\lambda_c}
   \end {equation}

   and

   \begin {equation}
      \lambda_c = \frac {\rm RIPK}{\rm ORDER}
   \end {equation}

   The nature of the standard ripple function is that $x$ is only
   formally meaningful in the range ($-\pi$, $+\pi$).

   See the \verb+RIPA+, \verb+RIPC+ and \verb+RIPK+ parameter descriptions.
}

\iueparameter{XL}
{
   [\_DOUBLE,\_DOUBLE]
}{
   This specifies the data limits used for plotting in the $x$-direction.
   This parameter is only read if the display has been reset, and
   the axes are being redrawn.
   If both values are the same ({\it{e.g.,}}\ \verb+[0, 0]+),
   then the data limits in the $x$-direction will be determined from the
   data being plotted.
}

\iueparameter{XP}
{
   [\_INTEGER,\_INTEGER]
}{
   This specifies the pixel limits along the $x$-direction used for
   image display.
   Values in decreasing order will cause the image to be inverted
   along the $x$-direction.
   If the values are undefined, the pixel limits will default to include
   the whole extent of the image along the $x$-direction.
}

\iueparameter{YEAR}
{
   \_INTEGER
}{
   This is the year (A.D.) used in constructing dates.
}

\iueparameter{YL}
{
   [\_DOUBLE,\_DOUBLE]
}{
   This specifies the data limits used for plotting in the $y$-direction.
   This parameter is only read if the display has been reset, and
   the axes are being redrawn.
   If both values are the same ({\it{e.g.,}}\ \verb+[0, 0]+),
   then the data limits in the $y$-direction will be determined from the
   data being plotted.
}

\iueparameter{YP}
{
   [\_INTEGER,\_INTEGER]
}{
   This specifies the pixel limits along the $y$-direction used for
   image display.
   Values in decreasing order will cause the image to be inverted
   along the $y$-direction.
   If the values are undefined, the pixel limits will default to include
   the whole extent of the image along the $y$-direction.
}

\iueparameter{ZL}
{
   [\_DOUBLE,\_DOUBLE]
}{
   This specifies the data limits used for display of images.
   If the values are given in decreasing order, then high data
   values will be represented by low (dark) display intensities,
   and vice-versa.
   If the values are undefined, then the full intensity range of the
   image will be used.

   Data values which fall at or below the lowest display intensity are drawn
   {\bf black,} those which are at the highest display intensity are drawn
   {\bf white} and those which are above the highest display intensity are
   drawn {\bf blue.}
}

\iueparameter{ZONE}
{
   \_INTEGER
}{
   This specifies the zone to be used for plotting.
   The zone numbers range
   from 0 to 8 and correspond to those defined by the TZONE command in
   DIPSO (see \xref{SUN/50}{sun50}{}), {\it{e.g.,}}

   \begin {quote}
   \begin {description}
      \item {\tt 0} --- entire display surface.
      \item {\tt 1} --- top left quarter.
      \item {\tt 2} --- top right quarter.
      \item {\tt 3} --- bottom left quarter.
      \item {\tt 4} --- bottom right quarter.
      \item {\tt 5} --- top half.
      \item {\tt 6} --- bottom half.
      \item {\tt 7} --- left half.
      \item {\tt 8} --- right half.
   \end {description}
   \end {quote}

   This parameter has a default value of 0.
}
}

\newpage
%------------------------------------------------------------------------------

\appendix
\section{\xlabel{parameter_defaults}\label{se:parameter_defaults}Parameter
          defaults}
\markboth{Parameter defaults}{\stardocname}

Some IUEDR parameters have default values.  Some have no default value and one
{\bf must} be provided.  Other parameters values are either calculated or
simply set by the program.  The default values and/or behaviour of each
parameter are listed here.

\begin {description}

\item [\htmlref{ABSFILE}{ABSFILE}] \lmbox
   No default value exists, a file name must be provided.
   The parameter is cancelled each time it is used.
\item [\htmlref{APERTURE}{APERTURE}] \lmbox
   Has no automatic default value.  An \verb+APERTURE+ must be selected.
   If only one is present in an image then this is taken as \verb+APERTURE+ by
   default.
\item [\htmlref{APERTURES}{APERTURES}] \lmbox
   Used only by \verb+READIUE+ and \verb+READSIPS+\@.  The value must be
   supplied by reading the IUE GO header `by eye'.
\item [\htmlref{AUTOSLIT}{AUTOSLIT} = TRUE] \lmbox
   Whether \verb+GSLIT+, \verb+BDIST+ and \verb+BSLIT+ are determined
   automatically.
\item [\htmlref{BADITF}{BADITF} (= TRUE)] \lmbox
   Has no default value, however it is recommended to be set \verb+TRUE+ as
   this will ensure any ITF error correction available for the particular
   \verb+ITF+ will be used.
   This may seem counter-intuitive, however, data using error-free ITF
   information will not be affected by setting \verb+BADITF=TRUE+\@.
\item [\htmlref{BDIST}{BDIST}] \lmbox
   No automatic default.  The value is calculated if \verb+AUTOSLIT=TRUE+,
   otherwise it should be estimated by looking at \verb+SCAN+ plots.
\item [\htmlref{BKGAV}{BKGAV} = 30.0] \lmbox
   Background averaging filter FWHM (geometric pixels).
\item [\htmlref{BKGIT}{BKGIT} = 1] \lmbox
   Number of background smoothing iterations.
\item [\htmlref{BKGSD}{BKGSD} = 2.0] \lmbox
   Discrimination level for background pixels (standard deviations).
\item [\htmlref{BSLIT}{BSLIT}] \lmbox
   No automatic default.  The value is calculated if \verb+AUTOSLIT=TRUE+,
   otherwise it should be estimated by looking at \verb+SCAN+ plots.
\item [\htmlref{CAMERA}{CAMERA}] \lmbox
   The value is read by the program from an IUE GO header.  The default value
   presented in a prompt will be the value found in the header.
\item [\htmlref{CENAV}{CENAV} = 30.0] \lmbox
   Centroid averaging filter FWHM (geometric pixels).
\item [\htmlref{CENIT}{CENIT} = 1] \lmbox
   Number of centroid tracking iterations.
\item [\htmlref{CENSD}{CENSD} = 4.0] \lmbox
   Discrimination level for signal to be used for centroids
   (standard deviations).
\item [\htmlref{CENSH}{CENSH} = FALSE] \lmbox
   Whether the spectrum template is just shifted linearly.
\item [\htmlref{CENSV}{CENSV} = FALSE] \lmbox
   Whether the spectrum template is saved in the dataset.
\item [\htmlref{CENTM}{CENTM} = FALSE] \lmbox
   Whether an existing centroid template is used.
\item [\htmlref{COL}{COL} (= 1)] \lmbox
   The value of \verb+COL+ will be calculated for each command requiring it.
   If \verb+COLROT=FALSE+ then \verb+COL+ will take the value 1, otherwise it
   will be incremented for each plotting command.
\item [\htmlref{COLOUR}{COLOUR} = FALSE] \lmbox
   Whether a spectrum-style false colour look-up table is used by
   \verb+DRIMAGE+\@.
\item [\htmlref{COLROT}{COLROT} = TRUE] \lmbox
   Whether the line colour is changed after the next plot.
\item [\htmlref{CONTINUUM}{CONTINUUM} = TRUE] \lmbox
   Whether the object spectrum is expected to have a ``continuum''.
\item [\htmlref{COVERGAP}{COVERGAP} = FALSE] \lmbox
   Whether gaps can be filled by covering orders.
\item [\htmlref{CUTFILE}{CUTFILE}] \lmbox
   No default value exists, a file name must be provided.
   The parameter is cancelled each time it is used.
\item [\htmlref{CUTWV}{CUTWV} = TRUE] \lmbox
   Whether wavelength cutoff data is to be used for the extraction grid.
\item [\htmlref{DATASET}{DATASET}] \lmbox
   No automatic default value.  The last value of \verb+DATASET+ used will be
   taken as the default.  The exception to this rule is when using
   \verb+READIUE+ or \verb+READSIPS+ when a suggested default value will be
   presented by the program in the parameter prompt.
\item [\htmlref{DAY}{DAY}] \lmbox
   No default value.  The program will attempt to extract the \verb+DAY+ from
   the IUE GO tape/file header and present this as the default value for that
   particular command.
\item [\htmlref{DEVICE}{DEVICE}] \lmbox
   No automatic default value.  The last value of \verb+DEVICE+ used will be
   taken as the default.
\item [\htmlref{DISPFILE}{DISPFILE}] \lmbox
   No default value exists, a file name must be provided.
   The parameter is cancelled each time it is used.
\item [\htmlref{DRIVE}{DRIVE}] \lmbox
   No automatic default value.  The last value of \verb+DRIVE+ used will be
   taken as the default.
\item [\htmlref{ESHIFT}{ESHIFT}] \lmbox
   No automatic default value.  If an \verb+ESHIFT+ has previously been set for
   the current \verb+DATASET+, this will be presented as the default value in
   the prompt.
\item [\htmlref{EXPOSURE}{EXPOSURE}] \lmbox
   No automatic default value.  The value of \verb+EXPOSURE+ previously set when
   using \verb+READIUE+ or \verb+READSIPS+ for the current \verb+APERTURE+ will
   be presented as the default value in the prompt.
\item [\htmlref{EXPOSURES}{EXPOSURES}] \lmbox
   No default value.  The program will attempt to extract the \verb+EXPOSURES+
   from the IUE GO tape/file header and present this as the default value for
   that particular command.
\item [\htmlref{EXTENDED}{EXTENDED} = FALSE] \lmbox
   Whether the object spectrum is expected to be extended.
\item [\htmlref{FIDFILE}{FIDFILE}] \lmbox
   No default value exists, a file name must be provided.
   The parameter is cancelled each time it is used.
\item [\htmlref{FIDSIZE}{FIDSIZE}] \lmbox
   The default \verb+FIDSIZE+ is read from the appropriate file in the
   \verb+$IUEDR_DATA+ directory.  This is presented as the default in parameter
   prompts.
\item [\htmlref{FILE}{FILE}] \lmbox
   No automatic default.  Takes the value 1 when reading a new tape or file.
\item [\htmlref{FILLGAP}{FILLGAP} = FALSE] \lmbox
   Whether gaps can be filled within an order.
\item [\htmlref{FLAG}{FLAG} = TRUE] \lmbox
   Whether data quality for faulty pixels is displayed.
\item [\htmlref{FN}{FN}] \lmbox
   No default.  A value must be supplied.
\item [\htmlref{FSCALE}{FSCALE} = 1] \lmbox
   No scale factor is applied by default.
\item [\htmlref{GSAMP}{GSAMP} = 1.414] \lmbox
   Spectrum grid sampling rate (geometric pixels).
\item [\htmlref{GSHIFT}{GSHIFT}] \lmbox
   No automatic default value.  The value of \verb+GSHIFT+ previously set when
   using \verb+CGSHIFT+ will be presented as the default value in the prompt.
\item [\htmlref{GSLIT}{GSLIT}] \lmbox
   No automatic default.  The value is calculated if \verb+AUTOSLIT=TRUE+,
   otherwise it should be estimated by looking at \verb+SCAN+ plots.
\item [\htmlref{HALAV}{HALAV} = 30.0] \lmbox
   FWHM of triangle function used for spectrum smoothing.
\item [\htmlref{HALC}{HALC}] \lmbox
   No default value.  Values must be positive or zero.
\item [\htmlref{HALTYPE}{HALTYPE} = POWER] \lmbox
   Currently this parameter can only take the value \verb+POWER+\@.
\item [\htmlref{HALW0}{HALW0} = 1400.0 or 1800.0] \lmbox
   Takes the value \verb+1400.0+ for the SWP camera, \verb+1800.0+ for
   LWP and LWR cameras.
\item [\htmlref{HALWC}{HALWC} = 1200.0 or 2400.0] \lmbox
   Takes the value \verb+1200.0+ for the SWP camera, \verb+2400.0+ for
   LWP and LWR cameras.
\item [\htmlref{HIST}{HIST} = TRUE] \lmbox
   Whether lines are to be drawn as histograms during plotting.
\item [\htmlref{IMAGE}{IMAGE}] \lmbox
   No default value.  The program will attempt to extract the \verb+IMAGE+
   number from the IUE GO tape/file header and present this as the default
   value for that particular command.
\item [\htmlref{ITF}{ITF}] \lmbox
   No default value.  Refer to page~\pageref{ITF} for details of
   working out which transfer function to use.
\item [\htmlref{ITFMAX}{ITFMAX}] \lmbox
   No default value.  Refer to page~\pageref{ITFMAX} for details of
   working out which transfer function saturation value to use.
\item [\htmlref{LINE}{LINE} (=SOLID)] \lmbox
   The value of LINE will be calculated for each command requiring it.
   If \verb+LINEROT=FALSE+ then \verb+LINE+ will take the value \verb+SOLID+,
   otherwise it will be rotated for each plotting command.
\item [\htmlref{LINEROT}{LINEROT} = FALSE] \lmbox
   Whether line style is to be changed after the next plot.
\item [\htmlref{ML}{ML}] \lmbox
   No default value.  Limits {\bf must} be specified.
\item [\htmlref{MONTH}{MONTH}] \lmbox
   No default value.  The program will attempt to extract the \verb+MONTH+ from
   the IUE GO tape/file header and present this as the default value for that
   particular command.
\item [\htmlref{MSAMP}{MSAMP}] \lmbox
   No default value exists.
\item [\htmlref{NFILE}{NFILE} = 1] \lmbox
   Number of tape files to be processed.
\item [\htmlref{NGEOM}{NGEOM} = 5] \lmbox
   When reading IUE GO tapes/files this is the value suggested.
\item [\htmlref{NLINE}{NLINE} = 10] \lmbox
   Number of IUE header lines to be printed.
\item [\htmlref{NORDER}{NORDER} = 0] \lmbox
   Number of \'{e}chelle orders to be processed.
\item [\htmlref{NSKIP}{NSKIP} = 1] \lmbox
   Number of tape marks to be skipped over.
\item [\htmlref{OBJECT}{OBJECT}] \lmbox
   The program will attempt to extract the \verb+OBJECT+ description text from
   the IUE GO header.  This is presented as the default in parameter prompts.
\item [\htmlref{ORDER}{ORDER}] \lmbox
   No default value. A valid \verb+ORDER+ {\bf must} be specified.
   Use \verb+SHOW V=S+ to find out which orders have been extracted from a
   HIRES image.
\item [{\htmlref{ORDERS}{ORDERS} (=[125,66])}] \lmbox
   Takes the values given by default for a new dataset and otherwise in
   response to a parameter cancel.
\item [\htmlref{OUTFILE}{OUTFILE}] \lmbox
   No automatic default.  The program will construct a suggested file name
   based on the \verb+CAMERA+, \verb+RESOLUTION+, \verb+APERTURE+ and
   \verb+ORDER+ as appropriate.
\item [\htmlref{QUAL}{QUAL} = TRUE] \lmbox
   Whether data quality information is plotted.
\item [\htmlref{RESOLUTION}{RESOLUTION}] \lmbox
   The program will attempt to extract the spectrograph \verb+RESOLUTION+ from
   the IUE GO header.  This is presented as the default in parameter prompts.
\item [\htmlref{RIPA}{RIPA} (=1)] \lmbox
   This parameter takes the value 1 when a new dataset is created.
\item [{\htmlref{RIPC}{RIPC} (=[1,0,0,0,0,0])}] \lmbox
   Takes the default values given above when a new dataset is created.
\item [\htmlref{RIPFILE}{RIPFILE}] \lmbox
   No default value exists, a file name must be provided.
   The parameter is cancelled each time it is used.
\item [\htmlref{RIPK}{RIPK}] \lmbox
   Takes the central wavelength value of an order by default.
\item [\htmlref{RL}{RL}] \lmbox
   No default.  \verb+RL=[0.0, 0.0]+ indicates the program should calculate
   values.
\item [\htmlref{RM}{RM} = TRUE] \lmbox
   Whether the mean spectrum is reset before averaging.
\item [\htmlref{RS}{RS} = TRUE] \lmbox
   Whether the display is reset before plotting.
\item [\htmlref{RSAMP}{RSAMP} = 1.414] \lmbox
   Radial coordinate sampling rate for \verb+LBLS+ grid (pixels).
\item [\htmlref{SCANAV}{SCANAV} = 5] \lmbox
   Averaging filter HWHM for image scan (geometric pixels).
\item [\htmlref{SCANDIST}{SCANDIST} (=0)] \lmbox
   Takes the last value used.  The first time \verb+SCAN+ is used the
   program will suggest a value of zero.
\item [\htmlref{SCANWV}{SCANWV}] \lmbox
   The value of \verb+SCANWV+ taken by default is calculated as the centre
   of the wavelength scale for the current \verb+APERTURE+\@.
\item [\htmlref{SKIPNEXT}{SKIPNEXT} = FALSE] \lmbox
   Whether skip to next tape file.
\item [\htmlref{SPECTYPE}{SPECTYPE} = 0] \lmbox
   DIPSO SP format file type (0, 1 or 2). Type 0 is a Starlink NDF.
\item [\htmlref{TEMFILE}{TEMFILE}] \lmbox
   No default value exists, a file name must be provided.
   The parameter is cancelled each time it is used.
\item [\htmlref{THDA}{THDA} (=0.0)] \lmbox
   No default, however, \verb+THDA=0.0+ implies that no value is available
   and the program will select a suitable mean \verb+THDA+ for the camera
   being used.
\item [\htmlref{THRESH}{THRESH}] \lmbox
   No default value.  The last value used will be presented as the default
   in parameter prompts.
\item [\htmlref{TYPE}{TYPE}] \lmbox
   No default.  This must be evaluated from the IUE GO file header contents.
\item [\htmlref{V}{V} = H] \lmbox
   This parameter is cancelled each time it is used.
\item [\htmlref{VSHIFT}{VSHIFT}] \lmbox
   No automatic default value.  If an VSHIFT has previously been set for
   the current \verb+DATASET+, this will be presented as the default value in
   the prompt.
\item [\htmlref{WCUT}{WCUT}] \lmbox
   The wavelength cut-off values are normally read from a \verb+.cut+ file and
   these are used for \verb+WCUT+ prompt values by default.
\item [\htmlref{WSHIFT}{WSHIFT} = 0] \lmbox
   No automatic default value.  If a \verb+WSHIFT+ has previously been set for
   the current \verb+DATASET+, this will be presented as the default value in
   the parameter prompt.
\item [{\htmlref{XCUT}{XCUT} =[-3.0,3.0]}] \lmbox
   Takes the value \verb+[-3.0,3.0]+ for a new dataset.
\item [{\htmlref{XL}{XL} (=[0,0])}] \lmbox
   No default value.  The limits are taken as the full x-range in the data
   to be plotted if no value of \verb+YL+ is set.
\item [{\htmlref{XP}{XP} (=[0,0])}] \lmbox
   No default value.  The limits are taken as the full image width if \verb+YP+
   is not set.
\item [\htmlref{YEAR}{YEAR}] \lmbox
   No default value.  The program will attempt to extract the \verb+YEAR+ from
   the IUE GO tape/file header and present this as the default value for that
   particular command.
\item [{\htmlref{YL}{YL} (=[0,0])}] \lmbox
   No default value.  The limits are taken as the full y-range in the data
   to be plotted if no value of \verb+YL+ is set.
\item [{\htmlref{YP}{YP} (=[0,0])}] \lmbox
   No default value.  The limits are taken as the full image height if
   \verb+YP+ is not set.
\item [{\htmlref{ZL}{ZL} (=[0,0])}] \lmbox
   No default value.  The limits are taken as the full intensity range for the
   current \verb+DATASET+ if \verb+ZL+ is not set.
\item [\htmlref{ZONE}{ZONE} = 0] \lmbox
   Graphics zone to be used for plotting.

\end {description}

\newpage
\section{\xlabel{vms_data}\label{se:vmsunix}Handling of VMS IUEDR data files}
\markboth{Handling of VMS IUEDR data files}{\stardocname}

IUEDR data files have changed format in order to allow
inter-machine operation. However, VMS IUEDR will still read the old format
files if they are present (this is only useful on the VAX as all old
format files will have been created on VAXen). If you have old format files
then you should use \verb+iuecnv+ to convert them to the new format by
following the procedure described below.
The resulting files can then be used with UNIX IUEDR.

During the conversion of an IUE dataset \verb+iuecnv+ will create a number of
binary data files. Their filenames are as follows:

\begin {description}
   \item \verb+<dataset>.UEC+ --- calibration file.
   \item \verb+<dataset>_UED.SDF+ --- image data and quality file.
   \item \verb+<dataset>_UES.SDF+ --- uncalibrated spectrum file.
   \item \verb+<dataset>_UEM.SDF+ --- calibrated mean spectrum file.
\end {description}

where \verb+<dataset>+ refers to the IUEDR DATASET parameter.

The {\tt .SDF} files are STARLINK NDF format files and can be read and
processed by any of the standard  packages ({\it{e.g.,}}\ KAPPA,FIGARO).
These files are in a machine independent format and can be freely
copied between any of the platforms which STARLINK supports.

\subsection{Moving IUEDR files to UNIX systems}

The file formats used by UNIX IUEDR are based on the STARLINK standard
NDF library. This makes the files transportable between all supported
systems. If you have old VMS IUEDR files ({\it{i.e.}}\ {\tt .UEC}, {\tt
.UED}, {\tt .UES}, {\tt .UEM} files) then these will need to be
converted into the new format {\bf before} they are transferred to a UNIX
system.

There is a VMS executable provided for this purpose, and a command file
to use it. To use the executable you must first copy:

\begin{verbatim}
   /star/bin/iuedr/iuecnv.exe
   /star/bin/iuedr/iuecnv.com
\end{verbatim}

onto your VMS system (use binary transfer for {\tt iuecnv.exe}).

When {\tt iuecnv.exe} is installed, you can then move to a directory
where your IUEDR datasets are stored and type:

\begin{verbatim}
   $ @somedisk:[somedir]iuecnv  dataset
\end{verbatim}

where {\tt somedisk:[somedir]} is wherever you copied {\tt iuecnv} to, and
{\tt dataset} is the name of an IUE dataset ({\it{e.g.,}}\ SWP23456).

The program will then locate and convert all the IUEDR datafiles
associated with the named dataset. Note that the {\tt .UEC} file is also
converted (although its name stays the same).

When conversion is complete you may copy the files ({\tt .UEC} and {\tt
*\_UE*.SDF}) to your UNIX system. The {\tt .UEC} files MUST be
transferred in ascii mode, and the {\tt .SDF} files MUST be transferred
in binary mode.

UNIX NDF expects that NDF container files end in the extension \verb+.sdf+ and
does not yet recognise \verb+.SDF+ files. Thus you may need to rename
files to have the lowercase \verb+.sdf+ extension (depending upon how
you do the transfer).

\subsection{VAX-UNIX IUEDR image file exchange}

An IUEDR image file is one of \verb+RAW+, \verb+PHOT+, or \verb+GPHOT+ type
and consist of 768 $\times$ 768 pixels each stored in a 1 or 2-byte integer.

The transfer of files between VAX and UNIX systems is complicated by
the sophistication of the VAX file system. Under VMS the system
records a complex description of the precise format of all the  files
(and stores it in  the directory entry). Under UNIX this information
has to be provided by the user of the file when it is  opened. Because
of this difference it is sometimes necessary to use the following
format conversion utilities.

\subsubsection{VAX to UNIX}

If you wish to transfer IUE image data from a VAX onto a UNIX machine
in order to use the UNIX IUEDR then the transfer should be done using
FTP (in {\bf binary} mode).

If you intend to copy the file using some other method ({\it{e.g.,}}\ via NFS)
then you should first use the command:

\begin{verbatim}
   UNIX_FORMAT image-name
\end{verbatim}

to ensure the file is properly transferred.

Note that this also applies if you wish to just access the file
via NFS without explicitly transferring it.

\subsubsection{UNIX to VAX}

If you wish to transfer IUE image data from a UNIX machine onto a VAX
in order to use the VAX IUEDR then the transfer should be done using
FTP (in {\bf binary} mode) and the command:

\begin{verbatim}
   VAX_FORMAT  image-name
\end{verbatim}

should then be used to ensure the file has the correct format.

If you use some other method of transferring the file ({\it{e.g.,}}\ NFS) then
the above command is {\bf still} required.

\subsubsection{What will work?}

In general the following two commands will allow you to use any disk
based IUE image with any machine running IUEDR:

\begin{itemize}
   \item {\tt VAX\_FORMAT} sets the file format as required by VAX IUEDR
   \item {\tt UNIX\_FORMAT} sets the file format as required by UNIX IUEDR
\end{itemize}

Both commands operate only on the VAX.

\subsection{\label{se:nfs}Accessing data via NFS}

UNIX machines commonly provide disk sharing amongst remote machines
using the NFS protocol.

For example your data frame may reside on a DECstation local  disk
called \verb+iuedata+ in the Rutherford cluster on machine \verb+adam4+\@. In
order to get IUEDR to read it directly you could enter the following
in response to the DRIVE prompt:

\begin{verbatim}
   DRIVE> /adam4/iuedata/swp12345.raw
\end{verbatim}

To see which disks you have NFS access to you should use the {\tt
\%~df} command. In general any disks whose entry does not start with
\verb+/dev+ are being served by a remote machine.

{\bf IMPORTANT NOTE\\}
IUEDR allows you to use this method of data access with the following proviso.

If the data resides on a VAX served disk then you must first  convert
its directory entry (on the VAX) using the following command:

\begin{verbatim}
   $ UNIX_FORMAT image-name
\end{verbatim}

This command does not change the data in any way.  It merely alters the
description of the file format as stored in the VAX directory.

If at some later stage you wish to use the VAX version of IUEDR on the
same data file it first be necessary to use the command:

\begin{verbatim}
   $ VAX_FORMAT image-name
\end{verbatim}

to convert back.

\twocolumn[
\section{\label{se:index}Command and parameter index}
]
\markboth{Index}{\stardocname}
\findexentry{A}{ABSFILE}{46}
\indexentry{AESHIFT}{14}
\indexentry{AGSHIFT}{14}
\indexentry{APERTURES}{46}
\indexentry{APERTURE}{46}
\indexentry{AUTOSLIT}{47}
\findexentry{B}{BADITF}{47}
\indexentry{BARKER}{15}
\indexentry{BDIST}{47}
\indexentry{BKGAV}{47}
\indexentry{BKGIT}{47}
\indexentry{BKGSD}{48}
\indexentry{BSLIT}{48}
\findexentry{C}{CAMERA}{48}
\indexentry{CENAV}{48}
\indexentry{CENIT}{48}
\indexentry{CENSD}{48}
\indexentry{CENSH}{48}
\indexentry{CENSV}{48}
\indexentry{CENTM}{49}
\indexentry{CENTREWAVE}{49}
\indexentry{CGSHIFT}{15}
\indexentry{CLEAN}{16}
\indexentry{COLOUR}{49}
\indexentry{COLROT}{49}
\indexentry{COL}{49}
\indexentry{CONTINUUM}{50}
\indexentry{COVERGAP}{50}
\indexentry{CULIMITS}{16}
\indexentry{CURSOR}{17}
\indexentry{CUTFILE}{50}
\indexentry{CUTWV}{50}
\findexentry{D}{DATASET}{50}
\indexentry{DAY}{50}
\indexentry{DELTAWAVE}{50}
\indexentry{DEVICE}{50}
\indexentry{DISPFILE}{50}
\indexentry{DRIMAGE}{17}
\indexentry{DRIVE}{51}
\indexentry{EDIMAGE}{18}
\findexentry{E}{EDMEAN}{19}
\indexentry{EDSPEC}{20}
\indexentry{ERASE}{20}
\indexentry{ESHIFT}{51}
\indexentry{EXIT}{21}
\indexentry{EXPOSURES}{51}
\indexentry{EXPOSURE}{51}
\indexentry{EXTENDED}{51}
\findexentry{F}{FIDFILE}{51}
\indexentry{FIDSIZE}{51}
\indexentry{FILE}{51}
\indexentry{FILLGAP}{52}
\indexentry{FLAG}{52}
\indexentry{FN}{52}
\indexentry{FSCALE}{52}
\findexentry{G}{GSAMP}{52}
\indexentry{GSHIFT}{52}
\indexentry{GSLIT}{53}
\findexentry{H}{HALAV}{53}
\indexentry{HALC}{53}
\indexentry{HALTYPE}{53}
\indexentry{HALW0}{53}
\indexentry{HALWC}{53}
\indexentry{HELP}{21}
\indexentry{HIST}{53}
\findexentry{I}{IMAGE}{53}
\indexentry{ITFMAX}{54}
\indexentry{ITF}{54}
\findexentry{L}{LBLS}{21}
\indexentry{LINEROT}{55}
\indexentry{LINE}{54}
\indexentry{LISTIUE}{22}
\findexentry{M}{MAP}{23}
\indexentry{ML}{55}
\indexentry{MODIMAGE}{23}
\indexentry{MONTH}{55}
\indexentry{MSAMP}{55}
\indexentry{MTMOVE}{24}
\indexentry{MTREW}{24}
\indexentry{MTSHOW}{24}
\indexentry{MTSKIPEOV}{25}
\indexentry{MTSKIPF}{25}
\findexentry{N}{NEWABS}{25}
\indexentry{NEWCUT}{26}
\indexentry{NEWDISP}{26}
\indexentry{NEWFID}{26}
\indexentry{NEWRIP}{27}
\indexentry{NEWTEM}{27}
\indexentry{NFILE}{55}
\indexentry{NGEOM}{55}
\indexentry{NLINE}{55}
\indexentry{NORDER}{55}
\indexentry{NSKIP}{55}
\findexentry{O}{OBJECT}{56}
\indexentry{ORDERS}{56}
\indexentry{ORDER}{56}
\indexentry{OUTEM}{27}
\indexentry{OUTFILE}{56}
\indexentry{OUTLBLS}{28}
\indexentry{OUTMEAN}{28}
\indexentry{OUTNET}{29}
\indexentry{OUTRAK}{29}
\indexentry{OUTSCAN}{30}
\indexentry{OUTSPEC}{30}
\findexentry{P}{PLCEN}{31}
\indexentry{PLFLUX}{32}
\indexentry{PLGRS}{33}
\indexentry{PLMEAN}{34}
\indexentry{PLNET}{35}
\indexentry{PLSCAN}{36}
\indexentry{PRGRS}{37}
\indexentry{PRLBLS}{37}
\indexentry{PRMEAN}{37}
\indexentry{PRSCAN}{38}
\indexentry{PRSPEC}{38}
\findexentry{Q}{QUAL}{56}
\indexentry{QUIT}{38}
\findexentry{R}{READIUE}{39}
\indexentry{READSIPS}{40}
\indexentry{RESOLUTION}{56}
\indexentry{RIPA}{57}
\indexentry{RIPC}{57}
\indexentry{RIPFILE}{57}
\indexentry{RIPK}{57}
\indexentry{RL}{58}
\indexentry{RM}{58}
\indexentry{RSAMP}{58}
\indexentry{RS}{58}
\findexentry{S}{SAVE}{40}
\indexentry{SCANAV}{58}
\indexentry{SCANDIST}{58}
\indexentry{SCANWV}{58}
\indexentry{SCAN}{41}
\indexentry{SETA}{41}
\indexentry{SETD}{42}
\indexentry{SETM}{42}
\indexentry{SGS}{43}
\indexentry{SHOW}{43}
\indexentry{SKIPNEXT}{58}
\indexentry{SPECTYPE}{59}
\findexentry{T}{TEMFILE}{59}
\indexentry{THDA}{59}
\indexentry{THRESH}{59}
\indexentry{TRAK}{44}
\indexentry{TYPE}{59}
\findexentry{V}{VSHIFT}{60}
\indexentry{V}{59}
\findexentry{W}{WCUT}{60}
\indexentry{WSHIFT}{60}
\findexentry{X}{XCUT}{60}
\indexentry{XL}{61}
\indexentry{XP}{61}
\findexentry{Y}{YEAR}{61}
\indexentry{YL}{61}
\indexentry{YP}{61}
\findexentry{Z}{ZL}{61}
\indexentry{ZONE}{61}

%\input{sg3.ind}

\typeout{  }
\typeout{*****************************************************}
\typeout{  }
\typeout{Reminder: run this document through Latex three times}
\typeout{to resolve cross references.}
\typeout{  }
\typeout{*****************************************************}
\typeout{  }

\end {document}


\typeout{  }
\typeout{*****************************************************}
\typeout{  }
\typeout{Reminder: run this document through Latex three times}
\typeout{to resolve cross references.}
\typeout{  }
\typeout{*****************************************************}
\typeout{  }

\end {document}


\typeout{  }
\typeout{*****************************************************}
\typeout{  }
\typeout{Reminder: run this document through Latex three times}
\typeout{to resolve cross references.}
\typeout{  }
\typeout{*****************************************************}
\typeout{  }

\end {document}


\typeout{  }
\typeout{*****************************************************}
\typeout{  }
\typeout{Reminder: run this document through Latex three times}
\typeout{to resolve cross references.}
\typeout{  }
\typeout{*****************************************************}
\typeout{  }

\end {document}
