\documentstyle[11pt,twoside]{article}
\pagestyle{myheadings}

%
%   This version reformatted and prepared for processing into hypertext
%   for Starlink release by Martin Bly & Mike Lawden, May 1996.
%
%   Based on original ircamdr.doc by Colin Aspin, and sun41.tex v41.0
%   by Frossie, April 1996.
%

% -----------------------------------------------------------------------------
% ? Document identification
\newcommand{\stardoccategory}  {Starlink User Note}
\newcommand{\stardocinitials}  {SUN}
\newcommand{\stardocsource}    {sun41.4}
\newcommand{\stardocnumber}    {41.4}
\newcommand{\stardocauthors}   {Colin Aspin}
\newcommand{\stardocdate}      {23rd May 1996}
\newcommand{\stardoctitle}     {IRCAMDR \\[1ex] IRCAM3 Data Reduction Software}
\newcommand{\stardocversion}   {Version 1.1}
\newcommand{\stardocmanual}    {User's Manual}
% ? End of document identification
% -----------------------------------------------------------------------------

\newcommand{\stardocname}{\stardocinitials /\stardocnumber}
\markright{\stardocname}
\setlength{\textwidth}{160mm}
\setlength{\textheight}{230mm}
\setlength{\topmargin}{-2mm}
\setlength{\oddsidemargin}{0mm}
\setlength{\evensidemargin}{0mm}
\setlength{\parindent}{0mm}
\setlength{\parskip}{\medskipamount}
\setlength{\unitlength}{1mm}

\setcounter{tocdepth}{2}
\setcounter{secnumdepth}{3}

% -----------------------------------------------------------------------------
%  Hypertext definitions.
%  ======================
%  These are used by the LaTeX2HTML translator in conjunction with star2html.

%  Comment.sty: version 2.0, 19 June 1992
%  Selectively in/exclude pieces of text.
%
%  Author
%    Victor Eijkhout                                      <eijkhout@cs.utk.edu>
%    Department of Computer Science
%    University Tennessee at Knoxville
%    104 Ayres Hall
%    Knoxville, TN 37996
%    USA

%  Do not remove the %\begin{rawtex} and %\end{rawtex} lines (used by
%  star2html to signify raw TeX that latex2html cannot process).
%\begin{rawtex}
\makeatletter
\def\makeinnocent#1{\catcode`#1=12 }
\def\csarg#1#2{\expandafter#1\csname#2\endcsname}

\def\ThrowAwayComment#1{\begingroup
    \def\CurrentComment{#1}%
    \let\do\makeinnocent \dospecials
    \makeinnocent\^^L% and whatever other special cases
    \endlinechar`\^^M \catcode`\^^M=12 \xComment}
{\catcode`\^^M=12 \endlinechar=-1 %
 \gdef\xComment#1^^M{\def\test{#1}
      \csarg\ifx{PlainEnd\CurrentComment Test}\test
          \let\html@next\endgroup
      \else \csarg\ifx{LaLaEnd\CurrentComment Test}\test
            \edef\html@next{\endgroup\noexpand\end{\CurrentComment}}
      \else \let\html@next\xComment
      \fi \fi \html@next}
}
\makeatother

\def\includecomment
 #1{\expandafter\def\csname#1\endcsname{}%
    \expandafter\def\csname end#1\endcsname{}}
\def\excludecomment
 #1{\expandafter\def\csname#1\endcsname{\ThrowAwayComment{#1}}%
    {\escapechar=-1\relax
     \csarg\xdef{PlainEnd#1Test}{\string\\end#1}%
     \csarg\xdef{LaLaEnd#1Test}{\string\\end\string\{#1\string\}}%
    }}

%  Define environments that ignore their contents.
\excludecomment{comment}
\excludecomment{rawhtml}
\excludecomment{htmlonly}
%\end{rawtex}

%  Hypertext commands etc. This is a condensed version of the html.sty
%  file supplied with LaTeX2HTML by: Nikos Drakos <nikos@cbl.leeds.ac.uk> &
%  Jelle van Zeijl <jvzeijl@isou17.estec.esa.nl>. The LaTeX2HTML documentation
%  should be consulted about all commands (and the environments defined above)
%  except \xref and \xlabel which are Starlink specific.

\newcommand{\htmladdnormallinkfoot}[2]{#1\footnote{#2}}
\newcommand{\htmladdnormallink}[2]{#1}
\newcommand{\htmladdimg}[1]{}
\newenvironment{latexonly}{}{}
\newcommand{\hyperref}[4]{#2\ref{#4}#3}
\newcommand{\htmlref}[2]{#1}
\newcommand{\htmlimage}[1]{}
\newcommand{\htmladdtonavigation}[1]{}

%  Starlink cross-references and labels.
\newcommand{\xref}[3]{#1}
\newcommand{\xlabel}[1]{}

%  LaTeX2HTML symbol.
\newcommand{\latextohtml}{{\bf LaTeX}{2}{\tt{HTML}}}

%  Define command to re-centre underscore for Latex and leave as normal
%  for HTML (severe problems with \_ in tabbing environments and \_\_
%  generally otherwise).
\newcommand{\latex}[1]{#1}
\newcommand{\setunderscore}{\renewcommand{\_}{{\tt\symbol{95}}}}
\latex{\setunderscore}

%  Redefine the \tableofcontents command. This procrastination is necessary
%  to stop the automatic creation of a second table of contents page
%  by latex2html.
\newcommand{\latexonlytoc}[0]{\tableofcontents}

% -----------------------------------------------------------------------------
%  Debugging.
%  =========
%  Remove % on the following to debug links in the HTML version using Latex.

% \newcommand{\hotlink}[2]{\fbox{\begin{tabular}[t]{@{}c@{}}#1\\\hline{\footnotesize #2}\end{tabular}}}
% \renewcommand{\htmladdnormallinkfoot}[2]{\hotlink{#1}{#2}}
% \renewcommand{\htmladdnormallink}[2]{\hotlink{#1}{#2}}
% \renewcommand{\hyperref}[4]{\hotlink{#1}{\S\ref{#4}}}
% \renewcommand{\htmlref}[2]{\hotlink{#1}{\S\ref{#2}}}
% \renewcommand{\xref}[3]{\hotlink{#1}{#2 -- #3}}
% -----------------------------------------------------------------------------
% ? Document specific \newcommand or \newenvironment commands.
\newcommand{\bull}{$\bullet$}
\begin{htmlonly}
\renewcommand{\bull}{}
\end{htmlonly}
% ? End of document specific commands
% -----------------------------------------------------------------------------
%  Title Page.
%  ===========
\renewcommand{\thepage}{\roman{page}}
\begin{document}
\thispagestyle{empty}

%  Latex document header.
%  ======================
\begin{latexonly}
   CCLRC / {\sc Rutherford Appleton Laboratory} \hfill {\bf \stardocname}\\
   {\large Particle Physics \& Astronomy Research Council}\\
   {\large Starlink Project\\}
   {\large \stardoccategory\ \stardocnumber}
   \begin{flushright}
   \stardocauthors\\
   \stardocdate
   \end{flushright}
   \vspace{-4mm}
   \rule{\textwidth}{0.5mm}
   \vspace{5mm}
   \begin{center}
   {\Huge\bf  \stardoctitle \\ [2.5ex]}
   {\LARGE\bf \stardocversion \\ [4ex]}
   {\Huge\bf  \stardocmanual}
   \end{center}
   \vspace{5mm}

% ? Heading for abstract if used.
   \vspace{10mm}
   \begin{center}
      {\Large\bf Abstract}
   \end{center}
% ? End of heading for abstract.
\end{latexonly}

%  HTML documentation header.
%  ==========================
\begin{htmlonly}
   \xlabel{}
   \begin{rawhtml} <H1> \end{rawhtml}
      \stardoctitle\\
      \stardocversion\\
      \stardocmanual
   \begin{rawhtml} </H1> \end{rawhtml}

% ? Add picture here if required.
% ? End of picture

   \begin{rawhtml} <P> <I> \end{rawhtml}
   \stardoccategory \stardocnumber \\
   \stardocauthors \\
   \stardocdate
   \begin{rawhtml} </I> </P> <H3> \end{rawhtml}
      \htmladdnormallink{CCLRC}{http://www.cclrc.ac.uk} /
      \htmladdnormallink{Rutherford Appleton Laboratory}
                        {http://www.cclrc.ac.uk/ral} \\
      \htmladdnormallink{Particle Physics \& Astronomy Research Council}
                        {http://www.pparc.ac.uk} \\
   \begin{rawhtml} </H3> <H2> \end{rawhtml}
      \htmladdnormallink{Starlink Project}{http://www.starlink.ac.uk/}
   \begin{rawhtml} </H2> \end{rawhtml}
   \htmladdnormallink{\htmladdimg{source.gif} Retrieve hardcopy}
      {http://www.starlink.ac.uk/cgi-bin/hcserver?\stardocsource}\\

%  HTML document table of contents.
%  ================================
%  Add table of contents header and a navigation button to return to this
%  point in the document (this should always go before the abstract \section).
  \label{stardoccontents}
  \begin{rawhtml}
    <HR>
    <H2>Contents</H2>
  \end{rawhtml}
  \renewcommand{\latexonlytoc}[0]{}
  \htmladdtonavigation{\htmlref{\htmladdimg{contents_motif.gif}}
        {stardoccontents}}

% ? New section for abstract if used.
  \section{\xlabel{abstract}Abstract}
% ? End of new section for abstract
\end{htmlonly}

% -----------------------------------------------------------------------------
% ? Document Abstract. (if used)
%   ==================

The {\sc ircam3} data reduction and analysis software package, {\sc
IrcamDR} (the old {\tt ircam\_clred} updated to work under Unix
and renamed) will analyse and display any 2D data image with a {\tt .sdf}
 file extension stored in the standard Starlink NDF data format.

It reduces and analyses {\sc ircam1/2} data images of 62$\times$58
pixels and {\sc ircam3} images of 256$\times$256 size.  Almost all the
applications (with the exception of {\bf med3d}) will work on NDF
images of any physical (pixel) dimensions, for example,
1024$\times$1024 CCD images can be processed ({\bf med3d} median
filters stacks of images up to 256$\times$256 in size at present).

% ? End of document abstract
% -----------------------------------------------------------------------------
% ? Latex document Table of Contents (if used).
%  ===========================================
 \newpage
 \begin{latexonly}
   \setlength{\parskip}{0mm}
   \latexonlytoc
   \setlength{\parskip}{\medskipamount}
   \markboth{IrcamDR}{\stardocname}
 \end{latexonly}
% ? End of Latex document table of contents
% -----------------------------------------------------------------------------
\newpage
\renewcommand{\thepage}{\arabic{page}}
\setcounter{page}{1}

\section{\label{se:soft_structure}\xlabel{soft_structure}{\sc IrcamDR} software philosophy and structure}

The {\sc IrcamDR} software runs on Unix/VMS machines under the Starlink
Infrastructure and utilizes the ICL command line environment.  {\sc ADAM}
tasks perform most of the functions available in {\sc IrcamDR} and are
controlled using ICL procedures.

The {\sc ADAM} tasks (written in Fortran 77) used by {\sc IrcamDR} are
called {\bf plt2d}, {\bf rapi2d}, {\bf obsrap} and {\bf polrap}.

\begin{description}

\item [plt2d] performs graphics display for {\sc IrcamDR} and will plot
images and line graphics on the selected graphics device.  Applications
within {\bf plt2d} are exclusively executed via ICL command line
procedures rather than by calling the {\bf plt2d} applications directly.

\item [rapi2d] is the original a-task monolith and contains numerous
applications to manipulate and process images; {\bf rapi2d} was the
original kernel source for the Starlink KAPPA reduction task.

\item [obsrap] is an additional reduction and analysis a-task monolith
more specific to the processing of {\sc ircam1/2} and {\sc ircam3} data,
although many of the applications can be used on other datasets.

\item [polrap] is an {\sc ADAM} a-task monolith with applications specific
to the reduction and analysis of 2D imaging polarimetry data taken with
{\sc ircam1/2/3} using the {\sc ukirt} {\sc irpol1/2} polarimeter module.

\end{description}

Currently, {\bf plt2d} contains 48 applications, {\bf rapi2d} contains
53 applications, {\bf obsrap} contains 55 applications and {\bf polrap}
contains 12 applications.  There are also currently 230 ICL procedures
in {\sc IrcamDR}.

What follows are instructions on how to run {\sc IrcamDR}, descriptions
of all the ICL procedures and {\sc ADAM} task applications, and notes
on the recommended methods of reducting and analysing {\sc ircam3}
data.

\section{\label{se:running_ircamdr}\xlabel{running_ircamdr}Running {\sc IrcamDR}}

In the Starlink environment you can run {\sc IrcamDR} by typing the command
name:

\begin{verbatim}
      % ircamdr
\end{verbatim}

where {\tt `\%'} is the shell prompt.  If this fails, check that you have
sourced the Starlink login files.  If {\sc IrcamDR} is not installed
at your site, you will see a polite message telling
you so, and what you should do about it.

You can set the current directory to the location of your data before
you execute the command {\tt ircamdr} (using the {\tt cd} command), or
you can just run {\sc IrcamDR} which will prompt you for the name of
the directory containing the data to be worked on.  You may also change
directory later from within {\sc IrcamDR} by using the ICL {\tt def}
command .

Next, {\sc IrcamDR} will run ICL and load some of the ICL procedures
and the {\sc ADAM} \mbox{a-task} monolith {\bf obsrap}.  This is
probably the step that would fail if there is a problem with the software
installation. Next, {\sc IrcamDR} will ask you to select a graphics
device; choose device No.\ 1 for an X-windows GKS graphics window.  The
GKS graphics window (titled GKS\_3800) will be created at this time and
be of the default size.  However, you can pre-create the GKS X-windows
graphics window using the command:

\begin{verbatim}
      % xmake GKS_3800 -geom 380x380 -col 64 -fg white -bg black
\end{verbatim}

where, in this example, the window will be of size 380$\times$380
pixels.  These numbers can be adjusted to suit the application you have
in mind or the size of the images being worked on.

Once you have selected the graphics device, {\sc IrcamDR} loads the
plotting tasks {\bf plt2d} and opens plotting to the selected device.
It then asks you to define whether you are working on new or old
format {\sc ircam} data ({\sc ircam3} is new, {\sc ircam1/2} is old).

If you select new format then you will be prompted for the UT date for
the raw {\sc ircam3} ({\sc ro}) observation images, in the format YYMMDD, and
then the default pixel scale of 0.286 arcseconds/pixel is defined
automatically.  Once the prompt \verb+IrcamDR>+ is given, you can
proceed with {\sc IrcamDR} data reduction and analysis.

The following is an example of an {\sc IrcamDR} startup:

\begin{small}
\begin{verbatim}
      % ircamdr

      Welcome to IRCAM_CLRED/Portable-IRCAMDR V1.0-0

      Working directory //home/frossie/data/cjs/ >

      ICL (UNIX) Version 3.1 14/09/95

      Loading installed package definitions...

        - Type HELP package_name for help on specific Starlink packages
        -   or HELP PACKAGES for a list of all Starlink packages
        - Type HELP [command] for help on ICL and its commands


         KAPPA commands are now available (Version 0.9-0).

         Type `help kappa' or `kaphelp' for help on KAPPA commands.

      Welcome to Unix IRCAM_CLRED/Portable-IRCAMDR !

      Loading /ukirt_sw/IrcamDR/obsrap into obsrap29159 (attached)
      Loading Unix utilities
      Loading SUN Unix open plotting procedure...
      Select a plotting device from the list below ...
      Loading procedure file $LIRCAMDIR/popen_unix.icl
      Plotting workstations currently supported :
      X-Windows base device                   =  1
      X-Windows 2 device                      =  2
      X-Windows 3 device                      =  3
      X-Windows 4 device                      =  4
      Canon laser printer (P or L)            = 10 or 11
      Canon TeX laser printer (P or L)        = 12 or 13
      Postscript printer (P or L)             = 14 or 15
      Encapsulated Postscript (P or L)        = 16 or 17
      Colour Postscript (P or L)              = 18 or 19
      Encapsulated Colour Postscript (P or L) = 20 or 21
      Give workstation number for plotting ?
      Workstation Number (0=NONE AT THE MOMENT) \0\ : 1
      Loading /ukirt_sw/IrcamDR/plt2d into plt2d (attached)

       Maximum number of colour cells available =   62
        Workstation centre coordinates are  390 ,256
        Workstation supports CURSORING ...
        Workstation supports IMAGE DISPLAY ...
          Loading colour table  $LIRCAMDIR/col19  please wait
          Colour table loaded ...
      O.K. Plotting OPEN on workstation  X-WINDOWS
      New (RO940422) or Old (IRCAM_22APR94_1C) data format :
      New or Old Format (N,O) \N\ ?
      Default UT date =  960226
      UT date of files to work on (e.g. 940422) [RETURN=DEFAULT] :
      UT Date \-1\ ? 960105
      Loading procedure file $LIRCAMDIR/setps.icl
      O.K. have set the PLATE SCALE to  0.286  arcsecs/pixel ...
      IrcamDR >
\end{verbatim}
\end{small}

\section{\label{se:soft_notes}\xlabel{soft_notes}Software Notes}

\subsection{\label{ss:data_input_format_raw}\xlabel{data_input_format_raw}Data input format -- raw data}

{\sc ircam3} data images are stored on disk in standard Starlink NDF
format with the file extension {\tt .sdf}.  {\sc ircam1/2} images are
also stored on disk in a Starlink HDS format, but are non-standard in
that one data file contains many observations (the old {\sc ircam}
container files). Raw {\sc ircam3} images that are to be processed by
the {\sc IrcamDR} software have names in the format {\tt ro950102.sdf}
where the {\tt 950102} is the UT date of the observation.  Once the
file type and format have been defined, either on startup of {\sc
IrcamDR} (via the questions {\sc IrcamDR} asks you) or via running the
command setfile, the user can access the raw data image in many
commands by just entering the observation number alone.

\subsection{\label{ss:data_input_format_processed}\xlabel{data_input_format_processed}Data input format -- processed data}

If you want to access processed images (any images for that
matter, {\sc ircam1/2} or {\sc ircam3} processed images or other 2D
images), you can enter 0 as the observation number.  You are then
prompted for the full name of the image to be accessed.

In many ICL procedural commands (\emph{e.g.}, those ending with the word
`{\bf lot}' or such commands as {\bf jitreg}, {\bf accoff}, {\bf coff},
\emph{etc.},) you will need to access processed or semi-processed data.
You can do so by entering a prefix, a range of observation numbers
and a suffix which together will form the sequence of images to be
accessed.

For example, if you input a prefix of {\tt image\_} and a range of
observation numbers 100 to 110 and a suffix of {\tt dfzm,} then the
images accessed by the command will be {\tt image\_100dfzm}, {\tt
image\_101dfzm}, {\tt image\_102dfzm}, \ldots, {\tt image\_110dfzm}.

In the above example, the {\tt dfzm} part actually means something and
comprise letters added by various commands in the standard reduction
procedure for {\sc ircam3} images ({\tt d} means dark subtracted, {\tt
f} means flat-fielded, {\tt z} means dc level corrected and {\tt m}
means bad-pixel mask applied).  The command {\bf stred} adds these
letters to the prefix you have given and the observation number range
to form the processed image names.

\subsection{\label{ss:lot_procedures}\xlabel{lot_procedures}The ``lot'' family of ICL procedures}

There are numerous ICL procedures in {\sc IrcamDR} whose names end with
the word ``{\bf lot}''.  These perform a specific task on a series (or
a lot) of images defined by a consecutive observation number sequence
(either a start and end observation number, or a start observation
number and the number of consecutive images).

An example of the ``{\bf lot}'' family of procedures is {\bf flatlot} which
flat-fields a series of images (dark subtracted) defined by a filename
prefix, a series of observation numbers and a filename suffix.  In the
case of {\sc ircam3} data and the procedure {\bf flatlot}, {\bf
rodarklot} would have been used to dark subtract and scale to unit
exposure time a series of {\sc ro} images and the output dark subtracted
images would be called {\tt im\_50d}, {\tt im\_51d}, \ldots, {\tt
im\_55d}, if the output filename prefix had been specified as {\tt
im\_} and the observation number range as {\tt 50-55}.  The suffix
letter {\tt d} is added automatically by {\bf rodarklot}.  These images
would be fed into {\bf rodarklot} (say), and upon flat-fielding the
letter {\tt f} would be added to the name so that the output images
would be called {\tt im\_50df}, {\tt im\_51df}, \ldots, {\tt im\_55df}.

\section{\label{se:ircam3_data_analysis}\xlabel{se:ircam3_data_analysis}The reduction and analysis of {\sc ircam3} data}

The sequence used at {\sc ukirt} to reduce raw {\sc ircam3}
images is as follows:

\subsection{\label{ss:standard_imaging}\xlabel{standard_imaging}Standard imaging observation using {\tt J}, {\tt H}, {\tt K}, {\tt nbL}, or 1-2um {\tt nb} filters.}

\begin{enumerate}

\item Dark subtraction using dark of same on-chip exposure time as
object image(s),

Commands to use are {\bf darklot} (for {\sc ircam1/2}
images) or {\bf rodarklot} for {\sc ircam3} images,

\item Flat-field creation using either:

\begin{enumerate}

\item a median-filtered set of object images,
\item a median-filtered set of separate sky observations,
\item a single dark subtracted sky observation (hopefully with no
astronomical features present unless object images are point sources
and they do not overlap with the features in the separate single sky
image).

\end{enumerate}

Commands to use are {\bf romed} or {\bf med3d} (called by {\bf
romed}) for median-filtering objects.

\item Flat-fielding of the dark subtracted object images.

Commands to
use are {\bf flatlot} for flat-fielding a sequence of object images or
{\bf flat2} for flat-fielding two images one with the other.  To flat-field
manually, use the commands {\bf stats} (to get the median of the whole dark-subtracted sky image), {\bf cdiv} (to divide the dark-subtracted sky image by
its median), and {\bf div} (to divide the dark-subtracted object image by the
dark subtracted scaled sky image).

\item Airmass correction (assuming extinction coefficients are known
and if a standard at the same airmass is not available) using the
commands {\bf amcorr} or {\bf amcorrlot}.

\item Bad pixel masking using the commands {\bf applymask} or {\bf
applymasklot}.  This sets the known bad pixels on the array to the
magic number {\tt -1.0e-20}.

\item Mosaicing, if required, using {\bf mosaic}, {\bf wmosaic}, {\bf
crequilt}, {\bf quilt} or {\bf wquilt}.

\end{enumerate}

For many mosaic-imaging applications (large field or jitter
mosaic-imaging), the command {\bf stred} can be used to perform all
these tasks automatically.  This is the recommended method of data
reduction in {\sc IrcamDR}.

\subsection{\label{ss:thermal_ir_imaging}\xlabel{thermal_ir_imaging}Thermal
IR imaging using {\tt L'}, {\tt nbM} filters and 64$\times$64 sub-array readout}

Generally, thermal imaging with {\sc ircam3} will be taken in chop mode
using the {\sc ukirt} chopping secondary.  For point sources, chopping
often takes place on the array in that the chop offset is small and the
object remains in the array image in both chop phases (A and B or
OBJECT and SKY).  It is usual to take numerous repeat chopped A and B
pairs of observations on a source in the thermal IR and then to coadd
them later with any necessary spatial alignment.  Nodding of the
telescope is also generally used to help define any thermal imbalance
across the source.

You should attempt to use the {\sc IrcamDR} procedure {\bf chred} to
reduce your chopped thermal imaging data.  This allows for multiple A
and B pairs and also for nodding if appropriate.  Once the set of
chopped imaging data has been reduced -- the process is just to subtract
chop phases A and B for each nod position and sum them separately
giving, in nodding mode, two output images that can be combined later
-- many of the other reduction steps detailed above can be applied
(say, from step 4 above).

\subsection{\label{ss:imaging_polarimetry_observations}\xlabel{imaging_polarimetry_observations}Imaging polarimetry observations}

Users should see the internal {\sc ukirt} software user note on
polarimetry reduction of {\sc ircam} data.  This is available via the {\sc
ukirt} on-line WWW information system.

\newpage

%  LISTS

\section{\label{se:command_lists}\xlabel{command_lists}Lists of Commands}

%  LIST1

\subsection{\label{ss:icl_procedure_commands_alpha}\xlabel{icl_procedure_commands_alpha}ICL procedure commands -- alphabetical}

\begin{description}
\begin{description}

\item [\htmlref{ABLOCK}{ABLOCK}]: plot a colour table (LUT) colour block

\item [\htmlref{ACCOFF}{ACCOFF}]: plot sequentially pairs of a series
of images from a raster mosaic and calculate accurate spatial offsets
using cursor input and peak pixel in search box

\item [\htmlref{AGAIN}{AGAIN}]: re-display the last image in the same
manner as previously requested

\item [\htmlref{AMCORRLOT}{AMCORRLOT}]: airmass correction on a series
of images

\item [\htmlref{ANNCOL}{ANNCOL}]: define colour for use in image annotation

\item [\htmlref{APPLYMASKLOT}{APPLYMASKLOT}]: apply a bad pixel mask to
a series of images

\item [\htmlref{ARRAY\_TESTS}{ARRAY_TESTS}]: calculate the STARE and
ND\_STARE readout noise, and the dark current from engineering data
taken with the EXEC named {\tt ARRAY\_TESTS}

\item [\htmlref{BORDER}{BORDER}]: plot a boarder round an image

\item [\htmlref{BOX}{BOX}]: plot a box on an image

\item [\htmlref{CABLOCK}{CABLOCK}]: plot a colour table (LUT) colour
block at cursor position

\item [\htmlref{CALEXP}{CALEXP}]: calculate the optimum exposure time

\item [\htmlref{CALMAG}{CALMAG}]: calculate object magnitude from counts

\item [\htmlref{CALZER}{CALZER}]: calculate a zeropoint from a
user-defined actual object magnitude

\item [\htmlref{CBOX}{CBOX}]: plot a box on an image, centered on cursor

\item [\htmlref{CCIRCLE}{CCIRCLE}]: plot a circle, centered on cursor

\item [\htmlref{CCROSS}{CCROSS}]: plot a cross, centered on cursor

\item [\htmlref{CCUT}{CCUT}]: plot a line cut (intensity v. position),
cursor defined positions

\item [\htmlref{CELLIPSE}{CELLIPSE}]: plot an ellipse, cursor defined position

\item [\htmlref{CENT1}{CENT1}]: define a position within an image

\item [\htmlref{CENT2}{CENT2}]: define two positions within an image

\item [\htmlref{CHDISP}{CHDISP}]: plot CHOP phase difference

\item [\htmlref{CHRED}{CHRED}]: reduce a set of CHOP observations

\item [\htmlref{CIRCLE}{CIRCLE}]: plot a circle on an image

\item [\htmlref{CLEAR}{CLEAR}]: clear current image display

\item [\htmlref{CLEARIT}{CLEARIT}]: clear a selected portion of the
current image display

\item [\htmlref{CLINE}{CLINE}]: plot a line, cursor defined position

\item [\htmlref{CNSIGMA}{CNSIGMA}]: plot an image with nsigma scaling

\item [\htmlref{COFF}{COFF}]: calculate accurate spatial offsets for a
series of images taken at supposedly the same spatial position

\item [\htmlref{COLINV}{COLINV}]:  invert the colours in a colour table (LUT)

%  \item [\htmlref{COLL}{COLL}]: collimation program for {\sc ircam3}

\item [\htmlref{COLTAB}{COLTAB}]: change/load colour table (LUT)

\item [\htmlref{CONT\_TITLE}{CONT_TITLE}]: set the title string for a
line graphics plot

\item [\htmlref{CONTOFF}{CONTOFF}]: relocate contour plot on an image

\item [\htmlref{CONTOUR}{CONTOUR}]: plot a contour map on an image

\item [\htmlref{CPLOT}{CPLOT}]: plot an image with user-defined {\it
max} and {\it min}

\item [\htmlref{CRANPLOT}{CRANPLOT}]: plot an image with user-defined
range on the mean

\item [\htmlref{CROSS}{CROSS}]: plot a cross on an image

\item [\htmlref{CROSSCUT}{CROSSCUT}]: plot RA,Dec (X and Y) cross-cut
through a cursor selected point on an image

\item [\htmlref{CROSSCUT\_PEAK}{CROSSCUT_PEAK}]: plot RA,Dec (X and Y)
cross-cut through the peak pixel in a small box centred on a cursor
selected point on an image

\item [\htmlref{CSTATS}{CSTATS}]: return statistical information from a
sub-area of an image

\item [\htmlref{CURHOT}{CURHOT}]: automatically remove bad/hot pixels
from a sub-area of an image located at a cursor defined position

\item [\htmlref{CURSOR}{CURSOR}]: display cursor on an image and
provide X,Y,Z data at cursor position

\item [\htmlref{CUT}{CUT}]: plot a cut/slice through an image

\item [\htmlref{CUT2FF}{CUT2FF}]: switch on/off writing of cut/slice
pixel data to a file

\item [\htmlref{CUT\_TITLE}{CUT_TITLE}]: set the title string for a
{\bf \htmlref{CUT}{CUT}/\htmlref{CCUT}{CCUT}} plot through an image

\item [\htmlref{CVARGREY}{CVARGREY}]: plots an image using variable
index scaling

\item [\htmlref{DAOCEN}{DAOCEN}]: {\sc IrcamDR} DAOPHOT utility

\item [\htmlref{DAOFIND}{DAOFIND}]: {\sc IrcamDR} DAOPHOT utility

\item [\htmlref{DAOGID}{DAOGID}]: {\sc IrcamDR} DAOPHOT utility

\item [\htmlref{DAOGID2}{DAOGID2}]: {\sc IrcamDR} DAOPHOT utility

\item [\htmlref{DARKLOT}{DARKLOT}]: dark subtract a series of raw {\sc
ircam1/2} format images from a container file

\item [\htmlref{DEGLOT}{DEGLOT}]: de-glitch a series of images using a
bad pixel mask

\item [\htmlref{DISP}{DISP}]: plot the difference of two images (A-B)

\item [\htmlref{DISPICK}{DISPICK}]: extract and store a sub-area of an
image into a new image

\item [\htmlref{ELLIPSE}{ELLIPSE}]: plot an ellipse on an image

\item [\htmlref{FLAT2}{FLAT2}]: flat-field two images A,B;  A with B
and B with A

\item [\htmlref{FLATLOT}{FLATLOT}]: flat-field a number of images with
the same flat-field image

\item [\htmlref{GETCM}{GETCM}]: get current image scalling and {\it
max}/{\it min} values

\item [\htmlref{GETOFF}{GETOFF}]: get the RA,Dec spatial offsets from
the header of a series of raw {\sc ro} {\sc ircam3} images

\item [\htmlref{GETRAST}{GETRAST}]: check size of imaging window in
raster units

\item [\htmlref{GLITCHMARK}{GLITCHMARK}]: interactive removal of bad/hot pixels
from an image

\item [\htmlref{GRID}{GRID}]: plot an X,Y (RA,Dec) grid on an image

\item [\htmlref{HARDCOPY}{HARDCOPY}]: produce hardcopy of current
image/graphics

\item [\htmlref{HARDLOT}{HARDLOT}]: produces hardcopies of series of
observations/images

\item [\htmlref{IO2RO}{IO2RO}]: converts raw integrations/observation
files to reduced observation files for reduction by {\sc IrcamDR}

\item [\htmlref{JITREG}{JITREG}]: calculate accurate spatial offsets
between consecutive images in a sequence or mosaic, and write them to
an ASCII file for later use

\item [\htmlref{LAGAIN}{LAGAIN}]: replot last line graphics plot

\item [\htmlref{LIMAG}{LIMAG}]: calculate limiting magnitude of image
from sky noise

\item [\htmlref{LINE}{LINE}]: plot a straight line on an image

\item [\htmlref{LINE\_WIDTH}{LINE_WIDTH}]: select line width for line graphics

\item [\htmlref{LINECOL}{LINECOL}]: select line colour for line graphics

\item [\htmlref{MCURSOR}{MCURSOR}]: display multiple cursors for pixel
inquiries

\item [\htmlref{MEDLOT}{MEDLOT}]: subtract the median value from a
sub-area of an image from the whole image, for a series of images

\item [\htmlref{MOFFLOT}{MOFFLOT}]: calculate dc and spatial offsets
between mosaic tiles

\item [\htmlref{MORENSIGMA}{MORENSIGMA}]: replot the last {\bf
\htmlref{NSIGMA}{NSIGMA}} image display using a different sigma level

\item [\htmlref{MOREPLOT}{MOREPLOT}]: replot the last {\bf
\htmlref{PLOT}{PLOT}} image display using a different {\it max},{\it
min} scaling values

\item [\htmlref{MORERANPLOT}{MORERANPLOT}]: replot the last {\bf
\htmlref{RANPLOT}{RANPLOT}} image display using a different range value

\item [\htmlref{MOREVARGREY}{MOREVARGREY}]: replot the last {\bf
\htmlref{VARGREY}{VARGREY}} image display using a different pair of
variable index scaling parameters

\item [\htmlref{MOS2}{MOS2}]: mosaic two images together using an
interactive selection of common points

\item [\htmlref{NOMAG}{NOMAG}]: sets pixel scale to 0.286"/pixel for no
{\sc ircam3} external magnifier

\item [\htmlref{NSIGMA}{NSIGMA}]: plots an image with nsigma scaling

\item [\htmlref{OADD}{OADD}]: add two raw observation data images

\item [\htmlref{OCADD}{OCADD}]: add a constant to a raw observation data image

\item [\htmlref{OCDIV}{OCDIV}]: divide a raw observation data image by
a constant

\item [\htmlref{OCMULT}{OCMULT}]: multiply a raw observation data image
by a constant

\item [\htmlref{OCSTATS}{OCSTATS}]: produce sub-area (rectangular)
statistics for a raw observation data image

\item [\htmlref{OCSUB}{OCSUB}]: subtract a constant from a raw
observation data image

\item [\htmlref{ODIST}{ODIST}]: calculate distance between two pixels
in an image

\item [\htmlref{ODIV}{ODIV}]: divide one raw observation data image by another

\item [\htmlref{OHISTO}{OHISTO}]: produce sub-area statistics for a raw
observation data image

\item [\htmlref{OLOOK}{OLOOK}]: print to screen sub-area pixel values
from a raw observation data image

\item [\htmlref{OMULT}{OMULT}]: multiply two raw observation data
images together

\item [\htmlref{OSTATS}{OSTATS}]: produce sub-area statistics for a raw
observation data image

\item [\htmlref{OSUB}{OSUB}]: subtract one raw observation data image
from another

\item [\htmlref{PCLOSE}{PCLOSE}]: close current image/graphics device

\item [\htmlref{PENCOL}{PENCOL}]: change colour of one pen on current
device (8 colours)

\item [\htmlref{PENINT}{PENINT}]: change colour of one pen on current
device (RGB)

\item [\htmlref{PHO}{PHO}]: single aperture photometry on an object in
an image

\item [\htmlref{PHO2}{PHO2}]: double aperture photometry on an object
in an image

\item [\htmlref{PHO3}{PHO3}]: tripple aperture photometry on an object
in an image

\item [\htmlref{PICKLOT}{PICKLOT}]: extract to separate independent
images a sub-area of a series of images

\item [\htmlref{PLOT}{PLOT}]: plot an image with linear {\it max},{\it
min} scaling

\item [\htmlref{PLOTGLITCH}{PLOTGLITCH}]: interactive bad/hot pixel
removal on sequence of images

\item [\htmlref{PLOTLOT}{PLOTLOT}]: plot a series of images with linear
{\it max},{\it min} scaling

\item [\htmlref{POL2}{POL2}]: aperture polarimetry for dual-beam
polarization images

\item [\htmlref{POLREG}{POLREG}]: spatially register polarization
images at 4 waveplate positions

\item [\htmlref{POPEN}{POPEN}]: open plotting to a new
workstation/device

\item [\htmlref{RANPLOT}{RANPLOT}]: plot an image with user-defined
range scalling

\item [\htmlref{REMOS}{REMOS}]: re-mosaic a series of tile images

\item [\htmlref{RFLOT}{RFLOT}]: rotate and flip a series of images

\item [\htmlref{ROCENT}{ROCENT}]: calculate accurate offsets between
the same feature (star) in a series of images

\item [\htmlref{RODARKLOT}{RODARKLOT}]: dark subtract a series of raw
{\sc ircam3} format images

\item [\htmlref{ROINDEX}{ROINDEX}]: produces list of observational
parameters from  series of raw {\sc ircam3} format images

\item [\htmlref{ROMED}{ROMED}]: median filter a series of raw ({\sc
ro}) {\sc ircam3} images

\item [\htmlref{ROPAR}{ROPAR}]: extract basic observational parameters
from a single raw ({\sc ro}) {\sc ircam3} observation file

\item [\htmlref{ROTLOT}{ROTLOT}]: rotate a series of images by an
arbitrary amount

\item [\htmlref{ROUT}{ROUT}]: extract UT time information from a series
of raw ({\sc ro}) {\sc ircam3} (or old format {\sc ircam1/2}) data
images

\item [\htmlref{SCALEDARK}{SCALEDARK}]: scale dark exposures from {\sc
ircam1/2} old style container files.

\item [\htmlref{SCALELOT}{SCALELOT}]: scale a series of images

\item [\htmlref{SEE}{SEE}]: measure the FWHM of a stellar image (uses
\xref{KAPPA}{sun95}{} \xref{PSF}{sun95}{PSF} application)

\item [\htmlref{SETAREA}{SETAREA}]: select an area or sub-area of an image

\item [\htmlref{SETCEN}{SETCEN}]: define the position of the centre of
an image on the current graphics device

\item [\htmlref{SETCOMORI}{SETCOMORI}]: defines orientation of text
written with {\bf \htmlref{WRCOM}{WRCOM}} and {\bf \htmlref{WRCCOM}{WRCCOM}}

\item [\htmlref{SETCONT}{SETCONT}]: define parameters associated with a
contour line plot

\item [\htmlref{SETCONTIC}{SETCONTIC}]: define contour axis tick mark
length

\item [\htmlref{SETCUR}{SETCUR}]: define where cursor selection of an
image position refers to

\item [\htmlref{SETCURMARK}{SETCURMARK}]: define marking of cursor
selected points in image

\item [\htmlref{SETCUT}{SETCUT}]: defines the parameters associated
with a {\bf \htmlref{CUT}{CUT}} (or {\bf \htmlref{CCUT}{CCUT}}) through
an image

\item [\htmlref{SETCUTAXRAT}{SETCUTAXRAT}]: defines cut Y/X axis ratio

\item [\htmlref{SETFILE}{SETFILE}]: define type of {\sc ircam} data
files to be processed

\item [\htmlref{SETFONT}{SETFONT}]: defines font for text on images

\item [\htmlref{SETHARD}{SETHARD}]: define hardcopy device to use in
{\bf \htmlref{HARDCOPY}{HARDCOPY}} command

\item [\htmlref{SETMAG}{SETMAG}]: define image magnification for display

\item [\htmlref{SETMAX}{SETMAX}]: define maximum intensity in plot image

\item [\htmlref{SETMIN}{SETMIN}]: define minimum intensity in plot image

\item [\htmlref{SETMM}{SETMM}]: define both maximum and minimum
intensity in plot image

\item [\htmlref{SETNUM}{SETNUM}]: define axis annotation setup for
{\bf \htmlref{SURROUND}{SURROUND}} annotation

\item [\htmlref{SETNUMORI}{SETNUMORI}]: define number orientation in
{\bf \htmlref{SURROUND}{SURROUND}} axis annotation

\item [\htmlref{SETNUMSCA}{SETNUMSCA}]: define number size in
{\bf \htmlref{SURROUND}{SURROUND}} axis annotation

\item [\htmlref{SETPOLCOL}{SETPOLCOL}]: define polarization vector
colours for multi-colour vector plot

\item [\htmlref{SETPS}{SETPS}]: define the arcsecond/pixel scale of an image

\item [\htmlref{SETQUAD}{SETQUAD}]: define quadrant image display
option (4 images together)

\item [\htmlref{SETRADEC}{SETRADEC}]: defines RA and Dec for
{\bf \htmlref{SURROUND}{SURROUND}} image axis annotation

\item [\htmlref{SETSTD}{SETSTD}]: define the reduction/analysis setup
for the automatic processing and photometry command {\bf
\htmlref{STDRED}{STDRED}}

\item [\htmlref{SETVAR}{SETVAR}]: define numerous variables for image
analysis

\item [\htmlref{SETVARGREY}{SETVARGREY}]: define {\bf
\htmlref{VARGREY}{VARGREY}} image scaling parameters

\item [\htmlref{SETVEC}{SETVEC}]: define parameters needed for a
polarization vector map line plot

\item [\htmlref{SHIFT2}{SHIFT2}]: register two images to same spatial
position

\item [\htmlref{SHIFT3}{SHIFT3}]: register three images to same spatial
position

\item [\htmlref{SKYSUB4}{SKYSUB4}]: subtract sky from 4 images, usually
images at four waveplate positions in polarimetry mode

\item [\htmlref{SPLITLOT}{SPLITLOT}]: splits images from a series of images

\item [\htmlref{STDPROC}{STDPROC}]: automatic reduction of nights
single star photometry mosaic data sets

\item [\htmlref{STDRED}{STDRED}]: reduce single star photometry mosaic
image sets defined by {\bf \htmlref{SETSTD}{SETSTD}}

\item [\htmlref{STLOT}{STLOT}]: subtracts a constant and thresholds a
series of images

\item [\htmlref{STRED}{STRED}]: fully process/reduce a set of {\sc
ircam3} images forming a mosaic

\item [\htmlref{STREHL}{STREHL}]: calculate strehl coefficients for
stellar images

\item [\htmlref{SURROUND}{SURROUND}]: displays a border and tick
marks/numbers around an image

\item [\htmlref{TICKIO}{TICKIO}]: define tick mark position in
{\bf \htmlref{SURROUND}{SURROUND}} axis annotation

\item [\htmlref{TICKLEN}{TICKLEN}]: define tick mark length in
{\bf \htmlref{SURROUND}{SURROUND}} image annotation

\item [\htmlref{TOMAG}{TOMAG}]: convert images in intensity (DN/second)
to magnitudes per pixel and per square arcsecond

\item [\htmlref{TSLOT}{TSLOT}]: thresholds and scales a series of images

\item [\htmlref{VANS}{VANS}]: plot a vector array image from
shift-and-add imaging

\item [\htmlref{VARGREY}{VARGREY}]: display an image using vargrey
scaling

\item [\htmlref{VEC}{VEC}]: plot a vector map of polarization data

\item [\htmlref{VEC\_TITLE}{VEC_TITLE}]: define title for vector plot

\item [\htmlref{WRAPLOT}{WRAPLOT}]: correct a series of images for
16-bit wrap-around

\item [\htmlref{WRCCOM}{WRCCOM}]: write a comment (text) string on the
current graphics device (cursor origin)

\item [\htmlref{WRCOM}{WRCOM}]: write a comment (text) string on the
current graphics device (pixel origin)

\item [\htmlref{WRITELUT}{WRITELUT}]: write a colour table (LUT) to the
current graphics device

\item [\htmlref{X2MAG}{X2MAG}]: defines pixel scale when using
$\times$2 external magnifier (0.143"/pixel)

\item [\htmlref{X5MAG}{X5MAG}]: defines pixel scale when using
$\times$5 external magnifier (0.0571"/pix)

\item [\htmlref{ZAPLOT}{ZAPLOT}]: remove bad columns/rows from a series
of images

\end{description}
\end{description}

% LIST2

\newpage
\subsection{\label{ss:atask_applications}\xlabel{atask_applications}Fortran a-task monolith commands -- alphabetical}

\begin{description}
\begin{description}

\item [\htmlref{ABCOM}{ABCOM}]: combine two images, the first with odd
channel data, the second with even channel data.

\item [\htmlref{ABSEP}{ABSEP}]: sepatate an image into two, the first
with odd channel data, the second with even channel data.

\item [\htmlref{ADD}{ADD}]: add two images, forming a third with the result

\item [\htmlref{AMCORR}{AMCORR}]: correct image intensity for the
effect of atmospheric extinction (airmass effect).

\item [\htmlref{ANNSTATS}{ANNSTATS}]: produce annular statistics from
user defined centre point on image

\item [\htmlref{APERADD}{APERADD}]: perform aperture statsistics on an image

\item [\htmlref{APERPHOT}{APERPHOT}]: aperture photometry

\item [\htmlref{APERPOL}{APERPOL}]: aperture polarimetry

\item [\htmlref{APPLYMASK}{APPLYMASK}]: apply a bad-pixel mask to an image

\item [\htmlref{ASCIILIST}{ASCIILIST}]: create plain-text file of pixel
values of an image

\item [\htmlref{AUTOMOS}{AUTOMOS}]: automatically correct dc-level
offsets for a set of images in a mosaic

\item [\htmlref{BINUP}{BINUP}]: bin an input image by a user-specified
factor in X and Y

\item [\htmlref{BLOCK}{BLOCK}]: block applies a block smooth to an image

\item [\htmlref{CADD}{CADD}]: Add a user-defined constant to every
pixel in an image

\item [\htmlref{CALCOL}{CALCOL}]: calculate a colour image from two
magnitude images

\item [\htmlref{CDIV}{CDIV}]: divides all pixels in an input image by a
constant value

\item [\htmlref{CENTROID}{CENTROID}]: calculate an accurate centroid
position of a stellar source or stellar-like feature in an image

\item [\htmlref{CHPIX}{CHPIX}]: change the value of a user-defined
pixel or pixels within an image

\item [\htmlref{CMULT}{CMULT}]: Multiplies every pixel in an image by a
constant value

\item [\htmlref{COLCYCLE}{COLCYCLE}]: create a cyclic colour table
(LUT)

\item [\htmlref{COLMED}{COLMED}]: median filters down the columns of an image

\item [\htmlref{COMPADD}{COMPADD}]: increase S/N in an image by
reducing spatial resolution

\item [\htmlref{COMPAVE}{COMPAVE}]: increase S/N in an image by
reducing spatial resolution

\item [\htmlref{COMPICK}{COMPICK}]: decrease size of image but retain
original S/N

\item [\htmlref{COMPRESS}{COMPRESS}]: decrease size of image and
improve S/N by degrading spatial resolution

\item [\htmlref{CRECOLT}{CRECOLT}]: create colour table (LUT)

\item [\htmlref{CREFRAME}{CREFRAME}]: create test frame

\item [\htmlref{CREQUILT}{CREQUILT}]: create a plain-text file for
mosaic assembly

\item [\htmlref{CSFIT}{CSFIT}]: fit a centro-symmetric polarization
pattern to a polarization map image pair

\item [\htmlref{CSGEN}{CSGEN}]: generate a centro-symmetric
polarization position angle image with a user-defined centre (pixel)

\item [\htmlref{CSUB}{CSUB}]: subtracts a constant from each pixel in an image

\item [\htmlref{DEFGRAD}{DEFGRAD}]: remove background gradients along
columns or rows in an image

\item [\htmlref{DEVFCS}{DEVFCS}]: calculate deviation from
centro-symmetry of pixels in a polarization map

\item [\htmlref{DIST}{DIST}]: calculate offset and position (RA,Dec) of
second point in image given position (RA,Dec) of first point

\item [\htmlref{DIV}{DIV}]: Divides an image by another image pixel by
pixel

\item [\htmlref{EXP10}{EXP10}]: take the base 10 exponential of each
pixel of an image

\item [\htmlref{EXPE}{EXPE}]: take the base E exponential of each pixel
of an image

\item [\htmlref{EXPON}{EXPON}]: take the user specified base
exponential of each pixel of an image

\item [\htmlref{FCOADD}{FCOADD}]: coadd (sum and average) a sequence of
NDF images

\item [\htmlref{FINDPEAK}{FINDPEAK}]: find the peak signal/flux in a
user-defined sub-area if an image

\item [\htmlref{FLIP}{FLIP}]: flip an image either horizontally (H) or
vertically (V)

\item [\htmlref{GAUSS}{GAUSS}]: smooth an image using a gaussian filter

\item [\htmlref{GAUSSTH}{GAUSSTH}]: smooth an image using a gaussian
filter, user-defined upper threshold

\item [\htmlref{GLITCH}{GLITCH}]: removes bad/hot pixels or change the
value of any other pixel

\item [\htmlref{HISTEQ}{HISTEQ}]: histogram equalization of an image

\item [\htmlref{HISTGEN}{HISTGEN}]: histogram of an image with user
defined binning

\item [\htmlref{HISTO}{HISTO}]: statistics on any image or sub-area of
any image

\item [\htmlref{HOTSHOT}{HOTSHOT}]: automatic search and removal of
bad/hot pixels using a sigma cut method

\item [\htmlref{IMCOMB}{IMCOMB}]: combine two images to replace bad
area of one with same (good) area of second.

\item [\htmlref{INDEX}{INDEX}]: produces an index listing of old {\sc
ircam} container file observations

\item [\htmlref{INSETB}{INSETB}]: set a region (box) of an image to a
magic number

\item [\htmlref{INSETC}{INSETC}]: set a region (circle) of an image to
a magic number

\item [\htmlref{INTLK}{INTLK}]: prints an integer representation of any
sub-area of any image to the terminal

% \item [\htmlref{JSDCLEAN}{JSDCLEAN}]: removal of PSF from images for
% enhanced spatial resolution (*** Under Devlopment *** )

\item [\htmlref{LAPLACE}{LAPLACE}]: apply a laplacian convolution to an
image, as an edge detector

\item [\htmlref{LINCONT}{LINCONT}]: linearize the images in an old {\sc
ircam1/2} format container file

\item [\htmlref{LINIMAG\_NDR}{LINIMAG_NDR}]: linearize individual
images from an old {\sc ircam1/2} style container file

\item [\htmlref{LOG10}{LOG10}]: Base 10 log of pixels in an image

\item [\htmlref{LOGAR}{LOGAR}]: User-defined base log of pixels in an image

\item [\htmlref{LOGE}{LOGE}]: Base e log of pixels in an image

\item [\htmlref{LOOK}{LOOK}]: write the pixel values for a user-defined
sub-area of an image to the terminal

\item [\htmlref{MAKEBAD}{MAKEBAD}]: create a glitch style bad pixel
list using an n-sigma cut on an image

\item [\htmlref{MAKEGLITCH}{MAKEGLITCH}]: convert a bad pixel mask to
bad pixel list file

\item [\htmlref{MAKEMASK}{MAKEMASK}]: create a bad pixel mask

\item [\htmlref{MANIC}{MANIC}]: re-size and/or re-dimension an image

\item [\htmlref{MANYCOL}{MANYCOL}]: combine colour tables (LUTs) in
user-defined sequence

\item [\htmlref{MED3D}{MED3D}]: median filter through a stack of images

\item [\htmlref{MEDIAN}{MEDIAN}]: median filters (spatially) an image

\item [\htmlref{MOFF}{MOFF}]: determine the best X,Y spatial and dc sky
offsets for a pair of mosaic images

\item [\htmlref{MOSAIC}{MOSAIC}]: mosaic (combine) up to 50 images
(tiles) in one image

\item [\htmlref{MOSAIC2}{MOSAIC2}]: mosaic (combine) a pair of images
(tiles) in one image

\item [\htmlref{MOSCOR}{MOSCOR}]: determine best dc level correction
between two overlapping images

\item [\htmlref{MULT}{MULT}]: multiply two images together

\item [\htmlref{NUMB}{NUMB}]: count the number of pixels in an image
with values greater than an input value

\item [\htmlref{OBSEXT}{OBSEXT}]: extract individual observation images
from an old format {\sc ircam1/2} container file

\item [\htmlref{OBSLIST}{OBSLIST}]: list one user-defined item in the
header of all the observations in an old format {\sc ircam1/2}
container file

\item [\htmlref{OEFIX}{OEFIX}]: correct odd/even channel problem in
{\sc ircam1/2} data

\item [\htmlref{OUTSETB}{OUTSETB}]: set the pixel values outside a
user-defined sub-area of an image to a user-defined value

\item [\htmlref{OUTSETC}{OUTSETC}]: set the pixel values outside a
user-defined circle in an image to a user-defined value

\item [\htmlref{PICKIM}{PICKIM}]: extract sub-area of image from larger image

\item [\htmlref{PIXDUPE}{PIXDUPE}]: expand the X,Y dimensions of an
image by pixel duplication

\item [\htmlref{POLCAL}{POLCAL}]: calculate polarization images

\item [\htmlref{POLLY}{POLLY}]: calculate polarization parameters from
four input intensity (numeric) values

\item [\htmlref{POLLY2}{POLLY2}]: same as {\bf \htmlref{POLLY}{POLLY}}
but for intensity values taken from images taken at the four waveplate
positions using {\sc irpol2} and Wollaston prism

\item [\htmlref{POLSEP}{POLSEP}]: separate the four regions of a image
taken using the {\sc irpol2} polarimeter with the Wollaston prism/focal plane
mask at one of the four required waveplate positions

\item [\htmlref{POLSHOT}{POLSHOT}]: calculate shot-noise polarization
on a waveplate+polarizer observation

\item [\htmlref{POLTHRESH}{POLTHRESH}]: threshold a set of
{\sc irpol} plus polarizer polarization data

\item [\htmlref{POW}{POW}]: raises each pixel in an image to a
user-defined power {\it x}

\item [\htmlref{QUILT}{QUILT}]: assemble mosaic tiles into a final mosaic image

\item [\htmlref{RADIM}{RADIM}]: expand a radial cut on a linear axis

\item [\htmlref{ROOT}{ROOT}]: square root an image

\item [\htmlref{ROTATE}{ROTATE}]: rotate an image

\item [\htmlref{ROWMED}{ROWMED}]: median filter along the rows of an image

\item [\htmlref{SETVAL}{SETVAL}]: set all pixels with a value to another value

\item [\htmlref{SHADOW}{SHADOW}]: enhance image details with shadow effect

\item [\htmlref{SHIFT}{SHIFT}]: X and/or Y shift an image

\item [\htmlref{SHSIZE}{SHSIZE}]: query image size (dimensions)

\item [\htmlref{SQORST}{SQORST}]: change the size (dimensions) of an
image by spline interpolation

\item [\htmlref{STATS}{STATS}]: statisitcal information on an image

\item [\htmlref{STCOADD}{STCOADD}]: coadd observations from an {\sc
ircam1/2} container file

\item [\htmlref{STEPIM}{STEPIM}]: create an intensity step (contour) image

\item [\htmlref{SUB}{SUB}]: subtract one image from another

\item [\htmlref{THETAFIX}{THETAFIX}]: position angle range correction

\item [\htmlref{THRESH}{THRESH}]: threshold an image

\item [\htmlref{THRESH0}{THRESH0}]: set pixels outside upper and lower
thresholds to zero

\item [\htmlref{TRANDAT}{TRANDAT}]: create NDF image from ASCII image

\item [\htmlref{TRIG}{TRIG}]: perform (numerous) trig functions pixel
by pixel on an image

\item [\htmlref{WMOSAIC}{WMOSAIC}]: mosaic weighted images

\item [\htmlref{WQUILT}{WQUILT}]: assemble weighted mosaic tiles into a
final mosaic image

\item [\htmlref{WRAPCOR}{WRAPCOR}]: correct for wrap-around in image readout

\item [\htmlref{XGROW}{XGROW}]: grows a Y slice into an image for subtraction

\item [\htmlref{YADD}{YADD}]: add up the rows in an image to create an
image slice

\item [\htmlref{YGROW}{YGROW}]: grows an X slice into an image for subtraction

\item [\htmlref{ZAPLIN}{ZAPLIN}]: remove bad columns and rows from an image

\end{description}
\end{description}

% LIST3

\newpage
\subsection{\label{ss:commands_by_usage}\xlabel{commands_by_usage}ICL procedures and Fortran a-task commands -- by usage}

\begin{description}
\begin{description}

\item[Image Display :] again chdisp clear clearit cnsigma cplot cranplot
cvargrey disp getcm morensigma moreplot moreranplot morevargrey nsigma
plot plotlot ranplot setarea setcen setcur setmag setmax setmin setmm
setquad setradec setvargrey surround vans vargrey

\item[Line Graphics :] anncol border box cbox ccircle ccross ccut cellipse
circle cline cont\_title contoff contour cross crosscut crosscut\_peak cut
cut2ff cut\_title ellipse grid histgen lagain line line\_width linecol
setcont setcontic setcurmark setcut setcutaxrat setnum setnumori
setnumsca tickio ticklen

\item[Colour Table Display :] ablock cablock colcycle colinv coltab crecolt
manycol pencol penint writelut

\item[Data Reduction :] abcom absep amcorr amcorrlot applymask applymasklot
automos calcol chred colmed darklot defgrad deglot fcoadd flat2 flatlot
flip histeq med3d medlot picklot rflot rodarklot romed rotlot scaledark
scalelot splitlot stlot stred thresh thresh0 tslot wraplot

\item[Image Registration :] shift, shift2, shift3

\item[Single Star Photometry Automatic Reduction/Analysis :] setstd, stdproc,

\item[Hardcopy Commands :] hardcopy hardlot sethard

\item[Statistics :] annstats cstats findpeak histo limag numb ocstats ohisto
ostats shsize stats strehl

\item[Mosaicing :] accoff coff crequilt getoff jitreg moff mofflot mos2 mosaic
mosaic2 moscor quilt remos wmosaic wquilt

\item[Position Determination :] centroid cent1 cent2 cursor dist mcursor odist
olook rocent

\item[Bad/hot Pixel Removal :] curhot glitch glitchmark hotshot insetb insetc
outsetb outsetc make\-bad makeglitch makemask plotglitch setval zaplin
zaplot

\item[Daophot Commands :] daocen daofind daogid daogid2

\item[Image Extraction :] dispick pickim

\item[Image Arithmetic :] add cadd cdiv cmult csub div exp10 expe expon log10
logar loge mult oadd ocadd ocdiv ocmult ocsub odiv omult osub pow root
sub trig

\item[Image Dimension Change :] binup compadd compave compick compress
manic \\ pixdupe rotate sqorst xgrow ygrow yadd

\item[Image Smoothing :] block gauss gaussth median
\item[Photometry Commands :] aperadd aperphot pho pho2 pho3 stdred

\item[Polarimetry Commands :] aperpol csfit csgen devfcs pol2 polcal polly
polly2 polsep polreg polshot polthresh setpolcol setvec skysub4 thetafix
vec vec\_title

\item[Text Commands :] setcomori setfont wrccom wrcom

\item[Miscellaneous :] array\_tests asciilist chpix calexp calmag
calzer creframe getrast imcomb index io2ro
%  jsdclean
laplace lincont
linimag\_ndr look nomag obsext obslist oefix pclose popen radim roindex
ropar rout rowmed see setfile setps setvar shadow stcoadd stepim tomag
trace trandat wrapcor x2mag x5mag

\end{description}
\end{description}

%  DESCRIPTIONS

\newpage
\section{\label{se:command_descriptions}\xlabel{command_descriptions}Command Descriptions}
%
\subsection{\label{ss:icl_procedure_descriptions}\xlabel{icl_procedure_descriptions}ICL procedure descriptions}

\hrule
\subsubsection*{\label{ABLOCK}\xlabel{ABLOCK}ABLOCK}

\begin{description}

\item[Description :] Plots a colour table (LUT) colour block on active
workstation showing the translation of colour to intensity via annotation
on colour block.  Plots colour block at user-specified coordinates in
workstation raster units.  Block can be horizontal or vertical, have
annotation or have no annotation and be of a user-defined size.

\item[Usage :] To plot colour block and annotation on image.
\item[Associated commands :] -
\item[Short version of command :] -
\item[Invocation :]

\begin{quote}{\tt ablock} \\
or \\
{\tt ablock 128 375 4 }
\end{quote}

\begin{itemize}

\item {\tt 128 375 }\/are the X,Y device (raster) coordinates position
for colour block

\item {\tt 4 }\/is the size scaling factor for the colour block

\end{itemize}

\end{description}

\hrule
\subsubsection*{\label{ACCOFF}\xlabel{ACCOFF}ACCOFF}

\begin{description}

\item[Description :] Plots sequentially pairs of a series of image from
a raster mosaic and calculates accurate spatial offsets using cursor
input and peak pixel in search box.  Image names are defined by prefix,
range of numbers and suffix.  The user clicks on the same feature
(star) in the two images displayed and accurate spatial offsets in
arcsec are calculated.\ {\bf accoff} writes offset file with list of
accurate RA,Dec spatial offsets in arcseconds wrt first image in
sequence. At each of the positions in the images selected using the
cursor, the peak pixel is found within a small rectangular region and
that position is used in the two images for the offset calculation.
Offset file produced can be fed into command remos to re-mosaic a set
of reduced images using a new/improved spatial offset file.

\item[Usage :] {\bf accoff} is useful for producing accurate spatial offset for
re-mosaicing large scale mosaics reduced using command stred.

\item[Notes :] Use {\bf accoff} to get spatial offsets of a sequence of images
in a mosaic pattern (\emph{i.e.}, with large offsets to survey a large
area of sky).  Use {\bf coff} for a sequence of images taken at supposedly the
same spatial position. Use {\bf jitreg} for a sequence of images taken with a
small jitter offset between them.

\item[Associated commands :] {\tt \htmlref{setps}{SETPS}},
{\tt \htmlref{histo}{HISTO}}, {\tt \htmlref{coff}{COFF}},
{\tt \htmlref{jitreg}{JITREG}}, {\tt \htmlref{stred}{STRED}},
{\tt \htmlref{remos}{REMOS}}
\item[Short version of command :] -
\item[Invocation :]

\begin{quote}{\tt accoff }\end{quote}

\end{description}

\hrule
\subsubsection*{\label{AGAIN}\xlabel{AGAIN}AGAIN}

\begin{description}

\item[Description :] Re-displays the last image in the same manner as
previously requested.

\item[Usage :] Re-plot the last image displayed without changing anything.

\item[Associated commands :] {\tt \htmlref{nsigma}{NSIGMA}},
{\tt \htmlref{plot}{PLOT}}, {\tt \htmlref{ranplot}{RANPLOT}},
{\tt \htmlref{vargrey}{VARGREY}}, {\tt \htmlref{cnsigma}{CNSIGMA}},
{\tt \htmlref{cplot}{CPLOT}}, {\tt \htmlref{cranplot}{CRANPLOT}}, \\
{\tt \htmlref{cvargrey}{CVARGREY}}, {\tt \htmlref{moren}{MORENSIGMA}},
{\tt \htmlref{morep}{MOREP}}, {\tt \htmlref{morer}{MORER}},
{\tt \htmlref{morev}{MOREV}}, {\tt \htmlref{clear}{CLEAR}},
{\tt \htmlref{setmag}{SETMAG}}, {\tt \htmlref{setcen}{SETCEN}}

\item[Short version of command :] {\tt ag}
\item[Invocation :]

\begin{quote}{\tt again }\end{quote}

\end{description}

\hrule
\subsubsection*{\label{AMCORRLOT}\xlabel{AMCORRLOT}AMCORRLOT}

\begin{description}

\item[Description :] Corrects a series of images whose names are defined
by a prefix, a range of numbers and a suffix, to represent the flux at
unit airmass.  See above for the definition of the operation of the
``{\bf lot}'' family of images. Images must by in DN/sec.  Correction uses
default extinction coefficients in ASCII file {\tt \$LIRCAMDIR/extinct.list}.

\item[Usage :] When you have a lot of images (say forming a mosaic) to
airmass correct.

\item[Associated commands :] {\tt \htmlref{amcorr}{AMCORR}}
\item[Short version of command :] -
\item[Invocation :]

\begin{quote}{\tt  amcorrlot }\end{quote}

\end{description}

\hrule
\subsubsection*{\label{ANNCOL}\xlabel{ANNCOL}ANNCOL}

\begin{description}

\item[Description :] Defines the colour of the annotation created using
command surround.  Defines colour to be used, use command surround to plot
annotation itself in new colour.

\item[Usage :] Change colour of surround annotation for pretty pictures.
\item[Associated commands :] -
\item[Short version of command :] -
\item[Invocation :]

\begin{quote}{\tt anncol }\\
or \\
{\tt anncol g }
\end{quote}

\begin{itemize}

\item {\tt g }\/is for green annotation

\end{itemize}

\end{description}

\hrule
\subsubsection*{\label{APPLYMASKLOT}\xlabel{APPLYMASKLOT}APPLYMASKLOT}

\begin{description}

\item[Description :] Applies a bad pixel mask to a series of images whose
names are defined by a prefix, a range of numbers and a suffix.  See above
for the definition of the operation of the ``{\bf lot}'' family of images. Bad
pixels in the mask are set to 1, good pixels set to 0.  A bad pixel in the
output images is set to the magic number {\tt -1.0e-20}.

\item[Usage :] To mask out bad pixels in a series of images.
\item[Associated commands:] {\tt \htmlref{applymask}{APPLYMASK}},
{\tt \htmlref{makemask}{MAKEMASK}}, \ldots
\item[Short version of command :] -
\item[Invocation :]

\begin{quote}{\tt  applymasklot }\end{quote}

\end{description}

\hrule
\subsubsection*{\label{ARRAY_TESTS}\xlabel{ARRAY_TESTS}ARRAY\_TESTS}

\begin{description}

\item[Description :] Calculates the STARE and ND\_STARE readout noises and
the dark current from a series of engineering data taken with the EXEC
named ARRAY\_TESTS\@.  Results are compared to the nominally expected
values and the user is notified if the values obtained are within limits.
The results are also logged to an engineering file for archival use.

\item[Usage :] To analyse data taken with ARRAY\_TESTS exec and
determine current performance characteristics of {\sc ircam3} array.

\item[Associated commands :] -
\item[Short version of command :] -
\item[Invocation :]

\begin{quote}{\tt  array\_tests }\\
or \\
{\tt array\_tests 9 }
\end{quote}

\begin{itemize}

\item {\tt 9 } is start observation number of array test image sequence

\end{itemize}

\end{description}

\hrule
\subsubsection*{\label{BORDER}\xlabel{BORDER}BORDER}

\begin{description}

\item[Description :] Plots a line around an image (a border) of
user-defined width.  The line width is in pixels.

\item[Usage :] To plot a line around your image.
\item[Associated commands :] {\tt \htmlref{surround}{SURROUND}}
\item[Short version of command :] {\tt bo}
\item[Invocation :]

\begin{quote}{\tt  border }\\
or \\
{\tt border 2 }
\end{quote}

\begin{itemize}

\item {\tt 2 } is the pixel width of border

\end{itemize}

\end{description}

\hrule
\subsubsection*{\label{BOX}\xlabel{BOX}BOX}

\begin{description}

\item[Description :] Plots a box at a user-defined position in an image.
The box is plotted wither centred or with its bottom-left corner on a
user-selected pixel (selected by its X,Y pixel coordinates) in the image
and is of a user-defined size (in arcseconds).  To set arcsecond/pixel
scale in image use command setps.  The X,Y pixel coordinates are 1,1 for
the bottom left corner (origin) of the image and 128,128 for the centre of
the {\sc ircam3} 256$\times$256 array.

\item[Usage :] To, say, highlight a source or image feature.
\item[Associated commands :] {\tt \htmlref{cbox}{CBOX}},
{\tt \htmlref{line\_width}{LINE_WIDTH}}, {\tt \htmlref{setps}{SETPS}}
\item[Short version of command :] -
\item[Invocation :]

\begin{quote}{\tt  box }\\
or \\
{\tt box 1 128 130 15 10 }
\end{quote}

\begin{itemize}

\item {\tt 1} means put centre of box on given pixel
(2 would mean put bottom-left corner of box on given pixel)
\item {\tt 128 130 } are the pixel position of centre/bottom-left corner
\item {\tt 15 10 } are the X,Y size of box in arcseconds
\end{itemize}
\end{description}

\hrule
\subsubsection*{\label{CABLOCK}\xlabel{CABLOCK}CABLOCK}

\begin{description}

\item[Description :] Plots a colour table (LUT) colour block on active
workstation showing the translation of colour to intensity via annotation
on colour block.  Plots colour block at user-specified position using
cursor. Block can be horizontal or vertical, have annotation or have no
annotation and be of a user-defined size.

\item[Usage :] To put your colour index block somewhere special on your plot.

\item[Associated commands :] -
\item[Short version of command :] -
\item[Invocation :]

\begin{quote}{\tt  cablock }\\
or \\
{\tt  cablock 4 }
\end{quote}

\begin{itemize}

\item {\tt 4 } is the size scaling factor of colour block

\end{itemize}
\end{description}

\hrule
\subsubsection*{\label{CALEXP}\xlabel{CALEXP}CALEXP}

\begin{description}

\item[Description :] Calculates the optimum exposure time to give 80\%
full-well on a user-defined source in a raw ({\sc ro}) image.  User is asked
for the observation number to be plotted, {\bf calexp} plots it and
then asks the user to select a source in the image with the cursor.
The peak pixel in a small box around the cursor position is found and
from the header of the raw observation file the exposure time is
extracted.\ {\bf calexp} then tells the user the optimum exposure time
for 80\% full-well on that object.

\item[Usage :] To give you an idea of the optimum on-chip exposure
time from a short test image.

\item[Associated commands :] -
\item[Short version of command :] -
\item[Invocation :]

\begin{quote}{\tt  calexp } \\
or \\
{\tt calexp 42 5 }
\end{quote}

\begin{itemize}

\item {\tt 42 } is the observation number for exposure calculation
\item {\tt 5 } is the sigma level to be used for the {\bf nsigma} display of
image No.\ 42

\end{itemize}

\end{description}

\hrule
\subsubsection*{\label{CALMAG}\xlabel{CALMAG}CALMAG}

\begin{description}

\item[Description :] Calculates a magnitude from a user-defined flux
value in DN/sec using the zeropoint defined for the filter selected by
the command {\bf setvar}. The default zeropoints are all set to zero
and hence by default, {\bf calmag} gives instrumental magnitudes.  The
result is given in a straight magnitude, a magnitude/pixel, a
magnitude/sq arcsec (using the pixel scale defined by the command {\bf
setps}) and a magnitude in a 2 and 4 arcsec aperture.

\item[Usage :] To calculate object magnitude from counts (DN/sec).
\item[Associated commands :] -
\item[Short version of command :] -
\item[Invocation :]

\begin{quote}{\tt  calmag }\\
or \\
{\tt calmag J 10234 }
\end{quote}

\begin{itemize}

\item {\tt J } is the filter used for the observation
\item {\tt 10234 } is the intensity to be converted to a magnitude

\end{itemize}

\end{description}

\hrule
\subsubsection*{\label{CALZER}\xlabel{CALZER}CALZER}

\begin{description}

\item[Description :] Calculates a zeropoint from a user-defined actual
object magnitude (the true brightness of a star in the filter used) and
a flux in DN/sec.  The calculation performed is zp = act.mag. +
2.5log10(flux)

\item[Usage :] To determine a zeropoint from stars magnitude and
observed counts (DN/sec).
\item[Associated commands :] -
\item[Short version of command :] -
\item[Invocation :]

\begin{quote}{\tt  calzer }\\
or \\
{\tt calzer 10.6 10234 }
\end{quote}

\begin{itemize}

\item {\tt 10.6 } is the actual magnitude of object in filter used
\item {\tt 10234 } is the intensity (in DN/sec) for object in filter used

\end{itemize}
\end{description}

\hrule
\subsubsection*{\label{CBOX}\xlabel{CBOX}CBOX}

\begin{description}

\item[Description :] Plots a box at a user-defined position in an image.  The
box is plotted either centred or with its bottom-left corner on a user-selected
pixel (selected with the cursor) in the image and is of a user-defined
size (in arcseconds).  To set the arcsecond/pixel scale in the image
use the command {\bf setps}.

\item[Usage :] Put a box on your image.
\item[Associated commands :] {\tt \htmlref{box}{BOX}},
{\tt \htmlref{line\_width}{LINE_WIDTH}}, {\tt \htmlref{setps}{SETPS}}
\item[Short version of command :] -
\item[Invocation :]

\begin{quote}{\tt  cbox }\\
or \\
{\tt cbox 1 10 12 }
\end{quote}

\begin{itemize}

\item {\tt 1} means plot box with centre at cursor input position
(2 would mean plot bottom-left corner at cursor position)
\item {\tt 10 12 } are the X,Y size of the box plotted in arcseconds

\end{itemize}

\end{description}

\hrule
\subsubsection*{\label{CCIRCLE}\xlabel{CCIRCLE}CCIRCLE}

\begin{description}

\item[Description :] Plots a circle at a user-defined position in an image.
The circle is plotted on a user-selected pixel (selected with the cursor) in
the image and is of a user-defined diameter (in arcseconds).  To set the
arcsecond/pixel scale in the image use the command {\bf setps}.

\item[Usage :] Put a circle on your image.
\item[Associated commands :] {\tt \htmlref{circle}{CIRCLE}},
{\tt \htmlref{line\_width}{LINE_WIDTH}}, {\tt \htmlref{setps}{SETPS}}
\item[Short version of command :] -
\item[Invocation :]

\begin{quote}{\tt  ccircle }\\
or \\
{\tt ccircle 25 }
\end{quote}

\begin{itemize}

\item {\tt 25 } is the diameter of the circle plotted in arcseconds

\end{itemize}

\end{description}

\hrule
\subsubsection*{\label{CCROSS}\xlabel{CCROSS}CCROSS}

\begin{description}

\item[Description :] Plots a cross at a user-defined position in an image.
The cross is plotted on a user-selected pixel (selected with the
cursor) in the image and is of a user-defined size (in arcseconds).  To
set the arcsecond/pixel scale in the image use the command {\bf setps}.

\item[Usage :] Put a cross on your image.
\item[Associated commands :] {\tt \htmlref{cross}{CROSS}},
{\tt \htmlref{line\_width}{LINE_WIDTH}}, {\tt \htmlref{setps}{SETPS}}
\item[Short version of command :] -
\item[Invocation :]

\begin{quote}{\tt  ccross }\\
or \\
{\tt  ccross 16 }
\end{quote}

\begin{itemize}

\item {\tt 16 } is the size of the cross plotted in arcseconds

\end{itemize}

\end{description}

\hrule
\subsubsection*{\label{CCUT}\xlabel{CCUT}CCUT}

\begin{description}

\item[Description :] Plots a line cut (intensity vs.\ position) between two
user-selected points in an image.  The points are selected using the
cursor.  The user has several options to do with how the cut is
displayed, these are set by the command setcut which has to be executed
at least once before the {\bf ccut} can be performed.

\item[Usage :] View a cross-cut through the image currently displayed.
\item[Associated commands :] {\tt \htmlref{setcut}{SETCUT}},
{\tt \htmlref{cut}{CUT}}, {\tt \htmlref{setps}{SETPS}},
{\tt \htmlref{line\_width}{LINE_WIDTH}}, {\tt \htmlref{setnum}{SETNUM}}
\item[Short version of command :] -
\item[Invocation :]

\begin{quote}{\tt  ccut }\\
or \\
{\tt  ccut 10 }
\end{quote}

\begin{itemize}

\item {\tt 10 } is the observation number of the image to have a cut
extracted from and plotted.

\emph{NB} {\tt ccut 0}  would prompt the user for the full name of the image
(instead of observation number) to have cut extracted from and plotted.

\end{itemize}

\end{description}

\hrule
\subsubsection*{\label{CELLIPSE}\xlabel{CELLIPSE}CELLIPSE}

\begin{description}

\item[Description :] Plots a ellipse at a user-defined position in an
image.  The ellipse is plotted on a user-selected pixel (selected with
the cursor) in the image and is of a user-defined major axis (in
arcseconds) and eccentricity.  To set the arcsecond/pixel scale in the
image use the command {\bf setps}.

\item[Usage :] Plot an ellipse (IRAS error ellipse for example).
\item[Associated commands :] {\tt \htmlref{ellipse}{ELLIPSE}},
{\tt \htmlref{line\_width}{LINE_WIDTH}}, {\tt \htmlref{setps}{SETPS}}
\item[Short version of command :] -
\item[Invocation :]

\begin{quote}{\tt  cellipse }\\
or \\
{\tt cellipse 42 0.89 30.2 }
\end{quote}

\begin{itemize}

\item {\tt 42} is the size of major axis of ellipse in arcseconds
\item {\tt 0.89} is the eccentricity of the ellipse
\item {\tt 30.2} is the position angle of major axis of ellipse

\end{itemize}

\end{description}

\hrule
\subsubsection*{\label{CENT1}\xlabel{CENT1}CENT1}

\begin{description}

\item[Description :] Allows you to define within the current image
displayed, one  position \emph{i.e.}, star.  The centroid of the
stellar position is calculated and returned to the user.  You can
select multiple centroiding positions in {\bf cent1}.

\item[Usage :] Determine the accurate position of sources in your image.
\item[Associated commands :] {\tt \htmlref{cent2}{CENT2}},
{\tt \htmlref{cursor}{CURSOR}}, {\tt \htmlref{centroid}{CENTROID}}
\item[Short version of command :] -
\item[Invocation :]

\begin{quote}{\tt  cent1 }\\
or \\
{\tt cent1 junkimage 9 }
\end{quote}

\begin{itemize}

\item {\tt junkimage } is the name of the image to be analysed
\item {\tt 9 } is size of the box (in pixels) for centroid calculation

\end{itemize}

\end{description}

\hrule
\subsubsection*{\label{CENT2}\xlabel{CENT2}CENT2}

\begin{description}

\item[Description :] Allows you to define within the current image
displayed, two positions \emph{i.e.}, stars. The centroid of the
stellar positions are calculated and the RA,Dec offsets between them
and together with the radial distance,position angle (east of north) is
returned.

\item[Usage :] Determine the accurate offsets between two sources in your image.
\item[Associated commands :] {\tt \htmlref{cent1}{CENT1}},
{\tt \htmlref{cursor}{CURSOR}}, {\tt \htmlref{centroid}{CENTROID}}

\item[Short version of command :] -
\item[Invocation :]

\begin{quote}{\tt  cent2 }\\ or \\ {\tt cent2 junkimage 9 } \end{quote}

\begin{itemize}

\item {\tt junkimage } is the name of the image to be analysed \item
{\tt 9 } is size of the box (in pixels) for centroid calculation

\end{itemize}

\end{description}

\hrule
\subsubsection*{\label{CHDISP}\xlabel{CHDISP}CHDISP}

\begin{description}

\item[Description :] Takes as input a chop observation number and
subtracts the two phases (object + sky or object + object if chopping
onto the array) and displays the result on the workstation.  {\bf
chdisp} then allows you to do photometry on the difference image using
zeropoints defined by the command {\bf setvar}.  The observation
information is read from the {\sc ro} file header.

\item[Usage :] To subtract chop phase B from chop phase A and to
display result and do crude aperture photometry on-line.
\item[Associated commands :] {\tt \htmlref{setvar}{SETVAR}}
\item[Short version of command :] -
\item[Invocation :]

\begin{quote}{\tt  chdisp }\\
or \\
{\tt chdisp 47 }
\end{quote}

\begin{itemize}

\item {\tt 47 } is the CHOP mode observation number
\end{itemize}

\end{description}

\hrule
\subsubsection*{\label{CHRED}\xlabel{CHRED}CHRED}

\begin{description}

\item[Description :] Reduces a set of CHOP observations to produce a
final object-sky (or object-object if chopping onto the array) image
which can be the sum of a series of independent (yet consecutive)
observations. Nodding between object and sky position (for three
position chopping) is also supported and produces two independent
images one for the main nod beam and one for the offset nod beam.

\item[Usage :] Reduce a set of chop observations at the same spatial position.
\item[Associated commands :] {\tt \htmlref{chdisp}{CHDISP}}
\item[Short version of command :] -
\item[Invocation :]

\begin{quote}{\tt  chred }\\
or \\
{\tt chred ch\_ 21 5 0 1 1.123 gl961\_nbm }
\end{quote}

\begin{itemize}

\item {\tt ch\_ } is the prefix for the individual processed CHOP images
\item {\tt 21 } is the start observation number of the sequence
\item {\tt 5 } is the number of individual CHOP observations in sequence
\item {\tt 0 }means CHOP observations were not nodded
(1 would mean CHOP observations were  nodded)
\item {\tt 1 } means airmass correct images
(0 would mean don't airmass correct images)
\item {\tt 1.123 } is the average airmass of the observations
\item {\tt gl961\_nbm } is the name of the final CHOP output image
\end{itemize}

\end{description}

\hrule
\subsubsection*{\label{CIRCLE}\xlabel{CIRCLE}CIRCLE}

\begin{description}

\item[Description :] Plots a circle at a user-defined position in an
image.  The circle is plotted on a user-selected pixel (selected by the
X,Y pixel position of the centre of the circle within the current
image) in the image and is of a user-defined diameter (in arcseconds).
To set the arcsecond/pixel scale in the image use the command {\bf setps}.

\item[Usage :] Plot a circle centred on a specific image pixel.
\item[Associated commands :] {\tt \htmlref{ccircle}{CCIRCLE}},
{\tt \htmlref{line\_width}{LINE_WIDTH}}, {\tt \htmlref{setps}{SETPS}}
\item[Short version of command :] -
\item[Invocation :]

\begin{quote}{\tt  circle }\\
or \\
{\tt circle 126 94 18 }
\end{quote}

\begin{itemize}

\item {\tt 126 94} are the X,Y pixel coordinates where circle is plotted
\item {\tt 18 } is the diameter of circle in arcseconds
\end{itemize}

\end{description}

\hrule
\subsubsection*{\label{CLEAR}\xlabel{CLEAR}CLEAR}

\begin{description}

\item[Description :] Clears/removes all images and graphics from the
current graphics device to produce an empty/blank window.

\item[Usage :] To clear off all graphics from display.
\item[Associated commands :] {\tt \htmlref{clearit}{CLEARIT}}
\item[Short version of command :] {\tt cl}
\item[Invocation :]

\begin{quote}{\tt  clear }\end{quote}

\end{description}

\hrule
\subsubsection*{\label{CLEARIT}\xlabel{CLEARIT}CLEARIT}

\begin{description}

\item[Description :] Clears a selected portion of the graphics device
of images/graphics. The section of the graphics device cleared is
defined by the cursor (the bottom left corner of the box then the top
right corner of the box) and the rectangular area defined is set to the
background colour (generally black).

\item[Usage :] To remove graphics from a section of the display area.
\item[Associated commands :] {\tt \htmlref{clear}{CLEAR}}
\item[Short version of command :] -
\item[Invocation :]

\begin{quote}{\tt  clearit }\end{quote}

\end{description}

\hrule
\subsubsection*{\label{CLINE}\xlabel{CLINE}CLINE}

\begin{description}

\item[Description :] Plots a straight line between two user-defined
positions in an image.  The line is plotted from pixel selected with
the cursor in the current image.  The line width used is defined by the
command {\bf line\_width}.

\item[Usage :] Plot a line between two position in an image.
\item[Associated commands :] {\tt \htmlref{line}{LINE}},
{\tt \htmlref{line\_width}{LINE_WIDTH}}, {\tt \htmlref{setcurmark}{SETCURMARK}}
\item[Short version of command :] -
\item[Invocation :]

\begin{quote}{\tt  cline }\end{quote}

\end{description}

\hrule
\subsubsection*{\label{CNSIGMA}\xlabel{CNSIGMA}CNSIGMA}

\begin{description}

\item[Description :] An image display command to display on the current
graphics device a user-defined image stored in Starlink NDF format.

The scaling of the image display (sometimes referred to as the cut or
{\it max},{\it min} levels) is determined by a parameter called the
sigma level and is input by the user.  The {\it max},{\it min} levels
of the image display is determined to be n standard deviations above
and below the mean of the image; the standard deviation is that from
all the pixels in the image.  The value of n is input by the user.

Thus, a 5-sigma display using {\bf cnsigma} plots the image with {\it
max},{\it min} equal to m+5s,m-5s, respectively where m is the mean (or
average) of the pixels in the image and s is the standard deviation of
those pixels about the mean.  The sigma level input can by fractional
\emph{e.g.}, it could be 0.1 or 0.057 or 10 or 100 or 50.3 \emph{etc.}
If the sigma level is input to be 0 then the user is prompted for the
upper and lower sigma level to be used. This terminology means that
instead of in the above example, the {\it max},{\it min} level being 5
standard deviations both above and below the mean, the user can chose
the sigma levels independently thus, could define the {\it max},{\it
min} values to be m+10s,m-1s, respectively.

Under this version of the command (the alternative is just called {\bf
nsigma}) the position the image is plotted on the graphics device is
determined by the cursor ({\bf nsigma} plots the image at the centre of
the graphics device).  The size of the image on the current graphics
device is determined by the value of the image magnification defined by
the command {\bf setmag}.  A magnification of 0 indicates that the
program auto-scales the image size to fill 80\% of the current graphics
device.  Fractional magnification are allowed so that images larger
than the size of the current graphics device can be displayed.

\item[Usage :] To plot a image with nsigma scaling.

\item[Associated commands :] {\tt \htmlref{nsigma}{NSIGMA}},
{\tt \htmlref{moren}{MORENSIGMA}}, {\tt \htmlref{clear}{CLEAR}},
{\tt \htmlref{plot}{PLOT}}, {\tt \htmlref{cplot}{CPLOT}},
{\tt \htmlref{ranplot}{RANPLOT}}, {\tt \htmlref{cranplot}{CRANPLOT}}, \\
{\tt \htmlref{vargrey}{VARGREY}}, {\tt \htmlref{cvargrey}{CVARGREY}},
{\tt \htmlref{setmag}{SETMAG}}, {\tt \htmlref{again}{AGAIN}},
{\tt \htmlref{setps}{SETPS}}, {\tt \htmlref{setnum}{SETNUM}},
{\tt \htmlref{setcen}{SETCEN}}, {\tt \htmlref{surround}{SURROUND}}

\item[Short version of command :] {\tt cns}
\item[Invocation :]

\begin{quote}{\tt  cnsigma }\\
or \\
{\tt cnsigma 42 }\\
or \\
{\tt  cnsigma 0 }
\end{quote}

\begin{itemize}
\item {\tt 42 } is the observation number of the image to be plotted
\item {\tt 0 } will prompt you for the full name of image to be plotted
\end{itemize}

\end{description}

\hrule
\subsubsection*{\label{COFF}\xlabel{COFF}COFF}

\begin{description}

\item[Description :] Calculates accurate spatial offsets for a series
of images taken at supposedly the same spatial position. The user
clicks on object (star) in the first image in the series and from
either the centroid of the image or the peak pixel in the region (the
user has the choice) accurate spatial offsets of the subsequent images
are calculated relative to the spatial position of the feature in the
first image.  For example, a series of repeat images images of the same
object taken at the same spatial position can be input and the spatial
offsets (due to telescope oscillation or tracking errors) will be
calculated relative to the first image in the sequence.  The offsets in
the user-named output ASCII file are in arcseconds.  To set the pixel
scale use the command {\bf setps}.

\item[Hint :] Use {\bf coff} for a sequence of images taken at supposedly the
same spatial position. Use {\tt jitreg} for a sequence of images taken with a
small jitter offset between them. Use {\bf accoff} to get spatial
offsets of a sequence of images in a mosaic pattern (\emph{i.e.}, with
large offsets to survey a large area of sky).

\item[Usage :] To work out offsets between subsequent images all of
which are supposedly at same spatial position.
\item[Associated commands :] {\tt \htmlref{setps}{SETPS}},
{\tt \htmlref{centroid}{CENTROID}}, {\tt \htmlref{histo}{HISTO}},
{\tt \htmlref{accoff}{ACCOFF}}, {\tt \htmlref{jitreg}{JITREG}}
\item[Short version of command :] -
\item[Invocation :]

\begin{quote}{\tt  coff }\end{quote}

\end{description}

\hrule
\subsubsection*{\label{COLINV}\xlabel{COLINV}COLINV}

\begin{description}

\item[Description :] Inverts the colours in a colour table (LUT).  The
current colour table in use is inverted and displayed immediately.  All
subsequent colour tables displayed are displayed inverted.  If a
greyscale colour table is in use going from black (low signal) to white
(high signal) then after {\bf colinv} the colour table will be white (low
signal) to black (high signal).  To re-establish the normal colour
table display run command {\bf colinv} again.

\item[Usage :] To invert colour table (b-w goes to w-b) and visa versa.
\item[Associated commands :] {\tt \htmlref{coltab}{COLTAB}},
{\tt \htmlref{writelut}{WRITELUT}},
{\tt grey}, {\tt col19}, {\tt heat}, {\tt col13}
\item[Short version of command :] -
\item[Invocation :]

\begin{quote}{\tt  colinv }\end{quote}

\end{description}

\hrule
\subsubsection*{\label{COLTAB}\xlabel{COLTAB}COLTAB}

\begin{description}

\item[Description :] Basic command to write a colour table (LUT) to the
current graphics device.  {\bf coltab} asks for the name of the colour table
to be written to the graphics device and first checks if there is a
file by that name in the current directory.  If there isn't then {\bf coltab}
looks in the {\sc IrcamDR} directory with environment variable name
{\tt \$LIRCAMDIR}. The colour tables (LUTs) themselves are in FIGARO
format with intensities for the Red, Green and Blue guns being defined
in a 2D DATA\_ARRAY component of size 3$\times$256 with values between
0 and 1.  The default set of colour tables in {\tt \$LIRCAMDIR} ({\tt
col1} through {\tt 49} plus {\tt grey} and {\tt heat}) can also be used
with the FIGARO command {\bf colour}.

\item[Usage :] Plot an {\sc IrcamDR} colour table of your own
created with {\bf crecolt}.
\item[Associated commands :] {\tt \htmlref{colinv}{COLINV}},
{\tt \htmlref{ablock}{ABLOCK}}, {\tt \htmlref{cablock}{CABLOCK}},
{\tt \htmlref{writelut}{WRITELUT}}, {\tt grey}, {\tt col19}, {\tt heat}, \\
{\tt col13}, {\tt \htmlref{crecolt}{CRECOLT}}
\item[Short version of command :] ct
\item[Invocation :]

\begin{quote}{\tt  coltab }\\
or \\
{\tt coltab col13 }
\end{quote}

\begin{itemize}

\item {\tt col13 } is the name of the {\sc IrcamDR} colour table (LUT) to be
displayed
\end{itemize}

\end{description}

\hrule
\subsubsection*{\label{CONT_TITLE}\xlabel{CONT_TITLE}CONT\_TITLE}

\begin{description}

\item[Description :] Sets the title string for line graphics plots
executed using the command contour which produces a contour plot of an
image on the current graphics device.

\item[Usage :] To change title of a contour map without running through
all of {\bf setcont}.
\item[Associated commands :] {\tt \htmlref{contour}{CONTOUR}},
{\tt \htmlref{setfont}{SETFONT}}
\item[Short version of command :] -
\item[Invocation :]

\begin{quote}{\tt  cont\_title }\\
or \\
{\tt cont\_title 'GL961 K band image' }
\end{quote}

\begin{itemize}

\item {\tt 'GL961 K band image' } is the title string to be displayed
under contour command
\end{itemize}

\end{description}

\hrule
\subsubsection*{\label{CONTOFF}\xlabel{CONTOFF}CONTOFF}

\begin{description}

\item[Description :] Moves the location of a contour map plot on the
current graphics device by an integer number of pixels.  This is useful
when aligning a contour map with an image.  In X (\emph{i.e.}, RA), a
positive offset moves the contour map to the left (\emph{i.e.},
increasing RA).  In Y (\emph{i.e.}, Dec), a positive offset moves the
contour map down (\emph{i.e.}, to decreasing Dec).

\item[Usage :] Move a contour map slightly with respect to
say, underlying image.

\item[Associated commands :] {\tt \htmlref{setcont}{}SETCONT},
{\tt \htmlref{contour}{CONTOUR}}, {\tt \htmlref{line\_width}{LINE_WIDTH}}
\item[Short version of command :] -
\item[Invocation :]

\begin{quote}{\tt  contoff }\\
or \\
{\tt contoff -5 8 }
\end{quote}

\begin{itemize}

\item {\tt -5 8 } are the X,Y spatial offsets to be used between image
display and contour map display
\end{itemize}

\end{description}

\hrule
\subsubsection*{\label{CONTOUR}\xlabel{CONTOUR}CONTOUR}

\begin{description}

\item[Description :] Plots a contour map on the current graphics
device.  The contour levels together with several other associated
parameters a set by the command {\bf setcont}.  The width of the line
used to plot the contours is set with the command {\bf line\_width}.

\item[Usage :] To plot a contour map.
\item[Associated commands :] {\tt \htmlref{setcont}{SETCONT}},
{\tt \htmlref{line\_width}{LINE_WIDTH}}, {\tt \htmlref{contoff}{CONTOFF}}
\item[Short version of command :] {\tt cont}
\item[Invocation :]

\begin{quote}{\tt  contour }\\
or \\
{\tt contour 42 }\\
or \\
{\tt contour 0 }
\end{quote}

\begin{itemize}
\item {\tt 42} is the observation number to be contoured
\item {\tt 0} will prompt you for the full name of image to be contoured
\end{itemize}

\end{description}

\hrule
\subsubsection*{\label{CPLOT}\xlabel{CPLOT}CPLOT}

\begin{description}

\item[Description :] Plots an image using scaling {\it max},{\it min}
values input by the user.  The user defines the {\it max},{\it min}
values to be used via the command {\bf setmm}.  The image centre is
defined by cursor input from the user.  The size of the image on the
current graphics device is determined by the value of the image
magnification defined by the command {\bf setmag}.  A magnification of
0 indicates that the program auto-scales the image size to fill 80\% of
the current graphics device.  Fractional magnification are allowed so
that images larger than the size of the current graphics device can be
displayed.

\item[Usage :] To plot an image using plot {\it max},{\it min} scaling.

\item[Associated commands :] {\tt \htmlref{plot}{PLOT}},
{\tt \htmlref{morep}{MOREPLOT}}, {\tt \htmlref{clear}{CLEAR}},
{\tt \htmlref{cnsigma}{CNSIGMA}}, {\tt \htmlref{nsigma}{NSIGMA}},
{\tt \htmlref{ranplot}{RANPLOT}}, {\tt \htmlref{cranplot}{CRANPLOT}}, \\
{\tt \htmlref{vargrey}{VARGREY}}, {\tt \htmlref{cvargrey}{CVARGREY}},
{\tt \htmlref{setmag}{SETMAG}}, {\tt \htmlref{setcen}{SETCEN}},
{\tt \htmlref{setmm}{SETMM}}, {\tt \htmlref{again}{AGAIN}},
{\tt \htmlref{setps}{SETPS}}, {\tt \htmlref{setnum}{SETNUM}},
{\tt \htmlref{surround}{SURROUND}}

\item[Short version of command :] {\tt cpl}
\item[Invocation :]

\begin{quote}{\tt  cplot }\\
or \\
{\tt cplot 42 }\\
or\\
{\tt cplot 0 }
\end{quote}

\begin{itemize}

\item {\tt 42} is the number of the observation to be plotted
\item {\tt 0} will prompt you for the full name of image to be plotted
\end{itemize}

\end{description}

\hrule
\subsubsection*{\label{CRANPLOT}\xlabel{CRANPLOT}CRANPLOT}

\begin{description}

\item[Description :] Plots an image using a range scaling value input
by the user.  The scaling {\it max},{\it min} values are determined by
the program to be + (max) and - (min) the range value on the mean of
the image \emph{i.e.}, {\it max} = mean+range, {\it min} = mean-range.
The image centre is defined by cursor input from the user.  The size of
the image on the current graphics device is determined by the value of
the image magnification defined by the command {\bf setmag}.  A
magnification of 0 indicates that the program auto-scales the image
size to fill 80\% of the current graphics device.  Fractional
magnification are allowed so that images larger than the size of the
current graphics device can be displayed.

\item[Usage :] Plot an image using range scaling.

\item[Associated commands :] {\tt \htmlref{ranplot}{RANPLOT}},
{\tt \htmlref{morer}{MORERANPLOT}}, {\tt \htmlref{clear}{CLEAR}},
{\tt \htmlref{cnsigma}{CNSIGMA}}, {\tt \htmlref{nsigma}{NSIGMA}},
{\tt \htmlref{plot}{PLOT}}, {\tt \htmlref{cplot}{CPLOT}}, \\
{\tt \htmlref{vargrey}{VARGREY}}, {\tt \htmlref{cvargrey}{CVARGREY}},
{\tt \htmlref{setmag}{SETMAG}}, {\tt \htmlref{setcen}{SETCEN}},
{\tt \htmlref{again}{AGAIN}}, {\tt \htmlref{setps}{SETPS}},
{\tt \htmlref{setnum}{SETNUM}}, {\tt \htmlref{surround}{SURROUND}}

\item[Short version of command :] {\tt cran}

\item[Invocation :]

\begin{quote}{\tt cranplot}\\
or \\
{\tt cranplot 42}\\
or \\
{\tt cranplot 0}
\end{quote}

\begin{itemize}
\item {\tt 42} is the observation number of the image to be plotted
\item {\tt 0} will prompt you for the full name of image to be plotted
\end{itemize}

\end{description}

\hrule
\subsubsection*{\label{CROSS}\xlabel{CROSS}CROSS}

\begin{description}

\item [Description :] Plots a cross at a user-defined position in an
image.  The cross is plotted on a user-selected pixel (selected by its
X,Y pixel number) in the image and is of a user-defined size (in
arcseconds).  To set the arcsecond/pixel scale in the image use the
command {\bf setps}.

\item[Usage :] Plot a cross on an image.

\item[Associated commands :] {\tt \htmlref{ccross}{CCROSS}}, {\tt
\htmlref{line\_width}{LINE_WIDTH}}, {\tt \htmlref{setps}{SETPS}}

\item[Short version of command :] -

\item[Invocation :]

\begin{quote}{\tt  cross }\\
or \\
{\tt cross 18 37 5 }
\end{quote}

\begin{itemize}

\item {\tt 18 37 } are the X,Y pixel coordinates for the cross
\item {\tt 5 } is the size of the cross in arcseconds
\end{itemize}

\end{description}

\hrule
\subsubsection*{\label{CROSSCUT}\xlabel{CROSSCUT}CROSSCUT}

\begin{description}

\item[Description :] Plots RA and Dec (X and Y) cross-cuts through a
cursor selected point on an image.  A cross-cut is a slice through the
image and displayed as a line graph.  The results are plotted on the
current graphics device and afterwards the users is asked if they
require a hardcopy of the cross-cuts on the current hardcopy device
(defined by the command {\bf sethard}).

\item[Usage :] To plot an RA,Dec cross-cut through a star to see profile
and FWHM value.
\item[Associated commands :] {\tt \htmlref{crosscut\_peak}{CROSSCUT_PEAK}}
\item[Short version of command :] {\tt cc}
\item[Invocation :]

\begin{quote}{\tt  crosscut }\\
or \\
{\tt crosscut N }
\end{quote}

\begin{itemize}

\item {\tt N } means don't clear current image display before plotting
cross-cuts (no parameter or anything other than N means
do clear current image display before plotting cross-cuts)
\end{itemize}

\end{description}

\hrule
\subsubsection*{\label{CROSSCUT_PEAK}\xlabel{CROSSCUT_PEAK}CROSSCUT\_PEAK}

\begin{description}

\item[Description :] Plots RA and Dec (X and Y) cross-cuts through the
peak pixel in a small box centred on a cursor selected point on an
image.  A cross-cut is a slice through the image and displayed as a
line graph.  The results are plotted on the current graphics device and
afterwards the users is asked if they require a hardcopy of the
cross-cuts on the current hardcopy device (defined by the command {\bf
sethard}).

\item[Usage :] Plot RA,Dec cross-cuts through peak pixel on a star.
\item[Associated commands :] {\tt \htmlref{crosscut}{CROSSCUT}}
\item[Short version of command :] {\tt ccp}
\item[Invocation :]

\begin{quote}{\tt  crosscut\_peak }\\
or \\
{\tt crosscut\_peak 8 35 }
\end{quote}

\begin{itemize}

\item {\tt 8} is the sigma level of the {\bf nsigma} plot to display image
through which {\bf crosscut\_peak} is required
\item {\tt 35 } is the number of pixels n-s and e-w covering full
cross-cut through feature (full cross-cut is in this case
70 pixel long in both directions centred on peak pixel
near cursor selected position
\end{itemize}\end{description}

\hrule
\subsubsection*{\label{CSTATS}\xlabel{CSTATS}CSTATS}

\begin{description}

\item[Description :] Returns statistical information ({\it max},{\it
min}, locations of {\it max},{\it min}, mean, median, 1-sigma standard
deviation) from a sub-area of an image.  The sub-area size is defined
in pixels by the user.  The sub-area location is defined by cursor
input.  A rectangular box is drawn on the current image at the location
of the cursor hit and of the size selected by the user.  If the command
is executed as {\tt cstats~20}, say, then a sub-area of size
20$\times$20 pixels is used for the statistical determination.  If the
user just invokes the command as {\tt cstats} then both the X and Y
size of the sub-area are prompted for.

\item[Usage :] Calculate statistics in sub-area of image currently displayed.
\item[Associated commands :] {\tt \htmlref{stats}{STATS}},
{\tt \htmlref{limag}{LIMAG}}
\item[Short version of command :] -
\item[Invocation :]

\begin{quote}{\tt  cstats }
or \\
{\tt cstats 20 }
\end{quote}

\begin{itemize}

\item {\tt 20 }is the sub-area size (in pixel) in which statistics are
calculated.

\end{itemize}

\end{description}

\hrule
\subsubsection*{\label{CURHOT}\xlabel{CURHOT}CURHOT}

\begin{description}

\item[Description :] Calls a-task application {\bf hotshot} to automatically
remove bad/hot pixels from a sub-area of an image located at a cursor
defined position.  {\bf curhot} works on the current image displayed on the
current graphics device and once the location of the sub-area is
selected using the cursor it feeds the information to {\bf hotshot}.
{\bf hotshot} asks for several additional parameters viz: X and Y size
of the search box (centre on the cursor location) in pixels, the X and
Y bin size for the search, the sigma level for hot pixel detection, the
upper threshold level for bad pixel removal and whether to replace the
hot pixel by the mean value in the bin or a bad pixel magic number.
See {\bf hotshot} documentation for more details. The output is the
image after bad/hot pixels have been removed.

\item[Usage :] To remove hot/bad pixels from a sub-area of current image
displayed.
\item[Associated commands :] {\tt \htmlref{hotshot}{HOTSHOT}}
\item[Short version of command :] -
\item[Invocation :]

\begin{quote}{\tt  curhot }\end{quote}

\end{description}

\hrule
\subsubsection*{\label{CURSOR}\xlabel{CURSOR}CURSOR}

\begin{description}

\item[Description :] Displays the cursor on the current graphics device
and returns the pixel (intensity) value and location (X,Y pixel) at
that position. If you click the mouse off the current image a value of
{\tt -9999} is returned indicating that the pixel chosen does not
exist. This command only gives you one cursor, if you want several use
{\bf mcursor}. Note that {\bf cursor}/{\bf mcursor} do not continually
update, they return values for the pixel that the user selects with the
left mouse button.

\item[Usage :] Get one cursor for returning one pixel position/value.
\item[Associated commands :] {\tt \htmlref{mcursor}{MCURSOR}}
\item[Short version of command :] {\tt c}
\item[Invocation :]

\begin{quote}{\tt  cursor }\end{quote}

\end{description}

\hrule
\subsubsection*{\label{CUT}\xlabel{CUT}CUT}

\begin{description}

\item[Description :] Plots a cut/slice through an image between two
user-defined pixels.  The pixels end-points are defined by their X,Y
pixel coordinates with 1,1 being the bottom left corner of the image.
Use {\bf ccut} if you want to define the end-points using the cursor.
The graph produced by {\bf cut} is defined by the command {\bf setcut}
which must be execute before {\bf cut} will operate.  {\bf setcut} has
numerous setup options for the graph produced.  Use {\bf line\_width}
to define the thickness of the line used to draw the graph.

\item[Usage :] Plot cuts between specific pixels in an image.
\item[Associated commands :] {\tt \htmlref{setcut}{SETCUT}},
{\tt \htmlref{ccut}{CCUT}}
\item[Short version of command :] -
\item[Invocation :]

\begin{quote}{\tt  cut }\\
or \\
{\tt cut 42 }\\
or \\
{\tt cut 0 }
\end{quote}

\begin{itemize}
\item {\tt 42} is the observation number of image from which a cut is to
be extracted and plotted
\item {\tt 0} prompts for the full name of image for {\bf cut}
\end{itemize}

\end{description}

\hrule
\subsubsection*{\label{CUT2FF}\xlabel{CUT2FF}CUT2FF}

\begin{description}

\item[Description :] Defines whether the pixel (intensity) values along
a cut/slice graph produced by the command {\bf cut} are written to an ASCII
file named by the user as well as being plotted.  {\bf cut2ff} switches on or
off this option and so has to be executed before and after a {\bf cut} plot
is performed.  The ASCII file is a list of X,Y,intensity values along
the cut.  The cut can be at any orientation within the image, if it is
not along rows or columns then {\bf cut} interpolates the intensity values
along the arbitrary direction selected.

\item[Usage :] To get an ASCII file of cut pixels and values.
\item[Associated commands :] {\tt \htmlref{cut}{CUT}},
{\tt \htmlref{ccut}{CCUT}}, {\tt \htmlref{setcut}{SETCUT}}
\item[Short version of command :] -
\item[Invocation :]

\begin{quote}{\tt  cut2ff }\\
or \\
{\tt cut2ff 1 ngc1068.cut }
\end{quote}

\begin{itemize}

\item {\tt 1} means enable {\bf cut} writing to ASCII file
(0 means disable {\bf cut} writing)
\item {\tt ngc1068.cut } is the name of the output ASCII file for {\bf cut}
\end{itemize}

\end{description}

\hrule
\subsubsection*{\label{CUT_TITLE}\xlabel{CUT_TITLE}CUT\_TITLE}

\begin{description}

\item[Description :] Sets the title string for a {\bf cut}/{\bf ccut}
plot through an image.  This string can also be set in {\bf setcut}.

\item[Usage :] Change {\bf cut} title rather than going through full
{\bf setcut} options.
\item[Associated commands :] {\tt \htmlref{cut}{CUT}},
{\tt \htmlref{ccut}{CCUT}}, {\tt \htmlref{setcut}{SETCUT}}
\item[Short version of command :] -
\item[Invocation :]

\begin{quote}{\tt  cut\_title }\\
or \\
{\tt cut\_title 'this is the title string' }
\end{quote}

\end{description}

\hrule
\subsubsection*{\label{CVARGREY}\xlabel{CVARGREY}CVARGREY}

\begin{description}

\item[Description :] Plots an image on the current graphics device
using variable index scaling.  Variable index scaling maps a percentage
of the dynamic range in the image onto a different percentage of the
colour table range available.  See command {\bf vargrey} for a more
complete definition of variable index scaling.  cvargrey plots the
image with its centre located at the position of the cursor input from
the user.  The variable index scaling parameters are set using the
command {\bf setvargrey}.

\item[Usage :] Plot an image using {\bf vargrey} scaling.

\item[Associated commands :] {\tt \htmlref{vargrey}{VARGREY}},
{\tt \htmlref{morev}{MOREVARGREY}}, {\tt \htmlref{setvargrey}{SETVARGREY}},
{\tt \htmlref{clear}{CLEAR}}, {\tt \htmlref{cnsigma}{CNSIGMA}},
{\tt \htmlref{nsigma}{NSIGMA}}, {\tt \htmlref{plot}{PLOT}}, \\
{\tt \htmlref{cplot}{CPLOT}}, {\tt \htmlref{setmag}{SETMAG}},
{\tt \htmlref{setcen}{SETCEN}}, {\tt \htmlref{again}{AGAIN}},
{\tt \htmlref{setps}{SETPS}}, {\tt \htmlref{setnum}{SETNUM}},
{\tt \htmlref{surround}{SURROUND}}

\item[Short version of command :] {\tt cvg}
\item[Invocation :]

\begin{quote}{\tt  cvargrey }\\
or \\
{\tt cvargrey 42 }\\
or \\
{\tt cvargrey 0 }
\end{quote}

\begin{itemize}
\item {\tt 42} is the observation number of the image to be plotted
\item {\tt 0 } will prompt you for the full name of image to be plotted
\end{itemize}

\end{description}

\hrule
\subsubsection*{\label{DAOCEN}\xlabel{DAOCEN}DAOCEN}

\begin{description}

\item[Description :] Cursors on current image {\tt n} times and returns
centroid of object at cursor position using {\bf rapi2d} application
{\bf centroid}.

\item[Usage :] Attempting to identify DAO found objects (using find
option) in your image.

\item[Associated commands :] {\tt \htmlref{daofind}{DAOFIND}},
{\tt \htmlref{daogid}{DAOGID}}, {\tt \htmlref{daogid2}{DAOGID2}}

\item[Short version of command :] -

\item[Invocation :]

\begin{quote}{\tt  daocen }\end{quote}

\end{description}

\hrule
\subsubsection*{\label{DAOFIND}\xlabel{DAOFIND}DAOFIND}

\begin{description}

\item[Description :] Analyses the output ASCII file from the DAOPHOT
find command (to auto-find stars in an image) and plots symbols
(circles, crosses, boxes) or id number from DAOPHOT file on object
identified.

\item[Usage :] Identifying stars found by DAOPHOT find command on an image.

\item[Associated commands :] {\tt \htmlref{daocen}{DAOCEN}},
{\tt \htmlref{daogid}{DAOGID}}, {\tt \htmlref{daogid2}{DAOGID2}}

\item[Short version of command :] -

\item[Invocation :]

\begin{quote}{\tt  daofind }\end{quote}

\end{description}

\hrule
\subsubsection*{\label{DAOGID}\xlabel{DAOGID}DAOGID}

\begin{description}

\item[Description :] Analyses the output ASCII file from the DAOPHOT
find command (to auto-find stars in an image) and after cursor
selection of star in currently displayed image, calculates which
DAOPHOT star is the one clicked on by user.

\item[Usage :] Identifying stars found by the DAOPHOT find command on an image.

\item[Associated commands :] {\tt \htmlref{daocen}{DAOCEN}},
{\tt \htmlref{daofind}{DAOFIND}}, {\tt \htmlref{daogid2}{DAOGID2}}

\item[Short version of command :] -

\item[Invocation :]

\begin{quote}{\tt  daogid }\end{quote}

\end{description}

\hrule
\subsubsection*{\label{DAOGID2}\xlabel{DAOGID2}DAOGID2}

\begin{description}

\item[Description :] Analyses two output ASCII files from DAOPHOT find
command (to auto-find stars in an image) and after cursor selection of
star in currently displayed image, calculates which DAOPHOT star is the
one clicked on by user.  Useful to id stars that have been found using
DAOPHOT find command in say, two filters (\emph{e.g.}, J and K).

\item[Usage :] Identifying stars found by the DAOPHOT find command on an image.
\item[Associated commands :] {\tt \htmlref{daocen}{DAOCEN}},
{\tt \htmlref{daofind}{DAOFIND}}, {\tt \htmlref{daogid}{DAOGID}}

\item[Short version of command :] -

\item[Invocation :]

\begin{quote}{\tt  daogid2 }\end{quote}

\end{description}

\hrule
\subsubsection*{\label{DARKLOT}\xlabel{DARKLOT}DARKLOT}

\begin{description}

\item[Description :] {\bf darklot} dark subtracts a series of raw {\sc
ircam1/2} format images taken from a container file (\emph{i.e.}, a
file with name viz:  {\tt ircam\_28dec94\_1c.sdf}).  Description of the
``{\bf lot}'' family of procedures that work via a name prefix, a range
of observation numbers and a name suffix is given above.  {\bf darklot}
takes a range of observation numbers and the observation number of an
associated dark exposure and subtracts the dark current image from the
observations and scales the result to DN/sec/coadd reading the on-chip
exposure time and the number of coadd from the observation header.  The
output dark subtracted files are named via an output filename prefix,
the range of observation numbers input and the suffix letter {\tt
`d'}.  For example, dark subtracting observations 20-25 from the
current {\sc ircam1/2} container file (defined by the command {\bf
setfile}), with an output filename prefix of {\tt im\_} would result in
dark subtracted images with names {\tt im\_20d}, {\tt im\_21d}, \ldots,
to {\tt im\_25d}.

To dark subtract a series of {\sc ircam3} ({\sc ro}) images use the command
{\bf rodarklot}.

\item[Usage :] Dark subtract a series of {\sc ircam1/2} images from
container file.
\item[Associated commands :] {\tt \htmlref{rodarklot}{RODARKLOT}},
{\tt \htmlref{setfile}{SETFILE}}
\item[Short version of command :] -
\item[Invocation :]

\begin{quote}{\tt  darklot }\end{quote}

\end{description}

\hrule
\subsubsection*{\label{DEGLOT}\xlabel{DEGLOT}DEGLOT}

\begin{description}

\item[Description :] Deglitches a series of images defined by a name
prefix, a range of observation numbers and a name suffix.  See above
for definition of the operation of the ``{\bf lot}'' family of images.
The input images are analysed and any pixels set to the bad pixel magic
number value ({\tt -1.0e-20} in {\sc IrcamDR}) are removed by replacement
with the median of the surrounding 8 pixels.  {\bf deglot} calls the {\bf
rapi2d} program {\bf glitch} to do the work.  Bad pixels (or hot pixels) can
be set to the bad pixel magic number using the command {\bf applymask} or
{\bf applymasklot} (to apply a bad pixel mask image).  {\bf deglot}
adds a letter {\tt `g'} to the input filenames and writes the deglitched
images to files of these names.

\item[Usage :] De-glitching a series of images using a bad pixel mask.
\item[Associated commands :] {\tt \htmlref{glitch}{GLITCH}},
{\tt \htmlref{applymask}{APPLYMASK}},
{\tt \htmlref{applymasklot}{APPLYMASKLOT}}
\item[Short version of command :] -
\item[Invocation :]

\begin{quote}{\tt  deglot }\end{quote}

\end{description}

\hrule
\subsubsection*{\label{DISP}\xlabel{DISP}DISP}

\begin{description}

\item[Description :] This command takes as input two raw observation
numbers, the second is subtracted from the first and the result is
displayed on the current graphics device.  The two observations can be
object and sky or two objects with spatial offsets between them.  Once
the difference image is displayed the option is presented to the user
to do aperture photometry on any sources present.  The parameters of
the raw observations are reads from the header of the {\sc ro} files
and are used when photometry is performed to get counts/sec for
application of the zeropoint currently defined in the system.  On-line
at {\sc ukirt} {\bf disp} reads default zeropoints (the magnitude
corresponding to 1DN/sec).  Off-line users should define zeropoints
using standard star observations and aperture photometry and put them
into the system using the command {\bf setvar}.

\item[Usage :] Quick look display of object-sky images and simple
aperture photometry on result.
\item[Associated commands :] {\tt \htmlref{setvar}{SETVAR}},
{\tt \htmlref{pho}{PHO}}, {\tt \htmlref{pho2}{PHO2}},
{\tt \htmlref{pho3}{PHO3}}, {\tt \htmlref{stdred}{STDRED}},
{\tt \htmlref{stred}{STRED}}
\item[Short version of command :] -
\item[Invocation :]

\begin{quote}{\tt  disp }\\
or \\
{\tt disp 45 46 }
\end{quote}

\begin{itemize}

\item {\tt 45 } is the observation number of the object image
\item {\tt 46 } is the observation number of the sky image
\end{itemize}

\end{description}

\hrule
\subsubsection*{\label{DISPICK}\xlabel{DISPICK}DISPICK}

\begin{description}

\item[Description :] Using cursor selected pixels on the current image
displayed on the current graphics device, {bf dispick} extracts a sub-area
of an images and creates a separate image to contain it.  The user
selects the bottom-left and top-right corners of the rectangular area
to be extracted and {\bf dispick} calls the {\bf rapi2d} application
{\bf pickim} to extract the defined area.  The output image name
containing the sub-area selected is named by the user.

\item[Usage :] To interactively extract and store a sub-area of an image
into a new image.
\item[Associated commands :] {\tt \htmlref{dispick}{DISPICK}}
\item[Short version of command :] -
\item[Invocation :]

\begin{quote}{\tt  dispick }\\
or \\
{\tt dispick 42 }\\
or \\
{\tt dispick 0 }
\end{quote}

\begin{itemize}
\item {\tt 42 } is the observation number of the image to be displayed
and have sub-area extracted from
\item {\tt 0 } will prompt you for the full name of image to be displayed
and have sub-area extracted from
\end{itemize}

\end{description}

\hrule
\subsubsection*{\label{ELLIPSE}\xlabel{ELLIPSE}ELLIPSE}

\begin{description}

\item[Description :] Plots a ellipse at a user-defined position in an
image. The ellipse is plotted on a user-selected pixel (selected by the
pixel X,Y coordinates, 1,1 being the bottom-left corner of the image) in
the image and is of a user-defined major axis (in arcseconds) and
eccentricity. To set the arcsecond/pixel scale in the image use the
command {\bf setps}.

\item[Usage :] To plot an ellipse centred on a pixel of an image
(\emph{e.g.}, an IRAS error ellipse).
\item[Associated commands :] {\tt \htmlref{cellipse}{CELLIPSE}},
{\tt \htmlref{line\_width}{LINE_WIDTH}}, {\tt \htmlref{setps}{SETPS}}
\item[Short version of command :] -
\item[Invocation :]

\begin{quote}{\tt  ellipse }\\
or \\
{\tt ellipse 47 201 10 0.982 42.3 }
\end{quote}

\begin{itemize}

\item {\tt 47 201 } are the X,Y pixel coordinates at which the ellipse
is plotted
\item {\tt 10 } is the major axis of the ellipse in arcseconds
\item {\tt 0.982 } is the eccentricity of the ellipse
\item {\tt 42.3 } is the position angle of the ellipse (e or n)
\end{itemize}

\end{description}

\hrule
\subsubsection*{\label{FLAT2}\xlabel{FLAT2}FLAT2}

\begin{description}

\item[Description :] {\bf flat2} takes as input two images and with
optional normalization to a median of unity, flat-fields the first with
the second and the second with the first.  The output images have the
same name as the input images with a letter {\tt `f'} added.  This is a
useful command for flat-fielding if you have two images of a compact
object (\emph{e.g.}, a star) at two different spatial offsets.  The
images should be dark subtracted using {\bf darklot} or {\bf rodarklot}
(depending on whether you are dealing with data from {\sc ircam1/2} or
{\sc ircam3}) and then self flat-fielded using {\bf flat2}. The
normalization to unity for the image to be divided into the object
image is determined by the {\bf rapi2d} application {\bf stats}.  The
division is performed using the {\bf rapi2d} application {\bf div}.

\item[Usage :] Flat-field two images A,B A with B and B with A.
A and B could be say, two images of a star with a spatial offset between them.
\item[Associated commands :] {\tt \htmlref{stats}{STATS}},
{\tt \htmlref{div}{DIV}}
\item[Short version of command :] -
\item[Invocation :]

\begin{quote}{\tt  flat2 }\\
or \\
{\tt flat2 im\_50d im\_51d 1 }
\end{quote}

\begin{itemize}

\item {\tt im\_50d im\_51d } are the two images to be flat-fielded by
each other
\item {\tt 1 } means normalize flat-field in each case
(0 would mean don't normalize flat-fields)
\end{itemize}

\end{description}

\hrule
\subsubsection*{\label{FLATLOT}\xlabel{FLATLOT}FLATLOT}

\begin{description}

\item[Description :] Flat-fields a series of images defined by a name
prefix, a range of observation numbers and a name suffix.  See above
for the definition of the operation of the ``{\bf lot}'' family of
images.  {\bf flatlot} asks for an image to be used as a flat-field and also
whether that images should be normalized to a median of unity.  It then
divides the flat-field images (normalized if requested) into the series
of object images and adds the letter {\tt `f'} to the input image name for
the output flat-fielded images.  The input images should be dark
subtracted using the commands {\bf darklot} or {\bf rodarklot} (depending on
whether you are reducing {\sc ircam1/2} or {\sc ircam3} data).  The
median of the flat-field image, used for normalization to a median of
unity, is calculated using the {\bf rapi2d} application {\tt stats}.  The
flat-fielding is performed using the {\bf rapi2d} application {\bf div}.

\item[Usage :] To flat-field a number of images with the same
flat-field image.
\item[Associated commands :] {\tt \htmlref{stats}{STATS}},
{\tt \htmlref{div}{DIV}}
\item[Short version of command :] -
\item[Invocation :]

\begin{quote}{\tt  flatlot }\end{quote}

\end{description}

\hrule
\subsubsection*{\label{GETCM}\xlabel{GETCM}GETCM}

\begin{description}

\item[Description :] Returns to the user the current calculated image
display {\it max},{\it min} scaling values.  The calculated {\it
max},{\it min} scaling values are defined by the last image display
command used to display an image \emph{e.g.}, {\bf nsigma}, {\bf plot},
{\bf ranplot}, {\bf vargrey}, \emph{etc.}

\item[Usage :] This command is useful when you have plotted an image
using say, nsigma display, and you want to know what the actual {\it
max},{\it min} scaling values are for a specific sigma-level.

\item[Associated commands :] {\tt \htmlref{nsigma}{NSIGMA}},
{\tt \htmlref{cnsigma}{CNSIGMA}}, {\tt \htmlref{plot}{PLOT}},
{\tt \htmlref{cplot}{CPLOT}}, {\tt \htmlref{ranplot}{RANPLOT}},
{\tt \htmlref{cranplot}{CRANPLOT}}, {\tt \htmlref{vargrey}{VARGREY}}, \\
{\tt \htmlref{cvargrey}{CVARGREY}}

\item[Short version of command :] -
\item[Invocation :]

\begin{quote}{\tt  getcm }\end{quote}

\end{description}

\hrule
\subsubsection*{\label{GETOFF}\xlabel{GETOFF}GETOFF}

\begin{description}

\item[Description :] Gets the RA,Dec spatial offsets from the header of
a series of raw {\sc ro} {\sc ircam3} images.  The RA,Dec spatial
offset are shown on the screen and also written to an ASCII file called
{\tt ro.off}.  This file can be used in such commands as {\bf remos} (to
re-mosaic a set of images) or {\bf jitreg} (to refine the offsets and correct
for telescope drifting and pointing changes.

\item[Usage :] Check spatial offsets of set of observations.
\item[Associated commands :] {\tt \htmlref{remos}{REMOS}},
{\tt \htmlref{jitreg}{JITREG}}
\item[Short version of command :] -
\item[Invocation :]

\begin{quote}{\tt  getoff }\end{quote}

\end{description}

\hrule
\subsubsection*{\label{GETRAST}\xlabel{GETRAST}GETRAST}

\begin{description}

\item[Description :] Gets the size (in raster units or device pixels)
of the current graphics device. The size is printed to the screen.
Some of the graphic display routines in {\sc IrcamDR} require that the
plotting location on the screen be defined in terms of the device
raster units rather than a users world coordinates.  There one should
think of the plotting/graphics device as having world coordinates in
X,Y from 1,1 to {\it xmax},{\it ymax} where {\it xmax} and {\it ymax}
are returned by {\bf getrast}.

\item[Usage :] Check size of imaging window in raster units.
\item[Associated commands :] -
\item[Short version of command :] -
\item[Invocation :]

\begin{quote}{\tt  getrast }\end{quote}

\end{description}

\hrule
\subsubsection*{\label{GLITCHMARK}\xlabel{GLITCHMARK}GLITCHMARK}

\begin{description}

\item[Description :] Allows interactive removal of bad/hot
pixels from an image.  An image should be displayed on the current
graphics device and once {\bf glitchmark} is invoked it displays the cursor
on the image and asks the user to select with the cursor all the pixels
that are to be defined as needing de-glitching (\emph{i.e.}, hot and
bad pixel or undesirable features).  The user click the cursor off the
image displayed (on the background to the image but still within the
current graphics device display area) to terminate pixel selection.
Once this has been done the selected pixels coordinates (1,1 bottom
left corner of image 256,256 top right corner of {\sc ircam3} image)
are written to a user-named ASCII file and the user is asked if this
list of bad/hot pixels is to be applied to a specific image.  If the
user wants to apply the list of bad/hot pixels to an image he/she input
a name and a name for the output image and {\bf glitchmark} calls the {\bf
rapi2d} application {\bf glitch} to replace all identified bad/hot pixels by
the median of the 8 surrounding pixels.

\item[Usage :] To identify bad/hot pixels interactively.
\item[Associated commands :] {\tt \htmlref{glitch}{GLITCH}}
\item[Short version of command :] -
\item[Invocation :]

\begin{quote}{\tt  glitchmark }\end{quote}

\end{description}

\hrule
\subsubsection*{\label{GRID}\xlabel{GRID}GRID}

\begin{description}

\item[Description :] Plots a X,Y (RA,Dec) grid on an image displayed on
the current graphics device.  The spacing of the grid lines are in
arcseconds and the arcsecond/pixel scale is defined by the command
{\bf setps}.  The grid starts at the first pixel in X and Y (bottom left
corner) of the image (this is a feature that is due to be eliminated so
that the grid can originate at any pixel in the image).  The X and Y
spacing are independent and can take different values (\emph{e.g.}, 15"
in ra (1 sec of time) and 10" in dec say).  The width of the line used
to draw the grid is defined by the command {\bf line\_width}.

\item[Usage :] Put a grid of user-defined spacing on image displayed.
\item[Associated commands :] {\tt \htmlref{setps}{SETPS}},
{\tt \htmlref{line\_width}{LINE_WIDTH}}
\item[Short version of command :] -
\item[Invocation :]

\begin{quote}{\tt  grid }\\
or \\
{\tt grid 15 10 }
\end{quote}

\begin{itemize}

\item {\tt 15 10 } are the X,Y spacing of grid in arcseconds
\end{itemize}

\end{description}

\hrule
\subsubsection*{\label{HARDCOPY}\xlabel{HARDCOPY}HARDCOPY}

\begin{description}

\item[Description :] Plots last image displayed or last line graphics
displayed or combination of the two to a hardcopy device (generally a
file \emph{e.g.}, a Postscript file).  {\bf hardcopy} presents you with
a menu that allows you to select what you want hard copying.  The {\bf
hardcopy} output device is defined by the command {\bf sethard}.  {\bf
hardcopy} closes plotting to the current graphics device and opens it
on the hardcopy device and plots the output desired and then re-opens
plotting to the previous graphics device.

\item[Usage :] Produce a paper copy of current graphics.
\item[Associated commands :] {\tt \htmlref{sethard}{SETHARD}},
{\tt \htmlref{hardlot}{HARDLOT}}
\item[Short version of command :] {\tt ha}
\item[Invocation :]

\begin{quote}{\tt  hardcopy }\\
or \\
{\tt hardcopy 3 }
\end{quote}

\begin{itemize}
\item {\tt 3} means plot last image to hardcopy device
\item {\tt 1} would mean plot last line graph)
\item {\tt 2} would mean plot last contour map + last vector map
\item {\tt 4} would mean plot last image + last contour map
\item {\tt 5} would mean plot last image + last vector map
\item {\tt 6} would mean plot last image + last contour map + last vector map)
\end{itemize}

\end{description}

\hrule
\subsubsection*{\label{HARDLOT}\xlabel{HARDLOT}HARDLOT}

\begin{description}

\item[Description :] Makes hardcopies of a series of images defined by
a name prefix, range of observation numbers, and a name suffix.  Allows
user to choose the plot method ({\bf nsigma}, {\bf plot}, {\bf
ranplot}, {\bf vargrey}) and their associated parameters.  Also allows
the user to plot a border ({\bf surround}) around images.  The
hardcopies are written to one output file but come out on separate
pages of the hardcopy device.

\item[Usage :] To produces paper copies of series of observations/images.
\item[Associated commands :] {\tt \htmlref{hardcopy}{HARDCOPY}},
{\tt \htmlref{sethard}{SETHARD}}
\item[Short version of command :] -
\item[Invocation :]

\begin{quote}{\tt  hardlot }\end{quote}

\end{description}

\hrule
\subsubsection*{\label{IO2RO}\xlabel{IO2RO}IO2RO}

\begin{description}

\item[Description :] Command to convert {\sc ircam3} {\sc i} and {\sc
o} primitive data files (produced by the data acquisition system) into
user-friendly {\sc ro} files (processed by {\sc IrcamDR}.  The {\sc i}
and {\sc o} files are automatically converted to {\sc ro} files on-line
at the telescope and these are what the user generally goes away from
JAC with.  However, if archive data is required, this command is needed
since only {\sc i} and {\sc o} files are archived to DAT and read-write
CDs.

\item[Usage :] Converting raw integrations/observation files to reduced
observation files for reduction by {\sc IrcamDR}.
\item[Associated commands :] -
\item[Short version of command :] -
\item[Invocation :]

\begin{quote}{\tt  io2ro }\end{quote}

\end{description}

\hrule
\subsubsection*{\label{JITREG}\xlabel{JITREG}JITREG}

\begin{description}

\item[Description :] Calculates accurate spatial offsets
between consecutive images in a sequence or, for example, a mosaic, and
writes them to an ASCII file for later use.  {\bf jitreg} takes as input an
ASCII file with the actual telescope spatial offsets (in arcseconds)
used to take the mosaic image and once the user has identified a common
feature (star) that appears in ALL mosaic tiles, {\bf jitreg} looks at the
correct spatial offsets in each image and refines the spatial offsets
to be accurate.  {\bf jitreg} works on a series of images defined by a name
prefix, a range of observation number and a name suffix.  It will work
best on processed \emph{i.e.}, reduced images (\emph{e.g.}, reduced
with {\bf stred} for example).  {\bf jitreg} uses the centroid of the feature
(star) to calculate accurate spatial offsets between images.  The
offsets written to the output ASCII file are all relative to the
location of the object in the first image in the sequence and are in
arcseconds.  The output ASCII file can be used in the command {\bf remos} to
re-mosaic together a series of images.

\item[Usage :] To refine spatial offsets between mosaic tiles when a
feature (star) appears in all mosaic tiles.

\item[Notes :] Use {\bf accoff} to get spatial offsets of a sequence of
images in a mosaic pattern (\emph{i.e.}, with large offsets to survey a
large area of sky).  Use {\bf coff} for a sequence of images taken at
supposedly the same spatial position. Use {\bf jitreg} for a sequence
of images taken with a small jitter offset between them.

\item[Associated commands :] {\tt \htmlref{setps}{SETPS}},
{\tt \htmlref{histo}{HISTO}}, {\tt \htmlref{coff}{COFF}},
{\tt \htmlref{jitreg}{JITREG}}, {\tt \htmlref{stred}{STRED}},
{\tt \htmlref{remos}{REMOS}}
\item[Short version of command :] -
\item[Invocation :]

\begin{quote}{\tt  jitreg }\end{quote}

\end{description}

\hrule
\subsubsection*{\label{LAGAIN}\xlabel{LAGAIN}LAGAIN}

\begin{description}

\item[Description :] Plots the last line graphics plot (\emph{e.g.},
cut, contour map, vector map) again.

\item[Usage :] When refining parameters for a line plot, you can
display the plot (cut, contour, vec) on the current graphics device,
clear the screen ({\bf clear}), change the parameters ({\bf setcut},
{\bf setcont}, {\bf setvec}) and re-display the line plot using
lagain.

\item[Associated commands :] {\tt \htmlref{cut}{CUT}},
{\tt \htmlref{contour}{CONTOUR}}, {\tt \htmlref{vec}{VEC}},
{\tt \htmlref{setcut}{SETCUT}}, {\tt \htmlref{setcont}{SETCONT}},
{\tt \htmlref{setvec}{SETVEC}}, {\tt \htmlref{clear}{CLEAR}}
\item[Short version of command :] {\tt lag}
\item[Invocation :]

\begin{quote}{\tt  lagain }\end{quote}

\end{description}

\hrule
\subsubsection*{\label{LIMAG}\xlabel{LIMAG}LIMAG}

\begin{description}

\item[Description :] Calculates limiting magnitudes of an image given a
zeropoint value (set with command {\bf setvar}), a filter, exposure time of
image (in sec), and a sub-area size (in pixels). Command {\bf limag} looks at
current image displayed and within the rectangular sub-area (the
position of which is defined by cursor input and the size defined by
user input pixels sizes in X and Y) which should be position on sky
(\emph{i.e.}, without any astronomical objects included) calculates
statistical values (\emph{e.g.}, mean, standard deviations \emph{etc}).
From the 1-sigma standard deviation within the sub-area, {\bf limag}
calculates the 1-, 5-, 10- and 100- sigma limiting magnitudes per
pixel, per sq arcsecond and per 3" aperture.

\item[Usage :] To calculate limiting magnitude of image from sky noise.
\item[Associated commands :] {\tt \htmlref{stats}{STATS}},
{\tt \htmlref{cstats}{CSTATS}}, {\tt \htmlref{histo}{HISTO}}
\item[Short version of command :] -
\item[Invocation :]

\begin{quote}{\tt  limag }\\
or \\
{\tt limag 20 }
\end{quote}

\begin{itemize}

\item {\tt 20 } is the X and Y sub-area size (in pixels) for stats
\end{itemize}

\end{description}

\hrule
\subsubsection*{\label{LINE}\xlabel{LINE}LINE}

\begin{description}

\item[Description :] Plots a straight line between two user-defined
positions in an image.  The line is plotted from pixel selected by
their X,Y pixel coordinates within the current image (1,1 is the
bottom-left corner of image, 256,256 is the top-right corner of an {\sc
ircam3} image).  The line width used is defined by the command {\bf
line\_width}.

\item[Usage :] Draws line on image displayed.
\item[Associated commands :] {\tt \htmlref{cline}{CLINE}},
{\tt \htmlref{line\_width}{LINE_WIDTH}}
\item[Short version of command :] -
\item[Invocation :]

\begin{quote}{\tt  line }\\
or \\
{\tt line 10 21 157 210 }
\end{quote}

\begin{itemize}

\item {\tt 10 21 } are the X,Y start pixels for the line
\item {\tt 157 210 } are the X,Y end pixels for the line
\end{itemize}

\end{description}

\hrule
\subsubsection*{\label{LINE_WIDTH}\xlabel{LINE_WIDTH}LINE\_WIDTH}

\begin{description}

\item[Description :] Defines the thickness of any line graphics plotted
within {\sc IrcamDR}.  The factor entered is a multiple of the default
(1 unit) line width.

\item[Usage :] For hardcopy production of line graphics (contours,
vector maps \emph{etc}).  A good setting for hardcopies is typically 2
or 3.

\item[Associated commands :] -
\item[Short version of command :] {\tt lw}
\item[Invocation :]

\begin{quote}{\tt  line\_width }\\
or \\
{\tt line\_width 3 }
\end{quote}

\begin{itemize}

\item {\tt 3 } is the new line width/thickness

\end{itemize}

\end{description}

\hrule
\subsubsection*{\label{LINECOL}\xlabel{LINECOL}LINECOL}

\begin{description}

\item[Description :] Sets the colour (immediately) of any line graphics
plotted on the current graphics device.  Colours are selected from the
standard set referred to as (WRGBYPCS [W=white, R=red, G=blue, B=blue,
Y=yellow, P=purple, C=cyan, S=pink and N=black]).  The setting of the
line graphics colour is only temporary since any new line graphics will
change the line colour to its own default colour.

\item[Usage :] To change line graphics colour when making a
presentation slide/picture from the device screen.
\item[Associated commands :] -
\item[Short version of command :] -
\item[Invocation :]

\begin{quote}{\tt  linecol }\\
or \\
{\tt linecol r }
\end{quote}

\begin{itemize}

\item {\tt r } is red, the new colour of the line graphics
\end{itemize}

\end{description}

\hrule
\subsubsection*{\label{MCURSOR}\xlabel{MCURSOR}MCURSOR}

\begin{description}

\item[Description :] Displays N cursors consecutively on the current
graphics device (assuming it supports cursoring) and returns the pixel
location within the current image and the pixel value at that location.
If you click the mouse off the current image a value of {\tt -9999} is
returned indicating that the pixel chosen does not exist.  {\bf mcursor}
gives you the option to specify at the start the number of cursors you
want or to keep hitting the return key to get another one.

\item[Usage :] To put up multiple cursors for pixel inquiries.
\item[Associated commands :] {\tt \htmlref{cursor}{CURSOR}}
\item[Short version of command :] {\tt mc}
\item[Invocation :]

\begin{quote}{\tt  mcursor }\\
or \\
{\tt mcursor 5 }
\end{quote}

\begin{itemize}

\item {\tt 5} is the number of cursors displayed consecutively
\end{itemize}

\end{description}

\hrule
\subsubsection*{\label{MEDLOT}\xlabel{MEDLOT}MEDLOT}

\begin{description}

\item[Description :] Subtracts the median value from a
sub-area of an image from the whole image.  It works on a series of
images whose names are defined by a name prefix, a series of
observation numbers and a name suffix. See above for the definition of
the operation of the ``{\bf lot}'' family of images.  The median value
is calculated using the {\bf rapi2d} application {\bf stats}.  The
sub-area of the image in which the median is calculated is defined by
its start and end X,Y pixel coordinates, 1,1 is the bottom-left corner
of the image, 256,256 is the top-right corner of an {\sc ircam3}
image.  The output images have the same names as the input images but
with the letter `{\tt m}' added, thus, if {\tt im\_50dfzm} was the
input image name (with prefix {\tt im\_}, a number range 50-54 say, and
a suffix {\tt dfzm}) then the output image would be {\tt im\_50dfzmm}.

\item[Usage :] Sky subtraction in flat-fielded image.
\item[Associated commands :] {\tt \htmlref{stats}{STATS}}
\item[Short version of command :] -
\item[Invocation :]

\begin{quote}{\tt  medlot }\end{quote}

\end{description}

\hrule
\subsubsection*{\label{MOFFLOT}\xlabel{MOFFLOT}MOFFLOT}

\begin{description}

\item[Description :] Calculates accurate spatial and dc level offsets
between consecutive images in a series of image from, say, a mosaic.
{\bf mofflot} uses the {\bf rapi2d} application {\bf moff}.  The image
sequence is defined by a name prefix, a range of observation numbers
and a name suffix.  See above for the definition of the operation of
the ``{\bf lot}'' family of images.  {\bf mofflot} requires an ASCII
telescope spatial offset (in arcseconds) which could be created using
the command {\bf getoff} from the raw {\sc ro} files of a spatial
mosaic.

\item[Usage :] Calculating dc and spatial offsets between mosaic tiles.
\item[Associated commands :] {\tt \htmlref{jitreg}{JITREG}},
{\tt \htmlref{accoff}{ACCOFF}}, {\tt \htmlref{coff}{COFF}}
\item[Short version of command :] -
\item[Invocation :]

\begin{quote}{\tt  mofflot }\end{quote}

\end{description}

\hrule
\subsubsection*{\label{MORENSIGMA}\xlabel{MORENSIGMA}MORENSIGMA}

\begin{description}

\item[Description :] Replots the last nsigma image display using a
different sigma level.  See command {\bf nsigma} for details of the nsigma
image display method.

\item[Usage :] Repeats last nsigma image display with new sigma-level.
\item[Associated commands :] {\tt \htmlref{nsigma}{NSIGMA}},
{\tt \htmlref{cnsigma}{CNSIGMA}}
\item[Short version of command :] {\tt moren}, {\tt mn}
\item[Invocation :]

\begin{quote}{\tt  morensigma }\\
or \\
{\tt morensigma 10 }
\end{quote}

\begin{itemize}

\item {\tt 10} is the new sigma level for the nsigma display of the last
image plotted
\end{itemize}

\end{description}

\hrule
\subsubsection*{\label{MOREPLOT}\xlabel{MOREPLOT}MOREPLOT}

\begin{description}

\item[Description :] Replots the last {\bf plot} image display using a
different {\it max},{\it min} scaling values.  See command {\bf plot}
for details of the plot image display method.

\item[Usage :] To repeat last plot image display with new {\it max},{\it min}.
\item[Associated commands :] {\tt \htmlref{plot}{PLOT}},
{\tt \htmlref{cplot}{CPLOT}}
\item[Short version of command :] {\tt morep}, {\tt mp}
\item[Invocation :]

\begin{quote}{\tt  moreplot }\\
or \\
{\tt moreplot 200 1000 }
\end{quote}

\begin{itemize}

\item {\tt 200 1000 } are the new {\it min} and {\it max} scaling values for the plot
\end{itemize}

\end{description}

\hrule
\subsubsection*{\label{MORERANPLOT}\xlabel{MORERANPLOT}MORERANPLOT}

\begin{description}

\item[Description :] Replots the last {\bf ranplot} image display using a
different range value.  See command {\bf ranplot} for details of the ranplot
display method.

\item[Usage :] To repeat last {\bf ranplot} image display with new range.
\item[Associated commands :] {\tt \htmlref{ranplot}{RANPLOT}},
{\tt \htmlref{cranplot}{CRANPLOT}}
\item[Short version of command :] {\tt morer}
\item[Invocation :]

\begin{quote}{\tt  moreranplot }\\
or \\
{\tt moreranplot 200 }
\end{quote}

\begin{itemize}

\item {\tt 200 } is the new range scaling parameter for the plot
\end{itemize}

\end{description}

\hrule
\subsubsection*{\label{MOREVARGREY}\xlabel{MOREVARGREY}MOREVARGREY}

\begin{description}

\item[Description :] Replots the last {\bf vargrey} image display using a
different pair of variable index scaling parameters.  See command
{\bf vargrey} for details of the vargrey scaling method.

\item[Usage :] To repeat last {\bf vargrey} image display with new parameters.
\item[Associated commands :] {\tt \htmlref{vargrey}{VARGREY}},
{\tt \htmlref{cvargrey}{CVARGREY}}
\item[Short version of command :] {\tt morev}
\item[Invocation :]

\begin{quote}{\tt  morevargrey }\\
or \\
{\tt morevargrey 60 20 }
\end{quote}

\begin{itemize}

\item {\tt 60 } is the new percentage intensity cut value

\item {\tt 20 } is the new colour index cut value
(the bottom 60\% of the intensity range of the image  is
mapped onto the bottom 20\% of the colour indices).

\end{itemize}

\end{description}

\hrule
\subsubsection*{\label{MOS2}\xlabel{MOS2}MOS2}

\begin{description}

\item[Description :] Mosaics two images together using an interactive
selection of common points.  The two images are displayed on the
current graphics device side-by-side.  The user is asked to selected a
common feature (star) in the right-hand image and then the same feature
in the left-hand image. The peak pixel in a small box centred at those
locations is used to calculate the spatial offsets between the images.
Once this has been done, the user has the option to try and dsiplay the
mosaic of the two images, the result is displayed on the graphics
device and saved with image name mos2.  The mosaicing part of mos2 is
performed by the {\bf rapi2d} application {\bf mosaic2}.

\item[Usage :] Test mosaic two overlapping images.
\item[Associated commands :] {\tt \htmlref{mosaic2}{MOSAIC2}},
{\tt \htmlref{mosaic}{MOSAIC}}, {\tt \htmlref{quilt}{QUILT}}
\item[Short version of command :] -
\item[Invocation :]

\begin{quote}{\tt  mos2 }\\
or \\
{\tt mos2 im1 im2 }
\end{quote}

\begin{itemize}

\item {\tt im1 and im2 } are the names of the two images to be analysed
 and mosaiced together
\end{itemize}

\end{description}

\hrule
\subsubsection*{\label{NOMAG}\xlabel{NOMAG}NOMAG}

\begin{description}

\item[Description :] Command to tell the system that {\sc ircam3} does not
have an external magnifier attached to the cryostat.  This command therefore
sets the pixel scale (arcseconds/pixel) to 0.286"/pixel.

\item[Usage :] To set pixel scale to 0.286".
\item[Associated commands :] {\tt \htmlref{x2mag}{X2MAG}},
{\tt \htmlref{x5mag}{X5MAG}}
\item[Short version of command :] -
\item[Invocation :]

\begin{quote}{\tt  nomag }\end{quote}

\end{description}

\hrule
\subsubsection*{\label{NSIGMA}\xlabel{NSIGMA}NSIGMA}

\begin{description}

\item[Description :] An image display command to display on the current
graphics device a user-defined image stored in Starlink NDF format.
The scaling of the image display (sometimes referred to as the cut or
{\it max},{\it min} levels) is determined by a parameter called the
sigma level and is input by the user.  The {\it max},{\it min} levels
of the image display is determined to be n standard deviations above
and below the mean of the image; the standard deviation is that from
all the pixels in the image.  The value of n is input by the user.
Thus, a 5-sigma display using nsigma plots the image with {\it
max},{\it min} equal to m+5s,m-5s, respectively where m is the mean (or
average) of the pixels in the image and s is the standard deviation of
those pixels about the mean.  The sigma level input can by fractional
\emph{e.g.}, it could be 0.1 or 0.057 or 10 or 100 or 50.3 \emph{etc.}
If the sigma level is input to be 0 then the user is prompted for the
upper and lower sigma level to be used. This terminology means that
instead of in the above example, the {\it max},{\it min} level being 5
standard deviations both above and below the mean, the user can chose
the sigma levels independently thus, could define the {\it max},{\it
min} values to be m+10s,m-1s, respectively.  Under this version of the
command (the alternative is just called {\bf nsigma}) the position the image
is plotted on the graphics device is determined by the X,Y raster
coordinates on the current graphics device ({\bf cnsigma} plots the image at
the cursor position). The size of the image on the current graphics
device is determined by the value of the image magnification defined by
the command {\bf setmag}.  A magnification of 0 indicates that the program
auto-scales the image size to fill 80\% of the current graphics device.
Fractional magnification are allowed so that images larger than the
size of the current graphics device can be displayed.

\item[Usage :] Plots an image with nsigma scaling.

\item[Associated commands :] {\tt \htmlref{cnsigma}{CNSIGMA}},
{\tt \htmlref{moren}{MORENSIGMA}}, {\tt \htmlref{clear}{CLEAR}},
{\tt \htmlref{plot}{PLOT}}, {\tt \htmlref{cplot}{CPLOT}},
{\tt \htmlref{ranplot}{RANPLOT}}, {\tt \htmlref{cranplot}{CRANPLOT}},
{\tt \htmlref{vargrey}{VARGREY}}, {\tt \htmlref{cvargrey}{CVARGREY}},
{\tt \htmlref{setmag}{SETMAG}}, {\tt \htmlref{again}{AGAIN}},
{\tt \htmlref{setps}{SETPS}}, {\tt \htmlref{setnum}{SETNUM}},
{\tt \htmlref{setcen}{SETCEN}}, {\tt \htmlref{surround}{SURROUND}}

\item[Short version of command :] {\tt ns}
\item[Invocation :]

\begin{quote}{\tt  nsigma }\\
or \\
{\tt nsigma 42 }\\
or \\
{\tt nsigma 0 }
\end{quote}

\begin{itemize}
\item {\tt 42 } is the observation number of the image to be plotted
\item {\tt 0 } will prompt you for the full name of image to be plotted
\end{itemize}

\end{description}

\hrule
\subsubsection*{\label{OADD}\xlabel{OADD}OADD}

\begin{description}

\item[Description :] Wraparound routine to add (using {\bf rapi2d}
application {\bf add}) two images together.  The {\bf o} at the start of the
command indicates processing of two raw observation ({\sc ro}) images
referred to by their observation number rather than their full name.

\item[Usage :] Adds two observations together.

\item[Associated commands :] {\tt \htmlref{osub}{OSUB}},
{\tt \htmlref{odiv}{ODIV}}, {\tt \htmlref{omult}{OMULT}},
{\tt \htmlref{add}{ADD}}, {\tt \htmlref{sub}{SUB}},
{\tt \htmlref{div}{DIV}}, {\tt \htmlref{mult}{MULT}}

\item[Short version of command :] -
\item[Invocation :]

\begin{quote}{\tt  oadd }\\
or \\
{\tt oadd 42 43 }
\end{quote}

\begin{itemize}

\item {\tt 42 43 } are the two images referred to by their observation
 numbers to be added together
\end{itemize}

\end{description}

\hrule
\subsubsection*{\label{OCADD}\xlabel{OCADD}OCADD}

\begin{description}

\item[Description :] Wraparound routine to add (using {\bf rapi2d}
application {\bf cadd}) constant to an image. The {\bf o} at the start of the
command indicates processing of raw observation ({\sc ro}) images
referred to by their observation number rather than their full name.

\item[Usage :] Adds a constant to an observation.

\item[Associated commands :] {\tt \htmlref{ocsub}{OCSUB}},
{\tt \htmlref{ocdiv}{OCDIV}}, {\tt \htmlref{ocmult}{OCMULT}},
{\tt \htmlref{cadd}{CADD}}, {\tt \htmlref{cdiv}{CDIV}},
{\tt \htmlref{cmult}{CMULT}}

\item[Short version of command :] -
\item[Invocation :]

\begin{quote}{\tt  ocadd }\\
or \\
{\tt ocadd 42 }
\end{quote}

\begin{itemize}

\item {\tt 42 } is the image referred to by its observation number that
 is to have a constant added to it.
\end{itemize}

\end{description}

\hrule
\subsubsection*{\label{OCDIV}\xlabel{OCDIV}OCDIV}

\begin{description}

\item[Description :] Wraparound routine to divide (using {\bf rapi2d}
application {\bf cdiv}) an image by a constant. The {\bf o} at the start of the
command indicates processing of raw observation ({\sc ro}) images
referred to by their observation number rather than their full name.

\item[Usage :] Divides an observation by a constant.

\item[Associated commands :] {\tt \htmlref{ocadd}{OCADD}},
{\tt \htmlref{ocmult}{OCMULT}}, {\tt \htmlref{ocsub}{OCSUB}},
{\tt \htmlref{add}{ADD}}, {\tt \htmlref{mult}{MULT}}, {\tt \htmlref{sub}{SUB}}

\item[Short version of command :] -
\item[Invocation :]

\begin{quote}{\tt  ocdiv }\\
or \\
{\tt ocdiv 42 }
\end{quote}

\begin{itemize}

\item {\tt 42 } is the image referred to by its observation number that
is being divided by constant.
\end{itemize}

\end{description}

\hrule
\subsubsection*{\label{OCMULT}\xlabel{OCMULT}OCMULT}

\begin{description}

\item[Description :] Wraparound routine to multiple (using {\bf rapi2d}
application {\bf cmult}) an image by a constant. The {\bf o} at the
start of the command indicates processing of raw observation ({\sc ro})
images referred to by their observation number rather than their full
name.

\item[Usage :] Multiples an observation by an constant.

\item[Associated commands :] {\tt \htmlref{ocadd}{OCADD}},
{\tt \htmlref{ocdiv}{OCDIV}}, {\tt \htmlref{ocsub}{OCSUB}},
{\tt \htmlref{add}{ADD}}, {\tt \htmlref{div}{DIV}}, {\tt \htmlref{sub}{SUB}}

\item[Short version of command :] -
\item[Invocation :]

\begin{quote}{\tt  ocmult }\\
or \\
{\tt ocmult 42 }
\end{quote}

\begin{itemize}

\item {\tt 42 } is the image referred to by its observation number that
is being multiplied by a constant.
\end{itemize}

\end{description}

\hrule
\subsubsection*{\label{OCSTATS}\xlabel{OCSTATS}OCSTATS}

\begin{description}

\item[Description :] Wraparound routine to calculate statistics at a
cursor defined position in an image within a user-defined rectangular
sub-area.  Uses {\bf rapi2d} application {\bf stats}. The {\bf o} at
the start of the command indicates processing of raw observation ({\sc
ro}) images referred to by their observation number rather than their
full name.

\item[Usage :] Getting sub-area statistics from an observation.
\item[Associated commands :] {\tt \htmlref{cstats}{CSTATS}},
{\tt \htmlref{stats}{STATS}}
\item[Short version of command :] -
\item[Invocation :]

\begin{quote}{\tt  ocstats }\\
or \\
{\tt ocstats 42 }
\end{quote}

\begin{itemize}

\item {\tt 42 } is the image referred to by its observation number that
 is being analysed.
\end{itemize}

\end{description}

\hrule
\subsubsection*{\label{OCSUB}\xlabel{OCSUB}OCSUB}

\begin{description}

\item[Description :] Wraparound routine to subtract (using {\bf rapi2d}
application {\bf csub}) a constant from an images. The {\bf o} at the
start of the command indicates processing of raw observation ({\sc ro})
images referred to by their observation number rather than their full
name.

\item[Usage :] Subtracting a constant from an observation.

\item[Associated commands :] {\tt \htmlref{ocadd}{OCADD}},
{\tt \htmlref{ocdiv}{OCDIV}}, {\tt \htmlref{ocmult}{OCMULT}},
{\tt \htmlref{cadd}{CADD}}, {\tt \htmlref{cdiv}{CDIV}},
{\tt \htmlref{cmult}{CMULT}}

\item[Short version of command :] -
\item[Invocation :]

\begin{quote}{\tt  ocsub }\\
or \\
{\tt ocsub 42 }
\end{quote}

\begin{itemize}

\item {\tt 42 } is the image referred to by its observation number that
  is having a constant subtracted from it.
\end{itemize}

\end{description}

\hrule
\subsubsection*{\label{ODIST}\xlabel{ODIST}ODIST}

\begin{description}

\item[Description :] Calculates the distance and position angle between
two cursor selected points in the current image displayed on the
graphics device.  The RA and Dec of the points are calculated for the
second point from the values given for the first point by the command
{\bf setnum} (RA Dec option).

\item[Usage :] Calculating the distance between two pixels in am image.
\item[Associated commands :] {\tt \htmlref{dist}{DIST}},
{\tt \htmlref{cent2}{CENT2}}
\item[Short version of command :] -
\item[Invocation :]

\begin{quote}{\tt  odist }\end{quote}

\end{description}

\hrule
\subsubsection*{\label{ODIV}\xlabel{ODIV}ODIV}

\begin{description}

\item[Description :] Wraparound routine to divide (using {\bf rapi2d}
application {\bf div}) two images together. The {\bf o} at the start of
the command indicates processing of two raw observation ({\sc ro})
images referred to by their observation number rather than their full
name.

\item[Usage :] Dividing one observation by another.

\item[Associated commands :] {\tt \htmlref{osub}{OSUB}},
{\tt \htmlref{oadd}{OADD}}, {\tt \htmlref{omult}{OMULT}},
{\tt \htmlref{add}{ADD}}, {\tt \htmlref{sub}{SUB}},
{\tt \htmlref{div}{DIV}}, {\tt \htmlref{mult}{MULT}}

\item[Short version of command :] -
\item[Invocation :]

\begin{quote}{\tt  odiv }\\
or \\
{\tt odiv 42 43 }
\end{quote}

\begin{itemize}

\item {\tt 42 43 } are the images referred to by their observation number
 that are being divided.
\end{itemize}

\end{description}

\hrule
\subsubsection*{\label{OHISTO}\xlabel{OHISTO}OHISTO}

\begin{description}

\item[Description :] Wraparound routine to calculate statistical
information (using {\bf rapi2d} application {\bf histo}) in a sub-area of an
image. The {\bf o} at the start of the command indicates processing of raw
observation ({\sc ro}) images referred to by their observation number
rather than their full name.

\item[Usage :] To obtain sub-area statistics from observation.
\item[Associated commands :] {\tt \htmlref{histo}{HISTO}},
{\tt \htmlref{stats}{STATS}}
\item[Short version of command :] -
\item[Invocation :]

\begin{quote}{\tt  ohisto }\\
or \\
{\tt ohisto 42 }
\end{quote}

\begin{itemize}

\item {\tt 42 } is the image referred to by its observation number that
  is being analysed.
\end{itemize}

\end{description}

\hrule
\subsubsection*{\label{OLOOK}\xlabel{OLOOK}OLOOK}

\begin{description}

\item[Description :] Wraparound routine to print to screen sub-area
pixel values (using {\bf rapi2d} application {\bf look}) from an image. The
{\bf o} at the start of the command indicates processing of raw observation
({\sc ro}) images referred to by their observation number rather than
their full name.

\item[Usage :] To inspect values in sub-area of observation.
\item[Associated commands :] {\tt \htmlref{look}{LOOK}}
\item[Short version of command :] -
\item[Invocation :]

\begin{quote}{\tt  olook }\\
or \\
{\tt olook 42 }
\end{quote}

\begin{itemize}

\item {\tt 42 } is the image referred to by its observation number that
  is being analysed.
\end{itemize}

\end{description}

\hrule
\subsubsection*{\label{OMULT}\xlabel{OMULT}OMULT}

\begin{description}

\item[Description :] Wraparound routine to multiple (using {\bf rapi2d}
application {\bf mult}) two images together. The {\bf o} at the start of the
command indicates processing of two raw observation ({\sc ro}) images
referred to by their observation number rather than their full name.

\item[Usage :] Multiplies two images together.

\item[Associated commands :] {\tt \htmlref{oadd}{OADD}},
{\tt \htmlref{odiv}{ODIV}}, {\tt \htmlref{osub}{OSUB}},
{\tt \htmlref{add}{ADD}}, {\tt \htmlref{div}{DIV}}, {\tt \htmlref{sub}{SUB}}

\item[Short version of command :] -
\item[Invocation :]

\begin{quote}{\tt  omult }\\
or \\
{\tt omult 42 43 }
\end{quote}

\begin{itemize}

\item {\tt 42 43 } are the images referred to by its observation numbers
 that  are being multiplied together.
\end{itemize}

\end{description}

\hrule
\subsubsection*{\label{OSTATS}\xlabel{OSTATS}OSTATS}

\begin{description}

\item[Description :] Wraparound routine to calculate statistical
information (using {\bf rapi2d} application {\bf stats}) in a sub-area of an
image. The {\bf o} at the start of the command indicates processing of raw
observation ({\sc ro}) images referred to by their observation number
rather than their full name.

\item[Usage :] To obtain statistics on sub-area of observation.
\item[Associated commands :] {\tt \htmlref{stats}{STATS}},
{\tt \htmlref{cstats}{CSTATS}}
\item[Short version of command :] -
\item[Invocation :]

\begin{quote}{\tt  ostats }\\
or \\
{\tt ostats 42 }
\end{quote}

\begin{itemize}

\item {\tt 42 } is the image referred to by its observation number that
  is being analysed.
\end{itemize}

\end{description}

\hrule
\subsubsection*{\label{OSUB}\xlabel{OSUB}OSUB}

\begin{description}

\item[Description :] Wraparound routine to subtract (using {\bf rapi2d}
application {\bf sub}) one image from another.  The {\bf o} at the start of
the command indicates processing of two raw observation ({\sc ro})
images referred to by their observation number rather than their full
name.

\item[Usage :] Subtracts one observation image from another.

\item[Associated commands :] {\tt \htmlref{oadd}{OADD}},
{\tt \htmlref{odiv}{ODIV}}, {\tt \htmlref{omult}{OMULT}},
{\tt \htmlref{add}{ADD}}, {\tt \htmlref{sub}{SUB}},
{\tt \htmlref{div}{DIV}}, {\tt \htmlref{mult}{MULT}}

\item[Short version of command :] -
\item[Invocation :]

\begin{quote}{\tt  osub }\\
or \\
{\tt osub 42 43 }
\end{quote}

\begin{itemize}

\item {\tt 42 43 } are the images referred to by their observation
 number that  are being analysed.
\end{itemize}

\end{description}

\hrule
\subsubsection*{\label{PCLOSE}\xlabel{PCLOSE}PCLOSE}

\begin{description}

\item[Description :] Closes plotting to the current graphics device in
preparation for re-opening a different (or even the same) graphics
device.

\item[Usage :] To close plotting to current workstation/device.
\item[Associated commands :] {\tt \htmlref{popen}{POPEN}}
\item[Short version of command :] -
\item[Invocation :]

\begin{quote}{\tt  pclose }\end{quote}

\end{description}

\hrule
\subsubsection*{\label{PENCOL}\xlabel{PENCOL}PENCOL}

\begin{description}

\item[Description :] Sets the colour of one user selected colour table
pen to a user-defined colour.  Colours are referred to by letters,
WRGBYPCSN [W=white, R=red, G=blue, B=blue, Y=yellow, P=purple, C=cyan,
S=pink and N=black]).  The pen is selected by a number between the {\it
min} and {\it max} allowed for the current graphics device.

\item[Usage :] To change colour of one pen on current device.
\item[Associated commands :] {\tt \htmlref{penint}{PENINT}}
\item[Short version of command :] -
\item[Invocation :]

\begin{quote}{\tt  pencol }\\
or \\
{\tt pencol 0 B }
\end{quote}

\begin{itemize}

\item {\tt 0 } is the pen number (0=background pen) to have colour set
\item {\tt B } is the colour (blue) to set pen to
\end{itemize}

\end{description}

\hrule
\subsubsection*{\label{PENINT}\xlabel{PENINT}PENINT}

\begin{description}

\item[Description :] Sets the colour of one user selected colour table
pen to a user-defined colour.  Colours are referred to by the intensity
of the three (RGB) guns used to device colours.  The intensities are
defined to be between 0 and 1 (0=nothing, 1=full on).  The pen is
selected by a number between the {\it min} and {\it max} allowed for
the current graphics device.

\item[Usage :] To change the colour of one pen on current workstation/device.
\item[Associated commands :] {\tt \htmlref{pencol}{PENCOL}}
\item[Short version of command :] -
\item[Invocation :]

\begin{quote}{\tt  penint }\\
or \\
{\tt penint 0 1.0 0.5 0.2 }
\end{quote}

\begin{itemize}

\item {\tt 0 } is the pen number (0=background pen) to have colour set
\item {\tt 1.0 } is the intensity of the red gun (0-1)
\item {\tt 0.5 } is the intensity of the green gun (0-1)
\item {\tt 0.2 } is the intensity of the blue gun (0-1)
\end{itemize}

\end{description}

\hrule
\subsubsection*{\label{PHO}\xlabel{PHO}PHO}

\begin{description}

\item[Description :] Performs aperture photometry on an image.  {\bf
pho} prompts for an aperture size (in arcseconds) to be used and
defaults to the pre-defined size defined by the command {\bf setvar}.
The default image to be used is the last one displayed on the current
graphics device.  Other required input is the exposure time of the
image in seconds (the default is 1.0 since users will tend to work on
reduced images scaled to an equivalent exposure time of 1 sec; this is
done by {\bf darklot}, {\bf rodarklot} or {\bf stred}), and the filter
of the observations.  The filter is required since {\bf pho} uses
pre-defined zeropoints for each filter set with the command {\bf
setvar}. {\bf pho} asks you to select an location for the 'star'
aperture using the cursor and draws a circle of the defined size at
that location. It then asks you to select a 'sky' aperture again with
the cursor.  The signal in these two identically sized aperture is used
to calculate the signal from the object alone which is then converted
to a magnitude using the zeropoint.

On invocation, if the user gives {\bf pho} a command line parameter (any
parameter actually) then {\bf pho} centres the star aperture on the peak pixel
in the local neighbourhood of the cursor selected point (using the
command {\bf histo} to get the location of the peak pixel).

\item [Usage :] Aperture photometry using single aperture object,sky.

\item [Associated commands :] {\tt \htmlref{pho2}{PHO2}},
{\tt \htmlref{pho3}{PHO3}}, {\tt \htmlref{disp}{DISP}},
{\tt \htmlref{stdred}{STDRED}}

\item [Short version of command :] -

\item[Invocation :]

\begin{quote}{\tt  pho }\\
or \\
{\tt pho Y }
\end{quote}

\end{description}

\hrule
\subsubsection*{\label{PHO2}\xlabel{PHO2}PHO2}

\begin{description}

\item[Description :] Performs aperture photometry on an image.  pho2
prompts for two aperture sizes (in arcseconds) to be used and defaults
to the pre-defined sizes defined by the command setvar.  The two
aperture sizes refer to the star aperture diameter and the sky annulus
diameter that immediately surrounds the star aperture. The default
image to be used is the last one displayed on the current graphics
device.  Other required input is the exposure time of the image in
seconds (the default is 1.0 since users will tend to work on reduced
images scaled to an equivalent exposure time of 1 sec; this is done by
darklot, rodarklot or stred), and the filter of the observations.  The
filter is required since pho2 uses pre-defined zeropoints for each
filter set with the command setvar. pho2 asks you to select a location
for the 'star' aperture and 'sky' annulus using the cursor and it draws
circles of the defined sizes at that location. The signal in the star
aperture and sky annulus is used to calculate the signal from the
object alone which is then converted to a magnitude using
the zeropoint.

On invocation, if the user gives {\bf pho2} a command line parameter
(any parameter actually) then {\bf pho2} centres the star aperture on
the peak pixel in the local neighbourhood of the cursor selected point
(using the command {\bf histo} to get the location of the peak pixel).

\item [Usage :] Aperture photometry using object aperture and concentric sky
annulus.
\item [Associated commands :] {\tt \htmlref{pho}{PHO}},
{\tt \htmlref{pho3}{PHO3}}, {\tt \htmlref{disp}{DISP}},
{\tt \htmlref{stdred}{STDRED}}
\item [Short version of command :] -

\item[Invocation :]

\begin{quote}{\tt  pho2 }\\
or \\
{\tt pho2 Y }
\end{quote}

\end{description}

\hrule
\subsubsection*{\label{PHO3}\xlabel{PHO3}PHO3}

\begin{description}

\item[Description :] Performs aperture photometry on an image.  {\bf
pho3} prompts for three aperture sizes (in arcseconds) to be used and
defaults to the pre-defined sizes defined by the command {\bf setvar}.
The three aperture sizes refer to the star aperture diameter and the
inner and outer sky annulus diameter.  The default image to be used is
the last one displayed on the current graphics device.  Other required
input is the exposure time of the image in seconds (the default is 1.0
since users will tend to work on reduced images scaled to an equivalent
exposure time of 1 sec; this is done by {\bf darklot}, {\bf rodarklot}
or {\bf stred}), and the filter of the observations.  The filter is
required since {\bf pho3} uses pre-defined zeropoints for each filter
set with the command {\bf setvar}. {\bf pho3} asks you to select a
location for the 'star' aperture and 'sky' annulus using the cursor and
it draws circles of the defined sizes at that location. The signal in
the star aperture and sky annulus is used to calculate the signal from
the object alone which is then converted to a magnitude using the
zeropoint.  On invocation, if the user gives {\bf pho3} a command line
parameter (any parameter actually) then {\bf pho3} centres the star
aperture on the peak pixel in the local neighbourhood of the cursor
selected point (using the command {\bf histo} to get the location of
the peak pixel).

\item [Usage :] Aperture photometry using object aperture and
concentric sky annulus separated from object aperture.

\item [Associated commands :] {\tt \htmlref{pho}{PHO}},
{\tt \htmlref{pho2}{PHO2}}, {\tt \htmlref{disp}{DISP}},
{\tt \htmlref{stdred}{STDRED}}

\item [Short version of command :] -

\item[Invocation :]

\begin{quote}{\tt  pho3 }\\
or \\
{\tt pho3 Y }
\end{quote}

\end{description}

\hrule
\subsubsection*{\label{PICKLOT}\xlabel{PICKLOT}PICKLOT}

\begin{description}

\item[Description :] Extracts a sub-area from a series of images whose
names are defined by a prefix, a range of numbers and a suffix (see
above for the definition of the operation of the ``{\bf lot}'' family
of images) and writes the sub-area to separate independent images.
{\bf picklot} calls the {\bf rapi2d} application {\bf pickim}.  {\bf
picklot} allows to to select the centre pixel of the sub-area to be
extracted and stored separately with the cursor (on the current image
displayed) and you then select the size of the image extracted via
keyboard input of the X,Y pixel size.  Alternatively, the user can
input the X,Y start pixel for the extracted sub-area and then the X,Y
size (in pixels) of the extracted sub-area.  The output images are
stored in images of the same name as the input images but with a letter
{\tt `p'} added \emph{e.g.}, prefix={\tt im\_}, start number=50, end
number=54, suffix={\tt dfzm}, would extract a sub-area from images {\tt
im\_50dfzm}, {\tt im\_51dfzm} \emph{etc.}, and write the sub-areas to
images {\tt im\_50dfzmp}, {\tt im\_51dfzmp}, \emph{etc.}  If the user
enters a command line parameter with value Y then that is taken as
wanting to centre the sub-area on a peak pixel close to the cursor
selected point.  If a second command line parameter is entered as value
Y then {\bf picklot} can allow the selection of the centre of each
sub-area using the cursor in each image from which a sub-area is being
extracted.  If either of the above two parameters is N or undefined on
the command line then that option is turned off.

\item[Usage :] Extract an area of image to separate image.
\item[Associated commands :] {\tt \htmlref{dispick}{DISPICK}},
{\tt \htmlref{pickim}{PICKIM}}
\item[Short version of command :] -
\item[Invocation :]

\begin{quote}{\tt  picklot }\\
or \\
{\tt picklot Y Y }
\end{quote}

\begin{itemize}

\item {\tt first Y } switches on centring of sub-area on local peak pixel
\item {\tt second Y } switch on cursor selection of sub-area centre in
 each image (rather than just in first image of series).
\end{itemize}

\end{description}

\hrule
\subsubsection*{\label{PLOT}\xlabel{PLOT}PLOT}

\begin{description}

\item[Description :] Plots an image on the current graphics device
using linear {\it max},{\it min} scaling.  The plot is scaled between
the two values set using the command {\bf setmm} \emph{i.e.}, the {\it max}
and {\it min} signal level in the image that is to be mapped onto the
range of colour/greyscale pens available.  {\bf plot} image displays can be
re-plotted with different scaling parameters by either using the
command {\bf moreplot} ({\bf morep}) or {\bf setmm}/{\bf again}.

\item[Usage :] {\bf plot} is useful when making hardcopies since you can set
the minimum signal level to just below the sky and the maximum to some
appropriate value above the sky.  This produces a plot where the sky is
close to white on paper and gives good contrast (as opposed to
hardcopying {\bf nsigma} plots) of features against background.

\item[Associated commands :] {\tt \htmlref{cplot}{CPLOT}},
{\tt \htmlref{nsigma}{NSIGMA}}, {\tt \htmlref{ranplot}{RANPLOT}},
{\tt \htmlref{vargrey}{VARGREY}}, {\tt \htmlref{moreplot}{MOREPLOT}},
{\tt \htmlref{again}{AGAIN}}, {\tt \htmlref{setmm}{SETMM}}

\item[Short version of command :] {\tt pl}
\item[Invocation :]

\begin{quote}{\tt  pl }\\
or \\
{\tt pl 42 }\\
or \\
{\tt pl 0 }
\end{quote}

\begin{itemize}
\item {\tt 42} is the observation number of the raw {\sc ro} image to be
plotted
\item if {\tt 0 } is entered here then plot prompts for the full name
 of the image to be plotted
\end{itemize}

\end{description}

\hrule
\subsubsection*{\label{PLOTGLITCH}\xlabel{PLOTGLITCH}PLOTGLITCH}

\begin{description}

\item[Description :] Plots a series of images whose full image names
are input by the user, and performs interactive bad/hot pixel selection
and subsequent de-glitching on the images.  The bad/hot pixels are
identified on the image, which is displayed on the current graphics
device, by cursor selection.  The output images with bad/hot pixels
removed have the same names as the input images but with the letter {\tt `g'}
added.  The user can use the command {\bf setarea} to plot just a subsection
of the current image so that individual bad/hot pixels can be selected
unambiguously with the cursor.

\item[Usage :] Interactive bad/hot pixel removal on sequence of images.
\item[Associated commands :] {\tt \htmlref{glitch}{GLITCH}},
{\tt \htmlref{glitchmark}{GLITCHMARK}}, {\tt \htmlref{setarea}{SETAREA}}
\item[Short version of command :] -
\item[Invocation :]

\begin{quote}{\tt  plotglitch }\end{quote}

\end{description}

\hrule
\subsubsection*{\label{PLOTLOT}\xlabel{PLOTLOT}PLOTLOT}

\begin{description}

\item[Description :] Plot a series of images, whose names are defined
by a prefix, a range of numbers and a suffix. See above for the
definition of the operation of the ``{\bf lot}'' family of images.  The
images can be displayed using the nsigma or plot method and can have
one of a number of things executed on them \emph{i.e.}, cursoring,
statistics in sub-area, or aperadd aperture photometry.  By default,
the user has to hit the return key between each image in the display
sequence.  If the user puts anything as a command line parameter then
this disables the return and the images are plotted consecutively as
quickly as possible.  This is useful for a pseudo-movie display to scan
through a series of images.

\item[Usage :] Plotting a lot of images using plot {\it max},{\it min} scaling.
\item[Associated commands :] {\tt \htmlref{nsigma}{NSIGMA}},
{\tt \htmlref{plot}{PLOT}}
\item[Short version of command :] -
\item[Invocation :]

\begin{quote}{\tt  plotlot }\\
or \\
{\tt plotlot N }
\end{quote}

\begin{itemize}

\item {\tt N } is the command line parameter (this can be anything
  at all) which disables the necessity to hit return between
image displays.
\end{itemize}

\end{description}

\hrule
\subsubsection*{\label{POL2}\xlabel{POL2}POL2}

\begin{description}

\item[Description :] Aperture polarimetry command for dual-beam
polarization images taken with {\sc ircam3} and IRPOL2 using Wollaston
prism and (possibly) focal plane mask. The images required by {\bf
pol2} are the four images taken at different rotational positions of
the warm {\sc IRPOL2} half-wave plate \emph{i.e.}, at positions 0
degrees, 45 degrees, 22.5 degrees and 67.5 degrees (in that specific
order).  Two images of each object (star for example) will be present
on each image due to the splitting of the orthogonal components of
polarization by the Wollaston prism.  {\bf pol2} displays the first (0
degrees) image on the current graphics device and asks the user to
select with the cursor the southern-most of the two images of the
object (star) that he/she wants aperture polarization measurements of.
{\bf pol2} then calculates the sky subtracted signal in both orthogonal
polarization components in each of the four images at the different
waveplate positions using the {\bf obsrap} application {\bf aperphot}
and calculates the Stokes parameters Q,U,I and the percentage
polarization polarization and associated position angle (in the
instrumental coordinate system) using the {\bf polrap} application {\bf
polly2}.  The default apertures used to calculate the object signal and
associated sky contribution for each object are defined by the command
{\bf setvar} and are those used by the command {\bf disp}.  Three
apertures are defined, a 'star' aperture diameter, and the inner and
outer sky annulus diameter. The apertures are prompted for but the
defaults values are set with {\bf setvar}. If {\bf pol2} receives a
command line parameter then it will automatically optimize the aperture
position on the peak pixel in the local region around the cursor
selected pixel location.  If you are using standard J, H or K filters
or the narrow band (nb) L filter then {\bf pol2}  knows (empirically)
what the X,Y (RA,Dec) displacements are between the o-ray and e-ray
images (the two orthogonal polarization state images on each
polarization image).  If you are using any other filter then you must
enter the displacements in pixels.  It would be best to calculate these
values prior to running {\bf pol2}  using {\bf nsigma} to display the
image and {\bf cent2} to work out accurate centroided
offsets between the two images.

\item[Usage :] {\bf pol2} is useful on-line to get estimates of the level of
polarization (if any) you are measuring and its reproducibility.  On
faint sources (radio galaxies for example) 9-point jitter mosaics (with
say, 5-8 arcsecond offsets) should be taken at each of the four waveplate
position and each mosaic reduced with the command {\bf stred}.  The output from
{\bf stred} can then be used in {\bf pol2}.

\item[Associated commands :] {\tt \htmlref{setvar}{SETVAR}},
{\tt \htmlref{stred}{STRED}}
\item[Short version of command :] -
\item[Invocation :]

\begin{quote}{\tt  pol2 }\\
or \\
{\tt pol2 Y }
\end{quote}

\begin{itemize}

\item {\tt Y } (or anything else for that matter) means optimize the
 polarization aperture position on the peak pixel in the
 local region around the cursor selected pixel position.
\end{itemize}

\end{description}

\hrule
\subsubsection*{\label{POLREG}\xlabel{POLREG}POLREG}

\begin{description}

\item[Description :] Registers and trims (removes undefined edges after
shifting) a set of four polarization images taken at the four waveplate
positions for a polarization measurement \emph{e.g.}, 0 degs, 45 degs,
22.5 degs, and 67.5 degs.  {\bf polreg} asks for the names of the four
images, displays the first on the current graphics device, asks the
user to select via cursor input the common feature (star) in all four
images, calculates accurate (centroid) position of the source in the
four images (using the {\bf rapi2d} application {\bf centroid}), shifts
the images to the average of the four positions (using the {\bf rapi2d}
application {\bf shift}) and trims the undefined edges left after the
shifting (using the {\bf rapi2d} application {\bf pickim}).  The output
images are generally of a smaller size then the input images due to the
trimming of the undefined edges after the shift.  The output image
names are the same as the input image names with the letter {\tt `r'}
added to the end, thus, if the input images were called {\tt pol0},
{\tt pol45}, {\tt pol22} and {\tt pol67} then the output registered
images would be called {\tt pol0r}, {\tt pol45r}, {\tt pol22r and
pol67r}.

\item[Usage :] Registering spatially polarization images at 4 waveplate
positions.
\item[Associated commands :] {\tt \htmlref{centroid}{CENTROID}},
{\tt \htmlref{shift}{SHIFT}}, {\tt \htmlref{pickim}{PICKIM}}
\item[Short version of command :] -
\item[Invocation :]

\begin{quote}{\tt  polreg }\end{quote}

\end{description}

\hrule
\subsubsection*{\label{POPEN}\xlabel{POPEN}POPEN}

\begin{description}

\item[Description :] Opens plotting to a user-defined graphics device.
The devices are referred to by an integer number given in the list
presented to the user upon invocation of the command {\bf popen}.  The most
common device is an X-windows plotting window called {\tt GKS\_3800}.
If the X-windows window exists (say, created from Unix via the {\sc
IrcamDR} commands {\bf ixcs}, {\bf ixcm} and {\bf ixcl} for small,
medium and large plotting windows, respectively) then that window is
used by {\sc IrcamDR}/{\bf plt2d} (the graphics tasks within {\sc
IrcamDR}).  If the X-windows window does not exist then it is created
for you.  Plotting can be closed using the command {\bf pclose}.  This
is needed when switching between output graphics devices.  The command
{\bf hardcopy} does this closing and opening for for you (the hardcopy
device is defined by the command {\bf sethard}).

\item[Usage :] Opening plotting to a new workstation/device.
\item[Associated commands :] {\tt \htmlref{pclose}{PCLOSE}},
{\tt \htmlref{sethard}{SETHARD}}, {\tt ixcs}, {\tt ixcm}, {\tt ixcl}
(last three from Unix).
\item[Short version of command :] -
\item[Invocation :]

\begin{quote}{\tt  popen }\\
or \\
{\tt popen 1 }
\end{quote}

\begin{itemize}

\item {\tt 1 } is the graphics device id number, 1  is the X-windows
 base window
\end{itemize}

\end{description}

\hrule
\subsubsection*{\label{RANPLOT}\xlabel{RANPLOT}RANPLOT}

\begin{description}

\item[Description :] Plots an image using a range scaling value input
by the user.  The scaling {\it max},{\it min} values are determined by
the program to be {\it +max} and {\it -min} the range value on
the mean of the image \emph{i.e.}, {\it max} = {\it mean+range}, {\it min} =
{\it mean-range}. The image centre is defined by pixel input and defaults to
the centre of the graphics device. The size of the image on the current
graphics device is determined by the value of the image magnification
defined by the command {\bf setmag}.  A magnification of 0 indicates that the
program auto-scales the image size to fill 80\% of the current graphics
device.  Fractional magnification are allowed so that images larger
than the size of the current graphics device can be displayed.

\item[Usage :] Plotting an image with range scaling on mean signal.

\item[Associated commands :] {\tt \htmlref{cranplot}{CRANPLOT}},
{\tt \htmlref{morer}{MORERANPLOT}}, {\tt \htmlref{clear}{CLEAR}},
{\tt \htmlref{nsigma}{NSIGMA}}, {\tt \htmlref{plot}{PLOT}},
{\tt \htmlref{vargrey}{VARGREY}}, {\tt \htmlref{setmag}{SETMAG}},
{\tt \htmlref{setcen}{SETCEN}}, {\tt \htmlref{again}{AGAIN}},
{\tt \htmlref{setps}{SETPS}}, {\tt \htmlref{setnum}{SETNUM}},
{\tt \htmlref{surround}{SURROUND}}

\item[Short version of command :] {\tt ran}
\item[Invocation :]

\begin{quote}{\tt  ranplot }\\
or \\
{\tt ranplot 42 }\\
or \\
{\tt ranplot 0 }
\end{quote}

\begin{itemize}
\item {\tt 42 } is the observation number of the image to be plotted
\item {\tt 0} will prompt you for the full name of image to be plotted
\end{itemize}

\end{description}

\hrule
\subsubsection*{\label{REMOS}\xlabel{REMOS}REMOS}

\begin{description}

\item[Description :] Re-mosaics together a series of tile images that
form a larger mosaic image.  The images to be mosaiced together would
generally be reduced images \emph{e.g.}, dark subtracted, flat-fielded
\emph{etc.}, the output from the command {\bf stred} for example.  The
image names of the images to be -remosaiced together are defined by a
name prefix, a range of observation numbers and a name suffix as in the
``{\bf lot}'' family of images (see above).  {\bf remos} takes the
input images and re-scales them, via dc subtraction, to have matching
sky (dc) levels in overlap regions (using the {\bf obsrap} application
{\bf automos}).  {\bf remos} then applies the bad pixel mask to the
images (using the {\bf obsrap} application {\bf applymask}) to mask off
bad/hot pixels and then it re-mosaics the images together using the
{\bf rapi2d} application {\bf quilt}.

\item[Usage :] For refining spatial offsets between tiles of a
mosaic using {\bf jitreg}, {\bf accoff} or {\bf coff} for example.  Use one
of these and then use the output file created to re-mosaic the tiles
using remos.

\item[Associated commands :] {\tt \htmlref{stred}{STRED}},
{\tt \htmlref{jitreg}{JITREG}}, {\tt \htmlref{accoff}{ACCOFF}},
{\tt \htmlref{coff}{COFF}}, {\tt \htmlref{quilt}{QUILT}}

\item[Short version of command :] -
\item[Invocation :]

\begin{quote}{\tt  remos }\end{quote}

\end{description}

\hrule
\subsubsection*{\label{RFLOT}\xlabel{RFLOT}RFLOT}

\begin{description}

\item[Description :] Rotates and flips a series of images whose names
are defined by a prefix, a range of numbers and a suffix. See above for
the definition of the operation of the ``{\bf lot}'' family of images.
The rotation is defined in degrees clockwise and the flip is either
about a horizontal axis (\emph{i.e.}, north-south) or about a vertical
axis (\emph{i.e.}, east-west).  The rotation is performed using the
{\bf rapi2d} application {\bf rotate}.  The flip is performed using the
{\bf rapi2d} application {\bf flip}.  The output images are of the same
name as the input images but with the letters {\tt `rf'} added to them,
thus {\tt im\_50dfzam} would be {\tt im\_50dfzmrf} after rotation and
flipping.

\item[Usage :] To change orientation of an image (say, North-East to top-left).
\item[Associated commands :] {\tt \htmlref{rotate}{ROTATE}},
{\tt \htmlref{flip}{FLIP}}
\item[Short version of command :] -
\item[Invocation :]

\begin{quote}{\tt  rflot }\end{quote}

\end{description}

\hrule
\subsubsection*{\label{ROCENT}\xlabel{ROCENT}ROCENT}

\begin{description}

\item[Description :] Calculates accurate offsets (using {\bf rapi2d}
application {\bf centroid}) between the same feature (star) in a series
of images specified with their full name.  {\bf rocent} plots the
images one-by-one and asks the user to select with the cursor the
common feature (star) and it then calculates a centroid of those
features and writes the results pixel to the screen and an ASCII file
called {\tt rocent.dat} (results in terms of accurate pixel positions
for the features). {\bf rocent} also writes an ASCII file called {\tt
rocent.off} which contains accurate spatial arcsecond offsets of the
features from the location of the feature in the first image in the
series.  Additionally, {\bf rocent} creates an ASCII file called {\tt
rocent.im} which contains a list of the images entered into in {\bf
rocent} for centroiding.  This can be used to re-mosaic the series of
images together, for example.

\item[Usage :] Determining the motion of an object
over a series of images (\emph{i.e.}, due to telescope tracking
problems or oscillations \emph{etc}).  Also useful to create a spatial
offset file (in arcseconds) for re-mosaicing a series of tiles of a
mosaic together using {\bf remos} or {\bf quilt} or {\bf mosaic}
\emph{etc.}  The files {\tt rocent.im} and {\tt rocent.off} can be fed
into {\bf crequilt} which creates a quilt style ASCII file which can be
fed into {\bf quilt} for mosaicing images together.

\item[Associated commands :] {\tt \htmlref{centroid}{CENTROID}},
{\tt \htmlref{remos}{REMOS}}, {\tt \htmlref{quilt}{QUILT}},
{\tt \htmlref{crequilt}{CREQUILT}}

\item[Short version of command :] -
\item[Invocation :]

\begin{quote}{\tt  rocent }\end{quote}

\end{description}

\hrule
\subsubsection*{\label{RODARKLOT}\xlabel{RODARKLOT}RODARKLOT}

\begin{description}

\item[Description :] Dark subtracts a series of raw {\sc ircam3} format
individual images with names viz:  {\tt ro950412\_12}, {\tt
ro940412\_13} \emph{etc.} Description of the ``{\bf lot}'' series of
procedures that work via a name prefix, a range of observation numbers
and a name suffix is given above.  {\bf rodarklot} takes a range of
observation numbers and the observation number of an associated dark
exposure and subtracts the dark current image from the observations and
scales the result to DN/sec/coadd reading the on-chip exposure time
from the observation header.  The {\sc ro} images are pre-scaled in the
ALICE Transputer system to represent 1 coadd.  The output dark
subtracted files are named via a output filename prefix, the range of
observation numbers input and the suffix letter {\tt `d'}.  For
example, dark subtracting observations 20-25 from the current {\sc
ircam3} image name format (defined by the command {\bf setfile}), with
an output filename prefix of {\tt im\_} would result in dark subtracted
images with names {\tt im\_20d}, {\tt im\_21d},\ldots {\tt im\_25d}.

\item[Usage :] Dark subtracting a series of {\sc ircam3} raw ({\sc ro})
images using a specified dark image.  To dark subtract a series of {\sc
ircam1/2} container file images use the command {\bf darklot}.

\item[Associated commands :] {\tt \htmlref{darklot}{DARKLOT}},
{\tt \htmlref{setfile}{SETFILE}}
\item[Short version of command :] -
\item[Invocation :]

\begin{quote}{\tt  rodarklot }\end{quote}

\end{description}

\hrule
\subsubsection*{\label{ROINDEX}\xlabel{ROINDEX}ROINDEX}

\begin{description}

\item[Description :] Produces an ASCII file containing the basic
observational parameters of a series of raw ({\sc ro}) {\sc ircam3}
data images.  The user entered the range of observations numbers to be
scanned and {\bf roindex} extracts the observational parameters from
the {\sc ro} files and write the results to the file.  Once complete,
the ASCII file is typed to the screen.  The UT date of the {\sc ro}
files used in {\bf roindex} (and in general) is set using the command
{\bf setfile}.

\item[Usage :] Producing a list of observational parameters from {\sc ro} image
range.

\item[Associated commands :] {\tt \htmlref{ropar}{ROPAR}},
{\tt \htmlref{rout}{ROUT}}, {\tt \htmlref{getoff}{GETOFF}},
{\tt \htmlref{setfile}{SETFILE}}

\item[Short version of command :] -
\item[Invocation :]

\begin{quote}{\tt  roindex }\\
or \\
{\tt roindex 24 37 }
\end{quote}

\begin{itemize}

\item {\tt 24 37 } are the start and end observation numbers to be
 indexed
\end{itemize}

\end{description}

\hrule
\subsubsection*{\label{ROMED}\xlabel{ROMED}ROMED}

\begin{description}

\item[Description :] Median filters a series of raw ({\sc ro}) {\sc
ircam3} images.  The images are defined by a range of observation
numbers and a appropriate dark is also required.  The dark is first
subtracted from the raw {\sc ro} images before they are median
filtered.  The resultant median filtered image is named by the user and
is itself normalized to a median (over the whole area of the image) of
unity.

\item[Usage :] Creating a separate
sky flat-field image from a set of sky jitter mosaic observations.  The
median filtered output image can be used in such commands as {\bf
stred} instead of {\bf stred} median filtering the object images and
using that as a flat-field.

\item[Associated commands :] {\tt \htmlref{rodarklot}{RODARKLOT}},
{\tt \htmlref{med3d}{MED3D}}, {\tt \htmlref{setfile}{SETFILE}},
{\tt \htmlref{stred}{STRED}}

\item[Short version of command :] -
\item[Invocation :]

\begin{quote}{\tt  romed }\\
or \\
{\tt romed 24 5 23 skyflat }
\end{quote}

\begin{itemize}

\item {\tt 24 } is the start observation number of the series to be
 median filtered
\item {\tt 5 } is the number of images in the series to be median
 filtered
\item {\tt 23 } is the observation number of the dark to be used
\item {\tt skyflat } is the name of the output normalized median
 filtered image
\end{itemize}

\end{description}

\hrule
\subsubsection*{\label{ROPAR}\xlabel{ROPAR}ROPAR}

\begin{description}

\item[Description :] Extracts the basic observational parameters from
one raw ({\sc ro}) {\sc ircam3} observation file.  The values used to
acquire that observation are written to the screen only.  The
observation is defined by its observation number and the UT date set
with the command {\bf setfile}.

\item[Usage :] To check the parameters used for a specific observation.

\item[Associated commands :] {\tt \htmlref{roindex}{ROINDEX}},
{\tt \htmlref{rout}{ROUT}}, {\tt \htmlref{getoff}{GETOFF}},
{\tt \htmlref{setfile}{SETFILE}}

\item[Short version of command :] -
\item[Invocation :]

\begin{quote}{\tt  ropar }\\
or \\
{\tt ropar 42 }
\end{quote}

\begin{itemize}

\item {\tt 42 } is the observation number of the image to be accessed
\end{itemize}

\end{description}

\hrule
\subsubsection*{\label{ROTLOT}\xlabel{ROTLOT}ROTLOT}

\begin{description}

\item[Description :] Rotates a series of images by an arbitrary
amount.  The series of images are defined by a prefix, a range of
numbers and a suffix. See above for the definition of the operation of
the ``{\bf lot}'' family of images.  The rotation is performed by the
{\bf rapi2d} application {\bf rotate} and the value is entered in
degrees clockwise.  The output images have the same names as the input
images but with the letter {\tt `r'} added to them.

\item[Usage :] Rotating a sequence of images.

\item[Associated commands :] {\tt \htmlref{rotate}{ROTATE}},
{\tt \htmlref{rflot}{RFLOT}}

\item[Short version of command :] -
\item[Invocation :]

\begin{quote}{\tt  rotlot }\end{quote}

\end{description}

\hrule
\subsubsection*{\label{ROUT}\xlabel{ROUT}ROUT}

\begin{description}

\item[Description :] Extracts the UT time information from a series of
raw ({\sc ro}) {\sc ircam3} (or old format {\sc ircam1/2}) data images.
The UT time is written to an ASCII file called {\tt rout.dat}.  The raw
images to be processed are defined by an observation number range and
the UT date defined by the command {\bf setfile}.

\item[Usage :] Obtaining UT time on a sequence of {\sc ro} images.

\item[Associated commands :] {\tt \htmlref{ropar}{ROPAR}},
{\tt \htmlref{roindex}{ROINDEX}}, {\tt \htmlref{getoff}{GETOFF}},
{\tt \htmlref{setfile}{SETFILE}}

\item[Short version of command :] -
\item[Invocation :]

\begin{quote}{\tt  rout }\\
or \\
{\tt rout 40 42 }
\end{quote}

\begin{itemize}

\item {\tt 40 42 } are the start and end observation number for
 extraction of the UT time information
\end{itemize}

\end{description}

\hrule
\subsubsection*{\label{SCALEDARK}\xlabel{SCALEDARK}SCALEDARK}

\begin{description}

\item[Description :] Scales dark exposures from {\sc ircam1/2} old
style container files. {\bf scaledark} scales the exposure time from one
value (the one used to take the dark) to another value (generally one
close to the one used to take the dark).

\item[Usage :] Useful if you forgot to take a dark of a specific exposure
time but you have one close to it in exposure length.
\item[Associated commands :] -
\item[Short version of command :] -
\item[Invocation :]

\begin{quote}{\tt  scaledark }\end{quote}

\end{description}

\hrule
\subsubsection*{\label{SCALELOT}\xlabel{SCALELOT}SCALELOT}

\begin{description}

\item[Description :] Scales a series of images, whose names are defined by a
prefix, a range of numbers and a suffix. See above for the definition
of the operation of the ``{\bf lot}'' family of images.  The required
input is the exposure time in seconds and the number of coadds (always
1 for {\sc ircam3} data).  {\bf scalelot} scales the images by division
to unit exposure time and unit coadds using the {\bf rapi2d}
application {\bf cdiv}.

\item[Usage :] Scaling a sequence of images.

\item[Associated commands :] {\tt \htmlref{scaledark}{SCALEDARK}},
{\tt \htmlref{cdiv}{CDIV}}

\item[Short version of command :] -
\item[Invocation :]

\begin{quote}{\tt  scalelot }\end{quote}

\end{description}

\hrule
\subsubsection*{\label{SEE}\xlabel{SEE}SEE}

\begin{description}

\item[Description :] Measures the FWHM of a stellar image
using the KAPPA application {\bf psf}.  {\bf see} will take an
observation number as input (and then uses the appropriate raw ({\sc
ro}) image (using the UT date set by setfile), it will also take the
full image name if the user wants to work on processed images.  The
image input by the user is displayed on the current graphics device and
the cursor is then displayed.  The user has to select N stars in the
image ({\bf psf} averages the seeing profile over a number of images)
and when the stars have been selected the user must click off the image
(but on the background to the graphics window) to terminate star
selection.  The star positions are fed into {\bf psf} which fits the
profiles with a 2D gaussian profile.  {\bf psf} (hence the command {\bf see})
returns the mean axis ratio and axis orientation of the star (since the
images may not be circular) together with the FWHM of the profile (in
pixels and arcseconds) and the gamma of the profile. The conversion
from FWHM in pixels to arcseconds is done using the value of
arcseconds/pixels set using the command {\bf setps}.  {\bf psf} also plots
the radial profile from the stellar image together with the best fit
profile found.  This plotting is done by default to the X-windows
window called {\tt GKS\_3800}.

\item[Usage :] Calculating FWHM of stellar images in arcseconds.

\item[Associated commands :] {\tt \xref{psf}{sun95}{PSF}},
{\tt \htmlref{setps}{SETPS}}

\item[Short version of command :] -
\item[Invocation :]

\begin{quote}{\tt  see }\\
or \\
{\tt see 42 }\\
or \\
{\tt see 0 }
\end{quote}

\begin{itemize}
\item {\tt 42 } is the observation number of the raw image to be analysed
\item {\tt 0 } means the user wants to enter full name of the image to
 be analysed.
\end{itemize}

\end{description}

\hrule
\subsubsection*{\label{SETAREA}\xlabel{SETAREA}SETAREA}

\begin{description}

\item[Description :] Defines the area (or sub-area) of an image
displayed on the current graphics device when the commands {\bf
nsigma}, {\bf plot}, {\bf ranplot}, {\bf vargrey} (and their associated
cursor commands \emph{e.g.} {\bf cnsigma}) are invoked.  The user can
tell the software to just display a sub-image from the image give
rather than displaying the whole image as is the default.  The
sub-image is defined by an X,Y start pixel and a sub-area size (in
pixels).  To switch off sub-area display the user must run {\bf
setarea} again and select the full image option.

\item[Usage :] Setting area of an image to be plotted with {\bf nsigma},
{\bf plot}, {\bf ranplot}, {\bf vargrey} \emph{etc.}, image display commands.

\item[Associated commands :] {\tt \htmlref{nsigma}{NSIGMA}},
{\tt \htmlref{plot}{PLOT}}, {\tt \htmlref{ranplot}{RANPLOT}},
{\tt \htmlref{vargrey}{VARGREY}}

\item[Short version of command :] -
\item[Invocation :]

\begin{quote}{\tt  setarea }\end{quote}

\end{description}

\hrule
\subsubsection*{\label{SETCEN}\xlabel{SETCEN}SETCEN}

\begin{description}

\item[Description :] Defines the position of the centre of an image on
the current graphics device when it is displayed using any of the {\sc
IrcamDR} image display commands \emph{e.g.}, {\bf nsigma}, {\bf plot},
{\bf ranplot}, {\bf vargrey} \emph{etc.}  The centre of the image is
defined in terms of the device raster (pixel) units, these are the
physical size of the graphics device in device pixels and are returned
from within {\sc IrcamDR} by the command {\bf getrast}.  Once execute,
subsequent images will be displayed centred on the input location.
Note that auto-sizing image using a magnification of 0 will likely not
work well here so users should use the command {\bf setmag} to define
an appropriate image magnification for images plotted at a location
other than on the centre of the array.

\item[Usage :] To plot several images on one display device.
Users should use the command {\bf getrast} to determine the raster (pixel)
size of the current graphics display device and then use {\bf setcen} (or
{\bf setquad}) to define a new centre location for subsequent image display.

\item[Associated commands :] {\tt \htmlref{setmag}{SETMAG}},
{\tt \htmlref{nsigma}{NSIGMA}}, {\tt \htmlref{plot}{PLOT}},
{\tt \htmlref{ranplot}{RANPLOT}}, {\tt \htmlref{vargrey}{VARGREY}},
{\tt \htmlref{setquad}{SETQUAD}}

\item[Short version of command :] -
\item[Invocation :]

\begin{quote}{\tt  setcen }\\
or \\
{\tt setcen 128 192 }
\end{quote}

\begin{itemize}

\item {\tt 128 192 } are the X,Y raster unit pixels for the centre of
 subsequent image displays
\end{itemize}

\end{description}

\hrule
\subsubsection*{\label{SETCOMORI}\xlabel{SETCOMORI}SETCOMORI}

\begin{description}

\item[Description :] Defines the orientation of subsequent comment
strings written to the current graphics device using the commands {\bf
wrcom} and {\bf wrccom}.  The orientation defines whether, for example,
text is written horizontally or vertically and starting at the bottom
writing upwards downwards \emph{etc.} Horizontal-right is the default
and produces normal direction and orientation text.

\item[Usage :] Defining orientation of text written with {\bf wrcom} and
{\bf wrccom}.

\item[Associated commands :] {\tt \htmlref{wrcom}{WRCOM}},
{\tt \htmlref{wrccom}{WRCCOM}}

\item[Short version of command :] -
\item[Invocation :]

\begin{quote}{\tt  setcomori }\\
or \\
{\tt setcomori 3 }
\end{quote}

\begin{itemize}

\item 3 makes subsequent comment strings write vertically from
 the right
\item {\tt 1-4 } are horizontal-right, horizontal-left, vertical-right
 and vertical-left
\end{itemize}

\end{description}

\hrule
\subsubsection*{\label{SETCONT}\xlabel{SETCONT}SETCONT}

\begin{description}

\item[Description :] Defines the parameters associated with a contour
line plot executed by the command {\bf contour}.  The parameters
definable with {\bf setcont} are the contour level definition method,
the contour number, the contour base and increment, the contour
colours, the annotation plotted, the tick mark start and interval
(calling command {\bf setnum}) and the position of the contour map on
the current graphics device.  {\bf setcont} has to be executed once
before a contour map can be plotted.  The line thickness use in the
contour map is defined by the command {\bf line\_width}.

\item[Usage :] Setting contour levels to be plotted (amongst other
things) using command {\bf contour}.

\item[Associated commands :] {\tt \htmlref{contour}{CONTOUR}},
{\tt \htmlref{line\_width}{LINE_WIDTH}}, {\tt \htmlref{setnum}{SETNUM}},
{\tt \htmlref{setps}{SETPS}}, {\tt \htmlref{cont\_title}{CONT_TITLE}},
{\tt \htmlref{contoff}{CONTOFF}}

\item[Short version of command :] -
\item[Invocation :]

\begin{quote}{\tt  setcont }\end{quote}

\end{description}

\hrule
\subsubsection*{\label{SETCONTIC}\xlabel{SETCONTIC}SETCONTIC}

\begin{description}

\item[Description :] Defines whether small or large tick marks are
plotted on contour maps and annotation produced by the command {\bf
surround}.  The tick marks are the fiducial marks along both X and Y
axes of the image (RA and Dec) and small tick marks are just of small
length while large tick marks form a complete grid across the image.
The width of the line used is set with the command {\bf line\_width}.
Tick marks are defined using the command {\bf setnum}.

\item[Usage :] Defining contour axis tick mark length.

\item[Associated commands :] {\tt \htmlref{setnum}{SETNUM}},
{\tt \htmlref{line\_width}{LINE_WIDTH}}, {\tt \htmlref{contour}{CONTOUR}},
{\tt \htmlref{surround}{SURROUND}}, {\tt \htmlref{ticklen}{TICKLEN}}

\item[Short version of command :] -
\item[Invocation :]

\begin{quote}{\tt  setcontic }\\
or \\
{\tt setcontic s }
\end{quote}

\begin{itemize}
\item {\tt s } means plot small tick marks
\item {\tt f } would mean plot full - grid - tick marks
\end{itemize}

\end{description}

\hrule
\subsubsection*{\label{SETCUR}\xlabel{SETCUR}SETCUR}

\begin{description}

\item[Description :] Defines where cursor selection of an image
position in {\bf cnsigma}, {\bf cplot}, {\bf cranplot} and {\bf
cvargrey} refers to \emph{i.e.}, the image centre at the cursor
selected point or the bottom-left or top-right corner of the image at
the cursor selected point.

\item[Usage :] Positioning several images on one display
area with a small magnification \emph{i.e.}, not auto-scaling 0.

\item[Associated commands :] -
\item[Short version of command :] -
\item[Invocation :]

\begin{quote}{\tt  setcur }\end{quote}

\end{description}

\hrule
\subsubsection*{\label{SETCURMARK}\xlabel{SETCURMARK}SETCURMARK}

\begin{description}

\item[Description :] Defines whether a small cross is plotted when a
point/pixel is selected with the cursor under commands like {\bf cursor},
{\bf mcursor}, {\bf glitchmark}, {\bf ccircle}, {\bf cbox} \emph{etc.}
The user can switch on and off the plotting of the small cross marking
the cursor position.

\item[Usage :] Marking of cursor selected points in image.

\item[Associated commands :] {\tt \htmlref{cursor}{CURSOR}},
{\tt \htmlref{mcursor}{MCURSOR}}, {\tt \htmlref{glitchmark}{GLITCHMARK}},
{\tt \htmlref{cbox}{CBOX}}, {\tt \htmlref{ccircle}{CCIRCLE}},
{\tt \htmlref{ccross}{CCROSS}}, \\ {\tt \htmlref{cellipse}{CELLIPSE}},
{\tt \htmlref{cline}{CLINE}}

\item[Short version of command :] -
\item[Invocation :]

\begin{quote}{\tt  setcurmark }\\
or \\
{\tt setcurmark 2 }
\end{quote}

\begin{itemize}

\item {\tt 2 } means do not plot a cross at cursor pixel
\item {\tt 1 } would mean do plot a cross at cursor pixel)
\end{itemize}

\end{description}

\hrule
\subsubsection*{\label{SETCUT}\xlabel{SETCUT}SETCUT}

\begin{description}

\item[Description :] Defines the parameters associated with a {\bf cut}
(or {\bf ccut}) through an image to produce a line graph of intensity
vs. position.  Parameters that can be defined include the cut line or
marker type, the intensity axis scaling method, the annotation plotted,
and the size and position of the plot on the current graphics device.
{\bf setcut} has to be executed once before a {\bf cut} or {\bf ccut}
can be produced.  The line thickness of the cut is set with the command
{\bf line\_width}.

\item[Usage :] Defining how a line cut through an image is plotted.

\item[Associated commands :] {\tt \htmlref{cut}{CUT}},
{\tt \htmlref{ccut}{CCUT}}, {\tt \htmlref{line\_width}{LINE_WIDTH}}

\item[Short version of command :] -
\item[Invocation :]

\begin{quote}{\tt  setcut }\end{quote}

\end{description}

\hrule
\subsubsection*{\label{SETCUTAXRAT}\xlabel{SETCUTAXRAT}SETCUTAXRAT}

\begin{description}

\item[Description :] Defines the axis ratio Y/X of cut plots produced
by the commands {\bf cut} and {\bf ccut}.  A value of {\tt 1} means the
cuts will be plotted in a square box, values \verb+<1+ will give a
longer X axis, values \verb+>1+  will give a longer Y axis.

\item[Usage :] Defining cut axis ratio on current workstation/device.

\item[Associated commands :] {\tt \htmlref{setcut}{SETCUT}},
{\tt \htmlref{cut}{CUT}}, {\tt \htmlref{ccut}{CCUT}},
{\tt \htmlref{line\_width}{LINE_WIDTH}}

\item[Short version of command :] -
\item[Invocation :]

\begin{quote}{\tt  setcutaxrat }\\
or \\
{\tt setcutaxrat 2.5 }
\end{quote}

\begin{itemize}

\item {\tt 2.5 } is the new cut axis ratio y/x value
\end{itemize}

\end{description}

\hrule
\subsubsection*{\label{SETFILE}\xlabel{SETFILE}SETFILE}

\begin{description}

\item[Description :] Defines what type of {\sc ircam} data files are to be
processed and analysed, old type {\sc ircam1/2} container files or new
type {\sc ircam3} {\sc ro} files.  {\bf setfile} also defines the {\sc
ircam1/2} container filename or the UT date for the {\sc ro} filename
prefix.  Once {\bf setfile} has been executed (it is run automatically
on startup of {\sc IrcamDR} but it can also be run any time from within
{\sc IrcamDR} to change the current settings), users can refer to raw
images by their observation number alone \emph{e.g.}, {\tt setfile n
950714} would allow you to plot observation number {\tt 42} in image
{\tt ro950714\_42.sdf} by the command {\tt ns 42}.

\item[Usage :] Selecting raw ({\sc ro}) image UT date or old container file
prefix.
\item[Associated commands :] -
\item[Short version of command :] -
\item[Invocation :]

\begin{quote}{\tt  setfile }\\
or \\
{\tt setfile n 950714 }
\end{quote}

\begin{itemize}

\item {\tt n } means use new {\sc ircam3} type image names
\item {\tt o } would mean use old {\sc ircam1/2} type images names)
\item {\tt 950714 } is the UT date for the new {\sc ircam3} type raw {\sc ro} images
\end{itemize}

\end{description}

\hrule
\subsubsection*{\label{SETFONT}\xlabel{SETFONT}SETFONT}

\begin{description}

\item[Description :] Selects which font is used when plotting text on
the current graphics device.  The font is selected by the associated
GKS font number.  Font number 106 is a good quality fancy font.

Please note that on Unix systems, the font numbers are negative; for the
fancy font described above, the command is {\tt setfont -106}.

\item[Usage :] Defining font for text on images.

\item[Associated commands :] {\tt \htmlref{wrcom}{WRCOM}},
{\tt \htmlref{wrccom}{WRCCOM}}

\item[Short version of command :] -
\item[Invocation :]

\begin{quote}{\tt  setfont }\end{quote}

\end{description}

\hrule
\subsubsection*{\label{SETHARD}\xlabel{SETHARD}SETHARD}

\begin{description}

\item[Description :] Defines the hardcopy device used to produce
hardcopies of images and line graphics using the command {\bf hardcopy}.  The
hardcopy devices are referred to by an integer number and are (these
days) mostly Postscript devices (normal Postscript, Encapsulated
Postscript, colour Postscript \emph{etc}).  This command should be
executed once prior to any hardcopy production using {\sc IrcamDR}.
The {\sc IrcamDR} software defaults to Postscript (portrait) as the
hardcopy output device if this command has not been executed.  The
Postscript output is written to ASCII files on disk and under Unix
these files are called {\tt fort.1}, {\tt fort.1.1}, {\tt fort.1.2}
\emph{etc}, for the first and subsequent Postscript output files
created.  The user has to do the printing of these files to their own
Postscript printer manually.

\item[Usage :] Selecting hardcopy device to use in hardcopy command.
\item[Associated commands :] {\tt \htmlref{hardcopy}{HARDCOPY}}
\item[Short version of command :] -
\item[Invocation :]

\begin{quote}{\tt  sethard }\\
or \\
{\tt sethard 14 }
\end{quote}

\begin{itemize}

\item {\tt 14 } is output to a Postscript (portrait mode) file
\end{itemize}

\end{description}

\hrule
\subsubsection*{\label{SETMAG}\xlabel{SETMAG}SETMAG}

\begin{description}

\item[Description :] Sets the magnification used to display images on
the current graphics device.  The magnification is a real number and
refers to the size of the image on the graphics device with respect to
the window area available.  A magnification of 0 tells the software to
auto-scale the image to fill 80\% of the available display area.  This
is the default.

\item[Usage :] Defining image magnification for image display.

\item[Associated commands :] {\tt \htmlref{nsigma}{NSIGMA}},
{\tt \htmlref{plot}{PLOT}}, {\tt \htmlref{ranplot}{RANPLOT}},
{\tt \htmlref{vargrey}{VARGREY}}

\item[Short version of command :] -
\item[Invocation :]

\begin{quote}{\tt  setmag }\\
or \\
{\tt setmag 1 }
\end{quote}

\begin{itemize}

\item {\tt 1 } is the new image scaling magnification factor
\item {\tt 0 } would mean auto-scale to fill 80\% of the display  area
\end{itemize}

\end{description}

\hrule
\subsubsection*{\label{SETMAX}\xlabel{SETMAX}SETMAX}

\begin{description}

\item[Description :] Sets the maximum intensity value mapped onto the
maximum colour index (pen) on the current graphics device under the
image display command {\bf plot}.  The minimum intensity value is set
with the command {\bf setmin}.  Both these parameters can be set
together with the command {\bf setmm}.

\item[Usage :] Defining maximum intensity in {\bf plot} image display.

\item[Associated commands :] {\tt \htmlref{setmin}{SETMIN}},
{\tt \htmlref{setmm}{SETMM}}, {\tt \htmlref{plot}{PLOT}}

\item[Short version of command :] -
\item[Invocation :]

\begin{quote}{\tt  setmax }\\
or \\
{\tt setmax 1023 }
\end{quote}

\begin{itemize}

\item {\tt 1023 } is the intensity level mapped onto the maximum colour
 index for the current graphics display device
\end{itemize}

\end{description}

\hrule
\subsubsection*{\label{SETMIN}\xlabel{SETMIN}SETMIN}

\begin{description}

\item[Description :] Sets the minimum intensity value mapped onto the
minimum colour index (pen) on the current graphics device under the
image display command {\bf plot}.  The maximum intensity value is set with
the command {\bf setmax}.  Both these parameters can be set together with the
command {\bf setmm}.

\item[Usage :] Defining minimum intensity in {\bf plot} image display.

\item[Associated commands :] {\tt \htmlref{setmax}{SETMAX}},
{\tt \htmlref{setmm}{SETMM}}, {\tt \htmlref{plot}{PLOT}}

\item[Short version of command :] -
\item[Invocation :]

\begin{quote}{\tt  setmin }\\
or \\
{\tt setmin 25 }
\end{quote}

\begin{itemize}

\item {\tt 25 } is the intensity level mapped onto the minimum colour
 index for the current graphics display device
\end{itemize}

\end{description}

\hrule
\subsubsection*{\label{SETMM}\xlabel{SETMM}SETMM}

\begin{description}

\item[Description :] Executes both {\bf setmax} and {\bf setmin} to set
the maximum and minimum intensity values to be mapped onto the maximum
and minimum colour indices for the current graphics display device
under the command {\bf plot}.

\item[Usage :] Running both {\bf setmax} and {\bf setmin}.

\item[Associated commands :] {\tt \htmlref{setmin}{SETMIN}},
{\tt \htmlref{setmax}{SETMAX}}, {\tt \htmlref{plot}{PLOT}}

\item[Short version of command :] -
\item[Invocation :]

\begin{quote}{\tt  setmm }\\
or \\
{\tt setmm 0 1023 }
\end{quote}

\begin{itemize}

\item {\tt 0 and 1023 } are the intensity values mapped onto the
minimum and maximum colour indices available for the current graphics
display device for the command {\bf plot}

\end{itemize}

\end{description}

\hrule
\subsubsection*{\label{SETNUM}\xlabel{SETNUM}SETNUM}

\begin{description}

\item[Description :] Defines the axis annotation used in {\bf surround} (for
images) and contour and vector maps.  The annotation consists of a
border around the graphics and user-defined tick marks (user-defined
X,Y start positions and X,Y intervals).  The tick marks can represent
arcsecond offsets from a selected image pixel or RA,Dec positions from
that same position.  {\bf setnum} therefore defines the parameters
required to produce these image annotations.  The arcsecond/pixel scale
has to be set using the command {\bf setps} before the command {\bf
setnum} is executed so that the correct offsets in arcseconds or
seconds of time/arcseconds can be produced.

\item[Usage :] Defining axis annotation setup for {\bf surround} annotation.

\item[Associated commands :] {\tt \htmlref{setps}{SETPS}}

\item[Short version of command :] -
\item[Invocation :]

\begin{quote}{\tt  setnum }\end{quote}

\end{description}

\hrule
\subsubsection*{\label{SETNUMORI}\xlabel{SETNUMORI}SETNUMORI}

\begin{description}

\item[Description :] Defines the orientation of Y-axis annotation numbers on
annotation plotted around images using the command {\bf surround} or on
contour/vector maps.  The two possibilities are horizontal so that the
numbers can be read normally or vertically so that the numbers are
written along the y-axis; the latter is the most useful when additional
axis comments/title strings are given.

\item[Usage :] Defining number orientation in {\bf surround} axis annotation.

\item[Associated commands :] {\tt \htmlref{surround}{SURROUND}},
{\tt \htmlref{contour}{CONTOUR}}, {\tt \htmlref{vec}{VEC}},
{\tt \htmlref{setfont}{SETFONT}}, {\tt \htmlref{setnumsca}{SETNUMSCA}}

\item[Short version of command :] -
\item[Invocation :]

\begin{quote}{\tt  setnumori }\end{quote}

\end{description}

\hrule
\subsubsection*{\label{SETNUMSCA}\xlabel{SETNUMSCA}SETNUMSCA}

\begin{description}

\item[Description :] Defines the size of numeric annotation on the axes
of images produced by the command {\bf surround} or around
contour/vector maps produced by the commands {\bf contour} and {\bf
vec}.  The input parameter is a scaling factor which if $<1$ means you
get larger number and if $>1$ you get smaller numbers.

\item[Usage :] Defining number size in {\bf surround} axis annotation.

\item[Associated commands :] {\tt \htmlref{surround}{SURROUND}},
{\tt \htmlref{contour}{CONTOUR}}, {\tt \htmlref{vec}{VEC}},
{\tt \htmlref{setfont}{SETFONT}}, {\tt \htmlref{setnumori}{SETNUMORI}}

\item[Short version of command :] -
\item[Invocation :]

\begin{quote}{\tt  setnumsca }\\
or \\
{\tt setnumsca 2 }
\end{quote}

\begin{itemize}

\item {\tt 2 } means the numbers on the axes will be 2$\times$ smaller
than the default size. A value of 0.5 would give numbers 2$\times$
larger than the default size.

\end{itemize}

\end{description}

\hrule
\subsubsection*{\label{SETPOLCOL}\xlabel{SETPOLCOL}SETPOLCOL}

\begin{description}

\item[Description :] Defines the colours of the polarization vectors
plotted using the command {\bf vec}.  The vectors can be all the same
colour or, if there are enough colour indices (pens) available on the
current graphics device (\emph{e.g.}, 256), different polarization
levels/ranges can have different colours. For example, 0-10\% could be
white, 11-20\% could be red, 21-30\% could be green, 31-40\% could be
yellow and everything above 41\% could be blue.  This helps delineate
regions of similar polarization in vector maps.

\item[Usage :] Defining polarization vector colours for multi-colour
vector plot.

\item[Associated commands :] {\tt \htmlref{vec}{VEC}},
{\tt \htmlref{setvec}{SETVEC}}

\item[Short version of command :] -
\item[Invocation :]

\begin{quote}{\tt  setpolcol }\end{quote}

\end{description}

\hrule
\subsubsection*{\label{SETPS}\xlabel{SETPS}SETPS}

\begin{description}

\item[Description :] Defines the arcsecond/pixel scale of the images
being process/analysed.  This value input is a real number and is the
arcseconds per pixel in the data.  {\bf setps} has to be executed
before a border with ticks marks in arcseconds offsets from a specific
location (or RA,Dec values) are plotted using the command {\bf
surround} (or automatically in {\bf contour} and {\bf vec}).  On
startup of {\sc IrcamDR}, the pixel scale for {\sc ircam3} images
without an external magnifier (\emph{i.e.}, 0.286"/pixel is
automatically defined).

\item[Usage :] Selecting pixel scale for {\bf setnum} \emph{etc.}
\item[Associated commands :] {\tt \htmlref{setnum}{SETNUM}}
\item[Short version of command :] -
\item[Invocation :]

\begin{quote}{\tt  setps }\\
or \\
{\tt setps 0.286 }
\end{quote}

\begin{itemize}

\item {\tt 0.286 } is the arcsecond/pixel value to be set
\end{itemize}

\end{description}

\hrule
\subsubsection*{\label{SETQUAD}\xlabel{SETQUAD}SETQUAD}

\begin{description}

\item[Description :] Allows you to display images in the four
quadrants of an image / graphics display device, \emph{i.e.}, top-right
(=1), top\_left (=2), bottom-left (=3) and bottom-right (=4).  The
quadrant is set using an integer value 1-4 (see above).  The
magnification of the images (set using {\bf setmag}) may also need to
be adjusted since the images will need to be displayed smaller to fit
into a quadrant of the current graphics device.  {\bf setquad} with the
argument 5 means reset image display to be on the centre position of
the current graphics device.

\item[Usage :] Selecting quadrant image display option.

\item[Associated commands :] {\tt \htmlref{setmag}{SETMAG}},
{\tt \htmlref{setcen}{SETCEN}}

\item[Short version of command :] {\tt sq}
\item[Invocation :]

\begin{quote}{\tt  setquad }\\
or \\
{\tt setquad 4 }
\end{quote}

\begin{itemize}

\item 4 means plot images in the bottom-right quadrant of the
 current image/graphics display device
\end{itemize}

\end{description}

\hrule
\subsubsection*{\label{SETRADEC}\xlabel{SETRADEC}SETRADEC}

\begin{description}

\item[Description :] Defines the RA,Dec of the origin point of axes
offsets plotted using the command {\bf surround} or under {\bf contour}
and {\bf vec} plots.  The RA,Dec are entered in the standard manner
with commas separating each of the components (\emph{e.g.}, degrees,
arcminutes, arcseconds).  Once the RA,Dec has been set the user can
define the origin and interval of the axes tick marks using the command
{\bf setnum}.

\item[Usage :] Setting RA and Dec for {\bf surround} image axis annotation.

\item[Associated commands :] {\tt \htmlref{setps}{SETPS}},
{\tt \htmlref{setnum}{SETNUM}}, {\tt \htmlref{setcomori}{SETCOMORI}}

\item[Short version of command :] -
\item[Invocation :]

\begin{quote}{\tt  setradec }\end{quote}

\end{description}

\hrule
\subsubsection*{\label{SETSTD}\xlabel{SETSTD}SETSTD}

\begin{description}

\item[Description :] Defines the reduction/analysis setup for the
automatic processing and photometry command {\bf stdred}.  The
parameters required are the number of images in a photometric sequence
(at different spatial offsets), the diameter of the star, inner and
outer sky apertures to be used, the sigma level for {\bf nsigma}
display of the images, whether to reduce raw ({\sc ro}) images or do
photometry on reduced images only, whether to do photometry on the
individual reduced images plus the mosaiced image or just the mosaic
image, whether to median filter the object images to create a
flat-field or use a separate flat-field (if using a separate flat-field
then the user can input the names of the dark subtracted, normalized to
unity flat-field for each waveband, J, H, K, nbL, Lp, nbM).

\item[Usage :] Defining how standard stars photometry mosaics are
reduced and analysed in {\bf stdred}.

\item[Associated commands :] {\tt \htmlref{stdred}{STDRED}}
\item[Short version of command :] -
\item[Invocation :]

\begin{quote}{\tt  setstd }\end{quote}

\end{description}

\hrule
\subsubsection*{\label{SETVAR}\xlabel{SETVAR}SETVAR}

\begin{description}

\item[Description :] Defines a variety of {\sc IrcamDR} variables,
specifically, the arcsecond/pixel scale, the magnification used for the
command {\bf disp}, the sigma-level used for {\bf nsigma} image display
in command {\bf disp}, the star and inner,outer sky annulus diameters
for command {\bf pho3} (and others), and the zeropoints for J, H, K,
nbL, Lp, and nbM.

\item[Usage :] Setting numerous variables for image analysis.

\item[Associated commands :] {\tt \htmlref{disp}{DISP}},
{\tt \htmlref{pho}{PHO}}, {\tt \htmlref{pho2}{PHO2}},
{\tt \htmlref{pho3}{PHO3}}, {\tt \htmlref{setps}{SETPS}}

\item[Short version of command :] -
\item[Invocation :]

\begin{quote}{\tt  setvar }\end{quote}

\end{description}

\hrule
\subsubsection*{\label{SETVARGREY}\xlabel{SETVARGREY}SETVARGREY}

\begin{description}

\item[Description :] Defines the parameters for {\bf vargrey}/{\bf
cvargrey} image display.  The two parameters required are the
percentage X of the intensity to be mapped onto the percentage of the
colour indices Y.  See {\bf vargrey} description for more details.

\item[Usage :] Setting {\bf vargrey} image scaling parameters.

\item[Associated commands :] {\tt \htmlref{vargrey}{VARGREY}},
{\tt \htmlref{cvargrey}{CVARGREY}}

\item[Short version of command :] {\tt setvg}
\item[Invocation :]

\begin{quote}{\tt  setvargrey }\\
or \\
{\tt setvargrey 20 80 }
\end{quote}

\begin{itemize}

\item 20 80 means map the lower 20\% of the intensity range onto
 the lower 80\% of the colour indices

\end{itemize}

\end{description}

\hrule
\subsubsection*{\label{SETVEC}\xlabel{SETVEC}SETVEC}

\begin{description}

\item[Description :] Defines the parameters needed for a polarization
vector map line plot using the command {\bf vec}.  The parameters
required by {\bf setvec} are the vector map magnification (0 for
auto-sizing to 80\% of the display area), the polarization position
angle correction value to rotate the polarization from the instrumental
to equatorial coordinate systems, the vector density (1 means plot a
vector at every pixel in the image, 2 means plot a vector at every
other pixel in the image \emph{etc}), the length of 100\% polarization
in units of pixels, whether you want annotation around the vector map
plot (annotation means axes, tick marks, axis annotation), how you want
to position the vector map on the current graphics display device, and
the colours used to produce the vector map plot (see {\bf setpolcol}
for more details).

\item[Usage :] Defining how vector plots using command {\bf vec} are displayed.

\item[Associated commands :] {\tt \htmlref{vec}{VEC}},
{\tt \htmlref{setpolcol}{SETPOLCOL}}, {\tt \htmlref{setps}{SETPS}}

\item[Short version of command :] -
\item[Invocation :]

\begin{quote}{\tt  setvec }\end{quote}

\end{description}

\hrule
\subsubsection*{\label{SHIFT2}\xlabel{SHIFT2}SHIFT2}

\begin{description}

\item[Description :] Shifts via linear interpolation (using {\bf
rapi2d} application {\bf shift}) two images to an average position.
The user can either input the accurate positions of the same feature
(star) in the two images or, by putting anything as a command line
parameter, use the cursor to select the common feature (star) in the
two images. In the latter case, the accurate positions of the features
are calculated using the {\bf rapi2d} application {\bf centroid}.  {\bf
shift2} also calculates the residuals after shifting between the two
features.

\item[Usage :] Registering two images to same spatial position.

\item[Associated commands :] {\tt \htmlref{shift}{SHIFT}},
{\tt \htmlref{shift3}{SHIFT3}}
\item[Short version of command :] -
\item[Invocation :]

\begin{quote}{\tt  shift2 }\\
or \\
{\tt shift2 Y }
\end{quote}

\begin{itemize}

\item {\tt Y } means use cursor selection of the common feature (star)
\end{itemize}

\end{description}

\hrule
\subsubsection*{\label{SHIFT3}\xlabel{SHIFT3}SHIFT3}

\begin{description}

\item[Description :] Shifts via linear interpolation (using {\bf
rapi2d} application {\bf shift}) three images to an average position.
The user can either input the accurate positions of the same feature
(star) in the three images or, by putting anything as a command line
parameter, use the cursor to select the common feature (star) in the
three images. In the latter case, the accurate positions of the
features are calculated using the {\bf rapi2d} application {\bf
centroid}.  {\bf shift3} also calculates the residuals after shifting
between the three features.

\item[Usage :] Registering three images to same spatial position.

\item[Associated commands :] {\tt \htmlref{shift}{SHIFT}},
{\tt \htmlref{shift2}{SHIFT2}}

\item[Short version of command :] -
\item[Invocation :]

\begin{quote}{\tt  shift3 }\\
or \\
{\tt shift3 Y }
\end{quote}

\begin{itemize}

\item {\tt Y }  means use cursor selection of the common feature (star)
\end{itemize}

\end{description}

\hrule
\subsubsection*{\label{SKYSUB4}\xlabel{SKYSUB4}SKYSUB4}

\begin{description}

\item[Description :] Subtracts the median value in a sub-area in each
of the four images from the image itself.  This command is useful for
sky subtracting the four images taken at the four different waveplate
positions in polarization observations.  The images are defined by
their full name and the first is displayed on the current graphics
device.  The user selects the sub-area to be used as the statistics
area for the median calculation (\emph{i.e.}, on sky for example) using
the cursor, he/she selects the bottom-left corner and top-right corner
of the sub-area with the cursor.  The median is calculated using the
{\bf rapi2d} application {\bf stats} and the value found in each image is
subtracted from the image itself using the {\bf rapi2d} application
{\bf csub}.  The output images are named the same as the input images but
with the letter 's' added to them.

\item[Usage :] Subtracting sky from 4 images, usually images at four
waveplate positions in polarimetry mode.

\item[Associated commands :] {\tt \htmlref{stats}{STATS}},
{\tt \htmlref{csub}{CSUB}}

\item[Short version of command :] -
\item[Invocation :]

\begin{quote}{\tt  skysub4 }\end{quote}

\end{description}

\hrule
\subsubsection*{\label{SPLITLOT}\xlabel{SPLITLOT}SPLITLOT}

\begin{description}

\item[Description :] Splits images from a series of images,
whose names are defined by a prefix, a range of numbers and a suffix
(see above for the definition of the operation of the ``{\bf lot}''
family of images), into four separate images each containing one
quadrant of the original image. Thus, for a 256$\times$256 {\sc ircam3}
image named {\tt im\_50dfzm}, {\bf splitlot} would create four output
images called {\tt im\_50dfzma}, {\tt im-50dfzmb}, {\tt im\_50dfzmc}
and {\tt im\_50dfzmd} each containing one quadrant of the original
image viz: {\tt a} contains data from pixel 1,1 (bottom-left corner of
image) to 128,128, {\tt b} contains data from pixel 129,1 to 256,128,
{\tt c} contains data from pixel 129,1 to 256,128 and {\tt d} contains
data from pixel 129,129 to 256,256.  The extraction of the sub-areas
into separate images is performed using the {\bf rapi2d} application
{\bf pickim}.

\item[Usage :] Splitting very large images from other sources
(say CCD images or large scale mosaic images), making the data more
manageable.

\item[Associated commands :] {\tt \htmlref{pickim}{PICKIM}}
\item[Short version of command :] -
\item[Invocation :]

\begin{quote}{\tt  splitlot }\end{quote}

\end{description}

\hrule
\subsubsection*{\label{STDPROC}\xlabel{STDPROC}STDPROC}

\begin{description}

\item[Description :] Creates an ICL procedure file for the automatic
reduction of a complete nights {\sc ircam3} data with pre-defined
stellar photometry extracted and stored.  {\bf stdproc} runs an {\sc
IrcamDR} Fortran 77 program called {\bf crestd} which analyses the raw
observation ({\sc ro}) index file (created by the GCS4DR utility {\tt
ircam3list} which is run automatically at the end of each nights
observing) whose name is of the form {\tt ro950714.lis}.  This file has
all the information about each observation taken stored in a one line
per observation summary.  {\bf stdproc} via {\bf crestd} analyses this
file and creates an ICL batch file called {\tt crestd.icl} which will
contain the necessary commands to fully reduce mosaics of specified
objects and extract photometric information on stellar sources within
the individual images and the final mosaic image.  The data reduction
and photometric information extraction is performed by the command {\bf
stdred} (see below).  The ASCII ICL batch file created by {\bf stdproc}
for subsequent data reduction will automatically execute the {\bf
stdred} setup procedure {\bf setstd} before reducing any images.  The
program {\bf crestd} will queue for reduction user specified objects.
It looks for a file called {\tt objects.list} in the current directory
and if it finds one it will look in their for the object names or
object name prefixes to be reduced.  By default, if no {\tt
objects.list} file is found, {\bf crestd} will queue image mosaics with
object names starting with the letters {\tt `fs'} or {\tt `hd'}
\emph{i.e.}, generally standard stars.  Users can enter their own names
into an {\tt objects.list} file so, for example, an {\tt objects.list}
file could contain lines like:

\begin{small}
\begin{verbatim}
      cyg
      ngc
      fs
      hd
      gl
      s
      -sky
\end{verbatim}
\end{small}

and hence, all image mosaics with object names starting with the given
letters will be queued for reduction using {\bf stdred}.  The {\tt
-sky} entry allows the user to tell {\bf crestd} not to reduce images
beginning with the letters {\tt sky} even though, on the previous line,
he/she has told the program to reduce all image mosaics beginning with
the letter {\tt s}.  Once {\bf stdproc} has run, it will ask if you
want to execute the file {\tt crestd.icl}.   If you do not want to
execute it immediately then you can do so at a later date using the ICL
command:

\begin{verbatim}
      load crestd.icl
\end{verbatim}

\item[Usage :] Executing automatic reduction of nights single star photometry
mosaic data sets.
\item[Associated commands :] {\tt \htmlref{stdred}{STDRED}}
\item[Short version of command :] -
\item[Invocation :]

\begin{quote}{\tt  stdproc }\end{quote}

\end{description}

\hrule
\subsubsection*{\label{STDRED}\xlabel{STDRED}STDRED}

\begin{description}

\item[Description :] Command to fully process/reduce a set of {\sc
ircam3} images forming a mosaic.  Also, {\bf stdred} will extract
photometric information from the reduced images.  {\bf stdred} was
written to reduce and analyse standard star data from the {\sc ukirt}
faint standard definition program but can be used on any set of data
given the appropriate setup.  Setup of {\bf stdred} is performed using
the command {\bf setstd}.  {\bf stdred} takes the following steps in
reducing image mosaics:

\begin{enumerate}

\item dark subtracts the images,

\item creates a median filtered flat-field from the images
(optional, user can use a separate sky flat-field -- option defined in
{\bf setstd}),

\item applies the flat-field to the dark subtracted images,

\item airmass corrects the images to unit airmass (optional, user
can switch airmass correction on/off in {\bf stdred} itself),

\item dc sky level offsets between mosaic tiles are calculated
using the {\bf obsrap} application {\bf automos} and applied to the
images -- corrections are dc subtractions only,

\item the bad/hot pixels are masked out using a bad pixel magic
number (the value used in {\sc IrcamDR} is {\tt -1.0e-20}),

\item mosaics the tiles together using the {\bf rapi2d}
application {\bf quilt}.  Bad/hot pixels masked out above are replaced
by good pixels from overlapping images,

\item photometry is extracted from the images (either the
individual tile images plus the final mosaiced image or just the final
mosaiced image).  Photometry is setup in {\bf setstd}.

Users can perform photometry on the stellar source located near the
centre of the first image in the mosaic (and at the offset position in
the mosaic tiles) or he/she can select the source for photometry using
the cursor.

\end{enumerate}

The results are appended to the file {\tt stdred\_photometry.results}
in the current directory, and written to the screen.  This give a
complete set of information on the images and the photometry
extracted.  At present, only one source per image can have
auto-photometry performed on it.  The option to select more than one
source per image is under development.

\item[Dark selection :] {\bf stdred} assumes you took a dark exposure
immediately prior to the last sequence of J images taken (it assume you
took dark, N J images of a standard, N H images of the standard, then N
K images of the standard).  It therefore looks at the image N-1 if the
current sequence of images are using the J filter, N*2-1 if the current
series of images are using the H filter, and N*3-1 if the current
series of images are using the K filter.  If the above image is not a
dark, stdred will prompt the user for the dark observation number.

{\bf stdred}  can also be used to perform photometry on an already processed
image. It will then write the results to the same output ASCII file in
the same format as when analysing raw images reduced in {\bf stdred}.  To
define whether to perform photometry on raw ({\bf stdred} processed) images or
already reduced images use the command {\bf setstd}.

\item[Usage :] Reducing single star photometry mosaic image sets as
defined by {\bf setstd}.

\item[Associated commands :] {\tt \htmlref{stdproc}{STDPROC}},
{\tt \htmlref{setstd}{SETSTD}}

\item[Short version of command :] -
\item[Invocation :]

\begin{quote}{\tt  stdred }\\
or \\
{\tt stdred 42 3 41 } \\
or \\
{\tt stdred im\_50dfzm K 0.286 1.34 1}
\end{quote}

\begin{itemize}

\item {\tt 42 } is the start observation number of the mosaic series
\item {\tt 3 } is the number of images in the mosaic series
\item {\tt 41 } is the dark to be used in the reduction.
\item {\tt im\_50dfzm } is the reduced image to perform photometry on
\item {\tt K } is the filter used to take the data in image
\item {\tt 0.286 } is the pixel scale in arcseconds/pixel
\item {\tt 1.34 } is the airmass at which the image was taken
\item {\tt 1 } means use peak positive features for aperture centering
(2 would mean use peak negative features for aperture centering)
\end{itemize}

\end{description}

\hrule
\subsubsection*{\label{STLOT}\xlabel{STLOT}STLOT}

\begin{description}

\item[Description :] Subtracts a constant (sky for example) and
thresholds a series of images whose names are defined by a prefix, a
range of numbers and a suffix. See above for the definition of the
operation of the ``{\bf lot}'' family of images.  The median signal in
a user-defined sub-area of the image (defined by cursor selection) is
subtracted from the whole image and then the images is thresholded
\emph{e.g.}, all pixels with values $<$ 10-sigma below 0 (the median
value in the sub-area after constant subtraction) are set to zero.  The
output images are called the same as the input images but have the
letters {\tt `st'} added to their name.

\item[Usage :] Sky subtracting and thresholding a series of images.

\item[Associated commands :] {\tt \htmlref{stats}{STATS}},
{\tt \htmlref{histo}{HISTO}}, {\tt \htmlref{csub}{CSUB}},
{\tt \htmlref{thresh}{THRESH}}

\item[Short version of command :] -
\item[Invocation :]

\begin{quote}{\tt  stlot }\end{quote}

\end{description}

\hrule
\subsubsection*{\label{STRED}\xlabel{STRED}STRED}

\begin{description}

\item[Description :] Command to fully process/reduce a set of {\sc
ircam3} images forming a mosaic. {\bf stred} takes the following steps in
reducing image mosaics:

\begin{enumerate}

\item dark subtracts the images,

\item creates a median filtered flat-field from the images (optionally,
user can use a separate sky flat-field),

\item applies the flat-field to the dark subtracted images,

\item airmass corrects the images to unit airmass
(optionally, user can switch airmass correction on/off in stred itself),


\item dc sky level offsets between mosaic tiles are calculated using
the obsrap application automos and applied to the images - corrections
are dc subtractions only,

\item the bad/hot pixels are masked out using a bad pixel magic number
(the value used in {\sc IrcamDR} is {\tt -1.0e-20}),

\item the tiles are mosaiced together using the {\bf rapi2d}
application {\bf quilt}.

Bad/hot pixels masked out in 5 above are replaced
by good pixels from overlapping images.

\end{enumerate}

\item[Note :] The output images from {\bf stred} are scaled to unit
exposure time.
\item[Usage :] Reducing imaging mosaics completely.
\item[Associated commands :] {\tt \htmlref{stdred}{STDRED}}
\item[Short version of command :] -
\item[Invocation :]

\begin{quote}{\tt  stred }\end{quote}

\end{description}

\hrule
\subsubsection*{\label{STREHL}\xlabel{STREHL}STREHL}

\begin{description}

\item[Description :] Used to calculate strehl coefficients for stellar
images.  used mostly for images taken with the 2$\times$ or 5$\times$
magnifier in say, shift-and-add mode of {\sc ircam3}.  {\bf strehl}
works on raw ({\sc ro}) images (like those produced by shift-and-add
mode).  It takes as a parameter the observation number of the image to
be analysed (the UT date for the {\sc ro} filename definition is set in
{\bf setfile} or on {\sc IrcamDR} {\bf sunup}).  The {\bf strehl}
coefficient is dependent on the filter and pixel scale used and
actually refers to the closeness to theoretical the peak signal to
total signal in a stellar image comes.  A value is returned from {\bf
strehl} (the Strehl ratio) and this is a fractional value referring to
the observed to theoretical comparison.

\item[Usage :] Determining strehl value of stellar imaging observation
generally using external magnifiers in shift-and-add mode.
\item[Associated commands :] -
\item[Short version of command :] -
\item[Invocation :]

\begin{quote}{\tt  strehl }\\
or \\
{\tt strehl 42 }
\end{quote}

\begin{itemize}

\item {\tt 42 } is the observation number of the raw ({\sc ro}) image input
 into {\bf strehl} for analysis
\end{itemize}

\end{description}

\hrule
\subsubsection*{\label{SURROUND}\xlabel{SURROUND}SURROUND}

\begin{description}

\item[Description :] Displays a border and tick marks/numbers around
the image currently displayed on the graphics device.  The tick marks
are defined using the command {\bf setps} and {\bf setnum}; {\bf setps}
defined the arcsecond/pixel scale while {\bf setnum} defines the tick
mark start position along the image axes and the interval between
them.  The thickness of the image border and tick marks are defined by
the command {\bf line\_width}.

\item[Usage :] Plotting axis annotation around current image as defined by
{\bf setnum} command.

\item[Associated commands :] {\tt \htmlref{setps}{SETPS}},
{\tt \htmlref{setnum}{SETNUM}}, {\tt \htmlref{line\_width}{LINE_WIDTH}}

\item[Short version of command :] {\tt sur}
\item[Invocation :]

\begin{quote}{\tt  surround }\end{quote}

\end{description}

\hrule
\subsubsection*{\label{TICKIO}\xlabel{TICKIO}TICKIO}

\begin{description}

\item[Description :] Defines whether tick marks produced by the command
{\bf surround} or under contour or vector map plots (using commands
{\bf contour} and {\bf vec}) are plotted inside or outside the border
around the image/plot.

\item[Usage :] Setting tick mark position in {\bf surround} axis annotation.

\item[Associated commands :] {\tt \htmlref{surround}{SURROUND}},
{\tt \htmlref{setps}{SETPS}}, {\tt \htmlref{setnum}{SETNUM}},
{\tt \htmlref{contour}{CONTOUR}}, {\tt \htmlref{vec}{VEC}}

\item[Short version of command :] -
\item[Invocation :]

\begin{quote}{\tt  tickio }\\
or \\
{\tt tickio 1 }
\end{quote}

\begin{itemize}

\item {\tt 1 } means plot tick marks inside border box
\item {\tt 2 } would mean plot tick marks outside border box
\end{itemize}

\end{description}

\hrule
\subsubsection*{\label{TICKLEN}\xlabel{TICKLEN}TICKLEN}

\begin{description}

\item[Description :] Defines whether small or large tick marks are
plotted on contour maps and annotation produced by the command {\bf
surround}.  The tick marks are the fiducial marks along both X and Y
axes of the image (RA and Dec) and small tick marks are just of small
length while large tick marks form a complete grid across the image.
The width of the line used is set with the command {\bf line\_width}.
Tick marks are defined using the command {\bf setnum}.

\item[Usage :] Setting tick mark length in {\bf surround} image annotation.

\item[Associated commands :] {\tt \htmlref{setnum}{SETNUM}},
{\tt \htmlref{line\_width}{LINE_WIDTH}}, {\tt \htmlref{contour}{CONTOUR}},
{\tt \htmlref{surround}{SURROUND}}, {\tt \htmlref{ticklen}{TICKLEN}}

\item[Short version of command :] -
\item[Invocation :]

\begin{quote}{\tt  ticklen }\\
or \\
{\tt ticklen 1 }
\end{quote}

\begin{itemize}

\item {\tt 1 } means plot small tick marks
\item {\tt 2 } 2 would mean plot full - grid - tick marks
\end{itemize}

\end{description}

\hrule
\subsubsection*{\label{TOMAG}\xlabel{TOMAG}TOMAG}

\begin{description}

\item[Description :] Converts images in intensity (DN/second) to
magnitudes per pixel and per square arcsecond using input zeropoints.
User-specifies full name of input image, and both output images.

\item[Usage :] Converting linear intensity images to
magnitudes for colour (\emph{e.g.}, J-K) calculations.  Images are best
thresholded ({\bf thresh}) before conversion to eliminate low, noisy levels.

\item[Associated commands :] -
\item[Short version of command :] -
\item[Invocation :]

\begin{quote}{\tt  tomag }\end{quote}

\end{description}

\hrule
\subsubsection*{\label{TSLOT}\xlabel{TSLOT}TSLOT}

\begin{description}

\item[Description :] Thresholds and scales a series of images whose
names are defined by a prefix, a range of numbers and a suffix. See
above for the definition of the operation of the ``{\bf lot}'' family
of images.  The user has the option to scale so that the intensity
range becomes 0-255 or to a user-defined range of values.  The images
input first are analysed and the median value in the images is
subtracted off the images. Next, pixels below zero are set to zero and
then the new (modified) {\it max},{\it min} range in the data is scaled
to be 0,255 or whatever the user has input).  The output images are of
the same name as the input images but with the letters {\tt `ts'} added to
the names.

\item[Usage :] Thresholding and scaling a series of images.

\item[Associated commands :] {\tt \htmlref{histo}{HISTO}},
{\tt \htmlref{stats}{STATS}}, {\tt \htmlref{csub}{CSUB}},
{\tt \htmlref{thresh}{THRESH}}, {\tt \htmlref{cdiv}{CDIV}}

\item[Short version of command :] -
\item[Invocation :]

\begin{quote}{\tt tslot }\end{quote}

\end{description}

\hrule
\subsubsection*{\label{VANS}\xlabel{VANS}VANS}

\begin{description}

\item[Description :] Plots an vector array image from shift-and-add
data on the current image display device.  Vector array images define
the motion of the peak pixel over the bid-area of the array during a
shift-and-add exposure.  Vector array images are stored as {\tt i}
images (raw integrations) and hence, {\bf vans} plots the {\tt i}
images associated with the ({\sc ro}) shift-and-add image on the image
display.  For example, the vector array images associated with the raw
observation image {\tt ro950412\_10} would be {\tt i950410\_10\_1v}.

\item[Usage :] Plotting a vector array image from shift-and-add imaging.
\item[Associated commands :] {\tt \htmlref{strehl}{STREHL}}
\item[Short version of command :] -
\item[Invocation :]

\begin{quote}{\tt  vans }\\
or \\
{\tt vans 10 }
\end{quote}

\begin{itemize}

\item {\tt 10 } is the observation number of the vector array image
 to be displayed
\end{itemize}

\end{description}

\hrule
\subsubsection*{\label{VARGREY}\xlabel{VARGREY}VARGREY}

\begin{description}

\item[Description :] Plots an image on the current graphics device
using variable index scaling.  Variable index scaling maps a percentage
of the dynamic range in the image onto a different percentage of the
colour table range available.  This means that one could get the bottom
N\% of the intensity range in an image mapped onto the bottom M\% of
the colours; this for example, will significantly enhance the
visibility of the low level features in the image.

A specific example of {\bf vargrey} plotting would be, if you want to
map the bottom 20\% of the intensity range in an image onto the bottom
80\% of the colour indices (available pens) available.  Let us assume
that the range of signal in your image is 100 to 1000 counts and the
range of colour indices is 1 to 64 (\emph{i.e.}, the graphic display
window was created with 64 colour indices).  The bottom 20\% of the
intensity range in the image would be 100-280 counts (min+0.2*(max-min)).
The bottom 80\% of the colour indices would be $\approx$1-51 pens.
Hence, in a {\bf vargrey} plot the range of colours covering pens 1
through 51 would represent the intensity range 100-280 while the range
of colours covering pens 52-64 would represent
the intensity range 282-1000.  Clearly this will enhance the visibility
of the low level features since more colours (pens) will be used to
show the low intensity levels.

{\bf vargrey} plots the image with its centre located at the device
coordinates input by the user.  Generally this is the centre of the
graphics display device.  The variable index scaling parameters are set
using the command {\bf setvargrey}.

\item[Usage :] Displaying an image using vargrey scaling.

\item[Associated commands :] {\tt \htmlref{cvargrey}{VARGREY}},
{\tt \htmlref{setvargrey}{SETVARGREY}},
{\tt \htmlref{morevargrey}{MOREVARGREY}}, {\tt \htmlref{plot}{PLOT}},
{\tt \htmlref{cplot}{CPLOT}}, {\tt \htmlref{nsigma}{NSIGMA}}, \\
{\tt \htmlref{cnsigma}{CNSIGMA}}, {\tt \htmlref{ranplot}{RANPLOT}},
{\tt \htmlref{cranplot}{CRANPLOT}}, {\tt \htmlref{setmag}{SETMAG}}, \ldots

\item[Short version of command :] {\tt vg}
\item[Invocation :]

\begin{quote}{\tt  vargrey }\\
or \\
{\tt vargrey 23 }
\end{quote}

\begin{itemize}

\item {\tt 23 } is the observation number of the image to be plotted
 using vargrey scaling
\end{itemize}

\end{description}

\hrule
\subsubsection*{\label{VEC}\xlabel{VEC}VEC}

\begin{description}

\item[Description :] Plots a polarization vector map from either images
of percentage polarization and position angle or Q and U Stokes
parameters.  The setup for the vector map plot is performed by the
command {\bf setvec}.  The polarization images are formed from the four
intensity waveplate position images for single-beam (IRPOL1) data by
{\bf polcal} and for dual-beam data by {\bf polcal2} (not yet
written).  The thickness of the lines plotted by vec are defined by the
command {\bf line\_width}.

\item[Usage :] Plotting a vector map of polarization data.
\item[Associated commands :] {\tt \htmlref{setvec}{SETVEC}},
{\tt \htmlref{line\_width}{LINE_WIDTH}}
\item[Short version of command :] -
\item[Invocation :]

\begin{quote}{\tt  vec }\end{quote}

\end{description}

\hrule
\subsubsection*{\label{VEC_TITLE}\xlabel{VEC_TITLE}VEC\_TITLE}

\begin{description}

\item[Description :] Defines the title string for the top line of a {\bf vec}
polarization vector map plot.  The title string can be any text sting
and it can also be set in the command {\bf setvec}.

\item[Usage :] Defining title for vector plot using command {\bf vec}.
\item[Associated commands :] {\tt \htmlref{setvec}{SETVEC}},
{\tt \htmlref{vec}{VEC}}
\item[Short version of command :] -
\item[Invocation :]

\begin{quote}{\tt  vec\_title }\\
or \\
{\tt vec\_title 'This is the title string' }
\end{quote}

\end{description}

\hrule
\subsubsection*{\label{WRAPLOT}\xlabel{WRAPLOT}WRAPLOT}

\begin{description}

\item[Description :] Corrects a series of images whose names are
defined by a prefix, a range of numbers and a suffix (see above for the
definition of the operation of the ``{\bf lot}'' family of images) for
16-bit wrap-around.  {\bf wraplot} adds a user-defined number to all pixel
values in an image below a user-defined intensity.  To correct for
16-bit wrap-around (whereby image pixel data is stored as 16-bit
numbers and values larger than 16-bits are made negative values)
{\bf wraplot} should add 65536 to all number in an image less than 0
(\emph{i.e.}, all the negative numbers).

\item[Usage :] Some optical CCD data images have this problem and have to be
corrected for 16-bit wrap-around.
\item[Associated commands :] -
\item[Short version of command :] -
\item[Invocation :]

\begin{quote}{\tt  wraplot }\end{quote}

\end{description}

\hrule
\subsubsection*{\label{WRCCOM}\xlabel{WRCCOM}WRCCOM}

\begin{description}

\item[Description :] Writes a comment (text) string on the current
graphics device.  The position of the string on the graphics screen is
determined cursor selection of the centre of the string.  {\bf wrccom}
allows you to specify not only the string content but also the size of
the characters, the orientation of the comment string and the font
used.

Note that GKS font numbers under Unix are negative and start at {\tt -101}.

\item[Usage :] Writing comments (text) on current workstation/device using
cursor selection of centre of string.
\item[Associated commands :] {\tt \htmlref{wrcom}{WRCOM}}
\item[Short version of command :] {\tt wrcc}
\item[Invocation :]

\begin{quote}{\tt  wrccom }\end{quote}

\end{description}

\hrule
\subsubsection*{\label{WRCOM}\xlabel{WRCOM}WRCOM}

\begin{description}

\item[Description :] Writes a comment (text) string on the current
graphics device. The position of the string on the graphics screen is
determined the pixel coordinates for the centre of the string.  {\bf
wrcom} allows you to specify not only the string content but also the
size of the characters, the orientation of the comment string and the
font used.

Note that GKS font numbers under Unix are negative and start at {\tt -101}.

\item[Usage :] Writing comments (text) on current workstation/device
using pixel coordinate selection of centre of string.

\item[Associated commands :] {\tt \htmlref{wrccom}{WRCCOM}}
\item[Short version of command :] {\tt wrc}
\item[Invocation :]

\begin{quote}{\tt  wrcom }\end{quote}

\end{description}

\hrule
\subsubsection*{\label{WRITELUT}\xlabel{WRITELUT}WRITELUT}

\begin{description}

\item[Description :] Writes a colour table (LUT) to the current graphics
device. The colour table (LUT) is taken from the sequence {\tt col1}
through {\tt col49} which are located in the {\tt \$LIRCAMDIR} {\sc
IrcamDR} directory.  Which colour table (LUT) to be displayed is
determined by the number associated with the col file \emph{e.g.}, {\tt
1} for {\tt col1}, {\tt 19} for {\tt col19} \emph{etc.} ({\tt col19} is
generally the most useful colour table in the {\sc IrcamDR} package).
The associated command {\bf coltab} writes a named colour table (LUT)
to the current graphics device.

\item[Usage :] Writing one of {\tt col1} to {\tt col49} colour tables
(LUTs) to current workstation/device using number along (1-49).
\item[Associated commands :] {\tt \htmlref{coltab}{COLTAB}}
\item[Short version of command :] -
\item[Invocation :]

\begin{quote}{\tt  writelut }\\
or \\
{\tt writelut 19 }
\end{quote}

\begin{itemize}

\item {\tt 19 } refers to {\tt col19} which  is the colour table selected by
 the user
\end{itemize}

\end{description}

\hrule
\subsubsection*{\label{X2MAG}\xlabel{X2MAG}X2MAG}

\begin{description}

\item[Description :] Defines the arcsecond/pixel scale for {\sc
IrcamDR} when the 2$\times$ external magnifier is installed on {\sc
ircam3}.  The pixel scale defined is 0.143"/pixel which is half that of
the default value 0.286"/pixel.  The default pixel scale is selected
using the command {\bf nomag}.  The 5$\times$ magnifier pixel scale is
selected using the command {\bf x5mag}.

\item[Usage :] Defineing pixel scale when using 2$\times$ external magnifier
(0.143"/pix).
\item[Associated commands :] {\tt \htmlref{nomag}{NOMAG}},
{\tt \htmlref{x5mag}{X5MAG}}
\item[Short version of command :] -
\item[Invocation :]

\begin{quote}{\tt  x2mag }\end{quote}

\end{description}

\hrule
\subsubsection*{\label{X5MAG}\xlabel{X5MAG}X5MAG}

\begin{description}

\item[Description :] Defines the arcsecond/pixel scale for {\sc
IrcamDR} when the 5$\times$ external magnifier is installed on {\sc
ircam3}.  The pixel scale defined is 0.0572"/pixel which is one fifth
that of the default value 0.286"/pixel.  The default pixel scale is
selected using the command {\bf nomag}.  The 2$\times$ magnifier pixel
scale is selected using the command {\bf x2mag}.

\item[Usage :] Defining pixel scale when using 5$\times$ external magnifier
(0.0571"/pix).
\item[Associated commands :] {\tt \htmlref{nomag}{NOMAG}},
{\tt \htmlref{x2mag}{X2MAG}}
\item[Short version of command :] -
\item[Invocation :]

\begin{quote}{\tt  x5mag }\end{quote}

\end{description}

\hrule
\subsubsection*{\label{ZAPLOT}\xlabel{ZAPLOT}ZAPLOT}

\begin{description}

\item[Description :] removes bad columns and rows from a series of
images whose names are defined by a prefix, a range of numbers and a
suffix (see above for the definition of the operation of the ``{\bf
lot}'' family of images).  The bad columns and rows are defined by
their column/row number and the number of consecutive columns/rows to
be replaced.  The bad columns/rows are replaced by linear interpolation
over then.  {\bf zaplot} uses the {\bf rapi2d} application {\bf zaplin}
to remove bad columns/rows.

\item[Usage :] Removing bad columns/rows from a series of images.
\item[Associated commands :] {\tt \htmlref{zaplin}{ZAPLIN}}
\item[Short version of command :] -
\item[Invocation :]

\begin{quote}{\tt  zaplot }\end{quote}

\end{description}

%  TASKS

\newpage
\subsection{\label{ss:atask_descriptions}\xlabel{atask_descriptions}A-TASK (rapi2d, obsrap, polrap) applications description}

\begin{quote}
NB. See interface files {\tt rapi2d.ifl}, {\tt obsrap.ifl} and {\tt
polrap.ifl} for full command line parameter usage/specification of all
a-task monolith applications.
\end{quote}

\hrule
\subsubsection*{\label{ABCOM}\xlabel{ABCOM}ABCOM}
\begin{description}

\item[Description :] Combines two images, the first containing columns
of data from the odd channels of a two output array, the second
containing column data from the even channels of a two output array,
into one image. Two output arrays (like the SBRC 62$\times$58 DRO
devices) can have difference gains (and different problems in the odd
and even column channels and so the user can use {\bf absep} to
separate the odd and even channels of the image/array, work on then
separately and then re-combine the odd and even channel images into one
image using {\bf abcom}.

\item[Usage :] Reconstructing an image from separate odd/even column images.

\item[Associated commands :] {\tt \htmlref{absep}{ABSEP}}

\item[Invocation :]
\begin{quote}{\tt
abcom}
\end{quote}

\end{description}

\hrule
\subsubsection*{\label{ABSEP}\xlabel{ABSEP}ABSEP}

\begin{description}

\item[Description :] Separates an image into two images, the first will
contain columns of data from the odd channels of a two output array,
the second will contain column data from the even channels of a two
output array.  Two output arrays (like the SBRC 62$\times$58 DRO
devices) can have difference gains (and different problems in the odd
and even column channels and so the user can use {\bf absep} to
separate the odd and even channels of the image/array, work on then
separately and then re-combine the odd and even channel images into one
image using {\bf abcom}.

\item[Usage :] Separating odd/even column channels into two separate images.

\item[Associated commands :] {\tt \htmlref{abcom}{ABCOM}}

\item[Invocation :]
\begin{quote}{\tt
absep}
\end{quote}

\end{description}

\hrule
\subsubsection*{\label{ADD}\xlabel{ADD}ADD}

\begin{description}

\item[Description :] Adds to images together forming a new image with
the result.

\item[Usage :] To add to images together.

\item[Associated commands :] {\tt \htmlref{sub}{SUB}},
{\tt \htmlref{div}{DIV}}, {\tt \htmlref{mult}{MULT}}

\item[Invocation :]

\begin{quote}
{\tt add} \\
or \\
{\tt  add im1 im2 im3}
\end{quote}

\begin{itemize}
\item {\tt im1} and {\tt im2 } are the input images to be added together
\item {\tt im3} is the output image.
\end{itemize}

\end{description}

\hrule
\subsubsection*{\label{AMCORR}\xlabel{AMCORR}AMCORR}

\begin{description}

\item[Description :] Corrects an image intensity for the effect of
atmospheric extinction (airmass effect).  The correction is a
multiplicative one and corrects the intensity to unit airmass
(\emph{i.e.}, the zenith).  The correction applied takes the form of
multiplying the pixel data by the result of the equation $10^{(airmass
\times extval/2.5)}$ where {\it airmass} is the airmass of the
observation and {\it extval} is the extinction coefficient (in
magnitudes/airmass) for the filter used.

\item[Usage :] Correcting images for atmospheric extinction (airmass effect).

\item[Associated commands :] {\tt \htmlref{amcorrlot}{AMCORRLOT}}

\item[Invocation :]

\begin{quote}{\tt  amcorr }\end{quote}

\end{description}

\hrule
\subsubsection*{\label{ANNSTATS}\xlabel{ANNSTATS}ANNSTATS}

\begin{description}

\item[Description :] Returns statistics in annuli of increasing
diameter from a user-defined centre position in an image.  Useful for
annular stats on say, an elliptical galaxy.  The width, eccentricity
and position angle of the annuli are also user-defined.  {\bf annstats}
returns the number of pixels, the inner and outer radius, the mean and
median signal, the sum and {\it max},{\it min} over the pixels, and the
standard deviation of the pixels in the annulus. An image is produced
by {\bf annstats} called {\tt annstats\_map}.  This contains a map of
the annuli within which the statistics have been calculated. Alternate
annuli have values of 0 and 1 in {\tt annstats\_map}. Also output from
{\bf annstats} is a text (ASCII) file containing the statistical
information calculated in the annuli considered.

\item[Usage :] For annular stats in extended objects \emph{e.g.},
elliptical galaxies.

\item[Associated commands :] {\tt \htmlref{stats}{STATS}},
{\tt \htmlref{cstats}{CSTATS}}, {\tt \htmlref{aperadd}{APERADD}}

\item[Invocation :]

\begin{quote}{\tt  annstats }\end{quote}

\end{description}

\hrule
\subsubsection*{\label{APERADD}\xlabel{APERADD}APERADD}

\begin{description}

\item[Description :] Performs aperture statistics within the named
image. The input required is the name of the image to have statistics
extracted from, the X,Y pixel number for the centre of the aperture to
be used for statistics calculation, the diameter of the circular
statistics aperture (in arcseconds) and the arcsecond/pixel scale
within the named image. {\bf aperadd} only used pixels that lie completely
within the circular bounds of the user-defined aperture \emph{i.e.},
fractional pixels are not considered. The output from {\bf aperadd} is
returned to the terminal and consists of: the number of pixel in the
aperture, the total intensity, the mean signal, and the noise (standard
deviation) before and after binning (by the number of pixels in the
aperture.  The noise after binning is just the noise before binning
(the standard deviation of all the pixels) divided by the root of the
number of pixels in the aperture.

\item[Usage :] For statistics within single circular aperture in
an image.

\item[Associated commands :] {\tt \htmlref{aperphot}{APERPHOT}},
{\tt \htmlref{annstats}{ANNSTATS}}, {\tt \htmlref{stats}{STATS}},
{\tt \htmlref{cstats}{CSTATS}}

\item[Invocation :]

\begin{quote}{\tt  aperadd }\end{quote}

\end{description}

\hrule
\subsubsection*{\label{APERPHOT}\xlabel{APERPHOT}APERPHOT}

\begin{description}

\item[Description :] Aperture photometry program.  {\bf aperphot} takes a
user-defined aperture (location in image, size, shape) and optionally a
concentric annulus for sky, and returns photometric information from
the image.  {\bf aperphot} input prompts are as in the example below:

\begin{small}
\begin{verbatim}
      IrcamDR > aperphot
      INPIC - Image to be analysed /@JUNK/ > f39m40
      Title = DISP:OBJECT-SKY
      Image is 256 by 256 pixels
      DATASOURCE - Source of input terminal=T or file=F /'T'/ >
      XCEN - X-centre of object aperture /128/ > 171
      YCEN - Y-centre of object aperture /128/ > 52
      ECC - Eccentricity of object aperture/sky annulus /0/ >
      POSANG - Position angle of major axis wrt N (in degrees) /0/ >
      SKYANNUL - Subtract sky in concentric annulus (Y/N) ? /YES/ >
      MAJAX1 - Major axis of object aperture (in arcsec) /10/ > 5
      MAJAX2 - Major axis of inner diameter of concentric sky aperture
        (arcsec) /10/ >
      MAJAX3 - Major axis of outer diameter of concentric sky aperture
        (arcsec) /20/ >
      SCALE - Pixel scale (arcsec/pixel) /1/ > .286
      USEWHAT - Use MEDIAN or MEAN in sky calculation ? /'MEDIAN'/ >
      USEBAD - Use bad pixel value in calculation (Y/N) ? /YES/ >
      BADVAL - Bad pixel value ? /-9.9999997E-2/ > -1.0e-20
\end{verbatim}
\end{small}

Output from {\bf aperphot} for terminal input mode (as above) is as follows:

\begin{small}
\begin{verbatim}
      STAR APERTURE
      *************
      Number of pixels in aperture         = 241/0
      Total intensity in aperture          = 24795.99
      Mean intensity over aperture         = 102.8879
      Median intensity over aperture       = 51.99756
      Standard deviation in aperture       = 142.035

      SKY ANNULUS
      ***********
      Number of GOOD/BAD pixels in annulus = 2868/0
      Total intensity in annulus           = 57773.33
      Mean intensity over annulus          = 20.14412
      Median intensity over annulus        = 24.00037
      Standard deviation over annulus      = 46.72461

      Sky in object aperture (MEDIAN)      = 5784.088

      Object-Sky value                     = 19011.9

      Magnitude from absolute value        = -10.69756

      AGAIN - Another aperture in same image (Y/N) ? /YES/ > n:
\end{verbatim}
\end{small}

In the example, input is given from the terminal.  Alternatively input
can be made to {\bf aperphot} in batch mode from a text (ASCII) file.  An
example of the format of the ASCII file is as below:

\begin{small}
\begin{verbatim}
      * test file for aperphot file input
      * a starred line is a comment line anywhere in the file
      128 156 0.2 45.0 1 10.0 15.0 20.0 0.286 1 1 -1.0e-20
      * another comment line...
      197 51 0.11 37.0 1 5.0 12.0 18.0 0.286 0 0 -1.0e-20
\end{verbatim}
\end{small}

where, for the first numeric input line:

\begin{itemize}

\item {\tt 128 156} are the X,Y pixel location of the centre of the
aperture/annulus,

\item {\tt 0.2 45} are the aperture/annulus eccentricity and position
angle (e of n),

\item {\tt 1} means subtract sky from a concentric sky annulus ({\tt 0}
would mean do not subtract sky),

\item {\tt 10 15 20} are the star aperture diameter, and the inner and
outer sky annulus diameters (arcseconds),

\item {\tt 0.286} is the pixel scale in arcsec/pixel,

\item {\tt 1} means use the median in the sky annulus for sky
subtraction ({\tt 0} would mean use the mean value instead),

\item {\tt 1} means do not use pixels with a magic (bad) value ({\tt 0}
would mean consider all pixels regardless of value), and

\item {\tt -1.0e-20} is the magic (bad) pixel value to be used.

\end{itemize}

{\bf aperphot} can write output (in file mode) to the terminal, a text
(ASCII) file or both.  The output file for the above input test file is
as below:

\begin{small}
\begin{verbatim}
      X  Y  E  POS  D1  D2  D3  N1  BN1  N2  BN2  TOT1  TOT2  TOT3 MAG
      * test file for aperphot file input
      * a starred line is a comment line anywhere in the file
      128.0 156.0 0.20 45.0 10.0 15.0 20.0 941 0 1656 0 25095 24991  104 -5.042
      * another comment line...
      197.0  51.0 0.11 37.0  5.0 12.0 18.0 241 0 1736 0  8803 10423 1620 -8.024
\end{verbatim}
\end{small}

The output file from aperphot is named by adding a letter {\tt `a'} to the
prefix of the input filename viz: if the input text file was called
{\tt ngc2071.inp} then the output file would be called {\tt ngc2071a.inp}.
The output file contains the same comment lines as the input file.  The
first line of the output file contains what the various output columns mean:

\begin{itemize}

\item {\tt X Y} are the input pixel coordinates of the centre of the
aperture/annulus,

\item {\tt E POS} are the eccentricity and position angle of the
aperture and annulus,

\item {\tt D1 D2 D3} are the three apertures defined (star+two sky),

\item {\tt N1} is the number of good pixels in the star aperture,

\item {\tt BN1} is the number of bad pixels in the sky aperture,

\item {\tt N2} and {\tt BN2} are the number of good and bad pixels in
the sky annulus,

\item {\tt TOT1} and {\tt TOT2} are total counts in
the star aperture and sky annulus,

\item {\tt TOT3} is the star-sky contribution counts, and

\item {\tt MAG} is the instrumental magnitude ({\tt -2.5*log10(TOT3)}) of the
source.

\end{itemize}

\item[Usage :] For aperture photometry in images especially in batch
mode.
\item[Associated commands :] {\tt \htmlref{aperadd}{APERADD}}
\item[Invocation :]

\begin{quote}{\tt  aperphot }\end{quote}

\end{description}

\hrule
\subsubsection*{\label{APERPOL}\xlabel{APERPOL}APERPOL}

\begin{description}

\item[Description :] Aperture polarimetry program similar to {\bf
aperphot}.  Takes four images as input, one at each of the four
half-waveplate positions required for polarimetry observations
\emph{i.e.}, {\tt 0}, {\tt 45}, {\tt 22.5} and {\tt 67.5} degrees from
datum.  The aperture/sky annulus input is as in {\bf aperphot}.
Additional input required is the position angle correction factor for
transformation from instrumental to equatorial coordinate systems, and
the electrons/DN in the image.  {\bf aperpol} works in both interactive
terminal mode and in batch mode.  The batch file text file has the
extra parameter included, an example of which is:

\begin{small}
\begin{verbatim}
      * test file for aperpol file input
      * a starred line is a comment line anywhere in the file
      128 156 0.2 45.0 1 10.0 15.0 0.286 1 1 -1.0e-20 27.2 6.0
      * another comment line...
      197 51 0.11 37.0 1 5.0 12.0 0.286 0 0 -1.0e-20 27.2 6.0
\end{verbatim}
\end{small}

The additional parameters are {\tt 27.2} an example of the coordinate
system transformation for the position angle, and {\tt 6.0} the
electrons/DN in the image.  Note that {\bf aperpol} only has two
aperture/annulus input parameters, the diameter of the star aperture
and the outer diameter of the concentric sky annulus.

The output text file from {\bf aperpol} is very similar to the one from
{\bf aperphot} but contains the polarization values instead of the
photometric values.

\item[Note :] the polarization error calculated in aperpol is incorrect
if your images have more than {\tt 1} coadded exposure.  The input
electrons/DN converts the photon shot-noise on the signal in the
aperture to an associated error but this is only valid (for {\sc
ircam3}) if you have 1 coadd since more than one coadd would give lower
shot-noise even though the counts in the image are the same ({\sc
ircam3} averages coadds unlike {\sc ircam1/2} which added the coadd
images together).  To get the correct photon shot-noise for the
polarization calculated you must divide the electrons/DN value input by
the root of the number of coadds.  Also, if you have averaged several
separate images to get a final set of four waveplate position images
you must also scale the electrons/DN value by the root of the number of
images averaged to make the final set of images.

\item[Usage :] Aperture polarimetry on point-sources

\item[Associated commands :] {\tt \htmlref{polly}{POLLY}},
{\tt \htmlref{polly2}{POLLY2}}, {\tt \htmlref{polcal}{POLCAL}}
{\tt polcal2}

\item[Invocation :]

\begin{quote}{\tt  aperpol }\end{quote}

\end{description}

\hrule
\subsubsection*{\label{APPLYMASK}\xlabel{APPLYMASK}APPLYMASK}

\begin{description}

\item[Description :] Applies a bad pixel mask (image with 0's for good
pixels and 1's for bad pixels) to an image.  In {\bf applymask}, the word
`apply' means that all identified bad pixels in the bad pixel mask are
set to a magic number ({\tt -1.0e-20} in {\sc IrcamDR}).  The output is
an image with these bad pixels set to the magic number.

\item[Usage :] Identifying bad/hot pixels in an image for subsequent mosaic
removal by replacement.

\item[Associated commands :] {\tt \htmlref{mosaic}{MOSAIC}},
{\tt \htmlref{quilt}{QUILT}}, {\tt \htmlref{makemask}{MAKEMASK}}

\item[Invocation :]

\begin{quote}{\tt  applymask }\end{quote}

\end{description}

\hrule
\subsubsection*{\label{ASCIILIST}\xlabel{ASCIILIST}ASCIILIST}

\begin{description}

\item[Description :] Creates a text (ASCII) file containing the pixel
values of all pixels in the input image.  There are two possible
formats for the output ASCII file, the first has as the first line the
number of pixels in X and Y in the image and then on subsequent lines
has a list of pixel values starting from pixel {\tt 1,1} (bottom-left
corner) and incrementing fastest in X.  The second format has three
columns of numbers, column one is the X pixel location of the pixel,
column two is the Y pixel location of the pixel and column three is the
pixel value itself.  The output ASCII file is named by the user.

\item[Usage :] For data conversion between different formats.
\item[Associated commands :] -
\item[Invocation :]

\begin{quote}{\tt  asciilist }\end{quote}

\end{description}

\hrule
\subsubsection*{\label{AUTOMOS}\xlabel{AUTOMOS}AUTOMOS}

\begin{description}

\item[Description :] Program to automatically correct dc level offsets
(sky changes) between a set of images in a mosaic.  {\bf automos} takes as
input:

\begin{enumerate}

\item a list of images, one per line, in a text (ASCII) file,

\item a list of telescope spatial offsets (in arcseconds), again,
one per line in a text (ASCII) file.

\end{enumerate}

Also required as input are the pixel scale (arcseconds/pixel), whether
to use the median, mean or mode as the value considered in overlap
regions, and the number of iterations to use in calculating the
correction values for each image (to obtain a ``flat'' sky).

{\bf automos} first works out the chosen value (median, mean,
mode) in each overlap region (the overlap regions are defined by the
spatial offsets and image sizes) and then works out the optimum
correction for each overlap region. It then applies a fraction of that
correction to each image overlap value (median, mean, mode) and repeats
the process until the specified number of iterations has been reached.
The user has the option to apply the final corrections values to the
input images creating new output images with the same name as the input
images but with the letter {\tt `z'} added the the names.

\item[Usage :] To correct mosaic images for dc sky level changes between
tiles.
\item[Associated commands :] -
\item[Invocation :]

\begin{quote}{\tt automos }\end{quote}

\end{description}

\hrule
\subsubsection*{\label{BINUP}\xlabel{BINUP}BINUP}

\begin{description}

\item[Description :] Bins an input image by a user-specified factor in
X and Y.  The user-specified factor can be different in X and Y.
Binning means that pixels are averaged (hence reducing the spatial
resolution but increasing the signal:noise) and the output image is
smaller in extent by the binning factors.  Two output images are
created by {\bf binup}, the first is the binned output image of lower
spatial resolution, the second is an image of the standard deviations
over the binned region.  This is useful to give an idea of the scatter
of values over an image.

\item[Usage :] To increase S/N in image by degrading spatial resolution.

\item[Associated commands :] {\tt \htmlref{compave}{COMPAVE}},
{\tt \htmlref{compadd}{COMPADD}}, {\tt \htmlref{manic}{MANIC}},
{\tt \htmlref{sqorst}{SQORST}}

\item[Invocation :]

\begin{quote}{\tt  binup }\end{quote}

\end{description}

\hrule
\subsubsection*{\label{BLOCK}\xlabel{BLOCK}BLOCK}

\begin{description}

\item[Description :] Applies a block smooth to an image.
Pixels in the output image are therefore less noisy than in the input
image.  The input parameter to {\bf block} is the pixel size of the
block smooth applied.  The larger the pixel size the more severe the
smoothing.

\item[Usage :] Smoothing an image
\item[Associated commands :] {\tt \htmlref{gauss}{GAUSS}}
\item[Invocation :]

\begin{quote}{\tt block }\end{quote}

\end{description}

\hrule
\subsubsection*{\label{CADD}\xlabel{CADD}CADD}

\begin{description}

\item[Description :] Adds a user-defined constant to every pixel in an image.

\item[Usage :] To correct dc level of image for sky subtraction

\item[Associated commands :] {\tt \htmlref{cmult}{CMULT}},
{\tt \htmlref{cdiv}{CDIV}}, {\tt \htmlref{csub}{CSUB}}

\item[Invocation :]

\begin{quote}{\tt  cadd  }\end{quote}

\end{description}

\hrule
\subsubsection*{\label{CALCOL}\xlabel{CALCOL}CALCOL}

\begin{description}

\item[Description :] Calculates a colour (\emph{e.g.}, {\tt
J-K}, {\tt H-K} \emph{etc}) image from two magnitude images
(\emph{e.g.}, a {\tt J} magnitude/sq arcsec image and a {\tt K}
magnitude/sq arcsec image).  {\bf calcol} takes a threshold value and
only considers pixels in each image which have magnitude values
brighter than the threshold in both.  This helps eliminates noise due
to poor signal:noise region being considered.  Filter specific
magnitude images can be created using the {\sc IrcamDR} command {\bf
tomag} which converts intensity images to magnitude images using an
input zeropoint.  For {\bf calcol} to work correctly, the two filter
images need to be shifted to a common centre (using say, the {\sc
IrcamDR} command {\bf shift2}) before running {\bf calcol}.

\item[Usage :] Creating a colour image from two magnitude images.

\item[Associated commands :] {\tt \htmlref{tomag}{TOMAG}},
{\tt \htmlref{shift2}{SHIFT2}}

\item[Invocation :]

\begin{quote}{\tt calcol }\end{quote}

\end{description}

\hrule
\subsubsection*{\label{CDIV}\xlabel{CDIV}CDIV}

\begin{description}

\item[Description :] Divides all pixels in an input image by a constant value.

\item[Usage :] Scaling an image by division.

\item[Associated commands :] {\tt \htmlref{cadd}{CADD}},
{\tt \htmlref{cmult}{CMULT}}, {\tt \htmlref{csub}{CSUB}}

\item[Invocation :]

\begin{quote}{\tt cdiv }\end{quote}

\end{description}

\hrule
\subsubsection*{\label{CENTROID}\xlabel{CENTROID}CENTROID}

\begin{description}

\item[Description :] Calculates an accurate centroid
position of a stellar source or stellar-like feature in an image.  It
takes as input a guess of the position of the object and then
calculates the accurate position.

\item[Usage :] Determining spatial offsets between images in a mosaic.
\item[Associated commands :] {\tt \htmlref{cent1}{CENT1}},
{\tt \htmlref{cent2}{CENT2}}
\item[Invocation :]

\begin{quote}{\tt centroid }\end{quote}

\end{description}

\hrule
\subsubsection*{\label{CHPIX}\xlabel{CHPIX}CHPIX}

\begin{description}

\item[Description :] Changes the value of a user-defined
pixel or pixels within an image.  The pixel is defined by its X and Y
pixel coordinates ({\tt 1,1} is the bottom-left corner of the image).
The new value for the pixel is defined by the user also.

\item[Usage :] Changing the counts in a bad/hot pixel.

\item[Associated commands :] {\tt \htmlref{zaplin}{ZAPLIN}},
{\tt \htmlref{glitchmark}{GLITCHMARK}}, {\tt \htmlref{glitch}{GLITCH}}

\item[Invocation :]

\begin{quote}{\tt  chpix }\end{quote}

\end{description}

\hrule
\subsubsection*{\label{CMULT}\xlabel{CMULT}CMULT}

\begin{description}

\item[Description :] Multiplies every pixel in an image by a constant value.

\item[Usage :] Scaling images.

\item[Associated commands :] {\tt \htmlref{cadd}{CADD}},
{\tt \htmlref{cdiv}{CDIV}}, {\tt \htmlref{csub}{CSUB}}

\item[Invocation :]

\begin{quote}{\tt cmult }\end{quote}

\end{description}

\hrule
\subsubsection*{\label{COLCYCLE}\xlabel{COLCYCLE}COLCYCLE}

\begin{description}

\item[Description :] Creates a colour table (LUT) which contains
another pre-defined colour table (LUT) {\bf n} times where {\tt n} is
defined by the user.  Thus, a colour table that goes from black to
white cycled 3 times would go from black to white then black to white
then black to white.  The input colour table is defined by the user
({\sc IrcamDR} colour tables are in the directory {\tt \$LIRCAMDIR}).
The output colour table is named by the user. {\sc IrcamDR} colour
tables are NDF images with dimensions {\tt 3,256}; the {\tt 3} refers
to RGB colour intensities.  The range of values for each colour (pen)
is {\tt 0-1}.  {\sc IrcamDR} colour tables are the same format as
FIGARO LUTs and are interchangeable.

\item[Usage :] Stretching display of an image by displaying multiple colour
tables.

\item[Associated commands :] {\tt \htmlref{coltab}{COLTAB}},
{\tt \htmlref{writelut}{WRITELUT}}, {\tt \htmlref{crecolt}{CRECOLT}}

\item[Invocation :]

\begin{quote}{\tt  colcycle }\end{quote}

\end{description}

\hrule
\subsubsection*{\label{COLMED}\xlabel{COLMED}COLMED}

\begin{description}

\item[Description :] Median filters down the columns of an image
and produces a N$\times$1 output image (where N is the original X
dimension of the input image) with containing the median values for
each column.  The user can define an exclusion region (a range of rows)
which is not included in the median calculation, hence, if there is a
bright object near the centre of the array, that region can be
excluded.

\item[Usage :] Can be useful for removing residual background structure in an
image.  Use {\bf colmed} to define the column structure via the median value
in the column then {\bf ygrow} to grow the N$\times$ image to N$\times$M
where N and M are the original X and Y dimensions of the input image.
The result can be subtracted from the original image using the command
{\bf sub}.

\item[Associated commands :] {\tt \htmlref{ygrow}{YGROW}},
{\tt \htmlref{sub}{SUB}}, {\tt \htmlref{rowmed}{ROWMED}},
{\tt \htmlref{xgrow}{XGROW}}

\item[Invocation :]

\begin{quote}{\tt  colmed} \\
or \\
{\tt colmed image1 out1 y 50 150 }
\end{quote}

\begin{itemize}

\item {\tt image1 } is the input N$\times$M image
\item {\tt out1 } is the output N$\times$1 image containing the medians
\item {\tt y } means use an exclusion region ({\tt n} would mean no exclusion
 region)
\item {\tt 50 150 } are the range of rows in each column to be excluded
 from the median calculation
\end{itemize}

\end{description}

\hrule
\subsubsection*{\label{COMPADD}\xlabel{COMPADD}COMPADD}

\begin{description}

\item[Description :] Reduces the X,Y dimensions of an image by an
integer amount using pixel addition to form new (larger) pixels.  The
scaling factor used is input by the user and is the same for both X and
Y.

\item[Usage :] To increase S/N in an image by reducing spatial resolution.

\item[Associated commands :] {\tt \htmlref{compave}{COMPAVE}},
{\tt \htmlref{compick}{COMPICK}}, {\tt \htmlref{compress}{COMPRESS}},
{\tt \htmlref{binup}{BINUP}}, {\tt \htmlref{manic}{MANIC}},
{\tt \htmlref{sqorst}{SQORST}}

\item[Invocation :]

\begin{quote}{\tt  compadd }\end{quote}

\end{description}

\hrule
\subsubsection*{\label{COMPAVE}\xlabel{COMPAVE}COMPAVE}

\begin{description}

\item[Description :] Reduces the X,Y dimensions of an image by an
integer amount using pixel averaging to form new (larger) pixels.  The
scaling factor used is input by the user and is the same for both X and
Y.

\item[Usage :] To increase S/N in an image by reducing spatial resolution.

\item[Associated commands :] {\tt \htmlref{compadd}{COMPADD}},
{\tt \htmlref{compick}{COMPICK}}, {\tt \htmlref{compress}{COMPRESS}},
{\tt \htmlref{binup}{BINUP}}, {\tt \htmlref{manic}{MANIC}},
{\tt \htmlref{sqorst}{SQORST}}

\item[Invocation :]

\begin{quote}{\tt  compave }\end{quote}

\end{description}

\hrule
\subsubsection*{\label{COMPICK}\xlabel{COMPICK}COMPICK}

\begin{description}

\item[Description :] Reduces the X,Y dimensions of an image by an
integer amount using pixel selection to form new (larger) pixels.  The
scaling factor used is input by the user and is the same for both X and
Y.  By pixel selection, we mean that say, every second or third pixel
is put in the output image rather than those pixels being summed or
averaged.

\item[Usage :] To decrease size of image but retain original S/N.

\item[Associated commands :] {\tt \htmlref{compadd}{COMPADD}},
{\tt \htmlref{compave}{COMPAVE}}, {\tt \htmlref{compress}{COMPRESS}},
{\tt \htmlref{binup}{BINUP}}, {\tt \htmlref{manic}{MANIC}},
{\tt \htmlref{sqorst}{SQORST}}

\item[Invocation :]

\begin{quote}{\tt  compick }\end{quote}

\end{description}

\hrule
\subsubsection*{\label{COMPRESS}\xlabel{COMPRESS}COMPRESS}

\begin{description}

\item[Description :] Reduces the X,Y dimensions of an image by an
integer amount using pixel averaging to form new (larger) pixels.  The
scaling factor used is input by the user and is the same for both X and
Y.

\item[Usage :] To decrease size of image and improve S/N by degrading
spatial resolution.

\item[Associated commands :] {\tt \htmlref{compadd}{COMPADD}},
{\tt \htmlref{compave}{COMPAVE}}, {\tt \htmlref{compick}{COMPICK}},
{\tt \htmlref{binup}{BINUP}}, {\tt \htmlref{manic}{MANIC}},
{\tt \htmlref{sqorst}{SQORST}}

\item[Invocation :]

\begin{quote}{\tt  compress }\end{quote}

\end{description}

\hrule
\subsubsection*{\label{CRECOLT}\xlabel{CRECOLT}CRECOLT}

\begin{description}

\item[Description :] Creates an {\sc IrcamDR}/FIGARO type colour table
(LUT) by selecting colours (from standard colour list) and pens
(between {\tt 1-250}).  The output colour table image (NDF) is named by
the user.  The {\sc IrcamDR} released colour tables ({\tt col1-49})
were all created with {\bf crecolt}.  Colour tables created with {\bf
crecolt} can also be used in the latest FIGARO (NDF) application called
{\bf colour}.

\item[Usage :] Creating user-specific colour tables (LUTs)
\item[Associated commands :] -
\item[Invocation :]

\begin{quote}{\tt  crecolt }\end{quote}

\end{description}

\hrule
\subsubsection*{\label{CREFRAME}\xlabel{CREFRAME}CREFRAME}

\begin{description}

\item[Description :] Used to create a test image with either gaussian
distribution of stars, random numbers ({\tt 0-1} or {\it min-max}),
noise with user-specified sigma and mean, poissionian noise about mean,
ramps (left to right, right to left, top to bottom or bottom to top)
with user-defined range, flat (constant value) or blank, images.
Useful to test software or simulate real data.

\item[Usage :] To create images for testing software or simulating data.
\item[Associated commands :] -
\item[Invocation :]

\begin{quote}{\tt  creframe }\end{quote}

\end{description}

\hrule
\subsubsection*{\label{CREQUILT}\xlabel{CREQUILT}CREQUILT}

\begin{description}

\item[Description :] Creates an ASCII (text) file in the correct format
for input into the mosaic assembly {\sc IrcamDR} program {\bf quilt}.
{\bf crequilt} takes as input an ASCII (text) file with a list of {\tt
N} image names, an ASCII (text) file with a list of {\tt N} spatial
offsets (as used to take the data \emph{i.e.}, fed to the telescope -
first line should be {\tt 0 0} \emph{e.g.}, the original of the mosaic
with subsequent offsets relative to that), the pixel scale used, and a
title string for the quilt file.  The name of the output quilt format
file is also specified by the user.

\item[Usage :] Creating {\bf quilt} format mosaic files from image list
file and telescope offset list file

\item[Associated commands :] {\tt \htmlref{quilt}{QUILT}}

\item[Invocation :]

\begin{quote}{\tt  crequilt }\end{quote}

\end{description}

\hrule
\subsubsection*{\label{CSFIT}\xlabel{CSFIT}CSFIT}

\begin{description}

\item[Description :] Fits a centro-symmetric polarization pattern to a
polarization map image pair (percentage polarization, polarization
position angle) as created by the {\sc IrcamDR} programs {\bf polcal}
and {\bf polcal2}.  The fitting allows the user to define a region of
the input images to be scanned pixel by pixel for the centre of the
centro-symmetric polarization pattern and also the minimum percentage
polarization level to be considered in the fit (values of {\tt p} below
that specified level are not considered since they likely to too small
a S/N).  The sum of the squares of the residuals are calculated over
the map for each location of the centre point of the pattern and the
one with the best-fit (smallest residuals) is considered the best-fit
centro-symmetric pattern.  The polarization position angle image will
have to have the position angles transformed to be on the equatorial
coordinate system prior to input into {\bf csfit}.  This can be
achieved using the commands {\bf cadd} and {\bf thetafix}.

In the {\sc IrcamDR} software, the percentage polarization image has
values between 0 and 100\% and the polarization position angle image
has values between 0 and 180 degrees.


\item[Usage :] Statistically determining the source of the radiation in a
polarization map.

\item[Associated commands :] {\tt \htmlref{polcal}{POLCAL}},
{\tt polcal2}, {\tt \htmlref{csgen}{CSGEN}}

\item[Invocation :]

\begin{quote}{\tt  csfit }\end{quote}

\end{description}

\hrule
\subsubsection*{\label{CSGEN}\xlabel{CSGEN}CSGEN}

\begin{description}

\item[Description :] Generates a centro-symmetric polarization position
angle image with a user-defined centre (pixel).  The output is a
polarization position angle image with values of theta in the
equatorial coordinate system, between 0 and 180 degrees.

In the {\sc IrcamDR} software, the percentage polarization image has
values between 0 and 100\% and the polarization position angle image
has values between 0 and 180 degrees.

\item[Usage :] To compare observations with theoretical results.

\item[Associated commands :] {\tt \htmlref{csfit}{CSFIT}},
{\tt \htmlref{polcal}{POLCAL}}, {\tt polcal2},
{\tt \htmlref{thetafix}{THETAFIX}}

\item[Invocation :]

\begin{quote}{\tt  csgen }\end{quote}

\end{description}

\hrule
\subsubsection*{\label{CSUB}\xlabel{CSUB}CSUB}

\begin{description}

\item[Description :] Subtracts a constant from each pixel in an image.
The result of the subtraction is stored in the output image.

\item[Usage :] Sky subtraction, dc-level correction \emph{etc.}

\item[Associated commands :] {\tt \htmlref{cadd}{CADD}},
{\tt \htmlref{cmult}{CMULT}}, {\tt \htmlref{cdiv}{CDIV}}

\item[Invocation :]

\begin{quote}{\tt  csub }\end{quote}

\end{description}

\hrule
\subsubsection*{\label{DEFGRAD}\xlabel{DEFGRAD}DEFGRAD}

\begin{description}

\item[Description :] Analyses an input image and defines an output
image containing a gradient defined from the first and last {\tt N}
columns or rows of the input image.  The median value of the pixels in
each column or row in the defined range of columns or rows is taken at
both the top and bottom (for columns) or right and left (for rows) and
the values of all pixels along columns or rows are set by linear
interpolation between these end values. The output image can be
subtracted form the input image to remove any background gradients in
the input image.

\item[Usage :] Removing background gradients along columns or rows in an image.

\item[Associated commands :] {\tt \htmlref{colmed}{COLMED}},
{\tt \htmlref{rowmed}{ROMED}}

\item[Invocation :]

\begin{quote}{\tt  defgrad }\end{quote}

\end{description}

\hrule
\subsubsection*{\label{DEVFCS}\xlabel{DEVFCS}DEVFCS}

\begin{description}

\item[Description :] Calculates the deviation of a pixel in an
polarization position angle image (calculated using {\bf polcal} or
{\bf polcal2}) from centro-symmetry with the centre of the
centro-symmetric pattern generated defined by the user.  Hence, once
the centre of a pattern is known (in terms of the pixel position in an
image) using, say, the command {\bf csfit}, the deviation of a position
angle from centro-symmetry at any other pixel position in the image can
be calculated using that information and the pixel position to be
studied.

\item[Usage :] Calculating deviation from centro-symmetry of pixels in a
polarization map.

\item[Associated commands :] {\tt \htmlref{csfit}{CSFIT}},
{\tt \htmlref{csgen}{CSGEN}}

\item[Invocation :]

\begin{quote}{\tt  devfcs }\end{quote}

\end{description}

\hrule
\subsubsection*{\label{DIST}\xlabel{DIST}DIST}

\begin{description}

\item[Description :] Given two pixels in an image and the pixel scale,
dist calculates the offsets between in arcseconds and also the RA,Dec
of the second position given the RA,Dec of the first.

\item[Note :] There is some doubt that {\bf dist} actually works correctly
for all position angles of the two pixel positions in terms of RA,Dec
(you need to check it yourself).

\item[Usage :] Calculating offset and position (RA,Dec) of second point in
an image given the position (RA,Dec) of a first point
\item[Associated commands :] {\tt \htmlref{cent2}{CENT2}}
\item[Invocation :]

\begin{quote}{\tt  dist }\end{quote}

\end{description}

\hrule
\subsubsection*{\label{DIV}\xlabel{DIV}DIV}

\begin{description}

\item[Description :] Divides an image by another image pixel by pixel.
The two images are named by the user and the output is also named by
the user.

\item[Usage :] Manual flat-fielding for example.

\item[Associated commands :] {\tt \htmlref{add}{ADD}},
{\tt \htmlref{mult}{MULT}}, {\tt \htmlref{sub}{SUB}}

\item[Invocation :]

\begin{quote}{\tt  div }\end{quote}

\end{description}

\hrule
\subsubsection*{\label{EXP10}\xlabel{EXP10}EXP10}

\begin{description}

\item[Description :] Take every pixel in an image as the exponent {\it x} in
the calculation $10^{x}$.  The output image represents the result of the
calculation.

\item[Usage :] -

\item[Associated commands :] {\tt \htmlref{expe}{EXPE}},
{\tt \htmlref{expon}{EXPON}}

\item[Invocation :]

\begin{quote}{\tt  exp10 }\end{quote}

\end{description}

\hrule
\subsubsection*{\label{EXPE}\xlabel{EXPE}EXPE}

\begin{description}

\item[Description :] Take every pixel in an image as the exponent {\it x} in
the calculation $e^{x}$.  The output image represents the result of the
calculation.

\item[Usage :] -

\item[Associated commands :] {\tt \htmlref{exp10}{EXP10}},
{\tt \htmlref{expon}{EXPON}}

\item[Invocation :]

\begin{quote}{\tt  expe }\end{quote}

\end{description}

\hrule
\subsubsection*{\label{EXPON}\xlabel{EXPON}EXPON}

\begin{description}

\item[Description :] Take every pixel in an image as the exponent {\it x} in the
calculation $y^{x}$ where {\it y} is input by the user. The output image
represents the result of the calculation.

\item[Usage :] -

\item[Associated commands :] {\tt \htmlref{exp10}{EXP10}},
{\tt \htmlref{expe}{EXPE}}

\item[Invocation :]

\begin{quote}{\tt  expon }\end{quote}

\end{description}

\hrule
\subsubsection*{\label{FCOADD}\xlabel{FCOADD}FCOADD}

\begin{description}

\item[Description :] Coadds (sums and averages) a sequence of NDF
images.  The sequence can be defined as a filename prefix, range of
numbers and filename suffix, or a list of named images.  The result is
therefore the average of the set of images input and is named by the
user.

\item[Usage :] Summing/averaging a set of images to increase S/N, for example

\item[Associated commands :] {\tt \htmlref{add}{ADD}}

\item[Invocation :]

\begin{quote}{\tt  fcoadd }\end{quote}

\end{description}

\hrule
\subsubsection*{\label{FINDPEAK}\xlabel{FINDPEAK}FINDPEAK}

\begin{description}

\item[Description :] Finds the peak signal/flux in a user-defined
sub-area if an image, The sub-area is defined by its start X,Y pixel
coordinates and the X,Y box size (in pixels).  The peak value and
location of the peak value is returned to the user.

\item[Usage :] Finding stars, for example.

\item[Associated commands :] {\tt \htmlref{cstats}{CSTATS}}

\item[Invocation :]

\begin{quote}{\tt  findpeak }\end{quote}

\end{description}

\hrule
\subsubsection*{\label{FLIP}\xlabel{FLIP}FLIP}

\begin{description}

\item[Description :] Flips an image either horizontally (H) or
vertically (V). An horizontal {\bf flip} flips around a vertical axis
through the centre of the image.  A vertical {\bf flip} flips an image
through a horizontal axis through the centre of the image.  The output
image is named by the user.

\item[Usage :] Putting North to the top and East to the left in some images
\item[Associated commands :] -
\item[Invocation :]

\begin{quote}{\tt flip} \\
or \\
{\tt flip im1 h out1 }
\end{quote}

\begin{itemize}

\item {\tt im1 } is the input image
\item {\tt h } is a horizontal flip ({\tt v} would be a vertical flip)
\item {\tt out1 } is the output image
\end{itemize}

\end{description}

\hrule
\subsubsection*{\label{GAUSS}\xlabel{GAUSS}GAUSS}

\begin{description}

\item[Description :] Smooths an image using a gaussian filter.  The
sigma of the gaussian used is defined by the user (in terms of
pixels).  The area that the gaussian works over (the area around each
pixel considered) is also input by the user.

\item[Usage :] Smoothing images, increasing S/N
\item[Associated commands :] {\tt \htmlref{block}{BLOCK}}
\item[Invocation :]

\begin{quote}{\tt  gauss }\\
or \\
{\tt gauss im1 1.5 7 out1 }
\end{quote}

\begin{itemize}

\item {\tt im1 } is the input image to be smoothed
\item {\tt 1.5 } is the sigma of the gaussian applied
\item {\tt 7 } is the area around each pixels gaussian considers
\item {\tt out1 } is the output smoothed image
\end{itemize}

\end{description}

\hrule
\subsubsection*{\label{GAUSSTH}\xlabel{GAUSSTH}GAUSSTH}

\begin{description}

\item[Description :] As {\bf gauss} but only smooths pixels that have a
signal smaller than a user-specified value.  This allows users to
smooth low signal (low surface brightness) areas and not smooth bright
objects/regions.

\item[Usage :] Smoothing faint features, not brighter object
\item[Associated commands :] {\tt \htmlref{gauss}{GAUSS}}
\item[Invocation :]

\begin{quote}{\tt  gaussth }\\
or \\
{\tt gaussth im1 1.5 7 8000 out1 }
\end{quote}

\begin{itemize}

\item {\tt im1 } is the input image to be smoothed
\item {\tt 1.5 } is the sigma of the gaussian applied
\item {\tt 7 } is the area around each pixels gaussian considers
\item {\tt 8000 } is the threshold below which pixels are smoothed
\item {\tt out1 } is the output smoothed image
\end{itemize}\end{description}

\hrule
\subsubsection*{\label{GLITCH}\xlabel{GLITCH}GLITCH}

\begin{description}

\item[Description :] Removes bad/hot pixels or changes the value of any
other pixel in a user named image.  {\bf glitch} has three modes of
determining the pixel values to be modified, first they can be listed
in an ASCII (text) file with the pixel coordinates one per line with
each line containing X Y pixel coordinates (\emph{e.g.}, {\tt 10 20}
for pixel {\tt 10,20}), second they can be input from the terminal by
the user one by one, and third {\bf glitch} will perform an auto-search
and consider any pixel with a user-specified value (a magic number say)
as bad.  The default {\sc IrcamDR} magic number value is {\tt
-1.0e-20}.  Auto-search is useful for removing the fixed bad pixels
after the bad pixel mask has been applied using the {\sc IrcamDR}
command {\bf applymask}.  {\bf glitch} removes bad/hot pixels by
replacement of them by the median value of the {\tt 8} surrounding
pixels.  The modified image is written to a user named output image.

\item[Usage :] Removing bad/hot pixels or fudging data.
\item[Associated commands :] {\tt \htmlref{glitchmark}{GLITCHMARK}},
{\tt \htmlref{applymask}{APPLYMASK}}
\item[Invocation :]

\begin{quote}{\tt  glitch }\end{quote}

\end{description}

\hrule
\subsubsection*{\label{HISTEQ}\xlabel{HISTEQ}HISTEQ}

\begin{description}

\item[Description :] Performs an histogram equalization of an image and
writes the output to a new image.  An histogram equalization is a
technique for enhancing faint features in an image, for example.  A
histogram of the equalized image will have the same number of pixels in
each intensity range whereas in the input image the originally measured
intensity structure will be present.

\item[Usage :] Enhancement of faint features with respect to bright features.
\item[Associated commands :] -
\item[Invocation :]

\begin{quote}{\tt  histeq }\end{quote}

\end{description}

\hrule
\subsubsection*{\label{HISTGEN}\xlabel{HISTGEN}HISTGEN}

\begin{description}

\item[Description :] Generates a histogram of an image using
user-defined number of bins and {\it max}, {\it min} range.  The output
is an image of dimensions N$\times$2 where {\tt N} is the number of bins
specified.  The row of numbers are the number of pixels in the bin
while the second row of numbers are the bin centre intensities.  The
image can be converted to ASCII (text) using the command {\bf asciilist}.

\item[Usage :] Producing histogram of an image
\item[Associated commands :] {\tt \htmlref{asciilist}{ASCIILIST}}
\item[Invocation :]

\begin{quote}{\tt  histgen }\end{quote}

\end{description}

\hrule
\subsubsection*{\label{HISTO}\xlabel{HISTO}HISTO}

\begin{description}

\item[Description :] Returns statistics on any image or sub-area of any
image.  The statistics include the {\it max},{\it min} value and the
location of the {\it max},{\it min} values in the image together with
the mean, median and mode of the pixels in the image.

\item[Usage :] Statistics

\item[Associated commands :] {\tt \htmlref{stats}{STATS}},
{\tt \htmlref{cstats}{CSTATS}}

\item[Invocation :]

\begin{quote}{\tt  histo }\\
or \\
{\tt histo im1 10 20 30 40 }
\end{quote}

\begin{itemize}

\item {\tt im1 } is the image for statistical analysis
\item {\tt 10 20 } are the start X,Y coordinates of the sub-area
\item {\tt 20 30 } are the X,Y size of the sub-area
\end{itemize}\end{description}

\hrule
\subsubsection*{\label{HOTSHOT}\xlabel{HOTSHOT}HOTSHOT}

\begin{description}

\item[Description :] Automatic search and removal of bad/hot pixels
using a sigma cut method.  {\bf hotshot} take in a user-defined image
and several other parameters defining the search sensitivity.  The
first two parameters are the X and Y box size for the search, usually
3$\times$3 is what is required.  This is the size of the sub-area of
the array considered around the current pixel when forming the standard
deviation for the identification of bad/hot pixels.  The next parameter
is the sigma-level of the search.  The default value is {\tt 3}.  This
parameter means that bad/hot pixels are identified when their intensity
is more than 3-sigma (assuming the default is taken) from the mean of
the surrounding {\tt N} pixels where {\tt N} is defined by the X,Y box
size ({\tt N=8} for a 3$\times$3, the pixel under consideration is not
considered in the mean,std calculation).  The next parameter is a
threshold intensity value so that only pixels with values less than the
input threshold are considered at even possibly bad/hot.  This
parameter helps keep star peaks from being removed by {\bf hotshot} in
the auto-search.  The final parameter defines whether to replace a
bad/hot pixel by the mean or median of the pixels in the X,Y search
box.  Once the parameters of the search are defined, {\bf hotshot}
scans the image and determines how many pixels satisfy the badness
criteria.  This number is relayed to the user.  Next, {\bf hotshot}
removes those pixels and re-scans the resulting image for additional
bad/hot pixels.  This iterative search stops when either:

\begin{enumerate}

\item the number of bad/hot pixels is less than 10\% of the original
number of bad/hot pixels,

\item when the number of bad pixels increases between iterations rather
than decreases, or

\item the number of bad/hot pixels in the first scan of the image is zero.

\end{enumerate}

It is useful to subtract the result of {\bf hotshot} from the input
image to {\bf hotshot} since this gives you a map of the bad/hot pixels
removed.

\item[Usage :] Removing isolated bad/hot pixels or cosmic ray events, for
example.

\item[Associated commands :] {\tt \htmlref{glitch}{GLITCH}},
{\tt \htmlref{glitchmark}{GLITCHMARK}}

\item[Invocation :]

\begin{quote}{\tt  hotshot }\end{quote}

\end{description}

\hrule
\subsubsection*{\label{IMCOMB}\xlabel{IMCOMB}IMCOMB}

\begin{description}

\item[Description :] Takes two images as input (images must be same
dimensions) and creates an output image which has a user-defined bad
sub-area of the first image replaced with the same sub-area from the
second image.  This allows users to replace a bad regions of one image
with a good region of another image and create a new output image with
the result.  {\bf imcomb} scales by addition, the region of the second image
to be transplanted into the first image where the sum added is
determined by the median of the pixels surrounding the region to be
implanted.

\item[Usage :] Putting one image inside another.
\item[Associated commands :] -
\item[Invocation :]

\begin{quote}{\tt  imcomb }\end{quote}

\end{description}

\hrule
\subsubsection*{\label{INDEX}\xlabel{INDEX}INDEX}

\begin{description}

\item[Description :] Produces an ASCII (text) of observational
parameters for the old style {\sc ircam1/2} {\sc ircam} container
files.  The output ASCII file is named by the user.

\item[Usage :] Producing an index listing of old {\sc ircam} container file
observations.
\item[Associated commands :] -
\item[Invocation :]

\begin{quote}{\tt  index }\end{quote}

\end{description}

\hrule
\subsubsection*{\label{INSETB}\xlabel{INSETB}INSETB}

\begin{description}

\item[Description :] Sets the pixel values with a user-defined sub-area
of an image to a user-defined value.  The sub-area within which to set
the values is defined by its X,Y start pixel coordinates and the X,Y
box size in pixels.  The output image has the sub-area values set to
the value chosen by the user.

\item[Usage :] Setting a region (box) of an image to a magic number

\item[Associated commands :] {\tt \htmlref{insetc}{INSETC}},
{\tt \htmlref{outsetb}{OUTSETB}}, {\tt \htmlref{outsetc}{OUTSETC}}

\item[Invocation :]

\begin{quote}{\tt  insetb }\end{quote}

\end{description}

\hrule
\subsubsection*{\label{INSETC}\xlabel{INSETC}INSETC}

\begin{description}

\item[Description :] Sets the pixel values with a user-defined circular
area of an image to a user-defined value.  The circle within which to
set the values is defined by its X,Y pixel coordinates and a radius in
pixels. The output image has the sub-area values set to the value
chosen by the user.

\item[Usage :] Setting a region (circle) of an image to a magic number.

\item[Associated commands :] {\tt \htmlref{insetb}{INSETB}},
{\tt \htmlref{outsetb}{OUTSETB}}, {\tt \htmlref{outsetc}{OUTSETC}}

\item[Invocation :]

\begin{quote}{\tt  insetc }\end{quote}

\end{description}

\hrule
\subsubsection*{\label{INTLK}\xlabel{INTLK}INTLK}

\begin{description}

\item[Description :] Prints an integer representation of any sub-area
of any image to the terminal.  Useful to inspect regions of images.

\item[Usage :] Identification of bad pixels or star positions?
\item[Associated commands :] {\tt \htmlref{look}{LOOK}}
\item[Invocation :]

\begin{quote}{\tt  intlk }\end{quote}

\end{description}

% \hrule
% \subsubsection*{\label{JSDCLEAN}\xlabel{JSDCLEAN}JSDCLEAN}
%
% \begin{description}
%
% \item[Description :] Under development.  Will eventually deconvolve PSF
% from images using modified clean algorithm developed from James Dunlop
% (JSD) of ROE.
%
% \item[Usage :] Removing of PSF from images for enhanced spatial resolution.
%
% \item[Associated commands :] -
%
% \item[Invocation :] -
%
% \end{description}

\hrule
\subsubsection*{\label{LAPLACE}\xlabel{LAPLACE}LAPLACE}

\begin{description}

\item[Description :] Applies a Laplacian convolution as an edge
detector to an image. This routine works out the Laplacian of an input
image and subtracts it from the original image to create a new output
image. This can be done an integer number of times, as given by an
input parameter. This operation can be thought of as a convolution by

\begin{verbatim}
                          -N   -N   -N
                          -N   +8N  -N
                          -N   -N   -N
\end{verbatim}

where {\tt N} is the integer number of times the Laplacian is
subtracted. This convolution is used as a unidirectional edge detector.
Areas where the input image is flat come out as zero in the output.

\item[Usage :] Edge detection
\item[Associated commands :] -
\item[Invocation :]

\begin{quote}{\tt  laplace }\end{quote}

\end{description}

\hrule
\subsubsection*{\label{LINCONT}\xlabel{LINCONT}LINCONT}

\begin{description}

\item[Description :] Applies a linearization algorithm to the images in
an old {\sc ircam1/2} format container file.  The linearization itself
is the application of a $5^{th}$ order polynomial to make the signal vs.
integration time curve linear.  The process reads linearization
coefficients from a ASCII (text) file, by default called {\tt
lincoeff.list}.  The linearization of {\sc ircam1/2}3 data changes in
July 1988 and so different linearization coefficients are used for data
before and after that time.

The ASCII files in the {\tt \$LIRCAMDIR} directory called:

\begin{quote}
{\tt lincoeff\_fpa118\_prejul88.list}, \\
{\tt lincoeff\_fpa118\_prejul88\_ndr.list}, \\
{\tt lincoeff\_\-fpa118\_postjul88.list}, \\
{\tt lincoeff\_fpa118\_postjul88\_ndr.list}, \\
{\tt lincoeff\_fpa175.\-list}, and \\
{\tt lincoeff\_fpa175\_ndr.list}
\end{quote}

should be used depending on the time of observations, the array in use,
and whether ndr readout was being used.  All this info in in the header
of the {\sc ircam} container file and can be accessed using the index
command. An example of the format of a linearization coefficient file
is:

\begin{small}
\begin{verbatim}
      Title - IRCAM linearization coefficients - FPA175 - Post 01-Jun-1989 NDR
      -1.985e-6
      0.0
      0.0
      25000.0
      50000.0
\end{verbatim}
\end{small}

\item[Usage :] Linearising images in an old {\sc ircam1/2} format container
file.

\item[Associated commands :] {\tt \htmlref{linimag\_ndr}{LINIMAG_NDR}}

\item[Invocation :]

\begin{quote}{\tt  lincont }\end{quote}

\end{description}

\hrule
\subsubsection*{\label{LINIMAG_NDR}\xlabel{LINIMAG_NDR}LINIMAG\_NDR}

\begin{description}

\item[Description :] Linearizes individual images from an old {\sc ircam1/2}
style container file.  Linearization is as in lincont command above.

\item[Usage :] Linearization of {\sc ircam1/2} data images.

\item[Associated commands :] {\tt \htmlref{lincont}{LINCONT}}

\item[Invocation :]

\begin{quote}{\tt  linimag\_ndr }\end{quote}

\end{description}

\hrule
\subsubsection*{\label{LOG10}\xlabel{LOG10}LOG10}

\begin{description}

\item[Description :] Takes the log to the base 10 of all the pixels in
an image and writes the result to an output image named by the user.

\item[Usage :] Intensity-magnitude conversion.
\item[Associated commands :] {\tt \htmlref{loge}{LOGE}},
{\tt \htmlref{logar}{LOGAR}}
\item[Invocation :]

\begin{quote}{\tt  log10 }\end{quote}

\end{description}

\hrule
\subsubsection*{\label{LOGAR}\xlabel{LOGAR}LOGAR}

\begin{description}

\item[Description :] Takes the log to a user-defined base of all the
pixels in an image and writes the result to an output image named by
the user.

\item[Usage :] -
\item[Associated commands :] {\tt \htmlref{loge}{LOGE}},
{\tt \htmlref{log10}{LOG10}}
\item[Invocation :]

\begin{quote}{\tt  logar }\end{quote}

\end{description}

\hrule
\subsubsection*{\label{LOGE}\xlabel{LOGE}LOGE}

\begin{description}

\item[Description :] Takes the log to the base e of all the pixels in
an image and writes the result to an output image named by the user.

\item[Usage :] -
\item[Associated commands :] {\tt \htmlref{logar}{LOGAR}},
{\tt \htmlref{log10}{LOG10}}
\item[Invocation :]

\begin{quote}{\tt  loge }\end{quote}

\end{description}

\hrule
\subsubsection*{\label{LOOK}\xlabel{LOOK}LOOK}

\begin{description}

\item[Description :] Writes the pixel values for a user-defined
sub-area of an image to the terminal. The sub-area is defined by the
X,Y start pixel and the X,Y sub-area size in pixels.

\item[Usage :] Inspection of images and features.
\item[Associated commands :] {\tt \htmlref{intlk}{INTLK}}
\item[Invocation :]

\begin{quote}{\tt  look }\end{quote}

\end{description}

\hrule
\subsubsection*{\label{MAKEBAD}\xlabel{MAKEBAD}MAKEBAD}

\begin{description}

\item[Description :] Scans an input image and creates a {\bf glitch}
style bad pixel list using an n-sigma cut on the image.  The n-sigma
cut (where {\tt n} is input by the user) determines a bad pixel to be
any pixel that has a value larger or smaller than n-sigma (1-sigma is
the standard deviation of all the pixels in the image) on the mean of
the whole image.  Useful to work on flat-fields or darks to determine
bad/hot pixels in an image.  The output ASCII (text) file is in the
format accepted by the program glitch and has one title line and then a
list of X Y pixel coordinates for the bad/hot pixels identified.

\item[Usage :] Creating bad pixel lists.
\item[Associated commands :] {\tt \htmlref{makemask}{MAKEMASK}}
\item[Invocation :]

\begin{quote}{\tt  makebad }\end{quote}

\end{description}

\hrule
\subsubsection*{\label{MAKEGLITCH}\xlabel{MAKEGLITCH}MAKEGLITCH}

\begin{description}

\item[Description :] Converts a bad pixel mask image into a bad pixel
ASCII (text) list file in the format accepted by the program {\bf
glitch}.  The mask can be created by the program {\bf makemask} or any
other way; good pixels have the value 0 and bad pixels have the value
1.

\item[Usage :] Converting a bad pixel mask to bad pixel list file.

\item[Associated commands :] {\tt \htmlref{makemask}{MAKEMASK}},
{\tt \htmlref{makebad}{MAKEBAD}}, {\tt \htmlref{glitch}{GLITCH}}

\item[Invocation :]

\begin{quote}{\tt  makeglitch }\end{quote}

\end{description}

\hrule
\subsubsection*{\label{MAKEMASK}\xlabel{MAKEMASK}MAKEMASK}

\begin{description}

\item[Description :] Scans an input image and creates a bad pixel mask
image using an n-sigma cut on the image.  The n-sigma cut (where {\tt n} is
input by the user) determines a bad pixel to be any pixel that has a
value larger or smaller than n-sigma (1-sigma is the standard deviation
of all the pixels in the image) on the mean of the whole image.  Useful
to work on flat-fields or darks to determine bad/hot pixels in an
image.  The output image has good pixels set to 0 and bad/hot pixels
set to 1.  The mask can be applied to an image using the command
{\bf applymask}.  Once applied, the bad pixels defined by the mask can be
removed by either mosaicing ({\bf quilt}, {\bf mosaic}) or replacement by the
median of the surrounding pixels using {\bf glitch}.

\item[Usage :] Creating bad pixel masks.

\item[Associated commands :] {\tt \htmlref{makebad}{MAKEBAD}},
{\tt \htmlref{makeglitch}{MAKEGLITCH}}, {\tt \htmlref{applymask}{APPLYMASK}}

\item[Invocation :]

\begin{quote}{\tt  makemask }\end{quote}

\end{description}

\hrule
\subsubsection*{\label{MANIC}\xlabel{MANIC}MANIC}

\begin{description}

\item[Description :] Writes all or part of a 1, 2 or 3 dimensional
NDF image to an output image of 1, 2 or 3 dimensions. Windows may be
set in any of the dimensions of the input image. All or part of the
input image may be projected on to any of the rectangular planes or
axes of the input before being written to an output image; or a 1 or 2
dimensional image may be grown to more dimensions to fill an output
image. Many output images, each of a different configuration if
required, may be extracted from a single input image with one call to
the routine.

\item[Usage :] Re-sizing, dimensioning of images

\item[Associated commands :] {\tt \htmlref{compadd}{COMPADD}},
{\tt \htmlref{compave}{COMPAVE}}, {\tt \htmlref{compress}{COMPRESS}},
{\tt \htmlref{binup}{BINUP}}, {\tt \htmlref{sqorst}{SQORST}}

\item[Invocation :]

\begin{quote}{\tt  manic }\end{quote}

\end{description}

\hrule
\subsubsection*{\label{MANYCOL}\xlabel{MANYCOL}MANYCOL}

\begin{description}

\item[Description :] Takes {\tt n} input {\sc IrcamDR} colour tables
(LUTs) and combines them by placing them in the input sequence given in
an output colour table with suitable interpolation.  Allows the user to
combine several colour tables.

\item[Usage :] Colour table manipulation
\item[Associated commands :] {\tt \htmlref{coltab}{COLTAB}},
{\tt \htmlref{crecolt}{CRECOLT}}, {\tt \htmlref{writelut}{WRITELUT}}
\item[Invocation :]

\begin{quote}{\tt  manycol }\end{quote}

\end{description}

\hrule
\subsubsection*{\label{MED3D}\xlabel{MED3D}MED3D}

\begin{description}

\item[Description :] Median filters through a stack of image to produce
a final image that contains median of each input pixel in the stack.
The stack of input images are input using an ASCII (text) file with the
image names listed one per line.  {\bf med3d} has various parameters
associated with monitoring what is taking place and also (at the end)
an option to normalize the output image to have a median (over all the
pixels in it) of unity.  This is very useful when using {\bf med3d} to
produce a flat-field image which should have a median of unity.  The
output median filtered image is named by the user.

\item[Usage :] Creating median filtered flat-fields.
\item[Associated commands :] -
\item[Invocation :]

\begin{quote}{\tt  med3d }\end{quote}

\end{description}

\hrule
\subsubsection*{\label{MEDIAN}\xlabel{MEDIAN}MEDIAN}

\begin{description}

\item[Description :] Filters (spatially) an image.  {\bf median}
filters the input image with a Weighted Median Filter (WMF) of a type
defined by the user, {\tt MODE = -1}, or selected from some predefined
types, {\tt MODE = 0} to {\tt MODE = 7}.  If the WMF is to be user
defined then the parameters for the weighting function, {\tt CORNER},
{\tt SIDE} and {\tt CENTRE}, will be requested.  A stepsize, {\tt
STEP}, has to be specified, this determines the spacing of the elements
of the weighting function. The weighting function has the form :

\begin{small}
\begin{verbatim}
               CORNER  .   SIDE   .  CORNER
                  .          .           .
                SIDE   .  CENTRE  .   SIDE
                  .          .           .
               CORNER  .   SIDE   .  CORNER
\end{verbatim}
\end{small}

The {\tt .} indicates that the weights are separated by {\tt (STEP-1)}
zeros. A threshold, {\tt DIFF}, for replacement of a value by the
median can be set. If the absolute value of the difference between the
actual value and the median is less than {\tt DIFF} the replacement
will not occur.  The way in which the image boundary is dealt with is
given by {\tt BOUND}, the choices are pixel replication or a reflection
about the edge pixels of the image. The WMF can be repeated {\tt NUMIT}
times before the filtered version is written to the output image.

\item[Usage :] Spatial filtering of an image.
\item[Associated commands :] -
\item[Invocation :]

\begin{quote}{\tt  median }\end{quote}

\end{description}

\hrule
\subsubsection*{\label{MOFF}\xlabel{MOFF}MOFF}

\begin{description}

\item[Description :] Determines the best X,Y spatial and
d.c. sky offsets for a pair of mosaic images.  Starting from guessed
initial offsets, the best fit is calculated by minimising the mean
square difference between the values measured at (supposedly) the same
point on the sky in the two images.

\item[Usage :] Mosaicing image tiles in a mosaic.

\item[Associated commands :] {\tt \htmlref{mosaic}{MOSAIC}},
{\tt \htmlref{quilt}{QUILT}}, {\tt \htmlref{csub}{CSUB}},
{\tt \htmlref{accoff}{ACCOFF}} \ldots

\item[Invocation :]

\begin{quote}{\tt  moff }\end{quote}

\end{description}

\hrule
\subsubsection*{\label{MOSAIC}\xlabel{MOSAIC}MOSAIC}

\begin{description}

\item[Description :] Up to 50 non-congruent images may be input, along
with their relative offsets from the first image, and these are then
mosaiced together into one (usually larger) output frame. Where the
frames overlap, a mean value is inserted into the output image. Bad
pixels are optionally handled -- bad data in one input may be replaced
by good data from another.  All the input images have to be of the same
size in mosaic.

\item[Usage :] Mosaicing tiles from a mosaic

\item[Associated commands :] {\tt \htmlref{quilt}{QUILT}},
{\tt \htmlref{moff}{MOFF}}, {\tt \htmlref{accoff}{ACCOFF}} \ldots

\item[Invocation :]

\begin{quote}{\tt  mosaic }\end{quote}

\end{description}

\hrule
\subsubsection*{\label{MOSAIC2}\xlabel{MOSAIC2}MOSAIC2}

\begin{description}

\item[Description :] Same as {\bf mosaic} but mosaics just two images
together.  The input images can be of different dimensions in {\bf mosaic2}.

\item[Usage :] Mosaicing two images together.

\item[Associated commands :] {\tt \htmlref{mosaic}{MOSAIC}},
{\tt \htmlref{quilt}{QUILT}}

\item[Invocation :]

\begin{quote}{\tt  mosaic2 }\end{quote}

\end{description}

\hrule
\subsubsection*{\label{MOSCOR}\xlabel{MOSCOR}MOSCOR}

\begin{description}

\item[Description :] Determines best dc level correction between two
overlapping images. The spatial offsets between the two images are
entered (in pixels) and the dc level offset (using the median or mean)
is calculated in the overlap region.  {\bf moscor} can test the
mosaicing and also correct the second input image with the dc level
correction calculated.

\item[Usage :] Mosaicing tiles of a mosaic.

\item[Associated commands :] {\tt \htmlref{mosaic}{MOSAIC}},
{\tt \htmlref{accoff}{ACCOFF}}, {\tt \htmlref{moff}{MOFF}},
{\tt \htmlref{quilt}{QUILT}} \ldots

\item[Invocation :]

\begin{quote}{\tt  moscor }\end{quote}

\end{description}

\hrule
\subsubsection*{\label{MULT}\xlabel{MULT}MULT}

\begin{description}

\item[Description :] Multiplies the first input image by the second
input image putting the result in the output image.

\item[Usage :] Multiplying two images together.

\item[Associated commands :] {\tt \htmlref{add}{ADD}},
{\tt \htmlref{div}{DIV}}, {\tt \htmlref{sub}{SUB}}

\item[Invocation :]

\begin{quote}{\tt  mult }\\
or \\
{\tt mult im1 im2 out1 }
\end{quote}

\begin{itemize}

\item {\tt im1 } and {\tt im2 } are the input images
\item {\tt out1 } is the output image
\end{itemize}

\end{description}

\hrule
\subsubsection*{\label{NUMB}\xlabel{NUMB}NUMB}

\begin{description}

\item[Description :] Calculates the number of pixels in an image with
values greater than an input value.

\item[Usage :] -
\item[Associated commands :] -
\item[Invocation :]

\begin{quote}{\tt  numb }\end{quote}

\end{description}

\hrule
\subsubsection*{\label{OBSEXT}\xlabel{OBSEXT}OBSEXT}

\begin{description}

\item[Description :] Extracts individual observation images from an old
format {\sc ircam1/2} container file.  The individual images are put
into separate NDF output images.

\item[Usage :] Image extraction from a old format {\sc ircam1/2}
container file.

\item[Associated commands :] -
\item[Invocation :]

\begin{quote}{\tt  obsext }\end{quote}

\end{description}

\hrule
\subsubsection*{\label{OBSLIST}\xlabel{OBSLIST}OBSLIST}

\begin{description}

\item[Description :] Lists one user-defined item in the header of all
the observations in an old format {\sc ircam1/2} container file.  The
list if written to the terminal and optionally to an ASCII file.

\item[Usage :] Listing one item in headers of a set of observations.
\item[Associated commands :] -
\item[Invocation :]

\begin{quote}{\tt  obslist }\end{quote}

\end{description}

\hrule
\subsubsection*{\label{OEFIX}\xlabel{OEFIX}OEFIX}

\begin{description}

\item[Description :] Fixes the odd and even channel dc offset problem
encountered with the old 62$\times$58 {\sc ircam1/2} (dual-channel
readout) array cameras.  The odd and even channels are extracted and
the medians of the pixels in them are calculated.  The dc offset
between the median of the odd channel pixels and the median of the even
channel pixels is added to the even channel and the array odd/even
channel structure is re-assembled into the output file.

\item[Usage :] Correcting odd/even channel problem in {\sc ircam1/2}.
\item[Associated commands :] -
\item[Invocation :]

\begin{quote}{\tt  oefix }\end{quote}

\end{description}

\hrule
\subsubsection*{\label{OUTSETB}\xlabel{OUTSETB}OUTSETB}

\begin{description}

\item[Description :] Sets the pixel values outside a user-defined
sub-area of an image to a user-defined value.  The sub-area outside of
which the pixels are set is defined by its X,Y start pixel coordinates
and the X,Y box size in pixels.  The output image has the pixels
outside the sub-area set to the value chosen by the user.

\item[Usage :] Setting pixels outside region (box) of an image to a
magic number.

\item[Associated commands :] {\tt \htmlref{outsetc}{OUTSETC}},
{\tt \htmlref{insetb}{INSETB}}, {\tt \htmlref{insetc}{INSETC}}

\item[Invocation :]

\begin{quote}{\tt  outsetb }\end{quote}

\end{description}

\hrule
\subsubsection*{\label{OUTSETC}\xlabel{OUTSETC}OUTSETC}

\begin{description}

\item[Description :] Sets the pixel values outside a user-defined
circle in an image to a user-defined value.  The circle outside of
which the pixels are set is defined by its X,Y centre pixel coordinates
and the diameter in pixels.  The output image has the pixels outside
the circle set to the value chosen by the user.

\item[Usage :] Setting pixels outside region (circle) of an image to a
magic number.

\item[Associated commands :] {\tt \htmlref{outsetb}{OUTSETB}},
{\tt \htmlref{insetb}{INSETB}}, {\tt \htmlref{insetc}{INSETC}}

\item[Invocation :]

\begin{quote}{\tt  outsetc }\end{quote}

\end{description}

\hrule
\subsubsection*{\label{PICKIM}\xlabel{PICKIM}PICKIM}

\begin{description}

\item[Description :] Extracts a sun-area of any input image and puts a
it in the output image as a stand-alone image.  The sub-area is defined
by the  X,Y start pixels (1,1 is the bottom left corner of the image)
and the X,Y sub-area size in pixels.

\item[Usage :] Extracting useful sub-areas of images from larger images.
\item[Associated commands :] {\tt \htmlref{dispick}{DISPICK}}
\item[Invocation :]

\begin{quote}{\tt dispick }\end{quote}

\end{description}

\hrule
\subsubsection*{\label{PIXDUPE}\xlabel{PIXDUPE}PIXDUPE}

\begin{description}

\item[Description :] Expands the X,Y dimensions of any image by pixel
duplication. The output image is expanded by an integer factor, say for
example 2, and will have X,Y dimensions 2$\times$ that of the input
image.  No interpolation takes place, the pixels are merely duplicated
N times where N is input by the user.

\item[Usage :] Expanding image sizes.
\item[Associated commands :] -
\item[Invocation :]

\begin{quote}{\tt  pixdupe }\end{quote}

\end{description}

\hrule
\subsubsection*{\label{POLCAL}\xlabel{POLCAL}POLCAL}

\begin{description}

\item[Description :] Calculates polarization images (normalized Stokes
parameters Q,U, percentage polarization and polarization position angle
P, theta, total intensity I, polarized and unpolarized intensity PI
UPI, and the percentage polarization and position angle shot-noise
error Pe, thetaE). {\bf polcal} takes four input images (one at each of
the four waveplate position for a polarization observation -- {\tt 0},
{\tt 45}, {\tt 22.5} and {\tt 67.5} degrees) and calculates the polarization
images from them.

The percentage polarization error image and associated polarization
position angle image require the user to enter the electrons/DN gain
factor for the original intensity images.  If your data is an average of
numerous coadds then the value entered should be the original gain factor
divided by the square root of the number of coadds (or separate averaged
images) since to calculate the accurate shot-noise error associated with
the polarization image one needs to use an equivalent signal for all the
coadds added together.  This is compensated by dividing the gain factor
by the root of the number of coadds/image.

{\bf polcal} operates on polarization images taken with the OLD {\sc
irpol} system \emph{i.e.}, four images one at each of the four
waveplate position ({\tt 0}, {\tt 45}, {\tt 22.5}, {\tt 67.5} degrees
from datum) using a cold polarizer rather than the current Wollaston
prism.  To analyse data from {\sc irpol} using the Wollaston prism use
the command {\bf polcal2}.

\item[Usage :] Calculating polarization images from processed intensity
images for {\sc irpol} plus polarizer (old format) combination

\item[Associated commands :] {\tt polcal2}

\item[Invocation :]

\begin{quote}{\tt  polcal }\end{quote}

\end{description}

\hrule
\subsubsection*{\label{POLLY}\xlabel{POLLY}POLLY}

\begin{description}

\item[Description :] Calculates polarization parameters {\tt Q}, {\tt
U}, {\tt P}, {\tt Theta}, {\tt I}, {\tt PI}, {\tt UPI}, {\tt PE}, {\tt
TE} from four input intensity (numeric) values.  The four intensity
values should be at the four waveplate positions required for a
polarization observation \emph{i.e.}, {\tt 0}, {\tt 45}, {\tt 22.5} and
{\tt 67.5} degrees from datum.  {\bf polly} works for numeric values
taken from images taken with a waveplate/polarizer combination
\emph{i.e.}, {\sc irpol1} and polarizer (no Wollaston prism).  Use {\bf
polly2} to for intensity values taken from an {\sc irpol2} and Wollaston
prism data set.

\item[Usage :] Calculating polarization from input intensity values.

\item[Associated commands :] {\tt \htmlref{polly2}{POLLY2}},
{\tt \htmlref{polcal}{POLCAL}}, {\tt polcal2}

\item[Invocation :]

\begin{quote}{\tt  polly }\end{quote}

\end{description}

\hrule
\subsubsection*{\label{POLLY2}\xlabel{POLLY2}POLLY2}

\begin{description}

\item[Description :] Same as {\bf polly} but for intensity values taken
from images taken at the four waveplate positions using {\sc irpol2}
and Wollaston prism.  The user has to enter 8 values (rather than four
for {\bf polly}) since the o- and e- rays are imaged separately using
the Wollaston prism.

\item[Usage :] Calculating polarization parameters from input intensity values.

\item[Associated commands :] {\tt \htmlref{polly}{POLLY}},
{\tt polcal2}, {\tt \htmlref{polcal}{POLCAL}},
{\tt \htmlref{polsep}{POLSEP}}, {\tt polthresh2}

\item[Invocation :]

\begin{quote}{\tt  polly2 }\end{quote}

\end{description}

\hrule
\subsubsection*{\label{POLSEP}\xlabel{POLSEP}POLSEP}

\begin{description}

\item[Description :] Separates the four regions of a image taken using
the {\sc irpol2} polarimeter with the Wollaston prism/focal plane mask at one
of the four required waveplate positions.  The four regions are the o-
and e- rays for the two spatial different areas of sky imaged using the
mask.  The four regions are put into four separate images which can
then be processed further using {\bf polcal2} on each spatial regions
sequentially. {\bf polsep} will need to be run on all four different
waveplate position image to get the 8 images per spatial position
required for {\bf polcal2} analysis.

\item[Usage :] Creating separate images with o- e- rays of the two
spatial positions imaged using the Wollaston prism/focal plane mask in
a new polarimetry observation.

\item[Associated commands :] {\tt \htmlref{polly2}{POLLY2}}
{\tt polcal2}, {\tt polthresh2}

\item[Invocation :]

\begin{quote}{\tt  polsep }\end{quote}

\end{description}

\hrule
\subsubsection*{\label{POLSHOT}\xlabel{POLSHOT}POLSHOT}

\begin{description}

\item[Description :] Calculates polarization shot-noise error on a
polarization observation using {\sc irpol1} with the internal cold
polarizer (\emph{i.e.}, not the Wollaston prism).  The shot-noise is
the polarization error due to photon counting statistics and is there
the smallest error associated with a polarization observation.  The
four input images, one at each of the four waveplate positions ({\tt
0}, {\tt 45}, {\tt 22.5}, {\tt 67.5} degrees) are required to
calculates the shot-noise error.  The user should be aware that the
total intensity in each of the four waveplate positions is needed not
the intensity scaled to dn/sec \emph{i.e.}, with the coadds/image
combinations removed by averaging.

{\bf polshot} is not too useful since errors in this mode are dominated
by measurement errors like sky transparency changes over the four
waveplate positions.

\item[Usage :] Calculating shot-noise polarization on a waveplate plus
polarizer observation.

\item[Associated commands :] {\tt \htmlref{polcal}{POLCAL}},
{\tt \htmlref{polthresh}{POLTHRESH}}

\item[Invocation :]

\begin{quote}{\tt  polshot }\end{quote}

\end{description}

\hrule
\subsubsection*{\label{POLTHRESH}\xlabel{POLTHRESH}POLTHRESH}

\begin{description}

\item[Description :] Thresholds a set of 4 images each taken at one of
four waveplate positions ({\tt 0}, {\tt 45}, {\tt 22.5}, {\tt 67.5}
degrees) in a polarization observation using {\sc irpol} and internal
cold polarizer (\emph{i.e.}, not the Wollaston prism).  The user enters
a threshold intensity level which the all four pixels at the same
spatial position on the array have to be above if polarization is to be
calculable.  If any one of the four intensity pixels have values less
than the threshold the output pixel in the all four output images is
set to zero.  This thresholding allows only pixels with good S/N to be
utilized in polarization calculations in {\bf polcal}. The user should
experiment with the threshold level (in the sky subtracted reduced
polarization images) to see when the faint features of nebulosity say,
become noisy.

\item[Usage :] Thresholding a set of {\sc irpol} plus polarizer
polarization data.
\item[Associated commands :] {\tt \htmlref{polcal}{POLCAL}}
\item[Invocation :]

\begin{quote}{\tt  polthresh }\end{quote}

\end{description}

\hrule
\subsubsection*{\label{POW}\xlabel{POW}POW}

\begin{description}

\item[Description :] Raises each pixel to the power {\it x} where {\it
x} is input by the user. The calculation involved is (pixel value)**x
where {\it x} is the input exponent.

\item[Usage :] -
\item[Associated commands :] -
\item[Invocation :]

\begin{quote}{\tt  pow }\end{quote}

\end{description}

\hrule
\subsubsection*{\label{QUILT}\xlabel{QUILT}QUILT}

\begin{description}

\item[Description :] Assembles mosaic tiles into a final mosaic image.
{\bf quilt} is very similar to the {\sc IrcamDR} command {\bf mosaic}
but {\bf quilt} takes its input from an ASCII (text) file rather from
the terminal/user.  An example of the form of the quilt input text file
is:

\begin{small}
\begin{verbatim}
      test quilt file
      image1
         5
           13.9860    13.9860
          -13.9860   -13.9860
      image2
          -13.9860    13.9860
      image3
          -13.9860   -13.9860
      image4
           13.9860   -13.9860
      image5
           13.9860    13.9860
\end{verbatim}
\end{small}

where line 1 is a title string, line 2 the number of image tiles in the
mosaic (5), line 3 the maximum spatial offsets in RA,Dec, line 4 the
minimum spatial offsets in RA,Dec, line 5 the 2nd image in the mosaic,
line 6 the spatial offsets (in pixels) between the 1st image and the 2nd
image, line 7 the 3rd image in the mosaic, line 8 the spatial offsets (in
pixels) between the 1st image and 3rd image, line 9 the 4th image in the
mosaic, line 10 the spatial offsets (in pixels) between the 1st image and
4th image in the mosaic, line 11 the 5th image in the mosaic and line 12
the spatial offsets between the 1st image and 5th image in the mosaic.
The spatial offsets in pixels are X,Y with X and Y increasing up and
right, respectively.

The best way to create a quilt file however, is to use the command
{\bf crequilt} described above.  This command takes in simple ASCII files and
creates the more complex {\bf quilt} mosaic ASCII file.


\item[Usage :] Mosaicing image tiles from a mosaic together.

\item[Associated commands :] {\tt \htmlref{mosaic}{MOSAIC}},
{\tt \htmlref{mosaic2}{MOSAIC2}}, {\tt \htmlref{accoff}{ACCOFF}} \emph{etc.}

\item[Invocation :]

\begin{quote}{\tt  quilt }\end{quote}

\end{description}

\hrule
\subsubsection*{\label{RADIM}\xlabel{RADIM}RADIM}

\begin{description}

\item[Description :] Produces a radially expanded image from any
user-defined point in an image.  A radially expanded image is one in
which, from the pixel chosen by the user, values along radii from that
point to a user-defined extent (in pixels) are written sequentially to
an image, each radial cut (in 1 degree increments) corresponds to one
column of the output image, thus, the output image is 360$\times$N
pixels in size, the 360 corresponding to the 360 degrees around the
centre pixel and the N being the user-defined extent of the radial cut
along each radius.

\item[Usage :] Expanding a radial cut on a linear axis.
\item[Associated commands :] -
\item[Invocation :]

\begin{quote}{\tt  radim }\end{quote}

\end{description}

\hrule
\subsubsection*{\label{ROOT}\xlabel{ROOT}ROOT}

\begin{description}

\item[Description :] Takes the square root of the input image and
stores the result in the output image.

\item[Usage :] Decreasing dynamic range to aid data display.

\item[Associated commands :] {\tt \htmlref{add}{ADD}},
{\tt \htmlref{div}{DIV}}, {\tt \htmlref{mult}{MULT}},
{\tt \htmlref{sub}{SUB}}, {\tt \htmlref{pow}{POW}},
{\tt \htmlref{cadd}{CADD}}, {\tt \htmlref{cdiv}{CDIV}},
{\tt \htmlref{cmult}{CMULT}}, {\tt \htmlref{csub}{CSUB}},
{\tt \htmlref{expe}{EXPE}}, {\tt \htmlref{exp10}{EXP10}},
{\tt \htmlref{expon}{EXPON}}, {\tt \htmlref{log10}{LOG10}},
{\tt \htmlref{loge}{LOGE}}, {\tt \htmlref{logar}{LOGAR}}

\item[Invocation :]

\begin{quote}{\tt root }\end{quote}

\end{description}

\hrule
\subsubsection*{\label{ROTATE}\xlabel{ROTATE}ROTATE}

\begin{description}

\item[Description :] Rotates an image by a user-defined amount (in
degrees) clockwise. The output image is re-sampled onto the rotated
coordinate system and the output image is rotated with respect to to
the input image.

\item[Note :] The N-S axis of {\sc ircam3} is mis-aligned by 1.5 degrees.

\item[Usage :] Rotation to remove N-S mis-alignment of images.
\item[Associated commands :] -
\item[Invocation :]

\begin{quote}{\tt  rotate }\end{quote}

\end{description}

\hrule
\subsubsection*{\label{ROWMED}\xlabel{ROWMED}ROWMED}

\begin{description}

\item[Description :] Median filters along the rows of an image and
produces a 1$\times$N output image (where N is the original Y dimension
of the input image) with containing the median values for each row.
The user can define an exclusion region (a range of columns) which is
not included in the median calculation, hence, if there is a bright
object near the centre of the array, that region can be excluded.

\item[Usage :] For removing residual background structure
in an image.  Use {\bf rowmed} to define the row structure via the
median value in the rows then {\bf xgrow} to grow the 1$\times$N image
to M$\times$N where M and N are the original X and Y dimensions of the
input image.  The result can be subtracted from the original image
using the command {\bf sub}.

\item[Associated commands :] {\tt \htmlref{xgrow}{XGROW}},
{\tt \htmlref{sub}{SUB}}, {\tt \htmlref{colmed}{COLMED}},
{\tt \htmlref{ygrow}{YGROW}}

\item[Invocation :]

\begin{quote}{\tt  rowmed }\\
or \\
{\tt rowmed image1 out1 y 50 150 }
\end{quote}

\begin{itemize}

\item {\tt image1 } is the input M$\times$N image
\item {\tt out1 } is the output 1$\times$N image containing the medians
\item {\tt y } means use an exclusion region ({\tt n} would mean no exclusion
 region)
\item {\tt 50 150 } are the range of columns in each row to be excluded
 from the median calculation
\end{itemize}

\end{description}

\hrule
\subsubsection*{\label{SETVAL}\xlabel{SETVAL}SETVAL}

\begin{description}

\item[Description :] Sets all pixels with a user-specified value to
another value also chosen by the user.  The output image contains the
modified image.

\item[Usage :] Changing bad pixel magic number values, for example.
\item[Associated commands :] -
\item[Invocation :]

\begin{quote}{\tt  setval }\end{quote}

\end{description}

\hrule
\subsubsection*{\label{SHADOW}\xlabel{SHADOW}SHADOW}

\begin{description}

\item[Description :] This routine takes a 2d image and creates an
enhanced version of it. The enhancement is a shadow effect that causes
features in an image to appear as though they have been illuminated
from the side by some imaginary light source. The enhancement is useful
in locating edges and fine detail in an image.

\item[Usage :] Enhancing image details.
\item[Associated commands :] {\tt \htmlref{laplace}{LAPLACE}}
\item[Invocation :]

\begin{quote}{\tt  shadow }\end{quote}

\end{description}

\hrule
\subsubsection*{\label{SHIFT}\xlabel{SHIFT}SHIFT}

\begin{description}

\item[Description :] The input image is shifted, in either or both of
the X and Y axes, to produce the output image. The shifts in X and Y
are either input as absolute X and Y shifts by the user or
alternatively are calculated from the coordinates of two points
provided by the user. The output image is padded with zeros in the
regions not occupied by the shifted input image.

\item[Usage :] Aligning images of same spatial region but, say, in different
filters.
\item[Associated commands :] {\tt \htmlref{shift2}{SHIFT2}},
{\tt \htmlref{shift3}{SHIFT3}}
\item[Invocation :]

\begin{quote}{\tt  shift }\end{quote}

\end{description}

\hrule
\subsubsection*{\label{SHSIZE}\xlabel{SHSIZE}SHSIZE}

\begin{description}

\item[Description :] Informs the user of the dimensions of the image
given.  The dimensions are written to the terminal.

\item[Usage :] Determining image size.
\item[Associated commands :] -
\item[Invocation :]

\begin{quote}{\tt  shsize }\end{quote}

\end{description}

\hrule
\subsubsection*{\label{SQORST}\xlabel{SQORST}SQORST}

\begin{description}

\item[Description :] Changes the size (dimensions) of any input image
and make them any other user-defined value.  {\bf sqorst} uses spline
interpolation to change image sizes.

\item[Usage :] Changing image sizes.

\item[Associated commands :] {\tt \htmlref{compave}{COMPAVE}},
{\tt \htmlref{compadd}{COMPADD}}, {\tt \htmlref{compress}{COMPRESS}},
{\tt \htmlref{binup}{BINUP}}, {\tt \htmlref{compick}{COMPICK}},
{\tt \htmlref{manic}{MANIC}}

\item[Invocation :]

\begin{quote}{\tt  sqorst }\end{quote}

\end{description}

\hrule
\subsubsection*{\label{STATS}\xlabel{STATS}STATS}

\begin{description}

\item[Description :] Returns statistical information on any sub-area of
any image. The area is defined by its X,Y start pixel coordinate and
X,Y sub-area size (in pixels).

\item[Usage :] Statistical information on images.
\item[Associated commands :] {\tt \htmlref{histo}{HISTO}},
{\tt \htmlref{cstats}{CSTATS}}
\item[Invocation :]

\begin{quote}{\tt  stats }\end{quote}

\end{description}

\hrule
\subsubsection*{\label{STCOADD}\xlabel{STCOADD}STCOADD}

\begin{description}

\item[Description :] Coadds (sums) a set of observation images taken
from an old format {\sc ircam1/2} container file.  The output image is
the sum of the input images and the input images to be coadded are
defined by either a start and end observation number or a list of
observation numbers (\emph{e.g.}, {\tt 10}, {\tt 12}, {\tt 14}, {\tt
18}, {\tt 24} \emph{etc}).

\item[Usage :] Coadding observations from an {\sc ircam1/2} container file.
\item[Associated commands :] {\tt \htmlref{fcoadd}{FCOADD}}
\item[Invocation :]

\begin{quote}{\tt  stcoadd }\end{quote}

\end{description}

\hrule
\subsubsection*{\label{STEPIM}\xlabel{STEPIM}STEPIM}

\begin{description}

\item[Description :] Sets user-defined ranges of intensity (or whatever
the units of the image) in the input image to specific values in the
output.  An example is that pixels with intensity values in the ranges
{\tt 0-100}, {\tt 101-200}, {\tt 201-300} \emph{etc.}, have, in the
output images, the values {\tt 50}, {\tt 150}, {\tt 250}, \emph{etc.}
This essentially steps an image into contour-like intervals so that
pixel vales in the range are represented by one number.  The original
idea was to step percentage polarization images to get a better idea of
where similar polarization values occurred spatially.  {\tt stepim} can be
used for any image however.  The steps are defined by a base level and
interval between steps and the number of steps to make in that interval
is input also.

\item[Usage :] Degrading intensity resolution to better see features.
\item[Associated commands :] -
\item[Invocation :]

\begin{quote}{\tt  stepim }\end{quote}

\end{description}

\hrule
\subsubsection*{\label{SUB}\xlabel{SUB}SUB}

\begin{description}

\item[Description :] Subtracts the two input images and writes result
to output image.

\item[Usage :] Object-sky subtraction, say.

\item[Associated commands :] {\tt \htmlref{add}{ADD}},
{\tt \htmlref{div}{DIV}}, {\tt \htmlref{mult}{MULT}}

\item[Invocation :]

\begin{quote}{\tt  sub }\end{quote}

\end{description}

\hrule
\subsubsection*{\label{THETAFIX}\xlabel{THETAFIX}THETAFIX}

\begin{description}

\item[Description :] Corrects polarization position angle images so
that, after the constant value offset between instrumental and
equatorial coordinate system has been added (using the command {\bf cadd}),
polarization position angles lie in the correct range \emph{i.e.},
0-180 degrees.  The input is just the image containing the polarization
position angle data.  The output is the corrected polarization position
angle image.

\item[Usage :] Position angle range correction.

\item[Associated commands :] {\tt \htmlref{polcal}{POLCAL}},
{\tt polcal2}, {\tt \htmlref{cadd}{CADD}}

\item[Invocation :]

\begin{quote}{\tt  thetafix }\end{quote}

\end{description}

\hrule
\subsubsection*{\label{THRESH}\xlabel{THRESH}THRESH}

\begin{description}

\item[Description :] Thresholds an image writing the result to an
output image.  Thresholding means that all pixel values below a
user-defined value are set to another user-defined value and all pixel
values above a third user-defined value are set to a fourth
user-defined value.

\item[Usage :] Thresholding an image.
\item[Associated commands :] {\tt \htmlref{thresh0}{THRESH0}}
\item[Invocation :]

\begin{quote}{\tt  thresh }\end{quote}

\end{description}

\hrule
\subsubsection*{\label{THRESH0}\xlabel{THRESH0}THRESH0}

\begin{description}

\item[Description :] Same as {\bf thresh} but sets all values below first
user-defined value and above second user-defined value to zero in
output image.

\item[Usage :] Setting pixels outside range to zero.
\item[Associated commands :] {\tt \htmlref{thresh}{THRESH}}
\item[Invocation :]

\begin{quote}{\tt  thresh0 }\end{quote}

\end{description}

\hrule
\subsubsection*{\label{TRANDAT}\xlabel{TRANDAT}TRANDAT}

\begin{description}

\item[Description :] Reads in an ASCII (text) file with lists of pixel
locations and intensities (whatever) and creates an NDF image from the
data.  The input format required of the ASCII file is:

\begin{small}
\begin{verbatim}
      -40.0   40.0   1121.9
        0.0   30.0     56.3
      100.0   20.0   2983.2
      120.0   80.0    339.3
     etc...
\end{verbatim}
\end{small}

where each line contains X offset, Y offset and intensity.  The user
can select the intra-pixel spacing, for example, pixels spaced 10 units
apart would would have the first two lines of the above example spaced
in the image by 4 pixels and 1 pixel in X and Y, respectively.

\item[Usage :] Format conversion from ASCII to NDF image.
\item[Associated commands :] {\tt \htmlref{asciilist}{ASCIILIST}}
\item[Invocation :]

\begin{quote}{\tt  trandat }\end{quote}

\end{description}

\hrule
\subsubsection*{\label{TRIG}\xlabel{TRIG}TRIG}

\begin{description}

\item[Description :] Uses each pixel in an image as the value in one of
numerous trigonometrical expressions \emph{e.g.}, cos(z), sin(z),
tan(z), asin(z) \emph{etc.}, where z is the pixel value.  The output is
written to a new output image.

\item[Usage :] -

\item[Associated commands :] {\tt \htmlref{root}{ROOT}},
{\tt \htmlref{pow}{POW}}, {\tt \htmlref{expe}{EXPE}},
{\tt \htmlref{exp10}{EXP10}}, {\tt \htmlref{expon}{EXPON}},
{\tt \htmlref{loge}{LOGE}}, {\tt \htmlref{log10}{LOG10}},
{\tt \htmlref{logar}{LOGAR}}

\item[Invocation :]

\begin{quote}{\tt  trig }\end{quote}

\end{description}

\hrule
\subsubsection*{\label{WMOSAIC}\xlabel{WMOSAIC}WMOSAIC}

\begin{description}

\item[Description :] The same as {\bf mosaic} except that the images
can be weighted in the averaging overlap regions.  This means images of
different S/N can be combined with different weights.  See {\bf mosaic}
for details of the mosaicing process.  The weighting is applied using
the formula:

\[out=\frac{image1*weight1+image2*weight2+\ldots}{weight1+weight2+\ldots}\]

\item[Usage :] Weighted mosaicing of tile in mosaic.

\item[Associated commands :] {\tt \htmlref{mosaic}{MOSAIC}},
{\tt \htmlref{quilt}{QUILT}}, {\tt \htmlref{wquilt}{WQUILT}}

\item[Invocation :]

\begin{quote}{\tt  wmosaic }\end{quote}

\end{description}

\hrule
\subsubsection*{\label{WQUILT}\xlabel{WQUILT}WQUILT}

\begin{description}

\item[Description :] The same as {\bf quilt} except that the images
can be weighted in the averaging overlap regions.  This means images of
different S/N can be combined with different weights.  See {\bf quilt} for
details of the mosaicing process. The weighting is applied using the
formula:

\[out=\frac{image1*weight1+image2*weight2+\ldots}{weight1+weight2+\ldots}\]

\item[Usage :] Weighted mosaicing of tile in mosaic from ASCII file.

\item[Associated commands :] {\tt \htmlref{mosaic}{MOSAIC}},
{\tt \htmlref{wmosaic}{WMOSAIC}}, {\tt \htmlref{wquilt}{WQUILT}}

\item[Invocation :]

\begin{quote}{\tt  wquilt }\end{quote}

\end{description}

\hrule
\subsubsection*{\label{WRAPCOR}\xlabel{WRAPCOR}WRAPCOR}

\begin{description}

\item[Description :] Corrects images (mostly optical CCD images) for
wrap-around caused by bit wrapping when the A to D unit of the readout
wraps around from 16-bits to -16-bits instead.  An example is data that
in actuality goes from 0-65535 but the A to D sets the values from
0-32767 then -32767-0, wrapcor adds a number to all values below a
specified number and hence, the the above example, we would add 32768
to all values below 0.

\item[Usage :] Correcting for wrap-around in image readout.
\item[Associated commands :] -
\item[Invocation :]

\begin{quote}{\tt  wrapcor }\end{quote}

\end{description}

\hrule
\subsubsection*{\label{XGROW}\xlabel{XGROW}XGROW}

\begin{description}

\item[Description :] Grows a 1$\times$N image created with {\bf rowmed}
into a M$\times$N image by pixel duplication.  The output is written to
the new NDF image.  See {\bf rowmed} for more details.

\item[Usage :] Growing a Y slice into an image for subtraction.

\item[Associated commands :] {\tt \htmlref{ygrow}{YGROW}},
{\tt \htmlref{rowmed}{ROWMED}}, {\tt \htmlref{colmed}{COLMED}},
{\tt \htmlref{yadd}{YADD}}

\item[Invocation :]

\begin{quote}{\tt xgrow }\end{quote}

\end{description}

\hrule
\subsubsection*{\label{YADD}\xlabel{YADD}YADD}

\begin{description}

\item[Description :] Adds up the rows in an image to create an output
1$\times$N image.  This output image can then be grow to the original
size of the input image using {\bf xgrow}.  The output can be the sum or
average of the rows.

\item[Usage :] -

\item[Associated commands :] {\tt \htmlref{xgrow}{XGROW}},
{\tt \htmlref{rowmed}{ROWMED}}, {\tt \htmlref{colmed}{COLMED}},
{\tt \htmlref{ygrow}{YGROW}}

\item[Invocation :]

\begin{quote}{\tt  yadd }\end{quote}

\end{description}

\hrule
\subsubsection*{\label{YGROW}\xlabel{YGROW}YGROW}

\begin{description}

\item[Description :] Grows an N$\times$1 image created with {\bf colmed}
into an N$\times$M image by pixel duplication.  The output
image is written to a new NDF image.  See {\bf colmed} for more details of
usage.

\item[Usage :] Growing an X slice into an image for subtraction.

\item[Associated commands :] {\tt \htmlref{xgrow}{XGROW}},
{\tt \htmlref{colmed}{COLMED}}, {\tt \htmlref{rowmed}{ROWMED}},
{\tt \htmlref{yadd}{YADD}}

\item[Invocation :]

\begin{quote}{\tt  ygrow }\end{quote}

\end{description}

\hrule
\subsubsection*{\label{ZAPLIN}\xlabel{ZAPLIN}ZAPLIN}

\begin{description}

\item[Description :] Removes bad columns and rows from an image by
linear interpolation.  The bad columns/rows are defined by their
column/row number and a number of consecutive columns/rows can be
removed.  After each range of columns/rows has been entered, the
program asks if more columns/rows are to be entered.  If not, the
columns/rows entered are removed by linear interpolation.

\item[Usage :] Removing bad columns/rows in images.

\item[Associated commands :] {\tt \htmlref{glitch}{GLITCH}},
{\tt \htmlref{glitchmark}{GLITCHMARK}}

\item[Invocation :]

\begin{quote}{\tt  zaplin }\end{quote}

\end{description}

% END

\end{document}

