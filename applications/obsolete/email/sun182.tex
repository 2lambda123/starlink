\documentclass[twoside,11pt]{article}

% ? Specify used packages
% \usepackage{graphicx}        %  Use this one for final production.
% \usepackage[draft]{graphicx} %  Use this one for drafting.
% ? End of specify used packages

\pagestyle{myheadings}

% -----------------------------------------------------------------------------
% ? Document identification
% Fixed part
\newcommand{\stardoccategory}  {Starlink User Note}
\newcommand{\stardocinitials}  {SUN}
\newcommand{\stardocsource}    {sun\stardocnumber}
\newcommand{\stardoccopyright} 
{Copyright \copyright\ 2000 Council for the Central Laboratory of the Research Councils}

% Variable part - replace [xxx] as appropriate.
\newcommand{\stardocnumber}    {182.7}
\newcommand{\stardocauthors}   {M. J. Bly \\G. R. Mellor}
\newcommand{\stardocdate}      {18th February 2000}
\newcommand{\stardoctitle}     {EMAIL --- Email address searching}
\newcommand{\stardocversion}   {v1.4}
% \newcommand{\stardocmanual}    {[manual-type]}
\newcommand{\stardocabstract}  {
One of the most common activities on networked computers is the sending
and receiving of personal electronic mail (email).  Starlink nodes are
connected to the worldwide Internet network.  This document describes
how to find email addresses to communicate with other astronomers
and astronomy groups in the UK and the rest of the world.
}

% ? End of document identification
% -----------------------------------------------------------------------------

% +
%  Name:
%     sun.tex
%
%  Purpose:
%     Template for Starlink User Note (SUN) documents.
%     Refer to SUN/199
%
%  Authors:
%     AJC: A.J.Chipperfield (Starlink, RAL)
%     BLY: M.J.Bly (Starlink, RAL)
%     PWD: Peter W. Draper (Starlink, Durham University)
%
%  History:
%     17-JAN-1996 (AJC):
%        Original with hypertext macros, based on MDL plain originals.
%     16-JUN-1997 (BLY):
%        Adapted for LaTeX2e.
%        Added picture commands.
%     13-AUG-1998 (PWD):
%        Converted for use with LaTeX2HTML version 98.2 and
%        Star2HTML version 1.3.
%      1-FEB-2000 (AJC):
%        Add Copyright statement in LaTeX
%     {Add further history here}
%
% -

\newcommand{\stardocname}{\stardocinitials /\stardocnumber}
\markboth{\stardocname}{\stardocname}
\setlength{\textwidth}{160mm}
\setlength{\textheight}{230mm}
\setlength{\topmargin}{-2mm}
\setlength{\oddsidemargin}{0mm}
\setlength{\evensidemargin}{0mm}
\setlength{\parindent}{0mm}
\setlength{\parskip}{\medskipamount}
\setlength{\unitlength}{1mm}

% -----------------------------------------------------------------------------
%  Hypertext definitions.
%  ======================
%  These are used by the LaTeX2HTML translator in conjunction with star2html.

%  Comment.sty: version 2.0, 19 June 1992
%  Selectively in/exclude pieces of text.
%
%  Author
%    Victor Eijkhout                                      <eijkhout@cs.utk.edu>
%    Department of Computer Science
%    University Tennessee at Knoxville
%    104 Ayres Hall
%    Knoxville, TN 37996
%    USA

%  Do not remove the %begin{latexonly} and %end{latexonly} lines (used by 
%  LaTeX2HTML to signify text it shouldn't process).
%begin{latexonly}
\makeatletter
\def\makeinnocent#1{\catcode`#1=12 }
\def\csarg#1#2{\expandafter#1\csname#2\endcsname}

\def\ThrowAwayComment#1{\begingroup
    \def\CurrentComment{#1}%
    \let\do\makeinnocent \dospecials
    \makeinnocent\^^L% and whatever other special cases
    \endlinechar`\^^M \catcode`\^^M=12 \xComment}
{\catcode`\^^M=12 \endlinechar=-1 %
 \gdef\xComment#1^^M{\def\test{#1}
      \csarg\ifx{PlainEnd\CurrentComment Test}\test
          \let\html@next\endgroup
      \else \csarg\ifx{LaLaEnd\CurrentComment Test}\test
            \edef\html@next{\endgroup\noexpand\end{\CurrentComment}}
      \else \let\html@next\xComment
      \fi \fi \html@next}
}
\makeatother

\def\includecomment
 #1{\expandafter\def\csname#1\endcsname{}%
    \expandafter\def\csname end#1\endcsname{}}
\def\excludecomment
 #1{\expandafter\def\csname#1\endcsname{\ThrowAwayComment{#1}}%
    {\escapechar=-1\relax
     \csarg\xdef{PlainEnd#1Test}{\string\\end#1}%
     \csarg\xdef{LaLaEnd#1Test}{\string\\end\string\{#1\string\}}%
    }}

%  Define environments that ignore their contents.
\excludecomment{comment}
\excludecomment{rawhtml}
\excludecomment{htmlonly}

%  Hypertext commands etc. This is a condensed version of the html.sty
%  file supplied with LaTeX2HTML by: Nikos Drakos <nikos@cbl.leeds.ac.uk> &
%  Jelle van Zeijl <jvzeijl@isou17.estec.esa.nl>. The LaTeX2HTML documentation
%  should be consulted about all commands (and the environments defined above)
%  except \xref and \xlabel which are Starlink specific.

\newcommand{\htmladdnormallinkfoot}[2]{#1\footnote{#2}}
\newcommand{\htmladdnormallink}[2]{#1}
\newcommand{\htmladdimg}[1]{}
\newcommand{\hyperref}[4]{#2\ref{#4}#3}
\newcommand{\htmlref}[2]{#1}
\newcommand{\htmlimage}[1]{}
\newcommand{\htmladdtonavigation}[1]{}

\newenvironment{latexonly}{}{}
\newcommand{\latex}[1]{#1}
\newcommand{\html}[1]{}
\newcommand{\latexhtml}[2]{#1}
\newcommand{\HTMLcode}[2][]{}

%  Starlink cross-references and labels.
\newcommand{\xref}[3]{#1}
\newcommand{\xlabel}[1]{}

%  LaTeX2HTML symbol.
\newcommand{\latextohtml}{\LaTeX2\texttt{HTML}}

%  Define command to re-centre underscore for Latex and leave as normal
%  for HTML (severe problems with \_ in tabbing environments and \_\_
%  generally otherwise).
\renewcommand{\_}{\texttt{\symbol{95}}}

% -----------------------------------------------------------------------------
%  Debugging.
%  =========
%  Remove % on the following to debug links in the HTML version using Latex.

% \newcommand{\hotlink}[2]{\fbox{\begin{tabular}[t]{@{}c@{}}#1\\\hline{\footnotesize #2}\end{tabular}}}
% \renewcommand{\htmladdnormallinkfoot}[2]{\hotlink{#1}{#2}}
% \renewcommand{\htmladdnormallink}[2]{\hotlink{#1}{#2}}
% \renewcommand{\hyperref}[4]{\hotlink{#1}{\S\ref{#4}}}
% \renewcommand{\htmlref}[2]{\hotlink{#1}{\S\ref{#2}}}
% \renewcommand{\xref}[3]{\hotlink{#1}{#2 -- #3}}
%end{latexonly}
% -----------------------------------------------------------------------------
% ? Document specific \newcommand or \newenvironment commands.
% ? End of document specific commands
% -----------------------------------------------------------------------------
%  Title Page.
%  ===========
\renewcommand{\thepage}{\roman{page}}
\begin{document}
\thispagestyle{empty}

%  Latex document header.
%  ======================
\begin{latexonly}
   CCLRC / \textsc{Rutherford Appleton Laboratory} \hfill \textbf{\stardocname}\\
   {\large Particle Physics \& Astronomy Research Council}\\
   {\large Starlink Project\\}
   {\large \stardoccategory\ \stardocnumber}
   \begin{flushright}
   \stardocauthors\\
   \stardocdate
   \end{flushright}
   \vspace{-4mm}
   \rule{\textwidth}{0.5mm}
   \vspace{5mm}
   \begin{center}
   {\Huge\textbf{\stardoctitle \\ [2.5ex]}}
   {\LARGE\textbf{\stardocversion \\ [4ex]}}
%  {\Huge\textbf{\stardocmanual}}
   \end{center}
   \vspace{5mm}

% ? Add picture here if required for the LaTeX version.
%   e.g. \includegraphics[scale=0.3]{filename.ps}
% ? End of picture

% ? Heading for abstract if used.
   \vspace{10mm}
   \begin{center}
      {\Large\textbf{Abstract}}
   \end{center}
% ? End of heading for abstract.
\end{latexonly}

%  HTML documentation header.
%  ==========================
\begin{htmlonly}
   \xlabel{}
   \begin{rawhtml} <H1> \end{rawhtml}
      \stardoctitle\\
      \stardocversion\\
%     \stardocmanual
   \begin{rawhtml} </H1> <HR> \end{rawhtml}

% ? Add picture here if required for the hypertext version.
%   e.g. \includegraphics[scale=0.7]{filename.ps}
% ? End of picture

   \begin{rawhtml} <P> <I> \end{rawhtml}
   \stardoccategory\ \stardocnumber \\
   \stardocauthors \\
   \stardocdate
   \begin{rawhtml} </I> </P> <H3> \end{rawhtml}
      \htmladdnormallink{CCLRC / Rutherford Appleton Laboratory}
                        {http://www.cclrc.ac.uk} \\
      \htmladdnormallink{Particle Physics \& Astronomy Research Council}
                        {http://www.pparc.ac.uk} \\
   \begin{rawhtml} </H3> <H2> \end{rawhtml}
      \htmladdnormallink{Starlink Project}{http://www.starlink.ac.uk/}
   \begin{rawhtml} </H2> \end{rawhtml}
   \htmladdnormallink{\htmladdimg{source.gif} Retrieve hardcopy}
      {http://www.starlink.ac.uk/cgi-bin/hcserver?\stardocsource}\\

%  HTML document table of contents. 
%  ================================
%  Add table of contents header and a navigation button to return to this 
%  point in the document (this should always go before the abstract \section). 
  \label{stardoccontents}
  \begin{rawhtml} 
    <HR>
    <H2>Contents</H2>
  \end{rawhtml}
  \htmladdtonavigation{\htmlref{\htmladdimg{contents_motif.gif}}
        {stardoccontents}}

% ? New section for abstract if used.
  \section{\xlabel{abstract}Abstract}
% ? End of new section for abstract
\end{htmlonly}

% -----------------------------------------------------------------------------
% ? Document Abstract. (if used)
%  ==================
\stardocabstract
% ? End of document abstract

% -----------------------------------------------------------------------------
% ? Latex Copyright Statement
%  =========================
% \begin{latexonly}
% \newpage
% \end{latexonly}
% ? End of Latex copyright statement

% -----------------------------------------------------------------------------
% ? Latex document Table of Contents (if used).
%  ===========================================
  \newpage
  \begin{latexonly}
    \setlength{\parskip}{0mm}
    \tableofcontents
\vspace*{\fill}
\stardoccopyright
    \setlength{\parskip}{\medskipamount}
    \markboth{\stardocname}{\stardocname}
  \end{latexonly}
% ? End of Latex document table of contents
% -----------------------------------------------------------------------------

\cleardoublepage
\renewcommand{\thepage}{\arabic{page}}
\setcounter{page}{1}

% ? Main text

\section{\xlabel{introduction}Introduction}
\label{introduction}

In a manner similar to ordinary mail, all electronic mail (email)
requires both a name to identify the recipient and address for the
recipient -- a \textit{username} and \textit{sitename}.  Together these
are referred to as an \textbf{\textit{email address}} or a
\textbf{\textit{mailbox}}.

There are a large number of people (everyone on the internet!) who have
an \textbf{\textit{email address}} so if you want to contact someone by
email it can be tricky finding out where to send the message unless you
already know their email address.

The Starlink EMAIL utility provides an easy method of searching for 
an astronomers email address.

\textit{Please note that the information files that the email utility
uses are not distributed to non-Starlink sites and do not form part of
Starlink's CD distributions from EMAIL v1.4 (Spring 2000).}

\section{\xlabel{finding_email_addresses}Finding email addresses}
\label{finding_email_addresses}

There are a number of resources that can help you find someone's email 
address, particularly of members of the astronomy community in the UK and 
abroad.  

\subsection{\xlabel{starlink_user_lists}Starlink user lists}
Starlink software installations at UK sites have a copy of the two lists
of users registered at the Starlink sites.  The files are:

\begin{itemize}

\item \texttt{unixnames} -- a list of Starlink users giving a location
code and email address for each.  Location codes are used to associate
a postal address with a user.

\item \texttt{usernames} -- a list of the usernames Starlink users.

\end{itemize}

These files are found in \texttt{/star/admin} on Starlink systems and
are distributed by Starlink only to UK sites to help prevent abuse of
the lists.  The most useful is \texttt{unixnames} because you can
examine it to find an individual email address.

\subsection{\xlabel{starlink_sites}Starlink sites}
\label{starlink_sites}

Starlink maintains a database of contact details for the Starlink
sites.  Access to the data is available via the
\htmladdnormallink{Starlink WWW
server}{http://www.starlink.ac.uk/project.html}
\latex{(\texttt{http://www.starlink.ac.uk/project.html})}.
Select the \textbf{\textit{sites}} link for a list of sites then select
the site you're interested in.

Since the pages are created dynamically the information is as
up-to-date as that in the database, which is actively maintained by
Starlink.

\subsection{\xlabel{world_directory_of_astronomers}World directory of
Astronomers} \label{world_directory_of_astronomers}

Chris Benn of La Palma and Ralph Martin of the Royal Greenwich
Observatory produced an email directory of astronomers world-wide
(\texttt{astropersons.lis}) together with contact details and addresses
of their astronomical institutes (\texttt{astroplaces.lis} and
\texttt{astropostal.lis}).  These files are distributed to UK Starlink 
sites and are found on Starlink systems in the
directory \texttt{/star/etc/email}.

Since the closure of the RGO the files are maintained by the
Starlink Project at RAL.  A \htmladdnormallink{WWW interface is
available}{http://www.starlink.ac.uk/astrolist/astrosearch.html}
\latex{(\texttt{http://www.starlink.ac.uk/astrolist/astrosearch.html})}
which provides search facilities and the ability for astronomers to
provide an update for their entry.  The latest versions of the files are
distributed monthly to UK Starlink sites.

\subsection{\xlabel{other_resources}Other resources}
\label{other_resources}

The National Information Services and Systems (NISS) service at the 
University of Bath provides access to a large number of information 
services and databases.  These include links to all higher and further
education institutes and research sites which can be used to find general
contact email addresses.  NISS can be accessed at
\htmladdnormallink{\texttt{http://www.niss.ac.uk}}{http://www.niss.ac.uk}.

The ``Whole Internet User's Guide and Catalog'' by Ed Krol contains
much valuable information regarding the Internet, email and addresses.

\subsection{\xlabel{cant_find_an_email_address}Can't find an email address?}
\label{cant_find_an_email_address}

If you cannot find an email address using the above resources, you have a 
problem -- there is no reliable way to find an email address for a user
on a remote machine.  Often usernames and email addresses are quite different.

One solution is to try emaling the \texttt{postmaster} username at a site.
This is a special username created at many sites to handle misdirected and 
failed email and is used by system managers to monitor the status of their 
email systems.  It can sometimes be worth attempting to mail this account for
information about a user.

\section{\xlabel{using_the_email_search_tool}Using the EMAIL search tool}
\label{using_the_email_search_tool}

The Starlink EMAIL utility is a simple tool for finding email addresses in 
the Starlink user lists and the ex-RGO world list of astronomers.

The \texttt{email} command invokes the search tool -- you just give it a 
name to search for.  For example if you wanted the email address of 
David Rawlinson at RAL:

\begin{quote}
\begin{small}
\begin{verbatim}
% email rawlinson
email address searcher v1.4
Searching the ex-RGO directory of observatories...


Searching the directory of Starlink Users...

    Starlink User's Email addresses found in this list are likely
    to be more up-to-date than those found in the world search.

PRO   Rawlinson, David                 djr@star.rl.ac.uk 

Searching the ex-RGO world directory of astronomers...

Rawlinson, David       RALproj  I djr@star.rl.ac.uk                         97

%
\end{verbatim} 
\end{small}
\end{quote}

Note that you get two `hits'.  The first is on the Starlink \texttt{unixnames}
list and the second is on the ex-RGO world astronomers list.  

For Starlink registered users, the email address from \texttt{unixnames}
is likely to be more up-to-date because the username lists are actively
maintained by the Starlink Project.  Although Starlink maintains it, the 
ex-RGO list relies on users to provide updates for their own entries.
The number beside each `hit' on the ex-RGO world list shows which year
the entry was last updated.

% ? End of main text
\end{document}

