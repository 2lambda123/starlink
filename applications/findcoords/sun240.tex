\documentclass[twoside,11pt,nolof]{starlink}
%
% FINDCOORDS --- Finding the Coordinates of a Named Object.
%
% Copyright 2001  Starlink, CCLRC.
%
% A.C. Davenhall (Edinburgh), 4/6/01.

% -----------------------------------------------------------------------------

%
% Set the findcoords version number.

\providecommand{\findcoordsversion}{1.0~}

% -----------------------------------------------------------------------------

% ? Specify used packages
% ? End of specify used packages

% -----------------------------------------------------------------------------
% ? Document identification
% Fixed part
\stardoccategory    {Starlink User Note}
\stardocinitials    {SUN}
\stardocsource      {sun\stardocnumber}
\stardoccopyright
{Copyright \copyright\ 2001 Council for the Central Laboratory of the Research Councils}

% Variable part - replace [xxx] as appropriate.
\stardocnumber      {240.1}
\stardocauthors     {A.C.~Davenhall}
\stardocdate        {29 May 2001}
\stardoctitle     {FINDCOORDS --- \\
Finding the Coordinates of a \\ Named Object}
\stardocversion     {Version \findcoordsversion}
\stardocabstract
{\texttt{findcoords} is a utility for finding the equatorial coordinates of
a named astronomical object.  You simply enter the name of the object
and its coordinates are displayed.}
% ? End of document identification
% -----------------------------------------------------------------------------

% ? Document specific \providecommand or \newenvironment commands.
% ? End of document specific commands
% -----------------------------------------------------------------------------
%  Title Page.
%  ===========
\begin{document}
\scfrontmatter

\section{\xlabel{INTRO}Introduction}

\texttt{findcoords} is a utility for finding the equatorial coordinates of
a named astronomical object.  You simply enter the name of the object
and its coordinates are displayed.  \texttt{findcoords} works by submitting a
remote query via the Internet to the version of the
\htmladdnormallinkfoot{SIMBAD}{http://simbad.u-strasbg.fr/Simbad}
name-resolver provided by ESO (the basic SIMBAD is maintained by the
\htmladdnormallink{Centre de Donn\'{e}es astronomiques de Strasbourg}
{http://cdsweb.u-strasbg.fr/CDS.html}, CDS).  Consequently, \texttt{findcoords} will only work on computers with a suitable Internet
connection.  Also, the name given must be recognised by SIMBAD, though
the latter's dictionary of names is very extensive.  \texttt{findcoords} is a
simple wrap-around for the name-resolver function of application \texttt{catremote} in \htmladdnormallink{CURSA}{http://www.starlink.ac.uk/cursa/}
(see \xref{SUN/190}{sun190}{}\cite{SUN190} and
\xref{SSN/76}{ssn76}{}\cite{SSN76}).


\section{\xlabel{USAGE}Usage}

To find the equatorial coordinates of an astronomical object whose name
you know simply type:

\begin{quote}
\texttt{findcoords} ~ \textit{object-name}
\end{quote}

The \textit{object-name}\/ should be entered without embedded spaces.  The
case of letters (upper or lower) is not significant.  If the name is
recognised then the equatorial coordinates of the object will be displayed.
The Right Ascension is shown in sexagesimal hours and the Declination in
sexagesimal degrees; both are for equinox J2000.


\section{\xlabel{EXAMP}Examples}

\begin{terminalv}
findcoords ngc6240
findcoords iras20056+1834
findcoords bd+303639
findcoords pks1417-19
findcoords mkn477
findcoords altair
\end{terminalv}

% -- References ---------------------------------------------------------------

\begin{thebibliography}{99}

  \bibitem{SUN190} A.C.~Davenhall, 14 May 2001,
   \xref{SUN/190.9}{sun190}{}: \textit{CURSA --- Catalogue and Table
    Manipulation Applications}, Starlink.

  \bibitem{SSN76} A.C.~Davenhall, 24 May 2001,
   \xref{SSN/76.1}{ssn76}{}: \textit{CATREMOTE --- a Tool for Querying Remote
   Catalogues}, Starlink.

\end{thebibliography}

\typeout{  }
\typeout{*****************************************************}
\typeout{  }
\typeout{Reminder: run this document through Latex twice to}
\typeout{resolve the references.}
\typeout{  }
\typeout{*****************************************************}
\typeout{  }

\end{document}
