\newpage
\sstroutine{
   NDF2TIFF
}{
   Converts an NDF to an 8-bit TIFF-6.0-format file.
}{
   \sstdescription{
      This application converts an NDF to a Tag Image File Format (TIFF).
      One- or two-dimensional arrays can be handled.

      The routine first finds the brightest and darkest pixel values in the 
      image.  It then uses these to determine suitable scaling factors to
      convert the image into an 8-bit representation.  These are then output 
      to a simple greyscale TIFF-6.0 file.
   }
   \sstusage{
      ndf2tiff in out
   }
   \sstparameters{
      \sstsubsection{
         IN = NDF (Read)
      }{
         The name of the input NDF data structure (without the {\tt .sdf} 
         extension).  The suggested default is the current NDF if one exists,
         otherwise it is the current value.
      }
      \sstsubsection{
         OUT = \_CHAR (Read)
      }{
         The name of the TIFF file to be generated.
         The {\tt .tif} name extension is added to any output filename that
         does not contain it.     
      }
   }
   \sstexamples{
      \sstexamplesubsection{
         ndf2tiff old new
      }{
         This converts the NDF called old (in file {\tt old.sdf}) to the
         TIFF file called {\tt new.tif}.
      }
      \sstexamplesubsection{
         ndf2tiff in=spectre out=spectre.tif
      }{
         This converts the NDF called spectre (in file {\tt spectre.sdf}) 
         to the TIFF file called {\tt spectre.tif}.
      }
   }
   \sstnotes{
      This application generates only 256 grey levels and does not use 
      any image colour lookup table so absolute data values may be lost.

      No compression is applied.
   }
   \sstdiytopic{
      Related Applications
   }{
      \CONVERT: TIFF2NDF.
   }
   \sstimplementationstatus{
      Bad values in the data array are replaced with zero in the output
      TIFF file.
   }
}

