\newpage
\sstroutine{
   NDF2GIF
}{
   Converts an NDF into a GIF file.
}{
   \sstdescription{
      This Bourne-shell script converts an NDF into a 256 grey-level
      Graphics Interchange Format (GIF) file.  One- or two-dimensional
      images can be handled.  The script uses the CONVERT utility
      \htmlref{NDF2TIFF}{NDF2TIFF} to produce a TIFF file and then
      various PBMPLUS utilities to convert the TIFF file into a GIF file. 

      Error messages are converted into Starlink style (preceded by {\tt{!}}).
   }
   \sstusage{
      ndf2gif in [out]
   }
   \sstparameters{
      \sstsubsection{
         IN = NDF (Read)
      }{
         The name of the input NDF (without the {\tt .sdf} extension).
      }
      \sstsubsection{
         OUT = FILENAME (Write)
      }{
         The name of the GIF file to be generated (without the {\tt .gif}
         extension, which is appended).
         If this is omitted, the value of the IN parameter is used.
      }
   }
   \sstexamples{
      \sstexamplesubsection{
         ndf2gif old new
      }{
         This converts the NDF called old (in file {\tt old.sdf})
         into a GIF file {\tt new.gif}.
      }
   }
   \sstnotes{
      The following points should be remembered:
      \ssthitemlist{

         \sstitem
            This initial version of the script generates only 256 grey 
            levels and does not use the image colour lookup table so
            absolute data values may be lost.

         \sstitem
            The PBMPLUS utilities {\tt tifftopnm} and {\tt ppmtogif}
            must be available on your PATH.

         \sstitem
         At the time of writing, this utility uses a special Netpbm version
         of {\bf tifftopnm} on alpha OSF/1 due to a problem with {\bf tifftopnm}
         in the standard release of PBMPLUS.  Netpbm is based on PBMPLUS but
         contains many improvements and additions.
      }
   }
   \sstdiytopic{
      Related Applications
   }{
      \CONVERT: \htmlref{GIF2NDF}{GIF2NDF}.
   }
}

