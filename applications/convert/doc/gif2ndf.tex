\newpage
\sstroutine{
   GIF2NDF
}{
   Converts a GIF file into an NDF. 
}{
   \sstdescription{
      This Bourne-shell script converts a Graphics Interchange Format
      (GIF) file into an unsigned-byte (256 grey-level) NDF format file.
      It handles one- or two-dimensional images.  The script uses
      various PBMPLUS utilities to produce a FITS file, flipped top
      to bottom, and then \htmlref{FITS2NDF}{FITS2NDF} to produce the final NDF. 
      Error messages are converted into Starlink style (preceded by {\tt{!}}).
   }
   \sstusage{
      gif2ndf in [out]
   }
   \sstparameters{
      \sstsubsection{
         IN = FILENAME (Read)
      }{
         The name of the GIF file to be converted (without the {\tt .gif}
         extension, which is assumed).
      }
      \sstsubsection{
         OUT = NDF (Write)
      }{
         The name of the NDF to be generated (without the {\tt .sdf} extension).
         If the OUT parameter is omitted, the value of the IN parameter
         is used.
      }
   }
   \sstexamples{
      \sstexamplesubsection{
         gif2ndf old new
      }{
         This converts the GIF file {\tt old.gif} into an NDF called new
         (in file {\tt new.sdf}).
      }
      \sstexamplesubsection{
         gif2ndf horse
      }{
         This converts the GIF file {\tt horse.gif} into an NDF called horse
         (in file {\tt horse.sdf}).
      }
   }

   \sstnotes{ 
      The following points should be remembered:
      \ssthitemlist{ 

         \sstitem
            This initial version of the script generates images with at most
            256 grey levels.  It does not use the image colour lookup table.

         \sstitem
            Input image filenames must have the file extension {\tt .gif}.

         \sstitem
            The PBMPLUS utilities {\tt giftopnm}, {\tt ppmtopgm} and
            {\tt pgmtofits} must be available on your PATH.
      }
   }
   \sstdiytopic{
      Related Applications
   }{
      \CONVERT: \htmlref{NDF2GIF}{NDF2GIF}.
   }
}

