\newpage
\sstroutine{
   NDF2IRAF
}{
   Converts an NDF to an IRAF image
}{
   \sstdescription{
      This application converts an NDF to an IRAF image.  See the Notes
      for details of the conversion.
   }
   \sstusage{
      ndf2iraf in out [fillbad]
   }
   \sstparameters{
      \sstsubsection{
         IN = NDF (Read)
      }{
         The input NDF data structure.  The suggested default is the
         current NDF if one exists, otherwise it is the current value.
      }
      \sstsubsection{
         FILLBAD = \_REAL (Read)
      }{
         The value used to replace bad pixels in the NDF's data array
         before it is copied to the IRAF file.  A null value ({\tt !}) means
         no replacements are to be made.  This parameter is ignored if
         there are no bad values.  {\tt [0]}
      }
      \sstsubsection{
         OUT = LITERAL (Write)
      }{
         The name of the output IRAF image.  Two files are produced
         with the same name but different extensions. The {\tt ".pix"} file
         contains the data array, and {\tt ".imh"} is the associated header
         file that may contain a copy of the NDF's FITS extension.
         The suggested default is the current value.
      }
      \sstsubsection{
         PROFITS = \_LOGICAL (Read)
      }{
         If {\tt TRUE}, the contents of the FITS extension of the NDF are
         merged with the header information derived from the standard
         NDF components.  See the Notes for details of the merger.
         {\tt [TRUE]}
      }
      \sstsubsection{
         PROHIS = \_LOGICAL (Read)
      }{
         If {\tt TRUE}, any NDF history records are written to the primary
         FITS header as HISTORY cards.  These follow the mandatory
         headers and any merged FITS-extension headers (see parameter
         PROFITS).  {\tt [TRUE]}
      }
   }
   \sstexamples{
      \sstexamplesubsection{
         ndf2iraf abell119 a119
      }{
         Converts an NDF called abell119 into the IRAF image comprising
         the pixel file {\tt a119.pix} and the header file {\tt a119.imh}.  If there
         are any bad values present they are copied verbatim to the IRAF
         image.
      }
      \sstexamplesubsection{
         ndf2iraf abell119 a119 noprohis
      }{
         As the previous example, except that NDF HISTORY records are
         not transferred to the headers in {\tt a119.imh}.
      }
      \sstexamplesubsection{
         ndf2iraf qsospe qsospe fillbad=0
      }{
         Converts the NDF called qsospe to an IRAF image comprising the
         pixel file {\tt qsospe.imh} and the header file {\tt qsospe.pix}.  Any bad
         values in the data array are replaced by zero.
      }
      \sstexamplesubsection{
         ndf2iraf qsospe qsospe fillbad=0 profits=f
      }{
         As the previous example, except that any FITS airlock
         information in the NDF are not transferred to the headers in
         {\tt qsospe.imh}.
      }
   }
   \sstnotes{
      The rules for the conversion are as follows:
      \ssthitemlist{

         \sstitem
         The NDF data array is copied to the {\tt ".pix"} file.  Ancillary
         information listed below is written to the {\tt ".imh"} header file in
         FITS-like headers.

         \sstitem
         The IRAF ``Mini World Coordinate System'' (MWCS) is used to
         record axis information whenever one of the following criteria is
         satisfied:

         \begin{enumerate}
            \item the dataset has some linear axes (system=world);

            \item the dataset is one-dimensional with a non-linear axis, or is
            two-dimensional with the first axis non-linear and the
            second being some aperture number or index
            (system=multispec);

            \item the dataset has a linear spectral/dispersion axis along the
            first dimension and all other dimensions are pixel indices
            (system=equispec).
         \end{enumerate}

         \sstitem
         The NDF title, label, units are written to the header keywords
         TITLE, OBJECT, and BUNIT respectively if they are defined.
         Otherwise anys values for these keywords found in the FITS
         extension are used (provided parameter PROFITS is {\tt TRUE}).  There
         is a limit of twenty characters for each.

         \sstitem
         The NDF pixel origins are stored in keywords LBOUND$n$ for the
         nth dimension when any of the pixel origins is not equal to 1.

         \sstitem
         Keywords HDUCLAS1, HDUCLAS$n$ are set to {\tt "NDF"} and the
         array-component name respectively.

         \sstitem
         The BLANK keyword is set to the Starlink standard bad value,
         but only for the \_WORD data type and not for a quality array.  It
         appears regardless of whether or not there are bad values
         actually present in the array.

         \sstitem
         HISTORY headers are propagated from the FITS extension when
         PROFITS is {\tt TRUE}, and from the NDF history component when PROHIS
         is {\tt TRUE}.

         \sstitem
         If there is a FITS extension in the NDF, then the elements up
         to the first END keyword of this are added to the `user area' of
         the IRAF header file, when PROFITS={\tt TRUE}.  However, certain
         keywords are excluded: SIMPLE, NAXIS, NAXIS$n$, BITPIX, EXTEND,
         PCOUNT, GCOUNT, BSCALE, BZERO, END, and any already created from
         standard components of the NDF listed above.

         \sstitem
         A HISTORY record is added to the IRAF header file indicating
         that it originated in the named NDF and was converted by
         NDF2IRAF.

         \sstitem
         All other NDF components are not propagated.
      }
   }
   \sstdiytopic{
      Related Applications
   }{
      \CONVERT: \htmlref{IRAF2NDF}{IRAF2NDF}.
   }
   \sstdiytopic{
      Pitfalls
   }{
      The IMFORT routines refuse to overwrite an IRAF image if an image
      with the same name exists.  The application then aborts.

      Some of the routines required for accessing the IRAF header image
      are written in SPP. Macros are used to find the start of the
      header line section, this constitutes an `Interface violation' as
      these macros are not part of the IMFORT interface specification.
      It is possible that these may be changed in the future, so
      beware.
   }
   \sstdiytopic{
      References
   }{
      IRAF User Handbook Volume 1A: ``A User's Guide to FORTRAN
      Programming in IRAF, the IMFORT Interface'', by Doug Tody.
    }
   \sstimplementationstatus{
      \sstitemlist{

         \sstitem
         Only handles one-, two-, and three-dimensional NDFs.

         \sstitem
         Of the NDF's array components only the data array may be
         copied.

         \sstitem
         The IRAF image produced has type SIGNED WORD or REAL dependent
         of the type of the NDF's data array.  (The IRAF IMFORT FORTRAN
         subroutine library only supports these data types.)  For \_BYTE,
         \_UBYTE, and \_WORD data arrays the IRAF image will have type
         SIGNED WORD; for all other data types of the NDF data array a
         REAL IRAF image is made.  The pixel type of the image can be
         changed from within IRAF using the `chpixtype' task in the
         `images' package.

         \sstitem
         Bad values may arise due to type conversion.  These too are
         substituted by the (non-null) value of FILLBAD.
      }
   }
}

