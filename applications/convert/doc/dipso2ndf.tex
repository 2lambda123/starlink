\newpage
\sstroutine{
   DIPSO2NDF
}{
   Converts a DIPSO file to an NDF
}{
   \sstdescription{
      This application creates an NDF from a DIPSO-format file as
      produced by the DIPSO `WRITE' command.  See SUN/50.  The rules
      for the conversion are listed in the Notes.
   }
   \sstusage{
      DIPSO2NDF IN OUT
   }
   \sstparameters{
      \sstsubsection{
         IN = FILENAME (Read)
      }{
         Input DIPSO file.  File extension {\tt ".DAT"} is assumed.
      }
      \sstsubsection{
         OUT = NDF (Write)
      }{
         Output NDF data structure.  A file extension must not be given
         after the name.  It becomes the new current NDF.
      }
   }
   \sstexamples{
      \sstexamplesubsection{
         DIPSO2NDF OLD NEW
      }{
         This converts the DIPSO file OLD.DAT file to the NDF file
         NEW.SDF.
      }
      \sstexamplesubsection{
         DIPSO2NDF SPECTRE SPECTRE
      }{
         This converts the DIPSO file SPECTRE.DAT to the NDF called
         SPECTRE in file SPECTRE.SDF.
      }
   }
   \sstnotes{
      \sstitemlist{

         \sstitem
         The DIPSO title is written to the NDF TITLE.

         \sstitem
         The DIPSO main data array (often call the flux array) is
         copied to the NDF's data array.  DIPSO records bad values by
         means of breaks in the data array.  The number and positions of
         these breaks are stored in the DIPSO file.  This application
         inserts bad pixels at these break positions.  The number of bad
         pixels inserted is based on the size of the gap in the wavelength
         scale.  At least one bad pixel is inserted at every break point.

         \sstitem
         The $x$-axis array (otherwise known as the wavelength array) is
         copied to the NDF AXIS(1) centres.  Missing axis centres are
         generated by linear interpolation between the good $x$-axis values
         in the DIPSO file.

         \sstitem
         The data and the axis centres need not be fluxes and
         wavelengths respectively.
      }
   }
   \sstimplementationstatus{
      \sstitemlist{

         \sstitem
         The output NDF has a primitive data array.

         \sstitem
         The input wavelength and flux data are always of Fortran REAL
         type, the output data arrays are of HDS type \_REAL.

         \sstitem
         The application assumes that the bad-pixel padding will not
         cause the number of elements in the data array to exceed twice
         the original number.
      }
   }
}
