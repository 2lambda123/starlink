\newpage
\sstroutine{
   DST2NDF
}{
   Converts a Figaro (Version 2) DST file to an NDF
}{
   \sstdescription{
      This application converts a Figaro Version-2 DST file to a
      Version-3 file, {\it i.e.}\ to an NDF.  The rules for converting the
      various components of a DST are listed in the notes.  Since
      both are hierarchical formats most files can be be converted with
      little or no information lost.
   }
   \sstusage{
      DST2NDF IN OUT
   }
   \sstparameters{
      \sstsubsection{
         FORM = LITERAL (Read)
      }{
         The storage form of the NDF's data and variance arrays.
         FORM = {\tt "Simple"} gives the simple form, where the array of data
         and variance values is located in an ARRAY structure.  Here it
         can have ancillary data like the origin.  This is the normal
         form for an NDF.  FORM = {\tt "Primitive"} offers compatibility with
         earlier formats, such as IMAGE.  In the primitive form the
         data and variance arrays are primitive components at the top
         level of the NDF structure, and hence it cannot have
         ancillary information. {\tt ["Simple"]}
      }
      \sstsubsection{
         IN = Figaro file (Read)
      }{
         The file name of the version 2 file.  A file extension must
         not be given after the name, since {\tt ".dst"} is appended by the
         application.  The file name is limited to 80 characters.
      }
      \sstsubsection{
         OUT = NDF (Write)
      }{
         The file name of the output NDF file.  A file extension must
         not be given after the name, since {\tt ".sdf"} is appended by the
         application.  Since the NDF\_ library is not used, a section
         definition may not be given following the name.  The file
         name is limited to 80 characters.
      }
   }
   \sstexamples{
      \sstexamplesubsection{
         dst2ndf old new
      }{
         This converts the Figaro file old.dst to the NDF called new
         (in file new.sdf).  The NDF has the simple form.
      }
      \sstexamplesubsection{
         dst2ndf horse horse p
      }{
         This converts the Figaro file horse.dst to the NDF called
         horse (in file horse.sdf).  The NDF has the primitive form.
      }
   }
   \sstnotes{
      The rules for the conversion of the various components are as
      follows:
      \vspace{-\parskip}
      \begin{center}
      \begin{tabular}{|lcl|p{47mm}|}
      \hline 
      \multicolumn{1}{|c}{Figaro file} & & \multicolumn{1}{c}{NDF} &
      \multicolumn{1}{|c|}{Comments} \\ \hline
      .Z.DATA   & $\Rightarrow$ & .DATA\_ARRAY.DATA & when FORM = {\tt "SIMPLE"}\\
      .Z.DATA   & $\Rightarrow$ & .DATA\_ARRAY & when FORM = {\tt "PRIMITIVE"} \\
      .Z.ERRORS & $\Rightarrow$ & .VARIANCE.DATA & after processing when FORM = {\tt "SIMPLE"} \\
      .Z.ERRORS & $\Rightarrow$ & .VARIANCE & after processing when FORM = {\tt "PRIMITIVE"} \\
      .Z.QUALITY & $\Rightarrow$ & .QUALITY.QUALITY & must be BYTE array
                                  (see Bad-pixel handling below) \\
      & $\Rightarrow$ & .QUALITY.BADBITS = 255 & \\
      .Z.LABEL  & $\Rightarrow$ & .LABEL & \\
      .Z.UNITS  & $\Rightarrow$ & .UNITS & \\
      .Z.IMAGINARY & $\Rightarrow$ & .DATA\_ARRAY.IMAGINARY\_DATA & \\
      .Z.MAGFLAG & $\Rightarrow$ & .MORE.FIGARO.MAGFLAG & \\
      .Z.RANGE  & $\Rightarrow$ & .MORE.FIGARO.RANGE & \\
      .Z.xxxx   & $\Rightarrow$ & .MORE.FIGARO.Z.xxxx & \\ \hline
      \end{tabular}
      \end{center}

      \begin{center}
      \begin{tabular}{|lcl|p{43mm}|}
      \hline 
      \multicolumn{1}{|c}{Figaro file} & & \multicolumn{1}{c}{NDF} &
      \multicolumn{1}{|c|}{Comments} \\ \hline
      .X.DATA   & $\Rightarrow$ & .AXIS(1).DATA\_ARRAY & \\ 
      .X.ERRORS & $\Rightarrow$ & .AXIS(1).VARIANCE & after processing \\
      .X.WIDTH  & $\Rightarrow$ & .AXIS(1).WIDTH & \\
      .X.LABEL  & $\Rightarrow$ & .AXIS(1).LABEL & \\
      .X.UNITS  & $\Rightarrow$ & .AXIS(1).UNITS & \\
      .X.LOG    & $\Rightarrow$ & .AXIS(1).MORE.FIGARO.LOG & \\
      .X.xxxx   & $\Rightarrow$ & .AXIS(1).MORE.FIGARO.xxxx & \\
      & & & (Similarly for .Y .T .U .V or .W structures which are
             renamed to AXIS(2), \ldots, AXIS(6) in the NDF.) \\
      & & & \\
      .OBS.OBJECT & $\Rightarrow$ & .TITLE & \\
      .OBS.SECZ & $\Rightarrow$ & .MORE.FIGARO.SECZ & \\
      .OBS.TIME & $\Rightarrow$ & .MORE.FIGARO.TIME & \\
      .OBS.xxxx & $\Rightarrow$ & .MORE.FIGARO.OBS.xxxx & \\
      & & & \\
      .FITS.xxxx& $\Rightarrow$ & .MORE.FITS.xxxx & into value part of
         the string \\
      .COMMENTS.xxxx  & $\Rightarrow$ & .MORE.FITS.xxxx & into comment part of
         the string \\
      .FITS.xxxx.DATA & $\Rightarrow$ & .MORE.FITS.xxxx & into value part of
         the string \\
      .FITS.xxxx.DESCRIPTION & $\Rightarrow$ & .MORE.FITS.xxxx & into comment
         part of the string \\
      & & & \\
      .MORE.xxxx& $\Rightarrow$ & .MORE.xxxx & \\
      & & & \\
      .TABLE    & $\Rightarrow$ & .MORE.FIGARO.TABLE & \\
      .xxxx     & $\Rightarrow$ & .MORE.FIGARO.xxxx & \\ \hline
      \end{tabular}
      \end{center}

      Axis arrays with dimensionality greater than one are not
      supported by the NDF.  Therefore, if the application encounters
      such an axis array, it processes the array using the following
      rules, rather than those given above.
      \begin{center}
      \begin{tabular}{|lcl|p{51mm}|}
      \hline 
      \multicolumn{1}{|c}{Figaro file} & & \multicolumn{1}{c}{NDF} &
      \multicolumn{1}{|c|}{Comments} \\ \hline
      .X.DATA   & $\Rightarrow$ & .AXIS(1).MORE.FIGARO.DATA &
            AXIS(1).DATA\_ARRAY is filled with pixel co-ordinates \\
      .X.ERRORS & $\Rightarrow$ & .AXIS(1).MORE.FIGARO.VARIANCE & after
            processing \\
      .X.WIDTH  & $\Rightarrow$ & .AXIS(1).MORE.FIGARO.WIDTH & \\ \hline
      \end{tabular}
      \end{center}
   }

   \sstdiytopic{
   Bad-pixel handling
   }{
   The QUALITY array is only copied if the bad-pixel flag
   (.Z.FLAGGED) is false or absent.  A simple NDF with the bad-pixel
   flag set to false (meaning that there are no bad-pixels present)
   is created when .Z.FLAGGED is absent or false and FORM = {\tt "SIMPLE"}.
   }
   \sstimplementationstatus{
      The maximum number of dimensions is 6.
   }
}
