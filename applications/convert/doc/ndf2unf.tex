\newpage
\sstroutine{
   NDF2UNF
}{
   Converts an NDF to a sequential unformatted file
}{
   \sstdescription{
      This application converts an NDF to a sequential unformatted
      Fortran file.  Only one of the array components may be copied to
      the output file.  Preceding the data there is an optional header
      consisting of either the FITS extension with the values of
      certain keywords replaced by information derived from the NDF, or
      a minimal FITS header also derived from the NDF.
   }
   \sstusage{
      ndf2unf in out [comp] [noperec]
   }
   \sstparameters{
      \sstsubsection{
         COMP = LITERAL (Read)
      }{
         The NDF component to be copied.  It may be {\tt "Data"},
         {\tt "Quality"} or {\tt "Variance"}. {\tt ["Data"]}
      }
      \sstsubsection{
         FITS = \_LOGICAL (Read)
      }{
         If {\tt TRUE}, any FITS extension is written to start of the output
         file, unless there is no extension whereupon a minimal FITS
         header is written to the unformatted file. {\tt [FALSE]}
      }
      \sstsubsection{
         IN = NDF (Read)
      }{
         Input NDF data structure. The suggested default is the current
         NDF if one exists, otherwise it is the current value.
      }
      \sstsubsection{
         NOPEREC = \_INTEGER (Read)
      }{
         The number of data values per record of the output file.  It
         must be positive.  The suggested default is the current value.
         {\tt{[}}The first dimension of the NDF{\tt{]}}
      }
      \sstsubsection{
         OUT = FILENAME (Write)
      }{
         Name of the output sequential unformatted file.  The file will
         but always fixed-length records when there is no header.
      }
   }
   \sstexamples{
      \sstexamplesubsection{
         ndf2unf cluster cluster.dat
      }{
         This copies the data array of the NDF called cluster to an
         unformatted file called {\tt cluster.dat}.  The number of data values
         per record is equal to the size of the first dimension of the
         NDF.
      }
      \sstexamplesubsection{
         ndf2unf cluster cluster.dat v
      }{
         This copies the variance of the NDF called cluster to an
         unformatted file called {\tt cluster.dat}.  The number of variance
         values per record is equal to the size of the first dimension
         of the NDF.
      }
      \sstexamplesubsection{
         ndf2unf cluster cluster.dat noperec=12
      }{
         This copies the data array of the NDF called cluster to an
         unformatted file called {\tt cluster.dat}.  There are twelve data
         values per record in {\tt cluster.dat}.
      }
      \sstexamplesubsection{
         ndf2unf out=ndf234.dat fits in=@234
      }{
         This copies the data array of the NDF called 234 to an
         unformatted file called {\tt ndf234.dat}.  The number of data values
         per record is equal to the size of the first dimension of the
         NDF.  If there is a FITS extension, it is copied to {\tt ndf234.dat}
         with substitution of certain keywords, otherwise a minimal
         FITS header is produced.
      }
   }
   \sstnotes{
      The details of the conversion are as follows:
      \ssthitemlist{

         \sstitem
            the NDF array as selected by COMP is written to the
            unformatted file in records following an optional header.

         \sstitem
            HISTORY is not propagated.

         \sstitem
            ORIGIN information is lost.

         \sstitem
            When a header is to be made, it is composed of FITS-like card
            images as follows:
      
         \sstitemlist{

            \sstitem
               The number of dimensions of the data array is written
               to the keyword NAXIS, and the actual dimensions to NAXIS1,
               NAXIS2 {\it etc.} as appropriate.

            \sstitem
               If the NDF contains any linear axis structures the
               information necessary to generate these structures is
               written to the FITS-like headers. For example, if a linear
               AXIS(1) structure exists in the input NDF the value of the
               first data point is stored with the keyword CRVAL1,
               and the incremental value between successive axis data is
               stored in keyword CDELT1.  By definition the reference pixel is
               1.0 and is stored in keyword CRPIX1.  If there is an axis label
               it is written to keyword CTYPE1, and axis unit is written to CUNIT1.
               (Similarly for AXIS(2) structures {\it etc.}) FITS does not have
               a standard method of storing axis widths and variances, so these
               NDF components will not be propagated to the header.
               Non-linear axis data arrays cannot be represented by CRVAL{\em{n}}
               and CDELT{\em{n}}, and must be ignored.

            \sstitem
               If the input NDF contains TITLE, LABEL or UNITS components
               these are stored with the keywords TITLE, LABEL or BUNIT
               respectively.

            \sstitem
               If the input NDF contains a FITS extension, the FITS items
               may be written to the FITS-like header, with the following
               exceptions:
               \begin{itemize}
               \item BITPIX is derived from the type of the NDF data array,
               and so it is not copied from the NDF FITS extension.
               \item NAXIS, and NAXIS{\em{n}} are derived from the dimensions of the
               NDF data array as described above, so these items are not
               copied from the NDF FITS extension.
               \item The TITLE, LABEL, and BUNIT descriptors are only copied
               if no TITLE, LABEL, and UNITS NDF components respectively
               have already been copied into these headers.
               \item The CDELT{\em{n}}, CRVAL{\em{n}}, CTYPE{\em{n}},
               CUNIT{\em{n}}, and CRTYPE{\em{n}} descriptors
               in the FITS extension are only copied if the input NDF
               contained no linear axis structures.
               \item The standard order of the FITS keywords is preserved,
               thus BITPIX, NAXIS and NAXIS{\em{n}} appear immediately after the
               first card image, which should be SIMPLE.
               \item BSCALE and BZERO in a FITS extension are copied when
               BITPIX is positive, {\it i.e.} the array is not floating-point.
               \end{itemize}

            \sstitem
               An extra header record with keyword UNSIGNED and logical
               value T is added when the array data type is one of the HDS
               unsigned integer types.  This is done because standard FITS
               does not support unsigned integers, and allows (in conjunction
               with BITPIX) applications reading the unformatted file to
               determine the data type of the array.

            \sstitem
               The last header record card will be the standard FITS END.
         }

         \sstitem
            Other extensions are not propagated.
      }
   }
   \sstdiytopic{
      Related Applications
   }{
      \CONVERT: \htmlref{UNF2NDF}{UNF2NDF}.
   }
   \sstimplementationstatus{
      \sstitemlist{

         \sstitem
         The value of bad pixels is not written to a FITS-like header
         record with keyword BLANK.
      }
   }
}

