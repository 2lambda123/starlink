\newpage
\sstroutine{
   IRAF2NDF
}{
   Converts an IRAF image to an NDF
}{
   \sstdescription{
      This application converts an IRAF image to an NDF.
      See the Notes for details of the conversion.
   }
   \sstusage{
      iraf2ndf in out
   }
   \sstparameters{
      \sstsubsection{
         IN = LITERAL (Read)
      }{
         The name of the IRAF image.  Note that this excludes the extension.
      }
      \sstsubsection{
         OUT = NDF (Write).
      }{
         The name of the NDF to be produced.
      }
   }
   \sstexamples{
      \sstexamplesubsection{
         iraf2ndf ell\_galaxy new\_galaxy
      }{
         Converts the IRAF image ell\_galaxy (comprising files
         ell\_galaxy.imh and ell\_galaxy.hdr) to an NDF called new\_galaxy.
      }
   }
   \sstnotes{
      The rules for the conversion are as follows:
      \sstitemlist{

         \sstitem
         The NDF data array is copied from the {\tt "}.pix{\tt "} file.

         \sstitem
         The title of the IRAF image (object i\_title in the {\tt "}.imh{\tt "}
         header file) becomes the NDF title.

         \sstitem
         Lines from the IRAF image header file are transferred to the
         FITS extension of the NDF, any compulsory FITS keywords that are
         missing are added.

         \sstitem
         If there is a FITS extension in the NDF, then the elements up
         to the first END keyword of this are added to the `user area' of
         the IRAF header file.

         \sstitem
         Lines from the HISTORY section of the IRAF image are also
         extracted and added to the NDF's FITS extension as FITS HISTORY
         lines.  Two extra HISTORY lines are added to record the original
         name of the image and the date of the format conversion.
      }
   }
   \sstdiytopic{
      Related Applications
   }{
      CONVERT: NDF2IRAF.
   }
   \sstimplementationstatus{
      \sstitemlist{

         \sstitem
         It is only available on VMS, SunOS, and Ultrix systems.  On
         Solaris 2.3 systems the version built on SunOS can be used in
         compatibility mode, but there is no guarantee that this will
         work for all IRAF images.  At the time of writing there was no
         working IRAF imfort library available for Alpha/OSF1.

         \sstitem
         Only handles one-, two-, and three-dimensional IRAF files.

         \sstitem
         The NDF produced has type \_WORD or \_REAL corresponding to the
         type of the IRAF image.  (The IRAF imfort FORTRAN subroutine
         library only supports these data types: signed words and real.)
         The pixel type of the image can be changed from within IRAF using
         the {\bf chpixtype} task in the {\bf images} package.
      }
   }
}
