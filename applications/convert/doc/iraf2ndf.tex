\newpage
\sstroutine{
   IRAF2NDF
}{
   Converts an IRAF image to an NDF
}{
   \sstdescription{
      This application converts an IRAF image to an NDF.  See the Notes
      for details of the conversion.
   }
   \sstusage{
      iraf2ndf in out
   }
   \sstparameters{
      \sstsubsection{
         IN = LITERAL (Read)
      }{
         The name of the IRAF image.  Note that this excludes the
         {\tt ".imh"} file extension.
      }
      \sstsubsection{
         OUT = NDF (Write)
      }{
         The name of the NDF to be produced.
      }
      \sstsubsection{
         PROFITS = \_LOGICAL (Read)
      }{
         If {\tt TRUE}, the headers of the IRAF file are written
         verbatim to the NDF's FITS extension.  Any IRAF history
         records are also appended to FITS extension. {\tt [TRUE]}
      }
      \sstsubsection{
         PROHIS = \_LOGICAL (Read)
      }{
         This parameter decides whether or not to create NDF HISTORY
         records.  Only the IRAF headers with keyword HISTORY, and
         which originated from NDF HISTORY records are used.  If
         PROHIS={\tt TRUE}, NDF HISTORY records are created.  {\tt [TRUE]}
      }
   }
   \sstexamples{
      \sstexamplesubsection{
         iraf2ndf ell\_galaxy new\_galaxy
      }{
         Converts the IRAF image ell\_galaxy (comprising files
         {\tt ell\_galaxy.imh} and {\tt ell\_galaxy.pix}) to an NDF
         called new\_galaxy.
      }
      \sstexamplesubsection{
         iraf2ndf ell\_galaxy new\_galaxy noprofits noprohis
      }{
         As above, except no FITS extension is created, and NDF-style
         HISTORY lines in {\tt ell\_galaxy.imh} are not transferred to HISTORY
         records in NDF new\_galaxy.
      }
   }
   \sstnotes{
      The rules for the conversion are as follows:
      \ssthitemlist{

         \sstitem
         The NDF data array is copied from the {\tt ".pix"} file.

         \sstitem
         The title of the IRAF image (object i\_title in the {\tt ".imh"}
         header file) becomes the NDF title.  Likewise headers OBJECT and
         BUNIT become the NDF label and units respectively.

         \sstitem
         Lines from the IRAF image header file may be transferred to
         the FITS extension of the NDF, when PROFITS={\tt TRUE}.  Any
         compulsory FITS keywords that are missing are added.  Certain
         other keywords are not propagated.  These are the IRAF ``Mini
         World Coordinate System'' (MWCS) keywords WCSDIM, DC\_FLAG,
         WAT$d$\_$nnn$ ($d$ is dimension, $nnn$ is the line number).
         Certain NDF-style HISTORY lines in the header may also
         be ignored when PROHIS={\tt TRUE} (see two notes below).

         \sstitem
         When PROFITS={\tt TRUE}, lines from the HISTORY section of the IRAF
         image are also extracted and added to the NDF's FITS extension as
         FITS HISTORY lines.  Two extra HISTORY lines are added to record
         the original name of the image and the date of the format
         conversion.

         \sstitem
         When PROHIS={\tt TRUE}, any HISTORY lines in the IRAF headers, which
         originated from an NDF2IRAF conversion of NDF HISTORY records.
         Such headers are not transferred to the FITS airlock, when
         PROFITS={\tt TRUE}.

         \sstitem
         Most axis information can be propagated either from standard
         FITS-like keywords, or certain MCWS headers.  Supported systems
         and formats are listed below.

         \sstitemlist{
            \medskip

            \sstitem
            FITS

            \ssthitemlist{
               \vspace*{-\bigskipamount}
               \sstitem
               linear

               \sstitem
               log-linear
            }
            \medskip

            \sstitem
            Equispec

            \ssthitemlist{
               \vspace*{-\bigskipamount}
               \sstitem
               linear

               \sstitem
               log-linear
            }
            \medskip

            \sstitem
            Multispec

            \ssthitemlist{
               \vspace*{-\bigskipamount}
               \sstitem
               linear

               \sstitem
               log-linear

               \sstitem
               Chebyshev and Legendre polynomials

               \sstitem
               Linear and cubic Spline

               \sstitem
               Explicit list of co-ordinates
               \vspace*{\medskipamount}
            }
         }

         However, for Multispec axes, only the first (spec1) axis
         co-ordinates are transferred to the NDF AXIS centres.  Any
         spec2 \dots spec$n$ co-ordinates, present when the data array is not
         one-dimensional or multiple fits have been stored, are ignored.
         The weights for multiple fits are thus also ignored.  The data
         type of the axis centres is \_REAL or \_DOUBLE depending on the
         number of significant digits in the co-ordinates or coefficients.

         The axis labels and units are also propagated, where present, to
         the NDF AXIS structure.  In the FITS system, these are derived
         from the CTYPE$n$ and CUNIT$n$ keywords.  In the MWCS, these
         components originate in the label and units parameters.

         The redshift correction, when present, is applied to the MCWS
         axis co-ordinates.

      }
   }
   \sstdiytopic{
      Related Applications
   }{
      \CONVERT: \htmlref{NDF2IRAF}{NDF2IRAF}.
   }
   \sstdiytopic{
      Pitfalls
   }{
      \sstitemlist{

         \sstitem
         Bad pixels in the IRAF image are not replaced.

         \sstitem
         Some of the routines required for accessing the IRAF header
         file are written in SPP.  Macros are used to find the start of the
         header line section, this constitutes an `Interface violation' as
         these macros are not part of the IMFORT interface specification.
         It is possible that these may be changed in the future, so
         beware.
      }
   }
   \sstdiytopic{
      References
   }{
      IRAF User Handbook Volume 1A: ``A User's Guide to FORTRAN
      Programming in IRAF, the IMFORT Interface'', by Doug Tody.
   }
   \sstdiytopic{
      Keywords
   }{
      CONVERT, IRAF
   }
   \sstimplementationstatus{
      \sstitemlist{

         \sstitem
         Only handles one-, two-, and three-dimensional IRAF files.

         \sstitem
         The NDF produced has type \_WORD or \_REAL corresponding to the
         type of the IRAF image.  (The IRAF IMFORT FORTRAN subroutine
         library only supports these data types: signed words and real.)
         The pixel type of the image can be changed from within IRAF using
         the `chpixtype' task in the `images' package.
      }
   }
}

