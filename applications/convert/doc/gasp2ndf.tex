\newpage
\sstroutine{
   GASP2NDF
}{
   Converts an image in GASP format to an NDF
}{
   \sstdescription{
      This application converts a GAlaxy Surface Photometry (GASP)
      format file into an NDF.
   }
   \sstusage{
      gasp2ndf in out shape=?
   }
   \sstparameters{
      \sstsubsection{
         IN = LITERAL (Read)
      }{
         A character string containing the name of GASP file to convert.
         The extension should not be given, since {\tt "}.dat{\tt "} is assumed.
      }
      \sstsubsection{
         OUT = NDF (Write)
      }{
         The name of the output NDF.
      }
      \sstsubsection{
         SHAPE( 2 ) = \_INTEGER (Read)
      }{
         The dimensions of the GASP image (the number of columns
         followed by the number of rows).  Each dimension must be in the
         range 1 to 1024.  This parameter is only used if supplied on
         the command line, or if the header file corresponding to the
         GASP image does not exist or cannot be opened.
      }
   }
   \sstexamples{
      \sstexamplesubsection{
         gasp2ndf m31\_gasp m31
      }{
         Convert a GASP file called m31\_gasp.dat into an NDF called m31.
         The dimensions of the image are taken from the header file
         m31\_gasp.hdr.
      }
      \sstexamplesubsection{
         gasp2ndf n1068 ngc1068 shape=[256,512]
      }{
         Take the pixel values in the GASP file n1068.dat and create
         the NDF ngc1068 with dimensions 256 columns by 512 rows.
      }
   }
   \sstnotes{
      \sstitemlist{

         \sstitem
         A GASP image is limited to a maximum of 1024 by 1024 elements.
         It must be two dimensional.

         \sstitem
         The GASP image is written to the NDF's data array.  The data
         array has type \_WORD. No other NDF components are created.

         \sstitem
         If the header file is corrupted, the user must remove the
         offending {\tt "}.hdr{\tt "} file or specify the shape of the GASP image on the
         command line, otherwise the application will continually abort.
      }
   }
   \sstdiytopic{
      Related Applications
   }{
      NDF2GASP
   }
   \sstdiytopic{
      References
   }{
      GASP documentation (MUD/66).
   }
}
