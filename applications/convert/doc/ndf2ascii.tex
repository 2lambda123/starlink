\newpage
\sstroutine{
   NDF2ASCII
}{
   Writes an NDF to an ASCII file
}{
   \sstdescription{
      This application converts an NDF to an ASCII file.  Only one of
      the array components may be copied to the output file.  Preceding
      the data there is an optional header consisting of either the
      FITS extension with the values of certain keywords replaced by
      information derived from the NDF, or a minimal FITS header also
      derived from the NDF.  The output file uses LIST carriagecontrol.
   }
   \sstusage{
      NDF2ASCII IN OUT [COMP] [RECLEN] NOPEREC=?
   }
   \sstparameters{
      \sstsubsection{
         COMP = LITERAL (Read)
      }{
         The NDF component to be copied.  It may be {\tt "Data"},
         {\tt "Quality"} or {\tt "Variance"}. {\tt ["Data"]}
      }
      \sstsubsection{
         FITS = \_LOGICAL (Read)
      }{
         If true, any FITS extension is written to start of the output
         file, unless there is no extension whereupon a minimal FITS
         header is written to the ASCII file. {\tt [FALSE]}
      }
      \sstsubsection{
         FIXED = \_LOGICAL (Read)
      }{
         If true, the output file allocates a fixed number of
         characters per data value.  The number of characters chosen is
         the minimum that prevents any loss of precision, and hence is
         dependent on the data type of the NDF array.  When FIXED is
         {\tt FALSE}, data values are packed as efficiently as possible within
         each record. {\tt [FALSE]}
      }
      \sstsubsection{
         IN = NDF (Read)
      }{
         Input NDF data structure. The suggested default is the current
         NDF if one exists, otherwise it is the current value.
      }
      \sstsubsection{
         NOPEREC = \_INTEGER (Read)
      }{
         The number of data values per record of the output file, when
         FIXED is {\tt TRUE}.  It should be in the range 1 to 16383.
         The suggested default is the current value, or 8 when there
         is not one.
      }
      \sstsubsection{
         OUT = FILENAME (Write)
      }{
         Name of the output formatted Fortran file.  The file will
         normally have variable-length records when there is a header,
         but always fixed-length records when there is no header.  It
         has a default file extension of {\tt ".DAT"}.
      }
      \sstsubsection{
         RECLEN = \_INTEGER (Write)
      }{
         The maximum record length in bytes of the output file.  This
         must be between 32 and 32766.  The lower limit is further
         increased to 80 when there is a FITS header to be copied.  It
         is only used when FIXED is {\tt FALSE} and will default to the
         current value, or 512 if there is no current value.  When
         FIXED is {\tt TRUE} the application creates data records whose
         length is the product of the number of bytes per value plus
         one (for the space), times the number of values per record.
         {\tt []}
      }
   }
   \sstexamples{
      \sstexamplesubsection{
         NDF2ASCII CLUSTER CLUSTER.DAT
      }{
         This copies the data array of the NDF called CLUSTER to an
         ASCII file called CLUSTER.DAT.  The maximum recordlength of
         CLUSTER.DAT is 512 bytes, and the data values are packed into
         these records as efficiently as possible.
      }
      \sstexamplesubsection{
         NDF2ASCII CLUSTER CLUSTER.DAT V
      }{
         This copies the variance of the NDF called CLUSTER to an
         ASCII file called CLUSTER.DAT.  The maximum recordlength of
         CLUSTER.DAT is 512 bytes, and the variance values are packed
         into these records as efficiently as possible.
      }
      \sstexamplesubsection{
         NDF2ASCII CLUSTER CLUSTER.DAT FIXED NOPEREC=12
      }{
         This copies the data array of the NDF called CLUSTER to an
         ASCII file called CLUSTER.DAT.  There are twelve data values
         per record in CLUSTER.DAT.
      }
      \sstexamplesubsection{
         NDF2ASCII OUT=NDF234.DAT FITS RECLEN=128 IN=@234
      }{
         This copies the data array of the NDF called 234 to an
         ASCII file called NDF234.DAT.  The maximum recordlength of
         CLUSTER.DAT is 128 bytes, and the data values are packed into
         these records as efficiently as possible.  If there is a FITS
         extension, it is copied to NDF234.DAT with substitution of
         certain keywords, otherwise a minimal FITS header is produced.
      }
   }
   \sstnotes{
      The details of the conversion are as follows:
      \sstitemlist{

         \sstitem
            the NDF array as selected by COMP is written to the ASCII
            file in records following an optional header.  When FIXED is
            {\tt FALSE} all records are padded out to the recordlength.

         \sstitem
            HISTORY is not propagated.

         \sstitem
            ORIGIN information is lost.

         \sstitem
            When a header is to be made, it is composed of FITS-like card
            images as follows:
      
         \sstitemlist{

            \sstitem
               The number of dimensions of the data array is written
               to the keyword NAXIS, and the actual dimensions to NAXIS1,
               NAXIS2 {\it etc.} as appropriate.

            \sstitem
               If the NDF contains any linear axis structures the
               information necessary to generate these structures is
               written to the FITS-like headers. For example, if a linear
               AXIS(1) structure exists in the input NDF the value of the
               first data point is stored with the keyword CRVAL1,
               and the incremental value between successive axis data is
               stored in keyword CDELT1. If there is an axis label it is
               written to keyword CRTYPE1, and axis unit is written to CTYPE1.
               (Similarly for AXIS(2) structures {\it etc.}) FITS does not have
               a standard method of storing axis widths and variances, so these
               NDF components will not be propagated to the header.
               Non-linear axis data arrays cannot be represented by CRVAL$n$
               and CDELT$n$, and must be ignored.

            \sstitem
               If the input NDF contains TITLE, LABEL or UNITS components
               these are stored with the keywords TITLE, LABEL or BUNITS
               respectively.

            \sstitem
               If the input NDF contains a FITS extension, the FITS items
               may be written to the FITS-like header, with the following
               exceptions:
               \begin{itemize}
               \item BITPIX is derived from the type of the NDF data array,
               and so it is not copied from the NDF FITS extension.
               \item NAXIS, and NAXIS$n$ are derived from the dimensions of the
               NDF data array as described above, so these items are not
               copied from the NDF FITS extension.
               \item The TITLE, LABEL, and BUNITS descriptors are only copied
               if no TITLE, LABEL, and UNITS NDF components respectively
               have already been copied into these headers.
               \item The CDELT$n$, CRVAL$n$, CTYPE$n$, and CRTYPE$n$ descriptors
               in the FITS extension are only copied if the input NDF
               contained no linear axis structures.
               \item The standard order of the FITS keywords is preserved,
               thus BITPIX, NAXIS and AXIS$n$ appear immediately after the
               first card image, which should be SIMPLE.
               \item BSCALE and BZERO in a FITS extension are copied when
               BITPIX is positive, {\it i.e.} the array is not floating-point.
               \end{itemize}

            \sstitem
               An extra header record with keyword UNSIGNED and logical
               value T is added when the array data type is one of the HDS
               unsigned integer types.  This is done because standard FITS
               does not support unsigned integers, and allows (in conjunction
               with BITPIX) applications reading the unformatted file to
               determine the data type of the array.

            \sstitem
               The last header record card will be the standard FITS END.
         }

         \sstitem
            Other extensions are not propagated.
      }
   }
   \sstdiytopic{
      Related Applications
   }{
      CONVERT: ASCII2NDF; KAPPA: TRANDAT; SPECDRE: ASCIN and ASCOUT.
   }
   \sstimplementationstatus{
      \sstitemlist{

         \sstitem
         The value of bad pixels is not written to a FITS-like header
         record with keyword BLANK.
      }
   }
}
