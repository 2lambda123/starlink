\documentstyle[11pt]{article} 
\pagestyle{myheadings}

%------------------------------------------------------------------------------
\newcommand{\stardoccategory}  {Starlink User Note}
\newcommand{\stardocinitials}  {SUN}
\newcommand{\stardocnumber}    {55.5}
\newcommand{\stardocauthors}   {Malcolm J. Currie}
\newcommand{\stardocdate}      {1993 November 23}
\newcommand{\stardoctitle}     {CONVERT --- A Format-conversion Package}
%------------------------------------------------------------------------------

\newcommand{\stardocname}{\stardocinitials /\stardocnumber}
\renewcommand{\_}{{\tt\char'137}}     % re-centres the underscore
\markright{\stardocname}
\setlength{\textwidth}{160mm}
\setlength{\textheight}{230mm}
\setlength{\topmargin}{-2mm}
\setlength{\oddsidemargin}{0mm}
\setlength{\evensidemargin}{0mm}
\setlength{\parindent}{0mm}
\setlength{\parskip}{\medskipamount}
\setlength{\unitlength}{1mm}

%------------------------------------------------------------------------------
% Add any \newcommand or \newenvironment commands here
% SST definitions
% ---------------

%+
%  Name:
%     LAYOUT.TEX

%  Purpose:
%     Define Latex commands for laying out documentation produced by PROLAT.

%  Language:
%     Latex

%  Type of Module:
%     Data file for use by the PROLAT application.

%  Description:
%     This file defines Latex commands which allow routine documentation
%     produced by the SST application PROLAT to be processed by Latex. The
%     contents of this file should be included in the source presented to
%     Latex in front of any output from PROLAT. By default, this is done
%     automatically by PROLAT.

%  Notes:
%     The definitions in this file should be included explicitly in any file
%     which requires them. The \include directive should not be used, as it
%     may not then be possible to process the resulting document with Latex
%     at a later date if changes to this definitions file become necessary.

%  Authors:
%     RFWS: R.F. Warren-Smith (STARLINK)

%  History:
%     10-SEP-1990 (RFWS):
%        Original version.
%     10-SEP-1990 (RFWS):
%        Added the implementation status section.
%     12-SEP-1990 (RFWS):
%        Added support for the usage section and adjusted various spacings.
%     {enter_further_changes_here}

%  Bugs:
%     {note_any_bugs_here}

%-

%  Define length variables.
\newlength{\sstbannerlength}
\newlength{\sstcaptionlength}

%  Define a \tt font of the required size.
\font\ssttt=cmtt10 scaled 1095

%  Define a command to produce a routine header, including its name,
%  a purpose description and the rest of the routine's documentation.
\newcommand{\sstroutine}[3]{
   \goodbreak
   \rule{\textwidth}{0.5mm}
   \vspace{-7ex}
   \newline
   \settowidth{\sstbannerlength}{{\Large {\bf #1}}}
   \setlength{\sstcaptionlength}{\textwidth}
   \addtolength{\sstbannerlength}{0.5em}
   \addtolength{\sstcaptionlength}{-2.0\sstbannerlength}
   \addtolength{\sstcaptionlength}{-4.45pt}
   \parbox[t]{\sstbannerlength}{\flushleft{\Large {\bf #1}}}
   \parbox[t]{\sstcaptionlength}{\center{\Large #2}}
   \parbox[t]{\sstbannerlength}{\flushright{\Large {\bf #1}}}
   \begin{description}
      #3
   \end{description}
}

%  Format the description section.
\newcommand{\sstdescription}[1]{\item[Description:] #1}

%  Format the usage section.
\newcommand{\sstusage}[1]{\item[Usage:] \mbox{} \\[1.3ex] {\ssttt #1}}

%  Format the invocation section.
\newcommand{\sstinvocation}[1]{\item[Invocation:]\hspace{0.4em}{\tt #1}}

%  Format the arguments section.
\newcommand{\sstarguments}[1]{
   \item[Arguments:] \mbox{} \\
   \vspace{-3.5ex}
   \begin{description}
      #1
   \end{description}
}

%  Format the returned value section (for a function).
\newcommand{\sstreturnedvalue}[1]{
   \item[Returned Value:] \mbox{} \\
   \vspace{-3.5ex}
   \begin{description}
      #1
   \end{description}
}

%  Format the parameters section (for an application).
\newcommand{\sstparameters}[1]{
   \item[Parameters:] \mbox{} \\
   \vspace{-3.5ex}
   \begin{description}
      #1
   \end{description}
}

%  Format the output results parameters section (for an application).
\newcommand{\sstresparameters}[1]{
   \item[Results Parameters:] \mbox{} \\
   \vspace{-3.5ex}
   \begin{description}
      #1
   \end{description}
}

%  Format the graphics style parameters section (for an application).
\newcommand{\sstgraphparameters}[1]{
   \item[Graphics-style Parameters:] \mbox{} \\
   \vspace{-3.5ex}
   \begin{description}
      #1
   \end{description}
}

%  Format the examples section.
\newcommand{\sstexamples}[1]{
   \item[Examples:] \mbox{} \\
   \vspace{-3.5ex}
   \begin{description}
      #1
   \end{description}
}

%  Define the format of a subsection in a normal section.
\newcommand{\sstsubsection}[1]{\item[{#1}] \mbox{} \\}

%  Define the format of a subsection in the examples section.
\newcommand{\sstexamplesubsection}[1]{\item[{\ssttt #1}] \mbox{} \\}

%  Format the notes section.
\newcommand{\sstnotes}[1]{\item[Notes:] \mbox{} \\[1.3ex] #1}

%  Provide a general-purpose format for additional (DIY) sections.
\newcommand{\sstdiytopic}[2]{\item[{\hspace{-0.35em}#1\hspace{-0.35em}:}] \mbox{} \\[1.3ex] #2}

%  Format the implementation status section.
\newcommand{\sstimplementationstatus}[1]{
   \item[{Implementation Status:}] \mbox{} \\[1.3ex] #1}

%  Format the bugs section.
\newcommand{\sstbugs}[1]{\item[Bugs:] #1}

%  Format a list of items while in paragraph mode.
\newcommand{\sstitemlist}[1]{
  \mbox{} \\
  \vspace{-3.5ex}
  \begin{itemize}
     #1
  \end{itemize}
}

%  Define the format of an item.
\newcommand{\sstitem}{\item}

%  End of LAYOUT.TEX layout definitions.
%.

% End of SST definitions
% ----------------------

%------------------------------------------------------------------------------

\begin{document}
\thispagestyle{empty}
SCIENCE \& ENGINEERING RESEARCH COUNCIL \hfill \stardocname\\
RUTHERFORD APPLETON LABORATORY\\
{\large\bf Starlink Project\\}
{\large\bf \stardoccategory\ \stardocnumber}
\begin{flushright}
\stardocauthors\\
\stardocdate
\end{flushright}
\vspace{-4mm}
\rule{\textwidth}{0.5mm}
\vspace{5mm}
\begin{center}
{\Large\bf \stardoctitle}
\end{center}
\vspace{5mm}

%------------------------------------------------------------------------------
%  Add this part if you want a table of contents
  \setlength{\parskip}{0mm}
  \tableofcontents
  \setlength{\parskip}{\medskipamount}
  \markright{\stardocname}
%------------------------------------------------------------------------------

\newpage
\section{Introduction}

If life were simple there would only be one data format, but in reality
there are numerous formats for storing $n$-dimensional astronomical data
associated with various software packages.   In Starlink we have not
been immune to this, having the original INTERIM BDF, HDS IMAGE format,
and Figaro DSTs to name but three.  However, Starlink is now taking the
novel approach of supporting different packages sharing a common data
format---the NDF\footnote{See SUN/33 for an introduction to the NDF.}
(Extensible $n$-Dimensional-data format)---which most Starlink packages
are already using and more applications will follow. 

The purpose of {\footnotesize CONVERT} is the interchange of data files
to and from the NDF.  Thus it enables astronomers to select the best
applications from a variety of packages, including those originating
abroad like IRAF. In addition it assists packages that wish to move to
using the NDF.  {\footnotesize CONVERT} also gives a migration path for
old, Interim BDF datasets that lurk on many a tape, and that will not be
readable on UNIX. 

{\footnotesize CONVERT} is available from both VMS and UNIX platforms.
The supported conversions are currently as follows: 

\begin{center}
\begin{tabular}{l@{ -- }l}
\medskip
{\bf ASCII2NDF} &  Converts a text file to an NDF. \\ \medskip
{\bf BDF2NDF} &  Converts a Starlink Interim BDF file to an NDF. \\ \medskip
{\bf DIPSO2NDF} & Converts a DIPSO-format file to an NDF. \\ \medskip
{\bf DST2NDF} &  Converts a Figaro (Version 2) DST file to an NDF. \\ \medskip
{\bf GASP2NDF} &  Converts an image in GASP format to an NDF. \\ \medskip
{\bf IRAF2NDF} &  Converts an IRAF image to an NDF. \\ \medskip
{\bf IRCAM2NDF} &  Converts an IRCAM data file to a series of NDFs. \\ \medskip
{\bf NDF2ASCII} &  Converts an NDF to a text file. \\ \medskip
{\bf NDF2BDF} &  Converts an NDF to a Starlink Interim BDF file. \\ \medskip
{\bf NDF2DIPSO} &  Converts an NDF to a DIPSO-format file. \\ \medskip
{\bf NDF2DST} & Converts an NDF to a Figaro (Version 2) DST file. \\ \medskip
{\bf NDF2GASP} & Converts a two-dimensional NDF into a GASP image. \\ \medskip
{\bf NDF2IRAF} & Converts an NDF to an IRAF image. \\ \medskip
{\bf NDF2UNF} &  Converts an NDF to a sequential unformatted file. \\ \medskip
{\bf UNF2NDF} &  Converts a sequential unformatted file to an NDF. \\
\end{tabular}
\end{center}
In addition there are FITS readers within {\footnotesize KAPPA} (SUN/95)
which will convert FITS files to NDFs.

The various formats supported by {\footnotesize CONVERT} do not have
one-to-one correspondence and therefore in general it is not possible to
apply a forward and reverse conversion and finish with a duplicate of
the initial data file.  This hysteresis is particularly likely when
starting with an NDF, since many simpler formats have no way of storing
certain NDF data items, like variance and axis widths.  However, if you
are dealing with a simple file containing just a data array and linear
axis centres, then it should be possible to avoid loss of information. 

{\sl Note---the input data file is not deleted or altered in any way, 
with the exception of the input BDF in BDF2NDF, which can accumulate an 
extra `incarnation' of the data array as described in
Appendix~\ref{ap:full}.}


\section{Running CONVERT}

You have a choice of command language from where you may run
{\footnotesize CONVERT}.  On VMS platforms you can choose either from
{\footnotesize DCL} or {\footnotesize ICL} (the {\footnotesize ADAM}
command language), and from the shell on UNIX platforms ({\footnotesize
ICL} is expected soon).  Under VMS {\footnotesize DCL} is 
convenient for one-off conversions, and {\footnotesize ICL} is
recommended for a series of conversions because of its greater
efficiency, as the applications are part of a monolith that remains
loaded, and so after you invoke the first {\footnotesize CONVERT}
application subsequent ones run immediately. 

\subsection{Starting CONVERT from DCL}

To define the commands used by {\footnotesize CONVERT} you must first
issue the command\footnote{The startup command cannot be {\tt CONVERT}
since there is a VMS command called {\tt CONVERT}.} 

\small
\begin{verbatim}
      $ CONVERTSTART
\end{verbatim}
\normalsize 
Note that the {\tt \$} is the standard VMS prompt which you do not type.
This executes a procedure to set up symbols for {\footnotesize
CONVERT}'s command names, and defines some logical names to make help
information available.  The procedure, like the other setup commands
below, tells you the current version number and other introductory
information like this. 

\small
\begin{verbatim}

      --    Initialised for CONVERT    --
      --  Version 0.5, 1993 November  --

         Type CONHELP for CONVERT help 
\end{verbatim}
\normalsize 
Then you'll be able to mix {\footnotesize CONVERT} commands with the
familiar {\footnotesize DCL} ones. 

\subsection{Starting CONVERT from ICL}

To run {\footnotesize CONVERT} from {\footnotesize ICL} you must be
within the {\footnotesize ADAM} command language.  If you aren't
already, this requires just one extra command, namely 

\small
\begin{verbatim}
     $ ADAM
\end{verbatim}
\normalsize
You will see any messages produced by system and user procedures, followed
by the {\tt ICL>} prompt, something like the following.

\small
\begin{verbatim}
     Interactive Command Language   -   Version 1.5-6

       - Type HELP package_name for help on specific Starlink packages
       -   or HELP PACKAGES for a list of all Starlink packages
       - Type HELP [command] for help on ICL and its commands

\end{verbatim}
\normalsize
You will then get the {\tt ICL$>$} prompt.
Again there is a procedure for making the {\footnotesize CONVERT}
commands known to the command language.

\small
\begin{verbatim}
     ICL> CONVERT
\end{verbatim}
\normalsize
As there is no name clash, we can use the package name here, like other
packages, though {\tt CONVERTSTART} would still be recognised. 

\subsection{Starting CONVERT from the UNIX shell}

The command {\bf convert} defines {\footnotesize CONVERT} commands from
the UNIX shell.

\small
\begin{verbatim}
      % convert
\end{verbatim}
\normalsize 
Note that the {\tt \%} is the UNIX shell's prompt which you do not type.
Then you'll be able to mix {\footnotesize CONVERT} and UNIX commands.

\subsection{Issuing CONVERT Commands}

Having initialised {\footnotesize CONVERT} you are now ready to issue a
{\footnotesize CONVERT} command. To run an application you can just give
its name---you will be prompted for any required {\em parameters}. 
Alternatively, you may enter parameter values on the command line
specified by position or by keyword.  If you want to override any
defaulted parameters, then you specify the parameter's value on the
command line.  Note that from UNIX the commands are in lowercase, whereas
from VMS the commands are case-insensitive and are the same for both
{\footnotesize DCL} and {\footnotesize ICL}.

Most {\footnotesize CONVERT} applications can be run as simply as 

\small
\begin{verbatim}
     <application> <in> <out>
\end{verbatim}
\normalsize
where {\tt $<$application$>$} is the application's name, {\tt $<$in$>$}
is the input file, and {\tt $<$out$>$} is the output
file following the conversion.  For instance, 

\small
\begin{verbatim}
     $ DST2NDF OLD NEW
\end{verbatim}
\normalsize
or

\small
\begin{verbatim}
     ICL> DST2NDF OLD NEW
\end{verbatim}
\normalsize
both instruct the application DST2NDF to convert the DST file called
OLD.DST to the NDF called NEW.SDF.  From UNIX the command would be

\small
\begin{verbatim}
     % dst2ndf old new
\end{verbatim}
\normalsize
to convert old.dst to new.sdf.

Under VMS, commands may be abbreviated provided they are unambiguous
strings with at least four characters and including the {\tt 2}. 

The following example has the same effect as those immediately above,
only this time you are prompted for the filenames needed by DST2NDF. 

\small
\begin{verbatim}
     ICL> DST2NDF
     IN - Name of Figaro (.DST) file to be converted /' '/ > OLD
     OUT - Name of output NDF /@F1/ > NEW
\end{verbatim}
\normalsize
The value between the {\tt / /} delimiters is a suggested default.  You
can choose to accept the suggestion by pressing carriage return. 

The exceptions to the simple usage ({\tt <application> <in> <out>}) are
ASCII2NDF, where you give the shape of the dataset, either directly or
from a FITS-like header; UNF2NDF where the same applies, and you also
must specify the number of data values per record; and IRCAM2NDF where
you select which observations to convert and give the prefix of the
names of the output NDFs. See Section~4 of SUN/95 or Section~8 of SG/4
for details of how to use parameters for controlling these and other
options.  However, you should be able to get along using intuition
alone, or, perhaps by consulting the examples in Appendix~\ref{ap:full}.

Currently, one invocation of a {\footnotesize CONVERT} application is
required for each file conversion. 

You can find full documentation of each application, including its
usage, parameters, details of the conversion process and examples in
Appendix~\ref{ap:full}. 

\subsection{Obtaining Help}
You can see the top-level help for {\footnotesize CONVERT} by entering

\small
\begin{verbatim}
     ICL> HELP CONVERT
\end{verbatim}
\normalsize
or from {\footnotesize DCL}\footnote{The behaviour is different for DCL
and ICL because the DCL command {\tt HELP CONVERT} would display
information concerning DCL's CONVERT command, and not the CONVERT package.} 

\small
\begin{verbatim}
     $ CONHELP
\end{verbatim}
\normalsize
or from the UNIX shell\footnotet{here is no help facility akin to that
in VMS so you must invoke a special application: conhelp.} 

\vspace{-\bigskipamount}
\small
\begin{verbatim}
     % conhelp
\end{verbatim}
\normalsize
  From UNIX
On VMS, type {\tt ?} to get a list of
help topics.  On UNIX that will already be visible.\footnote{This
arises because UNIX conhelp uses the Starlink portable help system
whereas on VMS systems CONVERT uses VMS help.} 

The help topics are mostly detailed descriptions of the applications,
{\it e.g.}\ the following command gives help on the application UNF2NDF

\small
\begin{verbatim}
     $ HELP UNF2NDF
\end{verbatim}
\normalsize
under VMS and

\small
\begin{verbatim}
     % conhelp unf2ndf
\end{verbatim}
\normalsize
does likewise from the UNIX shell; but they also include global
information on matters such as using parameters. 

If you have commenced running an application you can still access the
help library whenver you are prompted for a parameter; you enter {\tt ?}.
Here is an example.

\small
\begin{verbatim}
     NOPEREC - Number of data values per output record /512/ > ?

     NDF2UNF

       Parameters

         NOPEREC = _INTEGER (Read)
            The number of data values per record of the output file.  It
            should be in the range 1 to 8191, unless the array is double
            precision, when the upper limit is 4095.  The suggested
            default is the current value. [The first dimension of the NDF]

     NOPEREC - Number of data values per output record /512/ > 
\end{verbatim}
\normalsize
 
\section{Acknowledgements}
Jo Murray wrote the original versions of the applications that convert
between DSTs or DIPSO files and NDFs.  Alan Chipperfield produced the
original STARIN and STAROUT upon which BDF2NDF and NDF2BDF are based.
Rhys Morris wrote the original versions of IRAF2NDF, NDF2IRAF, GASP2NDF
and NDF2GASP.

\newpage
\appendix

\section{Specifications of CONVERT applications}
\label{ap:full}
\subsection{Explanatory Notes}
The specification of parameters has the following format.

\begin{verbatim}
     name  =  type (access)
        description
\end{verbatim}
This format also includes a {\em Usage\/} entry.  This shows how the
application is invoked from the command line.   It lists the positional
parameters in order followed by any prompted keyword parameters using 
a {\mbox ``KEYWORD=?''} syntax.  Defaulted
keyword parameters do not appear.  Positional parameters
that are normally defaulted are indicated by being enclosed in square
brackets.   Keyword ({\it i.e.}\ not positional) parameters are needed 
where the number of parameters are large, and usually occur because
they depend on the value of another parameter.  An example should clarify.
\bigskip

{\ssttt \hspace*{1.0em}
        NDF2ASCII IN OUT [COMP] [RECLEN] NOPEREC=?
}
\bigskip

IN, OUT, COMP, and RECLEN are all positional
parameters.  Only IN, and OUT would be prompted if not given
on the command line. The remaining parameter, NOPEREC, depends on the
value of another parameter (it is FIXED), and will be prompted for when
FIXED is {\tt TRUE}. 

There is also an {\em Examples\/} section.  This shows how to run the
application from the command line.  More often you'll enter the
command name and just some of the parameters, and be prompted for
the rest.

The description entry has a notation scheme to indicate 
normally defaulted parameters, {\it i.e.}\ those for which there will
be no prompt.
For such parameters a matching pair of square brackets ({\tt []})
terminates the description.  The content between the brackets mean
\begin{description}
\item[{\tt []}]
Empty brackets means that the default is created dynamically
by the application, and may depend on the values of other parameters.
Therefore, the default cannot be given explicitly.
\item[{\tt [,]}]
As above, but there are two default values that are created dynamically.
\item[{\tt [}{\rm default}{\tt ]}]
Occasionally, a description of the default is given in normal type.
\item[{\tt [default]}]
If the brackets contain a value in teletype-fount, this is the explicit
default value.
\end{description}

\newpage
\small
\newpage

\newpage
\sstroutine{
   ASCII2NDF
}{
   Converts a text file to an NDF
}{
   \sstdescription{
      This application converts a text file to an NDF.  Only one of
      the array components may be created from the input file.
      Preceding the input data there may be an optional header.  This
      header may be skipped, or may consist of a simple FITS header.
      In the former case the shape of the NDF has be to be supplied.
   }
   \sstusage{
      ascii2ndf in out [comp] [skip] shape [type]
   }
   \sstparameters{
      \sstsubsection{
         COMP = LITERAL (Read)
      }{
         The NDF component to be copied.  It may be {\tt "Data"},
         {\tt "Quality"} or {\tt "Variance"}.  To create a variance or
         quality array the NDF must already exist. {\tt ["Data"]}
      }
      \sstsubsection{
         FITS = \_LOGICAL (Read)
      }{
         If true, the initial records of the formatted file are
         interpreted as a FITS header (with one card image per record)
         from which the shape, data type, and axis centres are derived.
         The last record of the FITS-like header must be terminated by
         an END keyword; subsequent records in the input file are
         treated as an array component given by COMP.  {\tt [FALSE]}
      }
      \sstsubsection{
         IN = FILENAME (Read)
      }{
         Name of the input text Fortran file.  The file will normally
         have variable-length records when there is a header, but
         always fixed-length records when there is no header.  On VMS
         platforms a default file extension of {\tt ".DAT"} is appended when
         parameter IN contains no file extension.
      }
      \sstsubsection{
         OUT = NDF (Read and Write)
      }{
         Output NDF data structure.  When COMP is not {\tt "Data"} the NDF
         is modified rather than a new NDF created.
         It becomes the new current NDF.
      }
      \sstsubsection{
         SHAPE = \_INTEGER (Read)
      }{
         The shape of the NDF to be created.  For example, {\tt [40,30,20]}
         would create 40 columns by 30 lines by 10 bands.  It is only
         accessed when FITS is {\tt FALSE}.
      }
      \sstsubsection{
         SKIP = INTEGER (Read)
      }{
         The number of header records to be skipped at the start of the
         input file before finding the data array or FITS-like header.
         {\tt [0]}
      }
      \sstsubsection{
         TYPE = LITERAL (Read)
      }{
         The data type of the output NDF.  It must be one of the
         following HDS types: {\tt "\_BYTE"}, {\tt "\_WORD"}, {\tt "\_REAL"},
         {\tt "\_INTEGER"}, {\tt "\_DOUBLE"}, {\tt "\_UBYTE"},
         {\tt "\_UWORD"} corresponding to signed byte,
         signed word, real, integer, double precision, unsigned byte,
         and unsigned word.  See SUN/92 for further details.  An
         unambiguous abbreviation may be given.  TYPE is ignored when
         COMP = {\tt "Quality"} since the QUALITY component must comprise
         unsigned bytes (equivalent to TYPE = {\tt "\_UBYTE"}) to be a valid
         NDF. The suggested default is the current value.  {\tt ["\_REAL"]}
      }
   }
   \sstexamples{
      \sstexamplesubsection{
         ascii2ndf ngc253.dat ngc253 shape=[100,60]
      }{
         This copies a data array from the text file ngc253.dat to the
         NDF called ngc253.  ngc253.dat does not contain a header
         section.  The NDF is two-dimensional: 100 elements in $x$ by 60
         in $y$.  Its data array has type \_REAL.
      }
      \sstexamplesubsection{
         ascii2ndf ngc253q.dat ngc253 q shape=[100,60]
      }{
         This copies a quality array from the text file ngc253q.dat to
         an existing NDF called ngc253 (such as created in the first
         example).  ngc253q.dat does not contain a header section.  The
         NDF is two-dimensional: 100 elements in $x$ by 60 in $y$.  Its
         data array has type \_UBYTE.
      }
      \sstexamplesubsection{
         ascii2ndf ngc253.dat ngc253 fits
      }{
         This copies a data array from the text file ngc253.dat
         to the NDF called ngc253.  ngc253.dat contains a FITS-like
         header section, which is copied to the FITS extension of the
         NDF.  The shape of the NDF is controlled by the mandatory FITS
         keywords NAXIS, AXIS1, \dots, AXIS$n$, and the data type by
         keywords BITPIX and UNSIGNED.
      }
      \sstexamplesubsection{
         ascii2ndf type="\_uword" in=ngc253.dat out=ngc253 $\backslash$
      }{
         This copies a data array from the text file ngc253.dat to the
         NDF called ngc253.  ngc253.dat does not contain a header
         section.  The NDF has the current shape and data type is
         unsigned word.
      }
      \sstexamplesubsection{
         ascii2ndf spectrum ZZ skip=2 shape=200
      }{
         This copies a data array from the text file spectrum to
         the NDF called ZZ.  spectrum contains two header records
         that are ignored.  The NDF is one-dimensional comprising 200
         elements of type \_REAL.
      }
      \sstexamplesubsection{
         ascii2ndf spectrum.lis ZZ skip=1 fits
      }{
         This copies a data array from the text file spectrum.lis to
         the NDF called ZZ.  spectrum.lis contains one header record,
         that is ignored, followed by a FITS-like header section, which
         is copied to the FITS extension of the NDF.  The shape of the
         NDF is controlled by the mandatory FITS keywords NAXIS, AXIS1,
         \dots, AXIS$n$, and the data type by keywords BITPIX and UNSIGNED.
      }
   }
   \sstnotes{
      The details of the conversion are as follows:
      \sstitemlist{

         \sstitem
            the ASCII-file array is written to the NDF array as
            selected by COMP.  When the NDF is being modified, the shape
            of the new component must match that of the NDF.

         \sstitem
            If the input file contains a FITS-like header, and a new
            NDF is created, {\it i.e.}\ COMP = {\tt "Data"}, the header records are
            placed within the NDF's FITS extension.  This enables more
            than one array (input file) to be used to form an NDF.  Note
            that the data array must be created first to make a valid NDF,
            and it's the FITS structure associated with that array that is
            wanted.  Indeed the application prevents you from doing
            otherwise.

         \sstitem
            The FITS-like header defines the properties of the NDF as
            follows:
            \begin{itemize}
            \item BITPIX defines the data type: 8 gives \_BYTE, 16 produces
            \_WORD, 32 makes \_INTEGER, $-$32 gives \_REAL, and $-$64 generates
            \_DOUBLE.  For the first two, if there is an extra header
            record with the keyword UNSIGNED and logical value T, these
            types become \_UBYTE and \_UWORD respectively.  UNSIGNED is
            non-standard, since unsigned integers would not follow in a
            proper FITS file.  However, here it is useful to enable
            unsigned types to be input into an NDF.  UNSIGNED may be
            created by this application's sister, NDF2ASCII.  BITPIX is
            ignored for QUALITY data; type \_UBYTE is used.
            \item NAXIS, and NAXIS$n$ define the shape of the NDF.
            \item The TITLE, LABEL, and BUNITS are copied to the NDF
            TITLE, LABEL, and UNITS NDF components respectively.
            \item The CDELT$n$, CRVAL$n$, CTYPE$n$, and CRTYPE$n$ keywords make
            linear axis structures within the NDF.  CTYPE$n$ define the
            axis units, and the axis labels are assigned to CRTYPE$n$ If
            some are missing, pixel co-ordinates are used for those
            axes.
            \item BSCALE and BZERO in a FITS extension are ignored.
            \item BLANK is not used to indicate which input array values
            should be assigned to a standard bad value.
            \item END indicates the last header record unless it
            terminates a dummy header, and the actual data is in an
            extension.
            \end{itemize}

         \sstitem
            Other data item such as HISTORY, data ORIGIN, and axis
            widths are not supported, because the text file has a simple
            structure to enable a diverse set of input files to be
            converted to NDFs, and to limitations of the standard FITS
            header.
      }
   }
   \sstdiytopic{
      Related Applications
   }{
      CONVERT: NDF2ASCII; KAPPA: TRANDAT; SPECDRE: ASCIN and ASCOUT.
   }
}

\newpage
\sstroutine{
   BDF2NDF
}{
   Converts a Starlink Interim BDF file to an NDF
}{
   \sstdescription{
      This application converts data files from the Bulk Data Frame
      (BDF) format used by the INTERIM Environment (see SUN/4) to the
      Starlink standard NDF format (see SUN/33 or SGP/38).  Type
      conversion may be performed at the same time.
   }
   \sstusage{
      BDF2NDF IN OUT [TYPE] [DESCRIP]
   }
   \sstparameters{
      \sstsubsection{
         IN = BDF (Read)
      }{
         The BDF to be converted to an NDF.  A file extension must not
         be given since {\tt ".BDF"} is assumed.
      }
      \sstsubsection{
         OUT = NDF (Write)
      }{
         The name of NDF converted from the BDF.  It becomes the new
         current NDF.
      }
      \sstsubsection{
         TYPE = LITERAL (Read)
      }{
         The data type of the output NDF.  It must be one of the
         following HDS types: {\tt "\_BYTE"}, {\tt "\_WORD"},
         {\tt "\_REAL"}, {\tt "\_INTEGER"}, {\tt "\_DOUBLE"},
         {\tt "\_UBYTE"}, {\tt "\_UWORD"} corresponding to signed byte,
         signed word, real, integer, double precision, unsigned byte,
         and unsigned word.  See SUN/92 for further details.  An
         unambiguous abbreviation may be given.  {\tt ["\_REAL"]}
      }
      \sstsubsection{
         DESCRIP = \_LOGICAL (Read)
      }{
         If true the descriptors in the BDF are reported as they are
         copied to the FITS extension within the output NDF. {\tt [FALSE]}
      }
      \sstsubsection{
         CONNECT = FILENAME (Write)
      }{
         The Interim connection file.  It is deleted when the
         application terminates.  {\tt [NDF2BDF.TMP]}
      }
      \sstsubsection{
         COMMAND = FILENAME (Write)
      }{
         The Interim command file.  It is deleted when the application
         terminates.  {\tt [USERCOM.TMP]}
      }
   }
   \sstexamples{
      \sstexamplesubsection{
         BDF2NDF OLD NEW
      }{
         This converts the BDF called OLD (in file OLD.BDF) to the NDF
         called NEW (in file NEW.SDF).  NEW's data array will have
         the \_REAL data type.  Descriptors are copied to the FITS
         extension but are not reported.
      }
      \sstexamplesubsection{
         BDF2NDF OLD NEW DESCRIP
      }{
         This converts the BDF called OLD (in file OLD.BDF) to the NDF
         called NEW (in file NEW.SDF).  NEW's data array will have
         the \_REAL data type.  Descriptors are copied to the FITS
         extension and are reported.
      }
      \sstexamplesubsection{
         BDF2NDF HORSE HORSE TYPE="\_WORD"
      }{
         This converts the BDF called HORSE (in file HORSE.BDF) to the
         NDF also called HORSE (in file HORSE.SDF).  The NDF's data
         array will contain signed 2-byte integers.  Descriptors are
         copied to the FITS extension but are not reported.
      }
   }
   \sstnotes{
      \sstitemlist{

         \sstitem
         The conversion rules can be summarised as follow:
         \begin{itemize}
            \item the BDF data array is copied to the NDF main data array;
            by default the output data array has the \_REAL data type.
            \item The BLANK descriptor is not used to flag pixels in the NDF
            with the bad value.  Use KAPPA's SETMAGIC to flag another
            value.
            \item The BDF descriptors are written to the NDF FITS extension.
            Long values may be truncated, 65 for characters and 20 for
            numbers.  The formating adheres to the FITS standard.
            Descriptors already in the FITS format are copied as is, so
            La Palma ING-format headers can be propagated.
            \item If the BDF descriptors contain the FITS keywords CRVAL$n$,
            CDELT$n$, the appropriate axis structures are generated in
            the output NDF. In addition should CRTYPE$n$ also be present
            the labels are added to these structures.
            \item If the BDF descriptors contain the FITS keywords TITLE or
            LABEL or BUNITS, the associated character string value is
            written to the NDF TITLE, LABEL or UNITS component as
            appropriate.
            \item HISTORY descriptors are not used to make an HISTORY
            component in the NDF.
         \end{itemize}

         \sstitem
         WARNING: the BDF size may grow when TYPE is specified due to
         an incarnation being created.  See below and SUN/4 for more
         details.
      }
   }
   \sstdiytopic{
      Incarnations
   }{
      A BDF contains one or more `incarnations' of the data array.  An
      incarnation of a data array is simply a copy of that data array
      stored with a particular data type.  For example, a BDF may
      contain incarnations of the same data array stored as a REAL
      array, and an INTEGER array.  This rather strange behaviour is a
      consequence of the way the INTERIM environment deals with data
      access.  For example, if an application attempts to map a BDF data
      array of INTEGER type as a REAL array, type conversion must take
      place.  Instead of doing this in virtual memory, a second
      incarnation of the data, this time of type REAL is created and
      stored in the original file (causing perhaps a doubling of the
      file size).  This second incarnation of the data array is stored
      to avoid performing the same type conversion in the future.  (It
      is perhaps significant that no such scheme has been employed with
      subsequent data systems.)  One consequence of this behaviour is
      that a BDF data array may not have a unique type.  BDF2NDF will
      use the first type it finds as the default.  Specify the
      TYPE parameter explicitly to ensure that the correct data type is
      created within the NDF.

      To prevent the BDF growing, just remove write access from the
      file.
   }
   \sstdiytopic{
      Related Applications
   }{
      CONVERT: NDF2BDF.
   }
   \sstimplementationstatus{
      \sstitemlist{

         \sstitem
         Note the data array in the NDF is of the primitive form.

         \sstitem
         Only available on VMS platforms.
      }
   }
   \sstbugs{
      \sstitemlist{

         \sstitem
         May give spurious error messages if running under ICL.  This
         happens if a non-existent BDF is given as the input file.  A
         subsequent invocation of the application may result in a repeat of
         the error message although the conversion is carried out
         correctly.
      }
   }
}

\newpage
\sstroutine{
   DIPSO2NDF
}{
   Converts a DIPSO-format file to an NDF
}{
   \sstdescription{
      This application creates an NDF from a DIPSO-format file as
      produced by the DIPSO `WRITE' command.  See SUN/50.  The rules
      for the conversion are listed in the Notes.
   }
   \sstusage{
      DIPSO2NDF IN OUT
   }
   \sstparameters{
      \sstsubsection{
         IN = FILENAME (Read)
      }{
         Input DIPSO file.  A default file extension of {\tt ".DAT"} is
         appended when parameter IN contains no file extension.
      }
      \sstsubsection{
         OUT = NDF (Write)
      }{
         Output NDF data structure.  A file extension must not be given
         after the name.  It becomes the new current NDF.
      }
   }
   \sstexamples{
      \sstexamplesubsection{
         DIPSO2NDF OLD NEW
      }{
         This converts the DIPSO file called OLD to the NDF called NEW
         (in file NEW.SDF).
      }
      \sstexamplesubsection{
         DIPSO2NDF SPECTRE.DAT SPECTRE
      }{
         This converts the DIPSO file SPECTRE.DAT to the NDF called
         SPECTRE (in file SPECTRE.SDF).
      }
   }
   \sstnotes{
      \sstitemlist{

         \sstitem
         The DIPSO title is written to the NDF TITLE.

         \sstitem
         The DIPSO main data array (often call the flux array) is
         copied to the NDF's data array.  DIPSO records bad values by
         means of breaks in the data array.  The number and positions of
         these breaks are stored in the DIPSO file.  This application
         inserts bad pixels at these break positions.  The number of bad
         pixels inserted is based on the size of the gap in the wavelength
         scale.  At least one bad pixel is inserted at every break point.

         \sstitem
         The $x$-axis array (otherwise known as the wavelength array) is
         copied to the NDF AXIS(1) centres.  Missing axis centres are
         generated by linear interpolation between the good $x$-axis values
         in the DIPSO file.

         \sstitem
         The data and the axis centres need not be fluxes and
         wavelengths respectively.
      }
   }
   \sstdiytopic{
      Related Applications
   }{
      CONVERT: NDF2DIPSO; DIPSO: commands OREAD and OWRITE.
   }
   \sstimplementationstatus{
      \sstitemlist{

         \sstitem
         The output NDF has a primitive data array.

         \sstitem
         The input wavelength and flux data are always of Fortran REAL
         type, the output data arrays are of HDS type \_REAL.

         \sstitem
         The application assumes that the bad-pixel padding will not
         cause the number of elements in the data array to exceed twice
         the original number.

         \sstitem
         Only available on VMS platforms.
      }
   }
}

\newpage
\sstroutine{
   DST2NDF
}{
   Converts a Figaro (Version 2) DST file to an NDF
}{
   \sstdescription{
      This application converts a Figaro Version-2 DST file to a
      Version-3 file, {\it i.e.}\ to an NDF.  The rules for converting the
      various components of a DST are listed in the notes.  Since
      both are hierarchical formats most files can be be converted with
      little or no information lost.
   }
   \sstusage{
      dst2ndf in out
   }
   \sstparameters{
      \sstsubsection{
         FORM = LITERAL (Read)
      }{
         The storage form of the NDF's data and variance arrays.
         FORM = {\tt "Simple"} gives the simple form, where the array of data
         and variance values is located in an ARRAY structure.  Here it
         can have ancillary data like the origin.  This is the normal
         form for an NDF.  FORM = {\tt "Primitive"} offers compatibility with
         earlier formats, such as IMAGE.  In the primitive form the
         data and variance arrays are primitive components at the top
         level of the NDF structure, and hence it cannot have
         ancillary information. {\tt ["Simple"]}
      }
      \sstsubsection{
         IN = Figaro file (Read)
      }{
         The file name of the version 2 file.  A file extension must
         not be given after the name, since {\tt ".dst"} is appended by the
         application.  The file name is limited to 80 characters.
      }
      \sstsubsection{
         OUT = NDF (Write)
      }{
         The file name of the output NDF file.  A file extension must
         not be given after the name, since {\tt ".sdf"} is appended by the
         application.  Since the NDF\_ library is not used, a section
         definition may not be given following the name.  The file
         name is limited to 80 characters.
      }
   }
   \sstexamples{
      \sstexamplesubsection{
         dst2ndf old new
      }{
         This converts the Figaro file old.dst to the NDF called new
         (in file new.sdf).  The NDF has the simple form.
      }
      \sstexamplesubsection{
         dst2ndf horse horse p
      }{
         This converts the Figaro file horse.dst to the NDF called
         horse (in file horse.sdf).  The NDF has the primitive form.
      }
   }
   \sstnotes{
      The rules for the conversion of the various components are as
      follows:
      \vspace{-\parskip}
      \begin{center}
      \begin{tabular}{|lcl|p{47mm}|}
      \hline 
      \multicolumn{1}{|c}{Figaro file} & & \multicolumn{1}{c}{NDF} &
      \multicolumn{1}{|c|}{Comments} \\ \hline
      .Z.DATA   & $\Rightarrow$ & .DATA\_ARRAY.DATA & when FORM = {\tt "SIMPLE"}\\
      .Z.DATA   & $\Rightarrow$ & .DATA\_ARRAY & when FORM = {\tt "PRIMITIVE"} \\
      .Z.ERRORS & $\Rightarrow$ & .VARIANCE.DATA & after processing when FORM = {\tt "SIMPLE"} \\
      .Z.ERRORS & $\Rightarrow$ & .VARIANCE & after processing when FORM = {\tt "PRIMITIVE"} \\
      .Z.QUALITY & $\Rightarrow$ & .QUALITY.QUALITY & must be BYTE array
                                  (see Bad-pixel handling below) \\
      & $\Rightarrow$ & .QUALITY.BADBITS = 255 & \\
      .Z.LABEL  & $\Rightarrow$ & .LABEL & \\
      .Z.UNITS  & $\Rightarrow$ & .UNITS & \\
      .Z.IMAGINARY & $\Rightarrow$ & .DATA\_ARRAY.IMAGINARY\_DATA & \\
      .Z.MAGFLAG & $\Rightarrow$ & .MORE.FIGARO.MAGFLAG & \\
      .Z.RANGE  & $\Rightarrow$ & .MORE.FIGARO.RANGE & \\
      .Z.xxxx   & $\Rightarrow$ & .MORE.FIGARO.Z.xxxx & \\ \hline
      \end{tabular}
      \end{center}

      \begin{center}
      \begin{tabular}{|lcl|p{43mm}|}
      \hline 
      \multicolumn{1}{|c}{Figaro file} & & \multicolumn{1}{c}{NDF} &
      \multicolumn{1}{|c|}{Comments} \\ \hline
      .X.DATA   & $\Rightarrow$ & .AXIS(1).DATA\_ARRAY & \\ 
      .X.ERRORS & $\Rightarrow$ & .AXIS(1).VARIANCE & after processing \\
      .X.WIDTH  & $\Rightarrow$ & .AXIS(1).WIDTH & \\
      .X.LABEL  & $\Rightarrow$ & .AXIS(1).LABEL & \\
      .X.UNITS  & $\Rightarrow$ & .AXIS(1).UNITS & \\
      .X.LOG    & $\Rightarrow$ & .AXIS(1).MORE.FIGARO.LOG & \\
      .X.xxxx   & $\Rightarrow$ & .AXIS(1).MORE.FIGARO.xxxx & \\
      & & & (Similarly for .Y .T .U .V or .W structures which are
             renamed to AXIS(2), \ldots, AXIS(6) in the NDF.) \\
      & & & \\
      .OBS.OBJECT & $\Rightarrow$ & .TITLE & \\
      .OBS.SECZ & $\Rightarrow$ & .MORE.FIGARO.SECZ & \\
      .OBS.TIME & $\Rightarrow$ & .MORE.FIGARO.TIME & \\
      .OBS.xxxx & $\Rightarrow$ & .MORE.FIGARO.OBS.xxxx & \\
      & & & \\
      .FITS.xxxx& $\Rightarrow$ & .MORE.FITS.xxxx & into value part of
         the string \\
      .COMMENTS.xxxx  & $\Rightarrow$ & .MORE.FITS.xxxx & into comment part of
         the string \\
      .FITS.xxxx.DATA & $\Rightarrow$ & .MORE.FITS.xxxx & into value part of
         the string \\
      .FITS.xxxx.DESCRIPTION & $\Rightarrow$ & .MORE.FITS.xxxx & into comment
         part of the string \\
      & & & \\
      .MORE.xxxx& $\Rightarrow$ & .MORE.xxxx & \\
      & & & \\
      .TABLE    & $\Rightarrow$ & .MORE.FIGARO.TABLE & \\
      .xxxx     & $\Rightarrow$ & .MORE.FIGARO.xxxx & \\ \hline
      \end{tabular}
      \end{center}

      Axis arrays with dimensionality greater than one are not
      supported by the NDF.  Therefore, if the application encounters
      such an axis array, it processes the array using the following
      rules, rather than those given above.
      \begin{center}
      \begin{tabular}{|lcl|p{51mm}|}
      \hline 
      \multicolumn{1}{|c}{Figaro file} & & \multicolumn{1}{c}{NDF} &
      \multicolumn{1}{|c|}{Comments} \\ \hline
      .X.DATA   & $\Rightarrow$ & .AXIS(1).MORE.FIGARO.DATA &
            AXIS(1).DATA\_ARRAY is filled with pixel co-ordinates \\
      .X.ERRORS & $\Rightarrow$ & .AXIS(1).MORE.FIGARO.VARIANCE & after
            processing \\
      .X.WIDTH  & $\Rightarrow$ & .AXIS(1).MORE.FIGARO.WIDTH & \\ \hline
      \end{tabular}
      \end{center}
   }
   \sstdiytopic{
      Related Applications
   }{
      CONVERT: NDF2DST.
   }
   \sstdiytopic{
   Bad-pixel handling
   }{
   The QUALITY array is only copied if the bad-pixel flag
   (.Z.FLAGGED) is false or absent.  A simple NDF with the bad-pixel
   flag set to false (meaning that there are no bad-pixels present)
   is created when .Z.FLAGGED is absent or false and FORM = {\tt "SIMPLE"}.
   }
   \sstimplementationstatus{
      The maximum number of dimensions is 6.
   }
}

\newpage
\sstroutine{
   GASP2NDF
}{
   Converts an image in GASP format to an NDF
}{
   \sstdescription{
      This application converts a GAlaxy Surface Photometry (GASP)
      format file into an NDF.
   }
   \sstusage{
      gasp2ndf in out shape=?
   }
   \sstparameters{
      \sstsubsection{
         IN = LITERAL (Read)
      }{
         A character string containing the name of GASP file to convert.
         The extension should not be given, since {\tt "}.dat{\tt "} is assumed.
      }
      \sstsubsection{
         OUT = NDF (Write)
      }{
         The name of the output NDF.
      }
      \sstsubsection{
         SHAPE( 2 ) = \_INTEGER (Read)
      }{
         The dimensions of the GASP image (the number of columns
         followed by the number of rows).  Each dimension must be in the
         range 1 to 1024.  This parameter is only used if supplied on
         the command line, or if the header file corresponding to the
         GASP image does not exist or cannot be opened.
      }
   }
   \sstexamples{
      \sstexamplesubsection{
         gasp2ndf m31\_gasp m31
      }{
         Convert a GASP file called m31\_gasp.dat into an NDF called m31.
         The dimensions of the image are taken from the header file
         m31\_gasp.hdr.
      }
      \sstexamplesubsection{
         gasp2ndf n1068 ngc1068 shape=[256,512]
      }{
         Take the pixel values in the GASP file n1068.dat and create
         the NDF ngc1068 with dimensions 256 columns by 512 rows.
      }
   }
   \sstnotes{
      \sstitemlist{

         \sstitem
         A GASP image is limited to a maximum of 1024 by 1024 elements.
         It must be two dimensional.

         \sstitem
         The GASP image is written to the NDF's data array.  The data
         array has type \_WORD. No other NDF components are created.

         \sstitem
         If the header file is corrupted, the user must remove the
         offending {\tt "}.hdr{\tt "} file or specify the shape of the GASP image on the
         command line, otherwise the application will continually abort.
      }
   }
   \sstdiytopic{
      Related Applications
   }{
      CONVERT: NDF2GASP.
   }
   \sstdiytopic{
      References
   }{
      GASP documentation (MUD/66).
   }
}

\newpage
\sstroutine{
   IRAF2NDF
}{
   Converts an IRAF image to an NDF
}{
   \sstdescription{
      This application converts an IRAF image to an NDF.
      See the Notes for details of the conversion.
   }
   \sstusage{
      iraf2ndf in out
   }
   \sstparameters{
      \sstsubsection{
         IN = LITERAL (Read)
      }{
         The name of the IRAF image.  Note that this excludes the extension.
      }
      \sstsubsection{
         OUT = NDF (Write).
      }{
         The name of the NDF to be produced.
      }
   }
   \sstexamples{
      \sstexamplesubsection{
         iraf2ndf ell\_galaxy new\_galaxy
      }{
         Converts the IRAF image ell\_galaxy (comprising files
         ell\_galaxy.imh and ell\_galaxy.hdr) to an NDF called new\_galaxy.
      }
   }
   \sstnotes{
      The rules for the conversion are as follows:
      \sstitemlist{

         \sstitem
         The NDF data array is copied from the {\tt "}.pix{\tt "} file.

         \sstitem
         The title of the IRAF image (object i\_title in the {\tt "}.imh{\tt "}
         header file) becomes the NDF title.

         \sstitem
         Lines from the IRAF image header file are transferred to the
         FITS extension of the NDF, any compulsory FITS keywords that are
         missing are added.

         \sstitem
         If there is a FITS extension in the NDF, then the elements up
         to the first END keyword of this are added to the `user area' of
         the IRAF header file.

         \sstitem
         Lines from the HISTORY section of the IRAF image are also
         extracted and added to the NDF's FITS extension as FITS HISTORY
         lines.  Two extra HISTORY lines are added to record the original
         name of the image and the date of the format conversion.
      }
   }
   \sstdiytopic{
      Related Applications
   }{
      CONVERT: NDF2IRAF.
   }
   \sstimplementationstatus{
      \sstitemlist{

         \sstitem
         Only handles one-, two-, and three-dimensional IRAF files.

         \sstitem
         The NDF produced has type \_WORD or \_REAL corresponding to the
         type of the IRAF image.  (The IRAF imfort FORTRAN subroutine
         library only supports these data types: signed words and real.)
         The pixel type of the image can be changed from within IRAF using
         the {\bf chpixtype} task in the {\bf images} package.
      }
   }
}

\newpage
\sstroutine{
   IRCAM2NDF
}{
   Converts an IRCAM data file to a series of NDFs
}{
   \sstdescription{
      This applications converts an HDS file in the IRCAM format listed
      in IRCAM User Note 11 to one or more NDFs.  See the Notes for a
      detailed list of the rules of the conversion.
   }
   \sstusage{
      ircam2ndf in prefix obs [config]
   }
   \sstparameters{
      \sstsubsection{
         CONFIG = LITERAL (Read)
      }{
         The choice of data array to place in the NDF.  It can have one
         of the following configuration values:
            \begin{itemize}
            \item {\tt "STARE"} --- the image of the object or sky;
            \item {\tt "CHOP"} --- the chopped image of the sky;
            \item {\tt "KTCSTARE"} --- the electronic pedestal or bias associated
                           with the stare image of the object or sky;
            \item{\tt "KTCCHOP"} --- the electronic pedestal or bias associated
                           with the chopped image of the sky.
            \end{itemize}
         Note that at the time of writing chopping has not been
         implemented for IRCAM.  For practical purposes CONFIG={\tt "STARE"}
         should be used.  The suggested default is the current value.
         {\tt ["STARE"]}
      }
      \sstsubsection{
         FMTCNV = \_LOGICAL (Read)
      }{
         This specifies whether or not format conversion may occur.
         If FMTCNV is false, the data type of the data array in the NDF
         will be the same as that in the IRCAM file, and there is no
         scale factor and offset applied.  If FMTCNV is true, whenever
         the IRCAM observation has non-null scale and offset values,
         the observation data array will be converted to type \_REAL in
         the NDF, and the scale and offset applied to the input data
         values to give the `true' data values.  A null scale factor is
         1 and a null offset is 0. {\tt [FALSE]}
      }
      \sstsubsection{
         IN = IRCAM (Read)
      }{
         The name of the input IRCAM file to convert to NDFs.  The
         suggested value is the current value.
      }
      \sstsubsection{
         OBS()  = LITERAL (Read)
      }{
         A list of the observation numbers to be converted into NDFs.
         Observations are numbered consecutively from 1 up to the
         actual number of observations in the IRCAM file.  Single
         observations or a set of adjacent observations may be
         specified, {\it e.g.}\ entering {\tt [4,6-9,12,14-16]} will read
         observations 4,6,7,8,9,12,14,15,16.  (Note that the brackets
         are required to distinguish this array of characters from a
         single string including commas.  The brackets are unnecessary
         when there only one item.)

         If you wish to extract all the observations enter the wildcard
         {\tt $*$}.  {\tt 5-$*$} will read from 5 to the last observation.  The
         processing will continue until the last observation is
         converted.  The suggested value is the current value.
      }
      \sstsubsection{
         PREFIX = LITERAL (Read)
      }{
         The prefix of the output NDFs.  The name of an NDF is the
         prefix followed by the observation number.  The suggested
         value is the current value.
      }
   }
   \sstexamples{
      \sstexamplesubsection{
         ircam2ndf ircam\_27aug89\_1cl rhooph obs=$*$
      }{
         This converts the IRCAM data file called ircam\_27aug89\_1cl into
         a series of NDFs called rhooph1, rhooph2 {\it etc.}\  There is no
         format conversion applied.
      }
      \sstexamplesubsection{
         ircam2ndf ircam\_27aug89\_1cl rhooph [32,34-36] fmtcnv
      }{
         This converts four observations in the IRCAM data file called
         ircam\_27aug89\_1cl into NDFs called rhooph32, rhooph34,
         rhooph35, rhooph36.  The scale and offset are applied.
      }
      \sstexamplesubsection{
         ircam2ndf in=ircam\_04nov90\_1c config="KTC" obs=5 prefix=bias
      }{
         This converts the fifth observation in the IRCAM data file
         called ircam\_04nov90\_1c into an NDF called bias5 containing
         the electronic pedestal in its data array.  There is no format
         conversion applied.
      }
   }
   \sstnotes{
      The rules for the conversion of the various components are as
      follows:
      \begin{center}
      \begin{tabular}{|lcl|p{35mm}|}
      \hline 
      \multicolumn{1}{|l}{IRCAM file} & & \multicolumn{1}{l}{NDF} &
      \multicolumn{1}{|c|}{Comments} \\ \hline
      .OBS.PHASEA.DATA\_ARRAY & $\Rightarrow$ &  .DATA\_ARRAY & 
          when parameter CONFIG={\tt "STARE"} \\
      .OBS.PHASEB.DATA\_ARRAY & $\Rightarrow$ &  .DATA\_ARRAY &
          when parameter CONFIG={\tt "CHOP"} \\
      .OBS.KTCA.DATA\_ARRAY   & $\Rightarrow$ &  .DATA\_ARRAY &
          when parameter CONFIG={\tt "KTCSTARE"} \\
      .OBS.KTCB.DATA\_ARRAY   & $\Rightarrow$ &  .DATA\_ARRAY &
          when parameter CONFIG={\tt "KTCCHOP"} \\
      & & & \\
      .OBS.DATA\_LABEL        & $\Rightarrow$ &  .LABEL & \\
      .OBS.DATA\_UNITS        & $\Rightarrow$ &  .UNITS & \\
      .OBS.TITLE              & $\Rightarrow$ &  .TITLE &
          If .OBS.TITLE is a blank string, OBS.DATA\_OBJECT is copied
          to the NDF title instead. \\
      & & & \\
      .OBS.AXIS1\_LABEL       & $\Rightarrow$ &  .AXIS(1).LABEL & \\
      .OBS.AXIS2\_LABEL       & $\Rightarrow$ &  .AXIS(2).LABEL & \\
      .OBS.AXIS1\_UNITS       & $\Rightarrow$ &  .AXIS(1).UNITS & \\
      .OBS.AXIS2\_UNITS       & $\Rightarrow$ &  .AXIS(2).UNITS & \\
      \multicolumn{3}{|p{111mm}|}{
      .GENERAL.INSTRUMENT.PLATE\_SCALE 
          becomes the increment between the axis centres, with co-ordinate
          (0.0,0.0) located at the image centre.  The NDF axis units both
          become {\tt "}arcseconds{\tt "}. } & \\
      & & & \\
      .GENERAL               & $\Rightarrow$ &  .MORE.IRCAM.GENERAL & \\
      .GENERAL.x             & $\Rightarrow$ &  .MORE.IRCAM.GENERAL.x & \\
      .GENERAL.x.y           & $\Rightarrow$ &  .MORE.IRCAM.GENERAL.x.y & \\
      & & & \\
      .OBS.x                 & $\Rightarrow$ &  .MORE.IRCAM.OBS.x &
          This excludes the components of OBS already listed above and
          DATA\_BLANK. \\ \hline
      \end{tabular}
      \end{center}

      \sstitemlist{

         \sstitem
         The data types of the IRCAM GENERAL structures have not been
         propagated to the NDF IRCAM extensions, because it would violate
         the rules of SGP/38.  In the IRCAM file these all have the same
         type STRUCTURE.  The new data types are as follows:

      \begin{center}
      \begin{tabular}{|l|l|}
      \hline 
      \multicolumn{1}{|c|}{Extension Name} & \multicolumn{1}{c|}{Data Type} \\ \hline
      IRCAM.GENERAL & IRCAM\_GENERAL \\
      IRCAM.GENERAL.INSTRUMENT & IRCAM\_INSTRUM \\
      IRCAM.GENERAL.ID & IRCAM\_ID \\
      IRCAM.GENERAL.TELESCOPE & IRCAM\_TELESCOPE \\ \hline
      \end{tabular}
      \end{center}

         \sstitem
         Upon completion the number of observations
         successfully converted to NDFs is reported.
      }
   }
   \sstdiytopic{
      Bad-pixel Handling
   }{
      Elements of the data array equal to the IRCAM component
      .OBS.DATA\_BLANK are replaced by the standard bad value.
   }
   \sstimplementationstatus{
      \sstitemlist{

         \sstitem
         The data array in the NDF is in the primitive form.

         \sstitem
         The application aborts if the data array chosen by parameter
         CONFIG does not exist in the observation.
      }
   }
}

\newpage
\sstroutine{
   NDF2ASCII
}{
   Converts an NDF to a text file
}{
   \sstdescription{
      This application converts an NDF to a text file.  Only one of
      the array components may be copied to the output file.  Preceding
      the data there is an optional header consisting of either the
      FITS extension with the values of certain keywords replaced by
      information derived from the NDF, or a minimal FITS header also
      derived from the NDF.  The output file uses LIST carriagecontrol.
   }
   \sstusage{
      ndf2ascii in out [comp] [reclen] noperec=?
   }
   \sstparameters{
      \sstsubsection{
         COMP = LITERAL (Read)
      }{
         The NDF component to be copied.  It may be {\tt "Data"},
         {\tt "Quality"} or {\tt "Variance"}. {\tt ["Data"]}
      }
      \sstsubsection{
         FITS = \_LOGICAL (Read)
      }{
         If true, any FITS extension is written to start of the output
         file, unless there is no extension whereupon a minimal FITS
         header is written to the ASCII file. {\tt [FALSE]}
      }
      \sstsubsection{
         FIXED = \_LOGICAL (Read)
      }{
         If true, the output file allocates a fixed number of
         characters per data value.  The number of characters chosen is
         the minimum that prevents any loss of precision, and hence is
         dependent on the data type of the NDF array.  When FIXED is
         {\tt FALSE}, data values are packed as efficiently as possible within
         each record. {\tt [FALSE]}
      }
      \sstsubsection{
         IN = NDF (Read)
      }{
         Input NDF data structure. The suggested default is the current
         NDF if one exists, otherwise it is the current value.
      }
      \sstsubsection{
         NOPEREC = \_INTEGER (Read)
      }{
         The number of data values per record of the output file, when
         FIXED is {\tt TRUE}.  It should be in the range 1 to 16383.
         The suggested default is the current value, or 8 when there
         is not one.
      }
      \sstsubsection{
         OUT = FILENAME (Write)
      }{
         Name of the output formatted Fortran file.  The file will
         normally have variable-length records when there is a header,
         but always fixed-length records when there is no header.   On
         VMS platforms a default file extension of {\tt ".DAT"} is appended
         when parameter OUT contains no file extension.
      }
      \sstsubsection{
         RECLEN = \_INTEGER (Read)
      }{
         The maximum record length in bytes of the output file.  This
         must be between 32 and 32766.  The lower limit is further
         increased to 80 when there is a FITS header to be copied.  It
         is only used when FIXED is {\tt FALSE} and will default to the
         current value, or 512 if there is no current value.  When
         FIXED is {\tt TRUE} the application creates data records whose
         length is the product of the number of bytes per value plus
         one (for the space), times the number of values per record.
         {\tt []}
      }
   }
   \sstexamples{
      \sstexamplesubsection{
         ndf2ascii cluster cluster.dat
      }{
         This copies the data array of the NDF called cluster to a text
         file called cluster.dat.  The maximum recordlength of
         cluster.dat is 512 bytes, and the data values are packed into
         these records as efficiently as possible.
      }
      \sstexamplesubsection{
         ndf2ascii cluster cluster.dat v
      }{
         This copies the variance of the NDF called cluster to a
         text file called cluster.dat.  The maximum recordlength of
         cluster.dat is 512 bytes, and the variance values are packed
         into these records as efficiently as possible.
      }
      \sstexamplesubsection{
         ndf2ascii cluster cluster.dat fixed noperec=12
      }{
         This copies the data array of the NDF called cluster to a
         text file called cluster.dat.  There are twelve data values
         per record in cluster.dat.
      }
      \sstexamplesubsection{
         ndf2ascii out=ndf234.dat fits reclen=128 in=@234
      }{
         This copies the data array of the NDF called 234 to a
         text file called ndf234.dat.  The maximum recordlength of
         ndf234.dat is 128 bytes, and the data values are packed into
         these records as efficiently as possible.  If there is a FITS
         extension, it is copied to ndf234.dat with substitution of
         certain keywords, otherwise a minimal FITS header is produced.
      }
   }
   \sstnotes{
      The details of the conversion are as follows:
      \sstitemlist{

         \sstitem
            the NDF array as selected by COMP is written to the ASCII
            file in records following an optional header.  When FIXED is
            {\tt FALSE} all records are padded out to the recordlength.

         \sstitem
            HISTORY is not propagated.

         \sstitem
            ORIGIN information is lost.

         \sstitem
            When a header is to be made, it is composed of FITS-like card
            images as follows:
      
         \sstitemlist{

            \sstitem
               The number of dimensions of the data array is written
               to the keyword NAXIS, and the actual dimensions to NAXIS1,
               NAXIS2 {\it etc.} as appropriate.

            \sstitem
               If the NDF contains any linear axis structures the
               information necessary to generate these structures is
               written to the FITS-like headers. For example, if a linear
               AXIS(1) structure exists in the input NDF the value of the
               first data point is stored with the keyword CRVAL1,
               and the incremental value between successive axis data is
               stored in keyword CDELT1. If there is an axis label it is
               written to keyword CRTYPE1, and axis unit is written to CTYPE1.
               (Similarly for AXIS(2) structures {\it etc.}) FITS does not have
               a standard method of storing axis widths and variances, so these
               NDF components will not be propagated to the header.
               Non-linear axis data arrays cannot be represented by CRVAL$n$
               and CDELT$n$, and must be ignored.

            \sstitem
               If the input NDF contains TITLE, LABEL or UNITS components
               these are stored with the keywords TITLE, LABEL or BUNITS
               respectively.

            \sstitem
               If the input NDF contains a FITS extension, the FITS items
               may be written to the FITS-like header, with the following
               exceptions:
               \begin{itemize}
               \item BITPIX is derived from the type of the NDF data array,
               and so it is not copied from the NDF FITS extension.
               \item NAXIS, and NAXIS$n$ are derived from the dimensions of the
               NDF data array as described above, so these items are not
               copied from the NDF FITS extension.
               \item The TITLE, LABEL, and BUNITS descriptors are only copied
               if no TITLE, LABEL, and UNITS NDF components respectively
               have already been copied into these headers.
               \item The CDELT$n$, CRVAL$n$, CTYPE$n$, and CRTYPE$n$ descriptors
               in the FITS extension are only copied if the input NDF
               contained no linear axis structures.
               \item The standard order of the FITS keywords is preserved,
               thus BITPIX, NAXIS and AXIS$n$ appear immediately after the
               first card image, which should be SIMPLE.
               \item BSCALE and BZERO in a FITS extension are copied when
               BITPIX is positive, {\it i.e.} the array is not floating-point.
               \end{itemize}

            \sstitem
               An extra header record with keyword UNSIGNED and logical
               value T is added when the array data type is one of the HDS
               unsigned integer types.  This is done because standard FITS
               does not support unsigned integers, and allows (in conjunction
               with BITPIX) applications reading the unformatted file to
               determine the data type of the array.

            \sstitem
               The last header record card will be the standard FITS END.
         }

         \sstitem
            Other extensions are not propagated.
      }
   }
   \sstdiytopic{
      Related Applications
   }{
      CONVERT: ASCII2NDF; KAPPA: TRANDAT; SPECDRE: ASCIN and ASCOUT.
   }
   \sstimplementationstatus{
      \sstitemlist{

         \sstitem
         The value of bad pixels is not written to a FITS-like header
         record with keyword BLANK.
      }
   }
}

\newpage
\sstroutine{
   NDF2BDF
}{
   Converts an NDF to a Starlink Interim BDF file
}{
   \sstdescription{
      This application converts an NDF (see SUN/33) to the Bulk Data
      Frame (BDF) format used by the INTERIM Environment (see SUN/4).
      Type conversion may be performed at the same time.
   }
   \sstusage{
      NDF2BDF IN OUT [TYPE] [DESCRIP]
   }
   \sstparameters{
      \sstsubsection{
         IN = NDF (Read)
      }{
         The NDF to be converted to a BDF.  The suggested default is
         the current NDF if one exists, otherwise it is the current
         value.
      }
      \sstsubsection{
         OUT = BDF (Write)
      }{
         The name of BDF converted from the NDF.  No file extension
         should be given, as the application will automatically give
         extension {\tt ".BDF"}.
      }
      \sstsubsection{
         TYPE = LITERAL (Read)
      }{
         The data type of the output BDF.  It must be one of the
         following Interim types: {\tt "SB"}, {\tt "SW"}, {\tt "R"},
         {\tt "SL"}, {\tt "DP"}, {\tt "UB"}, {\tt "UW"} corresponding
         to signed byte, signed word, real, signed
         longword, double precision, unsigned byte, unsigned word.
         See SUN/4 for further details.  The default is the type
         corresponding to that of the NDF.  {\tt []}
      }
      \sstsubsection{
         DESCRIP = \_LOGICAL (Read)
      }{
         If true the keyword and values in a FITS extension are copied
         to the BDF's descriptors with a number of exceptions listed
         in the Notes.  {\tt [FALSE]}
      }
      \sstsubsection{
         CONNECT = FILENAME (Write)
      }{
         The Interim connection file.  It is deleted when the
         application terminates.  {\tt [NDF2BDF.TMP]}
      }
      \sstsubsection{
         COMMAND = FILENAME (Write)
      }{
         The Interim command file.  It is deleted when the application
         terminates.  {\tt [USERCOM.TMP]}
      }
   }
   \sstexamples{
      \sstexamplesubsection{
         NDF2BDF NEW OLD
      }{
         This converts the NDF called NEW (in file NEW.SDF) to the
         BDF called OLD (in file OLD.BDF).  OLD's data array will have
         the same data type as that of NEW.  The FITS header within
         NEW is converted to descriptors within OLD.
      }
      \sstexamplesubsection{
         NDF2BDF NEW OLD DESCRIP
      }{
         This converts the NDF called NEW (in file NEW.SDF) to the
         BDF called OLD (in file OLD.BDF).  OLD's data array will have
         the same data type as that of NEW.  The FITS header within
         NEW is converted to descriptors within OLD, and are reported to 
         the user.
      }
      \sstexamplesubsection{
         NDF2BDF HORSE HORSE TYPE=R
      }{
         This converts the NDF called HORSE (in file HORSE.SDF) to the
         BDF also called HORSE (in file HORSE.BDF).  The BDF's data
         array will contain 4-byte floating-point numbers.  The FITS
         header within NEW is converted to descriptors within the HORSE
         BDF.
      }
   }
   \sstnotes{
      The details of the conversion are as follows:
      \sstitemlist{

         \sstitem
            the NDF main data array is written to the BDF data array.

         \sstitem
            QUALITY, and VARIANCE have no counterparts in the BDF, and
            so cannot be propagated.

         \sstitem
            HISTORY is not propagated.

         \sstitem
            UNITS is written to descriptor BUNITS.

         \sstitem
            The number of dimensions of the data array is written
            to the BDF descriptor NAXIS, and the actual dimensions to
            NAXIS1, NAXIS2 {\it etc.}\ as appropriate.

         \sstitem
            If the NDF contains any linear axis structures the
            information necessary to generate these structures is
            written to the BDF descriptors. For example, if a linear
            AXIS(1) structure exists in the input NDF the value of the
            first data point is stored in the BDF descriptor CRVAL1,
            and the incremental value between successive axis data is
            stored in CDELT1. If there is an axis label it is written to
            descriptor CRTYPE1, and axis unit is written to CTYPE1.
            (Similarly for AXIS(2) structures {\it etc.}) FITS does not have a
            standard method of storing axis widths and variances, so these
            NDF components will not be propagated.  Non-linear axis data
            arrays cannot be represented by CRVAL$n$ and CDELT$n$, and must be
            ignored.

         \sstitem
            If the input NDF contains TITLE and LABEL components these
            are stored in the BDF descriptors TITLE and LABEL.

         \sstitem
            If the input NDF contains a FITS extension, the FITS items
            may be written to the BDF descriptors, with the following
            exceptions:
            \begin{itemize}
               \item NAXIS, and NAXIS$n$ are derived from the dimensions of
               the NDF data array as described above, so these items
               are not copied from the NDF FITS extension.
               \item The TITLE, LABEL, and BUNITS descriptors are only copied
               if no TITLE, LABEL, and UNITS NDF components have already
               been copied into these descriptors.
               \item The CDELT$n$, CRVAL$n$, CTYPE$n$, and CRTYPE$n$
               descriptors in the FITS extension are only copied if the
               input NDF contained no linear axis structures.
               \item The standard order of the FITS keywords is preserved,
               thus NAXIS and NAXIS$n$ appear immediately after the second
               card image, which should be BITPIX.  No FITS comments are
               written following the values of the above exceptions for
               compatibility with certain INTERIM applications.
               FITS-header lines with blank keywords are not copied.
            \end{itemize}

         \sstitem
            Other extensions have no BDF counterparts and therefore are
            not propagated.

         \sstitem
            All character objects longer than 70 characters are
            truncated in a BDF descriptor.
      }
   }
   \sstdiytopic{
      Related Applications
   }{
      CONVERT: BDF2NDF.
   }
   \sstimplementationstatus{
      \sstitemlist{

         \sstitem
         Primitive NDFs are created.

         \sstitem
         The value of bad pixels is not written to the descriptor BLANK.

         \sstitem
         Only available on VMS platforms.
      }
   }
}

\newpage
\sstroutine{
   NDF2DIPSO
}{
   Converts an NDF to a DIPSO-format file
}{
   \sstdescription{
      The routine converts a 1-dimensional NDF data file into a
      DIPSO-format file.  The resultant file can be imported into DIPSO
      by its READ command.  See SUN/50. The rules for the conversion
      are listed in the Notes.
   }
   \sstusage{
      NDF2DIPSO IN OUT
   }
   \sstparameters{
      \sstsubsection{
         IN = NDF (Read)
      }{
         Input NDF data structure.  A file extension must not be given
         after the name.  The suggested default is the current NDF if
         one exists, otherwise it is the current value.
      }
      \sstsubsection{
         OUT = FILENAME (Write)
      }{
         Output DIPSO file.  A default file extension of {\tt ".DAT"}
         is appended when parameter OUT contains no file extension.
      }
   }
   \sstexamples{
      \sstexamplesubsection{
         NDF2DIPSO OLD NEW
      }{
         This converts the NDF called OLD (in file OLD.SDF) to the
         DIPSO file called NEW.DAT.
      }
      \sstexamplesubsection{
         NDF2DIPSO SPECTRE SPECTRE.DAT
      }{
         This converts the NDF called SPECTRE (in file SPECTRE.SDF) to
         the DIPSO file SPECTRE.DAT.
      }
   }
   \sstnotes{
      \sstitemlist{

         \sstitem
         The NDF TITLE object is to the DIPSO file.

         \sstitem
         The NDF data array becomes the main array in the DIPSO file.
         Bad pixels found in the NDF result in `breaks' in the DIPSO file.

         \sstitem
         The axis centres becomes the $x$-axis array in the DIPSO file.

         \sstitem
         Most NDF components are not supported by the DIPSO format,
         and therefore anything but the data array, axis centres, and
         data title will not be copied.
      }
   }
   \sstdiytopic{
      Related Applications
   }{
      CONVERT: DIPSO2NDF; DIPSO: commands OREAD and OWRITE.
   }
   \sstimplementationstatus{
      \sstitemlist{

         \sstitem
         If the NDF data array exceeds the DIPSO limits of 28000
         elements or 1000 breaks in the data, the application will abort
         with an appropriate error message.

         \sstitem
         If the NDF does not have a title, {\tt "Data from NDF"} is written
         as the DIPSO file's title.

         \sstitem
         If the NDF does not have an axis, the application will abort
         with an appropriate error message.

         \sstitem
         Only available on VMS platforms.
      }
   }
}

\newpage
\sstroutine{
   NDF2DST
}{
   Converts an NDF to a Figaro (Version 2) DST file
}{
   \sstdescription{
      This application converts an NDF to a Figaro (Version 2) `DST'
      file.  The rules for converting the various components of a DST
      are listed in the notes.  Since both are hierarchical formats
      most files can be be converted with little or no information
      lost.
   }
   \sstusage{
      ndf2dst in out
   }
   \sstparameters{
      \sstsubsection{
         IN = NDF (Read)
      }{
         Input NDF data structure.  The suggested default is the
         current NDF if one exists, otherwise it is the current value.
      }
      \sstsubsection{
         OUT = Figaro (Write)
      }{
         Output Figaro file name. This excludes the file extension.
         The file created will be given extension {\tt ".dst"}.
      }
   }
   \sstexamples{
      \sstexamplesubsection{
         ndf2dst old new
      }{
         This converts the NDF called old (in file old.sdf) to the
         Figaro file new.dst.
      }
      \sstexamplesubsection{
         ndf2dst spectre spectre
      }{
         This converts the NDF called spectre (in file spectre.sdf) to
         the Figaro file spectre.dst.
      }
   }
   \sstnotes{
      The rules for the conversion are as follows:

      \begin{center}
      \begin{tabular}{|lcl|p{56mm}|}
      \hline 
      \multicolumn{1}{|c}{NDF} & & Figaro file &
      \multicolumn{1}{|c|}{Comments} \\ \hline
      Main data array  & $\Rightarrow$ & .Z.DATA & \\
      Bad-pixel flag   & $\Rightarrow$ & .Z.FLAGGED & \\
      Units            & $\Rightarrow$ & .Z.UNITS & \\
      Label            & $\Rightarrow$ & .Z.LABEL & \\
      Variance         & $\Rightarrow$ & .Z.ERRORS & after processing \\
      Quality          & $\Rightarrow$ &  & It is not copied directly
                         though bad values indicated by QUALITY flags will
                         be flagged in .Z.DATA in addition to any flagged
                         values actually in the input main data array.
                         .Z.FLAGGED is set accordingly. \\
      Title            & $\Rightarrow$ & .OBS.OBJECT & \\
      & & & \\
      AXIS(1) structure & $\Rightarrow$ & .X & \\
      AXIS(1) Data  & $\Rightarrow$ & .X.DATA & unless there is a DATA
                          component of AXIS(1).MORE.FIGARO to allow for a 
                          non-1-dimensional array \\
      AXIS(1) Variance & $\Rightarrow$ & .X.VARIANCE & unless there is a
                          VARIANCE component of AXIS(1).MORE.FIGARO to
                          allow for a non-1-dimensional array \\
      AXIS(1) Width & $\Rightarrow$ & .X.WIDTH & unless there is a WIDTH
                          component of AXIS(1).MORE.FIGARO to
                          allow for a non-1-dimensional array \\
      AXIS(1) Units & $\Rightarrow$ & .X.UNITS & \\
      AXIS(1) Label & $\Rightarrow$ & .X.LABEL & \\ \hline
      \end{tabular}
      \end{center}

      \begin{center}
      \begin{tabular}{|lcl|p{56mm}|}
      \hline 
      \multicolumn{1}{|c}{NDF} & & Figaro file &
      \multicolumn{1}{|c|}{Comments} \\ \hline
      AXIS(1).MORE.FIGARO.xxx & $\Rightarrow$ & .X.xxx & \\
      & & & Similarly for AXIS(2), \dots, AXIS(6) which are renamed to
           .Y .T .U .V or .W \\ \hline
      & & & \\
      FIGARO extension: & & & \\
      .MORE.FIGARO.MAGFLAG & $\Rightarrow$ & .Z.MAGFLAG & \\
      .MORE.FIGARO.RANGE & $\Rightarrow$ & .Z.RANGE & \\
      .MORE.FIGARO.SECZ & $\Rightarrow$ & .OBS.SECZ & \\
      .MORE.FIGARO.TIME & $\Rightarrow$ & .OBS.TIME & \\
      .MORE.FIGARO.xxx & $\Rightarrow$ & .xxx & recursively \\
      & & & \\
      FITS extension: & & & \\
      .MORE.FITS & & & \\
      Items  & $\Rightarrow$ & .FITS.xxx & \\
      Comments & $\Rightarrow$ & .COMMENTS.xxx & \\
      & & & \\
      Other extensions: & & & \\
      .MORE.other & $\Rightarrow$ & .MORE.other & \\ \hline
      \end{tabular}
      \end{center}
   }
   \sstdiytopic{
      Related Applications
   }{
      CONVERT: DST2NDF.
   }
}

\newpage
\sstroutine{
   NDF2GASP
}{
   Converts a two-dimensional NDF into a GASP image
}{
   \sstdescription{
      This application converts a two-dimensional NDF into the GAlaxy
      Surface Photometry (GASP) package's format.  See the Notes for the
      details of the conversion.
   }
   \sstusage{
      ndf2gasp in out [fillbad]
   }
   \sstparameters{
      \sstsubsection{
         IN = NDF (Read)
      }{
         The input NDF data structure. The suggested default is the
         current NDF if one exists, otherwise it is the current value.
      }
      \sstsubsection{
         FILLBAD = \_INTEGER (Read)
      }{
         The value used to replace bad pixels in the NDF's data array
         before it is copied to the GASP file.  The value must be in the
         range of signed words (-32768 to 32767).  A null value ({\tt !})
         means no replacements are to be made.  This parameter is
         ignored if there are no bad values.  {\tt [!]}
      }
      \sstsubsection{
         OUT = LITERAL (Write)
      }{
         The name of the output GASP image.  Two files are produced
         with the same name but different extensions.  The {\tt "}.dat{\tt "} file
         contains the data array, and {\tt "}.hdr{\tt "} is the associated header
         file that specifies the dimensions of the image.  The
         suggested default is the current value.
      }
   }
   \sstexamples{
      \sstexamplesubsection{
         ndf2gasp abell1367 a1367
      }{
         Converts an NDF called abell1367 into the GASP image comprising
         the pixel file a1367.dat and the header file a1367.hdr.  If
         there are any bad values present they are copied verbatim to
         the IRAF image.
      }
      \sstexamplesubsection{
         ndf2gasp ngc253 ngc253 fillbad=-1
      }{
         Converts the NDF called ngc253 to a GASP image comprising the
         pixel file ngc253.dat and the header file ngc253.hdr.  Any bad
         values in the data array are replaced by minus one.
      }
   }
   \sstnotes{
      The rules for the conversion are as follows:
      \sstitemlist{

         \sstitem
         The NDF data array is copied to the {\tt "}.dat{\tt "} file.

         \sstitem
         The dimensions of the NDF data array is written to the {\tt "}.hdr{\tt "}
         header file.

         \sstitem
         All other NDF components are ignored.
      }
   }
   \sstdiytopic{
      Related Applications
   }{
      CONVERT: GASP2NDF.
   }
   \sstdiytopic{
      References
   }{
      GASP documentation (MUD/66).
   }
   \sstimplementationstatus{
      \sstitemlist{

         \sstitem
         The GASP image produced has by definition type SIGNED WORD.
         There is type conversion of the data array to this type.

         \sstitem
         Bad values may arise due to type conversion.  These too are
         substituted by the (non-null) value of FILLBAD.

         \sstitem
         For an NDF with an odd number of columns, the last column from
         the GASP image is trimmed.
      }
   }
}

\newpage
\sstroutine{
   NDF2IRAF
}{
   Converts an NDF to an IRAF image
}{
   \sstdescription{
      This application converts an NDF to an IRAF image. 
      See the Notes for details of the conversion.
   }
   \sstusage{
      ndf2iraf in out [fillbad]
   }
   \sstparameters{
      \sstsubsection{
         IN = NDF (Read)
      }{
         The input NDF data structure.  The suggested default is the
         current NDF if one exists, otherwise it is the current value.
      }
      \sstsubsection{
         FILLBAD = \_REAL (Read)
      }{
         The value used to replace bad pixels in the NDF's data array
         before it is copied to the IRAF file.  A null value ({\tt !}) means
         no replacements are to be made.  This parameter is ignored if
         there are no bad values.  {\tt [!]}
      }
      \sstsubsection{
         OUT = LITERAL (Write)
      }{
         The name of the output IRAF image.  Two files are produced
         with the same name but different extensions. The {\tt "}.pix{\tt "} file
         contains the data array, and {\tt "}.imh{\tt "} is the associated header
         file that may contain a copy of the NDF's FITS extension.
         The suggested default is the current value.
      }
   }
   \sstexamples{
      \sstexamplesubsection{
         ndf2iraf abell119 a119
      }{
         Converts an NDF called abell119 into the IRAF image comprising
         the pixel file a119.pix and the header file a119.imh.  If there
         are any bad values present they are copied verbatim to the IRAF
         image.
      }
      \sstexamplesubsection{
         ndf2iraf qsospe qsospe fillbad=0
      }{
         Converts the NDF called QSOSPE to an IRAF image comprising the
         pixel file qsospe.imh and the header file qsospe.pix.  Any bad
         values in the data array are replaced by zero.
      }
   }
   \sstnotes{
      The rules for the conversion are as follows:
      \sstitemlist{

         \sstitem
         The NDF data array is copied to the {\tt "}.pix{\tt "} file.

         \sstitem
         The NDF title is written to the header object i\_title in
         the {\tt "}.imh{\tt "} header file. There is a limit of twenty
         characters.

         \sstitem
         If there is a FITS extension in the NDF, then the elements up
         to the first END keyword of this are added to the `user area' of
         the IRAF header file.

         \sstitem
         A HISTORY record is added to the IRAF header file indicating
         that it originated in the named NDF and was converted by
         NDF2IRAF.

         \sstitem
         All other NDF components are ignored.
      }
   }
   \sstdiytopic{
      Related Applications
   }{
      CONVERT: IRAF2NDF.
   }
   \sstimplementationstatus{
      \sstitemlist{

         \sstitem
         Only handles one-, two-, and three-dimensional NDFs.

         \sstitem
         Of the NDF's array components only the data array may be
         copied.

         \sstitem
         The IRAF image produced has type SIGNED WORD or REAL dependent
         of the type of the NDF's data array.  (The IRAF imfort FORTRAN
         subroutine library only supports these data types.)  For \_BYTE,
         \_UBYTE, and \_WORD data arrays the IRAF image will have type
         SIGNED WORD; for all other data types of the NDF data array a
         REAL IRAF image is made.  The pixel type of the image can be
         changed from within IRAF using the {\bf chpixtype} task in the
         {\bf images} package.

         \sstitem
         Bad values may arise due to type conversion.  These too are
         substituted by the (non-null) value of FILLBAD.
      }
   }
}

\newpage
\sstroutine{
   NDF2UNF
}{
   Converts an NDF to a sequential unformatted file
}{
   \sstdescription{
      This application converts an NDF to a sequential unformatted
      file.  Only one of the array components may be copied to the
      output file.  Preceding the data there is an optional header
      consisting of either the FITS extension with the values of
      certain keywords replaced by information derived from the NDF, or
      a minimal FITS header also derived from the NDF.
   }
   \sstusage{
      ndf2unf in out [comp] [noperec]
   }
   \sstparameters{
      \sstsubsection{
         COMP = LITERAL (Read)
      }{
         The NDF component to be copied.  It may be {\tt "Data"},
         {\tt "Quality"} or {\tt "Variance"}. {\tt ["Data"]}
      }
      \sstsubsection{
         FITS = \_LOGICAL (Read)
      }{
         If true, any FITS extension is written to start of the output
         file, unless there is no extension whereupon a minimal FITS
         header is written to the unformatted file. {\tt [FALSE]}
      }
      \sstsubsection{
         IN = NDF (Read)
      }{
         Input NDF data structure. The suggested default is the current
         NDF if one exists, otherwise it is the current value.
      }
      \sstsubsection{
         NOPEREC = \_INTEGER (Read)
      }{
         The number of data values per record of the output file.  It
         should be in the range 1 to $n$, where $n$ is 32764 divided by
         the number of bytes per data value.  The suggested
         default is the current value. {\tt [}The first dimension of the NDF
         (or $n$ if this is smaller){\tt ]}
      }
      \sstsubsection{
         OUT = FILENAME (Write)
      }{
         Name of the output sequential unformatted file.  The file will
         but always fixed-length records when there is no header.  On
         VMS platforms a default file extension of {\tt ".dat"} is appended
         when parameter OUT contains no file extension.
      }
   }
   \sstexamples{
      \sstexamplesubsection{
         ndf2unf cluster cluster.dat
      }{
         This copies the data array of the NDF called cluster to an
         unformatted file called cluster.dat.  The number of data values
         per record is equal to the size of the first dimension of the
         NDF.
      }
      \sstexamplesubsection{
         ndf2unf cluster cluster.dat v
      }{
         This copies the variance of the NDF called cluster to an
         unformatted file called cluster.dat.  The number of variance
         values per record is equal to the size of the first dimension
         of the NDF.
      }
      \sstexamplesubsection{
         ndf2unf cluster cluster.dat noperec=12
      }{
         This copies the data array of the NDF called cluster to an
         unformatted file called cluster.dat.  There are twelve data
         values per record in cluster.dat.
      }
      \sstexamplesubsection{
         ndf2unf out=ndf234.dat fits in=@234
      }{
         This copies the data array of the NDF called 234 to an
         unformatted file called ndf234.dat.  The number of data values
         per record is equal to the size of the first dimension of the
         NDF.  If there is a FITS extension, it is copied to ndf234.dat
         with substitution of certain keywords, otherwise a minimal
         FITS header is produced.
      }
   }
   \sstnotes{
      The details of the conversion are as follows:
      \sstitemlist{

         \sstitem
            the NDF array as selected by COMP is written to the
            unformatted file in records following an optional header.

         \sstitem
            HISTORY is not propagated.

         \sstitem
            ORIGIN information is lost.

         \sstitem
            When a header is to be made, it is composed of FITS-like card
            images as follows:
      
         \sstitemlist{

            \sstitem
               The number of dimensions of the data array is written
               to the keyword NAXIS, and the actual dimensions to NAXIS1,
               NAXIS2 {\it etc.} as appropriate.

            \sstitem
               If the NDF contains any linear axis structures the
               information necessary to generate these structures is
               written to the FITS-like headers. For example, if a linear
               AXIS(1) structure exists in the input NDF the value of the
               first data point is stored with the keyword CRVAL1,
               and the incremental value between successive axis data is
               stored in keyword CDELT1. If there is an axis label it is
               written to keyword CRTYPE1, and axis unit is written to CTYPE1.
               (Similarly for AXIS(2) structures {\it etc.}) FITS does not have
               a standard method of storing axis widths and variances, so these
               NDF components will not be propagated to the header.
               Non-linear axis data arrays cannot be represented by CRVAL$n$
               and CDELT$n$, and must be ignored.

            \sstitem
               If the input NDF contains TITLE, LABEL or UNITS components
               these are stored with the keywords TITLE, LABEL or BUNITS
               respectively.

            \sstitem
               If the input NDF contains a FITS extension, the FITS items
               may be written to the FITS-like header, with the following
               exceptions:
               \begin{itemize}
               \item BITPIX is derived from the type of the NDF data array,
               and so it is not copied from the NDF FITS extension.
               \item NAXIS, and NAXIS$n$ are derived from the dimensions of the
               NDF data array as described above, so these items are not
               copied from the NDF FITS extension.
               \item The TITLE, LABEL, and BUNITS descriptors are only copied
               if no TITLE, LABEL, and UNITS NDF components respectively
               have already been copied into these headers.
               \item The CDELT$n$, CRVAL$n$, CTYPE$n$, and CRTYPE$n$ descriptors
               in the FITS extension are only copied if the input NDF
               contained no linear axis structures.
               \item The standard order of the FITS keywords is preserved,
               thus BITPIX, NAXIS and NAXIS$n$ appear immediately after the
               first card image, which should be SIMPLE.
               \item BSCALE and BZERO in a FITS extension are copied when
               BITPIX is positive, {\it i.e.} the array is not floating-point.
               \end{itemize}

            \sstitem
               An extra header record with keyword UNSIGNED and logical
               value T is added when the array data type is one of the HDS
               unsigned integer types.  This is done because standard FITS
               does not support unsigned integers, and allows (in conjunction
               with BITPIX) applications reading the unformatted file to
               determine the data type of the array.

            \sstitem
               The last header record card will be the standard FITS END.
         }

         \sstitem
            Other extensions are not propagated.
      }
   }
   \sstdiytopic{
      Related Applications
   }{
      CONVERT: UNF2NDF.
   }
   \sstimplementationstatus{
      \sstitemlist{

         \sstitem
         The value of bad pixels is not written to a FITS-like header
         record with keyword BLANK.
      }
   }
}

\newpage
\sstroutine{
   UNF2NDF
}{
   Converts a sequential unformatted file to an NDF
}{
   \sstdescription{
      This application converts a sequential unformatted file to an
      NDF.  Only one of the array components may be created from the
      input file.  Preceding the input data there may be an optional
      header.  This header may be skipped, or may consist of a simple
      FITS header.  In the former case the shape of the NDF has be to
      be supplied.
   }
   \sstusage{
      unf2ndf in out [comp] noperec [skip] shape [type]
   }
   \sstparameters{
      \sstsubsection{
         COMP = LITERAL (Read)
      }{
         The NDF component to be copied.  It may be {\tt "Data"},
         {\tt "Quality"} or {\tt "Variance"}.  To create a variance or
         quality array the NDF must already exist. {\tt ["Data"]}
      }
      \sstsubsection{
         FITS = \_LOGICAL (Read)
      }{
         If true, the initial records of the unformatted file are
         interpreted as a FITS header (with one card image per record)
         from which the shape, data type, and axis centres are derived.
         The last record of the FITS-like header must be terminated by
         an END keyword; subsequent records in the input file are
         treated as an array component given by COMP.  {\tt [FALSE]}
      }
      \sstsubsection{
         IN = FILENAME (Read)
      }{
         Name of the input sequential unformatted Fortran file.  The
         file will normally have variable-length records when there is
         a header, but always fixed-length records when there is no
         header.  On VMS platforms a default file extension of {\tt ".DAT"}
         is appended when parameter IN contains no file extension.
      }
      \sstsubsection{
         NOPEREC = \_INTEGER (Read)
      }{
         The number of data values per record of the input file.
         It should be in the range 1 to 16383.  The suggested default
         is the size of the first dimension of the array if there is
         no current value.  A null ({\tt !}) value for NOPEREC causes the
         size of first dimension to be used.
      }
      \sstsubsection{
         OUT = NDF (Read and Write)
      }{
         Output NDF data structure.  When COMP is not {\tt "Data"} the NDF
         is modified rather than a new NDF created.  It becomes the new
         current NDF.
      }
      \sstsubsection{
         SHAPE = \_INTEGER (Read)
      }{
         The shape of the NDF to be created.  For example, {\tt [40,30,20]}
         would create 40 columns by 30 lines by 10 bands.  It is only
         accessed when FITS is {\tt FALSE}.
      }
      \sstsubsection{
         SKIP = INTEGER (Read)
      }{
         The number of header records to be skipped at the start of the
         input file before finding the data array or FITS-like header.
         {\tt [0]}
      }
      \sstsubsection{
         TYPE = LITERAL (Read)
      }{
         The data type of the output NDF.  It must be one of the
         following HDS types: {\tt "\_BYTE"}, {\tt "\_WORD"}, {\tt "\_REAL"},
         {\tt "\_INTEGER"}, {\tt "\_DOUBLE"}, {\tt "\_UBYTE"},
         {\tt "\_UWORD"} corresponding to signed byte,
         signed word, real, integer, double precision, unsigned byte,
         and unsigned word.  See SUN/92 for further details.  An
         unambiguous abbreviation may be given.  TYPE is ignored when
         COMP = {\tt "Quality"} since the QUALITY component must comprise
         unsigned bytes (equivalent to TYPE = {\tt "\_UBYTE"}) to be a valid
         NDF. The suggested default is the current value.  {\tt ["\_REAL"]}
      }
   }
   \sstexamples{
      \sstexamplesubsection{
         unf2ndf ngc253.dat ngc253 shape=[100,60] noperec=8
      }{
         This copies a data array from the unformatted file ngc253.DAT
         to the NDF called ngc253.  ngc253.DAT does not contain a
         header section.  The NDF is two-dimensional: 100 elements in $x$
         by 60 in $y$.  Its data array has type \_REAL.  The data records
         each have 8 values.
      }
      \sstexamplesubsection{
         unf2ndf ngc253q.dat ngc253 q 100 shape=[100,60] update
      }{
         This copies a quality array from the unformatted file
         ngc253q.dat to an existing NDF called ngc253 (such as created
         in the first example).  ngc253q.dat does not contain a header
         section.  The NDF is two-dimensional: 100 elements in $x$ by 60
         in $y$.  Its data array has type \_UBYTE.  The data records
         each have 100 values.
      }
      \sstexamplesubsection{
         unf2ndf ngc253.dat ngc253 fits noperec=!
      }{
         This copies a data array from the unformatted file ngc253.dat
         to the NDF called ngc253.  ngc253.dat contains a FITS-like
         header section, which is copied to the FITS extension of the
         NDF.  The shape of the NDF is controlled by the mandatory FITS
         keywords NAXIS, AXIS1, \ldots, AXIS$n$, and the data type by
         keywords BITPIX and UNSIGNED.  Each data record has AXIS1
         values (except perhaps for the last).
      }
      \sstexamplesubsection{
         unf2ndf type='\_uword' in=ngc253.dat out=ngc253 $\backslash$
      }{
         This copies a data array from the unformatted file ngc253.dat
         to the NDF called ngc253.  ngc253.dat does not contain a
         header section.  The NDF has the current shape and data type
         is unsigned word.  The current number of values per record is
         used.
      }
      \sstexamplesubsection{
         unf2ndf spectrum zz skip=2 shape=200 noperec=!
      }{
         This copies a data array from the unformatted file spectrum
         to the NDF called zz.  spectrum contains two header
         records that are ignored.  The NDF is one-dimensional
         comprising 200 elements of type \_REAL.  There is one data
         record containing the whole array.
      }
      \sstexamplesubsection{
         unf2ndf spectrum.lis ZZ skip=1 fits noperec=20
      }{
         This copies a data array from the unformatted file spectrum.lis
         to the NDF called ZZ.  spectrum.lis contains one header
         record, that is ignored, followed by a FITS-like header
         section, which is copied to the FITS extension of the NDF.
         The shape of the NDF is controlled by the mandatory FITS
         keywords NAXIS, AXIS1, \ldots, AXIS$n$, and the data type by
         keywords BITPIX and UNSIGNED.  Each data record has AXIS1
         values (except perhaps for the last).
      }
   }
   \sstnotes{
      The details of the conversion are as follows:
      \sstitemlist{

         \sstitem
            the unformatted-file array is written to the NDF array as
            selected by COMP.  When the NDF is being modified, the shape
            of the new component must match that of the NDF.

         \sstitem
            If the input file contains a FITS-like header, and a new
            NDF is created, {\it i.e.}\ COMP = {\tt "Data"}, the header
            records are placed within the NDF's FITS extension.  This enables 
            more than one array (input file) to be used to form an NDF.  Note
            that the data array must be created first to make a valid NDF,
            and it's the FITS structure associated with that array that is
            wanted.  Indeed the application prevents you from doing
            otherwise.

         \sstitem
            The FITS-like header defines the properties of the NDF as
            follows:
            \begin{itemize}
            \item BITPIX defines the data type: 8 gives \_BYTE, 16 produces
            \_WORD, 32 makes \_INTEGER, $-$32 gives \_REAL, and $-$64 generates
            \_DOUBLE.  For the first two, if there is an extra header
            record with the keyword UNSIGNED and logical value T, these
            types become \_UBYTE and \_UWORD respectively.  UNSIGNED is
            non-standard, since unsigned integers would not follow in a
            proper FITS file.  However, here it is useful to enable
            unsigned types to be input into an NDF.  UNSIGNED may be
            created by this application's sister, NDF2UNF.  BITPIX is
            ignored for QUALITY data; type \_UBYTE is used.
            \item NAXIS, and NAXIS$n$ define the shape of the NDF.
            \item The TITLE, LABEL, and BUNITS are copied to the NDF
            TITLE, LABEL, and UNITS NDF components respectively.
            \item The CDELT$n$, CRVAL$n$, CTYPE$n$, and CRTYPE$n$ keywords make
            linear axis structures within the NDF.  CTYPE$n$ define the
            axis units, and the axis labels are assigned to CRTYPE$n$ If
            some are missing, pixel co-ordinates are used for those
            axes.
            \item BSCALE and BZERO in a FITS extension are ignored.
            \item BLANK is not used to indicate which input array values
            should be assigned to a standard bad value.
            \item END indicates the last header record unless it
            terminates a dummy header, and the actual data is in an
            extension.
            \end{itemize}

         \sstitem
            Other data item such as HISTORY, data ORIGIN, and axis
            widths are not supported, because the unformatted file has a
            simple structure to enable a diverse set of input files to be
            converted to NDFs, and to limitations of the standard FITS
            header.
      }
   }
   \sstdiytopic{
      Related Applications
   }{
      CONVERT: NDF2UNF.
   }
}

\newpage
\normalsize
\section{Release Notes---V0.5}

This the first release where {\footnotesize CONVERT} is available on
UNIX platforms.  There are four new commands. 

\subsection{Global changes}
\begin{itemize}
  \item  {\footnotesize CONVERT} is available on UNIX platforms
         SUN/SunOS, SUN/Solaris, DEC/Ultrix, Alpha/OSF1 with the
         following exceptions:

  \begin{itemize}

     \item BDF2NDF and NDF2BDF are not available because the Interim
           library is not being ported.

     \item DIPSO2NDF and NDF2DIPSO are not required because UNIX DIPSO
           processes NDFs.

     \item IRAF2NDF and NDF2IRAF are currently not available under OSF/1
           and Solaris, because at the time of development there were no
           IRAF IMFORT libraries released for these platforms.
  \end{itemize}

  \item  CONHELP runs an application on UNIX platforms and accesses a
         portable-help library.

  \item  The documentation is revised to reflect the lowercase usage and
         examples under UNIX.

  \item  Special handling of some pathological datafiles have been made,
         mostly in BDF2NDF.
\end{itemize}

\subsection{New applications}
There are four new applications.

\begin{description}
   \item [GASP2NDF] -- Converts an image in GASP format to an NDF.
   \item [IRAF2NDF] -- Converts an IRAF image to an NDF.
   \item [IRCAM2NDF] -- Converts an IRCAM data file to a series of NDFs.
   \item [NDF2GASP] -- Converts a two-dimensional NDF into a GASP image.
\end{description}

\subsection{Extended and corrected applications}
  
Here is a summary of the main modifications.
\begin{itemize}
  \item NDF2DST propagates axis-variance and width arrays, and the
        contents of any axis Figaro extension.  If there is only one
        object in the Figaro extension, it is now copied to the DST.

  \item BDF2NDF now ignores deleted descriptors rather than propagating
        them to the NDF's FITS extension.

  \item DST2NDF has a FORM parameter to select the NDF storage form of
        the output NDF.  It defaults to the simple form.  Previously,
        the quality and bad-pixel values decided.

  \item DST2NDF permits the output NDF to be placed inside an arbitrary
        HDS structure, rather than just being the sole top-level
        structure in the HDS container file.

  \item DST2NDF has had a number of bug fixes applied.  In V0.4 it was
        possible for an axis data array to have the wrong number of
        elements in certain circumstances; the FLAGGED value could have
        been set incorrectly when the FITS structure preceeded the .Z
        structure within the DST file.

  \item DST2NDF and NDF2DST now handle quotes in FITS character values.

  \item NDF2ASCII and NDF2UNF ensure that a SIMPLE card appears first in
        the FITS header.

  \item NDF2BDF has better handling of rotated axes.  These are restored
        from the FITS extension, by overwriting the descriptors derived
        from the NDF AXIS.  (An NDF AXIS does not support rotated axes,
        so BDF2NDF makes a default axis with pixel co-ordinates.  A
        subsequent invocation of the old NDF2BDF would lose the original
        axis information from the destination NDF.  This change prevents
        that hysteresis.)

  \item NDF2BDF has had a number of bug fixes applied.  It is now not
        possible to generate erroneous CRTYPE$n$ and CTYPE$n$ values,
        when the NDF AXIS does not contain LABEL or UNITS.  The
        application reprompts if an invalid Interim type is given.
        Lowercase types may now be entered.  An NDF with a FITS
        extension containing only the mandatory BDF descriptors (NAXIS,
        NAXIS$n$) will now produce a valid IMAGE BDF, {\it i.e.}\ NAXIS,
        NAXIS$n$ derived from the NDF's shape are no longer missing from
        the BDF descriptors.

  \item NDF2DST has improved handling of FITS headers, particularly
        character strings.  Duplicated keywords are ignored.  HISTORY
        and COMMENT cards are processed correctly.

  \item NDF2IRAF can now convert one- and three-dimensional NDFs.

  \item NDF2IRAF has a FILLBAD parameter to allow NDF bad values to
        be replaced.  FILLBAD defaults to null, meaning do not perform
        bad-value substitution.

  \item NDF2IRAF can produce a signed-word IRAF image.  The output data
        type depends on the data type of the NDF's data array.

\end{itemize}

\end{document}
