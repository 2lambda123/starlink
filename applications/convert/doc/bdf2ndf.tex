\sstroutine{
   BDF2NDF
}{
   Converts a Starlink Interim BDF file into an NDF
}{
   \sstdescription{
      This application converts data files from the Bulk Data Frame
      (BDF) format used by the INTERIM Environment (see SUN/4) to the
      Starlink standard NDF format (see SUN/33 or SGP/38).  Type
      conversion may be performed at the same time.
   }
   \sstusage{
      BDF2NDF IN OUT [TYPE] [DESCRIP]
   }
   \sstparameters{
      \sstsubsection{
         IN = BDF (Read)
      }{
         The BDF to be converted to an NDF.  A file extension must not
         be given since {\tt "}.BDF{\tt "} is assumed.
      }
      \sstsubsection{
         OUT = NDF (Write)
      }{
         The name of NDF converted from the BDF.  It becomes the new
         current NDF.
      }
      \sstsubsection{
         TYPE = LITERAL (Read)
      }{
         The data type of the output NDF.  It must be one of the
         following HDS types: {\tt "\_BYTE"}, {\tt "\_WORD"},
         {\tt "\_REAL"}, {\tt "\_INTEGER"}, {\tt "\_DOUBLE"},
         {\tt "\_UBYTE"}, {\tt "\_UWORD"} corresponding to signed byte,
         signed word, real, integer, double precision, unsigned byte,
         and unsigned word.  See SUN/92 for further details.  An
         unambiguous abbreviation may be given.  The default is the
         type corresponding to that of the BDF.  {\tt []}
      }
      \sstsubsection{
         DESCRIP = \_LOGICAL (Read)
      }{
         If true the descriptors in the BDF are reported as they are
         copied to the FITS extension within the output NDF. {\tt [FALSE]}
      }
      \sstsubsection{
         CONNECT = FILENAME (Write)
      }{
         The Interim connection file.  It is deleted when the
         application terminates.  {\tt [NDF2BDF.TMP]}
      }
      \sstsubsection{
         COMMAND = FILENAME (Write)
      }{
         The Interim command file.  It is deleted when the application
         terminates.  {\tt [USERCOM.TMP]}
      }
   }
   \sstexamples{
      \sstexamplesubsection{
         BDF2NDF OLD NEW
      }{
         This converts the BDF called OLD (in file OLD.BDF) to the NDF
         called NEW (in file NEW.SDF).  NEW's data array will have
         the same data type as that of OLD.  Descriptors are copied
         to the FITS extension but are not reported.
      }
      \sstexamplesubsection{
         BDF2NDF OLD NEW DESCRIP
      }{
         This converts the BDF called OLD (in file OLD.BDF) to the NDF
         called NEW (in file NEW.SDF).  NEW's data array will have
         the same data type as that of OLD.  Descriptors are copied
         to the FITS extension and are reported.
      }
      \sstexamplesubsection{
         BDF2NDF HORSE HORSE TYPE="\_WORD"
      }{
         This converts the BDF called HORSE (in file HORSE.BDF) to the
         NDF also called HORSE (in file HORSE.SDF).  The NDF's data
         array will contain signed 2-byte integers.  Descriptors are
         copied to the FITS extension but are not reported.
      }
   }
   \sstnotes{
      \sstitemlist{

         \sstitem
         The conversion rules can be summarised as follow:
         \begin{itemize}
            \item the BDF data array is copied to the NDF main data array;
            by default the output data array has the data type
            corresponding to that of the BDF.
            \item The BLANK descriptor is not used to flag pixels in the NDF
            with the bad value.  Use KAPPA's SETMAGIC to flag another
            value.
            \item The BDF descriptors are written to the NDF FITS extension.
            Long values may be truncated, 65 for characters and 20 for
            numbers.  The formating adheres to the FITS standard.
            Descriptors already in the FITS format are copied as is, so
            La Palma ING-format headers can be propagated.
            \item If the BDF descriptors contain the FITS keywords CRVALn,
            CDELTn, the appropriate axis structures are generated in
            the output NDF. In addition should CRTYPEn also be present
            the labels are added to these structures.
            \item If the BDF descriptors contain the FITS keywords TITLE or
            LABEL or BUNITS, the associated character string value is
            written to the NDF TITLE, LABEL or UNITS component as
            appropriate.
            \item HISTORY descriptors are not used to make an HISTORY
            component in the NDF.
         \end{itemize}

         \sstitem
         WARNING: the BDF size may grow when TYPE is specified due to
         an incarnation being created.  See below and SUN/4 for more
         details.
      }
   }
   \sstdiytopic{
      Incarnations
   }{
      A BDF contains one or more `incarnations' of the data array.  An
      incarnation of a data array is simply a copy of that data array
      stored with a particular data type.  For example, a BDF may
      contain incarnations of the same data array stored as a REAL
      array, and an INTEGER array.  This rather strange behaviour is a
      consequence of the way the INTERIM environment deals with data
      access.  For example, if an application attempts to map a BDF data
      array of INTEGER type as a REAL array, type conversion must take
      place.  Instead of doing this in virtual memory, a second
      incarnation of the data, this time of type REAL is created and
      stored in the original file (causing perhaps a doubling of the
      file size).  This second incarnation of the data array is stored
      to avoid performing the same type conversion in the future.  (It
      is perhaps significant that no such scheme has been employed with
      subsequent data systems.)  One consequence of this behaviour is
      that a BDF data array may not have a unique type.  BDF2NDF will
      use the first type it finds as the default.  Specify the
      TYPE parameter explicitly to ensure that the correct data type is
      created within the NDF.

      To prevent the BDF growing, just remove write access from the
      file.
   }
   \sstimplementationstatus{
      \sstitemlist{

         \sstitem
         Note the data array in the NDF is of the primitive form.
      }
   }
   \sstbugs{
      \sstitemlist{

         \sstitem
         May give spurious error messages if running under ICL.  This
         happens if a non-existent BDF is given as the input file.  A
         subsequent invocation of the application may result in a repeat of
         the error message although the conversion is carried out
         correctly.
      }
   }
}
\end{document}
