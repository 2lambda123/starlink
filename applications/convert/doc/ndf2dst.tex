\newpage
\sstroutine{
   NDF2DST
}{
   Converts an NDF to a Figaro (Version 2) DST file
}{
   \sstdescription{
      This application converts an NDF to a Figaro (Version 2) `DST'
      file.  The rules for converting the various components of a DST
      are listed in the notes.  Since both are hierarchical formats
      most files can be be converted with little or no information
      lost.
   }
   \sstusage{
      ndf2dst in out
   }
   \sstparameters{
      \sstsubsection{
         IN = NDF (Read)
      }{
         Input NDF data structure.  The suggested default is the
         current NDF if one exists, otherwise it is the current value.
      }
      \sstsubsection{
         OUT = Figaro (Write)
      }{
         Output Figaro file name. This excludes the file extension.
         The file created will be given extension {\tt ".dst"}.
      }
   }
   \sstexamples{
      \sstexamplesubsection{
         ndf2dst old new
      }{
         This converts the NDF called old (in file old.sdf) to the
         Figaro file new.dst.
      }
      \sstexamplesubsection{
         ndf2dst spectre spectre
      }{
         This converts the NDF called spectre (in file spectre.sdf) to
         the Figaro file spectre.dst.
      }
   }
   \sstnotes{
      The rules for the conversion are as follows:

      \begin{center}
      \begin{tabular}{|lcl|p{56mm}|}
      \hline 
      \multicolumn{1}{|c}{NDF} & & Figaro file &
      \multicolumn{1}{|c|}{Comments} \\ \hline
      Main data array  & $\Rightarrow$ & .Z.DATA & \\
      Bad-pixel flag   & $\Rightarrow$ & .Z.FLAGGED & \\
      Units            & $\Rightarrow$ & .Z.UNITS & \\
      Label            & $\Rightarrow$ & .Z.LABEL & \\
      Variance         & $\Rightarrow$ & .Z.ERRORS & after processing \\
      Quality          & $\Rightarrow$ &  & It is not copied directly
                         though bad values indicated by QUALITY flags will
                         be flagged in .Z.DATA in addition to any flagged
                         values actually in the input main data array.
                         .Z.FLAGGED is set accordingly. \\
      Title            & $\Rightarrow$ & .OBS.OBJECT & \\
      & & & \\
      AXIS(1) structure & $\Rightarrow$ & .X & \\
      AXIS(1) Data  & $\Rightarrow$ & .X.DATA & unless there is a DATA
                          component of AXIS(1).MORE.FIGARO to allow for a 
                          non-1-dimensional array \\
      AXIS(1) Variance & $\Rightarrow$ & .X.VARIANCE & unless there is a
                          VARIANCE component of AXIS(1).MORE.FIGARO to
                          allow for a non-1-dimensional array \\
      AXIS(1) Width & $\Rightarrow$ & .X.WIDTH & unless there is a WIDTH
                          component of AXIS(1).MORE.FIGARO to
                          allow for a non-1-dimensional array \\
      AXIS(1) Units & $\Rightarrow$ & .X.UNITS & \\
      AXIS(1) Label & $\Rightarrow$ & .X.LABEL & \\ \hline
      \end{tabular}
      \end{center}

      \begin{center}
      \begin{tabular}{|lcl|p{56mm}|}
      \hline 
      \multicolumn{1}{|c}{NDF} & & Figaro file &
      \multicolumn{1}{|c|}{Comments} \\ \hline
      AXIS(1).MORE.FIGARO.xxx & $\Rightarrow$ & .X.xxx & \\
      & & & Similarly for AXIS(2), \dots, AXIS(6) which are renamed to
           .Y .T .U .V or .W \\ \hline
      & & & \\
      FIGARO extension: & & & \\
      .MORE.FIGARO.MAGFLAG & $\Rightarrow$ & .Z.MAGFLAG & \\
      .MORE.FIGARO.RANGE & $\Rightarrow$ & .Z.RANGE & \\
      .MORE.FIGARO.SECZ & $\Rightarrow$ & .OBS.SECZ & \\
      .MORE.FIGARO.TIME & $\Rightarrow$ & .OBS.TIME & \\
      .MORE.FIGARO.xxx & $\Rightarrow$ & .xxx & recursively \\
      & & & \\
      FITS extension: & & & \\
      .MORE.FITS & & & \\
      Items  & $\Rightarrow$ & .FITS.xxx & \\
      Comments & $\Rightarrow$ & .COMMENTS.xxx & \\
      & & & \\
      Other extensions: & & & \\
      .MORE.other & $\Rightarrow$ & .MORE.other & \\ \hline
      \end{tabular}
      \end{center}
   }
   \sstdiytopic{
      Related Applications
   }{
      CONVERT: DST2NDF.
   }
}

