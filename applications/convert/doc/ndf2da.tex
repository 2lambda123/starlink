\newpage
\sstroutine{
   NDF2DA
}{
   Converts an NDF to a direct-access unformatted file
}{
   \sstdescription{
      This application converts an NDF to a direct-access unformatted
      file, which is equivalent to fixed-length records, or a data
      stream suitable for reading by C routines.  Only one of the array
      components may be copied to the output file.
   }
   \sstusage{
      ndf2da in out [comp] [noperec]
   }
   \sstparameters{
      \sstsubsection{
         COMP = LITERAL (Read)
      }{
         The NDF component to be copied.  It may be {\tt "Data"},
         {\tt "Quality"} or {\tt "Variance"}. {\tt ["Data"]}
      }
      \sstsubsection{
         IN = NDF (Read)
      }{
         Input NDF data structure.  The suggested default is the current
         NDF if one exists, otherwise it is the current value.
      }
      \sstsubsection{
         NOPEREC = \_INTEGER (Read)
      }{
         The number of data values per record of the output file.  It
         must be positive.  The suggested default is the current value.
         {\tt{[}}The first dimension of the NDF{\tt{]}}
      }
      \sstsubsection{
         OUT = FILENAME (Write)
      }{
         Name of the output direct-access unformatted file.
      }
   }
   \sstexamples{
      \sstexamplesubsection{
         ndf2da cluster cluster.dat
      }{
         This copies the data array of the NDF called cluster to a
         direct-access unformatted file called {\tt cluster.dat}.  The number
         of data values per record is equal to the size of the first
         dimension of the NDF.
      }
      \sstexamplesubsection{
         ndf2da cluster cluster.dat v
      }{
         This copies the variance of the NDF called cluster to a
         direct-access unformatted file called {\tt cluster.dat}.  The number
         of variance values per record is equal to the size of the
         first dimension of the NDF.
      }
      \sstexamplesubsection{
         ndf2da cluster cluster.dat noperec=12
      }{
         This copies the data array of the NDF called cluster to a
         direct-access unformatted file called cluster.dat.  There are
         twelve data values per record in {\tt cluster.dat}.
      }
   }
   \sstnotes{
      The details of the conversion are as follows:
      \sstitemlist{

         \sstitem
            the NDF array as selected by COMP is written to the
            unformatted file in records.

         \sstitem
            all other NDF components are lost.
      }
   }
   \sstdiytopic{
      Related Applications
   }{
      CONVERT: DA2NDF.
   }
}

