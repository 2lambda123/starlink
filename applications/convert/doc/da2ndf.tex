\newpage
\sstroutine{
   DA2NDF
}{
   Converts a direct-access unformatted file to an NDF
}{
   \sstdescription{
      This application converts a direct-access (fixed-length records)
      unformatted file to an NDF.  It can therefore also process
      unformatted data files generated by C routines.  Only one of the
      array components may be created from the input file.   The shape
      of the NDF has be to be supplied.
   }
   \sstusage{
      da2ndf in out [comp] noperec shape [type]
   }
   \sstparameters{
      \sstsubsection{
         COMP = LITERAL (Read)
      }{
         The NDF component to be copied.  It may be {\tt "Data"},
         {\tt "Quality"}
         or {\tt "Variance"}.  To create a variance or quality array the NDF
         must already exist. {\tt ["Data"]}
      }
      \sstsubsection{
         IN = FILENAME (Read)
      }{
         Name of the input direct-access unformatted file.
      }
      \sstsubsection{
         NOPEREC = \_INTEGER (Read)
      }{
         The number of data values per record of the input file.  It
         must be positive.  The suggested default is the size of the
         first dimension of the array.  A null ({\tt !}) value for NOPEREC
         causes the size of first dimension to be used.
      }
      \sstsubsection{
         OUT = NDF (Read and Write)
      }{
         Output NDF data structure.  When COMP is not {\tt "Data"} the NDF
         is modified rather than a new NDF created.  It becomes the new
         current NDF.  Unusually for an output NDF, there is a suggested
         default---the current value---to facilitate the inclusion of
         variance and quality arrays.
      }
      \sstsubsection{
         SHAPE = \_INTEGER (Read)
      }{
         The shape of the NDF to be created.  For example, {\tt [40,30,20]}
         would create 40 columns by 30 lines by 10 bands.
      }
      \sstsubsection{
         TYPE = LITERAL (Read)
      }{
         The data type of the direct-access file and the NDF.  It must
         be one of the following HDS types: {\tt "\_BYTE"}, {\tt"\_WORD"},
         {\tt "\_REAL"}, {\tt "\_INTEGER"}, {\tt "\_DOUBLE"},
         {\tt "\_UBYTE"}, {\tt "\_UWORD"} corresponding to
         signed byte, signed word, real, integer, double precision,
         unsigned byte, and unsigned word respectively.  See \xref{SUN/92}{sun92}{}
         for further details.  An unambiguous abbreviation may be given.
         TYPE is ignored when COMP = {\tt "Quality"} since the QUALITY
         component must comprise unsigned bytes (equivalent to TYPE =
         {\tt "\_UBYTE"}) to be a valid NDF. The suggested default is the
         current value. {\tt ["\_REAL"]}
      }
   }
   \sstexamples{
      \sstexamplesubsection{
         da2ndf ngc253.dat ngc253 shape=[100,60] noperec=8
      }{
         This copies a data array from the direct-access file {\tt ngc253.dat}
         to the NDF called ngc253.  The NDF is two-dimensional: 100
         elements in $x$ by 60 in $y$.  Its data array has type \_REAL.  The
         data records each have 8 real values.
      }
      \sstexamplesubsection{
         da2ndf ngc253q.dat ngc253 q 100 [100,60]
      }{
         This copies a quality array from the direct-access file
         {\tt ngc253q.dat} to an existing NDF called ngc253 (such as created
         in the first example).  The NDF is two-dimensional: 100
         elements in $x$ by 60 in $y$.  Its data array has type \_UBYTE.
         The data records each have 100 unsigned-byte values.
      }
      \sstexamplesubsection{
         da2ndf type="\_uword" in=ngc253.dat out=ngc253 $\backslash$
      }{
         This copies a data array from the direct-access file
         {\tt ngc253.dat}
         to the NDF called ngc253.  The NDF has the current shape and
         data type is unsigned word.  The current number of values per
         record is used.
      }
   }
   \sstnotes{
      The details of the conversion are as follows:
      \ssthitemlist{

         \sstitem
            the direct-access file{\tt '}s array is written to the NDF array
            as selected by COMP.  When the NDF is being modified, the
            shape of the new component must match that of the NDF.  This
            enables more than one array (input file) to be used to form an
            NDF.  Note that the data array must be created first to make a
            valid NDF.  Indeed the application prevents you from doing
            otherwise.

         \sstitem
            Other data items such as axes are not supported, because of
            the direct-access file{\tt '}s simple structure.
      }
   }
   \sstdiytopic{
      Related Applications
   }{
      \CONVERT: \htmlref{NDF2DA}{NDF2DA}.
   }
}

