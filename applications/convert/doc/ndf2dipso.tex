\newpage
\sstroutine{
   NDF2DIPSO
}{
   Writes an NDF to a DIPSO-format file, which can be read by the
   DIPSO READ command
}{
   \sstdescription{
      The routine reads a 1-dimensional NDF data file and writes a
      DIPSO format file.  The NDF TITLE object is also written to the
      DIPSO file.  Bad pixels found in the NDF result in `breaks' in
      the DIPSO file.
   }
   \sstusage{
      NDF2DIPSO IN OUT
   }
   \sstparameters{
      \sstsubsection{
         IN = NDF (Read)
      }{
         Input NDF data structure.  A file extension must not be given
         after the name.  The suggested default is the current NDF if
         one exists, otherwise it is the current value.
      }
      \sstsubsection{
         OUT = FILENAME (Write)
      }{
         Output DIPSO file.  File extension {\tt "}.DAT{\tt "} is assumed.
      }
   }
   \sstexamples{
      \sstexamplesubsection{
         NDF2DIPSO OLD NEW
      }{
         This converts the NDF file NEW.SDF to the DIPSO file OLD.DAT
         file.
      }
      \sstexamplesubsection{
         NDF2DIPSO SPECTRE SPECTRE
      }{
         This converts the NDF called SPECTRE in file SPECTRE.SDF to
         the DIPSO file SPECTRE.DAT.
      }
   }
   \sstnotes{
      \sstitemlist{

         \sstitem
         Most NDF components are not supported by the DIPSO format,
         and therefore anything but the data array, axis centres, and
         data title will not be copied.
      }
   }
   \sstimplementationstatus{
      \sstitemlist{

         \sstitem
         If the NDF data array exceeds the DIPSO limits of 28000
         elements or 1000 breaks in the data, the application will abort
         with an appropriate error message.

         \sstitem
         If the NDF does not have a title, {\tt "Data from NDF"} is written
         as the DIPSO file's title.

         \sstitem
         If the NDF does not have an axis, the application will abort
         with an appropriate error message.
      }
   }
}
\end{document}
