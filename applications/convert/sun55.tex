\documentclass[twoside,11pt]{article}

% ? Specify used packages
% \usepackage{graphicx}        %  Use this one for final production.
% \usepackage[draft]{graphicx} %  Use this one for drafting.
% ? End of specify used packages

\pagestyle{myheadings}

% -----------------------------------------------------------------------------
% ? Document identification
% Fixed part
\newcommand{\stardoccategory}  {Starlink User Note}
\newcommand{\stardocinitials}  {SUN}
\newcommand{\stardocsource}    {sun\stardocnumber}
\newcommand{\stardoccopyright} 
{Copyright \copyright\ 2005 Council for the Central Laboratory of the Research Councils}

% Variable part - replace [xxx] as appropriate.
\newcommand{\stardocnumber}    {55.23}
\newcommand{\stardocauthors}   {Malcolm J. Currie\\
                                G.J.Privett\\
                                A.J.Chipperfield\\
                                D.S.Berry\\
                                A.C.Davenhall}
\newcommand{\stardocdate}      {2008 February 12}
\newcommand{\stardoctitle}     {CONVERT\\
                                A Format-conversion Package}
\newcommand{\stardocversion}   {Version 1.5-13}
\newcommand{\stardocmanual}    {User's Manual}
\newcommand{\stardocabstract}  {
The \CONVERT\ package contains utilities for converting data files
between Starlink's Extensible {\em n}-dimensional Data Format
\xref{(NDF)}{sun33}{abstract},
which is used by most Starlink applications, and a number of other common
data formats.
Using these utilities, astronomers can process their data selecting the best 
applications from a variety of Starlink or other packages.
\par
Most of the \CONVERT\ utilities may be run from the shell or \ICLref\ in 
the normal way, or invoked automatically by the NDF library's `on-the-fly' 
data-conversion system, but there are also 
\IDLref\
procedures for handling NDFs from within IDL.}
% ? End of document identification
% -----------------------------------------------------------------------------

% +
%  Name:
%     sun.tex
%
%  Purpose:
%     Template for Starlink User Note (SUN) documents.
%     Refer to SUN/199
%
%  Authors:
%     AJC: A.J.Chipperfield (Starlink, RAL)
%     BLY: M.J.Bly (Starlink, RAL)
%     PWD: Peter W. Draper (Starlink, Durham University)
%
%  History:
%     17-JAN-1996 (AJC):
%        Original with hypertext macros, based on MDL plain originals.
%     16-JUN-1997 (BLY):
%        Adapted for LaTeX2e.
%        Added picture commands.
%     13-AUG-1998 (PWD):
%        Converted for use with LaTeX2HTML version 98.2 and
%        Star2HTML version 1.3.
%      1-FEB-2000 (AJC):
%        Add Copyright statement in LaTeX
%     {Add further history here}
%
%-

\newcommand{\stardocname}{\stardocinitials /\stardocnumber}
\markboth{\stardocname}{\stardocname}
\setlength{\textwidth}{160mm}
\setlength{\textheight}{230mm}
\setlength{\topmargin}{-2mm}
\setlength{\oddsidemargin}{0mm}
\setlength{\evensidemargin}{0mm}
\setlength{\parindent}{0mm}
\setlength{\parskip}{\medskipamount}
\setlength{\unitlength}{1mm}

% -----------------------------------------------------------------------------
%  Hypertext definitions.
%  ======================
%  These are used by the LaTeX2HTML translator in conjunction with star2html.

%  Comment.sty: version 2.0, 19 June 1992
%  Selectively in/exclude pieces of text.
%
%  Author
%    Victor Eijkhout                                      <eijkhout@cs.utk.edu>
%    Department of Computer Science
%    University Tennessee at Knoxville
%    104 Ayres Hall
%    Knoxville, TN 37996
%    USA

%  Do not remove the %begin{latexonly} and %end{latexonly} lines (used by 
%  LaTeX2HTML to signify text it shouldn't process).
%begin{latexonly}
\makeatletter
\def\makeinnocent#1{\catcode`#1=12 }
\def\csarg#1#2{\expandafter#1\csname#2\endcsname}

\def\ThrowAwayComment#1{\begingroup
    \def\CurrentComment{#1}%
    \let\do\makeinnocent \dospecials
    \makeinnocent\^^L% and whatever other special cases
    \endlinechar`\^^M \catcode`\^^M=12 \xComment}
{\catcode`\^^M=12 \endlinechar=-1 %
 \gdef\xComment#1^^M{\def\test{#1}
      \csarg\ifx{PlainEnd\CurrentComment Test}\test
          \let\html@next\endgroup
      \else \csarg\ifx{LaLaEnd\CurrentComment Test}\test
            \edef\html@next{\endgroup\noexpand\end{\CurrentComment}}
      \else \let\html@next\xComment
      \fi \fi \html@next}
}
\makeatother

\def\includecomment
 #1{\expandafter\def\csname#1\endcsname{}%
    \expandafter\def\csname end#1\endcsname{}}
\def\excludecomment
 #1{\expandafter\def\csname#1\endcsname{\ThrowAwayComment{#1}}%
    {\escapechar=-1\relax
     \csarg\xdef{PlainEnd#1Test}{\string\\end#1}%
     \csarg\xdef{LaLaEnd#1Test}{\string\\end\string\{#1\string\}}%
    }}

%  Define environments that ignore their contents.
\excludecomment{comment}
\excludecomment{rawhtml}
\excludecomment{htmlonly}

%  Hypertext commands etc. This is a condensed version of the html.sty
%  file supplied with LaTeX2HTML by: Nikos Drakos <nikos@cbl.leeds.ac.uk> &
%  Jelle van Zeijl <jvzeijl@isou17.estec.esa.nl>. The LaTeX2HTML documentation
%  should be consulted about all commands (and the environments defined above)
%  except \xref and \xlabel which are Starlink specific.

\newcommand{\htmladdnormallinkfoot}[2]{#1\footnote{#2}}
\newcommand{\htmladdnormallink}[2]{#1}
\newcommand{\htmladdimg}[1]{}
\newcommand{\hyperref}[4]{#2\ref{#4}#3}
\newcommand{\htmlref}[2]{#1}
\newcommand{\htmlimage}[1]{}
\newcommand{\htmladdtonavigation}[1]{}

\newenvironment{latexonly}{}{}
\newcommand{\latex}[1]{#1}
\newcommand{\html}[1]{}
\newcommand{\latexhtml}[2]{#1}
\newcommand{\HTMLcode}[2][]{}

%  Starlink cross-references and labels.
\newcommand{\xref}[3]{#1}
\newcommand{\xlabel}[1]{}

%  LaTeX2HTML symbol.
\newcommand{\latextohtml}{\LaTeX2\texttt{HTML}}

%  Define command to re-centre underscore for Latex and leave as normal
%  for HTML (severe problems with \_ in tabbing environments and \_\_
%  generally otherwise).
\newcommand{\setunderscore}{\renewcommand{\_}{{\tt\symbol{95}}}}
\latex{\setunderscore}

%  Redefine the \tableofcontents command. This procrastination is necessary 
%  to stop the automatic creation of a second table of contents page
%  by latex2html.
\newcommand{\latexonlytoc}[0]{\tableofcontents}


% -----------------------------------------------------------------------------
%  Debugging.
%  =========
%  Remove % on the following to debug links in the HTML version using Latex.

% \newcommand{\hotlink}[2]{\fbox{\begin{tabular}[t]{@{}c@{}}#1\\\hline{\footnotesize #2}\end{tabular}}}
% \renewcommand{\htmladdnormallinkfoot}[2]{\hotlink{#1}{#2}}
% \renewcommand{\htmladdnormallink}[2]{\hotlink{#1}{#2}}
% \renewcommand{\hyperref}[4]{\hotlink{#1}{\S\ref{#4}}}
% \renewcommand{\htmlref}[2]{\hotlink{#1}{\S\ref{#2}}}
% \renewcommand{\xref}[3]{\hotlink{#1}{#2 -- #3}}
%end{latexonly}
% -----------------------------------------------------------------------------
% ? Document-specific \newcommand or \newenvironment commands.

\newcommand{\dqt}[1]{{\texttt{"#1"}}}
\newcommand{\hash}{\dqt{\#}}

% centre an asterisk
\newcommand{\lsk}{\raisebox{-0.4ex}{\rm *}}

\begin{htmlonly}
  \newcommand{\hash}{\dqt{#}}
\end{htmlonly}

% conditional text
\newcommand{\latexelsehtml}[2]{#1}
\begin{htmlonly}
  \newcommand{\latexelsehtml}[2]{#2}
\end{htmlonly}

% A kind of list item, like description, but with an easily adjustable
% item separation.  Note that the paragraph and fount-size change are
% needed to make the revised \baselinestretch work.
\newlength{\menuwidth}
\newlength{\menuindent}
\newcommand{\menuitem}[2]
  {{\bf #1} \settowidth{\menuwidth}{{\bf #1} }
  \setlength{\menuindent}{-0.5em}
  \addtolength{\menuwidth}{-2\menuwidth}
  \addtolength{\menuwidth}{\textwidth}
  \addtolength{\menuwidth}{\menuindent}
  \hspace{\menuindent}\parbox[t]{\menuwidth}{
  \newcommand{\baselinestretch}{0.75}\small
  #2 \par \vspace{1.0ex}
  \newcommand{\baselinestretch}{1.0}\normalsize} \\ }

\begin{htmlonly}
\newcommand{\menuitem}[2]
  {\item [\htmlref{#1}{#1}] #2}
\end{htmlonly}

\newcommand{\classitem}[1]{\item [\htmlref{#1}{#1}]}

% an environment for references (for the SST sstdiytopic command).
\newenvironment{refs}{\vspace{-4ex} % normally 3ex
                      \begin{list}{}{\setlength{\topsep}{0mm}
                                     \setlength{\partopsep}{0mm}
                                     \setlength{\itemsep}{0mm}
                                     \setlength{\parsep}{0mm}
                                     \setlength{\leftmargin}{1.5em}
                                     \setlength{\itemindent}{-\leftmargin}
                                     \setlength{\labelsep}{0mm}
                                     \setlength{\labelwidth}{0mm}}
                    }{\end{list}}

\begin{htmlonly}
% an environment for references (for the SST sstdiytopic command).
   \newenvironment{refs}{\vspace{-4ex} % normally 3ex
                         \begin{list}{}{\setlength{\topsep}{-2ex}
                                     \setlength{\partopsep}{0mm}
                                     \setlength{\itemsep}{0mm}
                                     \setlength{\parsep}{0mm}
                                     \setlength{\leftmargin}{1.5em}
                                     \setlength{\itemindent}{-\leftmargin}
                                     \setlength{\labelsep}{0mm}
                                     \setlength{\labelwidth}{0mm}}
                       }{\end{list}}
\end{htmlonly}

% Shorthands for hypertext links.
% -------------------------------
\newcommand{\CONVERT}{{\footnotesize CONVERT}}
\newcommand{\BCONVERT}{{\normalsize CONVERT}}
\newcommand{\CURSA}{{\footnotesize CURSA}}
\newcommand{\FIGARO}{{\footnotesize FIGARO}}
\newcommand{\Figaroref}{\xref{{\footnotesize FIGARO}}{sun86}{abstract}}
\newcommand{\FITSref}{\htmladdnormallink{FITS}{http://fits.gsfc.nasa.gov/}}
\newcommand{\GIFref}{\htmladdnormallink{GIF}{http://en.wikipedia.org/wiki/GIF}}
\newcommand{\HDSref}{\xref{HDS}{sun92}{}}
\newcommand{\ICL}{{\footnotesize ICL}}
\newcommand{\ICLref}{\xref{\ICL}{sg5}{}}
\newcommand{\IRAF}{{\footnotesize IRAF}}
%\newcommand{\IRAFref}{\htmladdnormallink{{\footnotesize IRAF}}{http://iraf.noao.edu/iraf-homepage.html}}
\newcommand{\IRAFURL}{http://star-www.rl.ac.uk/iraf/web/iraf-homepage.html}
\newcommand{\IRAFref}{\htmladdnormallink{\IRAF}{\IRAFURL}}
\newcommand{\KAPPA}{{\footnotesize KAPPA}}
\newcommand{\KAPPAref}{\xref{\KAPPA}{sun95}{}}
\newcommand{\IDLURL}{http://www.rsinc.com/idl/}
\newcommand{\IDLref}{\htmladdnormallink{IDL}{\IDLURL}}
\newcommand{\IDLAULURL}{http://idlastro.gsfc.nasa.gov/homepage.html}
\newcommand{\NDFref}{\xref{NDF}{sun33}{}}
\newcommand{\Netpbm}{{\footnotesize NETPBM}}
\newcommand{\Netpbmref}{\htmladdnormallink{\Netpbm}{http://netpbm.sourceforge.net/}}
\newcommand{\PBMPLUS}{{\footnotesize PBMPLUS}}
\newcommand{\PBMPLUSref}{\htmladdnormallink{\PBMPLUS}{http://www.acme.com/software/pbmplus/}}
\newcommand{\SPECDRE}{\xref{{\footnotesize SPECDRE}}{sun140}{}}
\newcommand{\TIFFref}{\htmladdnormallink{TIFF}{http://en.wikipedia.org/wiki/TIFF}}
%+
%  Name:
%     SST.TEX

%  Purpose:
%     Define LaTeX commands for laying out Starlink routine descriptions.

%  Language:
%     LaTeX

%  Type of Module:
%     LaTeX data file.

%  Description:
%     This file defines LaTeX commands which allow routine documentation
%     produced by the SST application PROLAT to be processed by LaTeX and
%     by LaTeX2html. The contents of this file should be included in the
%     source prior to any statements that make of the sst commands.

%  Notes:
%     The commands defined in the style file html.sty provided with LaTeX2html 
%     are used. These should either be made available by using the appropriate
%     sun.tex (with hypertext extensions) or by putting the file html.sty 
%     on your TEXINPUTS path (and including the name as part of the  
%     documentstyle declaration).

%  Authors:
%     RFWS: R.F. Warren-Smith (STARLINK)
%     PDRAPER: P.W. Draper (Starlink - Durham University)
%     MJC: Malcolm J. Currie (STARLINK)

%  History:
%     10-SEP-1990 (RFWS):
%        Original version.
%     10-SEP-1990 (RFWS):
%        Added the implementation status section.
%     12-SEP-1990 (RFWS):
%        Added support for the usage section and adjusted various spacings.
%     8-DEC-1994 (PDRAPER):
%        Added support for simplified formatting using LaTeX2html.
%     1995 October 4 (MJC):
%        Added goodbreaks and pagebreak[3] in various places to improve
%        pages breaking before headings, not immediately after.
%        Corrected banner width.
%     {enter_further_changes_here}

%  Bugs:
%     {note_any_bugs_here}

%-

%  Define length variables.
\newlength{\sstbannerlength}
\newlength{\sstcaptionlength}
\newlength{\sstexampleslength}
\newlength{\sstexampleswidth}

%  Define a \tt font of the required size.
\latex{\newfont{\ssttt}{cmtt10 scaled 1095}}
\html{\newcommand{\ssttt}{\large\tt}}

%  Define a command to produce a routine header, including its name,
%  a purpose description and the rest of the routine's documentation.
\newcommand{\sstroutine}[3]{
   \goodbreak
   \markboth{{\stardocname}~ --- #1}{{\stardocname}~ --- #1}
   \rule{\textwidth}{0.5mm}
   \vspace{-7ex}
   \newline
   \settowidth{\sstbannerlength}{{\Large {\bf #1}}}
   \setlength{\sstcaptionlength}{\textwidth}
   \setlength{\sstexampleslength}{\textwidth}
   \addtolength{\sstbannerlength}{0.5em}
   \addtolength{\sstcaptionlength}{-2.0\sstbannerlength}
   \addtolength{\sstcaptionlength}{-5.0pt}
   \settowidth{\sstexampleswidth}{{\bf Examples:}}
   \addtolength{\sstexampleslength}{-\sstexampleswidth}
   \parbox[t]{\sstbannerlength}{\flushleft{\Large {\bf #1}}}
   \parbox[t]{\sstcaptionlength}{\center{\Large #2}}
   \parbox[t]{\sstbannerlength}{\flushright{\Large {\bf #1}}}
   \begin{description}
      #3
   \end{description}
}

%  Format the description section.
\newcommand{\sstdescription}[1]{\item[Description:] #1}

%  Format the usage section.
%\newcommand{\sstusage}[1]{\pagebreak[3] \item[Usage:] \mbox{} \\[1.3ex] {\ssttt #1}}
\newcommand{\sstusage}[1]{\item[Usage:] \mbox{}
\\[1.3ex]{\raggedright \ssttt #1}}

%  Format the invocation section.
\newcommand{\sstinvocation}[1]{\sloppy \item[Invocation:]\hspace{0.4em}{\tt #1}}

%  Format the arguments section.
\newcommand{\sstarguments}[1]{
   \item[Arguments:] \mbox{} \\
   \vspace{-3.5ex}
   \begin{description}
      #1
   \end{description}
}

%  Format the keywords section.
\newcommand{\sstkeywords}[1]{
   \item[Keywords:] \mbox{} \\
   \vspace{-3.5ex}
   \begin{description}
      #1
   \end{description}
}

%  Format the returned value section (for a function).
\newcommand{\sstreturnedvalue}[1]{
   \item[Returned Value:] \mbox{} \\
   \vspace{-3.5ex}
   \begin{description}
      #1
   \end{description}
}

%  Format the parameters section (for an application).
\newcommand{\sstparameters}[1]{
   \goodbreak 
   \item[Parameters:] \mbox{} \\
   \vspace{-3.5ex}
   \begin{description}
      #1
   \end{description}
}

%  Format the output results parameters section (for an application).
\newcommand{\sstresparameters}[1]{
   \goodbreak 
   \item[Results Parameters:] \mbox{} \\
   \vspace{-3.5ex}
   \begin{description}
      #1
   \end{description}
}

%  Format the examples section.
\newcommand{\sstexamples}[1]{
   \goodbreak
   \item[Examples:] \mbox{} \\
   \vspace{-3.5ex}
   \begin{description}
      #1
   \end{description}
}

%  Define the format of a subsection in a normal section.
\newcommand{\sstsubsection}[1]{ \item[{#1}] \mbox{} \\}

%  Define the format of a subsection in the examples section.
\newcommand{\sstexamplesubsection}[2]{\sloppy
\item[\parbox{\sstexampleslength}{\ssttt #1}] \mbox{} \vspace{0.5ex}
\\ #2 \vspace{1.0ex}}

%  Format the notes section.
\newcommand{\sstnotes}[1]{\pagebreak[3] \item[Notes:] \mbox{} \\[1.3ex] #1}

%  Provide a general-purpose format for additional (DIY) sections.
\newcommand{\sstdiytopic}[2]{\goodbreak \item[{\hspace{-0.35em}#1\hspace{-0.35em}:}] \mbox{} \\[1.3ex] #2}

%  Format the implementation status section.
\newcommand{\sstimplementationstatus}[1]{
   \pagebreak[3] \item[{Implementation Status:}] \mbox{} \\[1.3ex] #1}

%  Format the bugs section.
\newcommand{\sstbugs}[1]{\item[Bugs:] #1}

%  Specify a variant of the itemize environment where the top separation
%  is reduced.  It is needed because a \vspace is ignored in the
%  \sstitemlist command.
\newenvironment{sstitemize}{%
  \vspace{-4.3ex}\begin{itemize}}{\end{itemize}}

%  Format a list of items while in paragraph mode.
\newcommand{\sstitemlist}[1]{
  \mbox{} \\
  \vspace{-3.5ex}
  \begin{sstitemize}
     #1
  \end{sstitemize}
}

%  Format a list of items while in paragraph mode, and where there
%  is a heading, thus the negative vertical space is not needed.
\newcommand{\ssthitemlist}[1]{
  \latex{
  \mbox{} \\
  \vspace{-3.5ex}
  }
  \begin{itemize}
     #1
  \end{itemize}
}

%  Define the format of an item.
\newcommand{\sstitem}{\item}

% Now define html equivalents of those already set. These are used by
%  latex2html and are defined in the html.sty files.
\begin{htmlonly}

%  sstroutine.
   \newcommand{\sstroutine}[3]{
      \subsection{#1\xlabel{#1}-\label{#1}#2}
      \begin{description}
         #3
      \end{description}
   }

%  sstdescription
   \newcommand{\sstdescription}[1]{\item[Description:]
      \begin{description}
         #1
      \end{description}
      \\
   }

%  sstusage
   \newcommand{\sstusage}[1]{\item[\htmlref{Usage:}{app_usage}]
      \begin{description}
         {\ssttt #1}
      \end{description}
      \\
   }

%  sstinvocation
   \newcommand{\sstinvocation}[1]{\item[Invocation:]
      \begin{description}
         {\ssttt #1}
      \end{description}
      \\
   }

%  sstarguments
   \newcommand{\sstarguments}[1]{
      \item[Arguments:] \\
      \begin{description}
         #1
      \end{description}
      \\
   }

%  sstkeywords
   \newcommand{\sstkeywords}[1]{
      \item[Keywords:] \\
      \begin{description}
         #1
      \end{description}
      \\
   }

%  sstreturnedvalue
   \newcommand{\sstreturnedvalue}[1]{
      \item[Returned Value:] \\
      \begin{description}
         #1
      \end{description}
      \\
   }

%  sstparameters
   \newcommand{\sstparameters}[1]{
      \item[\xref{Parameters:}{sun95}{se_param}] \\
      \begin{description}
         #1
      \end{description}
   }


%  sstresparameters
   \newcommand{\sstresparameters}[1]{
      \item[\\ \xref{Results Parameters:}{sun95}{se_parout}] \\
      \begin{description}
         #1
      \end{description}
   }

%  sstexamples
   \newcommand{\sstexamples}[1]{
   \item[\vspace{0.35ex}\htmlref{Examples:\vspace{-0.5ex}}{app_example}]
      \begin{description}
         #1
      \end{description}
      \\
   }

%  sstsubsection
   \newcommand{\sstsubsection}[1]{\item[{#1}]}

%  sstexamplesubsection
   \newcommand{\sstexamplesubsection}[2]{
   \vspace{-1.0ex} \item[{\ssttt #1}] #2 \vspace{0.2ex}}

%  sstnotes
   \newcommand{\sstnotes}[1]{\item[Notes:]
      \begin{description}
         #1
      \end{description}
   }

%  sstdiytopic
   \newcommand{\sstdiytopic}[2]{\\ \item[{#1}:]
      \begin{description}
         #2
      \end{description}
   }

%  sstimplementationstatus
   \newcommand{\sstimplementationstatus}[1]{\\ \item[Implementation Status:]
      \begin{description}
         #1
      \end{description}
   }

%  sstitemlist
   \newcommand{\sstitemlist}[1]{
      \begin{itemize}
         #1
      \end{itemize}
   }

%  sstitem
   \newcommand{\sstitem}{\item}

\end{htmlonly}

%  End of "sst.tex" layout definitions.
%.
% ? End of document specific commands
% -----------------------------------------------------------------------------
%  Title Page.
%  ===========
\renewcommand{\thepage}{\roman{page}}
\begin{document}
\thispagestyle{empty}

%  Latex document header.
%  ======================
\begin{latexonly}
   CCLRC / \textsc{Rutherford Appleton Laboratory} \hfill \textbf{\stardocname}\\
   {\large Particle Physics \& Astronomy Research Council}\\
   {\large Starlink Project\\}
   {\large \stardoccategory\ \stardocnumber}
   \begin{flushright}
   \stardocauthors\\
   \stardocdate
   \end{flushright}
   \vspace{-4mm}
   \rule{\textwidth}{0.5mm}
   \vspace{5mm}
   \begin{center}
   {\Huge\textbf{\stardoctitle \\ [2.5ex]}}
   {\LARGE\textbf{\stardocversion \\ [4ex]}}
   {\Huge\textbf{\stardocmanual}}
   \end{center}
   \vspace{5mm}

% ? Add picture here if required for the LaTeX version.
%   e.g. \includegraphics[scale=0.3]{filename.ps}
% ? End of picture

% ? Heading for abstract if used.
   \vspace{10mm}
   \begin{center}
      {\Large\textbf{Abstract}}
   \end{center}
% ? End of heading for abstract.
\end{latexonly}

%  HTML documentation header.
%  ==========================
\begin{htmlonly}
   \xlabel{}
   \begin{rawhtml} <H1> \end{rawhtml}
      \stardoctitle\\
      \stardocversion\\
      \stardocmanual
   \begin{rawhtml} </H1> <HR> \end{rawhtml}

% ? Add picture here if required for the hypertext version.
%   e.g. \includegraphics[scale=0.7]{filename.ps}
% ? End of picture

   \begin{rawhtml} <P> <I> \end{rawhtml}
   \stardoccategory\ \stardocnumber \\
   \stardocauthors \\
   \stardocdate
   \begin{rawhtml} </I> </P> <H3> \end{rawhtml}
      \htmladdnormallink{CCLRC / Rutherford Appleton Laboratory}
                        {http://www.cclrc.ac.uk} \\
      \htmladdnormallink{Particle Physics \& Astronomy Research Council}
                        {http://www.pparc.ac.uk} \\
   \begin{rawhtml} </H3> <H2> \end{rawhtml}
      \htmladdnormallink{Starlink Project}{http://www.starlink.rl.ac.uk/}
   \begin{rawhtml} </H2> \end{rawhtml}
   \htmladdnormallink{\htmladdimg{source.gif} Retrieve hardcopy}
      {http://www.starlink.rl.ac.uk/cgi-bin/hcserver?\stardocsource}\\

%  HTML document table of contents. 
%  ================================
%  Add table of contents header and a navigation button to return to this 
%  point in the document (this should always go before the abstract \section). 
  \label{stardoccontents}
  \begin{rawhtml} 
    <HR>
    <H2>Contents</H2>
  \end{rawhtml}
  \htmladdtonavigation{\htmlref{\htmladdimg{contents_motif.gif}}
        {stardoccontents}}

% ? New section for abstract if used.
  \section{\xlabel{abstract}Abstract}
% ? End of new section for abstract
\end{htmlonly}

% -----------------------------------------------------------------------------
% ? Document Abstract. (if used)
%  ==================
\stardocabstract
% ? End of document abstract

% -----------------------------------------------------------------------------
% ? Latex Copyright Statement
%  =========================
\begin{latexonly}
\newpage
\vspace*{\fill}
\stardoccopyright
\end{latexonly}
% ? End of Latex copyright statement

% -----------------------------------------------------------------------------
% ? Latex document Table of Contents (if used).
%  ===========================================
  \newpage
  \begin{latexonly}
    \setlength{\parskip}{0mm}
    \tableofcontents
    \setlength{\parskip}{\medskipamount}
    \markboth{\stardocname}{\stardocname}
  \end{latexonly}
% ? End of Latex document table of contents
% -----------------------------------------------------------------------------

\cleardoublepage
\renewcommand{\thepage}{\arabic{page}}
\setcounter{page}{1}

\section{\label{introduction}Introduction}

If life were simple there would only be one data format, but in reality
there are numerous formats for storing {\em{n}}-dimensional astronomical data
associated with various software packages.   In Starlink we have not
been immune to this, having the original INTERIM BDF, HDS IMAGE format,
and \Figaroref\ DSTs to name but three.  However, Starlink is now taking the
novel approach of supporting different packages sharing a common data
format---the
\xref{NDF}{sun33}{abstract}\latex{\footnote{
See SUN/33 for an introduction to the NDF.}}
(Extensible {\em{n}}-Dimensional-data format)---which most Starlink packages
are already using.

The purpose of \CONVERT\ is the interchange of data files
to and from the NDF.  Thus it enables astronomers to select the best
applications from a variety of packages, including those originating
abroad like 
\IRAFref.
In addition it assists packages that wish to move to
using the NDF.

\CONVERT\ is available from all three UNIX platforms supported by
Starlink.  The supported conversions are currently as follows: 

%\begin{latexonly}
\begin{center}
\begin{small}
\begin{tabular}{l@{ -- }p{125mm}}
{\bf \htmlref{ASCII2NDF}{ASCII2NDF}}
& Converts a text file to an NDF. \\[\medskipamount]
{\bf \htmlref{AST2NDF}{AST2NDF}}
& Converts an Asterix data cube into a standard NDF. \\[\medskipamount]
{\bf \htmlref{DA2NDF}{DA2NDF}} 
& Converts a direct-access unformatted file to an NDF. \\[\medskipamount]
{\bf \htmlref{DST2NDF}{DST2NDF}}
& Converts a Figaro (Version 2) DST file to an NDF. \\[\medskipamount]
{\bf \htmlref{FITS2NDF}{FITS2NDF}}
& Converts FITS files into NDFs. \\[\medskipamount]
{\bf \htmlref{GASP2NDF}{GASP2NDF}}
& Converts an image in GASP format to an NDF. \\[\medskipamount]
{\bf \htmlref{GIF2NDF}{GIF2NDF}}
& Converts an image in GIF format to an NDF. \\[\medskipamount]
{\bf \htmlref{IRAF2NDF}{IRAF2NDF}}
& Converts an IRAF image to an NDF. \\[\medskipamount]
{\bf \htmlref{IRCAM2NDF}{IRCAM2NDF}}
& Converts an IRCAM data file to a series of NDFs. \\[\medskipamount]
{\bf \htmlref{MTFITS2NDF}{MTFITS2NDF}}
& Converts a FITS tape to one or more NDFs. \\[\medskipamount]
{\bf \htmlref{NDF2ASCII}{NDF2ASCII}}
& Converts an NDF to a text file. \\[\medskipamount]
{\bf \htmlref{NDF2DA}{NDF2DA}}
& Converts an NDF to a direct-access unformatted file. \\[\medskipamount]
{\bf \htmlref{NDF2DST}{NDF2DST}}
& Converts an NDF to a Figaro (Version 2) DST file. \\[\medskipamount]
{\bf \htmlref{NDF2FITS}{NDF2FITS}}
& Converts an NDF to a FITS file. \\[\medskipamount]
{\bf \htmlref{NDF2GASP}{NDF2GASP}}
& Converts a two-dimensional NDF into a GASP image. \\[\medskipamount]
{\bf \htmlref{NDF2GIF}{NDF2GIF}}
& Converts an NDF to a GIF file. \\[\medskipamount]
{\bf \htmlref{NDF2IRAF}{NDF2IRAF}}
& Converts an NDF to an IRAF image. \\[\medskipamount]
{\bf \htmlref{NDF2PGM}{NDF2PGM}}
& Converts an NDF to a PGM format. \\[\medskipamount]
{\bf \htmlref{NDF2TIFF}{NDF2TIFF}}
& Converts an NDF to a TIFF file. \\[\medskipamount]
{\bf \htmlref{NDF2UNF}{NDF2UNF}}
& Converts an NDF to a sequential unformatted file. \\[\medskipamount]
{\bf \htmlref{SPECX2NDF}{SPECX2NDF}}
& Converts a SPECX map into a simple data cube, or SPECX data files to individual spectra. \\[\medskipamount]
{\bf \htmlref{TIFF2NDF}{TIFF2NDF}}
& Converts an image in TIFF format to an NDF. \\[\medskipamount]
{\bf \htmlref{UNF2NDF}{UNF2NDF}}
& Converts a sequential unformatted file to an NDF. \\[\medskipamount]
\end{tabular}
\end{small}
\end{center}

In addition there are FITS readers within 
\KAPPAref \latex{ (see SUN/95)}
which will convert FITS files and tapes to NDFs.
These are more tolerant of `almost FITS' files, but lack support for
IMAGE and BINTABLE extensions.

\IDLref\
procedures for handling NDFs and other methods of converting 
\htmlref{NDFs to IDL}{app_idl}
format are described in 
\latex{Appendix~\ref{app_idl} of} this document.

Starting up the \CONVERT\ package will also set up defaults for the
\htmlref{automatic NDF conversion facilities}{sect_auto}
\latex{(described in Section \ref{sect_auto})} to enable applications 
which use the NDF library to read and write most of the file formats handled 
by the \CONVERT\ package, and some others.

The various formats supported by \CONVERT\ do not have
one-to-one correspondence and therefore in general it is not possible to
apply a forward and reverse conversion and finish with a duplicate of
the initial data file.  This hysteresis is particularly likely when
starting with an NDF, since many simpler formats have no way of storing
certain NDF data items, like variance and axis widths.  However, if you
are dealing with a simple file containing just a data array and linear
axis centres, then it should be possible to avoid loss of information except
with \GIFref\ and \TIFFref\ formats which will reduce the absolute data values to 256
greyscale levels.

{\sl Note -- the input data file is not deleted or altered in any way.}

\newpage 
\section[Running {\small \bf CONVERT}]{\label{running_convert}Running \BCONVERT}

\subsection[Starting {\small CONVERT} from the UNIX shell]
{\xlabel{starting_convert_from_the_unix_shell}
Starting \BCONVERT\ from the UNIX shell}

The command \texttt{convert} defines \CONVERT\ commands from
the UNIX shell.

\small
\begin{verbatim}
      % convert
\end{verbatim}
\normalsize 
Note that the \texttt{\%} is the UNIX shell's prompt which you do not type.

A message similar to:
\small
\begin{verbatim}

   CONVERT commands are now available -- (Version 1.0, 1997 August)

   Defaults for automatic NDF conversion are set.

   Type conhelp for help on CONVERT commands.
   Type "showme sun55" to browse the hypertext documentation.

\end{verbatim}
\normalsize 
will be displayed. You will then be able to mix \CONVERT\ and UNIX commands.

The \texttt{convert} command is defined by the Starlink startup procedures to
`source' file \texttt{convert.csh} in the \CONVERT\ executables directory. 
Non-Starlink sites must make their own arrangements.

\subsection[Starting {\small CONVERT} from {\small \bf ICL}]
{\xlabel{starting_convert_from_icl}
Starting \BCONVERT\ from ICL}
To start \ICLref, type:
\small
\begin{verbatim}
     % icl
\end{verbatim}
\normalsize
You will see any messages produced by system and user procedures, followed
by the \texttt{ICL>} prompt, something like the following.

\small
\begin{quote} \begin{verbatim}
ICL (UNIX) Version 3.1-5 20/05/97

Loading installed package definitions...

  - Type HELP package_name for help on specific Starlink packages
  -   or HELP PACKAGES for a list of all Starlink packages
  - Type HELP [command] for help on ICL and its commands

ICL>
\end{verbatim} \end{quote}
\normalsize
Then, to make the \CONVERT\ commands known to the command language, type:
\small
\begin{quote} \begin{verbatim}
ICL> CONVERT
\end{verbatim} \end{quote}
\normalsize
This will produce a \CONVERT\ startup message similar to:
\small
\begin{quote} \begin{verbatim}
CONVERT commands are now available -- (Version 1.0, 1997 August)

Defaults for automatic NDF conversion are set.

Type CONHELP or HELP CONVERT for help on CONVERT commands.
Type "showme sun55" to browse the hypertext documentation.
\end{verbatim} \end{quote}
\normalsize

The ICL command \texttt{CONVERT} is defined by the standard Starlink ICL login
files to \texttt{LOAD} file \texttt{convert.icl} in the \CONVERT\ executables 
directory. Non-Starlink sites must make their own arrangements.

\subsection[Issuing {\small CONVERT} Commands]
{\xlabel{issuing_convert_commands}
Issuing \BCONVERT\ Commands}
Having initialised \CONVERT\ you are now ready to issue a
\CONVERT\ command. To run an application you can just give
its name (or its name preceded by \texttt{con\_}\footnote{The
\texttt{con\_<name>}
form is defined for use where there may be confusion between commands of the
same name from different packages.})---you will be prompted for any required 
{\em parameters}. 
Alternatively, you may enter parameter values on the command line
specified by position or by keyword.  If you want to override any
defaulted parameters, then you specify the parameter's value on the
command line.  Note that {\em from UNIX the commands are in lowercase}, whereas
from \ICL\ the case does not matter.

Most \CONVERT\ applications can be run as simply as:

\small
\begin{verbatim}
     <application> <in> <out>
\end{verbatim}
\normalsize
where \texttt{$<$application$>$} is the application's name, \texttt{$<$in$>$}
is the input file, and \texttt{$<$out$>$} is the output
file following the conversion.  For instance, from the UNIX shell,

\small
\begin{verbatim}
     % dst2ndf old new
\end{verbatim}
\normalsize
or, from ICL,

\small
\begin{verbatim}
     ICL> DST2NDF old new
\end{verbatim}
\normalsize
both instruct the application \htmlref{DST2NDF}{DST2NDF} to convert the
DST file called \texttt{old.dst} to the NDF called \texttt{new.sdf}.  
Note that for UNIX, the case of the filename is significant.

The following example has the same effect as those immediately above,
only this time you are prompted for the filenames needed by DST2NDF. 

\small
\begin{verbatim}
     ICL> DST2NDF
     IN - Name of Figaro (.DST) file to be converted /' '/ > old
     OUT - Name of output NDF /@f1/ > new
\end{verbatim}
\normalsize
The value between the \texttt{/ /} delimiters is a suggested default.  You
can choose to accept the suggestion by pressing carriage return. 

The simple usage, (\texttt{<application> <in> <out>}), will usually produce a
result but many applications have additional parameters which you can set to
give finer control over the conversion. See the 
\htmlref{application specifications}{app_full}
\latex{in Appendix~\ref{app_full}} for details of the options available.

You can find details of how to use parameters for controlling Starlink program
options
\latexelsehtml{in Section~4 of SUN/95}{\xref{in SUN/95}{sun95}{se_param}}
or in
\latexelsehtml{Chapter~8}{section `Specifying Parameter Values'}
of 
\xref{SG/4}{sg4}{}.
However, you should be able to get along using intuition alone, or, perhaps 
by consulting the
\htmlref{application specifications}{app_full} 
\latex{in Appendix~\ref{app_full}}, which include usage, parameters,
examples and details of the conversion process.

In most cases, one invocation of a \CONVERT\ application is required for each 
file conversion but in some cases, inputs may be defined as 
\xref{`GROUPS'}{sun150}{} of names, including wildcards (see the 
\htmlref{application specifications}{app_full} for details).

\subsection{\xlabel{obtaining_help}Obtaining Help}
You can get the top-level help information for \CONVERT\  by typing:

\small
\begin{verbatim}
     % conhelp
\end{verbatim}
\normalsize
from the UNIX shell, or:
\small
\begin{verbatim}
     ICL> CONHELP
\end{verbatim}
\normalsize
from ICL. (You can also access \CONVERT\ help from ICL by using the ICL command,
HELP.)

The help topics are mostly detailed descriptions of the applications
but also include global information on matters such as using parameters. 
{\em{e.g.}}\ the following command gives help on the application
\htmlref{UNF2NDF}{UNF2NDF}.

\small
\begin{verbatim}
     % conhelp unf2ndf
\end{verbatim}
\normalsize

If you have commenced running an application you can still access the
help library whenever you are prompted for a parameter; you enter \texttt{?}.
Here is an example.

\small
\begin{verbatim}
     NOPEREC - Number of data values per output record /512/ > ?

     NDF2UNF

       Parameters

         NOPEREC = _INTEGER (Read)
            The number of data values per record of the output file.  It
            should be in the range 1 to 8191, unless the array is double
            precision, when the upper limit is 4095.  The suggested
            default is the current value. [The first dimension of the NDF]

     NOPEREC - Number of data values per output record /512/ > 
\end{verbatim}
\normalsize

\subsection{\xlabel{sun55_se_hypertext}Hypertext Help\label{se_hypertext}}

A modified version of this document exists in hypertext form.  One way
to access it is to use the
\xref{\texttt{showme}}{sun188}{displaying_parts_of_documents} command

\small
\begin{verbatim}
     % showme sun55
\end{verbatim}
\normalsize
and a Web browser will appear, presenting the index to the hypertext
form of this document.  The hypertext permits easy location of
referenced documents and applications.

\newpage 
\section{\label{sect_auto}\xlabel{sect_auto}Automatic Format Conversion with
the NDF Library}
\xref{SSN/20}{ssn20}{} describes a system incorporated into the 
\xref{NDF library}{sun33}{}
routines which enables applications written to read or write NDFs to handle
any arbitrary `foreign' format for which a conversion utility can be defined.
The system operates via environment variables which define the set of permitted 
conversions and the commands required to do them.

\subsection{\xlabel{the_default_conversion_commands}The Default Conversion Commands}
\CONVERT\ startup will define defaults for the NDF-conversion environment 
variables which permit automatic conversion of files in the formats handled by 
\CONVERT\ (except for AST, IRCAM, PGM. SPECX and FITS tapes).
It also allows data compression. 

\begin{htmlonly}
The list of format names and associated filename extensions defined by 
\CONVERT\ is set out in the table below---the filename extensions tell the 
system which format the file is in.
\end{htmlonly}
\begin{latexonly}
The list of format names and associated filename extensions
defined by \CONVERT\ is set out in Table~\ref{tab:formats}---the filename 
extensions tell the system which format the file is in. For the unformatted 
and ASCII conversions the format names and extensions are somewhat arbitrary.
\end{latexonly}
This list will nullify any existing list so private conversions must be added
{\em after\/} \CONVERT\ startup.

For the unformatted and ASCII conversions the format names and
extensions are somewhat arbitrary.  The FITS and STREAM formats have
synonym file extensions for the conversion to NDF.  The standard file
extension is required for the conversion to the foreign format.

\begin{table}[hb]
\begin{center}
\begin{tabular}{|llp{32mm}l|}
\hline
Format & Extension & Extension & Description \\
       &           & Synonyms  & \\
\hline
FITS        & .fit & .fits .fts .FITS .FIT
                     .lilo .lihi .silo .sihi
                     .mxlo .mxhi .rilo .rihi .vdlo .vdhi & FITS \\
FIGARO      & .dst & & Figaro (Version 2) DST \\
IRAF        & .imh & & IRAF \\
STREAM      & .das & .str & Unformatted direct-access or stream \\
UNFORMATTED & .unf & & Unformatted with FITS header \\
UNF0        & .dat & & Unformatted without FITS header \\
ASCII       & .asc & & ASCII with FITS header \\
TEXT        & .txt & & ASCII without FITS header \\
GIF         & .gif & & Graphics Interchange Format \\
TIFF        & .tif & & Tag Image File Format \\
GASP        & .hdr & & GASP \\
COMPRESSED  & .sdf.Z & & Compressed NDF \\
GZIP        & .sdf.gz & & {\bf gzip} compressed NDF \\
\hline
\end{tabular}
\caption{\label{tab:formats}Defined Formats and Extensions}
\end{center}
\end{table}

\begin{htmlonly}
The table below lists the utilities used to perform the conversions.
In general the default parameter values are used---non-default parameters 
(other than the input and output filenames) are listed in the table.
\end{htmlonly}
\begin{latexonly}
Table \ref{tab:conversions} lists the utilities used to perform the conversions.
In general the default parameter values are used---non-default parameters 
(other than the input and output filenames) are listed in the table.
\end{latexonly}

\begin{table}[ht]
\begin{center}
\begin{tabular}{|llllc|}
\hline
FORMAT & In/out & Utility & Non-default parameters & Variable \\
\hline
FITS & in & \htmlref{FITS2NDF}{FITS2NDF} & & \\
& out & \htmlref{NDF2FITS}{NDF2FITS} & bitpix=-1 proexts=t & \\
FIGARO & in & \htmlref{DST2NDF}{DST2NDF} & & \\
& out & \htmlref{NDF2DST}{NDF2DST} & & $\times$ \\
IRAF & in & \htmlref{IRAF2NDF}{IRAFNDF} & & \\
& out & \htmlref{NDF2IRAF}{NDF2IRAF} & & \\
STREAM & in & \htmlref{DA2NDF}{DA2NDF} & noperec=! & \\
& out & \htmlref{NDF2DA}{NDF2DA} & & \\
UNFORMATTED & in & \htmlref{UNF2NDF}{UNF2NDF} & fits=t noperec=! & \\
& out & \htmlref{NDF2UNF}{NDF2UNF} & fits=t & \\
UNF0 & in & \htmlref{UNF2NDF}{UNF2NDF} & fits=f noperec=! &  $\surd$ \\
& out & \htmlref{NDF2UNF}{NDF2UNF} & fits=f & \\
ASCII & in & \htmlref{ASCII2NDF}{ASCII2NDF} & fits=t & \\
& out & \htmlref{NDF2ASCII}{NDF2ASCII} & fits=t reclen=80 & \\
TEXT & in & \htmlref{ASCII2NDF}{ASCII2NDF} & fits=f &  $\surd$ \\
& out & \htmlref{NDF2ASCII}{ASCII2NDF} & fits=f reclen=80 & \\
GIF & in & \htmlref{GIF2NDF}{GIF2NDF} & & $\times$ \\
& out & \htmlref{NDF2GIF}{NDF2GIF} & & $\times$ \\
TIFF & in & \htmlref{TIFF2NDF}{TIFF2NDF} & & $\times$ \\
& out & \htmlref{NDF2TIFF}{NDF2TIFF} & & $\times$ \\
GASP & in & \htmlref{GASP2NDF}{GASP2NDF} & & \\
& out & \htmlref{NDF2GASP}{NDF2GASP} & fillbad=0 & \\
COMPRESSED & in & {\bf uncompress} & & $\times$ \\
& out & {\bf compress} & & $\times$ \\
GZIP & in & {\bf gzip} & & $\times$ \\
& out & {\bf gunzip} & & $\times$ \\
\hline
\end{tabular}
\caption{\label{tab:conversions}Conversion Commands.}
\end{center}
\end{table}

\begin{latexonly}
Table~\ref{tab:conversions} also contains a column headed `Variable'.
Most of the command lines issued to do the automatic conversion will include 
the translation of an environment variable named 
NDF\_FROM\_\-{\em fmt}\_\-PARS or NDF\_TO\_{\em fmt}\_PARS as appropriate
(where {\em fmt} is the format name).
This may be used to give additional parameters to the command if you do not
want to define a completely new command for yourself. 
\end{latexonly}
\begin{htmlonly}
The table also contains a column headed `Variable'.
Most of the command lines issued to do the automatic conversion will include 
the translation of an environment variable named NDF\_FROM\_{\em fmt}\_PARS or 
NDF\_TO\_{\em fmt}\_PARS as appropriate (where {\em fmt} is the format name).
This may be used to give additional parameters to the command if you do not
want to define a completely new command for yourself. 
\end{htmlonly}
Where the `Variable' column contains a tick, the variable {\em must} be used 
to supply the SHAPE parameter; where it contains a cross, additional parameters
cannot be specified.

For example, suppose application \texttt{rdndf} uses the NDF library to read
one NDF (named by the first parameter) and write another (named by the second
parameter). This application could be made to read a TEXT file 
(\texttt{data.txt})
containing the required values for a 50 $\times$ 10 data array, and write its 
results as a FITS file (\texttt{output.fit}) as follows:

\small
\begin{quote} \begin{verbatim}
% convert
CONVERT commands are now available -- (Version 1.4, 2001 November)

Defaults for automatic NDF conversion are set.

Type conhelp for help on CONVERT commands.
Type "showme sun55" to browse the hypertext documentation.

% setenv NDF_FROM_TEXT_PARS 'SHAPE=[50,10]'
% rdndf data.txt output.fit
\end{verbatim} \end{quote}
\normalsize
\bigskip

The order of the formats in the tables also defines a search path.  If
you omit the file extension, the system will search for an NDF of that
name.  If that is absent, it will try a \texttt{.fit} FITS file.  If neither
are present it tries an \IRAFref\ file, and so on. The recognised formats and
their order is defined through the environment variable
NDF\_FORMATS\_IN.  The shell \texttt{convert} startup defines
NDF\_FORMATS\_IN as given below.

\small
\begin{verbatim}
     'FITS(.fit),FIGARO(.dst),IRAF(.imh),STREAM(.das),UNFORMATTED(.unf),
     UNF0(.dat),ASCII(.asc),TEXT(.txt),GIF(.gif),TIFF(.tif),GASP(.hdr),
     COMPRESSED(.sdf.Z),GZIP(.sdf.gz),FITS(.fits),FITS(.fts),FITS(.FTS),
     FITS(.FITS),FITS(.FIT),FITS(.lilo),FITS(.lihi),
     FITS(.silo),FITS(.sihi),FITS(.mxlo),FITS(.mxhi),
     FITS(.rilo),FITS(.rihi),FITS(.vdlo),FITS(.vdhi),STREAM(.str)'
\end{verbatim}
\normalsize

but from ICL the \texttt{CONVERT} command does not define the synonyms due 
to a limitation of ICL.  Thus NDF\_FORMATS\_IN is defined to be the
following. 

\small
\begin{verbatim}
     'FITS(.fit),FIGARO(.dst),IRAF(.imh),STREAM(.das),UNFORMATTED(.unf),
      UNF0(.dat),ASCII(.asc),TEXT(.txt),GIF(.gif),TIFF(.tif),GASP(.hdr),
      COMPRESSED(.sdf.Z),GZIP(.sdf.gz)'
\end{verbatim}
\normalsize

When creating an output file, there is a similar list of recognised
formats.  The \CONVERT\ startup procedures define NDF\_FORMATS\_OUT
as follows.

\small
\begin{verbatim}
     '.,FITS(.fit),FIGARO(.dst),IRAF(.imh),STREAM(.das),UNFORMATTED(.unf),
     UNF0(.dat),ASCII(.asc),TEXT(.txt),GIF(.gif),TIFF(.tif),GASP(.hdr),
     COMPRESSED(.sdf.Z),GZIP(.sdf.gz)'
\end{verbatim}
\normalsize

The leading dot indicates that if you omit the file extension, the
output file will be an NDF.

There are some examples of the automatic system in action and use of
NDF\_FORMATS\_IN and NDF\_FORMATS\_OUT in
\latexelsehtml{SUN/95, Section~15.1.}{\xref{Automatic
Conversion.}{sun95}{se_autoconvert}}

\section{Acknowledgments}
Jo Murray wrote the original versions of the applications that convert
between DSTs and NDFs.  Alan Chipperfield produced the IDL converters.
Rhys Morris wrote the original versions of IRAF2NDF, NDF2IRAF, GASP2NDF
and NDF2GASP.  Grant Privett wrote the TIFF and contributed to the GIF
conversion utilities.  Clive Davenhall wrote AST2NDF and SPECX2NDF;
the latter was substantially revised by David Berry to work with the
AST SpecFrame.

Rodney Warren-Smith devised the format-conversion facilities for the NDF 
data-access library and Alan Chipperfield implemented the \CONVERT\ components
of it.

\newpage
\appendix

\section[Specifications of {\small \bf CONVERT} Applications]
{\label{app_full}Specifications of \BCONVERT\ Applications}
\subsection{\xlabel{explanatory_notes}Explanatory Notes}

The {\em Parameters\/} section \label{app_parameters}
lists the application's parameters, with the format:

\begin{verbatim}
     name  =  type (access)
        description
\end{verbatim}

The description entry has a notation scheme to indicate 
normally defaulted parameters, {\it i.e.}\ those for which there will
be no prompt.
For such parameters a matching pair of square brackets (\verb![]!)
terminates the description.  The content between the brackets mean
\begin{description}
\item[\texttt{[]}]
Empty brackets means that the default is created dynamically
by the application, and may depend on the values of other parameters.
Therefore, the default cannot be given explicitly.
\item[\texttt{[,]}]
As above, but there are two default values that are created dynamically.
\item[\texttt{[}{\rm default}\texttt{]}]
Occasionally, a description of the default is given in normal type.
\item[\texttt{[default]}]
If the brackets contain a value in teletype-fount, this is the explicit
default value.
\end{description}

There is also a {\em Usage\/} entry.   \label{app_usage}
This shows how the  application is invoked from the command line.   It
lists the positional parameters in order followed by any prompted
keyword parameters using  a \mbox{``KEYWORD=?''} syntax.  Defaulted
keyword parameters do not appear.  Positional parameters that are
normally defaulted are indicated by being enclosed in square brackets.  
Keyword ({\it i.e.}\ not positional) parameters are needed where the
number of parameters are large, and usually occur because they depend on
the value of another parameter.  An example should clarify.
\smallskip

\begin{verbatim}
     ndf2ascii in out [comp] [reclen] noperec=?
\end{verbatim}
\normalsize
\smallskip

IN, OUT, COMP, and RECLEN are all positional
parameters.  Only IN, and OUT would be prompted if not given
on the command line. The remaining parameter, NOPEREC, depends on the
value of another parameter (it is FIXED), and will be prompted for when
FIXED is \texttt{TRUE}. 

The {\em Examples\/} section  \label{app_example}
shows how to run the application from the command line.  More often
you'll enter the command name and just some of the parameters, and be
prompted for the rest. 

Examples give command lines as accepted by the tasks themselves.  From
the UNIX shell, {\em metacharacters\/} (notably \texttt{[}, \texttt{]} and
\texttt{"})
{\em must be escaped or enclosed in single quotes}.  For example:

\begin{quote} \begin{verbatim}
ascii2ndf ngc253q.dat ngc253 q shape='[100,60]'
fits2ndf '"abc.fit,def.fits"' 'fgh,ijk"' fmtcnv='"F,T"' noproexts
\end{verbatim} \end{quote}
\normalsize

\newpage
\sstroutine{
   ASCII2NDF
}{
   Converts a text file to an NDF
}{
   \sstdescription{
      This application converts a text file to an \NDFref.  Only one of
      the array components may be created from the input file.
      Preceding the input data there may be an optional header.  This
      header may be skipped, or may consist of a simple \FITSref\ header.
      In the former case the shape of the NDF has be to be supplied.
   }
   \sstusage{
      ascii2ndf in out [comp] [skip] shape [type]
   }
   \sstparameters{
      \sstsubsection{
         COMP = \xref{LITERAL}{sun95}{se_parmenu} (Read)
      }{
         The NDF component to be copied.  It may be \texttt{"Data"},
         \texttt{"Quality"} or \texttt{"Variance"}.  To create a variance or
         quality array the NDF must already exist. \texttt{["Data"]}
      }
      \sstsubsection{
         FITS = \_LOGICAL (Read)
      }{
         If \texttt{TRUE}, the initial records of the formatted file are
         interpreted as a FITS header (with one card image per record)
         from which the shape, data type, and axis centres are derived.
         The last record of the FITS-like header must be terminated by
         an END keyword; subsequent records in the input file are
         treated as an array component given by COMP.  \texttt{[FALSE]}
      }
      \sstsubsection{
         IN = FILENAME (Read)
      }{
         Name of the input text Fortran file.  The file will normally
         have variable-length records when there is a header, but
         always fixed-length records when there is no header.  The
         maximum record length allowed is 512 bytes.
      }
      \sstsubsection{
         MAXLEN = INTEGER (Read)
      }{
         The maximum record length in bytes of records within the input
         text file.  Unless the records are longer than 512 bytes, you
         can use the default value.  The suggested value is the current
         value.  \texttt{[512]}
      }
      \sstsubsection{
         OUT = NDF (Read and Write)
      }{
         Output NDF data structure.  When COMP is not \texttt{"Data"} the NDF
         is modified rather than a new NDF created.
         It becomes the new current NDF.
      }
      \sstsubsection{
         SHAPE = \_INTEGER (Read)
      }{
         The shape of the NDF to be created.  For example, \texttt{[40,30,20]}
         would create 40 columns by 30 lines by 10 bands.  It is only
         accessed when FITS is \texttt{FALSE}.
      }
      \sstsubsection{
         SKIP = INTEGER (Read)
      }{
         The number of header records to be skipped at the start of the
         input file before finding the data array or FITS-like header.
         \texttt{[0]}
      }
      \sstsubsection{
         TYPE = LITERAL (Read)
      }{
         The data type of the output NDF.  It must be one of the
         following HDS types: \texttt{"\_BYTE"}, \texttt{"\_WORD"},
         \texttt{"\_REAL"},
         \texttt{"\_INTEGER"}, \texttt{"\_DOUBLE"}, \texttt{"\_UBYTE"},
         \texttt{"\_UWORD"} corresponding to signed byte,
         signed word, real, integer, double precision, unsigned byte,
         and unsigned word.  See \xref{SUN/92}{sun92}{} for further details.
         An unambiguous abbreviation may be given.  TYPE is ignored when
         COMP = \texttt{"Quality"} since the QUALITY component must comprise
         unsigned bytes (equivalent to TYPE = \texttt{"\_UBYTE"}) to be a valid
         NDF. The suggested default is the current value.  TYPE is only
         accessed when FITS is \texttt{FALSE}.  \texttt{["\_REAL"]}
       }
   }
   \sstexamples{
      \sstexamplesubsection{
         ascii2ndf ngc253.dat ngc253 shape=[100,60]
      }{
         This copies a data array from the text file \texttt{ngc253.dat} to the
         NDF called ngc253.  The input file does not contain a header
         section.  The NDF is two-dimensional: 100 elements in \textit{x} by 60
         in \textit{y}.  Its data array has type \_REAL.
      }
      \sstexamplesubsection{
         ascii2ndf ngc253q.dat ngc253 q shape=[100,60]
      }{
         This copies a quality array from the text file \texttt{ngc253q.dat} to
         an existing NDF called ngc253 (such as created in the first
         example).  The input file does not contain a header section.  The NDF
         is two-dimensional: 100 elements in \textit{x} by 60 in \textit{y}.  
         Its data array has type \_UBYTE.
      }
      \sstexamplesubsection{
         ascii2ndf ngc253.dat ngc253 fits
      }{
         This copies a data array from the text file \texttt{ngc253.dat}
         to the NDF called ngc253.  The input file contains a FITS-like
         header section, which is copied to the FITS extension of the
         NDF.  The shape of the NDF is controlled by the mandatory FITS
         keywords NAXIS, AXIS1, \dots, AXIS{\em{n}}, and the data type by
         keywords BITPIX and UNSIGNED.
      }
      \sstexamplesubsection{
         ascii2ndf type="\_uword" in=ngc253.dat out=ngc253 maxlen=4000 $\backslash$
      }{
         This copies a data array from the text file \texttt{ngc253.dat} to the
         NDF called ngc253.  The input file does not contain a header
         section.  The NDF has the current shape and data type is
         unsigned word.  The maximum record length is 4000 bytes.
      }
      \sstexamplesubsection{
         ascii2ndf spectrum ZZ skip=2 shape=200
      }{
         This copies a data array from the text file \texttt{spectrum} to
         the NDF called ZZ.  The input file contains two header records
         that are ignored.  The NDF is one-dimensional comprising 200
         elements of type \_REAL.
      }
      \sstexamplesubsection{
         ascii2ndf spectrum.lis ZZ skip=1 fits
      }{
         This copies a data array from the text file \texttt{spectrum.lis} to
         the NDF called ZZ.  The input file contains one header 
         record, that is ignored, followed by a FITS-like header section, which
         is copied to the FITS extension of the NDF.  The shape of the
         NDF is controlled by the mandatory FITS keywords NAXIS, AXIS1,
         \dots, AXIS{\em{n}}, and the data type by keywords BITPIX and UNSIGNED.
      }
   }
   \sstnotes{
      The details of the conversion are as follows:
      \ssthitemlist{

         \sstitem
            the ASCII-file array is written to the NDF array as
            selected by COMP.  When the NDF is being modified, the shape
            of the new component must match that of the NDF.

         \sstitem
            If the input file contains a FITS-like header, and a new
            NDF is created, {\it i.e.}\ COMP = \texttt{"Data"}, the header 
            records are placed within the NDF's FITS extension.  
            This enables more
            than one array (input file) to be used to form an NDF.  Note
            that the data array must be created first to make a valid NDF,
            and it's the FITS structure associated with that array that is
            wanted.  Indeed the application prevents you from doing
            otherwise.

         \sstitem
            The FITS-like header defines the properties of the NDF as
            follows:
            \begin{itemize}
            \item BITPIX defines the data type: 8 gives \_BYTE, 16 produces
            \_WORD, 32 makes \_INTEGER, $-$32 gives \_REAL, and $-$64 generates
            \_DOUBLE.  For the first two, if there is an extra header
            record with the keyword UNSIGNED and logical value T, these
            types become \_UBYTE and \_UWORD respectively.  UNSIGNED is
            non-standard, since unsigned integers would not follow in a
            proper FITS file.  However, here it is useful to enable
            unsigned types to be input into an NDF.  UNSIGNED may be
            created by this application's sister, NDF2ASCII.  BITPIX is
            ignored for QUALITY data; type \_UBYTE is used.
            \item NAXIS, and NAXIS{\em{n}} define the shape of the NDF.
            \item The TITLE, LABEL, and BUNIT are copied to the NDF
            TITLE, LABEL, and UNITS NDF components respectively.
            \item The CDELT{\em{n}}, CRVAL{\em{n}}, CTYPE{\em{n}}, and 
            CUNIT{\em{n}} keywords make
            linear axis structures within the NDF.  CUNIT{\em{n}} define the
            axis units, and the axis labels are assigned to CTYPE{\em{n}}. 
            If some are missing, pixel co-ordinates are used for those
            axes.
            \item BSCALE and BZERO in a FITS extension are ignored.
            \item BLANK is not used to indicate which input array values
            should be assigned to a
            \xref{standard bad value}{sun95}{se_badmasking}.
            \item END indicates the last header record unless it
            terminates a dummy header, and the actual data is in an
            extension.
            \end{itemize}

         \sstitem
            Other data item such as HISTORY, data ORIGIN, and axis
            widths are not supported, because the text file has a simple
            structure to enable a diverse set of input files to be
            converted to NDFs, and to limitations of the standard FITS
            header.
      }
   }
   \sstdiytopic{
      Related Applications
   }{
      \CONVERT: \htmlref{NDF2ASCII}{NDF2ASCII};
      \KAPPA: \xref{TRANDAT}{sun95}{TRANDAT};
      \FIGARO: \xref{ASCIN}{sun86}{ASCIN} and
      \xref{ASCOUT}{sun86}{ASCOUT}.
   }
}

\newpage
\sstroutine{
   AST2NDF
}{
   Converts an Asterix data cube into a simple NDF
}{
   \sstdescription{
      This application converts an Asterix data cube into a standard
      \NDFref.  See Section \texttt{"}Notes\texttt{"} (below) for details
      of the conversion.
   }
   \sstusage{
      ast2ndf in out
   }
   \sstparameters{
      \sstsubsection{
         IN  =  NDF (Read)
      }{
         The name of the input Asterix data cube.  The file extension
         (\texttt{.sdf}) should not be included since it is appended
         automatically by the application.
      }
      \sstsubsection{
         OUT  =  NDF (Write)
      }{
         The name of the output NDF containing the data cube written
         by the application.  The file extension (\texttt{.sdf}) should not be
         included since it is appended automatically by the application.
      }
   }
   \sstexamples{
      \sstexamplesubsection{
         ast2ndf  ast\_cube  ndf\_cube
      }{
         This example generates NDF data cube ndf\_cube (in file 
         \texttt{ndf\_cube.sdf}) from Asterix cube ast\_cube (in file
        \texttt{ast\_cube.sdf}).
      }
   }
   \sstnotes{
      This application accepts data in the format used by the Asterix
      package (see \xref{SUN/98}{sun98}{}).
      These data are cubes, with two axes
      comprising a regular grid of positions on the sky and the third
      corresponding to energy or wavelength.  The data are Starlink
      \HDSref\ files which are very similar in format to a standard NDF.  The
      following points apply.

      \sstitemlist{

         \sstitem
         The Asterix QUALITY array is non-standard.  There is no QUALITY
         component in the output NDF.  Instead `bad' or `null' values
         are used to indicate missing or suspect values.

         \sstitem
         The VARIANCE component is copied if it is present.

         \sstitem
         The non-standard Asterix axis components are replaced with
         standard ones.

         \sstitem
         The order of the axes is rearranged.
      }
   }
   \sstdiytopic{
      References
   }{
      \begin{refs}
        \item D.J. Allan and R.J. Vallance, 1995, in 
          \xref{SUN/98}{sun98}{}: 
          {\it ASTERIX -- X-ray Data Processing System}, Starlink.
      \end{refs}
   }
   \sstdiytopic{
      Related Applications
   }{
      \KAPPA: \xref{AXCONV}{sun95}{AXCONV}.
   }
}

\newpage
\sstroutine{
   DA2NDF
}{
   Converts a direct-access unformatted file to an NDF
}{
   \sstdescription{
      This application converts a direct-access (fixed-length records)
      unformatted file to an \NDFref.  It can therefore also process
      unformatted data files generated by C routines.  Only one of the
      array components may be created from the input file.   The shape
      of the NDF has be to be supplied.
   }
   \sstusage{
      da2ndf in out [comp] noperec shape [type]
   }
   \sstparameters{
      \sstsubsection{
         COMP = \xref{LITERAL}{sun95}{se_parmenu} (Read)
      }{
         The NDF component to be copied.  It may be \texttt{"Data"},
         \texttt{"Quality"}
         or \texttt{"Variance"}.  To create a variance or quality array the NDF
         must already exist. \texttt{["Data"]}
      }
      \sstsubsection{
         IN = FILENAME (Read)
      }{
         Name of the input direct-access unformatted file.
      }
      \sstsubsection{
         NOPEREC = \_INTEGER (Read)
      }{
         The number of data values per record of the input file.  It
         must be positive.  The suggested default is the size of the
         first dimension of the array.  A null (\texttt{!}) value for NOPEREC
         causes the size of first dimension to be used.
      }
      \sstsubsection{
         OUT = NDF (Read and Write)
      }{
         Output NDF data structure.  When COMP is not \texttt{"Data"} the NDF
         is modified rather than a new NDF created.  It becomes the new
         current NDF.  Unusually for an output NDF, there is a suggested
         default---the current value---to facilitate the inclusion of
         variance and quality arrays.
      }
      \sstsubsection{
         SHAPE = \_INTEGER (Read)
      }{
         The shape of the NDF to be created.  For example, \texttt{[40,30,20]}
         would create 40 columns by 30 lines by 10 bands.
      }
      \sstsubsection{
         TYPE = LITERAL (Read)
      }{
         The data type of the direct-access file and the NDF.  It must
         be one of the following HDS types: \texttt{"\_BYTE"}, 
         \texttt{"\_WORD"},
         \texttt{"\_REAL"}, \texttt{"\_INTEGER"}, \texttt{"\_DOUBLE"},
         \texttt{"\_UBYTE"}, \texttt{"\_UWORD"} corresponding to
         signed byte, signed word, real, integer, double precision,
         unsigned byte, and unsigned word respectively.  See
         \xref{SUN/92}{sun92}{} for further details.  An unambiguous
         abbreviation may be given.
         TYPE is ignored when COMP = \texttt{"Quality"} since the QUALITY
         component must comprise unsigned bytes (equivalent to TYPE =
         \texttt{"\_UBYTE"}) to be a valid NDF. The suggested default is the
         current value. \texttt{["\_REAL"]}
      }
   }
   \sstexamples{
      \sstexamplesubsection{
         da2ndf ngc253.dat ngc253 shape=[100,60] noperec=8
      }{
         This copies a data array from the direct-access file \texttt{ngc253.dat}
         to the NDF called ngc253.  The NDF is two-dimensional: 100
         elements in \textit{x} by 60 in \textit{y}.
         Its data array has type \_REAL.  
         The data records each have 8 real values.
      }
      \sstexamplesubsection{
         da2ndf ngc253q.dat ngc253 q 100 [100,60]
      }{
         This copies a quality array from the direct-access file
         \texttt{ngc253q.dat} to an existing NDF called ngc253 (such as created
         in the first example).  The NDF is two-dimensional: 100
         elements in \textit{x} by 60 in \textit{y}.  
         Its data array has type \_UBYTE.
         The data records each have 100 unsigned-byte values.
      }
      \sstexamplesubsection{
         da2ndf type="\_uword" in=ngc253.dat out=ngc253 $\backslash$
      }{
         This copies a data array from the direct-access file
         \texttt{ngc253.dat}
         to the NDF called ngc253.  The NDF has the current shape and
         data type is unsigned word.  The current number of values per
         record is used.
      }
   }
   \sstnotes{
      The details of the conversion are as follows:
      \ssthitemlist{

         \sstitem
            the direct-access file's array is written to the NDF array
            as selected by COMP.  When the NDF is being modified, the
            shape of the new component must match that of the NDF.  This
            enables more than one array (input file) to be used to form an
            NDF.  Note that the data array must be created first to make a
            valid NDF.  Indeed the application prevents you from doing
            otherwise.

         \sstitem
            Other data items such as axes are not supported, because of
            the direct-access file's simple structure.
      }
   }
   \sstdiytopic{
      Related Applications
   }{
      \CONVERT: \htmlref{NDF2DA}{NDF2DA}.
   }
}

\newpage
\sstroutine{
   DST2NDF
}{
   Converts a Figaro (Version 2) DST file to an NDF
}{
   \sstdescription{
      This application converts a \Figaroref\ Version-2 DST file to a
      Version-3 file, {\it i.e.}\ to an \NDFref.  The rules for converting the
      various components of a DST are listed in the \texttt{"}Notes\texttt{"}.  Since
      both are hierarchical formats most files can be be converted with
      little or no information lost.
   }
   \sstusage{
      dst2ndf in out
   }
   \sstparameters{
      \sstsubsection{
         FORM = \xref{LITERAL}{sun95}{se_parmenu} (Read)
      }{
         The storage form of the NDF's data and variance arrays.
         FORM = \texttt{"Simple"} gives the simple form, where the array of data
         and variance values is located in an ARRAY structure.  Here it
         can have ancillary data like the origin.  This is the normal
         form for an NDF.  FORM = \texttt{"Primitive"} offers compatibility with
         earlier formats, such as IMAGE.  In the primitive form the
         data and variance arrays are primitive components at the top
         level of the NDF structure, and hence it cannot have
         ancillary information. \texttt{["Simple"]}
      }
      \sstsubsection{
         IN = Figaro file (Read)
      }{
         The file name of the version 2 file.  A file extension must
         not be given after the name, since \texttt{".dst"} is appended by the
         application.  The file name is limited to 80 characters.
      }
      \sstsubsection{
         OUT = NDF (Write)
      }{
         The file name of the output NDF file.  A file extension must
         not be given after the name, since \texttt{".sdf"} is appended by the
         application.  Since the NDF\_ library is not used, a section
         definition may not be given following the name.  The file
         name is limited to 80 characters.
      }
   }
   \sstexamples{
      \sstexamplesubsection{
         dst2ndf old new
      }{
         This converts the Figaro file \texttt{old.dst} to the NDF called new
         (in file \texttt{new.sdf}).  The NDF has the simple form.
      }
      \sstexamplesubsection{
         dst2ndf horse horse p
      }{
         This converts the Figaro file \texttt{horse.dst} to the NDF called
         horse (in file \texttt{horse.sdf}).  The NDF has the primitive form.
      }
   }
   \sstnotes{
      The rules for the conversion of the various components are as
      follows:
      \vspace{-\parskip}

      \latex{\scriptsize}
      \begin{center}
      \begin{tabular}{|lcl|p{43mm}|}
      \hline 
      \multicolumn{1}{|c}{Figaro file} & & \multicolumn{1}{c}{NDF} &
      \multicolumn{1}{|c|}{Comments} \\ \hline
      .Z.DATA   & $\Rightarrow$ & .DATA\_ARRAY.DATA & when FORM = \texttt{"SIMPLE"}\\
      .Z.DATA   & $\Rightarrow$ & .DATA\_ARRAY & when FORM = \texttt{"PRIMITIVE"} \\
      .Z.ERRORS & $\Rightarrow$ & .VARIANCE.DATA & after processing when FORM = \texttt{"SIMPLE"} \\
      .Z.ERRORS & $\Rightarrow$ & .VARIANCE & after processing when FORM = \texttt{"PRIMITIVE"} \\
      .Z.QUALITY & $\Rightarrow$ & .QUALITY.QUALITY & must be BYTE array
                                  (see Bad-pixel handling below) \\
      & $\Rightarrow$ & .QUALITY.BADBITS = 255 & \\
      .Z.LABEL  & $\Rightarrow$ & .LABEL & \\
      .Z.UNITS  & $\Rightarrow$ & .UNITS & \\
      .Z.IMAGINARY & $\Rightarrow$ & .DATA\_ARRAY.IMAGINARY\_DATA & \\
      .Z.MAGFLAG & $\Rightarrow$ & .MORE.FIGARO.MAGFLAG & \\
      .Z.RANGE  & $\Rightarrow$ & .MORE.FIGARO.RANGE & \\
      .Z.xxxx   & $\Rightarrow$ & .MORE.FIGARO.Z.xxxx & \\ 
      & & & \\
      .X.DATA   & $\Rightarrow$ & .AXIS(1).DATA\_ARRAY & \\ 
      .X.ERRORS & $\Rightarrow$ & .AXIS(1).VARIANCE & after processing \\
      .X.WIDTH  & $\Rightarrow$ & .AXIS(1).WIDTH & \\
      .X.LABEL  & $\Rightarrow$ & .AXIS(1).LABEL & \\
      .X.UNITS  & $\Rightarrow$ & .AXIS(1).UNITS & \\
      .X.LOG    & $\Rightarrow$ & .AXIS(1).MORE.FIGARO.LOG & \\
      .X.xxxx   & $\Rightarrow$ & .AXIS(1).MORE.FIGARO.xxxx & \\
      & & & (Similarly for .Y .T .U .V or .W structures which are
             renamed to AXIS(2), \ldots, AXIS(6) in the NDF.) \\
      & & & \\
      .OBS.OBJECT & $\Rightarrow$ & .TITLE & \\
      .OBS.SECZ & $\Rightarrow$ & .MORE.FIGARO.SECZ & \\
      .OBS.TIME & $\Rightarrow$ & .MORE.FIGARO.TIME & \\
      .OBS.xxxx & $\Rightarrow$ & .MORE.FIGARO.OBS.xxxx & \\
      & & & \\
      .FITS.xxxx& $\Rightarrow$ & .MORE.FITS.xxxx & into value part of
         the string \\
      .COMMENTS.xxxx  & $\Rightarrow$ & .MORE.FITS(\textit{n}) & 
         into comment part of the string \\
      .FITS.xxxx.DATA & $\Rightarrow$ & .MORE.FITS(\textit{n}) & 
         into value part of the string \\
      .FITS.xxxx.DESCRIPTION & $\Rightarrow$ & .MORE.FITS(\textit{n}) & 
         into comment part of the string \\
      .FITS.xxxx.yyyy & $\Rightarrow$ & .MORE.FITS(\textit{n}) & 
         into blank-keyword comment containing \texttt{yyyy=value} \\
      .MORE.xxxx& $\Rightarrow$ & .MORE.xxxx & \\
      & & & \\
      .TABLE    & $\Rightarrow$ & .MORE.FIGARO.TABLE & \\
      .xxxx     & $\Rightarrow$ & .MORE.FIGARO.xxxx & \\ \hline
      \end{tabular}
      \end{center}
      \normalsize

      Axis arrays with dimensionality greater than one are not
      supported by the NDF.  Therefore, if the application encounters
      such an axis array, it processes the array using the following
      rules, rather than those given above.

      \latex{\scriptsize}
      \begin{center}
      \begin{tabular}{|lcl|p{48mm}|}
      \hline 
      \multicolumn{1}{|c}{Figaro file} & & \multicolumn{1}{c}{NDF} &
      \multicolumn{1}{|c|}{Comments} \\ \hline
      .X.DATA   & $\Rightarrow$ & .AXIS(1).MORE.FIGARO.DATA\_ARRAY &
            AXIS(1).DATA\_ARRAY is filled with pixel co-ordinates \\
      .X.ERRORS & $\Rightarrow$ & .AXIS(1).MORE.FIGARO.VARIANCE & after
            processing \\
      .X.WIDTH  & $\Rightarrow$ & .AXIS(1).MORE.FIGARO.WIDTH & \\ \hline
      \end{tabular}
      \end{center}
      \normalsize

      In addition to creating a blank-keyword NDF FITS-extension
      header for each component of a non-standard DST FITS structure
      (.FITS.xxxx.yyyy where yyyy is not DATA or DESCRIPTION), this set
      of related headers are bracketed by blank lines and a comment
      containing the name of the structure ({\it i.e.}\ xxxx).
   }
   \sstdiytopic{
      Related Applications
   }{
      \CONVERT: \htmlref{NDF2DST}{NDF2DST}.
   }
   \sstdiytopic{
   \xref{Bad-pixel handling}{sun95}{se_masking}
   }{
   The QUALITY array is only copied if the bad-pixel flag
   (.Z.FLAGGED) is FALSE or absent.  A simple NDF with the bad-pixel
   flag set to FALSE (meaning that there are no bad-pixels present)
   is created when .Z.FLAGGED is absent or false and FORM = \texttt{"SIMPLE"}.
   }
   \sstimplementationstatus{
      The maximum number of dimensions is 6.
   }
}

\newpage
\sstroutine{
   FITS2NDF
}{
   Converts FITS files into NDFs
}{
   \sstdescription{
      This application converts one or more files in the \FITSref\ format
      into {\NDFref}s.  It can process an arbitrary FITS file to produce an
      NDF, using NDF extensions to store information conveyed in table
      and image components of the FITS file.  While no information is
      lost, in many common cases this would prove inconvenient
      especially as no meaning is attached to the NDF extension
      components.  Therefore, FITS2NDF recognises certain data products
      (currently \htmladdnormallink{IUE Final Archive}{http://www.vilspa.esa.es/iue/iuefa.html}, 
      \htmladdnormallink{INES}{http://ines.laeff.esa.es/}, 
      \htmladdnormallink{ISO}{http://www.iso.vilspa.esa.es/users/idc/IDC.html},
      and \htmladdnormallink{2dF}{http://www.aao.gov.au/local/www/2df/}), and provides
      tailored conversions that map the FITS data better on to the NDF
      components.  For instance, a FITS IMAGE extension storing data
      errors will have its data array transferred to the NDF's VARIANCE
      (after being squared).  In addition, FITS2NDF can restore
      NDFs converted to FITS by the sister task \htmlref{NDF2FITS}{NDF2FITS}.

      A more general facility is also provided to associate specified
      FITS extensions with NDF components by means of entries in a file
      (see the EXTABLE parameter).

      Details of the supported special formats and rules for processing
      them are given in topic 
      \htmlref{\texttt{"}Special Formats\texttt{"}}{special_formats}; 
      the general-case
      processing rules are described in the 
      \htmlref{\texttt{"}Notes\texttt{"}}{fits2ndf_notes}.

      FITS2NDF can also process both external and internal compressed
      FITS files.  The external compression applies to the whole file
      and FITS2NDF recognises {\bf gzip} (\texttt{.gz}) and UNIX 
      {\bf compress} (\texttt{.Z}) formats.  Internal compressions are
      where a large image is tiled and each tile is compressed.  The
      supported formats are Rice, the IRAF PLIO, and GZIP.

      Both NDF and FITS use the term extension, and they mean different
      things.  Thus to avoid confusion in the descriptions below, the term
      `sub-file' is used to refer to a FITS IMAGE, TABLE or BINTABLE Header
      and Data Unit (HDU).
   }
   \sstusage{
      fits2ndf in out
   }
   \sstparameters{
      \sstsubsection{
         CONTAINER = \_LOGICAL (Read)
      }{
         If \texttt{TRUE} causes each HDU from the FITS file to be written as
         a component of the HDS container file specified by the OUT
         parameter.  Each component will be named \texttt{HDU\_\textit{n}},
         where \texttt{\textit{n}} is the FITS HDU number. The primary HDU
         is numbered 0.  Primary and IMAGE HDUs will become NDFs and if the
         PROFITS parameter is TRUE, each NDF's FITS extension will be
         created from the header of the FITS sub-file.  It will have the form
         of a primary header and may include cards inherited from the primary
         header.  If the FITS HDU has no data array, an NDF will not be
         created---if PROFITS is \texttt{TRUE}, a structure of type FITS\_HEADER,
         containing the FITS header as an array of type \_CHAR*80, is created;
         if PROFITS is \texttt{FALSE}, no component is created.
         Binary and ASCII tables become components of type `TABLE',
         formatted as in the general rules under \texttt{"}Notes\texttt{"} below.
         \texttt{[FALSE]}
      }
      \sstsubsection{
         ENCODINGS = \xref{LITERAL}{sun95}{se_parmenu} (Read)
      }{
         Determines which FITS keywords should be used to define the
         world co-ordinate systems to be stored in the NDF's WCS
         component. The allowed values (case-insensitive) are:
         \ssthitemlist{

         \sstitem 
            \texttt{"FITS-IRAF"} --- This uses keywords CRVAL\textit{i},
            CRPIX\textit{i}, CD\textit{i\_j}, and is the
            system commonly used by IRAF. It is described in the document
            \textit{World Coordinate Systems Representations Within the FITS 
            Format} by R.J. Hanisch and D.G. Wells, 1988, available by ftp 
            from fits.cv.nrao.edu \texttt{/fits/documents/wcs/wcs88.ps.Z}.

         \sstitem
            \texttt{"FITS-WCS"} --- This is the FITS standard WCS encoding 
            scheme described in the paper 
            \htmladdnormallink{\textit{Representation of celestial coordinates in FITS.}}
            {http://www.atnf.csiro.au/people/mcalabre/WCS/}\\ \latex{
            (\texttt{http://www.atnf.csiro.au/people/mcalabre/WCS/})}  It is
            very similar to \texttt{"FITS-IRAF"} but supports a wider range of
            projections and co-ordinate systems.

         \sstitem
            \texttt{"FITS-PC"} --- This uses keywords CRVAL\textit{i},
            CDELT\textit{i}, CRPIX\textit{i}, PC\textit{iiijjj}, 
            \textit{etc}, as described in a previous (now superseded) draft of
            the above FITS world co-ordinate system paper by E.W.Greisen and 
            M.Calabretta.

         \sstitem
            \texttt{"FITS-AIPS"} --- This uses conventions described in the
            document "\textit{Non-linear Coordinate Systems in AIPS}" by
            Eric W. Greisen (revised 9th September, 1994), available by ftp
            from fits.cv.nrao.edu \texttt{/fits/documents/wcs/aips27.ps.Z}. 
            It is currently employed by the AIPS data-analysis facility
            (amongst others), so its use will facilitate data exchange with
            AIPS. This encoding uses CROTA\textit{i} and CDELT\textit{i}
            keywords to describe axis rotation and scaling.

         \sstitem
            \texttt{"FITS-AIPS++"} --- This is an extension to FITS-AIPS which 
            allows the use of a wider range of celestial projections, as used by
            the AIPS++ project.

         \sstitem
            \texttt{"FITS-CLASS"} --- This uses the conventions of the CLASS
            project.  CLASS is a software package for reducing 
            single-dish radio and sub-mm spectroscopic data.  It 
            supports double-sideband spectra.  See 
            \htmladdnormallink{the GILDAS 
            manual}{http://www.iram.fr/IRAMFR/GILDAS/doc/html/class-html/class.html}.

         \sstitem
            \texttt{"DSS"} --- This is the system used by the Digital Sky Survey,
            and uses keywords AMDX\textit{n}, AMDY\textit{n}, PLTRAH, 
            \emph{etc}.

         \sstitem
            \texttt{"NATIVE"} --- This is the native system used by the  
            AST library (see \xref{SUN/210}{sun210}{}), and provides a
            loss-free method for transferring WCS information between
            AST-based applications.  It allows more complicated WCS
            information to be stored and retrieved than any of the other
            encodings.
         }
         A comma-separated list of up to six values may be supplied, in
         which case the value actually used is the first in the list for
         which corresponding keywords can be found in the FITS header.

         A FITS header may contain keywords from more than one of these
         encodings, in which case it is possible for the encodings to be
         inconsistent with each other. This may happen for instance if
         an application modifies the keyword associated with one encoding
         but fails to make equivalent modifications to the others.

         If a null
         parameter value (\texttt{!}) is supplied for ENCODINGS, then an attempt
         is made to determine the most reliable encoding to use as follows. If
         both native and non-native encodings are available, then the
         first non-native encoding to be found which is inconsistent with the
         native encoding is used. If all encodings are consistent, then the
         native encoding is used (if present). \texttt{[!]}
      }
      \sstsubsection{
         EXTABLE = FILE (Read)
      }{
        This specifies the name of a text file containing a table associating
        sub-files from a multi-extension FITS file with specific NDF
        components. If the null value (!) is given for EXTABLE, FITS sub-files
        are treated as determined by the PROEXTS parameter (see below).

        An EXTABLE file contains records which may be:
        \sstitemlist {
        \latex{\vspace*{4.5ex}}
        \sstitem
        `\textbf{component specifier records}', which associate FITS sub-files
        with NDF components.
        \sstitem
        `\textbf{NDFNAMES records}', which specify the names of the NDFs to be
          created. Normally they will be created within the top-level HDS
          container file specified by the OUT parameter.
        \sstitem
        `\textbf{directive records}', which inform the table file parser.
        }
        Spaces are allowed between elements within records and blank records
        are ignored.
        \medskip

        \textbf{Component specifier records} have the form:
        \begin{quote}
          \texttt{\textit{component}; \textit{sub-file\_specifiers};
          \textit{transformation\_code}}
        \end{quote}
        where:
           \ssthitemlist{
           \sstitem
           \texttt{\textit{component}} (case-insensitive) specifies the NDF
              component and is \texttt{DATA}, \texttt{VARIANCE}, \texttt{QUALITY} or
              \texttt{EXTN}\textit{i}.\textit{name}. The \texttt{EXTN}\textit{i}.\textit{name}
              form specifies the name \textit{name} of an NDF extension to be
              created. \textit{name} may be omitted in which case 
              \texttt{FITS\_EXT\_\textit{n}} is assumed, where
              \texttt{\textit{n}} is the FITS sub-file number.
              \textit{i} comprises any characters and may be omitted; it serves to
              differentiate component specifiers where the default name is to
              be used.
           \sstitem
           \texttt{\textit{sub-file\_specifiers}} is a list of FITS sub-file
              specifiers, separated by commas. The \textit{n}th sub-file
              specifier from each component specifier record forms a
              `sub-file set' and each
              sub-file set will be used to create one NDF in the output
              file.

              Each sub-file specifier may be:
              \ssthitemlist{
              \sstitem
              An integer, specifying the FITS Header and Data Unit (HDU)
                 number. The primary HDU number is 0.
              \sstitem
              \textit{keyword}=\textit{value} (case-insensitive), specifying 
                 a FITS HDU
                 where the specified keyword has the specified value,
                 \textit{e.g.} \texttt{EXTNAME=IM2}.
                 The \textit{keyword=} may be omitted in which
                 case \texttt{EXTNAME} is assumed.
                 Multiple \textit{keyword}=\textit{value} pairs separated by
                 commas and enclosed
                 in \texttt{[]} may be given as a single sub-file specifier.
                 All the given keywords must match the sub-file header values.
              \sstitem
              Omitted, to indicate that the component is not required
                 for the corresponding NDF. (Commas may be needed to maintain
                 correct sub-file set alignment for later sub-file
                 specifiers.) If the last character of 
                 \texttt{\textit{sub-file\_specifiers}} is comma, it
                 indicates an omitted specifier at the end.
                 Note that if a sub-file is not specified for the DATA
                 component of an NDF, an error will be reported at closedown.
              }
           \sstitem
           \texttt{\textit{transformation\_code}} (case-insensitive) is a 
              character string specifying a  transformation to be applied to
              the FITS data before it is written into the NDF component.
              The code and
              preceding \texttt{";"} may be omitted in which case \texttt{"NONE"} (no
              transformation) is assumed.  Currently the only permitted code
              is \texttt{"NONE"}.
           }
          There may be more than one component specifier record for a given
          component, the sub-file specifiers will be concatenated. A
          sub-file specifier may not span records and only the transformation
          code specified by the last record for the component will be
          effective.
          \medskip

        \textbf{NDFNAMES records} have the format:
          \begin{quote}
          \texttt{NDFNAMES \textit{name\_list}}
          \end{quote}
           Where \texttt{\textit{name\_list}} is a list of names for the NDFs
           to be created, one
           for each sub-file set specified by the component specifier lines.
           The names are separated by commas. If any of the names are omitted,
           the last name specified is assumed to be a root name to which an
           integer counter is to be added until a new name is found.  If no
           names are specified, \texttt{EXTN\_SET} is used as the root name.
           For example,  \texttt{NDFNAMES NDF,,SET\_}
           would result in NDFs named NDF1, NDF2, SET\_1, SET\_2 \textit{etc.}
           up to the given number of sub-file sets.

           There may be multiple NDFNAMES records, the names will be
           concatenated.  A name may not span records and a comma as the last
           non-blank character indicates an omitted name. 

           If there is only one sub-file set, \textit{name\_list} may be
           `\texttt{*}',
           in which case the NDF will be created at the top level of the output
           file.
           \medskip

        \textbf{Directive records} have \texttt{\#} in column 1 and will
           generally be treated
           as comments and ignored.  An exception is a record starting with
           `\texttt{\#END}', which may optionally be used to terminate the file.
        
        Each HDU of the FITS file is processed in turn. If it matches one 
        of the sub-file specifiers in the table, it is used to create the 
        specified component of the appropriate NDF in the output file;
        otherwise the next HDU is processed. The table is searched in
        sub-file set order.
        If a table entry is matched it is removed from the table; this
        means that the same FITS sub-file specifier may be repeated for
        another NDF component but each FITS HDU can only be used once. 
        If sub-file specifiers remain unmatched at the end, a warning
        message is displayed.
        \bigskip

        A simple example of an EXTABLE is:
        \begin{quote} 
          \texttt{\# A simple example\\
          DATA;0,1,2,3,4,5,6\\
          \#END}
        \end{quote}
        The primary HDU and sub-files 1--6 of the FITS file will be written
        as the DATA components of NDFs EXTN\_SET1--EXTN\_SET7 within the HDS
        container file specified by the OUT parameter.

        A contrived example, showing more of the facilities, is:
        \begin{quote}
          \texttt{\# A contrived example\\
          NDFNAMES obs\_\\
          DATA; 1, EXTNAME=IM4, IM7; none\\
          VARIANCE; 2,im5, im8\\
          EXTN.CAL;3 ,,[extname=cal,extver=2]\\
          \#END}
        \end{quote}
        The HDS container file specified by the OUT parameter will contain
        three NDFs, the NDFNAMES record specifies that they will be named
        OBS\_1, OBS\_2 and OBS\_3.

        NDF OBS\_1 will have its DATA component created from the first
        extension  (HDU 1) of the FITS file specified by the IN parameter,
        and its VARIANCE from the second.  NDF OBS\_1 will have an extension
        named CAL created from the third FITS extension.

        NDF OBS\_2 has DATA and VARIANCE components created from the FITS
        sub-files whose EXTNAME keywords have the value IM4 and IM5
        respectively; no CAL extension is created in OBS\_2.

        OBS\_3 DATA and VARIANCE are created from FITS sub-files named IM7 and
        IM8 and the CAL extension from the FITS sub-file whose EXTNAME and
        EXTVER keywords have values \texttt{`CAL'} and \texttt{2} respectively.

        In all cases, if the PROFITS parameter is TRUE, the NDF's FITS
        extension will be created from the header of the sub-file
        associated with the DATA component of the NDF. It will have the
        form of a primary header and may include cards inherited from the
        primary header \texttt{[!]}
      }   
   \sstsubsection{
         FMTCNV = LITERAL (Read)
      }{
         This specifies whether or not format conversion will occur.
         The conversion applies the values of the FITS keywords BSCALE
         and BZERO to the FITS data to generate the `true' data values.
         This applies to IMAGE extensions, as well as the primary data
         array.  If BSCALE and BZERO are not given in the FITS header,
         they are taken to be 1.0 and 0.0 respectively.

         If FMTCNV=\texttt{"FALSE"}, the HDS type of the data array in the NDF
         will be the equivalent of the FITS data format on tape
         ({\em{e.g.}}\ BITPIX = 16 creates a \_WORD array).  If \texttt{"TRUE"},
         the data array in the NDF will be converted from the FITS data
         type to \_REAL or \_DOUBLE in the NDF.

         The special value FMTCNV=\texttt{"Native"} is a variant of \texttt{"FALSE"}, 
         that in addition creates a scaled form of NDF array, provided
         the array values are scaled through BSCALE and/or BZERO 
         keywords (\emph{i.e.} the keywords' values are not the null 1.0 
         and 0.0 respectively).  This NDF scaled array contains the 
         unscaled data values, and the scale and offset.

         The actual NDF data type for FMTCNV=\texttt{"TRUE"}, and the data 
         type after applying the scale and offset for FMTCNV=\texttt{"NATIVE"}
         are both specified by parameter TYPE.  However, if TYPE is a
         blank string or null ({\tt{!}}), then the choice of floating-point
         data type depends on the number of significant digits
         in the BSCALE and BZERO keywords.

         FMTCNV may be a list of comma-separated values, enclosed in
	 double quotes, to be applied to each conversion in turn. An
	 error results if more values than the number of input FITS
	 files are supplied.  If too few are given, the last value in
	 the list applied to all the conversions; thus a single value
	 is applied to all the input files.  If more than one line is
	 required to enter the information at a prompt then place a
	 \texttt{"-"} at the end of each line where a continuation
	 line is desired.  \texttt{["TRUE"]}
      }
      \sstsubsection{
         IN = LITERAL (Read)
      }{
         The names of the FITS-format files to be converted to NDFs.
         It may be a list of file names or indirection specifications
         separated by commas and enclosed in double quotes.  FITS file
         names may include the regular expressions (\texttt{"$*$"},
         \texttt{"?"}, \texttt{"[a-z]"} {\em etc.})  but a \texttt{"[]"}
         construct at the end of the name is assumed to be
         a sub-file specifier to specify a particular FITS sub-file to
         be converted. (See the description of an EXTABLE file above for
         allowed sub-file specifiers but note that only a single 
         \texttt{keyword=value} pair is allowed here.  Note also that if a
         specifier contains a \texttt{keyword=value} pair, the name(s) must
         be enclosed in double quotes.)
         If you really want to have an \texttt{[a-z]}-type regular expression
         at the end of the filename, you can put a null sub-file specifier 
         \texttt{"[]"} after it.

         Indirection may occur
         through text files (nested up to seven deep).  The indirection
         character is \texttt{"$\wedge$"}.  If extra prompt lines are
         required, append the continuation character \texttt{"-"} to the
         end of the line.  Comments
         in the indirection file begin with the character \texttt{"\#"}.
      }
      \sstsubsection{
         OUT = LITERAL (Write)
      }{
         The names for the output NDFs.  These may be enclosed in
         double quotes and specified as a list of comma-separated names,
         OR, using modification elements to specify output NDF names
         based on the input filenames.  Indirection may be used if
         required.

         The simplest modification element is the asterisk \texttt{"$*$"},
         which means call the output NDF files the same name (without any
         directory specification) as the corresponding input FITS file,
         but with file extension \texttt{".sdf"}.

         Other types of modification can also occur so OUT = \texttt{"x$*$"}
         would mean that the output files would have the same name as the
         input FITS files except for an \texttt{"x"} prefix.  You can also
         replace a specified string in the output filename, for example
         OUT=\texttt{"x$*$|cal|Starlink|"} replaces the string
         \texttt{"cal"} with \texttt{"Starlink"} in any of the output names
         \texttt{"x$*$"}.

         Some of the options create a series of NDFs in the original NDF,
         which becomes just an HDS container and no longer an NDF.
      }
      \sstsubsection{
         PROEXTS = \_LOGICAL (Read)
      }{
         This governs how any extensions within the FITS file are processed 
         in the general case.  If \texttt{TRUE}, any FITS-file extension is
         propagated to the NDF as an NDF extension
         called FITS\_EXT\_\textit{n}, where \textit{n} is the number of the 
         extension.
         If \texttt{FALSE}, any FITS-file extensions are ignored.  The
         \texttt{"}Notes\texttt{"} of the general conversion contain details of
         where and in what form the various FITS-file extensions are stored
         in the NDF.

         This parameter is ignored when the supplied FITS file is one
         of the special formats, including one defined by an EXTABLE but
         excluding NDF2FITS-created files, whose structure in terms of
         multiple FITS objects is defined. Specialist NDF extensions may be
         created in this case.  See topic 
         \htmlref{\texttt{"}Special Formats\texttt{"}}{special_formats}
         for details.

         It is also ignored if a sub-file is specified as the IN parameter,
         or parameter CONTAINER is TRUE.  \texttt{[TRUE]}
      }
      \sstsubsection{
         PROFITS = \_LOGICAL (Read)
      }{
         If TRUE, the headers of the FITS file are written to the NDF's
         FITS extension.  If a specific FITS sub-file has been specified or
         parameter CONTAINER is \texttt{TRUE} or an EXTABLE is in use, the FITS
         extension will appear as a primary header and may include cards
         inherited from the primary HDU; otherwise the FITS header is
         written verbatim. \texttt{[TRUE]}
      }
      \sstsubsection{
         TYPE = LITERAL (Read)
      }{
         The data type of the output NDF's data and variance arrays.
	 It is normally one of the following HDS types: \texttt{"\_BYTE"},
	 \texttt{"\_WORD"}, \texttt{"\_REAL"}, \texttt{"\_INTEGER"},
	 \texttt{"\_DOUBLE"}, \texttt{"\_UBYTE"}, \texttt{"\_UWORD"}
	 corresponding to signed byte, signed word, real, integer,
	 double precision, unsigned byte, and unsigned word.  See
	 \xref{SUN/92}{sun92}{} for further details.  An unambiguous
	 abbreviation may be given.  TYPE is ignored when COMP =
	 \texttt{"Quality"} since the QUALITY component must comprise
	 unsigned bytes (equivalent to TYPE = \texttt{"\_UBYTE"}) to
	 be a valid NDF.  The suggested default is the current value.
	 Note that setting TYPE may result in a loss of precision, and
	 should be used with care.

         A null value (\texttt{!}) or blank requests that the type be
	 propagated from the FITS (using the BITPIX keyword); or if
	 FMTCNV is \texttt{"TRUE"}, the type is either \_REAL or \_DOUBLE
	 depending on the precision of the BSCALE and BZERO keywords.

         TYPE may be a list of comma-separated values enclosed in 
         double quotes, that are applied to each conversion in turn.  An
         error results if more values than the number of input FITS 
         files are supplied.  If too few are given, the last value in 
         the list is applied to all the conversions; thus a single value
         is applied to all the input files.  If more than one line is 
         required to enter the information at a prompt then place a
         \texttt{"-"} at the end of each line where a continuation line
         is desired.  \texttt{[!]}
      }
      \sstsubsection{
         WCSATTRS = LITERAL (Read)
      }{
         A comma-separated list of keyword=value pairs which modify the 
         way WCS information is extracted from the FITS headers. Each of
         the keywords should be an attribute of an AST FitsChan.  This 
         is the object which is responsible for interpreting the FITS
         WCS headers, and is described full in the documentation for the AST 
         library (see \xref{SUN/210}{sun210}{}).  For instance, to force 
         CAR projections to be interpreted as simple linear mappings from
         pixel co-ordinates to celestial co-ordinates (rather than the 
         non-linear mapping implied by the FITS-WCS conventions), use
         WCSATTRS=\texttt{"CarLin=1"}.  A null  value (\texttt{!}) 
         results in all attributes using default values. \texttt{[!]}
      }
   }
   \sstexamples{
      \sstexamplesubsection{
         fits2ndf 256.fit f256 fmtcnv=f
      }{
         This converts the FITS file called \texttt{256.fit} to the NDF called
         f256.  The data type of the NDF's data array matches that of
         the FITS primary data array.  A FITS extension is created in
         f256, and FITS sub-files are propagated to NDF extensions.
      }
      \sstexamplesubsection{
         fits2ndf 256.fit f256 fmtcnv=native type=\_real
      }{
         As above but now a \_REAL type scaled data array is created,
         assuming that {\tt 256.fit} contains scaled integer data with 
         BITPIX=8 or 16 and non-default BSCALE and BZERO keywords.
      }
      \sstexamplesubsection{
         fits2ndf 256.fit f256 fmtcnv=t type=\_real
      }{
         As the first example, but now a \_REAL type data array is 
         created by applying the scale and offset from BSCALE and 
         BZERO keywords to the integer values stored in {\tt 256.fit}.
      }
      \sstexamplesubsection{
         fits2ndf 256.fit f256 noprofits noproexts
      }{
         As the previous example except there will be a format conversion
         from a FITS integer data type to floating point in the NDF
         using the BSCALE and BZERO keywords, and there will be no
         extensions written within f256.
      }
      \sstexamplesubsection{
         fits2ndf "$*$.fit,p$*$.fits" $*$
      }{
         This converts a set of FITS files given by the list
         \texttt{"$*$.fit,p$*$.fits"}, where \texttt{$*$} is the
         match-any-character wildcard.

         The resultant NDFs take the filenames of the FITS files, so if
         one of the FITS files was \texttt{parker.fits}, the resultant NDF
         would be called \texttt{parker}.
         Format conversion is performed on integer
         data types.  A FITS extension is created in each NDF and any
         FITS sub-files present are propagated to NDF extensions.
      }
      \sstexamplesubsection{
         fits2ndf swp25000.mxlo mxlo25000
      }{
         This converts the IUE MXLO FITS file called \texttt{swp25000.mxlo} to
         the NDF called \texttt{mxlo25000}.  Should the dataset comprise both
         the large- and small-aperture spectra, they will be found in
         NDFs \texttt{mxlo25000.large} and \texttt{mxlo25000.small}
         respectively.
      }
      \sstexamplesubsection{
         fits2ndf SWP19966.MXHI mxhi19966
      }{
        This converts the IUE MXHI FITS file called \texttt{SWP19966.MXHI} to a
        series of NDFs within a file \texttt{mxhi19966.sdf}.  Each NDF
        corresponds to an order.  Thus for instance the one hundredth
        order will be in the NDF called \texttt{mxhi19966.order100}.
      }
      \sstexamplesubsection{
         fits2ndf data/$*$.silo silo$*$$|$swp$|$$|$ noprofits
      }{
         This converts all the IUE SILO FITS files with file extension
         \texttt{.silo} in directory data to NDFs in the current directory.
         Each name of an NDF is derived from the corresponding FITS
         filename; the original name has the \texttt{"swp"} removed and
         \texttt{"silo"} is prefixed.  So for example, \texttt{swp25000.silo}
         would become an NDF called silo25000.  No FITS extension is created.
      }
      \sstexamplesubsection{
         fits2ndf "abc.fit,def.fts" "fgh,ijk" fmtcnv="F,T" noproexts
      }{
         This converts the FITS files \texttt{abc.fit} and \texttt{def.fts}
         to the NDFs called fgh and ijk respectively.
         Format conversion is applied
         to \texttt{abc.fit} but not to \texttt{def.fts}.
         FITS extensions are created
         in the NDFs but there are no extensions for any FITS sub-files
         that may be present.
         }
      \sstexamplesubsection{
         fits2ndf 256.fit f256 fmtcnv=f encodings=DSS
      }{
         This is the same as the first example except that it is specified 
         that the co-ordinate system information to be stored in the WCS 
         component of the NDF must be based on the FITS keywords written 
         with Digitised Sky Survey (DSS) images. If these keywords are not
         present in the FITS header then no WCS component will be created.
         All the earlier examples retained the default null value for the
         ENCODINGS parameter, resulting in the choice of keywords being
         based on the contents of the FITS header (see the description of
         the ENCODINGS parameter for details).
      }
      \sstexamplesubsection{
         fits2ndf 256.fit f256 fmtcnv=f encodings="DSS,native"
      }{
         This is the same as the previous example except that if no
         DSS keywords are available, then the co-ordinate system
         information stored in the NDF's WCS component will be based on
         keywords written by applications which use the AST library (see
         \xref{SUN/210}{sun210}{}).  One such application is
         \htmlref{NDF2FITS}{NDF2FITS}.   This `native'
         encoding provides a `loss-free' means of transferring
         information about co-ordinate systems (\textit{i.e.}\ no information is
         lost; this may not be the case with other encodings).  If the
         file \texttt{256.fit} contains neither DSS nor native AST keywords, then
         no WCS component will be created.
      }
      \sstexamplesubsection{
         fits2ndf "multifile.fit[extname=im3]" *
      }{
         This will create an NDF, multifile, from the first FITS
         extension in file \texttt{multifile.fit} whose EXTNAME keyword has
         the value "im3".
      }
      \sstexamplesubsection{
         fits2ndf multifile.fit  multifile extable=table1
      }{    
         This will create a series of NDFs in the container file
         \texttt{multifile.sdf} according to the specifications in the 
         EXTABLE-format file \texttt{table1}.
      }
   }
   \label{fits2ndf_notes}
   \sstnotes{
      Some sources of FITS files that require special conversion
      rules, particularly because they use binary tables, are
      recognised.  Details of the processing for these is given within
      topic 
      \htmlref{\texttt{"}Special Formats\texttt{"}}{special_formats}.

      Two other special cases are when a particular sub-file is specified by
      the IN parameter and when conversion is driven by an EXTABLE file.

      The general rules for the conversion apply if the FITS file is not
      one of the \texttt{"}Special Formats\texttt{"} (including one defined by an EXTABLE)
      and parameter CONTAINER is not TRUE.

      The general rules are as follows:
      \ssthitemlist{
         \sstitem 
         The primary data array of the FITS file becomes the NDF's data
         array. There is an option using parameter FMTCNV to convert
         integer data to floating point using the values of FITS keywords
         BSCALE and BZERO.
         \sstitem
         Any integer array elements with value equal to the FITS
         keyword BLANK become \xref{bad}{sun95}{se_badmasking} values in
         the NDF data array.  Likewise any floating-point data set to an
         IEEE not-a-number value also become bad values in the NDF's data
         array.  The BAD\_PIXEL flag is set appropriately.
         \sstitem
         NDF quality and variance arrays are not created.
         \sstitem
         A verbatim copy of the FITS primary header is placed in the
         NDF's FITS extension when parameter PROFITS is \texttt{TRUE}.
         \sstitem
         Here are details of the processing of standard items from the
         the FITS header, listed by FITS keyword.
         \ssthitemlist{
            \sstitem
            CRVAL\textit{n}, CDELT\textit{n}, CRPIX\textit{n}, CTYPE\textit{n},
            CUNIT\textit{n} --- define the NDF's WCS component (see parameter 
            ENCODINGS).
            \sstitem
            OBJECT, LABEL, BUNIT --- if present are equated to the NDF's
            TITLE, LABEL, and UNITS components respectively.
            \sstitem
            LBOUND\textit{n} --- if present, this specifies the pixel origin
            for the $\textit{n}^{\rm th}$ dimension.
         }
         \sstitem
         Additional sub-files within the FITS files are converted into
         extensions within the NDF if parameter PROEXTS is \texttt{TRUE}.
         These extensions are named FITS\_EXT\_\textit{m} for the 
         $\textit{m}^{\rm th}$ sub-file.
         \sstitem
         An IMAGE sub-file is treated like the primary data array, and
         follows the rules give above. However, the resultant NDF is an
         extension of the main NDF.
         \sstitem
         A BINTABLE or TABLE sub-file is converted into a structure
         of type TABLE.  This has a NROWS component specifying the
         number of rows, and a COLUMNS structure containing a series of
         further structures, each of which takes its name from the label
         of the corresponding column in the FITS table.  If there is no
         label for the $n$th column, the structure is called COLUMN$n$.
         These COLUMN structures contain a column of table data values in
         component DATA, preserving the original data type; and optional
         UNITS and COMMENT components which specify the column's units and
         the meaning of the column.  Thus for example, for the third
         sub-file of NDF called ABC, the data for column called RA would be
         located in ABC.MORE.FITS\_EXT\_3.COLUMNS.RA.DATA.
         \sstitem
         A random-group FITS file creates an NDF for each group.  As
         they are related observations the series of NDFs are stored in a
         single HDS container file whose name is still given by parameter
         OUT.  Each group NDF has component name FITS\_G\textit{n}, where 
         \textit{n} is the group number.

         Each group NDF contains the full header in the FITS extension,
         appended by the set of group parameters.  The group parameters
         are evaluated using their scales and offsets, and made to look
         like FITS cards.  The keywords of these FITS cards are derived
         from the values of PTYPE\textit{m} in the main header, where 
         \textit{m} is the number of the group parameter.
      }
   }
   \sstdiytopic{
      \label{special_formats}Special Formats
   }{
      \sstitemlist{
         \sstitem
         NDF2FITS
         \ssthitemlist{
            \sstitem
            This is recognised by the presence of an HDUCLAS1 keyword set
            to \texttt{'NDF'}.
            The conversion is similar to the general case, except
            the processing of FITS sub-files and HISTORY headers.
            \sstitem
            An IMAGE sub-file converts to an NDF VARIANCE component,
            provided the keyword HDUCLAS2 is present and has a value that is
            either \texttt{'VARIANCE'} or \texttt{'ERROR'}.
            \sstitem
            An IMAGE sub-file converts to an NDF QUALITY component,
            provided the keyword HDUCLAS2 is present and has value
            \texttt{'QUALITY'}.
            \sstitem
            FITS ASCII and binary tables become NDF extensions, however,
            the original structure path and data type are restored using
            the values of the EXTNAME and EXTTYPE keywords respectively.
            An extension may be an array of structures, the shape being stored
            in the EXTSHAPE keyword.  The shapes of multi-dimensional arrays
            within the extensions are also restored.
            \sstitem
            HISTORY cards in a special format created by NDF2FITS are
            converted back into NDF history records.  This will only work
            provided the HISTORY headers have not been tampered.  Such
            headers are not transferred to the FITS airlock, when
            PROFITS=\texttt{TRUE}.
            \sstitem
            Any SMURF package's ancillary IMAGE sub-files are restored
            to a SMURF extension, with the original names and structure
            contents.  Thus the global HISTORY present in each sub-file
            is not duplicated in each SMURF-extension NDF.
         }

         \bigskip
         \sstitem
         \htmladdnormallink{IUE Final Archive}{http://www.vilspa.esa.es/iue/iuefa.html}
         LILO, LIHI, SILO, SIHI
         \ssthitemlist{
            \sstitem
            This converts an IUE LI or SI product stored as a FITS primary
            data array and IMAGE extension containing the quality into an
            NDF.  Other FITS headers are used to create AXIS structures (SI
            products only), and character components.
            \sstitem
            Details of the conversion are:
            \ssthitemlist{
               \sstitem
               The primary data array of the FITS file becomes NDF main
               data array.  The value of parameter FMTCNV controls whether
               keywords BSCALE and BZERO are applied to scale the data;
               FMTCNV along with the number of significant characters in the
               keywords decide the data type of the array.  It is expected
               that this will be \_REAL if FMTCNV is \texttt{TRUE}, and \_WORD
               otherwise.
               \sstitem
               The quality array comes from the IMAGE extension of the
               FITS file.  The twos complement values are divided by $-$128 to
               obtain the most-significant 8 bits of the 14 in use.  There is
               no check that the dimension and axis-defining FITS headers in
               this extension match those of the main data array.  The
               standard indicates that they will be the same.
               \sstitem
               The FILENAME header value becomes the NDF's TITLE component.
               \sstitem
               The BUNIT header value becomes the NDF's UNITS component.
               \sstitem
               The CDELT\textit{n}, CRPIX\textit{n}, and CRVAL\textit{n} 
               define the axis centres.
               CTYPE\textit{n} defines the axis labels.  
               Axis information is only available for the SILO and SIHI 
               products.
               \sstitem
               The primary headers may be written to the NDF's FITS
               extension when parameter PROFITS is \texttt{TRUE}.
            }
         }
         \bigskip
         \pagebreak[2]
         \sstitem
         IUE Final Archive MXLO
         \ssthitemlist{
            \sstitem
            This will usually be a single 1-dimensional NDF, however, if
            the binary table contains two rows, a pair of NDFs are stored in
            a single HDS container file whose name is specified by parameter
            OUT.  The name of each NDF is either SMALL or LARGE depending on
            the size of the aperture used.  Thus for OUT=\texttt{ABC}, the
            small-aperture observation will be in an NDF called ABC.SMALL.
            \sstitem
            Only the most-significant 8 bits of the quality flags are
            transferred to the NDF.
            \sstitem
            The primary headers may be written to the standard FITS
            airlock extension when PROFITS is \texttt{TRUE}.
            \sstitem
            The conversion from binary-table columns and headers to NDF
            objects is as follows:
            \\[\medskipamount]
            \begin{tabular}{lp{90mm}}
            NPOINTS            &   Number of elements \\
            WAVELENGTH         &   Start wavelength, axis label and units \\
            DELTAW             &   Incremental wavelength \\
            FLUX               &   Data array, label, units, bad-pixel flag \\
            SIGMA              &   Data-error array \\
            QUALITY            &   Quality array \\
            remaining columns  &   Component in IUE\_MX extension (NET and
                                   BACKGROUND are NDFs) \\
            \end{tabular}
         }
         \bigskip
         \sstitem
         IUE Final Archive MXHI
         \ssthitemlist{
            \sstitem
            This creates a series of NDFs within a single HDS container
            file whose name is specified by parameter OUT.  Each NDF
            corresponds to a spectral order, and may be accessed individually.
            The name of each NDF is ORDER followed by the spectral-order
            number.  For instance, when OUT=\texttt{SWP}, the 85$^{th}$-order
            spectrum will be in an NDF called SWP.ORDER85.
            \sstitem
            Only the most-significant 8 bits of the quality flags are
            transferred to the NDF.
            \sstitem
            The primary headers may be written to the standard FITS
            airlock extension when PROFITS is \texttt{TRUE}.  To save space, this
            appears once in the NDF specified by parameter OUT.
            \sstitem
            The conversion from binary-table columns and headers to NDF
            objects is as follows:
            \\[\medskipamount]
            \begin{tabular}{p{32mm}p{90mm}}
            NPOINTS            &   Number of non-zero elements \\
            WAVELENGTH         &   Start wavelength of the non-zero
                                   elements, label, and units \\
            STARTPIX           &   Lower bound of the non-zero elements \\
            DELTAW             &   Incremental wavelength \\
            ABS\_CAL           &   Data array, label, and units \\
            QUALITY            &   Quality array \\
            remaining columns (except 14-17) &   Component in IUE\_MH extension
                                   (NOISE, NET, BACKGROUND, and RIPPLE are NDFs
                                   each comprising a data array, label,
                                   units and wavelength axis) \\
            \end{tabular}

            \sstitem
            It may be possible to evaluate an approximate error array for
            the absolutely calibrated data (ABS\_CAL), by multiplying the
            NOISE by the ratio ABS\_CAL / NET for each element.
            \sstitem
            The Chebyshev coefficients, limits, and scale factor in
            columns 14 to 17 are omitted as the evaluated background fit is
            propagated in BACKGROUND.
         }
         \bigskip
         \sstitem
         IUE \htmladdnormallink{INES}{http://ines.laeff.esa.es/} reduced spectra
         \ssthitemlist{
            \sstitem
            This generates a single 1-dimensional NDF. 
            \sstitem
            Only the most-significant 8 bits of the quality flags are
            transferred to the NDF.
            \sstitem
            The primary headers may be written to the standard FITS
            airlock extension when PROFITS is \texttt{TRUE}.
            \sstitem
            The conversion from binary-table columns and headers to NDF
            objects is as follows:
            \\[\medskipamount]
            \begin{tabular}{lp{90mm}}
            WAVELENGTH         &   Start wavelength, axis label and units \\
            FLUX               &   Data array, label, units, bad-pixel flag \\
            SIGMA              &   Data-error array \\
            QUALITY            &   Quality array \\
            \end{tabular}
         }
         \bigskip
         \sstitem
         \htmladdnormallink{ISO}{http://www.iso.vilspa.esa.es/users/idc/IDC.html}
         CAM auto-analysis (CMAP, CMOS)
         \ssthitemlist{
            \sstitem
            The CAM auto-analysis FITS products have a binary table
            using the `Green Bank' convention, where rows
            of the table represent a series of observations, and each
            row is equivalent to a normal simple header and data unit.  Thus
            most of the columns have the same names as
            the standard FITS keywords.
            \sstitem
            If there is only one observation, a normal NDF is produced; if
            there are more than one, the HDS container file of the supplied
            NDF is used to store a series of NDFs---one for each
            observation---called OBS\textit{n}, where \textit{n} is the 
            observation number.
            Each observation comprises three rows in the binary table
            corresponding to the flux, the r.m.s. errors, and the integration
            times.
            \sstitem
            The conversion from binary-table columns to NDF objects is as
            follows:
            \\[\medskipamount]
            \begin{tabular}{lp{90mm}}
            ARRAY              &   Data, error, exposure arrays depending
                                   on the value of column TYPE \\
            BLANK              &   Data blank (\textit{i.e.}\ undefined value) \\
            BUNIT              &   Data units \\
            BSCALE             &   Data scale factor \\
            BZERO              &   Data offset \\
            CDELT\textit{n}           &   
             Pixel increment along axis \textit{n} \\
            CRPIX\textit{n}           &   Axis \textit{n} reference pixel \\
            CRVAL\textit{n}           &   
             Axis \textit{n} co-ordinate of reference pixel \\
            CTYPE\textit{n}           &   Label for axis \textit{n} \\
            CUNIT\textit{n}           &   Units for axis \textit{n} \\
            NAXIS              &   Number of dimensions \\
            NAXIS\textit{n}           &   Dimension of axis \textit{n} \\
            remaining columns  &   keyword in FITS extension \\
            \end{tabular}
            \\[\medskipamount]
            Some of these remaining columns overwrite the (global) values
            in the primary headers.  The integration times are stored as
            an NDF within an extension called EXPOSURE.
            \\[\medskipamount]
            The creation of axis information and extensions does not occur
            for the error array, as these are already generated when the
            data-array row in the binary table is processed.
            \\[\medskipamount]
            The BITPIX column is ignored as the data type is determined
            through the use the TFORM\textit{n} keyword and the value of 
            FMTCNV in conjunction with the BSCALE and BZERO columns.
         }
         \bigskip
         \sstitem
         ISO LWS auto-analysis (LWS AN)
         \ssthitemlist{
            \sstitem
            The conversion from binary-table columns to NDF objects is
            as follows:
            \\[\medskipamount]
            \begin{tabular}{lp{90mm}}
               LSANFLX            &   Data array, label, and units \\
               LSANFLXU           &   Data errors, hence variance \\
               LSANDET            &   Quality (bits 1 to 4) \\
               LSANSDIR           &   Quality (bit 5) \\
               LSANRPID           &   Axis centres, labels, and units
                                      \textit{x}-\textit{y} positions---
                                      dimensions 1 and 2) \\
               LSANSCNT           &   Axis centre, label, and unit (scan
                                      (count index---dimension 4) \\
               LSANWAV            &   Axis centre, label, and unit
                                      (wavelength---dimension 3) \\
               LSANWAVU           &   Axis errors (wavelength---dimension 3) \\
               LSANFILL           &   not copied \\
               remaining columns  &   column name in LWSAN extension \\
            \end{tabular}
         }
         \bigskip
         \sstitem
         ISO SWS auto-analysis (SWS AA)
         \ssthitemlist{
            \sstitem
            The conversion from binary-table columns to NDF objects is
            as follows:
            \\[\medskipamount]
            \begin{tabular}{ll}
               SWAAWAVE           &   Axis centres, label, and units \\
               SWAAFLUX           &   Data array, label, and units \\
               SWAASTDV           &   Data errors, hence variance \\
               SWAADETN           &   Quality \\
               SWAARPID           &   not copied \\
               SWAASPAR           &   not copied \\
               remaining columns  &   column name in SWSAA extension \\
            \end{tabular}
         }
         \bigskip
         \sstitem
         AAO \htmladdnormallink{2dF}{http://www.aao.gov.au/local/www/2df/}
         \ssthitemlist{
            \sstitem
            The conversion is restricted to a 2dF archive FITS file
            created by task NDF2FITS.  FITS2NDF restores the original NDF.
            It creates the 2dF FIBRES extension and its constituent
            structures, and NDF\_CLASS extension.  In addition the variance,
            axes, and HISTORY records are converted.
            \sstitem
            The HISTORY propagation assumes that the FITS HISTORY headers
            have not been tampered.
            \sstitem
            Details of the conversion are:
            \ssthitemlist{
               \sstitem
               The primary data array becomes the NDF's data array.  Any
               NaN values present become \xref{bad}{sun95}{se_badmasking} values in the NDF.
               \sstitem
               The keywords CRVAL\textit{n}, CDELT\textit{n}, CRPIX\textit{n},
               CTYPE\textit{n}, CUNIT\textit{n} are
               used to create the NDF axis centres, labels, and units.
               \sstitem
               The OBJECT, LABEL, BUNIT keywords define the NDF's TITLE,
               LABEL, and UNITS components respectively, if they are defined.
               \sstitem
               HISTORY cards in a special format created by NDF2FITS are
               converted back into NDF history records.
               \sstitem
               The NDF variance is derived from the data array of an
               IMAGE extension (usually the first), if present, provided the
               IMAGE extension headers have an HDUCLAS2 keyword whose value
               is either \texttt{'VARIANCE'} or \texttt{'ERROR'}.
               \sstitem
               The NDF\_CLASS extension within the NDF is filled using the
               a FITS binary-table extension whose EXTNAME keyword's value is
               NDF\_CLASS.  Note: no error is reported if this extension does
               not exist within the FITS file.
               \sstitem
               The FIBRES extension is created from another FITS binary table
               whose EXTNAME keyword's value is FIBRES.  The OBJECT
               substructure's component names, data types, and values are
               taken from the binary-table columns themselves, and the
               components of the FIELD substructure are extracted from
               recognised keywords in the binary-table's header.  Note:
               no error is reported if this extension does not exist within
               the FITS file.
               \sstitem
               A FITS extension in the NDF may be written to store the
               primary data unit's headers when parameter PROFITS is
               \texttt{TRUE}.  This FITS airlock will not contain any
               NDF-style HISTORY records.
            }
         }
      }
   }
   \sstdiytopic{
      References
   }{
      \begin{refs}
      \item Bailey, J.A. 1997, 2dF Software Report 14, version 0.5.
      \item NASA Office of Standards and Technology, 1994, {\it A User's Guide
       for the Flexible Image Transport System (FITS)}, version 3.1.
      \item NASA Office of Standards and Technology, 1995, {\it Definition of
       the Flexible Image Transport System (FITS)}, version 1.1.
      \end{refs}
   }
   \sstdiytopic{
      Related Applications
   }{
      \CONVERT: \htmlref{MTFITS2NDF}{MTFITS2NDF}, \htmlref{NDF2FITS}{NDF2FITS};
      \CURSA: \xref{xcatview}{sun190}{XVIEW};
      \KAPPA: \xref{FITSDIN}{sun95}{FITSDIN}, \xref{FITSIN}{sun95}{FITSIN}.
   }
}
\newpage
\sstroutine{
   GASP2NDF
}{
   Converts an image in GASP format to an NDF
}{
   \sstdescription{
      This application converts a GAlaxy Surface Photometry (GASP)
      format file into an \NDFref.
   }
   \sstusage{
      gasp2ndf in out shape=?
   }
   \sstparameters{
      \sstsubsection{
         IN = FILENAME (Read)
      }{
         A character string containing the name of GASP file to convert.
         The extension should not be given, since \texttt{".dat"} is assumed.
      }
      \sstsubsection{
         OUT = NDF (Write)
      }{
         The name of the output NDF.
      }
      \sstsubsection{
         SHAPE( 2 ) = \_INTEGER (Read)
      }{
         The dimensions of the GASP image (the number of columns
         followed by the number of rows).  Each dimension must be in the
         range 1 to 1024.  This parameter is only used if supplied on
         the command line, or if the header file corresponding to the
         GASP image does not exist or cannot be opened.
      }
   }
   \sstexamples{
      \sstexamplesubsection{
         gasp2ndf m31\_gasp m31
      }{
         Convert a GASP file called \texttt{m31\_gasp.dat} into an NDF called
         m31.  The dimensions of the image are taken from the header file
         \texttt{m31\_gasp.hdr}.
      }
      \sstexamplesubsection{
         gasp2ndf n1068 ngc1068 shape=[256,512]
      }{
         Take the pixel values in the GASP file \texttt{n1068.dat} and create
         the NDF \texttt{ngc1068} with dimensions 256 columns by 512 rows.
      }
   }
   \sstnotes{
      \sstitemlist{

         \sstitem
         A GASP image is limited to a maximum of 1024 by 1024 elements.
         It must be two dimensional.

         \sstitem
         The GASP image is written to the NDF's data array.  The data
         array has type \_WORD. No other NDF components are created.

         \sstitem
         If the header file is corrupted, you must remove the offending
         \texttt{".hdr"} file or specify the shape of the GASP image on the
         command line, otherwise the application will continually abort.
      }
   }
   \sstdiytopic{
      Related Applications
   }{
      \CONVERT: \htmlref{NDF2GASP}{NDF2GASP}.
   }
   \sstdiytopic{
      References
   }{
      GASP documentation (MUD/66).
   }
}

\newpage
\sstroutine{
   GIF2NDF
}{
   Converts a GIF file into an NDF. 
}{
   \sstdescription{
      This Bourne-shell script converts a Graphics Interchange Format
      (\GIFref) file into an unsigned-byte (256 grey-level) \NDFref\ format file.
      It handles one- or two-dimensional images.  The script uses
      various \Netpbmref\ utilities to produce a FITS file, flipped top
      to bottom, and then \htmlref{FITS2NDF}{FITS2NDF} to produce the final
      NDF. 
      Error messages are converted into Starlink style (preceded by \texttt{!}).
   }
   \sstusage{
      gif2ndf in [out]
   }
   \sstparameters{
      \sstsubsection{
         IN = FILENAME (Read)
      }{
         The name of the GIF file to be converted. (A \texttt{.gif} name
         extension is assumed if omitted.)
      }
      \sstsubsection{
         OUT = NDF (Write)
      }{
         The name of the NDF to be generated (without the \texttt{.sdf}
         extension).
         If the OUT parameter is omitted, the value of the IN parameter
         is used.
      }
   }
   \sstexamples{
      \sstexamplesubsection{
         gif2ndf old new
      }{
         This converts the GIF file \texttt{old.gif} into an NDF called new
         (in file \texttt{new.sdf}).
      }
      \sstexamplesubsection{
         gif2ndf horse
      }{
         This converts the GIF file \texttt{horse.gif} into an NDF called horse
         (in file \texttt{horse.sdf}).
      }
   }

   \sstnotes{ 
      The following points should be remembered:
      \ssthitemlist{ 

         \sstitem
            This initial version of the script generates images with at most
            256 grey levels.  It does not use the image colour lookup table.

         \sstitem
            Input image filenames must have the file extension \texttt{.gif}.

         \sstitem
            The \Netpbm\ utilities \texttt{giftopnm}, \texttt{ppmtopgm},
            \texttt{pnmflip} and \texttt{pnmtofits} must be available on your
            PATH.
      }
   }
   \sstdiytopic{
      Related Applications
   }{
      \CONVERT: \htmlref{NDF2GIF}{NDF2GIF}.
   }
}

\newpage
\sstroutine{
   IRAF2NDF
}{
   Converts an IRAF image to an NDF
}{
   \sstdescription{
      This application converts an \IRAFref\ image to an \NDFref.  See the 
      \texttt{"}Notes\texttt{"} for details of the conversion. 
   }
   \sstusage{
      iraf2ndf in out
   }
   \sstparameters{
      \sstsubsection{
         IN = LITERAL (Read)
      }{
         The name of the IRAF image.  Note that this excludes the
         \texttt{".imh"} file extension.
      }
      \sstsubsection{
         OUT = NDF (Write)
      }{
         The name of the NDF to be produced.
      }
      \sstsubsection{
         PROFITS = \_LOGICAL (Read)
      }{
         If \texttt{TRUE}, the user headers of the IRAF file are written
         verbatim to the NDF's FITS extension.  Any IRAF history
         records are also appended to the FITS extension.  The FITS
         extension is not created if there are no user headers
         present in the IRAF file.  \texttt{[TRUE]}
      }
      \sstsubsection{
         PROHIS = \_LOGICAL (Read)
      }{
         This parameter decides whether or not to create NDF HISTORY
         records.  Only the IRAF headers with keyword HISTORY, and
         which originated from NDF HISTORY records are used.  If
         PROHIS=\texttt{TRUE}, NDF HISTORY records are created.  \texttt{[TRUE]}
      }
   }
   \sstexamples{
      \sstexamplesubsection{
         iraf2ndf ell\_galaxy new\_galaxy
      }{
         Converts the IRAF image ell\_galaxy (comprising files
         \texttt{ell\_galaxy.imh} and \texttt{ell\_galaxy.pix}) to an NDF
         called new\_galaxy.
      }
      \sstexamplesubsection{
         iraf2ndf ell\_galaxy new\_galaxy noprofits noprohis
      }{
         As above, except no FITS extension is created, and NDF-style
         HISTORY lines in \texttt{ell\_galaxy.imh} are not transferred to 
         HISTORY records in NDF new\_galaxy.
      }
   }
   \sstnotes{
      The rules for the conversion are as follows:
      \sstitemlist{
         \latex{\vspace{4.5ex}}

         \sstitem
         The NDF is created with bounds determined by any LBOUND$n$
         keywords in the IRAF image header.

         \sstitem
         The NDF data array is copied from the \texttt{".pix"} file.

         \sstitem
         The title of the IRAF image (object i\_title in the \texttt{".imh"}
         header file) becomes the NDF title.  Likewise headers OBJECT and
         BUNIT become the NDF label and units respectively.

         \sstitem
         The pixel origin is set if any LBOUND\textit{n} headers are present.

         \sstitem
         Lines from the IRAF image header file may be transferred to
         the FITS extension of the NDF, when PROFITS=\texttt{TRUE}.  Any
         compulsory FITS keywords that are missing are added.  Certain
         other keywords are not propagated.  These are the IRAF ``Mini
         World Co-ordinate System'' (MWCS) keywords WCSDIM, DC\_FLAG,
         WAT\textit{d}\_\textit{nnn} (\textit{d} is dimension, \textit{nnn}
         is the line number).
         Certain NDF-style HISTORY lines in the header are also
         be ignored when PROHIS=\texttt{TRUE} (see two notes below).

         \sstitem
         When PROFITS=\texttt{TRUE}, lines from the HISTORY section of the IRAF
         image are also extracted and added to the NDF's FITS extension as
         FITS HISTORY lines.  Two extra HISTORY lines are added to record
         the original name of the image and the date of the format
         conversion.

         \sstitem
         When PROHIS=\texttt{TRUE}, any HISTORY lines in the IRAF headers, which
         originated from an NDF2IRAF conversion of NDF HISTORY records.
         Such headers are not transferred to the FITS airlock, when
         PROFITS=\texttt{TRUE}.

         \sstitem
         Most axis information can be propagated either from standard
         FITS-like keywords, or certain MCWS headers.  Supported systems
         and formats are listed below.
         \ssthitemlist{

            \sstitem
            FITS
            \ssthitemlist{
               \sstitem
               linear

               \sstitem
               log-linear
            }

            \sstitem
            Equispec
            \ssthitemlist{

               \sstitem
               linear

               \sstitem
               log-linear
            }

            \sstitem
            Multispec
            \ssthitemlist{

               \sstitem
               linear

               \sstitem
               log-linear

               \sstitem
               Chebyshev and Legendre polynomials

               \sstitem
               Linear and cubic Spline

               \sstitem
               Explicit list of co-ordinates
            }
         }

         However, for Multispec axes, only the first (spec1) axis
         co-ordinates are transferred to the NDF AXIS centres.  Any
         spec2 \dots spec\textit{n} co-ordinates, present when the data array 
         is not one-dimensional or multiple fits have been stored, are ignored.
         The weights for multiple fits are thus also ignored.  The data
         type of the axis centres is \_REAL or \_DOUBLE depending on the
         number of significant digits in the co-ordinates or coefficients.

         The axis labels and units are also propagated, where present, to
         the NDF AXIS structure.  In the FITS system, these are derived
         from the CTYPE\textit{n} and CUNIT\textit{n} keywords.  
         In the MWCS, these components originate in the label and units 
         parameters.

         The redshift correction, when present, is applied to the MCWS
         axis co-ordinates.

      }
   }
   \sstdiytopic{
      Related Applications
   }{
      \CONVERT: \htmlref{NDF2IRAF}{NDF2IRAF}.
   }
   \sstdiytopic{
      Pitfalls
   }{
      \sstitemlist{

         \sstitem
         \xref{Bad pixels}{sun95}{se_masking} in the IRAF image are not replaced.

         \sstitem
         Some of the routines required for accessing the IRAF header
         file are written in SPP.  Macros are used to find the start of the
         header line section, this constitutes an `Interface violation' as
         these macros are not part of the IMFORT interface specification.
         It is possible that these may be changed in the future, so
         beware.
      }
   }
   \sstdiytopic{
      References
   }{
      IRAF User Handbook Volume 1A: \textit{A User's Guide to FORTRAN
      Programming in IRAF, the IMFORT Interface}, by Doug Tody.
   }
   \sstdiytopic{
      Keywords
   }{
      CONVERT, IRAF
   }
   \sstimplementationstatus{
      \sstitemlist{

         \sstitem
         Only handles one-, two-, and three-dimensional IRAF files.

         \sstitem
         The NDF produced has type \_WORD or \_REAL corresponding to the
         type of the IRAF image.  (The IRAF IMFORT FORTRAN subroutine
         library only supports these data types: signed words and real.)
         The pixel type of the image can be changed from within IRAF using
         the `chpixtype' task in the `images' package.

         \sstitem
         See  \texttt{"}Release Notes\texttt{"} for IRAF Version compatibility.
      }
   }
}

\newpage
\sstroutine{
   IRCAM2NDF
}{
   Converts an IRCAM data file to a series of NDFs
}{
   \sstdescription{
      This applications converts an \HDSref\ file in the IRCAM format listed
      in IRCAM User Note 11 to one or more {\NDFref}s.  See the 
      \htmlref{\texttt{"}Notes\texttt{"}}{ircam2ndf_notes} for a detailed list
      of the rules of the conversion.
   }
   \sstusage{
      ircam2ndf in prefix obs [config]
   }
   \sstparameters{
      \sstsubsection{
         CONFIG = \xref{LITERAL}{sun95}{se_parmenu} (Read)
      }{
         The choice of data array to place in the NDF.  It can have one
         of the following configuration values:
            \begin{itemize}
            \item \texttt{"STARE"} --- the image of the object or sky;
            \item \texttt{"CHOP"} --- the chopped image of the sky;
            \item \texttt{"KTCSTARE"} --- the electronic pedestal or bias
                           associated with the stare image of the object or sky;
            \item\texttt{"KTCCHOP"} --- the electronic pedestal or bias
                           associated with the chopped image of the sky.
            \end{itemize}
         Note that at the time of writing chopping has not been
         implemented for IRCAM.  For practical purposes CONFIG=\dqt{STARE}
         should be used.  The suggested default is the current value.
         \texttt{["STARE"]}
      }
      \sstsubsection{
         FMTCNV = \_LOGICAL (Read)
      }{
         This specifies whether or not format conversion may occur.
         If FMTCNV is \texttt{FALSE}, the data type of the data array in the NDF
         will be the same as that in the IRCAM file, and there is no
         scale factor and offset applied.  If FMTCNV is \texttt{TRUE}, whenever
         the IRCAM observation has non-null scale and offset values,
         the observation data array will be converted to type \_REAL in
         the NDF, and the scale and offset applied to the input data
         values to give the `true' data values.  A null scale factor is
         1 and a null offset is 0. \texttt{[FALSE]}
      }
      \sstsubsection{
         IN = IRCAM (Read)
      }{
         The name of the input IRCAM file to convert to NDFs.  The
         suggested value is the current value.
      }
      \sstsubsection{
         OBS()  = LITERAL (Read)
      }{
         A list of the observation numbers to be converted into NDFs.
         Observations are numbered consecutively from 1 up to the
         actual number of observations in the IRCAM file.  Single
         observations or a set of adjacent observations may be
         specified, {\it e.g.}\ entering \texttt{[4,6-9,12,14-16]} will read
         observations 4,6,7,8,9,12,14,15,16.  (Note that the brackets
         are required to distinguish this array of characters from a
         single string including commas.  The brackets are unnecessary
         when there is only one item.)

         If you wish to extract all the observations enter the wildcard
         \texttt{$*$}.  \texttt{5-$*$} will read from 5 to the last
         observation.  The processing will continue until the last observation
         is converted.  The suggested value is the current value.
      }
      \sstsubsection{
         PREFIX = LITERAL (Read)
      }{
         The prefix of the output NDFs.  The name of an NDF is the
         prefix followed by the observation number.  The suggested
         value is the current value.
      }
   }
   \sstexamples{
      \sstexamplesubsection{
         ircam2ndf ircam\_27aug89\_1cl rhooph obs=$*$
      }{
         This converts the IRCAM data file called \texttt{ircam\_27aug89\_1cl}
         into a series of NDFs called rhooph1, rhooph2 {\it etc.}\  
         There is no format conversion applied.
      }
      \sstexamplesubsection{
         ircam2ndf ircam\_27aug89\_1cl rhooph [32,34-36] fmtcnv
      }{
         This converts four observations in the IRCAM data file called
         \texttt{ircam\_27aug89\_1cl} into NDFs called rhooph32, 
         rhooph34, rhooph35, rhooph36.  The scale and offset
         are applied.
      }
      \sstexamplesubsection{
         ircam2ndf in=ircam\_04nov90\_1c config="KTC" obs=5 prefix=bias
      }{
         This converts the fifth observation in the IRCAM data file
         called \texttt{ircam\_04nov90\_1c.sdf} into an NDF called bias5
         containing the electronic pedestal in its data array.  There is no 
         format conversion applied.
      }
   }
   \label{ircam2ndf_notes}
   \sstnotes{
      The rules for the conversion of the various components are as
      follows: \vspace*{-\medskipamount}
      \begin{center}
      \begin{tabular}{|lcl|p{38mm}|}
      \hline 
      \multicolumn{1}{|l}{IRCAM file} & & \multicolumn{1}{l}{NDF} &
      \multicolumn{1}{|c|}{Comments} \\ \hline
      .OBS.PHASEA.DATA\_ARRAY & $\Rightarrow$ &  .DATA\_ARRAY & 
          when parameter CONFIG=\texttt{"STARE"} \\
      .OBS.PHASEB.DATA\_ARRAY & $\Rightarrow$ &  .DATA\_ARRAY &
          when parameter CONFIG=\texttt{"CHOP"} \\
      .OBS.KTCA.DATA\_ARRAY   & $\Rightarrow$ &  .DATA\_ARRAY &
          when parameter CONFIG=\texttt{"KTCSTARE"} \\
      .OBS.KTCB.DATA\_ARRAY   & $\Rightarrow$ &  .DATA\_ARRAY &
          when parameter CONFIG=\texttt{"KTCCHOP"} \\
      & & & \\
      .OBS.DATA\_LABEL        & $\Rightarrow$ &  .LABEL & \\
      .OBS.DATA\_UNITS        & $\Rightarrow$ &  .UNITS & \\
      .OBS.TITLE              & $\Rightarrow$ &  .TITLE &
          If .OBS.TITLE is a blank string, OBS.DATA\_OBJECT is 
          copied to the NDF title instead. \\
      .OBS.AXIS1\_LABEL       & $\Rightarrow$ &  .AXIS(1).LABEL & \\
      .OBS.AXIS2\_LABEL       & $\Rightarrow$ &  .AXIS(2).LABEL & \\
      .OBS.AXIS1\_UNITS       & $\Rightarrow$ &  .AXIS(1).UNITS & \\
      .OBS.AXIS2\_UNITS       & $\Rightarrow$ &  .AXIS(2).UNITS & \\ \hline
      \end{tabular}
      \end{center}

      \begin{center}
      \begin{tabular}{|lcl|p{40mm}|}
      \hline 
      \multicolumn{1}{|l}{IRCAM file} & & \multicolumn{1}{l}{NDF} &
      \multicolumn{1}{|c|}{Comments} \\ \hline
      \multicolumn{3}{|p{100mm}|}{
      .GENERAL.INSTRUMENT.PLATE\_SCALE 
          becomes the increment between the axis centres, with co-ordinate
          (0.0,0.0) located at the image centre.  The NDF axis units both
          become \texttt{"arcseconds"}. } & \\
      & & & \\
      .GENERAL               & $\Rightarrow$ &  .MORE.IRCAM.GENERAL & \\
      .GENERAL.x             & $\Rightarrow$ &  .MORE.IRCAM.GENERAL.x & \\
      .GENERAL.x.y           & $\Rightarrow$ &  .MORE.IRCAM.GENERAL.x.y & \\
      & & & \\
      .OBS.x                 & $\Rightarrow$ &  .MORE.IRCAM.OBS.x &
          This excludes the components of OBS already listed above and
          DATA\_BLANK. \\ \hline
      \end{tabular}
      \end{center}

      \sstitemlist{

         \sstitem
         The data types of the IRCAM GENERAL structures have not been
         propagated to the NDF IRCAM extensions, because it would violate
         the rules of \xref{SGP/38}{sgp38}{}.  In the IRCAM file these all
         have the same type STRUCTURE.  The new data types are as follows:

      \begin{center}
      \begin{tabular}{|l|l|}
      \hline 
      \multicolumn{1}{|c|}{Extension Name} & \multicolumn{1}{c|}{Data Type} \\ \hline
      IRCAM.GENERAL & IRCAM\_GENERAL \\
      IRCAM.GENERAL.INSTRUMENT & IRCAM\_INSTRUM \\
      IRCAM.GENERAL.ID & IRCAM\_ID \\
      IRCAM.GENERAL.TELESCOPE & IRCAM\_TELESCOPE \\ \hline
      \end{tabular}
      \end{center}

         \sstitem
         Upon completion the number of observations
         successfully converted to NDFs is reported.
      }
   }
   \sstdiytopic{
      \xref{Bad-pixel}{sun95}{se_masking} Handling
   }{
      Elements of the data array equal to the IRCAM component
      .OBS.DATA\_BLANK are replaced by the 
      \xref{standard bad value}{sun95}{se_badmasking}.
   }
   \sstimplementationstatus{
      \sstitemlist{

         \sstitem
         The data array in the NDF is in the primitive form.

         \sstitem
         The application aborts if the data array chosen by parameter
         CONFIG does not exist in the observation.
      }
   }
}

\newpage
\sstroutine{
   MTFITS2NDF
}{
      Converts FITS magnetic tape files into NDFs.
}{
   \sstdescription{
      This application converts files from a \FITSref\ tape into {\NDFref}s by
      using shell commands \textbf{mt} and \textbf{dd} to position the
      tape and convert the selected tape files into FITS disk files and 
      then using \htmlref{FITS2NDF}{FITS2NDF} to
      produce the NDFs.  The intermediate FITS files may be saved.
   }
   \sstusage{
      mtfits2ndf in out block=n [of=fits\_file] [$<$fits2ndf\_pars$>$]
   }
   \sstparameters{
      \sstsubsection{
         BLOCK = \_INTEGER (Read)
      }{
         The FITS blocking factor, \textit{i.e.}\ the block size on the tape is
         this value multiplied by the standard FITS block size.  The
         suggested default is 10.
      }
      \sstsubsection{
         IN = DEVICE (Read)
      }{
         The name of a tape device.  For correct tape positioning, a no-rewind
         device must be used.  The device name may have file specifiers
         appended, separated by commas and enclosed in \texttt{[]}.  The file specifiers
         indicate which files from the tape are to be processed.
         For example:
         \texttt{[2]} indicates the second file on the tape.
         \texttt{[4-6]} indicates files 4 to 6.
         \texttt{[5-]} indicates file 5 to the last file on the tape.
         \texttt{[1,3-5,7-]} indicates files 1,3,4,5, and 7 to the end of the tape.
         If no file specifiers are given, all files on the tape will be
         processed.
      }
      \sstsubsection{
         OF = LITERAL (Read)
      }{
         Name(prefix) of the intermediate FITS file(s) copied from the tape.
         Only set this if you want to save the intermediate FITS file(s).  If a
         number of files are being produced, the name should contain a \texttt{*},
         which will be replaced by the corresponding FITS tape file number.
         If OF is not specified, \texttt{mtf2ndf*.fits} will be used and deleted.
         (See also \texttt{"}Notes\texttt{"}).
      }
      \sstsubsection{
         OUT = NDF (Write)
      }{
         The name of the NDF(s) to be produced by FITS2NDF.  This is passed to
         FITS2NDF but only a single element string can be specified---it can
         contain the matching patterns allowed for FITS2NDF, for example \texttt{"*"}.
      }
      \sstsubsection{
         \texttt{$<$fits2ndf\_pars$>$}
      }{
         Other parameters will be passed to FITS2NDF---see the description of
         \htmlref{FITS2NDF}{FITS2NDF}.
      }
   }

   \sstnotes{
      \sstitemlist{
      \sstitem This application is a \texttt{tcsh} script which calls an 
      ADAM A-task.
      \CONVERT\ startup sets alias \texttt{mtfits2ndf} to
      `\texttt{tcsh mtfits2ndf.tcsh}',
      and \texttt{tcsh} must be on the user's PATH.
      \sstitem The string specified for the intermediate FITS file name(s)
      will be  presented as the IN parameter for the FITS2NDF call. All files
      matching the string will be used, whether or not they were produced
      in this run. (See the \htmlref{FITS2NDF}{FITS2NDF} description for
      details.)
      }
   }

   \sstexamples{
      \sstexamplesubsection{
         mtfits2ndf /dev/nst0[2] f256 block=10 fmtcnv=f
      }{
         This converts the second file on the tape on device 
         \texttt{/dev/nst0} to an NDF called \texttt{f256}.  The FITS
         blocking factor of the tape is 10.
         As a result of the parameters passed to FITS2NDF, the data type of
         the NDF's data array matches that of the FITS primary data array,
         a FITS extension is created in \texttt{f256}, and FITS sub-files are
         propagated to NDF extensions.
      }
      \sstexamplesubsection{
         mtfits2ndf /dev/nst0 * block=1 of=ral256\_*.fit
      }{
         Will convert each file on the tape on device \texttt{/dev/nst0}
         (with a blocking factor of 1) to FITS disk files named 
         \texttt{ral256\_*.fit}, where \texttt{*} is replaced by each tape file number.
         The FITS files will be converted to NDFs named \texttt{ral256\_*} and
         retained.
      }
      \sstexamplesubsection{
         mtfits2ndf
      }{
        The user is prompted for the input device, the output NDF name and
        the FITS blocking factor.  All other parameters are defaulted.
      }
   }
   \sstdiytopic{
      References
   }{
      \begin{refs}
         \item NASA Office of Standards and Technology, 1997, \textit{A User's Guide
               for the Flexible Image Transport System (FITS)}, version 4.0.
         \item NASA Office of Standards and Technology, 1999, \textit{Definition of
               the Flexible Image Transport System (FITS)}.
      \end{refs}
   }
   \sstdiytopic{
      Deficiencies
   }{
      \sstitemlist{
         \item Facilities for naming multiple output files are limited.
         \item The command is not available from ICL, nor as an option in
                  the automatic (on-the-fly) conversion system.
         \item Extensions within a FITS file may not be specified.  However,
                  it is possible to pass an EXTABLE parameter to the FITS2NDF
                  operation to select extensions.
         \item Only tape devices for which the \textbf{mt} and \textbf{dd} commands will work
                  may be used.
      }
   }
   \sstdiytopic{
      Related Applications
   }{
      \KAPPA: \xref{FITSDIN}{sun95}{FITSDIN}, \xref{FITSIN}{sun95}{FITSIN};
      \CONVERT: \htmlref{FITS2NDF}{FITS2NDF}.
   }
}

\newpage
\sstroutine{
   NDF2ASCII
}{
   Converts an NDF to a text file
}{
   \sstdescription{
      This application converts an \NDFref\ to a Fortran text file.  Only one of
      the array components may be copied to the output file.  Preceding
      the data there is an optional header consisting of either the
      FITS extension with the values of certain keywords replaced by
      information derived from the NDF, or a minimal FITS header also
      derived from the NDF.  The output file uses LIST carriagecontrol.
   }
   \sstusage{
      ndf2ascii in out [comp] [reclen] noperec=?
   }
   \sstparameters{
      \sstsubsection{
         COMP = \xref{LITERAL}{sun95}{se_parmenu} (Read)
      }{
         The NDF component to be copied.  It may be \texttt{"Data"},
         \texttt{"Quality"} or \texttt{"Variance"}. \texttt{["Data"]}
      }
      \sstsubsection{
         FITS = \_LOGICAL (Read)
      }{
         If \texttt{TRUE}, any FITS extension is written to start of the output
         file, unless there is no extension whereupon a minimal FITS
         header is written to the ASCII file. \texttt{[FALSE]}
      }
      \sstsubsection{
         FIXED = \_LOGICAL (Read)
      }{
         When FIXED is \texttt{TRUE}, the output file allocates a fixed number
         of characters per data value.  The number of characters chosen
         is the minimum that prevents any loss of precision, and hence
         is dependent on the data type of the NDF array.  These widths
         in characters for each HDS data type are as follows: \_UBYTE, 3;
         \_BYTE, 4; \_UWORD, 5; \_WORD, 6; \_INTEGER, 11; \_REAL, 16; and
         \_DOUBLE, 24.  The record length is the product of the number
         of characters per value plus one (for a delimiting space),
         times the number of values per record given by parameter
         NOPEREC, up to a maximum of 512.

         When FIXED is \texttt{FALSE}, data values are packed as efficiently
         as possible within each record.  The length of each record is
         given by parameter RECLEN.  \texttt{[FALSE]}
      }
      \sstsubsection{
         IN = NDF (Read)
      }{
         Input NDF data structure. The suggested default is the current
         NDF if one exists, otherwise it is the current value.
      }
      \sstsubsection{
         NOPEREC = \_INTEGER (Read)
      }{
         The number of data values per record of the output file, when
         FIXED is \texttt{TRUE}.  It should be positive on UNIX platforms.
         The suggested default is the current value, or 8 when there
         is not one.  The upper limit is given by 512 divided by the
         number of characters per value plus 1 (see parameter FIXED).
      }
      \sstsubsection{
         OUT = FILENAME (Write)
      }{
         Name of the output formatted Fortran file.  The file will
         normally have variable-length records when there is a header,
         but always fixed-length records when there is no header.
      }
      \sstsubsection{
         RECLEN = \_INTEGER (Read)
      }{
         The maximum record length in bytes (characters) of the output
         file.  This has a maximum length of 512 (for efficiency
         reasons), and must be greater than 31 on UNIX systems.  The
         lower limit is further increased to 80 when there is a FITS
         header to be copied.  It is only used when FIXED is \texttt{FALSE} and
         will default to the current value, or 132 if there is no
         current value.  \texttt{[]}
      }
   }
   \sstexamples{
      \sstexamplesubsection{
         ndf2ascii cluster cluster.dat
      }{
         This copies the data array of the NDF called cluster to a text
         file called \texttt{cluster.dat}.  The maximum recordlength of
         \texttt{cluster.dat} is 132 bytes, and the data values are packed into
         these records as efficiently as possible.
      }
      \sstexamplesubsection{
         ndf2ascii cluster cluster.dat v
      }{
         This copies the variance of the NDF called cluster to a
         text file called \texttt{cluster.dat}.  The maximum recordlength of
         \texttt{cluster.dat} is 132 bytes, and the variance values are packed
         into these records as efficiently as possible.
      }
      \sstexamplesubsection{
         ndf2ascii cluster cluster.dat fixed noperec=12
      }{
         This copies the data array of the NDF called cluster to a
         text file called \texttt{cluster.dat}.  There are twelve data values
         per record in \texttt{cluster.dat}.
      }
      \sstexamplesubsection{
         ndf2ascii out=ndf234.dat fits reclen=80 in=@234
      }{
         This copies the data array of the NDF called 234 to a
         text file called \texttt{ndf234.dat}.  The maximum recordlength of
         \texttt{ndf234.dat} is 80 bytes, and the data values are packed into
         these records as efficiently as possible.  If there is a FITS
         extension, it is copied to \texttt{ndf234.dat} with substitution of
         certain keywords, otherwise a minimal FITS header is produced.
      }
   }
   \sstnotes{
      The details of the conversion are as follows:
      \ssthitemlist{

         \sstitem
            the NDF array as selected by COMP is written to the ASCII
            file in records following an optional header.  When FIXED is
            \texttt{FALSE} all records are padded out to the recordlength.

         \sstitem
            HISTORY is not propagated.

         \sstitem
            ORIGIN information is lost.

         \sstitem
            When a header is to be made, it is composed of FITS-like card
            images as follows:
         \ssthitemlist{

            \sstitem
               The number of dimensions of the data array is written
               to the keyword NAXIS, and the actual dimensions to NAXIS1,
               NAXIS2 {\it etc.} as appropriate.

            \sstitem
               If the NDF contains any linear axis structures the
               information necessary to generate these structures is
               written to the FITS-like headers. For example, if a linear
               AXIS(1) structure exists in the input NDF the value of the
               first data point is stored with the keyword CRVAL1,
               and the incremental value between successive axis data is
               stored in keyword CDELT1.  By definition the reference pixel is
               1.0 and is stored in keyword CRPIX1.  If there is an axis label
               it is written to keyword CTYPE1, and axis unit is written to CUNIT1.
               (Similarly for AXIS(2) structures {\it etc.}) FITS does not have
               a standard method of storing axis widths and variances, so these
               NDF components will not be propagated to the header.
               Non-linear axis data arrays cannot be represented by CRVAL{\em{n}}
               and CDELT{\em{n}}, and must be ignored.

            \sstitem
               If the input NDF contains TITLE, LABEL, or UNITS components
               these are stored with the keywords TITLE, LABEL, or BUNIT
               respectively.

            \sstitem
               If the input NDF contains a FITS extension, the FITS items
               may be written to the FITS-like header, with the following
               exceptions:
               \begin{itemize}
               \item BITPIX is derived from the type of the NDF data array,
               and so it is not copied from the NDF FITS extension.
               \item NAXIS, and NAXIS{\em{n}} are derived from the dimensions of the
               NDF data array as described above, so these items are not
               copied from the NDF FITS extension.
               \item The TITLE, LABEL, and BUNIT descriptors are only copied
               if no TITLE, LABEL, and UNITS NDF components respectively
               have already been copied into these headers.
               \item The CDELT{\em{n}}, CRVAL{\em{n}}, CTYPE{\em{n}},
               CUNIT{\em{n}}, and CRTYPE{\em{n}} descriptors
               in the FITS extension are only copied if the input NDF
               contained no linear axis structures.
               \item The standard order of the FITS keywords is preserved,
               thus BITPIX, NAXIS and AXIS{\em{n}} appear immediately after the
               first card image, which should be SIMPLE.
               \item BSCALE and BZERO in a FITS extension are copied when
               BITPIX is positive, {\it i.e.}\ the array is not floating-point.
               \end{itemize}

            \sstitem
               An extra header record with keyword UNSIGNED and logical
               value T is added when the array data type is one of the \HDSref\
               unsigned integer types.  This is done because standard FITS
               does not support unsigned integers, and allows (in conjunction
               with BITPIX) applications reading the unformatted file to
               determine the data type of the array.

            \sstitem
               The last header record card will be the standard FITS END.
         }

         \sstitem
            Other extensions are not propagated.
      }
   }
   \sstdiytopic{
      Related Applications
   }{
      \CONVERT: \htmlref{ASCII2NDF}{ASCII2NDF};
      \KAPPA: \xref{TRANDAT}{sun95}{TRANDAT};
      \FIGARO: \xref{ASCIN}{sun86}{ASCIN} and
      \xref{ASCOUT}{sun86}{ASCOUT}.
   }
   \sstimplementationstatus{
      \sstitemlist{

         \sstitem
         All non-complex numeric data types are supported.

         \sstitem
         The value of \xref{bad pixels}{sun95}{se_badmasking} is not written to
         a FITS-like header record with keyword BLANK.
      }
   }
}

\newpage
\sstroutine{
   NDF2DA
}{
   Converts an NDF to a direct-access unformatted file
}{
   \sstdescription{
      This application converts an \NDFref\ to a direct-access unformatted
      file, which is equivalent to fixed-length records, or a data
      stream suitable for reading by C routines.  Only one of the array
      components may be copied to the output file.
   }
   \sstusage{
      ndf2da in out [comp] [noperec]
   }
   \sstparameters{
      \sstsubsection{
         COMP = \xref{LITERAL}{sun95}{se_parmenu} (Read)
      }{
         The NDF component to be copied.  It may be \texttt{"Data"},
         \texttt{"Quality"} or \texttt{"Variance"}. \texttt{["Data"]}
      }
      \sstsubsection{
         IN = NDF (Read)
      }{
         Input NDF data structure.  The suggested default is the current
         NDF if one exists, otherwise it is the current value.
      }
      \sstsubsection{
         NOPEREC = \_INTEGER (Read)
      }{
         The number of data values per record of the output file.  It
         must be positive.  The suggested default is the current value.
         \texttt{[}The first dimension of the NDF\texttt{]}
      }
      \sstsubsection{
         OUT = FILENAME (Write)
      }{
         Name of the output direct-access unformatted file.
      }
   }
   \sstexamples{
      \sstexamplesubsection{
         ndf2da cluster cluster.dat
      }{
         This copies the data array of the NDF called cluster to a
         direct-access unformatted file called \texttt{cluster.dat}.
         The number of data values per record is equal to the size of
         the first dimension of the NDF.
      }
      \sstexamplesubsection{
         ndf2da cluster cluster.dat v
      }{
         This copies the variance of the NDF called cluster to a
         direct-access unformatted file called \texttt{cluster.dat}.
         The number of variance values per record is equal to the size of the
         first dimension of the NDF.
      }
      \sstexamplesubsection{
         ndf2da cluster cluster.dat noperec=12
      }{
         This copies the data array of the NDF called cluster to a
         direct-access unformatted file called cluster.dat.  There are
         twelve data values per record in \texttt{cluster.dat}.
      }
   }
   \sstnotes{
      The details of the conversion are as follows:
      \ssthitemlist{

         \sstitem
            the NDF array as selected by COMP is written to the
            unformatted file in records.

         \sstitem
            all other NDF components are lost.
      }
   }
   \sstdiytopic{
      Related Applications
   }{
      \CONVERT: \htmlref{DA2NDF}{DA2NDF}.
   }
}

\newpage
\sstroutine{
   NDF2DST
}{
   Converts an NDF to a Figaro (Version 2) DST file
}{
   \sstdescription{
      This application converts an \NDFref\ to a \Figaroref\ (Version 2) `DST'
      file.  The rules for converting the various components of a DST
      are listed in the \texttt{"}Notes\texttt{"}.  Since both are hierarchical formats
      most files can be be converted with little or no information
      lost.
   }
   \sstusage{
      ndf2dst in out
   }
   \sstparameters{
      \sstsubsection{
         IN = NDF (Read)
      }{
         Input NDF data structure.  The suggested default is the
         current NDF if one exists, otherwise it is the current value.
      }
      \sstsubsection{
         OUT = Figaro (Write)
      }{
         Output Figaro file name. This excludes the file extension.
         The file created will be given extension \texttt{".dst"}.
      }
   }
   \sstexamples{
      \sstexamplesubsection{
         ndf2dst old new
      }{
         This converts the NDF called old (in file \texttt{old.sdf}) to the
         Figaro file \texttt{new.dst}.
      }
      \sstexamplesubsection{
         ndf2dst spectre spectre
      }{
         This converts the NDF called spectre (in file \texttt{spectre.sdf}) 
         to the Figaro file \texttt{spectre.dst}.
      }
   }
   \sstnotes{
      The rules for the conversion are as follows:\vspace*{-\medskipamount}

      \latex{\scriptsize}
      \begin{center}
      \begin{tabular}{|lcl|p{57mm}|}
      \hline 
      \multicolumn{1}{|c}{NDF} & & Figaro file &
      \multicolumn{1}{|c|}{Comments} \\ \hline
      Main data array  & $\Rightarrow$ & .Z.DATA & \\
      Imaginary array  & $\Rightarrow$ & .Z.IMAGINARY & \\
      Bad-pixel flag   & $\Rightarrow$ & .Z.FLAGGED & \\
      Units            & $\Rightarrow$ & .Z.UNITS & \\
      Label            & $\Rightarrow$ & .Z.LABEL & \\
      Variance         & $\Rightarrow$ & .Z.ERRORS & After processing \\
      Quality          & $\Rightarrow$ &  & It is not copied directly
                         though \xref{bad values}{sun95}{se_badmasking} indicated by
                         QUALITY flags will be flagged in .Z.DATA in addition to any
                         flagged values actually in the input main data array.
                         .Z.FLAGGED is set accordingly. \\ \hline
      \end{tabular}
      \end{center}

      \begin{center}
      \begin{tabular}{|lcl|p{54mm}|}
      \hline 
      \multicolumn{1}{|c}{NDF} & & Figaro file &
      \multicolumn{1}{|c|}{Comments} \\ \hline
      Title            & $\Rightarrow$ & .OBS.OBJECT & \\
      & & & \\
      AXIS(1) structure & $\Rightarrow$ & .X & \\
      AXIS(1) Data  & $\Rightarrow$ & .X.DATA\_ARRAY & unless there is a DATA
                          component of AXIS(1).MORE.FIGARO to allow for a 
                          non-1-dimensional array \\
      AXIS(1) Variance & $\Rightarrow$ & .X.VARIANCE & unless there is a
                          VARIANCE component of AXIS(1).MORE.FIGARO to
                          allow for a non-1-dimensional array \\
      AXIS(1) Width & $\Rightarrow$ & .X.WIDTH & unless there is a WIDTH
                          component of AXIS(1).MORE.FIGARO to
                          allow for a non-1-dimensional array \\
      AXIS(1) Units & $\Rightarrow$ & .X.UNITS & \\
      AXIS(1) Label & $\Rightarrow$ & .X.LABEL & \\
      AXIS(1).MORE.FIGARO.xxx & $\Rightarrow$ & .X.xxx & \\
      & & & Similarly for AXIS(2), \dots, AXIS(6) which are renamed to
           .Y .T .U .V or .W \\
      & & & \\
      FIGARO extension: & & & \\
      .MORE.FIGARO.MAGFLAG & $\Rightarrow$ & .Z.MAGFLAG & \\
      .MORE.FIGARO.RANGE & $\Rightarrow$ & .Z.RANGE & \\
      .MORE.FIGARO.SECZ & $\Rightarrow$ & .OBS.SECZ & \\
      .MORE.FIGARO.TIME & $\Rightarrow$ & .OBS.TIME & \\
      .MORE.FIGARO.xxx & $\Rightarrow$ & .xxx & recursively \\
      & & & \\
      FITS extension: & & & \\
      .MORE.FITS & & & \\
      Items  & $\Rightarrow$ & .FITS.xxx & \\
      Comments & $\Rightarrow$ & .COMMENTS.xxx & \\
      & & & \\
      Other extensions: & & & \\
      .MORE.other & $\Rightarrow$ & .MORE.other & \\ \hline
      \end{tabular}
      \end{center}
      \normalsize
   }
   \sstdiytopic{
      Related Applications
   }{
      \CONVERT: \htmlref{DST2NDF}{DST2NDF}.
   }
}

\newpage
\sstroutine{
   NDF2FITS
}{
   Converts NDFs into FITS files
}{
   \sstdescription{
      This application converts one or more \NDFref\ datasets into
      {\FITSref}-format files.  NDF2FITS stores any variance and quality
      information in IMAGE extensions (`sub-files') within the FITS
      file; and it uses binary tables to hold any NDF-extension data
      present, except for the \xref{FITS-airlock extension}{sun95}{se_fitsairlock}, 
      which may be merged into the output FITS file's headers.

      You can select which NDF array components to export to the FITS
      file, and choose the data type of the data and variance arrays.
      You can control whether or not to propagate extensions and
      history information.

      The application also accepts NDFs stored as top-level components 
      of an HDS container file.
   }
   \sstusage{
      ndf2fits in out [comp] [bitpix] [origin]
   }
   \sstparameters{
      \sstsubsection{
         BITPIX = LITERAL (Read)
      }{
         The FITS bits-per-pixel (BITPIX) value for each conversion.
         This specifies the data type of the output FITS file.  Permitted
         values are: \texttt{8} for unsigned byte, \texttt{16} for signed word,
         \texttt{32} for integer, \texttt{-32} for real, \texttt{-64} for double
         precision.  There are three other special values.  
         \ssthitemlist{

         \sstitem 
            BITPIX=\texttt{0} will cause the output file to have the
	    data type equivalent to that of the input NDF.

         \sstitem  
            BITPIX=\texttt{-1} requests that the output file has the
	    data type corresponding to the value of the BITPIX keyword
	    in the NDF's FITS extension.  If the extension or BITPIX
	    keyword is absent, the output file takes the data type of
	    the input array.

         \sstitem
            BITPIX=\texttt{"Native"} requests that any scaled arrays
	    in the NDF be copied to the scaled data type.  Otherwise
	    behaviour reverts to BITPIX=\texttt{-1}, which may in turn be
	    effectively BITPIX=\texttt{0}.  The case-insensitive value may
            be abbreviated to \texttt{"n"}.
         }

         BITPIX values must be enclosed in double quotes and may be a list 
         of comma-separated values to be applied to each conversion in turn.
         An error results if more values than the number of input NDFs are
         supplied.  If too few are given, the last value in the list is
         applied to the remainder of the NDFs; thus a single value is 
         applied to all the conversions.  
         The given values must be in the same order as that of the input NDFs.
         Indirection through a text file may be used.
         If more than one line is required to enter the information at a prompt
         then type a \texttt{"-"} at the end of each line where a
         continuation line is desired.
         \texttt{[0]}
      }
      \sstsubsection{
         CHECKSUM = \_LOGICAL (Read)
      }{
         If \texttt{TRUE}, each header and data unit in the FITS file will
         contain the integrity-check keywords CHECKSUM and DATASUM 
         immediately before the END card.  \texttt{[TRUE]}
      }
      \sstsubsection{
         COMP = LITERAL (Read)
      }{
         The list of array components to attempt to transfer to each
         FITS file.  The acceptable values are \texttt{"D"} for the main data
         array \texttt{"V"} for variance, \texttt{"Q"} for quality, or any
         permutation thereof.
         The special value \texttt{"A"} means all components,
         {\it i.e.}\ COMP=\dqt{DVQ}. Thus COMP=\dqt{VD} requests that 
         both the data array and variance are to be converted if present.  
         During processing at least one, if not all, of the requested
         components must be present, otherwise an error is reported and
         processing turns to the next input NDF.  If the DATA component
         is in the list, it will always be processed first into the
         FITS primary array.  The order of the variance and quality
         in COMP decides the order they will appear in the FITS file.

         COMP may be a list of comma-separated values to be applied to
         each conversion in turn.  The list must be enclosed in double quotes.
         An error results if more values than
         the number of input NDFs are supplied.  If too few are given,
         the last value in the list is applied to the remainder of the
         NDFs; thus a single value is applied to all the conversions.
         The given values must be in the same order as that of the
         input NDFs.  Indirection through a text file may be used.  If
         more than one line is required to enter the information at a prompt
         then type a \texttt{"-"} at the end of each line where a continuation 
         line is desired.  \texttt{["A"]}
      }
      \sstsubsection{
         CONTAINER = \_LOGICAL (Read)
      }{
         If \texttt{TRUE}, the supplied IN files are any multi-NDF HDS 
         container files, in which the NDFs reside as top-level 
         components.  This option is primarily intended to support 
         the UKIRT format where the NDFs are named .I$n$, 
         $n >=1$, and one named HEADER containing global metadata in 
         its FITS airlock.  The .I$n$ NDFs
         may also contain FITS airlocks, storing metadata pertinent to
         that NDF, such as observation times.  The individual NDFs often
         represent separate integrations nodded along a slit or
         spatially.  Note that this is not a group, so a single value
         applies to all the supplied input files.  \texttt{[FALSE]}
      }
      \sstsubsection{
         DUPLEX = \_LOGICAL (Read)
      }{
         This qualifies the effect of PROFITS=\texttt{TRUE}.  
          DUPLEX=\texttt{FALSE} means that the airlock headers only appear
          with the primary array.  DUPLEX=\texttt{TRUE}, propagates the 
          FITS airlock headers for other array components of the NDF. 
          \texttt{[FALSE]}
      }
      \sstsubsection{
         ENCODING = LITERAL (Read)
      }{
         Controls the FITS keywords which will be used to encode the 
         World Co-ordinate System (WCS) information within the FITS 
         header. The value supplied should be one of the encodings listed
         in the 
         \htmlref{\texttt{"}World Co-ordinate Systems\texttt{"}}
         {world_coordinate_systems}
         section below. In addition, 
         the value \texttt{"Auto"} may also be supplied, in which case a suitable
         default encoding is chosen based on the contents of the NDF's FITS
         extension and WCS component.  \texttt{["Auto"]}
      }
      \sstsubsection{
         IN = LITERAL (Read)
      }{
         The names of the NDFs to be converted into FITS format.  It
         may be a list of NDF names or direction specifications
         separated by commas and enclosed in double quotes.  
         NDF names may include wild-cards (\texttt{"*"}, \texttt{"?"}).
         Indirection may occur through text files (nested up to seven
         deep).  The indirection character is \texttt{"$\wedge$"}.  If extra
         prompt lines are required, append the continuation character
         \texttt{"-"} to the end of the line.
         Comments in the indirection file begin with the character \hash.
      }
      \sstsubsection{
         NATIVE = \_LOGICAL (Read)
      }{
         If a \texttt{TRUE} value is given for parameter NATIVE, then World
         Co-ordinate System (WCS) information will be written to the
         FITS header in the form of a `native' encoding (see 
         \htmlref{\texttt{"}World Co-ordinate Systems\texttt{"}}
         {world_coordinate_systems} below).  This will be in addition to the
         encoding specified using parameter ENCODING, and will usually result
         in  two descriptions of the WCS information being stored in the FITS 
         header (unless ENCODING parameter produces a native encoding in which
         case only one native encoding is stored in the header). Including a
         native encoding in the header will enable other 
         \xref{AST}{sun210}{abstract}-based software 
         (such as 
         \htmlref{FITS2NDF}{FITS2NDF}) 
         to reconstruct the full details of the WCS information. 
         The other non-native encodings will usually result in some 
         information being lost. \texttt{[FALSE]}
      }
      \sstsubsection{
         MERGE = \_LOGICAL (Read)
      }{
         Whether or not to merge the FITS-airlocks' headers of the 
         header NDF of a UKIRT multi-NDF container file with its sole 
         data NDF into the primary HDU.  This parameter is only used
         when CONTAINER is \texttt{TRUE}; and when the container file only has
         two component NDFs: one data NDF of arbitrary name, and the
         other called HEADER that stores the global headers of the
         dataset. \texttt{[TRUE]}
      }
      \sstsubsection{
         ORIGIN = LITERAL (Read)
      }{
         The origin of the FITS files.  This becomes the value of the
         ORIGIN keyword in the FITS headers.  If a null value is given
         it defaults to \texttt{"Starlink Project, U.K."}.
         \texttt{[!]}
      }
      \sstsubsection{
         OUT = LITERAL (Write)
      }{
         The names for the output FITS files.  These may be enclosed in double
         quotes and specified as a list of comma-separated names, or they may
         be created automatically on the basis of the input NDF names. To do
         this, the string supplied for this parameter should include an
         asterisk \texttt{"$*$"}. This character is a token which represents
         the name of the corresponding input NDF, but with a file type of
         \texttt{".fit"} instead of \texttt{".sdf"}, and with no directory
         specification. Thus, simply supplying \texttt{"*"} for this parameter
         will create a group of output files in the current directory with the 
         same names as the input NDFs, but with file type \texttt{".fit"}. 
         You can also specify some simple editing to be performed. For instance,
         \texttt{"new-*|.fit|.fits|"} will add the string \texttt{"new-"} to
         the start of every file name, and will substitute the string
         \texttt{".fits"} for the original string \texttt{".fit"}.
      }
      \sstsubsection{
         PROEXTS = \_LOGICAL (Read)
      }{
         If \texttt{TRUE}, the NDF extensions (other than the FITS extension)
         are propagated to the FITS files as FITS binary-table
         extensions, one per structure of the hierarchy. \texttt{[FALSE]}
      }
      \sstsubsection{
         PROFITS = \_LOGICAL (Read)
      }{
         If \texttt{TRUE}, the contents of the FITS extension of the NDF are
         merged with the header information derived from the standard
         NDF components.  See the \texttt{"}Notes\texttt{"} for details of the merger.
         \texttt{[TRUE]}
      }
      \sstsubsection{
         PROHIS = \_LOGICAL (Read)
      }{
         If \texttt{TRUE}, any NDF history records are written to the primary
         FITS header as HISTORY cards.  These follow the mandatory
         headers and any merged FITS-extension headers (see parameter
         PROFITS). \texttt{[TRUE]}
      }
      \sstsubsection{
         PROVENANCE = LITERAL (Read)
      }{
         This controls the export of NDF provenance information to the 
         FITS file.  Allowed values are as follows.
         \ssthitemlist{

            \sstitem  
            \texttt{"None"} --- No provenance is written.

            \sstitem
            \texttt{"CADC"} --- The CADC headers are written.  These
	    record the number and paths of both the direct parents
	    of the NDF being converted, and its root ancestors (the
	    ones without parents).
 
            \sstitem
            \texttt{"Generic"} --- Encapsulates the entire
	    PROVENANCE structure in FITS headers in sets of five
	    character-value indexed headers. there is a set for the
	    current NDF and each parent. See Section
	    \htmlref{\texttt{"}Provenance\texttt{"}}{ndf2fits_provenance}
	    for more details.
         }
         \texttt{["None"]}
      }
   }
   \sstexamples{
      \sstexamplesubsection{
         ndf2fits horse logo.fit d
      }{
         This converts the NDF called horse to the FITS file called
         \texttt{logo.fit}.  The data type of the FITS primary data array
         matches that of the NDF's data array.  The FITS extension in the NDF
         is merged into the FITS header of \texttt{logo.fit}.
      }
      \sstexamplesubsection{
         ndf2fits horse logo.fit d proexts
      }{
         This converts the NDF called horse to the FITS file called
         \texttt{logo.fit}.
         The data type of the FITS primary data array matches
         that of the NDF's data array.  The FITS extension in the NDF
         is merged into the FITS header of \texttt{logo.fit}.  In addition any
         NDF extensions (apart from FITS) are turned into binary tables
         that follow the primary header and data unit.
      }
      \sstexamplesubsection{
         ndf2fits horse logo.fit noprohis
      }{
         This converts the NDF called horse to the FITS file called
         \texttt{logo.fit}.
         The data type of the FITS primary data array matches
         that of the NDF's data array.  The FITS extension in the NDF
         is merged into the FITS header of \texttt{logo.fit}.  Should horse
         contain variance and quality arrays, these are written in IMAGE
         extensions.  Any history information in the NDF is not relayed
         to the FITS file.
      }
      \sstexamplesubsection{
         ndf2fits "data/a$*$z" $*$ comp=v noprofits bitpix=-32
      }{
         This converts the NDFs with names beginning with \texttt{"a"} and 
         ending in \texttt{"z"} in the directory called \texttt{data} into
         FITS files of the
         same name and with a file extension of \texttt{".fit"}.  The variance
         array becomes the data array of each FITS file.  The data type
         of the FITS primary data array single-precision floating
         point.  Any FITS extension in the NDF is ignored.
      }
      \sstexamplesubsection{
         ndf2fits "abc,def" "jvp1.fit,jvp2.fit" comp=d  bitpix="16,-64"
      }{
         This converts the NDFs called abc and def into FITS files
         called \texttt{jvp1.fit} and \texttt{jvp2.fit} respectively.  
         The data type of the FITS primary data array is signed integer words
         in \texttt{jvp1.fit}, and double-precision floating point in
         \texttt{jvp2.fit}.  The FITS extension in each NDFs is merged into the
         FITS header of the corresponding FITS file.
      }
      \sstexamplesubsection{
         ndf2fits horse logo.fit d native encoding="fits-wcs"
      }{
         This is the same as the first example except that the co-ordinate 
         system information stored in the NDF's WCS component is written
         to the FITS file twice; once using the FITS-WCS headers, and once 
         using a special set of `native' keywords recognised by the AST 
         library (see \xref{SUN/210}{sun210}{}). 
         The native encoding provides a `loss-free'
         means of transferring co-ordinate system information (\textit{i.e.}\
         no information is lost; other encodings may cause information to be 
         lost).  Only applications based on the AST library (such as 
         \htmlref{FITS2NDF}{FITS2NDF}) are able to interpret native encodings.
      }
      \sstexamplesubsection{
         ndf2fits u20040730\_00675 merge container accept
      }{
         This converts the UIST container file \texttt{u20040730\_00675.sdf}
         to FITS file \texttt{u20040730\_00675.fit}, merging its .I1 and 
         .HEADER structures into a single NDF before the conversion.  The 
         output file has only one header and data unit.
      }
      \sstexamplesubsection{
         ndf2fits in=c20011204\_00016 out=cgs4\_16.fit container
      }{
         This converts the CGS4 container file 
         \texttt{c20011204\_00016.sdf} to the multiple-extension FITS 
         file \texttt{cgs4\_16.fit}.  The primary HDU has the global 
         metadata from the .HEADER's FITS airlock.  The 
         four integrations in I1, I2, I3, and I4 components of the
         container file are converted to FITS IMAGE extensions.
      }
      \sstexamplesubsection{
         ndf2fits in=huge out=huge.fits comp=d bitpix=n
      }{
         This converts the NDF called huge to the FITS file called
         \texttt{huge.fits}.  The data type of the FITS primary data array 
         matches that of the NDF's scaled data array.  The scale and
         offset coefficients used to form the FITS array are also taken
         from the NDF's scaled array.
      }
      \sstexamplesubsection{
         ndf2fits in=huge out=huge.fits comp=d bitpix=-1
      }{
         As the previous example, except that the data type of the FITS 
         primary data array is that given by the BITPIX keyword in the
         FITS airlock of NDF huge and the scaling factors are
         determined.
      }
   }
   \sstnotes{
      The rules for the conversion are as follows:
      \ssthitemlist{

         \sstitem
         The NDF main data array becomes the primary data array of the
         FITS file if it is in value of parameter COMP, otherwise the first
         array defined by parameter COMP will become the primary data
         array.  A conversion from floating point to integer or to a
         shorter integer type will cause the output array to be scaled and
         offset, the values being recorded in keywords BSCALE and BZERO.
         There is an offset (keyword BZERO) applied to signed byte and
         unsigned word types to make them unsigned-byte and signed-word
         values respectively in the FITS array (this is because FITS does
         not support these data types).

         \sstitem
         The FITS keyword BLANK records the \xref{bad values}{sun95}{se_badmasking}
         for integer output types.  Bad values in floating-point output arrays are
         denoted by IEEE not-a-number values.

         \sstitem
         The NDF's quality and variance arrays appear in individual
         FITS IMAGE extensions immediately following the primary header
         and data unit, unless that component already appears as the
         primary data array.  The quality array will always be written as
         an unsigned-byte array in the FITS file, regardless of the value
         of the parameter BITPIX.

         \sstitem
         Here are details of the processing of standard items from the
         NDF into the FITS header, listed by FITS keyword.
         \ssthitemlist{

            \sstitem
            SIMPLE, EXTEND, PCOUNT, GCOUNT --- all take their default
              values.

            \sstitem
            BITPIX, NAXIS, NAXISn --- are derived directly from the NDF
              data array; however the BITPIX in the FITS airlock extension
              is transferred when parameter BITPIX=\texttt{-1}.

            \sstitem
            CRVAL\textit{n}, CDELT\textit{n}, CRPIX\textit{n}, CTYPE\textit{n},
            CUNIT\textit{n} --- are derived
              from the NDF WCS component if possible (see 
              \htmlref{\texttt{"}World Co-ordinate Systems\texttt{"}}
                 {world_coordinate_systems}).
              If this is not possible, and if PROFITS is \texttt{TRUE}, then
              they are copied from the NDF's FITS extension.

            \sstitem
            OBJECT, LABEL, BUNIT --- the values held in the NDF's TITLE,
              LABEL, and UNITS components respectively are used if
              they are defined; otherwise any values found in the FITS
              extension are used (provided parameter PROFITS is \texttt{TRUE}).
              For a variance array, BUNIT is assigned to 
              \texttt{($<unit>$)**2}, where $<unit>$ is the DATA unit; the
              BUNIT header is absent for a quality array.

            \sstitem
            DATE --- is created automatically.

            \sstitem
            ORIGIN --- inherits any existing ORIGIN card in the NDF FITS
              extension, unless you supply a value through parameter
              ORIGIN other than the default \texttt{"Starlink Project, U.K."}

            \sstitem
            EXTNAME --- is the array-component name when the EXTNAME
              appears in the primary header or an IMAGE extension.  In a
              binary-table derived from an NDF extension, EXTNAME is the
              path of the extension within the NDF, the path separator
              being the usual dot.  The path includes the indices to
              elements of any array structures present; the indices are in
              a comma-separated list within parentheses.

            \sstitem
            EXTLEVEL --- is the level in the hierarchical structure of the
              extensions.  Thus a top-level extension has value 1,
              sub-components of this extension have value 2 and so on.

            \sstitem
            EXTTYPE --- is the data type of the NDF extension used to
              create a binary table.

            \sstitem
            EXTSHAPE --- is the shape of the NDF extension used to
            create a binary table.  It is a comma-separated list of the
            dimensions, and is 0 when the extension is not an array.

            \sstitem
            HDUCLAS1, HDUCLAS{\em{n}} --- \texttt{"NDF"} and the
              array-component name respectively.

            \sstitem
            LBOUND{\textit{n}} --- is the pixel origin for the 
              $\textit{n}^{\rm th}$ dimension
              when any of the pixel origins is not equal to 1.  (This is not a
              standard FITS keyword.)

            \sstitem
            XTENSION, BSCALE, BZERO, BLANK and END --- are not propagated
              from the NDF's FITS extension.  XTENSION will be set for
              any extension.  BSCALE and BZERO will be defined based on
              the chosen output data type in comparison with the NDF
              array's type, but cards with values 1.0 and 0.0 respectively
              are written to reserve places in the header section.  These
              `reservation' cards are for efficiency and they can always
              be deleted later.  BLANK is set to the Starlink standard 
              \xref{bad value}{sun95}{se_badmasking} corresponding to the type
              specified by BITPIX, but only for integer types and not for the
              quality array.  It appears regardless of whether or not there are
              bad values actually present in the array; this is for the same
              efficiency reasons as before.  The END card terminates the FITS header.

            \sstitem
            HISTORY --- HISTORY headers are propagated from the FITS
              airlock when PROFITS is \texttt{TRUE}, and from the NDF
              history component when PROHIS is \texttt{TRUE}.

            \sstitem
            DATASUM and CHECKSUM --- data-integrity keywords are written 
            when parameter CHECKSUM is \texttt{TRUE}, replacing any existing
            values.  When parameter CHECKSUM is \texttt{FALSE} and PROFITS is
            \texttt{TRUE} any existing values inherited from the FITS airlock are 
            removed to prevent storage of invalid checksums relating to 
            another data file.

         }
         See also the sections
         \htmlref{\texttt{"}Provenance\texttt{"}}{ndf2fits_provenance} and
         \htmlref{\texttt{"}World Co-ordinate Systems\texttt{"}}{world_coordinate_systems}
         for details of headers used to describe the PROVENANCE extension 
         and WCS information respectively.

         \sstitem
         Extension information may be transferred to the FITS file when
         PROEXTS is \texttt{TRUE}.
         The whole hierarchy of extensions is propagated
         in order.  This includes substructures, and arrays of extensions
         and substructures.  However, at present, any extension structure
         containing only substructures is not propagated itself (as
         zero-column tables are not permitted), although its
         substructures may be converted.
 
         Each extension or substructure creates a one-row binary table,
         where the columns of the table correspond to the primitive
         (non-structure) components.  The name of each column is the
         component name.  The column order is the same as the component
         order.  The shapes of multi-dimensional arrays are recorded using
         the TDIM\textit{n} keyword, where \textit{n} is the column number.
         The HEASARCH convention for specifying the width of character arrays 
         (keyword TFORM\textit{n}='\textit{r}A\textit{w}', where \textit{r} is
         the total number of characters in the column and \textit{w} is the 
         width of an element) is used.  The EXTNAME,
         EXTTYPE, EXTSHAPE and EXTLEVEL keywords (see above) are written
         to the binary-table header.

      }
      There are additional rules if a multi-NDF container file is being
      converted (see parameter CONTAINER).  This excludes the case where
      there are but two NDFs---one data and the other just 
      headers---that have already been merged (see parameter MERGE):
      \ssthitemlist{

         \sstitem
            For multiple NDFs a header-only HDU may be created followed 
            by an IMAGE extension containing the data array (or 
            whichever other array is first specified by COMP).
         \sstitem
            BITPIX for the header HDU is set to an arbitrary 8.
         \sstitem
            Additional keywords are written for each IMAGE extension.
         \ssthitemlist{
            \sstitem
             HDSNAME --- is the NDF name for a component NDF in a multi-NDF
               container file, for example \texttt{I2}.
             \sstitem
             HDSTYPE --- is set to \texttt{NDF} for a component NDF in a 
               multi-NDF container file.
         }
      }      
   }
   \sstdiytopic{
      \label{world_coordinate_systems}World Co-ordinate Systems
   }{
      Any co-ordinate system information stored in the WCS component of the
      NDF is written to the FITS header using one of the following encoding
      systems (the encodings used are determined by parameters ENCODING and 
      NATIVE):
      \ssthitemlist{

         \sstitem 
         \texttt{"FITS-IRAF"} --- This uses keywords CRVAL\textit{i},
         CRPIX\textit{i}, and CD\textit{i\_j}, and is the
         system commonly used by IRAF. It is described in the document
         \textit{World Coordinate Systems Representations Within the FITS
         Format}\ by by R.J. Hanisch and D.G. Wells, 1988, available by ftp from
         fits.cv.nrao.edu \texttt{/fits/documents/wcs/wcs88.ps.Z}.


         \sstitem
         \texttt{"FITS-WCS"} --- This is the FITS standard WCS encoding 
         scheme described in the paper 
         \htmladdnormallink{\textit{Representation of celestial coordinates in FITS.}}
         {http://www.atnf.csiro.au/people/mcalabre/WCS/}\\ \latex{
         (\texttt{http://www.atnf.csiro.au/people/mcalabre/WCS/})}  It is
         very similar to \texttt{"FITS-IRAF"} but supports a wider range of
         projections and co-ordinate systems.

         \sstitem
         \texttt{"FITS-WCS(CD)"} --- This is the same as \texttt{"FITS-WCS"}
         except that the scaling and rotation of the data array is described by a 
         CD matrix instead of a PC matrix with associated CDELT values.

         \sstitem
         \texttt{"FITS-PC"} --- This uses keywords CRVAL\textit{i},
         CDELT\textit{i}, CRPIX\textit{i}, PC\textit{iiijjj}, 
         \textit{etc}, as described in a previous (now superseded) draft of
         the above FITS world co-ordinate system paper by E.W.Greisen and 
         M.Calabretta.

         \sstitem
         \texttt{"FITS-AIPS"} --- This uses conventions described in the
         document "\textit{Non-linear Coordinate Systems in AIPS}" by
         Eric W. Greisen (revised 9th September, 1994), available by ftp
         from fits.cv.nrao.edu \texttt{/fits/documents/wcs/aips27.ps.Z.}
         It is currently employed by the AIPS data-analysis facility
         (amongst others), so its use
         will facilitate data exchange with AIPS. This encoding uses
         CROTA\textit{i} and CDELT\textit{i} keywords to describe axis
         rotation and scaling.

         \sstitem 
         \texttt{"FITS-AIPS++"} --- This is an extension to FITS-AIPS which 
         allows the use of a wider range of celestial as used by the AIPS++ 
         project.

         \sstitem
         \texttt{"FITS-CLASS"} --- This uses the conventions of the CLASS
         project.  CLASS is a software package for reducing single-dish 
         radio and sub-mm spectroscopic data.  It supports double-sideband 
         spectra.  See \htmladdnormallink{the GILDAS 
         manual}{http://www.iram.fr/IRAMFR/GILDAS/doc/html/class-html/class.html}.

         \sstitem
         \texttt{"DSS"} --- This is the system used by the Digital Sky Survey, and
         uses keywords AMDX\textit{n}, AMDY\textit{n}, PLTRAH, etc.

         \sstitem
         \texttt{"NATIVE"} --- This is the native system used by the AST library (see
         SUN/210) and provides a loss-free method for transferring WCS
         information between AST-based applications.  It allows more
         complicated WCS information to be stored and retrieved than any of 
         the other encodings.
      }

      Values for FITS keywords generated by the above encodings will
      always be used in preference to any corresponding keywords found in
      the FITS extension (even if PROFITS is \texttt{TRUE}).
      If this is not what
      is required, the WCS component of the NDF should be erased using
      the \KAPPA\ command \xref{ERASE}{sun95}{ERASE} before running NDF2FITS. Note, if PROFITS
      is \texttt{TRUE}, then any WCS-related keywords in the FITS extension
      which are not replaced by keywords derived from the WCS component may
      appear in the output FITS file. If this causes a problem, then
      PROFITS should be set to \texttt{FALSE} or the offending keywords removed
      using \KAPPA\ 
      \xref{FITSEDIT}{sun95}{FITSEDIT}
      for example.
   }
   \sstdiytopic{
      \label{ndf2fits_provenance}Provenance
   }{
      The following PROVENANCE headers are written if parameter
      PROVENANCE is set to \texttt{"Generic"}.
      \ssthitemlist{

         \sstitem
         PRVP$n$ --- is the path of the $n$th NDF.

         \sstitem
         PRVIn --- is a comma-separated list of the identifiers of the 
           direct parents for the $n$th ancestor.

         \sstitem
         PRVDn --- is the creation date in ISO order of the $n$th ancestor.

         \sstitem
         PRVCn --- is the software used to create the $n$th ancestor
         
         \sstitem
         PRVMn --- lists the contents of the MORE structure of the $n$th
           parent.
      }
      All have value \texttt{<unknown>} if the information could not be found, 
      except for the PRVM$n$ header, which is omitted if there is no MORE 
      information to record.   The index $n$ used in each keyword's name 
      is the provenance identifier for the NDF, and starts at 0 for the
      NDF being converted to FITS.

      The following PROVENANCE headers are written if parameter
      PROVENANCE is set to \texttt{"CADC"}.
      \ssthitemlist{

         \sstitem
         PRVCNT --- is the number of immediate parents.

         \sstitem
         PRV$m$ --- is name of the $m$th immediate parent.

         \sstitem
         OBSCNT --- is the number of root ancestor OBS$m$ headers.

         \sstitem
         OBS$m$ --- is $m$th root ancestor identifier from its MORE.OBSIDSS
         component.
      }

       When PROFITS is \texttt{TRUE} any existing provenance keywords in the
       FITS airlock are not copied to the FITS file.
   }
   \sstdiytopic{
      Special Formats
   }{
      In the general case, NDF extensions (excluding the FITS extension)
      may be converted to one-row binary tables in the FITS file when
      parameter PROEXTS is \texttt{TRUE}.  This preserves the information,
      but it may not be accessible to the recipient's FITS reader.  Therefore,
      in some cases it is desirable to understand the meanings of
      certain NDF extensions, and create standard FITS products for
      compatibility.

      At present only one product is supported, but others may be added
      as required.

      \sstitemlist{

         \sstitem
         AAO \htmladdnormallink{2dF}{http://www.aao.gov.au/local/www/2df/}

         Standard processing is used except for the 2dF FIBRES extension
         and its constituent structures.  The NDF may be restored from the
         created FITS file using FITS2NDF.  The FIBRES extension converts
         to the second binary table in the FITS file (the NDF\_CLASS
         extension appears in the first).

         To propagate the OBJECT substructure, NDF2FITS creates a binary
         table of constant width (224 bytes) with one row per fibre.  The
         total number of rows is obtained from component NUM\_FIBRES.  If a
         possible OBJECT component is missing from the NDF, a null column
         is written for that component.  The columns inherit the data
         types of the OBJECT structure's components.  Column meanings and
         units are assigned based upon information in the reference given
         below.

         The FIELD structure components are converted into additional
         keywords of the same name in the binary-table header, with the
         exception that components with names longer than eight characters
         have abbreviated keywords: UNALLOC\textit{xxx} becomes
         UNAL-\textit{xxx} (\textit{xxx}=OBJ,
         GUI, or SKY), CONFIGMJD becomes CONFMJD, and \textit{x}SWITCHOFF
         becomes \textit{x}SWTCHOF (\textit{x}=X or Y).  If any FIELD 
         component is missing it is ignored.

         Keywords for the extension level, name, and type appear in the
         binary-table header.

         \sstitem
         JCMT SMURF

         Standard processing is used except for the SMURF-type extension.
         This contains NDFs such as EXP\_TIME and TSYS.  Each such NDF
         is treated like the main NDF except that it is assumed that
         these extension NDFs have no extensions of their own.  FITS
         airlock information and HISTORY are inherited from the parent
         NDF.  Also the extension keywords are written: EXTNAME gives the
         path to the NDF, EXTLEVEL records the extension hierarchy level,
         and EXTTYPE is set to \texttt{"NDF"}.  Any non-NDF components of 
         the SMURF extension are written to a binary table in the normal 
         fashion.
      }
   }
   \sstdiytopic{
      References
   }{
      \begin{refs}
      \item Bailey, J.A. 1997, 2dF Software Report 14, version 0.5.
      \item NASA Office of Standards and Technology, 1994, {\it A User's Guide
       for the Flexible Image Transport System (FITS)}, version 3.1.
      \item NASA Office of Standards and Technology, 1995, {\it Definition of
       the Flexible Image Transport System (FITS)}, version 1.1.
      \end{refs}
   }
   \sstdiytopic{
      Related Applications
   }{
      \CONVERT: \htmlref{FITS2NDF}{FITS2NDF};
      \KAPPA: \xref{FITSDIN}{sun95}{FITSDIN}, \xref{FITSIN}{sun95}{FITSIN}.
   }
   \sstimplementationstatus{
      \sstitemlist{

         \sstitem
         All NDF data types are supported.
      }
   }
}

\newpage
\sstroutine{
   NDF2GASP
}{
   Converts a two-dimensional NDF into a GASP image
}{
   \sstdescription{
      This application converts a two-dimensional \NDFref\ into the GAlaxy
      Surface Photometry (GASP) package's format.  See the
      \texttt{"}Notes\texttt{"} for the details of the conversion.
   }
   \sstusage{
      ndf2gasp in out [fillbad]
   }
   \sstparameters{
      \sstsubsection{
         IN = NDF (Read)
      }{
         The input NDF data structure. The suggested default is the
         current NDF if one exists, otherwise it is the current value.
      }
      \sstsubsection{
         FILLBAD = \_INTEGER (Read)
      }{
         The value used to replace \xref{bad pixels}{sun95}{se_badmasking} in the
         NDF's data array before it is copied to the GASP file.  The value must be 
         in the range of signed words ($-$32768 to 32767).  A null value (\texttt{!})
         means no replacements are to be made.  This parameter is
         ignored if there are no bad values.  \texttt{[!]}
      }
      \sstsubsection{
         OUT = FILENAME (Write)
      }{
         The name of the output GASP image.  Two files are produced
         with the same name but different extensions.  The \texttt{".dat"} file
         contains the data array, and \texttt{".hdr"} is the associated header
         file that specifies the dimensions of the image.  The
         suggested default is the current value.
      }
   }
   \sstexamples{
      \sstexamplesubsection{
         ndf2gasp abell1367 a1367
      }{
         Converts an NDF called abell1367 into the GASP image comprising
         the pixel file \texttt{a1367.dat} and the header file
         \texttt{a1367.hdr}.
         If there are any bad values present they are copied verbatim to
         the GASP image.
      }
      \sstexamplesubsection{
         ndf2gasp ngc253 ngc253 fillbad=-1
      }{
         Converts the NDF called ngc253 to a GASP image comprising the
         pixel file \texttt{ngc253.dat} and the header file \texttt{ngc253.hdr}.
         Any bad values in the data array are replaced by minus one.
      }
   }
   \sstnotes{
      The rules for the conversion are as follows:
      \ssthitemlist{

         \sstitem
         The NDF data array is copied to the \texttt{".dat"} file.

         \sstitem
         The dimensions of the NDF data array is written to the \texttt{".hdr"}
         header file.

         \sstitem
         All other NDF components are ignored.
      }
   }
   \sstdiytopic{
      Related Applications
   }{
      \CONVERT: \htmlref{GASP2NDF}{GASP2NDF}.
   }
   \sstdiytopic{
      References
   }{
      GASP documentation (MUD/66).
   }
   \sstimplementationstatus{
      \sstitemlist{

         \sstitem
         The GASP image produced has by definition type SIGNED WORD.
         There is type conversion of the data array to this type.

         \sstitem
         Bad values may arise due to type conversion.  These too are
         substituted by the (non-null) value of FILLBAD.

         \sstitem
         For an NDF with an odd number of columns, the last column from
         the GASP image is trimmed.
      }
   }
}

\newpage
\sstroutine{
   NDF2GIF
}{
   Converts an NDF into a GIF file.
}{
   \sstdescription{
      This Bourne-shell script converts an \NDFref\ into a 256 grey-level
      Graphics Interchange Format (\GIFref) file.  One- or two-dimensional
      images can be handled and various methods of scaling the data are
      provided.  The script uses the \CONVERT\ utility
      \htmlref{NDF2TIFF}{NDF2TIFF} to produce a \TIFFref\ file and then
      various \Netpbmref\ utilities to convert the TIFF file into a GIF file. 

      If the `high' scaling value is less than the `low' value, the output
      image will be a negative. \xref{Bad values}{sun95}{se_badmasking} are
      set to 0 for positives and 255 for negatives.

      Error messages are converted into Starlink style (preceded by \texttt{!}).
   }
   \sstusage{
      ndf2gif in [out] [scale] $\left\{\begin{tabular}{l}
                    low=? high=? \\
                    percentiles=? [nbins=?] \\
                    sigmas=?
                   \end{tabular}\right.$
   }
   \sstparameters{
      \sstsubsection{
         IN = NDF (Read)
      }{
         The name of the input NDF (the \texttt{.sdf} extension is not 
         required).
      }
      \sstsubsection{
         OUT = FILENAME (Write)
      }{
         The name of the GIF file to be generated (a \texttt{.gif} name
         extension is added if it is omitted).
         If OUT is omitted, the value of the IN parameter is used.
         Any existing file with the same name will be overwritten.
      }

   }
     The following parameters are actually parameters of the ADAM task
     \htmlref{NDF2TIFF}{NDF2TIFF}.
     Their values on the NDF2GIF command line are just passed 
     through to NDF2TIFF, which may prompt for other required values.
     The output parameters SCAHIGH and SCALOW will be found in NDF2TIFF's 
     parameter file.

   \begin{description}
      \sstsubsection{
        HIGH = \_DOUBLE (Read)
      }{
        Only required if SCALE is {\texttt{"Scale"}}.
        The array value that scales to 255 in the TIFF file.
        All larger array values are set to 255 when HIGH is greater than
        LOW, otherwise all array values less than HIGH are set to 255.
        The dynamic default is the maximum data value.  There is an
        efficiency gain when both LOW and HIGH are given on the
        command line, because the extreme values need not be computed.
        The highest data value is suggested in prompts.
      }
      \sstsubsection{
        LOW = \_DOUBLE (Read)
      }{
        Only required if SCALE is {\texttt{"Scale"}}.
        The array value that scales to 0 in the TIFF file.
        All smaller array values are also set to 0 when LOW is less than
        HIGH, otherwise all array values greater than LOW are set to 0.
        The dynamic default is the minimum data value.  There is an
        efficiency gain when both LOW and HIGH are given on the
        command line, because the extreme values need not be computed.
        The lowest data value is suggested in prompts.
      }
      \sstsubsection{
        MSG\_FILTER = LITERAL (Read)
      }{
        The output message filtering level, \texttt{QUIET}, \texttt{NORMAL} or
        \texttt{VERBOSE}. If set to verbose, the scaling limits used will be
        displayed. \texttt{[NORMAL]}
      } 
      \sstsubsection{
        NUMBIN  =  \_INTEGER (Read)
      }{
        Only used if SCALE is {\texttt{"Percentiles"}}.
        The number of histogram bins used to compute percentiles for
        scaling. (Percentiles mode) \texttt{[2048]}
      }
      \sstsubsection{
        PERCENTILES( 2 ) = \_REAL (Read)
      }{
        Only required if SCALE is {\texttt{"Percentiles"}}.
        The percentiles that define the scaling limits. For example,
        {\texttt{[25,75]}} would scale between the quartile values.
      }
      \sstsubsection{
         SCALE = LITERAL (Read)
      }{
        The type of scaling to be applied to the array.  The options, which
        may be abbreviated to an unambiguous string and are case-insensitive,
        are described below:
        \ssthitemlist{
           \sstitem
           {\texttt{"Range"}} --- The image is scaled between the minimum and
                          maximum data values. (This is the default.)
           \sstitem
           {\texttt{"Faint"}} --- The image is scaled from the mean minus one
                          standard deviation to the mean plus seven
                          standard deviations.  
           \sstitem
           {\texttt{"Percentiles"}} --- The image is scaled between the values
                          corresponding to two percentiles.  
           \sstitem
           {\texttt{"Scale"}} --- You define the upper and lower limits
                          between which the image is to be scaled.  The
                          application suggests the maximum and the
                          minimum values minimum values when prompting.
           \sstitem
           {\texttt{"Sigmas"}} --- The image is scaled between two
                          standard-deviation limits.  
         }
         \texttt{["Range"]}
      }
      \sstsubsection{
        SIGMAS( 2 ) = \_REAL (Read)
      }{
        Only required if SCALE is {\texttt{"Sigmas"}}.
        The standard-deviation bounds that define the scaling limits.
        To obtain values either side of the mean both a negative and
        a positive value are required.  Thus {\texttt{[-2,3]}} would scale
        between the mean minus two and the mean plus three standard
        deviations.  {\texttt{[3,-2]}} would give the negative of that.
      }
   \end{description}
   \sstresparameters{
      \sstsubsection{
        SCAHIGH = \_DOUBLE (Write)
      }{
        The array value scaled to the maximum colour index for display.
      }
      \sstsubsection{
        SCALOW = \_DOUBLE (Write)
      }{
        The array value scaled to the minimum colour index for display.
      }
   }
   \sstexamples{
      \sstexamplesubsection{
         ndf2gif old new
      }{
         This converts the NDF called old (in file \texttt{old.sdf})
         into a GIF file \texttt{new.gif}.
      }
      \sstexamplesubsection{
         ndf2gif horse scale=pe
      }{
        This converts the NDF called horse (in file \texttt{horse.sdf})
        into a GIF file \texttt{horse.gif} using percentile scaling.
        The user will be prompted for the percentiles to use.
      }
   }
   \sstnotes{
      The following points should be remembered:
      \ssthitemlist{

         \sstitem
            This initial version of the script generates only 256 grey 
            levels and does not use the image colour lookup table so
            absolute data values may be lost.

         \sstitem
            The \Netpbm\ utilities \texttt{tifftopnm} and \texttt{ppmtogif}
            must be available on your PATH.
      }
   }
   \sstdiytopic{
      Related Applications
   }{
      \CONVERT: \htmlref{GIF2NDF}{GIF2NDF}, \htmlref{NDF2TIFF}{NDF2TIFF}.
   }
}

\newpage
\sstroutine{
   NDF2IRAF
}{
   Converts an NDF to an IRAF image
}{
   \sstdescription{
      This application converts an \NDFref\ to an \IRAFref\ image.  See the
      \texttt{"}Notes\texttt{"} for details of the conversion.
   }
   \sstusage{
      ndf2iraf in out [fillbad]
   }
   \sstparameters{
      \sstsubsection{
         IN = NDF (Read)
      }{
         The input NDF data structure.  The suggested default is the
         current NDF if one exists, otherwise it is the current value.
      }
      \sstsubsection{
         FILLBAD = \_REAL (Read)
      }{
         The value used to replace \xref{bad pixels}{sun95}{se_badmasking} in the
         NDF's data array before it is copied to the IRAF file.  A null value
         (\texttt{!}) means no replacements are to be made.  This parameter is ignored if
         there are no bad values.  \texttt{[0]}
      }
      \sstsubsection{
         OUT = LITERAL (Write)
      }{
         The name of the output IRAF image.  Two files are produced
         with the same name but different extensions.  The \texttt{".pix"} file
         contains the data array, and \texttt{".imh"} is the associated header
         file that may contain a copy of the NDF's FITS extension.
         The suggested default is the current value.
      }
      \sstsubsection{
         PROFITS = \_LOGICAL (Read)
      }{
         If \texttt{TRUE}, the contents of the FITS extension of the NDF are
         merged with the header information derived from the standard
         NDF components.  See the \texttt{"}Notes\texttt{"} for details of the merger.
         \texttt{[TRUE]}
      }
      \sstsubsection{
         PROHIS = \_LOGICAL (Read)
      }{
         If \texttt{TRUE}, any NDF history records are written to the primary
         FITS header as HISTORY cards.  These follow the mandatory
         headers and any merged FITS-extension headers (see parameter
         PROFITS).  \texttt{[TRUE]}
      }
   }
   \sstexamples{
      \sstexamplesubsection{
         ndf2iraf abell119 a119
      }{
         Converts an NDF called abell119 into the IRAF image comprising
         the pixel file \texttt{a119.pix} and the header file 
         \texttt{a119.imh}.  If there
         are any bad values present they are copied verbatim to the IRAF
         image.
      }
      \sstexamplesubsection{
         ndf2iraf abell119 a119 noprohis
      }{
         As the previous example, except that NDF HISTORY records are
         not transferred to the headers in \texttt{a119.imh}.
      }
      \sstexamplesubsection{
         ndf2iraf qsospe qsospe fillbad=0
      }{
         Converts the NDF called qsospe to an IRAF image comprising the
         pixel file \texttt{qsospe.imh} and the header file 
         \texttt{qsospe.pix}. 
         Any bad values in the data array are replaced by zero.
      }
      \sstexamplesubsection{
         ndf2iraf qsospe qsospe fillbad=0 profits=f
      }{
         As the previous example, except that any FITS airlock
         information in the NDF are not transferred to the headers in
         \texttt{qsospe.imh}.
      }
   }
   \sstnotes{
      The rules for the conversion are as follows:
      \ssthitemlist{

         \sstitem
         The NDF data array is copied to the \texttt{".pix"} file.  Ancillary
         information listed below is written to the \texttt{".imh"} header
         file in FITS-like headers.

         \sstitem
         The IRAF ``Mini World Co-ordinate System'' (MWCS) is used to
         record axis information whenever one of the following criteria is
         satisfied:

         \begin{enumerate}
            \item the dataset has some linear axes (system=world);

            \item the dataset is one-dimensional with a non-linear axis, or is
            two-dimensional with the first axis non-linear and the
            second being some aperture number or index
            (system=multispec);

            \item the dataset has a linear spectral/dispersion axis along the
            first dimension and all other dimensions are pixel indices
            (system=equispec).
         \end{enumerate}

         \sstitem
         The NDF title, label, units are written to the header keywords
         TITLE, OBJECT, and BUNIT respectively if they are defined.
         Otherwise any values for these keywords found in the FITS
         extension are used (provided parameter PROFITS is \texttt{TRUE}).
         There is a limit of twenty characters for each.

         \sstitem
         The NDF pixel origins are stored in keywords LBOUND\textit{n} for the
         nth dimension when any of the pixel origins is not equal to 1.

         \sstitem
         Keywords HDUCLAS1, HDUCLAS\textit{n} are set to \texttt{"NDF"} and the
         array-component name respectively.

         \sstitem
         The BLANK keyword is set to the Starlink standard
         \xref{bad value}{sun95}{se_badmasking},
         but only for the \_WORD data type and not for a quality array.  It
         appears regardless of whether or not there are bad values
         actually present in the array.

         \sstitem
         HISTORY headers are propagated from the FITS extension when
         PROFITS is \texttt{TRUE}, and from the NDF HISTORY component when
         PROHIS is \texttt{TRUE}.

         \sstitem
         If there is a FITS extension in the NDF, then the elements up
         to the first END keyword of this are added to the `user area' of
         the IRAF header file, when PROFITS=\texttt{TRUE}.  However, certain
         keywords are excluded: SIMPLE, NAXIS, NAXIS\textit{n}, BITPIX, EXTEND,
         PCOUNT, GCOUNT, BSCALE, BZERO, END, and any already created from
         standard components of the NDF listed above.

         \sstitem
         A HISTORY record is added to the IRAF header file indicating
         that it originated in the named NDF and was converted by
         NDF2IRAF.

         \sstitem
         All other NDF components are not propagated.
      }
   }
   \sstdiytopic{
      Related Applications
   }{
      \CONVERT: \htmlref{IRAF2NDF}{IRAF2NDF}.
   }
   \sstdiytopic{
      Pitfalls
   }{
      The IMFORT routines refuse to overwrite an IRAF image if an image
      with the same name exists.  The application then aborts.

      Some of the routines required for accessing the IRAF header image
      are written in SPP. Macros are used to find the start of the
      header line section, this constitutes an `Interface violation' as
      these macros are not part of the IMFORT interface specification.
      It is possible that these may be changed in the future, so
      beware.
   }
   \sstdiytopic{
      References
   }{
      IRAF User Handbook Volume 1A: \texttt{A User's Guide to FORTRAN
      Programming in IRAF, the IMFORT Interface}, by Doug Tody.
    }
   \sstimplementationstatus{
      \sstitemlist{

         \sstitem
         Only handles one-, two-, and three-dimensional NDFs.

         \sstitem
         Of the NDF's array components only the data array may be
         copied.

         \sstitem
         The IRAF image produced has type SIGNED WORD or REAL dependent
         of the type of the NDF's data array.  (The IRAF IMFORT FORTRAN
         subroutine library only supports these data types.)  For \_BYTE,
         \_UBYTE, and \_WORD data arrays the IRAF image will have type
         SIGNED WORD; for all other data types of the NDF data array a
         REAL IRAF image is made.  The pixel type of the image can be
         changed from within IRAF using the `chpixtype' task in the
         `images' package.

         \sstitem
         Bad values may arise due to type conversion.  These too are
         substituted by the (non-null) value of FILLBAD.

         \sstitem
         See \texttt{"}Release Notes\texttt{"} for IRAF Version compatibility.
      }
   }
}

\newpage
\sstroutine{
   NDF2PGM
}{
   Converts an NDF to a PBMPLUS-style PGM-format file.
}{
   \sstdescription{
      This application converts an \NDFref\ to a \PBMPLUSref\ PGM-format file.
      The programme first finds the brightest and darkest pixel values 
      in the image.  It then uses these to determine suitable scaling
      factors to convert the image into an 8-bit representation.  These
      are then output to a simple greyscale \PBMPLUS\ PGM file.
   }
   \sstusage{
      ndf2pgm in out
   }
   \sstparameters{
      \sstsubsection{
         IN = NDF (Read)
      }{
         The name of the input NDF data structure (without the \texttt{.sdf} 
         extension).  The suggested default is the current NDF if one exists,
         otherwise it is the current value.
      }
      \sstsubsection{
         OUT = \_CHAR (Read)
      }{
         The name of the PGM file be generated.
         The \texttt{.pgm} name extension is added to any output filename that
         does not contain it.     
      }
   }
   \sstexamples{
      \sstexamplesubsection{
         ndf2pgm old new
      }{
         This converts the NDF called old (in file \texttt{old.sdf}) to the
         PGM file \texttt{new.pgm}.
      }
      \sstexamplesubsection{
         ndf2pgm in=spectre out=spectre.pgm
      }{
         This converts the NDF called spectre (in file \texttt{spectre.sdf}) 
         to the PGM file \texttt{spectre.pgm}.
      }
   }
   \sstnotes{
      This programme was written for diagnostic purposes and is included just
      in case someone finds it useful.
   }
   \sstimplementationstatus{
      \xref{Bad values}{sun95}{se_badmasking} in the data array are replaced 
      with zero in the output PGM file.
   }
}

\newpage
\sstroutine{
   NDF2TIFF
}{
   Converts an NDF to an 8-bit TIFF-6.0-format file.
}{
   \sstdescription{
      This application converts an \NDFref\ to a Tag Image File Format (\TIFFref).
      One- or two-dimensional arrays can be handled and various methods
      of scaling the data are provided.

      The routine first finds the brightest and darkest pixel values
      required by the particular scaling method in use. It then uses
      these to determine suitable scaling factors and converts the image
      into an 8-bit representation which is then output to a simple
      greyscale TIFF-6.0 file.

      If the `high' scaling value is less than the `low' value, the output
      image will be a negative. \xref{Bad values}{sun95}{se_badmasking} are
      set to 0 for positives and 255 for negatives.
   }
   \sstusage{
      ndf2tiff in out [scale] $\left\{\begin{tabular}{l}
                    low=? high=? \\
                    percentiles=? [nbins=?] \\
                    sigmas=?
                   \end{tabular}\right.$
   }
   \sstparameters{
      \sstsubsection{
        HIGH = \_DOUBLE (Read)
      }{
        The array value that scales to 255 in the TIFF file.  It is only
        required if SCALE is {\texttt{"Scale"}}.
        All larger array values are set to 255 when HIGH is greater than
        LOW, otherwise all array values less than HIGH are set to 255.
        The dynamic default is the maximum data value.  There is an
        efficiency gain when both LOW and HIGH are given on the
        command line, because the extreme values need not be computed.
        The highest data value is suggested in prompts.
      }
      \sstsubsection{
         IN = NDF (Read)
      }{
         The name of the input NDF data structure (without the \texttt{.sdf} 
         extension).  The suggested default is the current NDF if one exists,
         otherwise it is the current value.
      }
      \sstsubsection{
        LOW = \_DOUBLE (Read)
      }{
        The array value that scales to 0 in the TIFF file.  It is only
        required if SCALE is {\texttt{"Scale"}}.
        All smaller array values are also set to 0 when LOW is less than
        HIGH, otherwise all array values greater than LOW are set to 0.
        The dynamic default is the minimum data value.  There is an
        efficiency gain when both LOW and HIGH are given on the
        command line, because the extreme values need not be computed.
        The lowest data value is suggested in prompts.
      }
      \sstsubsection{
        MSG\_FILTER = LITERAL (Read)
      }{
        The output message filtering level, \texttt{QUIET}, \texttt{NORMAL} or
        \texttt{VERBOSE}. If set to verbose, the scaling limits used will be
        displayed.  \texttt{[NORMAL]}
      } 
      \sstsubsection{
        NUMBIN  =  \_INTEGER (Read)
      }{
        The number of histogram bins used to compute percentiles for
        scaling.  It is only used if SCALE is {\texttt{"Percentiles"}}.
        \texttt{[2048]}
      }
      \sstsubsection{
         OUT = \_CHAR (Read)
      }{
         The name of the TIFF file to be generated.  
         A \texttt{.tif} name extension is added to any output filename 
         that does not contain it.     
         Any existing file with the same name will be overwritten.
      }
      \sstsubsection{
        PERCENTILES( 2 ) = \_REAL (Read)
      }{
        The percentiles that define the scaling limits. For example,
        {\texttt{[25,75]}} would scale between the quartile values.
        It is only required if SCALE is {\texttt{"Percentiles"}}.

      }
      \sstsubsection{
         SCALE =  \xref{LITERAL}{sun95}{se_parmenu} (Read)
      }{
        The type of scaling to be applied to the array.
        The options, which may be abbreviated to an unambiguous string
        and are case-insensitive, are described below.
        \ssthitemlist{
           \sstitem
           {\texttt{"Range"}}  --- The image is scaled between the minimum and
                          maximum data values.  (This is the default.)
           \sstitem
           {\texttt{"Faint"}}  --- The image is scaled from the mean minus one
                          standard deviation to the mean plus seven
                          standard deviations.  
           \sstitem
           {\texttt{"Percentiles"}} --- The image is scaled between the values
                          corresponding to two percentiles.  
           \sstitem
           {\texttt{"Scale"}} --- You define the upper and lower limits
                          between which the image is to be scaled.  The
                          application suggests the maximum and the
                          minimum values minimum values when prompting.
           \sstitem
           {\texttt{"Sigmas"}} --- The image is scaled between two
                          standard-deviation limits.  
         }
         \texttt{["Range"]}
      }
      \sstsubsection{
        SIGMAS( 2 ) = \_REAL (Read)
      }{
        Only required if SCALE is {\texttt{"Sigmas"}}.
        The standard-deviation bounds that define the scaling limits.
        To obtain values either side of the mean both a negative and
        a positive value are required.  Thus {\texttt{[-2,3]}} would scale
        between the mean minus two and the mean plus three standard
        deviations.  {\texttt{[3,-2]}} would give the negative of that.
      }
   }
   \sstresparameters{
      \sstsubsection{
        SCAHIGH = \_DOUBLE (Write)
      }{
        The array value scaled to the maximum colour index for display.
      }
      \sstsubsection{
        SCALOW = \_DOUBLE (Write)
      }{
        The array value scaled to the minimum colour index for display.
      }
   }
   \sstexamples{
      \sstexamplesubsection{
         ndf2tiff old new
      }{
         This converts the NDF called old (in file \texttt{old.sdf}) to the
         TIFF file called \texttt{new.tif}.
      }
      \sstexamplesubsection{
         ndf2tiff horse horse pe
      }{
        This converts the NDF called horse (in file \texttt{horse.sdf})
        into a TIFF file \texttt{horse.tif} using percentile scaling.
        The user will be prompted for the percentiles to use.
      }
   }
   \sstnotes{
      This application generates only 256 grey levels and does not use 
      any image colour lookup table so absolute data values may be lost.

      No compression is applied.
   }
   \sstdiytopic{
      Related Applications
   }{
      \CONVERT: \htmlref{TIFF2NDF}{TIFF2NDF}.
   }
}

\newpage
\sstroutine{
   NDF2UNF
}{
   Converts an NDF to a sequential unformatted file
}{
   \sstdescription{
      This application converts an \NDFref\ to a sequential unformatted
      Fortran file.  Only one of the array components may be copied to
      the output file.  Preceding the data there is an optional header
      consisting of either the FITS extension with the values of
      certain keywords replaced by information derived from the NDF, or
      a minimal \FITSref\ header also derived from the NDF.
   }
   \sstusage{
      ndf2unf in out [comp] [noperec]
   }
   \sstparameters{
      \sstsubsection{
         COMP =  \xref{LITERAL}{sun95}{se_parmenu} (Read)
      }{
         The NDF component to be copied.  It may be \texttt{"Data"},
         \texttt{"Quality"} or \texttt{"Variance"}. \texttt{["Data"]}
      }
      \sstsubsection{
         FITS = \_LOGICAL (Read)
      }{
         If \texttt{TRUE}, any FITS extension is written to start of the output
         file, unless there is no extension whereupon a minimal FITS
         header is written to the unformatted file. \texttt{[FALSE]}
      }
      \sstsubsection{
         IN = NDF (Read)
      }{
         Input NDF data structure. The suggested default is the current
         NDF if one exists, otherwise it is the current value.
      }
      \sstsubsection{
         NOPEREC = \_INTEGER (Read)
      }{
         The number of data values per record of the output file.  It
         must be positive.  The suggested default is the current value.
         \texttt{[}The first dimension of the NDF\texttt{]}
      }
      \sstsubsection{
         OUT = FILENAME (Write)
      }{
         Name of the output sequential unformatted file.  The file will
         but always fixed-length records when there is no header.
      }
   }
   \sstexamples{
      \sstexamplesubsection{
         ndf2unf cluster cluster.dat
      }{
         This copies the data array of the NDF called cluster to an
         unformatted file called \texttt{cluster.dat}.  The number of data
         values per record is equal to the size of the first dimension of the
         NDF.
      }
      \sstexamplesubsection{
         ndf2unf cluster cluster.dat v
      }{
         This copies the variance of the NDF called cluster to an
         unformatted file called \texttt{cluster.dat}.  The number of variance
         values per record is equal to the size of the first dimension
         of the NDF.
      }
      \sstexamplesubsection{
         ndf2unf cluster cluster.dat noperec=12
      }{
         This copies the data array of the NDF called cluster to an
         unformatted file called \texttt{cluster.dat}.  There are twelve data
         values per record in \texttt{cluster.dat}.
      }
      \sstexamplesubsection{
         ndf2unf out=ndf234.dat fits in=@234
      }{
         This copies the data array of the NDF called 234 to an
         unformatted file called \texttt{ndf234.dat}.  The number of data values
         per record is equal to the size of the first dimension of the
         NDF.  If there is a FITS extension, it is copied to \texttt{ndf234.dat}
         with substitution of certain keywords, otherwise a minimal
         FITS header is produced.
      }
   }
   \sstnotes{
      The details of the conversion are as follows:
      \ssthitemlist{

         \sstitem
            the NDF array as selected by COMP is written to the
            unformatted file in records following an optional header.

         \sstitem
            HISTORY is not propagated.

         \sstitem
            ORIGIN information is lost.

         \sstitem
            When a header is to be made, it is composed of FITS-like card
            images as follows:
      
         \ssthitemlist{

            \sstitem
               The number of dimensions of the data array is written
               to the keyword NAXIS, and the actual dimensions to NAXIS1,
               NAXIS2 {\it etc.} as appropriate.

            \sstitem
               If the NDF contains any linear axis structures the
               information necessary to generate these structures is
               written to the FITS-like headers. For example, if a linear
               AXIS(1) structure exists in the input NDF the value of the
               first data point is stored with the keyword CRVAL1,
               and the incremental value between successive axis data is
               stored in keyword CDELT1.  By definition the reference pixel is
               1.0 and is stored in keyword CRPIX1.  If there is an axis label
               it is written to keyword CTYPE1, and axis unit is written to CUNIT1.
               (Similarly for AXIS(2) structures {\it etc.}) FITS does not have
               a standard method of storing axis widths and variances, so these
               NDF components will not be propagated to the header.
               Non-linear axis data arrays cannot be represented by CRVAL{\em{n}}
               and CDELT{\em{n}}, and must be ignored.

            \sstitem
               If the input NDF contains TITLE, LABEL or UNITS components
               these are stored with the keywords TITLE, LABEL or BUNIT
               respectively.

            \sstitem
               If the input NDF contains a FITS extension, the FITS items
               may be written to the FITS-like header, with the following
               exceptions:
               \begin{itemize}
               \item BITPIX is derived from the type of the NDF data array,
               and so it is not copied from the NDF FITS extension.
               \item NAXIS, and NAXIS{\em{n}} are derived from the dimensions of the
               NDF data array as described above, so these items are not
               copied from the NDF FITS extension.
               \item The TITLE, LABEL, and BUNIT descriptors are only copied
               if no TITLE, LABEL, and UNITS NDF components respectively
               have already been copied into these headers.
               \item The CDELT{\em{n}}, CRVAL{\em{n}}, CTYPE{\em{n}},
               CUNIT{\em{n}}, and CRTYPE{\em{n}} descriptors
               in the FITS extension are only copied if the input NDF
               contained no linear axis structures.
               \item The standard order of the FITS keywords is preserved,
               thus BITPIX, NAXIS and NAXIS{\em{n}} appear immediately after the
               first card image, which should be SIMPLE.
               \item BSCALE and BZERO in a FITS extension are copied when
               BITPIX is positive, {\it i.e.} the array is not floating-point.
               \end{itemize}

            \sstitem
               An extra header record with keyword UNSIGNED and logical
               value T is added when the array data type is one of the \HDSref\
               unsigned integer types.  This is done because standard FITS
               does not support unsigned integers, and allows (in conjunction
               with BITPIX) applications reading the unformatted file to
               determine the data type of the array.

            \sstitem
               The last header record card will be the standard FITS END.
         }

         \sstitem
            Other extensions are not propagated.
      }
   }
   \sstdiytopic{
      Related Applications
   }{
      \CONVERT: \htmlref{UNF2NDF}{UNF2NDF}.
   }
   \sstimplementationstatus{
      \sstitemlist{

         \sstitem
         The value of \xref{bad pixels}{sun95}{se_badmasking} is not written to a
         FITS-like header record with keyword BLANK.
      }
   }
}

\newpage
\sstroutine{
   SPECX2NDF
}{
   Converts a SPECX map into a simple data cube, or SPECX data files to individual spectra.
}{
   \sstdescription{
      This application converts a SPECX map file into a simple data cube
      formatted  as a standard \NDFref.  It works on map files in version 4.2
      or later of the SPECX format.  It can optionally write a schematic
      of the map grid to a text file.

      In addition, it will also convert an \HDSref\ container file containing
      an array of one-dimensional NDFs holding SPECX spectra into a similar
      container file holding individual, scalar NDFs each holding a single
      spectrum from the supplied array.

      In both cases, WCS components are added to the output NDFs
      describing the spectral and spatial axes.

   }
   \sstusage{
      specx2ndf in out [gridfile] [telescope] [latitude] [longitude]
   }
   \sstparameters{
      \sstsubsection{
         GRIDFILE  =  LITERAL (Read)
      }{
         The name of a text file to which a schematic of the SPECX map
         will be written.  This schematic shows those positions in the
         map grid where spectra were observed.  To indicate that a file
         containing the schematic is not to be written reply with an
         exclamation mark (\texttt{"!"}).  See Section 
         \htmlref{\texttt{"}Schematic of the map grid\texttt{"}}{SCHEMATICMAP}
         (below) for further details.  \texttt{[!]}
      }
      \sstsubsection{
         IN  =  NDF (Read)
      }{
         The name of the input SPECX map, or container file.  The file
         file extension (\texttt{.sdf}) should not be included since it
         is appended automatically by the application.
      }
      \sstsubsection{
         LATITUDE  =  LITERAL (Read)
      }{
         The geodetic (geographic) latitude of the telescope where the
         observation was made.  The value should be specified in
         sexagesimal degrees, with a colon (\texttt{":"}) to separate the
         degrees, minutes and seconds, and no embedded spaces.  Values in
         the northern hemisphere are positive.  The default corresponds
         to the latitude of the JCMT.  \texttt{["19:49:33"]}
      }
      \sstsubsection{
         LONGITUDE  =  LITERAL (Read)
      }{
         The geodetic (geographic) longitude of the telescope where the
         observation was made.  The value should be specified in
         sexagesimal degrees, with a colon (\texttt{":"}) to separate the
         degrees, minutes and seconds, and no embedded spaces.  Following
         the usual geographic convention longitudes west of Greenwich are
         positive.  The default corresponds to the longitude of the
         JCMT.  \texttt{["155:28:47"]}
      }
      \sstsubsection{
         OUT  =  NDF (Write)
      }{
         The name of the output NDF containing the data cube or spectra written
         by the application.  The file extension (\texttt{.sdf}) should not be
         included since it is appended automatically by the application.
      }
      \sstsubsection{
         TELESCOPE  =  LITERAL (Read)
      }{
         The name of the telescope where the observation was made.
         This parameter is used to look up the geodetic (geographical)
         latitude and longitude of the telescope.  See the documentation
         of subroutine SLA\_OBS in 
         \xref{SUN/67}{sun67}{} for a list of permitted values.
         Alternatively, if you wish to explicitly enter the latitude and
         longitude enter \texttt{"COORDS"}.  The values are not case sensitive.
         \texttt{["JCMT"]}
      }
   }
   \sstexamples{
      \sstexamplesubsection{
         specx2ndf  specx\_map  specx\_cube
      }{
         This example generates an NDF data cube called specx\_cube
         (in file \texttt{specx\_cube.sdf}) from the NDF SPECX map called
         specx\_map (in file \texttt{specx\_map.sdf}).  A text file
         containing a schematic of the map grid will not be produced.
      }
      \sstexamplesubsection{
         specx2ndf  specx\_map  specx\_cube  gridfile=map.grid
      }{
         This example generates an NDF data cube called specx\_cube
         (in file \texttt{specx\_cube.sdf}) from the NDF SPECX map called
         specx\_map (in file \texttt{specx\_map.sdf}).  A text file
         containing a schematic of the map grid will be written to file
         \texttt{map.grid}.
      }
   }
   \sstdiytopic{
      Input and output map formats
   }{
      SPECX map files are written by the SPECX package (see 
      \xref{SUN/17}{sun17}{}) for
      reducing spectra observed with heterodyne receivers operating in
      the mm and sub-mm wavelength range of the electromagnetic
      spectrum.  SPECX is usually used to process observations obtained
      with the \htmladdnormallink{James Clerk Maxwell
      Telescope}{http://www.jach.hawaii.edu/JCMT/index.html} (JCMT) in
      Hawaii.

      A SPECX map file comprises a regular `rectangular' two-dimensional
      grid of map positions on the sky, with spectra observed at the grid
      points.  However, a spectrum is not necessarily available at every
      grid position; at some positions a spectrum is not observed in
      order to save observing time.  For example, for a grid centred on
      a typical, roughly circular, object spectra may be omitted for the
      positions at the corners of the grid.  SPECX map files are standard
      Starlink NDF HDS structures.  The principal array of the NDF is a
      two-dimensional array of the grid positions.  The value of each
      element is either a pointer to the spectrum observed there (in
      practice the number of the spectrum in the array where they are
      stored) or a value indicating that a spectrum was not observed at
      this point.  In effect the SPECX map structure is an implementation
      of a sparse array.

      SPECX2NDF expands a SPECX map file into a simple three-dimensional
      data cube, again formatted as a standard NDF, in which the first
      and second pixel axes corresponds to the spatial axes and the third
      axes correspond to the spectral axis.  The advantage of
      this approach is that the converted file can be examined with
      standard applications, such as those in \KAPPA\ (see \xref{SUN/95}{sun95}{}) and
      easily imported into visualisation packages, such as Data Explorer
      (DX, see \xref{SUN/203}{sun203}{} and \xref{SC/2}{sc2}{}).  When the 
      output data cube is created
      the columns corresponding to the positions on the sky grid where
      spectra were not observed are filled with \xref{`bad'
      values}{sun95}{se_masking} (sometimes
      called `magic' or `null' values), to indicate that valid data are
      not available at these positions.  The \xref{standard Starlink
      bad value}{sun95}{se_badmasking}
      is used.  Because of the presence of these bad values the expanded
      cube is usually larger than the original map file.

      The created NDF cube has a WCS component in which axes 1 and 2 are
      RA and DEC, and axis 3 is frequency in units of GHz. The nature of
      these axes can be changed if necessary by subsequent use of the
      \xref{WCSATTRIB}{sun95}{WCSATTRIB} application within the \KAPPA\ package.
      For compatibility
      with older applications, AXIS structures are also added to the output
      cube. Axes 1 and 2 are offsets from the central position of the map,
      with units of seconds of arc, and axis 3 is frequency offset in GHz
      relative to the central frequency.  The pixel origin is placed at the
      source position on axes 1 and 2, and the central frequency on axis 3.

      SPECX2NDF reads map files in version 4.2 or later of the SPECX data 
      format. If it is given a map file in an earlier version of the
      data format it will terminate with an error message.  Note,
      however, that SPECX itself can read map files in earlier versions
      of the SPECX format and convert them to version 4.2.
   }
   \sstdiytopic{
      Schematic of the map grid   \label{SCHEMATICMAP}
   }{
      SPECX2NDF has an optional facility to write a crude schematic of
      the grid of points observed on the sky to an ASCII text file
      suitable for printing or viewing on a terminal screen.  This
      schematic can be useful in interpreting displays of the data cube.
      It shows the positions on the grid where spectra were observed.
      Each spectrum is numbered within the SPECX map structure and the
      first nine are shown using the digits one to nine.  The remaining
      spectra are shown using an asterisk (\texttt{"*"}).  You specify the name
      of the file to which the schematic is written.
      \html{The figure below}
      \latex{Figure~\ref{MAPSCHEMATIC}} shows an example of a schematic.
   }
   \sstdiytopic{
      Auxiliary information
   }{
      SPECX2NDF copies all the auxiliary information present in the
      original map file to the output data cube.  However, the arrays
      holding the original spectra are not copied in order to save
      disk space.
   }
   \sstdiytopic{
      Input and output spectra formats
   }{
      In addition to converting SPECX map files, this application can also
      convert HDS files which hold an array of one-dimensional NDF structures,
      each being a single spectrum extracted by SPECX.  Since arrays of NDFs
      are not easily accessed, this application extracts each NDF from the
      array and creates a new scalar NDF holding the same data within the
      output container file.  The name of the new NDF is \texttt{SPECTRUM\textit{n}} 
      where \textit{n} is its index within the original array of NDFs. 
      Each new scalar NDF is actually three-dimensional and has the format
      described above for an output cube (\textit{i.e.}\ axes 1 and 2 are RA and DEC,
      and axis 3 is frequency).  However, pixel axes 1 and 2 span only a
      single pixel (the size of this single spatial pixel is assumed to be
      half the size of the resolution of the JCMT at the central frequency).
      Inclusion of three-dimensional WCS information allows the individual
      spectra to be aligned on the sky (for instance using the \KAPPA\ 
      \xref{WCSALIGN}{sun95}{WCSALIGN} task). 
   }
}
\clearpage

%\begin{figure}[htbp]
\begin{figure}[t]
\begin{verbatim}
                       Schematic map grid for CO21

                           +---------+
                          9|         |
                          8| 8765432 |
                          7|*******9 |
                          6|******** |
                          5|****1*** |
                          4|******** |
                          3|******** |
                          2|******** |
                          1|         |
                           +---------+
                            123456789
\end{verbatim}
\caption{Example of a map schematic
\label{MAPSCHEMATIC} }
\end{figure}
\vspace*{\fill}
\newpage
\sstroutine{
   TIFF2NDF
}{
   Converts a TIFF file into an NDF. 
}{
   \sstdescription{
      This Bourne-shell script converts a 256 grey-level or
      black-and-white dithered Tag Image File Format (\TIFFref) into an
      unsigned-byte \NDFref\ file.  It handles one- or two-dimensional
      images.  The script uses various \Netpbmref\ utilities to produce a
      FITS file, flipped top to bottom, and then
      \htmlref{FITS2NDF}{FITS2NDF} to produce
      the final NDF.  Error messages are converted into Starlink style
      (preceded by \texttt{!}).
   }
   \sstusage{
      tiff2ndf in [out]
   }
   \sstparameters{
      \sstsubsection{
         IN = FILENAME (Read)
      }{
         The name of the TIFF file to be converted. (A \texttt{.tif} name
         extension is assumed if omitted.)
      }
      \sstsubsection{
         OUT = NDF (Write)
      }{
         The name of the NDF to be generated (without the \texttt{.sdf}
         extension).
         If this is omitted, the value of the IN parameter is used.
      }
   }
   \sstexamples{
      \sstexamplesubsection{
         tiff2ndf old new
      }{
         This converts the TIFF file \texttt{old.tif} into an NDF called new
         (in file \texttt{new.sdf}).
      }
      \sstexamplesubsection{
         tiff2ndf horse
      }{
         This converts the TIFF file \texttt{horse.tif} into an NDF called horse
         (in file \texttt{horse.sdf}).
      }
   }
   \sstnotes{
      The following points should be remembered:
      \ssthitemlist{

         \sstitem
            This initial version of the script handles only greyscale
            or b/w dithered images.  You are responsible for conversion
            of your images to this format prior to use, including 
            the conversion of RGB values to brightness values.

         \sstitem
            Input image file names must have the extension \texttt{.tif}.

         \sstitem
            The \Netpbm\ utilities \texttt{tifftopnm}, \texttt{ppmtopgm},
            \texttt{pnmflip} and \texttt{pnmtofits} must be available on your
            PATH.
      }
   }
   \sstdiytopic{
      Related Applications
   }{
      \CONVERT: \htmlref{NDF2TIFF}{NDF2TIFF}.
   }
}

\newpage
\sstroutine{
   UNF2NDF
}{
   Converts a sequential unformatted file to an NDF
}{
   \sstdescription{
      This application converts a sequential unformatted Fortran file to
      an \NDFref.  Only one of the array components may be created from the
      input file.  Preceding the input data there may be an optional
      header.  This header may be skipped, or may consist of a simple
      \FITSref\ header.  In the former case the shape of the NDF has be to
      be supplied.
   }
   \sstusage{
      unf2ndf in out [comp] noperec [skip] shape [type]
   }
   \sstparameters{
      \sstsubsection{
         COMP = \xref{LITERAL}{sun95}{se_parmenu} (Read)
      }{
         The NDF component to be copied.  It may be \texttt{"Data"},
         \texttt{"Quality"} or \texttt{"Variance"}.  To create a variance or
         quality array the NDF must already exist. \texttt{["Data"]}
      }
      \sstsubsection{
         FITS = \_LOGICAL (Read)
      }{
         If \texttt{TRUE}, the initial records of the unformatted file are
         interpreted as a FITS header (with one card image per record)
         from which the shape, data type, and axis centres are derived.
         The last record of the FITS-like header must be terminated by
         an END keyword; subsequent records in the input file are
         treated as an array component given by COMP.  \texttt{[FALSE]}
      }
      \sstsubsection{
         IN = FILENAME (Read)
      }{
         Name of the input sequential unformatted Fortran file.  The
         file will normally have variable-length records when there is
         a header, but always fixed-length records when there is no
         header.
      }
      \sstsubsection{
         NOPEREC = \_INTEGER (Read)
      }{
         The number of data values per record of the input file.
         It must be positive on UNIX systems.  The suggested default is the
         size of the first dimension of the array if there is no
         current value.  A null (\texttt{!}) value for NOPEREC causes the size
         of first dimension to be used.
      }
      \sstsubsection{
         OUT = NDF (Read and Write)
      }{
         Output NDF data structure.  When COMP is not \texttt{"Data"} the NDF
         is modified rather than a new NDF created.  It becomes the new
         current NDF.
      }
      \sstsubsection{
         SHAPE = \_INTEGER (Read)
      }{
         The shape of the NDF to be created.  For example, \texttt{[40,30,20]}
         would create 40 columns by 30 lines by 10 bands.  It is only
         accessed when FITS is \texttt{FALSE}.
      }
      \sstsubsection{
         SKIP = INTEGER (Read)
      }{
         The number of header records to be skipped at the start of the
         input file before finding the data array or FITS-like header.
         \texttt{[0]}
      }
      \sstsubsection{
         TYPE = LITERAL (Read)
      }{
         The data type of the input file and output NDF.  
         It must be one of the
         following HDS types: \texttt{"\_BYTE"}, \texttt{"\_WORD"},
         \texttt{"\_REAL"},
         \texttt{"\_INTEGER"}, \texttt{"\_DOUBLE"}, \texttt{"\_UBYTE"},
         \texttt{"\_UWORD"} corresponding to signed byte,
         signed word, real, integer, double precision, unsigned byte,
         and unsigned word.  See \xref{SUN/92}{sun92}{} for further details.
         An unambiguous abbreviation may be given.  TYPE is ignored when
         COMP = \texttt{"Quality"} since the QUALITY component must comprise
         unsigned bytes (equivalent to TYPE = \texttt{"\_UBYTE"}) to be a valid
         NDF. The suggested default is the current value.  TYPE is also only
         accessed when FITS is \texttt{FALSE}. \texttt{["\_REAL"]}
      }
   }
   \sstexamples{
      \sstexamplesubsection{
         unf2ndf ngc253.dat ngc253 shape=[100,60] noperec=8
      }{
         This copies a data array from the unformatted file \texttt{ngc253.dat}
         to the NDF called ngc253.  The input file does not contain a
         header section.  The NDF is two-dimensional: 100 elements in \textit{x}
         by 60 in \textit{y}.  Its data array has type \_REAL.  The data records
         each have 8 values.
      }
      \sstexamplesubsection{
         unf2ndf ngc253q.dat ngc253 q 100 shape=[100,60]
      }{
         This copies a quality array from the unformatted file
         \texttt{ngc253q.dat} to an existing NDF called ngc253 (such as 
         created in the first example).  The input file does not contain a 
         header section.  The NDF is two-dimensional: 100 elements in 
         \textit{x} by 60 in \textit{y}.  
         Its data array has type \_UBYTE.
         The data records each have 100 values.
      }
      \sstexamplesubsection{
         unf2ndf ngc253.dat ngc253 fits noperec=!
      }{
         This copies a data array from the unformatted file ngc253.dat
         to the NDF called ngc253.  The input file contains a FITS-like
         header section, which is copied to the FITS extension of the
         NDF.  The shape of the NDF is controlled by the mandatory FITS
         keywords NAXIS, AXIS1, \ldots, AXIS{\em{n}}, and the data type by
         keywords BITPIX and UNSIGNED.  Each data record has AXIS1
         values (except perhaps for the last).
      }
      \sstexamplesubsection{
         unf2ndf type="\_uword" in=ngc253.dat out=ngc253 $\backslash$
      }{
         This copies a data array from the unformatted file \texttt{ngc253.dat}
         to the NDF called ngc253.  The input file does not contain a
         header section.  The NDF has the current shape and data type
         is unsigned word.  The current number of values per record is
         used.
      }
      \sstexamplesubsection{
         unf2ndf spectrum zz skip=2 shape=200 noperec=!
      }{
         This copies a data array from the unformatted file \texttt{spectrum}
         to the NDF called zz.  The input file contains two header
         records that are ignored.  The NDF is one-dimensional
         comprising 200 elements of type \_REAL.  There is one data
         record containing the whole array.
      }
      \sstexamplesubsection{
         unf2ndf spectrum.lis ZZ skip=1 fits noperec=20
      }{
         This copies a data array from the unformatted file 
         \texttt{spectrum.lis}
         to the NDF called ZZ.  The input file contains one header
         record, that is ignored, followed by a FITS-like header
         section, which is copied to the FITS extension of the NDF.
         The shape of the NDF is controlled by the mandatory FITS
         keywords NAXIS, AXIS1, \ldots, AXIS{\em{n}}, and the data type by
         keywords BITPIX and UNSIGNED.  Each data record has AXIS1
         values (except perhaps for the last).
      }
   }
   \sstnotes{
      The details of the conversion are as follows:
      \ssthitemlist{

         \sstitem
            the unformatted-file array is written to the NDF array as
            selected by COMP.  When the NDF is being modified, the shape
            of the new component must match that of the NDF.

         \sstitem
            If the input file contains a FITS-like header, and a new
            NDF is created, {\it i.e.}\ COMP = \texttt{"Data"}, the header
            records are placed within the NDF's FITS extension.  This enables 
            more than one array (input file) to be used to form an NDF.  Note
            that the data array must be created first to make a valid NDF,
            and it's the FITS structure associated with that array that is
            wanted.  Indeed the application prevents you from doing
            otherwise.

         \sstitem
            The FITS-like header defines the properties of the NDF as
            follows:
            \begin{itemize}
            \item BITPIX defines the data type: 8 gives \_BYTE, 16 produces
            \_WORD, 32 makes \_INTEGER, $-$32 gives \_REAL, and $-$64 generates
            \_DOUBLE.  For the first two, if there is an extra header
            record with the keyword UNSIGNED and logical value T, these
            types become \_UBYTE and \_UWORD respectively.  UNSIGNED is
            non-standard, since unsigned integers would not follow in a
            proper FITS file.  However, here it is useful to enable
            unsigned types to be input into an NDF.  UNSIGNED may be
            created by this application's sister, NDF2UNF.  BITPIX is
            ignored for QUALITY data; type \_UBYTE is used.
            \item NAXIS, and NAXIS{\em{n}} define the shape of the NDF.
            \item The TITLE, LABEL, and BUNIT are copied to the NDF
            TITLE, LABEL, and UNITS NDF components respectively.
            \item The CDELT{\em{n}}, CRVAL{\em{n}}, CTYPE{\em{n}}, and CUNIT{\em{n}} keywords make
            linear axis structures within the NDF.  CUNIT{\em{n}} define the
            axis units, and the axis labels are assigned to CTYPE{\em{n}} If
            some are missing, pixel co-ordinates are used for those
            axes.
            \item BSCALE and BZERO in a FITS extension are ignored.
            \item BLANK is not used to indicate which input array values
            should be assigned to a \xref{standard bad
            value}{sun95}{se_badmasking}.
            \item END indicates the last header record unless it
            terminates a dummy header, and the actual data is in an
            extension.
            \end{itemize}

         \sstitem
            Other data item such as HISTORY, data ORIGIN, and axis
            widths are not supported, because the unformatted file has a
            simple structure to enable a diverse set of input files to be
            converted to NDFs, and to limitations of the standard FITS
            header.
      }
   }
   \sstdiytopic{
      Related Applications
   }{
      \CONVERT: \htmlref{NDF2UNF}{NDF2UNF}.
   }
}

\newpage
\section{\label{app_idl}Handling NDFs in \IDLref}
\subsection{\label{idl_easy}The Easy Way}
\emph{Note that this method cannot be used with the 64-bit Solaris version of
IDL (use \texttt{\% idl -32}}).

\IDLref\
function 
\htmlref{READ\_NDF}{READ_NDF}
is available to convert a component of an NDF to an IDL array, and IDL 
procedure 
\htmlref{WRITE\_NDF}{WRITE_NDF}
will create an NDF component from an IDL array. So, for example:
\begin{quote} \begin{verbatim}
IDL> tv,read_ndf('comwest')
\end{verbatim} \end{quote}
will display the data array of the NDF, \texttt{comwest.sdf}, using the IDL
command, \texttt{TV}, and:
\begin{quote} \begin{verbatim}
IDL> write_ndf,field,'stars'
\end{verbatim} \end{quote}
will create an NDF, \texttt{stars.sdf}, of a suitable type and size, and write
the IDL array, \texttt{field}, to its \texttt{DATA\_ARRAY} component.
\begin{quote} \begin{verbatim}
IDL> write_ndf,q,'stars','QUALITY'
\end{verbatim} \end{quote}
will write the IDL array, \texttt{q}, to the \texttt{QUALITY} component of an 
existing NDF, \texttt{stars.sdf}.

Both READ\_NDF and WRITE\_NDF can take special action on \xref{bad
values}{sun95}{se_masking}. For a full description of their arguments see 
\begin{htmlonly}
\htmlref{Specifications of \BCONVERT\ IDL Procedure}{idl_specs}.
\end{htmlonly}
\latex{Appendix~\ref{idl_specs}.}

Complete structures can be handled by function
\htmlref{HDS2IDL}{HDS2IDL}
and procedure
\htmlref{IDL2HDS}{IDL2HDS}.
So, for example:
\begin{quote} \begin{verbatim}
IDL> comwest=hds2idl('comwest') 
% Loaded DLM: HDS2IDL.
IDL> help,comwest,/str
** Structure <40071208>, 5 tags, length=262184, refs=1:
   HDSSTRUCTYPE    STRING    'IMAGE'
   TITLE           STRING    'Comet West, low resolution'
   DATA_ARRAY      FLOAT     Array[256, 256]
   DATA_MIN        FLOAT           3.89062
   DATA_MAX        FLOAT           245.937
IDL>tv,comwest.data_array
\end{verbatim} \end{quote}
will display the same image as the READ\_NDF example above, but the other
components of the NDF are also available in the IDL structure,
\texttt{comwest}, so that:
\begin{quote} \begin{verbatim}
IDL> idl2hds,comwest,'newcomwest'
\end{verbatim} \end{quote}
will create a duplicate of \texttt{comwest.sdf} in \texttt{newcomwest.sdf}.
Exact duplication of the type and structure is not always possible -- see the 
routine descriptions for details.

When \CONVERT\ is installed, the converter procedures and routines are
placed in \texttt{\$CONVERT\_DIR} so, to make them available to IDL,
that directory must be added to the IDL search paths, \texttt{IDL\_PATH} and
\texttt{IDL\_DLM\_PATH}. This will be done if the environment variable 
IDL\_PATH has been set (usually by `sourcing' the 
\texttt{idl\_setup} script) when you start the \CONVERT\ package by typing:
\begin{quote} \begin{verbatim}
% convert
\end{verbatim} \end{quote}
Note that \texttt{convert} must be run \emph{after} sourcing the 
\texttt{idl\_setup} script.

\emph{Note also that having started \CONVERT\, 
\xref{the NDF library}{sun33}{abstract}\latex{ (see SUN/33)},
which is ultimately used by READ\_NDF and WRITE\_NDF, will
allow them (but not HDS2IDL and IDL2HDS) to do 
\htmlref{on-the-fly conversion}{sect_auto}\latex{ (see Section \ref{sect_auto})}
of any files given as parameters. This opens up the possibility of using almost 
any data format with IDL.}

As an example:
\begin{quote} \begin{verbatim}
IDL> tv,read_ndf('moon.imh')
\end{verbatim} \end{quote}
will display the \IRAFref\ file, \texttt{moon.imh}.

\subsection{Other Methods}
\subsubsection{A simple route (but rather slow)}
The simplest route to use when generating data for IDL from NDFs is
to create an ASCII copy of the NDF you are interested in
using the \CONVERT\ package application 
\htmlref{NDF2ASCII}{NDF2ASCII} and then read 
the resulting file with IDL. The steps taken might be something like:

\begin{quote} \begin{verbatim}
% ndf2ascii in=imagein out=fileout  
\end{verbatim} \end{quote}
This will create a file, \texttt{fileout}, containing the DATA component 
of the NDF called \texttt{imagein}. If you want to store the VARIANCE or 
QUALITY components you would use \texttt{comp=v} or \texttt{comp=q} respectively
as additional parameters.

You can then employ a simple IDL batch file such as:
\begin{quote} \begin{verbatim}
; Create IMAGE an 339x244 single precision floating point array.
IMAGE=FLTARR(339,244)

; Open the existing NDF2ASCII file "fileout" for read only access. 
OPENR, UNIT, 'fileout', /GET_LUN
  
; Read formatted input from the specified file unit and 
; place in the variable "IMAGE".
READF, UNIT, IMAGE
 
; Closes the file unit used.
CLOSE, UNIT

; Display the image after suitable scaling. 
TVSCL, IMAGE
\end{verbatim} \end{quote} 
The above example assumes image dimensions of 339$\times$244 pixels. 
If you are in any doubt as to the dimensions of your image you can
determine them using the \KAPPA\ application 
\xref{NDFTRACE}{sun95}{NDFTRACE}. 

One advantage of this route is that the ASCII data can instead be read
directly into byte, integer or double-precision arrays/structures by 
simply substituting \texttt{INTARR}, \texttt{DBLARR} or \texttt{BYTARR} for 
\texttt{FLTARR}. Clearly, it should be remembered that attempting to represent 
floating-point values 
within a byte array will not work properly, whereas a double-precision 
array will accommodate double-precision, floating-point, integer or byte 
values (though somewhat inefficiently in terms of memory consumption).


\subsubsection{A faster route (but a little more complicated)}
However, the NDF2ASCII routine is not fast and makes this route awkward
if time is important. Consequently, you may want to use
\htmlref{NDF2UNF}{NDF2UNF} to create an F77 unformatted sequential file thus:

\begin{quote} \begin{verbatim}
% ndf2unf in=imagein out=fileout noperec=339
\end{verbatim} \end{quote}
Where the \texttt{noperec} number should be the size of the first axis of
the image.  

You can then read the created file using an IDL batch file similar to:

\begin{quote} \begin{verbatim}
; Supply the name of the image and its dimensions.
FNAME='fileout'
SD1=339
SD2=244
  
; Set up the main array and temporary array
; use NOZERO option to avoid initialisation 
IMAGE=INTARR(SD1,SD2,/NOZERO)
TEMP= INTARR(SD1,    /NOZERO)
 
; Display what is going on.
PRINT, "Converting file: ", FNAME
 
; Open the file generated by NDF2UNF for read access only.
OPENR, UNIT, fname, /GET_LUN, /F77_UNFORMATTED
 
; Read the image one record at a time.
FOR I=0,SD2-1  DO BEGIN  READU, UNIT, TEMP & $
  ; Transfer each line into the main image array.
  FOR J=0,SD1-1 DO BEGIN IMAGE(J,I)=TEMP(J) & $ 
ENDFOR & ENDFOR 
 
; Close the opened file unit.
CLOSE, 1
 
; Scale the image for display.
IMAGE=CONGRID(IMAGE,SD1,SD2,/INTERP)
WINDOW, 0,XSIZE=SD1,YSIZE=SD2
 
; Display the image.
TVSCL, IMAGE
\end{verbatim} \end{quote} 

The image is then contained in the integer array \texttt{IMAGE} and may be 
manipulated by IDL. This would allow such operations as storing it as 
a UNIX unformatted file where the image might subsequently be read in 
from disc in one go.

It should be remembered that the data type of the F77 unformatted
file created by NDF2UNF may differ depending on the type of data stored 
in the original NDF. If this is the case you might need to change the 
type definition of the arrays \texttt{image} and \texttt{temp} to reflect this. 
The data type used within each component of an NDF may be determined using 
\xref{NDFTRACE}{sun95}{NDFTRACE}. 


\subsubsection{Using the IDL Astronomy Users' Library}

If this all seems a bit tedious, then those of you with with the IDL 
Astronomy Users' Library installed on your machines might choose to 
take advantage of its FITS conversion procedures to make life easier 
still. Users wishing to obtain a copy of the library can find it at 
\htmladdnormallink{\texttt{\IDLAULURL}}{\IDLAULURL}.

The library contains IDL procedures from a number of sources that allow FITS 
format files to be read into IDL data structures. The routine chosen for the
example below was FXREAD which may be found in the \texttt{/pro/bintable} 
sub-directory. It is part of a comprehensive suite of FITS conversion programs
that seems particularly easy to use.

So if you first convert your NDF into a FITS file using
\htmlref{NDF2FITS}{NDF2FITS} like this:

\begin{quote} \begin{verbatim}
% ndf2fits in=m42 out=m42.fit comp=d
\end{verbatim} \end{quote}
you can then use the following code from within IDL to place the image
into an IDL structure and display it. 

\begin{quote} \begin{verbatim}
; Read the FITS file 
fxread, 'm42.fit', DATA, HEADER
 
; Determine the size of the image
SIZEX=fxpar(header, 'NAXIS1')
SIZEY=fxpar(header, 'NAXIS2')
 
; Find the data type being read
DTYPE=fxpar(header, 'BITPIX')
 
; Scale the image for display
IMAGE=congrid(DATA,SIZEX,SIZEY,/INTERP)
window, 0,xsize=SIZEX,ysize=SIZEY
 
; Display the image
TVSCL, IMAGE
\end{verbatim} \end{quote}
As can be seen above, various values contained within the FITS header 
of the original can be obtained using the FXPAR procedure. 

You should bear in mind that a number of other procedures (such as 
IEEE\_TO\_HOST and GET\_DATE) from the Astronomy Users' Library 
are also needed by IDL when compiling this code. Consequently it is 
essential that the whole library should be obtained from the archive.

\newpage
\section[Specifications of {\small \bf CONVERT} IDL Procedures]
{\label{idl_specs}Specifications of \BCONVERT\ IDL Procedures}
\sstroutine{
   READ\_NDF
}{
    Convert a Starlink NDF to an IDL array.
}{
   \sstdescription{
      This IDL function will convert a Starlink NDF file of up to seven
      dimensions to an IDL array of an appropriate type and shape. 
      \xref{Bad values}{sun95}{se_badmasking}
      in the NDF may be converted to specific values in the IDL array.

      If NDF on-the-fly conversion has been activated, the given filename may
      refer to a file of a different data format which is to be converted.
   }
   \sstusage{
	Result = READ\_NDF( Ndf\_name[, Bad\_value][,COMPONENT=Comp\_name])
   }
   \sstarguments{
      \sstsubsection{
         Ndf\_name
      }{
         A string expression specifying name of the NDF to be read.
      }
      \sstsubsection{
         Bad\_value
      }{
         Optional - A value to replace in the IDL array any occurrence of
         the PRIMDAT bad value in the NDF component.  The value must be the 
         same type as the array.
      }
   }
   \sstkeywords{
      \sstsubsection{
         COMPONENT
      }{
         Set this to a string expression specifying the NDF component to be 
         read.  It may be {\texttt{"DATA"}}, {\texttt{"VARIANCE"}} or 
         {\texttt{"QUALITY"}} and defaults to {\texttt{"DATA"}}. 
         The case of the string does not matter and it may be 
         abbreviated to one or more characters.
      }
   }
   \sstreturnedvalue{
      \sstitem{
         Result
      }{
         An IDL array of a size and type corresponding with the NDF. The type
         correspondence is as follows:
         \begin{description}
            \item[]\_REAL --$>$ floating
            \item[]\_DOUBLE --$>$ double-precision
            \item[]\_UBYTE --$>$  byte
            \item[]\_WORD --$>$  integer
            \item[]\_INTEGER --$>$  longword integer
         \end{description}
      }
   }
   \sstexamples{
      \item[] Assuming \texttt{my\_ndf.sdf} is an NDF of type \_REAL,
      \sstexamplesubsection{
         IDL> data\_array = read\_ndf('my\_ndf')
      }{
         creates an IDL floating array, data\_array, with the same dimensions
         as the NDF and containing the values from its DATA component.
      }
      \sstexamplesubsection{
         IDL> data\_array = read\_ndf('my\_ndf', !values.f\_nan)
      }{
         As above except that any occurrence of a bad value (VAL\_\_BADR as 
         defined by the Starlink PRIMDAT package) in the NDF will be replaced
         by NaN in the IDL array.
      }
      \sstexamplesubsection{
         IDL> quality = read\_ndf('my\_ndf',comp='q')
      }{
         creates an IDL byte array from the QUALITY component of the same NDF.
         (The QUALITY component is always type \_UBYTE.)
         Note that the keyword 'component' and the value 'QUALITY' are 
         case-independent and can be abbreviated.
      }
   }
   \sstdiytopic{
      Deficiencies
   }{
      No conversion of the given bad value to the appropriate type for
      the array will be attempted; instead an error will be reported.
   }
   \sstdiytopic{
      Related Applications
   }{
      \CONVERT: \htmlref{WRITE\_NDF}{WRITE_NDF}.
   }
}

\newpage
\sstroutine{
   WRITE\_NDF
}{
    Convert an IDL array to a Starlink NDF.
}{
   \sstdescription{
      This IDL procedure will write an IDL array of up to seven dimensions to
      a Starlink NDF. If NDF on-the-fly conversion has been activated, the 
      given filename may refer to a file of a different data format in which
      case the NDF is then automatically converted to the required file type.
   }
   \sstusage{
      IDL> write\_ndf, IDL\_array, Ndf\_name[, Bad\_value][, COMPONENT=Comp\_name]
   }
   \sstarguments{
      \sstsubsection{
         IDL\_array
      }{
         The IDL array to be converted. This may be an array name or constant
         of up to seven dimensions. The type of the NDF component created
         will depend on the type of the given array:
         \begin{description}
            \item[]floating --$>$ \_REAL
            \item[]double-precision --$>$ \_DOUBLE
            \item[]byte --$>$ \_UBYTE
            \item[]integer --$>$ \_WORD
            \item[]longword integer --$>$ \_INTEGER
         \end{description}
         No other types are allowed.
      }
      \sstsubsection{
         Ndf\_name
      }{
         A string expression specifying name of the NDF to be written.
      }
      \sstsubsection{
         Bad\_value
      }{
         Optional - A value any occurrence of which in the IDL array is to be
         replaced by the appropriate PRIMDAT \xref{bad
         value}{sun95}{se_badmasking} in the NDF component.
         If no such value is found, the NDF bad-pixel flag for the component
         is set FALSE.  The value must be the same type as the array.
      }
   }
   \sstkeywords{
      \sstsubsection{
         COMPONENT
      }{
         Set this to a string expression specifying the NDF component to be 
         written.  The following values are allowed:
         \begin{description}
          \item[\texttt{"DATA"}] --- A new NDF is created with the same dimensions as the
                         IDL array, and the DATA component written.
          \item[\texttt{"VARIANCE"}] --- An existing NDF is opened and a new component 
                         written.  The size of the given array must be the same
                         as the NDF.
          \item[\texttt{"QUALITY"}] --- An existing NDF is opened and a new component 
                         written. The size of the given array must be the same
                         as the NDF and the type of the IDL array must be Byte.
           \end{description}
           The case of the string does not matter and it may be abbreviated
           to one or more characters.
      }
   }
   \pagebreak
   \sstexamples{
      \item[] Assuming \texttt{my\_data} is an IDL floating array,
      \sstexamplesubsection{
	   IDL> write\_ndf, my\_data, 'my\_ndf'
      }{
         creates the NDF 'my\_ndf.sdf' with the same dimensions as the IDL
         array 'my\_data', and writes the array to its DATA component (of
         type \_REAL).  No checks on bad values are made.
      }
      \sstexamplesubsection{
         IDL> write\_ndf, my\_data, 'my\_ndf', !values.f\_nan
      }{
         As above except that any occurrence of the value NaN in the array
         will be replaced by the VAL\_\_BADR value as defined by the Starlink
         PRIMDAT package.
      }
      \sstexamplesubsection{
         IDL> write\_ndf, my\_variances, 'my\_ndf', comp='v'
      }{
         Writes the IDL array 'my\_variances' to the VARIANCES component of
         the NDF created above. A check is made that the size of the array 
         corresponds with the size of the NDF. (Note that the keyword 
         COMPONENT and the value \texttt{"VARIANCE"} are case-independent and can 
         be abbreviated.)
      }
   }
   \sstdiytopic{
      Deficiencies
   }{
      No conversion of the given bad value to the appropriate type for
      the array will be attempted; instead an error will be reported.
   }
   \sstdiytopic{
      Related Applications
   }{
      \CONVERT: \htmlref{READ\_NDF}{READ_NDF}.
   }
}

\newpage
\sstroutine{
   HDS2IDL
}{
    Convert a Starlink HDS file to an IDL variable.
}{
   \sstdescription{
      This \IDLref\ function will convert a Starlink HDS object into an IDL
      variable. The object may be a structure or primitive so a complete NDF
      structure can be obtained, instead of just the single component produced
      by function
      \htmlref{READ\_NDF}{READ_NDF}.
   }
   \sstusage{
	Result = HDS2IDL( filename )
   }
   \sstarguments{
      \sstsubsection{
         filename
      }{
         A string expression specifying an HDS object. The specification may
         include slices and cells of arrays.
      }
   }
   \sstreturnedvalue{
      \sstsubsection{
         Result
      }{
         An IDL variable corresponding to the HDS object.
         \emph{Structures and primitive types are not necessarily identical
         (see \texttt{"}Notes\texttt{"}).}
      }
   }
   \sstnotes{
      \sstitemlist{
         \sstitem Type correspondence is as follows:
         \begin{description}
            \item[]\_REAL --$>$ floating
            \item[]\_DOUBLE --$>$ double-precision
            \item[]\_UBYTE --$>$  byte
            \item[]\_WORD --$>$  integer
            \item[]\_INTEGER --$>$  longword integer
            \item[]\_LOGICAL --$>$  longword integer (with name change, see below)
            \item[]\_CHAR --$>$  string (see below)
            \item[]\_BYTE --$>$  integer (see below)
            \item[]\_UWORD --$>$  longword integer (see below)
         \end{description}
      \sstitem IDL structures will have an additional STRING component,
         named HDSSTRUCTYPE, specifying the type of the HDS structure to which
         it corresponds.
      \sstitem \_LOGICAL HDS components become IDL LONG components with
         LOGICAL\_ prefixed to their name.
         Any IDL byte, integer or longword structure component must have this
         name convention if it is to be converted to an HDS \_LOGICAL
         component by 
         \htmlref{\texttt{IDL2HDS}}{IDL2HDS}.
      \sstitem IDL strings created from HDS \_CHAR components will have 
         trailing spaces removed, so there is no way to determine the size of
         the original HDS component.
      \sstitem HDS types \_BYTE and \_UWORD become IDL components
         indistinguishable from components produced from HDS types \_WORD and
         \_INTEGER so, if they are converted back to HDS by
         \htmlref{\texttt{IDL2HDS}}{IDL2HDS},
         their HDS type will have changed.
      \sstitem \label{HDS2IDLstrarr} In an IDL array of structures, each 
         element must be exactly the
         same structure. For HDS this is not the case, therefore an HDS array
         of structures, \textit{NAME}, will become an IDL structure,
         \textit{NAME}, with component HDSSTRUCTYPE set to 
         \textit{TYPE}(\textit{n,m,...})
         (where \textit{TYPE} is the type of the HDS array of structures, and
         \textit{n}, \textit{m} \textit{etc.} are the dimensions) and 
         components
         \textit{NAME\_i\_j...} \textit{etc.} (where \textit{NAME\_i\_j...} 
         \textit{etc.} are structures, one for each element of the HDS array
         of structures).
      }
   }
   \sstexamples{
            \sstexamplesubsection{
         IDL> data\_struct = hds2idl('my\_file')
      }{
         Assuming \texttt{my\_file.sdf} is an HDS, file this creates an
         IDL structure corresponding to it.
      }
   }
   \sstdiytopic{
      Deficiencies
   }{
      \sstitemlist{
         \sstitem It is not possible to obtain an identical structure in all
            cases (see \texttt{"}Notes\texttt{"}).
         \sstitem Complex values are not handled.
      }
   }
   \sstdiytopic{
      Related Applications
   }{
      \CONVERT: \htmlref{IDL2HDS}{IDL2HDS}.
   }
}

\newpage
\sstroutine{
   IDL2HDS
}{
    Convert an IDL variable to a Starlink HDS file.
}{
   \sstdescription{
      This IDL procedure will create a Starlink HDS file corresponding to the
      IDL variable, which may be a scalar, array or structure. 
   }
   \sstusage{
      IDL> idl2hds, IDL\_struct, filename
   }
   \sstarguments{
      \sstsubsection{
         IDL\_struct
      }{
         The IDL variable to be written.
      }
      \sstsubsection{
         filename
      }{
         A string expression specifying name of the HDS file to be written.
         \emph{The HDS structure is not necessarily identical to the IDL
         structure (see \texttt{"}Notes\texttt{"}).}
      }
   }
   \sstnotes{
      \sstitemlist{
         \sstitem Type correspondence is as follows:
         \begin{description}
            \item[]floating --$>$ \_REAL
            \item[]double-precision --$>$ \_DOUBLE
            \item[]byte --$>$ \_UBYTE (unless \_LOGICAL, see below).
            \item[]integer --$>$ \_WORD (unless \_LOGICAL, see below).
            \item[]longword integer --$>$ \_INTEGER  (unless \_LOGICAL, see below).
            \item[]string --$>$ \_CHAR*(IDL string size)
         \end{description}
      \sstitem Each HDS structure will have a type as defined by the
         HDSSTRUCTYPE component of the IDL structure. If there is no such
         component, the HDS type is blank.
      \sstitem Byte, integer or longword integer components which have 
         LOGICAL\_ prefixed to their name will be converted to \_LOGICAL
         HDS components and the prefix will be removed from the name.
      \sstitem IDL strings will be converted into HDS \_CHAR  components
         whose size is the size of the IDL string. If a size greater than the
         used length is required, the IDL string must be padded with blanks.
      \sstitem HDS arrays of structures will be created from IDL structures
         having 
         \htmlref{the form produced from HDS arrays of structures by HDS2IDL}
         {HDS2IDLstrarr}.
      }
   }
   \sstexamples{
      \sstexamplesubsection{
	   IDL> idl2hds, data\_struct, 'my\_file'
      }{
         Assuming \texttt{data\_struct} is an IDL structure, this creates
         the HDS file '\texttt{my\_file.sdf}' with a corresponding
         structure.
      }
   }
   \sstdiytopic{
      Deficiencies
   }{
      \sstitemlist{
         \sstitem It is not possible to obtain an identical structure in all
         cases (see \texttt{"}Notes\texttt{"}).
         \sstitem It is not possible to produce HDS components of type
         \_UWORD or \_BYTE.
         \sstitem Complex values are not handled.
         \sstitem IDL arrays of more than 1 structure are not handled.
         \sstitem Only a complete HDS file can be written.
      }
   }
   \sstdiytopic{
      Related Applications
   }{
      \CONVERT: \htmlref{HDS2IDL}{HDS2IDL}.
   }
}

\newpage

\section[{\small IRAF} Versions]{\label{iraf_versions}{\normalsize IRAF} Versions}
The \CONVERT\ utilities 
\htmlref{NDF2IRAF}{NDF2IRAF}
and
\htmlref{IRAF2NDF}{IRAF2NDF}
are built using copies of relevant \IRAF\ libraries (which are included in
the \CONVERT\ release) so they exhibit the same behaviour as the \IRAF\ version
from which the libraries were extracted.
(There are also some \IRAF\ dependencies in the so-called SPP 
routines of \CONVERT\ -- these originate written in the \IRAF\ SPP language and 
include header files defining the layout of the \IRAF\ image.)
The versions of IRAF2NDF and NDF2IRAF which you use must therefore be
compatible with the version of \IRAF\ which you are using.

A new version of the \IRAFref\ image format was developed for 
\IRAF\ Version 2.11 onwards.
\IRAF\ Version 2.11 onwards will read either the old or new image formats but
will  produce the new format by default. (It can be made to produce old-format
images by setting environment variable \texttt{oifversion=1}.)

\CONVERT\ contains \IRAF\ V2.11 compatible versions of the SPP 
routines and the \IRAF\ libraries.
If you are still running \IRAF\ V2.10, set environment variable 
\texttt{oifversion=1} before running NDF2IRAF. (This includes when running 
Starlink programs from \IRAF\ cl if an output image is produced by 
`on-the-fly' conversion.)

Details of which IRAF version libraries are used in \CONVERT\ will be given in
the release notes.

\section{Release Notes -- V1.2}
\subsection{New Applications}
\begin{description}
\item[\htmlref{READ\_NDF}{READ_NDF}] An IDL function to read an NDF component
into an IDL array.
\item[\htmlref{WRITE\_NDF}{WRITE_NDF}] An IDL procedure to write an IDL array to
an NDF component.
\end{description}
The use of these converters for IDL is described in
\latexelsehtml{Section~\ref{app_idl}}{\htmlref{Handling NDFs in \IDLref}{app_idl}}

\subsection{Changed applications}
\begin{description}
\item[\htmlref{NDF2TIFF}{NDF2TIFF}] Various alternative methods of scaling the
NDF data are provided.
\item[\htmlref{NDF2GIF}{NDF2GIF}] Various alternative methods of scaling the
NDF data are provided (the scaling is performed by 
\htmlref{NDF2TIFF}{NDF2TIFF}).
\item[\htmlref{NDF2FITS}{NDF2FITS}] Now has an additional parameter, ENCODING,
to control the way WCS information is encoded within the FITS header.
\item[\htmlref{NDF2IRAF}{NDF2IRAF}
and
\htmlref{IRAF2NDF}{IRAF2NDF}]
These are re-built using libraries from \IRAFref\ release V2.11.1.
\end{description}


\subsection{Documentation}
This document and \CONVERT\ help has been modified to reflect these changes.

\section{Release Notes -- V1.2-4}
  
All the converters which are ADAM tasks have been combined into a monolith
to save disk space.

The FITS converters now use the CFITSIO library.

\htmlref{FITS2NDF}{FITS2NDF}
 will now report an error but continue to create the output NDF 
if an error occurs in creating history records.  It will also display 
warnings if it finds projections which include unsupported IRAF extensions.

A problem causing error "!! FTPSCL: Error defining the scale and offset."
when running 
\htmlref{NDF2FITS}{NDF2FITS}
has been corrected.

\IRAFref\ release V2.11.3 libraries are used in this release for building
\htmlref{NDF2IRAF}{NDF2IRAF}
and
\htmlref{IRAF2NDF}{IRAF2NDF}.

This document has been updated to include this section and a copyright 
statement. Some bugs in the HTML version header have been corrected.

\section{Release Notes -- V1.3-2}
Two new converters, 
\htmlref{HDS2IDL}{HDS2IDL}
and
\htmlref{IDL2HDS}{IDL2HDS},
are provided for IDL users. They handle complete structures, unlike 
\htmlref{READ\_NDF}{READ_NDF}
and
\htmlref{WRITE\_NDF}{WRITE_NDF}
which only handle the main arrays of NDFs. 

\htmlref{NDF2FITS}{NDF2FITS}
and
\htmlref{FITS2NDF}{FITS2NDF}
have been enhanced by the addition of FITS-AIPS and FITS-PC encodings. Also,
appropriate messages are  output if there are no valid input files.

\htmlref{FITS2NDF}{FITS2NDF}
has also been enhanced to allow more flexibility over the handling of
multi-extension FITS files. Syntax such as \texttt{filename.fit[1]} or
\texttt{"filename.fit[extname=im2]"} can be used to specify a single FITS
extension to be converted, and additional parameters EXTABLE and CONTAINER
allow FITS extensions to be combined to produce a single NDF or a series of
components of a top-level HDS container file.

The scripts associated with the NDF on-the-fly conversion system have been
modified to allow a single extension from a multi-extension FITS file to be
specified.  This feature requires applications to have been built with NDF 
V1.5-6 or later.

Different numbers of axes in the  BaseFrames of the NDF and FitsChan FrameSets
are now allowed when creating the NDF's WCS component.

\htmlref{NDF2FITS}{NDF2FITS} now uses the 
\xref{NDG}{sun2}{}
library to allow conversion of NDFs stored as sub-components within an 
HDS container file (\textit{e.g.} scuba data, \textit{etc}).

\IRAFref\ release V2.11.3 libraries are used in this release for building
\htmlref{NDF2IRAF}{NDF2IRAF}
and
\htmlref{IRAF2NDF}{IRAF2NDF}.

This document has been updated to describe the new applications and
enhancements, and to include the
\htmlref{`{\small IRAF} Versions' section}{iraf_versions}.
Early release-note sections have been removed.
On installation, only a link to the installed \LaTeX\ document is now retained
in the source directory.

\section{Release Notes -- V1.3-5}
\htmlref{FITS2NDF}{FITS2NDF}
now creates a component of type FITS\_HEADER, not NDF, in the container
file for multi-extension FITS files if the FITS HDU does not contain a data
array.

\htmlref{NDF2FITS}{NDF2FITS}
Corrects a problem with NDF2FITS if there is garbage beyond the END header
record in the NDF's FITS airlock, and stops a second ORIGIN header being
output (copied from the airlock) if the ORIGIN parameter has been used to
specify a non-default ORIGIN.

\htmlref{NDF2PGM}{NDF2PGM}
A bug which caused the image to be offset by three bytes has been fixed,

This document has been updated to include this section and to update the
description of the CONTAINER parameter of
\htmlref{FITS2NDF}{FITS2NDF}.

\section{Release Notes -- V1.3-6}
\htmlref{FITS2NDF}{FITS2NDF}
\begin{itemize}
\item Improve error reporting if no WCS encoding can be used.

\item Don't handle WCS if NDIM less than or = 0.

\item Correct handling of NDF history records in the FITS header.
\end{itemize}

\section{Release Notes -- V1.4}
Added application \htmlref{MTFITS2NDF}{MTFITS2NDF} to convert FITS tapes to NDFs.

\section{Release Notes -- V1.4-1}
Updated the IDL converters to work with IDL 5.5. Fix a problem causing a crash
on Linux and document the fact that the Starlink IDL converters cannot be used
with the 64-bit Solaris version of IDL.

Corrected an error in the section of this document describing the easy
way to do IDL/NDF conversions.  \htmlref{WRITE\_NDF}{WRITE_NDF} was
described as if it were a function rather than a procedure.

\section{Release Notes -- V1.4-2}
Parameter WCSATTRS is added to \htmlref{FITS2NDF}{FITS2NDF}.
This enables users to modify the way WCS information is extracted from the
FITS headers. This can be useful when the headers do not conform to conventions. 

\section{Release Notes -- V1.4-3}
Fixed a problem in \htmlref{NDF2FITS}{NDF2FITS} caused by a mismatch of
conventions used in the TFORM\textit{n} and TDIM\textit{n} header cards for
multi-dimensional character arrays.  A corresponding fix was required for
\htmlref{FITS2NDF}{FITS2NDF}.

\section{Release Notes -- V1.4-4}
Fixed a bug in \htmlref{FITS2NDF}{FITS2NDF} caused by a missing END card on a
merged header.
  
\section{Release Notes -- V1.5}

\htmlref{SPECX2NDF}{SPECX2NDF} has been revamped so that it now uses
the new AST \xref{SpecFrame functionality}{sun95}{se_wcsuse},
allowing translation between spectral frames without re-running
SPECX2NDF.  The SPECTRUM parameter (and associated parameters SOR,
DOPPLER) are no longer required since they can be changed after
conversion using \xref{WCSATTRIB}{sun95}{WCSATTRIB} application. {\em
Scripts using SPECX2NDF may need modification.} In addition to dealing
with map files SPECX2NDF has now been extended to deal with
\xref{SPECX}{sun17}{data_formats_in_and_data_migration_to_the_unix_version}
data files; each spectrum in the file is translated to an NDF spectrum
in the output HDS container file.

\htmlref{FITS2NDF}{FITS2NDF} supports INES archive IUE spectra.

Improved propagation of existing world co-ordinate system (WCS)
headers in \htmlref{NDF2FITS}{NDF2FITS} partially from improvements to
the \xref{AST subroutine library}{sun210}{}.  For example, long-slit
spectra with a three-dimensional WCS, but stored in a two-dimensional
image, retain their three-dimensional WCS headers.  Comments are
preserved where values have not changed significantly.

The references to the old VMS-only tasks have been removed from the
documentation because the residual VMS service no longer exists.

\section{Release Notes -- V1.5-4}
Added IRAF compatibility libs for Linux systems.  CONVERT should now
build on any ix86 Linux platform. 

\section{Release Notes -- V1.5-5}

Added FITS-AIPS$++$ and FITS-CLASS encodings to 
\htmlref{FITS2NDF}{FITS2NDF} and \htmlref{NDF2FITS}{NDF2FITS}.

Clarified the description of \htmlref{UNF2NDF}{UNF2NDF} parameter TYPE
so that it is clear that the type given should also be of the input
data, not just the output NDF.

\htmlref{SPECX2NDF}{SPECX2NDF} now creates NDF files using the new
double-sideband SpecFrame.  See the AST documentation for more details
on double-sideband spectra.

\section{Release Notes -- V1.5-6}

FITSGZ is a new on-the-fly conversion format for GZIP-compressed FITS
files.  The recognised extensions are \texttt{fits.gz}, \texttt{fit.gz}, and
\texttt{fts.gz}.

FITS2NDF supports the new AAO Instruments (AAOMEGA and FMOS) that use
the 2dF data structures.
  
\section{Release Notes -- V1.5-7}

\htmlref{FITS2NDF}{FITS2NDF} has a new TYPE group parameter to set the
data type of the NDF, overriding the value propagated from the FITS
BITPIX, or BSCALE and BZERO precision when FMTCNV is \texttt{TRUE}.

\htmlref{NDF2FITS}{NDF2FITS} now supports multi-NDFs HDS container files
through the new CONTAINER and MERGE parameters.

\section{Release Notes -- V1.5-8}

\htmlref{FITS2NDF}{FITS2NDF} parameter FMTCNV has a new allowed value
of \texttt{"Native"} requesting that there is no format conversion, and
the array of numbers stored in the FITS file are copied to a scaled array
within the NDF.  This preserves the data compression.  Parameter TYPE
continues to control the data type of the true unscaled values.

\htmlref{NDF2FITS}{NDF2FITS} parameter BITPIX has a new allowed value
of \texttt{"Native"}.  This requests that should any scaled-form arrays be 
converted, then the data type of the corresponding output FITS array 
is set to the scaled-form array's data type, and that the scale and
offset coefficients for the format conversion are taken from the NDF's
scaled array too.  This new facility preserves the data compression
of large files.  In the absence of a scaled array, the application
behaves as if BITPIX=\texttt{-1} were specified.

\section{Release Notes -- V1.5-9}

\htmlref{NDF2FITS}{NDF2FITS} parameter ENCODING has a new allowed
value of \texttt{"FITS-WCS(CD)"}.  This is the same as
\texttt{"FITS-WCS"} except that it uses the old CD matrix formalism 
to describe the data array's rotation and scaling.

\section{Release Notes -- V1.5-10}

\htmlref{FITS2NDF}{FITS2NDF} supports externally and internally compressed
FITS files.

\htmlref{NDF2FITS}{NDF2FITS} writes integrity check keywords CHECKSUM
and DATASUM at the end of each header if new parameter CHECKSUMS is 
\texttt{TRUE}.

\section{Release Notes -- V1.5-11}

\htmlref{NDF2FITS}{NDF2FITS} writes the correct BUNIT keyword value in 
the IMAGE extension storing the VARIANCE component.  The BUNIT
keyword is absent for a QUALITY array.

There is a new DUPLEX parameter.  When set \texttt{TRUE} (and PROFITS
is also \texttt{TRUE}), it permits the FITS airlock headers to appear
also in the IMAGE extensions for the VARIANCE and QUALITY arrays.

\section{Release Notes -- V1.5-12}

\htmlref{NDF2FITS}{NDF2FITS} now makes special provision for the JCMT
SMURF-package extension.  It treats the extension contents as NDFs
rather than arbitrary HDS structures.

\section{Release Notes -- V1.5-13}

\htmlref{NDF2FITS}{NDF2FITS} supports the propagation of provenance
information to FITS headers.  There is a choice of generic propagation
that attempts to propagate all the information, or to write
CADC-specific headers, or to exclude provenance (the default),
governed by the new PROVENANCE parameter.

NDF2FITS now handles extensions containing only NDFs by adding a dummy
FITS sub-file that retains the name and type of the wrapper
structure.
 
\htmlref{FITS2NDF}{FITS2NDF} processes SMURF-package data
better, permitting a roundtrip via FITS, perserving the original
data structures, save for some additional FITS headers.

\end{document}
