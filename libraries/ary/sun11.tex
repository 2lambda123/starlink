\documentclass[twoside,11pt]{article}

% ? Specify used packages
% \usepackage{graphicx}        %  Use this one for final production.
% \usepackage[draft]{graphicx} %  Use this one for drafting.
% ? End of specify used packages

\pagestyle{myheadings}

% -----------------------------------------------------------------------------
% ? Document identification
% Fixed part
\newcommand{\stardoccategory}  {Starlink User Note}
\newcommand{\stardocinitials}  {SUN}
\newcommand{\stardocsource}    {sun\stardocnumber}

% Variable part - replace [xxx] as appropriate.
\newcommand{\stardocnumber}    {11.3}
\newcommand{\stardocauthors}   {R.F. Warren-Smith}
\newcommand{\stardocdate}      {28th April 1998}
\newcommand{\stardoctitle}     {ARY\\[2.5ex]
                                A Subroutine Library for Accessing\\
                                ARRAY Data Structures}
\newcommand{\stardocversion}   {Version 1.1}
\newcommand{\stardocmanual}    {Reference Manual}
\newcommand{\stardocabstract}  {%
The ARY library is a set of routines for accessing Starlink ARRAY data 
structures built using the Hierarchical Data System (HDS).
}
% ? End of document identification
% -----------------------------------------------------------------------------

% +
%  Name:
%     sun.tex
%
%  Purpose:
%     Template for Starlink User Note (SUN) documents.
%     Refer to SUN/199
%
%  Authors:
%     AJC: A.J.Chipperfield (Starlink, RAL)
%     BLY: M.J.Bly (Starlink, RAL)
%
%  History:
%     17-JAN-1996 (AJC):
%        Original with hypertext macros, based on MDL plain originals.
%     16-JUN-1997 (BLY):
%        Adapted for LaTeX2e.
%        Added picture commands.
%     {Add further history here}
%
% -

\newcommand{\stardocname}{\stardocinitials /\stardocnumber}
\markboth{\stardocname}{\stardocname}
\setlength{\textwidth}{160mm}
\setlength{\textheight}{230mm}
\setlength{\topmargin}{-2mm}
\setlength{\oddsidemargin}{0mm}
\setlength{\evensidemargin}{0mm}
\setlength{\parindent}{0mm}
\setlength{\parskip}{\medskipamount}
\setlength{\unitlength}{1mm}

% -----------------------------------------------------------------------------
%  Hypertext definitions.
%  ======================
%  These are used by the LaTeX2HTML translator in conjunction with star2html.

%  Comment.sty: version 2.0, 19 June 1992
%  Selectively in/exclude pieces of text.
%
%  Author
%    Victor Eijkhout                                      <eijkhout@cs.utk.edu>
%    Department of Computer Science
%    University Tennessee at Knoxville
%    104 Ayres Hall
%    Knoxville, TN 37996
%    USA

%  Do not remove the %\begin{rawtex} and %\end{rawtex} lines (used by 
%  star2html to signify raw TeX that latex2html cannot process).
%\begin{rawtex}
\makeatletter
\def\makeinnocent#1{\catcode`#1=12 }
\def\csarg#1#2{\expandafter#1\csname#2\endcsname}

\def\ThrowAwayComment#1{\begingroup
    \def\CurrentComment{#1}%
    \let\do\makeinnocent \dospecials
    \makeinnocent\^^L% and whatever other special cases
    \endlinechar`\^^M \catcode`\^^M=12 \xComment}
{\catcode`\^^M=12 \endlinechar=-1 %
 \gdef\xComment#1^^M{\def\test{#1}
      \csarg\ifx{PlainEnd\CurrentComment Test}\test
          \let\html@next\endgroup
      \else \csarg\ifx{LaLaEnd\CurrentComment Test}\test
            \edef\html@next{\endgroup\noexpand\end{\CurrentComment}}
      \else \let\html@next\xComment
      \fi \fi \html@next}
}
\makeatother

\def\includecomment
 #1{\expandafter\def\csname#1\endcsname{}%
    \expandafter\def\csname end#1\endcsname{}}
\def\excludecomment
 #1{\expandafter\def\csname#1\endcsname{\ThrowAwayComment{#1}}%
    {\escapechar=-1\relax
     \csarg\xdef{PlainEnd#1Test}{\string\\end#1}%
     \csarg\xdef{LaLaEnd#1Test}{\string\\end\string\{#1\string\}}%
    }}

%  Define environments that ignore their contents.
\excludecomment{comment}
\excludecomment{rawhtml}
\excludecomment{htmlonly}
%\end{rawtex}

%  Hypertext commands etc. This is a condensed version of the html.sty
%  file supplied with LaTeX2HTML by: Nikos Drakos <nikos@cbl.leeds.ac.uk> &
%  Jelle van Zeijl <jvzeijl@isou17.estec.esa.nl>. The LaTeX2HTML documentation
%  should be consulted about all commands (and the environments defined above)
%  except \xref and \xlabel which are Starlink specific.

\newcommand{\htmladdnormallinkfoot}[2]{#1\footnote{#2}}
\newcommand{\htmladdnormallink}[2]{#1}
\newcommand{\htmladdimg}[1]{}
\newenvironment{latexonly}{}{}
\newcommand{\hyperref}[4]{#2\ref{#4}#3}
\newcommand{\htmlref}[2]{#1}
\newcommand{\htmlimage}[1]{}
\newcommand{\htmladdtonavigation}[1]{}

%  Starlink cross-references and labels.
\newcommand{\xref}[3]{#1}
\newcommand{\xlabel}[1]{}

%  LaTeX2HTML symbol.
\newcommand{\latextohtml}{{\bf LaTeX}{2}{\tt{HTML}}}

%  Define command to re-centre underscore for Latex and leave as normal
%  for HTML (severe problems with \_ in tabbing environments and \_\_
%  generally otherwise).
\newcommand{\latex}[1]{#1}
\newcommand{\setunderscore}{\renewcommand{\_}{{\tt\symbol{95}}}}
\latex{\setunderscore}

%  Redefine the \tableofcontents command. This procrastination is necessary 
%  to stop the automatic creation of a second table of contents page
%  by latex2html.
\newcommand{\latexonlytoc}[0]{\tableofcontents}

% -----------------------------------------------------------------------------
%  Debugging.
%  =========
%  Remove % on the following to debug links in the HTML version using Latex.

% \newcommand{\hotlink}[2]{\fbox{\begin{tabular}[t]{@{}c@{}}#1\\\hline{\footnotesize #2}\end{tabular}}}
% \renewcommand{\htmladdnormallinkfoot}[2]{\hotlink{#1}{#2}}
% \renewcommand{\htmladdnormallink}[2]{\hotlink{#1}{#2}}
% \renewcommand{\hyperref}[4]{\hotlink{#1}{\S\ref{#4}}}
% \renewcommand{\htmlref}[2]{\hotlink{#1}{\S\ref{#2}}}
% \renewcommand{\xref}[3]{\hotlink{#1}{#2 -- #3}}
% -----------------------------------------------------------------------------
% ? Document specific \newcommand or \newenvironment commands.

\newcommand{\noteroutine}[2]{\textbf{#1}\hspace*{\fill}\nopagebreak \\
                             \hspace*{3em}\emph{#2}\hspace*{\fill}\par}

% +
%  Name:
%     SST.TEX

%  Purpose:
%     Define LaTeX commands for laying out Starlink routine descriptions.

%  Language:
%     LaTeX

%  Type of Module:
%     LaTeX data file.

%  Description:
%     This file defines LaTeX commands which allow routine documentation
%     produced by the SST application PROLAT to be processed by LaTeX and
%     by LaTeX2HTML. The contents of this file should be included in the
%     source prior to any statements that make use of the SST commands.

%  Notes:
%     The commands defined in the style file html.sty provided with LaTeX2html 
%     are used. These should either be made available by using the appropriate
%     sun.tex (with hypertext extensions) or by putting the file html.sty 
%     on your TEXINPUTS path (and including the name as part of the  
%     documentstyle declaration).

%  Authors:
%     RFWS: R.F. Warren-Smith (STARLINK)
%     PDRAPER: P.W. Draper (Starlink - Durham University)

%  History:
%     10-SEP-1990 (RFWS):
%        Original version.
%     10-SEP-1990 (RFWS):
%        Added the implementation status section.
%     12-SEP-1990 (RFWS):
%        Added support for the usage section and adjusted various spacings.
%     8-DEC-1994 (PDRAPER):
%        Added support for simplified formatting using LaTeX2html.
%     {enter_further_changes_here}

%  Bugs:
%     {note_any_bugs_here}

% -

%  Define length variables.
\newlength{\sstbannerlength}
\newlength{\sstcaptionlength}
\newlength{\sstexampleslength}
\newlength{\sstexampleswidth}

%  Define a \tt font of the required size.
\newfont{\ssttt}{cmtt10 scaled 1095}

%  Define a command to produce a routine header, including its name,
%  a purpose description and the rest of the routine's documentation.
\newcommand{\sstroutine}[3]{
   \goodbreak
   \rule{\textwidth}{0.5mm}
   \vspace{-7ex}
   \newline
   \settowidth{\sstbannerlength}{{\Large {\bf #1}}}
   \setlength{\sstcaptionlength}{\textwidth}
   \setlength{\sstexampleslength}{\textwidth}
   \addtolength{\sstbannerlength}{0.5em}
   \addtolength{\sstcaptionlength}{-2.0\sstbannerlength}
   \addtolength{\sstcaptionlength}{-5.0pt}
   \settowidth{\sstexampleswidth}{{\bf Examples:}}
   \addtolength{\sstexampleslength}{-\sstexampleswidth}
   \parbox[t]{\sstbannerlength}{\flushleft{\Large {\bf #1}}}
   \parbox[t]{\sstcaptionlength}{\center{\Large #2}}
   \parbox[t]{\sstbannerlength}{\flushright{\Large {\bf #1}}}
   \begin{description}
      #3
   \end{description}
}

%  Format the description section.
\newcommand{\sstdescription}[1]{\item[Description:] #1}

%  Format the usage section.
\newcommand{\sstusage}[1]{\item[Usage:] \mbox{} \\[1.3ex] {\ssttt #1}}


%  Format the invocation section.
\newcommand{\sstinvocation}[1]{\item[Invocation:]\hspace{0.4em}{\tt #1}}

%  Format the arguments section.
\newcommand{\sstarguments}[1]{
   \item[Arguments:] \mbox{} \\
   \vspace{-3.5ex}
   \begin{description}
      #1
   \end{description}
}

%  Format the returned value section (for a function).
\newcommand{\sstreturnedvalue}[1]{
   \item[Returned Value:] \mbox{} \\
   \vspace{-3.5ex}
   \begin{description}
      #1
   \end{description}
}

%  Format the parameters section (for an application).
\newcommand{\sstparameters}[1]{
   \item[Parameters:] \mbox{} \\
   \vspace{-3.5ex}
   \begin{description}
      #1
   \end{description}
}

%  Format the examples section.
\newcommand{\sstexamples}[1]{
   \item[Examples:] \mbox{} \\
   \vspace{-3.5ex}
   \begin{description}
      #1
   \end{description}
}

%  Define the format of a subsection in a normal section.
\newcommand{\sstsubsection}[1]{ \item[{#1}] \mbox{} \\}

%  Define the format of a subsection in the examples section.
\newcommand{\sstexamplesubsection}[2]{\sloppy
\item[\parbox{\sstexampleslength}{\ssttt #1}] \mbox{} \\ #2 }

%  Format the notes section.
\newcommand{\sstnotes}[1]{\item[Notes:] \mbox{} \\[1.3ex] #1}

%  Provide a general-purpose format for additional (DIY) sections.
\newcommand{\sstdiytopic}[2]{\item[{\hspace{-0.35em}#1\hspace{-0.35em}:}] \mbox{} \\[1.3ex] #2}

%  Format the implementation status section.
\newcommand{\sstimplementationstatus}[1]{
   \item[{Implementation Status:}] \mbox{} \\[1.3ex] #1}

%  Format the bugs section.
\newcommand{\sstbugs}[1]{\item[Bugs:] #1}

%  Format a list of items while in paragraph mode.
\newcommand{\sstitemlist}[1]{
  \mbox{} \\
  \vspace{-3.5ex}
  \begin{itemize}
     #1
  \end{itemize}
}

%  Define the format of an item.
\newcommand{\sstitem}{\item}

%  Now define html equivalents of those already set. These are used by
%  latex2html and are defined in the html.sty files.
\begin{htmlonly}

%  Re-define \ssttt.
   \newcommand{\ssttt}{\tt}

%  \sstroutine.
   \renewcommand{\sstroutine}[3]{
      \subsection{#1\xlabel{#1}-\label{#1}#2}
      \begin{description}
         #3
      \end{description}
   }

%  \sstdescription
   \renewcommand{\sstdescription}[1]{\item[Description:]
      \begin{description}
         #1
      \end{description}
   }

%  \sstusage
   \renewcommand{\sstusage}[1]{\item[Usage:]
      \begin{description}
         {\ssttt #1}
      \end{description}
   }

%  \sstinvocation
   \renewcommand{\sstinvocation}[1]{\item[Invocation:]
      \begin{description}
         {\ssttt #1}
      \end{description}
   }

%  \sstarguments
   \renewcommand{\sstarguments}[1]{
      \item[Arguments:]
      \begin{description}
         #1
      \end{description}
   }

%  \sstreturnedvalue
   \renewcommand{\sstreturnedvalue}[1]{
      \item[Returned Value:]
      \begin{description}
         #1
      \end{description}
   }

%  \sstparameters
   \renewcommand{\sstparameters}[1]{
      \item[Parameters:]
      \begin{description}
         #1
      \end{description}
   }

%  \sstexamples
   \renewcommand{\sstexamples}[1]{
      \item[Examples:]
      \begin{description}
         #1
      \end{description}
   }

%  \sstsubsection
   \renewcommand{\sstsubsection}[1]{\item[{#1}]}

%  \sstexamplesubsection
   \renewcommand{\sstexamplesubsection}[2]{\item[{\ssttt #1}] \\ #2}

%  \sstnotes
   \renewcommand{\sstnotes}[1]{\item[Notes:]
      \begin{description}
         #1
      \end{description}
   }

%  \sstdiytopic
   \renewcommand{\sstdiytopic}[2]{\item[{#1}]
      \begin{description}
         #2
      \end{description}
   }

%  \sstimplementationstatus
   \renewcommand{\sstimplementationstatus}[1]{\item[Implementation Status:] 
      \begin{description}
         #1
      \end{description}
   }

%  \sstitemlist
   \newcommand{\sstitemlist}[1]{
      \begin{itemize}
         #1
      \end{itemize}
   }
\end{htmlonly}

%  End of "sst.tex" layout definitions.
% .
% @(#)sst.tex   1.4   95/06/06 11:46:41   95/06/06 11:49:58


% ? End of document specific commands
% -----------------------------------------------------------------------------
%  Title Page.
%  ===========
\renewcommand{\thepage}{\roman{page}}
\begin{document}
\thispagestyle{empty}

%  Latex document header.
%  ======================
\begin{latexonly}
   CCLRC / {\sc Rutherford Appleton Laboratory} \hfill {\bf \stardocname}\\
   {\large Particle Physics \& Astronomy Research Council}\\
   {\large Starlink Project\\}
   {\large \stardoccategory\ \stardocnumber}
   \begin{flushright}
   \stardocauthors\\
   \stardocdate
   \end{flushright}
   \vspace{-4mm}
   \rule{\textwidth}{0.5mm}
   \vspace{5mm}
   \begin{center}
   {\Huge\bf  \stardoctitle \\ [2.5ex]}
   {\LARGE\bf \stardocversion \\ [4ex]}
   {\Huge\bf  \stardocmanual}
   \end{center}
   \vspace{5mm}

% ? Add picture here if required for the LaTeX version.
%   e.g. \includegraphics[scale=0.3]{filename.ps}
% ? End of picture

% ? Heading for abstract if used.
   \vspace{10mm}
   \begin{center}
      {\Large\bf Abstract}
   \end{center}
% ? End of heading for abstract.
\end{latexonly}

%  HTML documentation header.
%  ==========================
\begin{htmlonly}
   \xlabel{}
   \begin{rawhtml} <H1> \end{rawhtml}
      \stardoctitle\\
      \stardocversion\\
      \stardocmanual
   \begin{rawhtml} </H1> \end{rawhtml}

% ? Add picture here if required for the hypertext version.
%   e.g. \includegraphics[scale=0.7]{filename.ps}
% ? End of picture

   \begin{rawhtml} <P> <I> \end{rawhtml}
   \stardoccategory \stardocnumber \\
   \stardocauthors \\
   \stardocdate
   \begin{rawhtml} </I> </P> <H3> \end{rawhtml}
      \htmladdnormallink{CCLRC}{http://www.cclrc.ac.uk} /
      \htmladdnormallink{Rutherford Appleton Laboratory}
                        {http://www.cclrc.ac.uk/ral} \\
      \htmladdnormallink{Particle Physics \& Astronomy Research Council}
                        {http://www.pparc.ac.uk} \\
   \begin{rawhtml} </H3> <H2> \end{rawhtml}
      \htmladdnormallink{Starlink Project}{http://star-www.rl.ac.uk/}
   \begin{rawhtml} </H2> \end{rawhtml}
   \htmladdnormallink{\htmladdimg{source.gif} Retrieve hardcopy}
      {http://star-www.rl.ac.uk/cgi-bin/hcserver?\stardocsource}\\

%  HTML document table of contents. 
%  ================================
%  Add table of contents header and a navigation button to return to this 
%  point in the document (this should always go before the abstract \section). 
  \label{stardoccontents}
  \begin{rawhtml} 
    <HR>
    <H2>Contents</H2>
  \end{rawhtml}
  \renewcommand{\latexonlytoc}[0]{}
  \htmladdtonavigation{\htmlref{\htmladdimg{contents_motif.gif}}
        {stardoccontents}}

% ? New section for abstract if used.
  \section{\xlabel{abstract}Abstract}
% ? End of new section for abstract
\end{htmlonly}

% -----------------------------------------------------------------------------
% ? Document Abstract. (if used)
%  ==================
\stardocabstract
% ? End of document abstract
% -----------------------------------------------------------------------------
% ? Latex document Table of Contents (if used).
%  ===========================================
  \newpage
  \begin{latexonly}
    \null\vspace {5mm}
    \begin {center}
    \rule{80mm}{0.5mm} \\ [1ex]
    {\Large\bf \stardoctitle \\ [2.5ex]
               \stardocversion} \\ [2ex]
    \rule{80mm}{0.5mm}
    \end{center}
    \setlength{\parskip}{0mm}
    \latexonlytoc
    \setlength{\parskip}{\medskipamount}
    \markboth{\stardocname}{\stardocname}
  \end{latexonly}
% ? End of Latex document table of contents
% -----------------------------------------------------------------------------
\cleardoublepage
\renewcommand{\thepage}{\arabic{page}}
\setcounter{page}{1}
\begin{latexonly}
\null\vspace {5mm}
\begin {center}
\rule{80mm}{0.5mm} \\ [1ex]
{\Large\bf \stardoctitle \\ [2.5ex]
           \stardocversion} \\ [2ex]
\rule{80mm}{0.5mm}
\end{center}
\end{latexonly}

% ? Main text

\section{\xlabel{introduction}Introduction}
\label{introduction}

This is a preliminary document describing a set of routines for accessing
Starlink ARRAY data structures built using the Hierarchical Data System HDS
(\xref{SUN/92}{sun92}{}).  Details of these structures and the design
philosophy behind them can be found in \xref{SGP/38}{sgp38}{}, although
familiarity with that document is not necessarily required in order to
use the routines described here.

This document currently lacks a descriptive section outlining the
philosophy behind the use of the ARY\_ routines.  It is nevertheless
being made available in this form because the ARY\_ system constitutes
an essential sub-component of the NDF\_ system, which contains routines
for accessing Starlink NDF (N-Dimensional Data Format) structures.
These NDF\_ routines are described in \xref{SUN/33}{sun33}{}.

The most likely reason for needing to use the ARY\_ routines directly at
present is for accessing ARRAY structures stored in NDF extensions.
Since most ARY\_ routines closely resemble the equivalent NDF\_ routine,
\xref{SUN/33}{sun33}{} should initially form an adequate introduction to 
their use, in conjunction with the detailed routine descriptions contained in 
Appendix~\ref{ss:routinedescriptions} of this document. 

Note that at present, the ARY\_ system only supports the ``primitive'' and
``simple'' forms of the ARRAY data structure. 

\section{\xlabel{compiling_and_linking}Compiling and Linking}
\label{compiling_and_linking}

Before compiling or linking applications which call ARY\_ routines, you
should first ``log in'' for software development using these routines. 
This is done using the \texttt{ary\_dev} command.

\begin{quote}\begin{verbatim}
% ary_dev
\end{verbatim}\end{quote}

This will generate links to the ARY include files required for 
programming (note that this is only necessary if you reference ARY\_
system include files, or call ARY\_ routines explicitly in your
application).

ADAM applications which call ARY\_ routines may be linked with the link
script \mbox{ary\_link\_adam}, which should be specified on the linker
command line.  For example, to compile and link an application called
\texttt{adamprog} using the \texttt{alink} command, the following might be
used:

\begin{quote}\begin{verbatim}
% alink adamprog.f -o adamprog `ary_link_adam`
\end{verbatim}\end{quote}

A ``stand-alone'' (\emph{i.e.} non-ADAM) version of the ARY\_ system is
also available and should be used by those applications which do not
use ADAM facilities.  This version may be obtained by specifying the
options file \mbox{ary\_link} on the linker command line.  For example,
to compile and link a stand-alone application called \texttt{prog}, the
following might be used:

\begin{quote}\begin{verbatim}
% f77 -O prog.f -o prog `ary_link`
\end{verbatim}\end{quote}

Both versions of the ARY\_ system contain the same set of user-callable
routines.  

\newpage
\appendix

\section{\xlabel{alphabetical_list_of_routines}Alphabetical list of Routines}
\label{alphabetical_list_of_routines}

\noteroutine{ARY\_ANNUL( IARY, STATUS )}
            {Annul an array identifier}
\noteroutine{ARY\_BAD( IARY, CHECK, BAD, STATUS )}
            {Determine if an array may contain bad pixels}
\noteroutine{ARY\_BASE( IARY1, IARY2, STATUS )}
            {Obtain an identifier for a base array}
\noteroutine{ARY\_BOUND( IARY, NDIMX, LBND, UBND, NDIM, STATUS )}
            {Enquire the pixel-index bounds of an array}
\noteroutine{ARY\_CLONE( IARY1, IARY2, STATUS )}
            {Clone an array identifier}
\noteroutine{ARY\_CMPLX( IARY, CMPLX, STATUS )}
            {Determine whether an array holds complex values}
\noteroutine{ARY\_COPY( IARY1, PLACE, IARY2, STATUS )}
            {Copy an array to a new location}
\noteroutine{ARY\_DELET( IARY, STATUS )}
            {Delete an array}
\noteroutine{ARY\_DIM( IARY, NDIMX, DIM, NDIM, STATUS )}
            {Enquire the dimension sizes of an array}
\noteroutine{ARY\_DUPE( IARY1, PLACE, IARY2, STATUS )}
            {Duplicate an array}
\noteroutine{ARY\_FIND( LOC, NAME, IARY, STATUS )}
            {Find an array in an HDS structure and import it into the ARY\_ system}
\noteroutine{ARY\_FORM( IARY, FORM, STATUS )}
            {Obtain the storage form of an array}
\noteroutine{ARY\_FTYPE( IARY, FTYPE, STATUS )}
            {Obtain the full data type of an array}
\noteroutine{ARY\_IMPRT( LOC, IARY, STATUS )}
            {Import an array into the ARY\_ system from HDS}
\noteroutine{ARY\_ISACC( IARY, ACCESS, ISACC, STATUS )}
            {Determine whether a specified type of array access is available}
\noteroutine{ARY\_ISBAS( IARY, BASE, STATUS )}
            {Enquire if an array is a base array}
\noteroutine{ARY\_ISMAP( IARY, MAPPED, STATUS )}
            {Determine if an array is currently mapped}
\noteroutine{ARY\_ISTMP( IARY, TEMP, STATUS )}
            {Determine if an array is temporary}
\noteroutine{ARY\_MAP( IARY, TYPE, MMOD, PNTR, EL, STATUS )}
            {Obtain mapped access to an array}
\noteroutine{ARY\_MAPZ( IARY, TYPE, MMOD, RPNTR, IPNTR, EL, STATUS )}
            {Obtain complex mapped access to an array}
\noteroutine{ARY\_MSG( TOKEN, IARY )}
            {Assign the name of an array to a message token}
\noteroutine{ARY\_NDIM( IARY, NDIM, STATUS )}
            {Enquire the dimensionality of an array}
\noteroutine{ARY\_NEW( FTYPE, NDIM, LBND, UBND, PLACE, IARY, STATUS )}
            {Create a new simple array}
\noteroutine{ARY\_NEWP( FTYPE, NDIM, UBND, PLACE, IARY, STATUS )}
            {Create a new primitive array}
\noteroutine{ARY\_NOACC( ACCESS, IARY, STATUS )}
            {Disable a specified type of access to an array}
\noteroutine{ARY\_OFFS( IARY1, IARY2, MXOFFS, OFFS, STATUS )}
            {Obtain the pixel offset between two arrays}
\noteroutine{ARY\_PLACE( LOC, NAME, PLACE, STATUS )}
            {Obtain an array placeholder}
\noteroutine{ARY\_RESET( IARY, STATUS )}
            {Reset an array to an undefined state}
\noteroutine{ARY\_SAME( IARY1, IARY2, SAME, ISECT, STATUS )}
            {Enquire if two arrays are part of the same base array}
\noteroutine{ARY\_SBAD( BAD, IARY, STATUS )}
            {Set the bad-pixel flag for an array}
\noteroutine{ARY\_SBND( NDIM, LBND, UBND, IARY, STATUS )}
            {Set new pixel-index bounds for an array}
\noteroutine{ARY\_SECT( IARY1, NDIM, LBND, UBND, IARY2, STATUS )}
            {Create an array section}
\noteroutine{ARY\_SHIFT( NSHIFT, SHIFT, IARY, STATUS )}
            {Apply pixel-index shifts to an array}
\noteroutine{ARY\_SIZE( IARY, NPIX, STATUS )}
            {Determine the size of an array}
\noteroutine{ARY\_SSECT( IARY1, IARY2, IARY3, STATUS )}
            {Produce a similar array section to an existing one}
\noteroutine{ARY\_STATE( IARY, STATE, STATUS )}
            {Determine the state of an array (defined or undefined)}
\noteroutine{ARY\_STYPE( FTYPE, IARY, STATUS )}
            {Set a new type for an array}
\noteroutine{ARY\_TEMP( PLACE, STATUS )}
            {Obtain a placeholder for a temporary array}
\noteroutine{ARY\_TRACE( NEWFLG, OLDFLG )}
            {Set the internal ARY\_ system error-tracing flag}
\noteroutine{ARY\_TYPE( IARY, TYPE, STATUS )}
            {Obtain the numeric type of an array}
\noteroutine{ARY\_UNMAP( IARY, STATUS )}
            {Unmap an array}
\noteroutine{ARY\_VALID( IARY, VALID, STATUS )}
            {Determine whether an array identifier is valid}
\noteroutine{ARY\_VERFY( IARY, STATUS )}
            {Verify that an array's data structure is correctly constructed}


\newpage
\section{\xlabel{classified_list_of_routines}Classified list of Routines}
\label{classified_list_of_routines}


\subsection{\xlabel{access_to_existing_arrays}Access to Existing Arrays}
\label{access_to_existing_arrays}

\noteroutine{ARY\_FIND( LOC, NAME, IARY, STATUS )}
            {Find an array in an HDS structure and import it into the ARY\_ system}
\noteroutine{ARY\_IMPRT( LOC, IARY, STATUS )}
            {Import an array into the ARY\_ system from HDS}


\subsection{\xlabel{enquiring_array_attributes}Enquiring Array Attributes}
\label{enquiring_array_attributes}

\noteroutine{ARY\_BAD( IARY, CHECK, BAD, STATUS )}
            {Determine if an array may contain bad pixels}
\noteroutine{ARY\_BOUND( IARY, NDIMX, LBND, UBND, NDIM, STATUS )}
            {Enquire the pixel-index bounds of an array}
\noteroutine{ARY\_CMPLX( IARY, CMPLX, STATUS )}
            {Determine whether an array holds complex values}
\noteroutine{ARY\_DIM( IARY, NDIMX, DIM, NDIM, STATUS )}
            {Enquire the dimension sizes of an array}
\noteroutine{ARY\_FORM( IARY, FORM, STATUS )}
            {Obtain the storage form of an array}
\noteroutine{ARY\_FTYPE( IARY, FTYPE, STATUS )}
            {Obtain the full data type of an array}
\noteroutine{ARY\_ISACC( IARY, ACCESS, ISACC, STATUS )}
            {Determine whether a specified type of array access is available}
\noteroutine{ARY\_ISMAP( IARY, MAPPED, STATUS )}
            {Determine if an array is currently mapped}
\noteroutine{ARY\_ISBAS( IARY, BASE, STATUS )}
            {Enquire if an array is a base array}
\noteroutine{ARY\_ISTMP( IARY, TEMP, STATUS )}
            {Determine if an array is temporary}
\noteroutine{ARY\_NDIM( IARY, NDIM, STATUS )}
            {Enquire the dimensionality of an array}
\noteroutine{ARY\_OFFS( IARY1, IARY2, MXOFFS, OFFS, STATUS )}
            {Obtain the pixel offset between two arrays}
\noteroutine{ARY\_SAME( IARY1, IARY2, SAME, ISECT, STATUS )}
            {Enquire if two arrays are part of the same base array}
\noteroutine{ARY\_SIZE( IARY, NPIX, STATUS )}
            {Determine the size of an array}
\noteroutine{ARY\_STATE( IARY, STATE, STATUS )}
            {Determine the state of an array (defined or undefined)}
\noteroutine{ARY\_TYPE( IARY, TYPE, STATUS )}
            {Obtain the numeric type of an array}
\noteroutine{ARY\_VALID( IARY, VALID, STATUS )}
            {Determine whether an array identifier is valid}
\noteroutine{ARY\_VERFY( IARY, STATUS )}
            {Verify that an array's data structure is correctly constructed}

\subsection{\xlabel{creating_and_deleting_arrays}Creating and Deleting Arrays}
\label{creating_and_deleting_arrays}

\noteroutine{ARY\_DELET( IARY, STATUS )}
            {Delete an array}
\noteroutine{ARY\_DUPE( IARY1, PLACE, IARY2, STATUS )}
            {Duplicate an array}
\noteroutine{ARY\_NEW( FTYPE, NDIM, LBND, UBND, PLACE, IARY, STATUS )}
            {Create a new simple array}
\noteroutine{ARY\_NEWP( FTYPE, NDIM, UBND, PLACE, IARY, STATUS )}
            {Create a new primitive array}


\subsection{\xlabel{setting_array_attributes}Setting Array Attributes}
\label{setting_array_attributes}

\noteroutine{ARY\_NOACC( ACCESS, IARY, STATUS )}
            {Disable a specified type of access to an array}
\noteroutine{ARY\_RESET( IARY, STATUS )}
            {Reset an array to an undefined state}
\noteroutine{ARY\_SBAD( BAD, IARY, STATUS )}
            {Set the bad-pixel flag for an array}
\noteroutine{ARY\_SHIFT( NSHIFT, SHIFT, IARY, STATUS )}
            {Apply pixel-index shifts to an array}
\noteroutine{ARY\_STYPE( FTYPE, IARY, STATUS )}
            {Set a new type for an array}


\subsection{\xlabel{access_to_array_values}Access to Array Values}
\label{access_to_array_values}

\noteroutine{ARY\_MAP( IARY, TYPE, MMOD, PNTR, EL, STATUS )}
            {Obtain mapped access to an array}
\noteroutine{ARY\_MAPZ( IARY, TYPE, MMOD, RPNTR, IPNTR, EL, STATUS )}
            {Obtain complex mapped access to an array}
\noteroutine{ARY\_UNMAP( IARY, STATUS )}
            {Unmap an array}


\subsection{\xlabel{creation_and_control_of_identifiers}%
Creation and Control of Identifiers}
\label{creation_and_control_of_identifiers}

\noteroutine{ARY\_ANNUL( IARY, STATUS )}
            {Annul an array identifier}
\noteroutine{ARY\_BASE( IARY1, IARY2, STATUS )}
            {Obtain an identifier for a base array}
\noteroutine{ARY\_CLONE( IARY1, IARY2, STATUS )}
            {Clone an array identifier}
\noteroutine{ARY\_SECT( IARY1, NDIM, LBND, UBND, IARY2, STATUS )}
            {Create an array section}
\noteroutine{ARY\_SSECT( IARY1, IARY2, IARY3, STATUS )}
            {Produce a similar array section to an existing one}
\noteroutine{ARY\_VALID( IARY, VALID, STATUS )}
            {Determine whether an array identifier is valid}


\subsection{\xlabel{message_system_routines}Message System Routines}
\label{message_system_routines}

\noteroutine{ARY\_MSG( TOKEN, IARY )}
            {Assign the name of an array to a message token}


\subsection{\xlabel{creating_placeholders}Creating Placeholders}
\label{creating_placeholders}

\noteroutine{ARY\_PLACE( LOC, NAME, PLACE, STATUS )}
            {Obtain an array placeholder}
\noteroutine{ARY\_TEMP( PLACE, STATUS )}
            {Obtain a placeholder for a temporary array}


\subsection{\xlabel{copying_arrays}Copying Arrays}
\label{copying_arrays}

\noteroutine{ARY\_COPY( IARY1, PLACE, IARY2, STATUS )}
            {Copy an array to a new location}
\noteroutine{ARY\_DUPE( IARY1, PLACE, IARY2, STATUS )}
            {Duplicate an array}


\subsection{\xlabel{miscellaneous}Miscellaneous}
\label{miscellaneous}

\noteroutine{ARY\_TRACE( NEWFLG, OLDFLG )}
            {Set the internal ARY\_ system error-tracing flag}

\newpage
\section{\xlabel{routine_descriptions}Routine Descriptions}
\label{routine_descriptions}
\label{ss:routinedescriptions}

\sstroutine{
   ARY\_ANNUL
}{
   Annul an array identifier
}{
   \sstdescription{
      The routine annuls the array identifier supplied so that it is no
      longer recognised as a valid identifier by the ARY\_ routines.
      Any resources associated with it are released and made available
      for re-use. If the array is mapped for access, then it is
      automatically unmapped by this routine.
   }
   \sstinvocation{
      CALL ARY\_ANNUL( IARY, STATUS )
   }
   \sstarguments{
      \sstsubsection{
         IARY = INTEGER (Given and Returned)
      }{
         The array identifier to be annulled. A value of ARY\_\_NOID is
         returned (as defined in the include file ARY\_PAR).
      }
      \sstsubsection{
         STATUS = INTEGER (Given and Returned)
      }{
         The global status.
      }
   }
   \sstnotes{
      \sstitemlist{

         \sstitem
         This routine attempts to execute even if STATUS is set on
         entry, although no further error report will be made if it
         subsequently fails under these circumstances. In particular, it
         will fail if the identifier supplied is not initially valid, but
         this will only be reported if STATUS is set to SAI\_\_OK on entry.

         \sstitem
         An error will result if an attempt is made to annul the last
         remaining identifier associated with an array which is in an
         undefined state (unless it is a temporary array, in which case it
         will be deleted at this point).
      }
   }
}
\sstroutine{
   ARY\_BAD
}{
   Determine if an array may contain bad pixels
}{
   \sstdescription{
      The routine returns a logical value indicating whether an array
      may contain bad pixels for which checks must be made when its
      values are processed. Only if the returned value is .FALSE. can
      such checks be omitted.  If the CHECK argument to this routine is
      set .TRUE., then it will perform an explicit check (if necessary)
      to see whether bad pixels are actually present.
   }
   \sstinvocation{
      CALL ARY\_BAD( IARY, CHECK, BAD, STATUS )
   }
   \sstarguments{
      \sstsubsection{
         IARY = INTEGER (Given)
      }{
         Array identifier.
      }
      \sstsubsection{
         CHECK = LOGICAL (Given)
      }{
         Whether to perform an explicit check to see if bad pixels are
         actually present.
      }
      \sstsubsection{
         BAD = LOGICAL (Returned)
      }{
         Whether it is necessary to check for bad pixels when processing
         the array{\tt '}s values.
      }
      \sstsubsection{
         STATUS = INTEGER (Given and Returned)
      }{
         The global status.
      }
   }
%   \newpage
   \sstnotes{
      \sstitemlist{

         \sstitem
         If CHECK is set .FALSE., then the returned value of BAD will
         indicate whether bad pixels might be present and should therefore
         be checked for during subsequent processing. However, even if BAD
         is returned .TRUE. in such circumstances, it is still possible
         that there may not actually be any bad pixels present (for
         instance, in an array section, the region of the base array
         accessed might happen to avoid all the bad pixels).

         \sstitem
         If CHECK is set .TRUE., then an explicit check will be made,
         if necessary, to ensure that BAD is only returned .TRUE. if bad
         pixels are actually present.

         \sstitem
         If the array is mapped for access through the identifier
         supplied, then the value of BAD will refer to the actual mapped
         values. It may differ from its original (unmapped) value if
         conversion errors occurred during the mapping process, or if an
         initialisation option of {\tt '}/ZERO{\tt '} was specified for an array which
         was initially undefined, or if the mapped values have subsequently
         been modified.

         \sstitem
         The BAD argument will always return a value of .TRUE. if the
         array is in an undefined state.
      }
   }
}
\sstroutine{
   ARY\_BASE
}{
   Obtain an identifier for a base array
}{
   \sstdescription{
      The routine returns an identifier for the base array with which
      an array section is associated.
   }
   \sstinvocation{
      CALL ARY\_BASE( IARY1, IARY2, STATUS )
   }
   \sstarguments{
      \sstsubsection{
         IARY1 = INTEGER (Given)
      }{
         Identifier for an existing array section (the routine will also
         work if this is already a base array).
      }
      \sstsubsection{
         IARY2 = INTEGER (Returned)
      }{
         Identifier for the base array with which the section is
         associated.
      }
      \sstsubsection{
         STATUS = INTEGER (Given and Returned)
      }{
         The global status.
      }
   }
   \sstnotes{
      \sstitemlist{

         \sstitem
         If this routine is called with STATUS set, then a value of
         ARY\_\_NOID will be returned for the IARY2 argument, although no
         further processing will occur. The same value will also be
         returned if the routine should fail for any reason. The ARY\_\_NOID
         constant is defined in the include file ARY\_PAR.
      }
   }
}
\sstroutine{
   ARY\_BOUND
}{
   Enquire the pixel-index bounds of an array
}{
   \sstdescription{
      The routine returns the lower and upper pixel-index bounds of
      each dimension of an array, together with the total number of
      dimensions.
   }
   \sstinvocation{
      CALL ARY\_BOUND( IARY, NDIMX, LBND, UBND, NDIM, STATUS )
   }
   \sstarguments{
      \sstsubsection{
         IARY = INTEGER (Given)
      }{
         Array identifier.
      }
      \sstsubsection{
         NDIMX = INTEGER (Given)
      }{
         Maximum number of pixel-index bounds to return (i.e. the
         declared size of the LBND and UBND arguments).
      }
      \sstsubsection{
         LBND( NDIMX ) = INTEGER (Returned)
      }{
         Lower pixel-index bounds for each dimension.
      }
      \sstsubsection{
         UBND( NDIMX ) = INTEGER (Returned)
      }{
         Upper pixel-index bounds for each dimension.
      }
      \sstsubsection{
         NDIM = INTEGER (Returned)
      }{
         Total number of array dimensions.
      }
      \sstsubsection{
         STATUS = INTEGER (Given and Returned)
      }{
         The global status.
      }
   }
   \sstnotes{
      \sstitemlist{

         \sstitem
         If the array has fewer than NDIMX dimensions, then any
         remaining elements of the LBND and UBND arguments will be filled
         with 1{\tt '}s.

         \sstitem
         If the array has more than NDIMX dimensions, then the NDIM
         argument will return the actual number of dimensions. In this
         case only the first NDIMX sets of bounds will be returned, and an
         error will result if the size of any of the remaining dimensions
         exceeds 1.

         \sstitem
         The symbolic constant ARY\_\_MXDIM may be used to declare the
         size of the LBND and UBND arguments so that they will be able to
         hold the maximum number of array bounds that this routine can
         return. This constant is defined in the include file ARY\_PAR.
      }
   }
}
\sstroutine{
   ARY\_CLONE
}{
   Clone an array identifier
}{
   \sstdescription{
      The routine produces a {\tt "}cloned{\tt "} copy of an array identifier (i.e.
      it produces a new identifier describing an array with identical
      attributes to the original).
   }
   \sstinvocation{
      CALL ARY\_CLONE( IARY1, IARY2, STATUS )
   }
   \sstarguments{
      \sstsubsection{
         IARY1 = INTEGER (Given)
      }{
         Array identifier to be cloned.
      }
      \sstsubsection{
         IARY2 = INTEGER (Returned)
      }{
         Cloned identifier.
      }
      \sstsubsection{
         STATUS = INTEGER (Given and Returned)
      }{
         The global status.
      }
   }
   \sstnotes{
      \sstitemlist{

         \sstitem
         If this routine is called with STATUS set, then a value of
         ARY\_\_NOID will be returned for the IARY2 argument, although no
         further processing will occur. The same value will also be
         returned if the routine should fail for any reason. The ARY\_\_NOID
         constant is defined in the include file ARY\_PAR.
      }
   }
}
\sstroutine{
   ARY\_CMPLX
}{
   Determine whether an array holds complex values
}{
   \sstdescription{
      The routine returns a logical value indicating whether an array
      holds complex values.
   }
   \sstinvocation{
      CALL ARY\_CMPLX( IARY, CMPLX, STATUS )
   }
   \sstarguments{
      \sstsubsection{
         IARY = INTEGER (Given)
      }{
         Array identifier.
      }
      \sstsubsection{
         CMPLX = LOGICAL (Returned)
      }{
         Whether the array holds complex values.
      }
      \sstsubsection{
         STATUS = INTEGER (Given and Returned)
      }{
         The global status.
      }
   }
}
\sstroutine{
   ARY\_COPY
}{
   Copy an array to a new location
}{
   \sstdescription{
      The routine copies an array to a new location and returns an
      identifier for the resulting new base array.
   }
   \sstinvocation{
      CALL ARY\_COPY( IARY1, PLACE, IARY2, STATUS )
   }
   \sstarguments{
      \sstsubsection{
         IARY1 = INTEGER (Given)
      }{
         Identifier for the array (or array section) to be copied.
      }
      \sstsubsection{
         PLACE = INTEGER (Given and Returned)
      }{
         An array placeholder (e.g. generated by the ARY\_PLACE routine)
         which indicates the position in the data system where the new
         array will reside. The placeholder is annulled by this
         routine, and a value of ARY\_\_NOPL will be returned (as defined
         in the include file ARY\_PAR).
      }
      \sstsubsection{
         IARY2 = INTEGER (Returned)
      }{
         Identifier for the new array.
      }
      \sstsubsection{
         STATUS = INTEGER (Given and Returned)
      }{
         The global status.
      }
   }
   \sstnotes{
      \sstitemlist{

         \sstitem
         If this routine is called with STATUS set, then a value of
         ARY\_\_NOID will be returned for the IARY2 argument, although no
         further processing will occur. The same value will also be
         returned if the routine should fail for any reason. In either
         event, the placeholder will still be annulled. The ARY\_\_NOID
         constant is defined in the include file ARY\_PAR.
      }
   }
}
\sstroutine{
   ARY\_DELET
}{
   Delete an array
}{
   \sstdescription{
      The routine deletes the specified array. If this is a base array,
      then the associated data object is erased and all array
      identifiers which refer to it (or to sections derived from it)
      become invalid. If the array is mapped for access, then it is
      first unmapped.  If an array section is specified, then this
      routine is equivalent to calling ARY\_ANNUL.
   }
   \sstinvocation{
      CALL ARY\_DELET( IARY, STATUS )
   }
%   \newpage
   \sstarguments{
      \sstsubsection{
         IARY = INTEGER (Given and Returned)
      }{
         Identifier for the array to be deleted. A value of ARY\_\_NOID
         is returned.
      }
      \sstsubsection{
         STATUS = INTEGER (Given and Returned)
      }{
         The global status.
      }
   }
   \sstnotes{
      \sstitemlist{

         \sstitem
         This routine attempts to execute even if STATUS is set on
         entry, although no further error report will be made if it
         subsequently fails under these circumstances.

         \sstitem
         A value of ARY\_\_NOID is always returned for the IARY argument,
         even if the routine should fail. This constant is defined in the
         include file ARY\_PAR.
      }
   }
}
\sstroutine{
   ARY\_DIM
}{
   Enquire the dimension sizes of an array
}{
   \sstdescription{
      The routine returns the size in pixels of each dimension of an
      array, together with the total number of dimensions (the size of
      a dimension is the difference between that dimension{\tt '}s upper and
      lower pixel-index bounds $+$ 1).
   }
   \sstinvocation{
      CALL ARY\_DIM( IARY, NDIMX, DIM, NDIM, STATUS )
   }
   \sstarguments{
      \sstsubsection{
         IARY = INTEGER (Given)
      }{
         Array identifier.
      }
      \sstsubsection{
         NDIMX = INTEGER (Given)
      }{
         Maximum number of dimension sizes to return (i.e. the declared
         size of the DIM argument).
      }
      \sstsubsection{
         DIM( NDIMX ) = INTEGER (Returned)
      }{
         Size of each dimension in pixels.
      }
      \sstsubsection{
         NDIM = INTEGER (Returned)
      }{
         Total number of array dimensions.
      }
      \sstsubsection{
         STATUS = INTEGER (Given and Returned)
      }{
         The global status.
      }
   }
   \sstnotes{
      \sstitemlist{

         \sstitem
         If the array has fewer than NDIMX dimensions, then any
         remaining elements of the DIM argument will be filled with 1{\tt '}s.

         \sstitem
         If the array has more than NDIMX dimensions, then the NDIM
         argument will return the actual number of dimensions. In this
         case only the first NDIMX dimension sizes will be returned, and
         an error will result if the size of any of the excluded
         dimensions exceeds 1.

         \sstitem
         The symbolic constant ARY\_\_MXDIM may be used to declare the
         size of the DIM argument so that it will be able to hold the
         maximum number of array dimension sizes that this routine can
         return. This constant is defined in the include file ARY\_PAR.
      }
   }
}
\sstroutine{
   ARY\_DUPE
}{
   Duplicate an array
}{
   \sstdescription{
      The routine duplicates an array, creating a new base array with
      the same attributes as an existing array (or array section). The
      new array is left in an undefined state.
   }
   \sstinvocation{
      CALL ARY\_DUPE( IARY1, PLACE, IARY2, STATUS )
   }
   \sstarguments{
      \sstsubsection{
         IARY1 = INTEGER (Given)
      }{
         Identifier for the array to be duplicated.
      }
      \sstsubsection{
         PLACE = INTEGER (Given and Returned)
      }{
         An array placeholder (e.g. generated by the ARY\_PLACE routine)
         which indicates the position in the data system where the new
         array will reside. The placeholder is annulled by this
         routine, and a value of ARY\_\_NOPL will be returned (as defined
         in the include file ARY\_PAR).
      }
      \sstsubsection{
         IARY2 = INTEGER (Returned)
      }{
         Identifier for the new duplicate array.
      }
      \sstsubsection{
         STATUS = INTEGER (Given and Returned)
      }{
         The global status.
      }
   }
   \sstnotes{
      \sstitemlist{

         \sstitem
         If this routine is called with STATUS set, then a value of
         ARY\_\_NOID will be returned for the IARY2 argument, although no
         further processing will occur. The same value will also be
         returned if the routine should fail for any reason.  In either
         event, the placeholder will still be annulled. The ARY\_\_NOID
         constant is defined in the include file ARY\_PAR.
      }
   }
}
\sstroutine{
   ARY\_FIND
}{
   Find an array in an HDS structure and import it into the ARY\_
   system
}{
   \sstdescription{
      The routine finds a named array within an HDS structure, imports
      it into the ARY\_ system and issues an identifier for it. The
      imported array may then be manipulated by the ARY\_ routines.
   }
   \sstinvocation{
      CALL ARY\_FIND( LOC, NAME, IARY, STATUS )
   }
   \sstarguments{
      \sstsubsection{
         LOC = CHARACTER $*$ ( $*$ ) (Given)
      }{
         Locator to the enclosing HDS structure.
      }
      \sstsubsection{
         NAME = CHARACTER $*$ ( $*$ ) (Given)
      }{
         Name of the HDS structure component to be imported.
      }
      \sstsubsection{
         IARY = INTEGER (Returned)
      }{
         Array identifier.
      }
      \sstsubsection{
         STATUS = INTEGER (Given and Returned)
      }{
         The global status.
      }
   }
   \sstnotes{
      \sstitemlist{

         \sstitem
         If this routine is called with STATUS set, then a value of
         ARY\_\_NOID will be returned for the IARY argument, although no
         further processing will occur. The same value will also be
         returned if the routine should fail for any reason. The ARY\_\_NOID
         constant is defined in the include file ARY\_PAR.
      }
   }
}
\sstroutine{
   ARY\_FORM
}{
   Obtain the storage form of an array
}{
   \sstdescription{
      The routine returns the storage form of an array as an upper-case
      character string (e.g. {\tt '}SIMPLE{\tt '}).
   }
   \sstinvocation{
      CALL ARY\_FORM( IARY, FORM, STATUS )
   }
   \sstarguments{
      \sstsubsection{
         IARY = INTEGER (Given)
      }{
         Array identifier.
      }
      \sstsubsection{
         FORM = CHARACTER $*$ ( $*$ ) (Returned)
      }{
         Storage form of the array.
      }
      \sstsubsection{
         STATUS = INTEGER (Given and Returned)
      }{
         The global status.
      }
   }
   \sstnotes{
      \sstitemlist{

         \sstitem
         The symbolic constant ARY\_\_SZFRM may be used for declaring the
         length of a character variable to hold the storage form of an
         array. This constant is defined in the include file ARY\_PAR.

         \sstitem
         At present, the ARY\_ routines only support {\tt "}primitive{\tt "} and
         {\tt "}simple{\tt "} arrays, so only the values {\tt '}PRIMITIVE{\tt '} and {\tt '}SIMPLE{\tt '} can
         be returned.
      }
   }
}
\sstroutine{
   ARY\_FTYPE
}{
   Obtain the full data type of an array
}{
   \sstdescription{
      The routine returns the full data type of an array as an
      upper-case character string (e.g. {\tt '}\_REAL{\tt '} or {\tt '}COMPLEX\_BYTE{\tt '}).
   }
   \sstinvocation{
      CALL ARY\_FTYPE( IARY, FTYPE, STATUS )
   }
   \sstarguments{
      \sstsubsection{
         IARY = INTEGER (Given)
      }{
         Array identifier.
      }
      \sstsubsection{
         FTYPE = CHARACTER $*$ ( $*$ ) (Returned)
      }{
         Full data type of the array.
      }
      \sstsubsection{
         STATUS = INTEGER (Given and Returned)
      }{
         The global status.
      }
   }
   \sstnotes{
      \sstitemlist{

         \sstitem
         The symbolic constant ARY\_\_SZFTP may be used for declaring the
         length of a character variable to hold the full data type of an
         array. This constant is defined in the include file ARY\_PAR.
      }
   }
}
\sstroutine{
   ARY\_IMPRT
}{
   Import an array into the ARY\_ system from HDS
}{
   \sstdescription{
      The routine imports an array into the ARY\_ system from HDS and
      issues an identifier for it. The array may then be manipulated by
      the ARY\_ routines.
   }
   \sstinvocation{
      CALL ARY\_IMPRT( LOC, IARY, STATUS )
   }
   \sstarguments{
      \sstsubsection{
         LOC = CHARACTER $*$ ( $*$ ) (Given)
      }{
         HDS locator to an array structure.
      }
      \sstsubsection{
         IARY = INTEGER (Returned)
      }{
         Array identifier.
      }
      \sstsubsection{
         STATUS = INTEGER (Given and Returned)
      }{
         The global status.
      }
   }
   \sstnotes{
      \sstitemlist{

         \sstitem
         The locator supplied as input to this routine may later be
         annulled without affecting the subsequent behaviour of the ARY\_
         system.

         \sstitem
         If this routine is called with STATUS set, then a value of
         ARY\_\_NOID will be returned for the IARY argument, although no
         further processing will occur. The same value will also be
         returned if the routine should fail for any reason. The ARY\_\_NOID
         constant is defined in the include file ARY\_PAR.
      }
   }
}
\sstroutine{
   ARY\_ISACC
}{
   Determine whether a specified type of array access is available
}{
   \sstdescription{
      The routine determines whether a specified type of access to an
      array is available, or whether it has been disabled. If access is
      not available, then any attempt to access the array in this way
      will fail.
   }
   \sstinvocation{
      CALL ARY\_ISACC( IARY, ACCESS, ISACC, STATUS )
   }
   \sstarguments{
      \sstsubsection{
         IARY = INTEGER (Given)
      }{
         Array identifier.
      }
      \sstsubsection{
         ACCESS = CHARACTER $*$ ( $*$ ) (Given)
      }{
         The type of array access required: {\tt '}BOUNDS{\tt '}, {\tt '}DELETE{\tt '},
         {\tt '}SHIFT{\tt '}, {\tt '}TYPE{\tt '} or {\tt '}WRITE{\tt '} (see the Notes section for
         details).
      }
      \sstsubsection{
         ISACC = LOGICAL (Returned)
      }{
         Whether the specified type of access is available.
      }
      \sstsubsection{
         STATUS = INTEGER (Given and Returned)
      }{
         The global status.
      }
   }
   \sstnotes{
      The valid access types control the following operations on the
      array:
      \sstitemlist{

         \sstitem
         {\tt '}BOUNDS{\tt '} permits the pixel-index bounds of a base array to be
         altered.

         \sstitem
         {\tt '}DELETE{\tt '} permits deletion of the array.

         \sstitem
         {\tt '}SHIFT{\tt '} permits pixel-index shifts to be applied to a base
         array.

         \sstitem
         {\tt '}TYPE{\tt '} permits the data type of the array to be altered.

         \sstitem
         {\tt '}WRITE{\tt '} permits new values to be written to the array, or the
         array{\tt '}s state to be reset.
      }
   }
}
\sstroutine{
   ARY\_ISBAS
}{
   Enquire if an array is a base array
}{
   \sstdescription{
      The routine returns a logical value indicating whether the array
      whose identifier is supplied is a base array (as opposed to an
      array section).
   }
   \sstinvocation{
      CALL ARY\_ISBAS( IARY, BASE, STATUS )
   }
   \sstarguments{
      \sstsubsection{
         IARY = INTEGER (Given)
      }{
         Array identifier.
      }
      \sstsubsection{
         BASE = LOGICAL (Returned)
      }{
         Whether the array is a base array.
      }
      \sstsubsection{
         STATUS = INTEGER (Given and Returned)
      }{
         The global status.
      }
   }
}
\sstroutine{
   ARY\_ISMAP
}{
   Determine if an array is currently mapped
}{
   \sstdescription{
      The routine returns a logical value indicating whether an array is
      currently mapped for access through the identifier supplied.
   }
   \sstinvocation{
      CALL ARY\_ISMAP( IARY, MAPPED, STATUS )
   }
   \sstarguments{
      \sstsubsection{
         IARY = INTEGER (Given)
      }{
         Array identifier.
      }
      \sstsubsection{
         MAPPED = LOGICAL (Returned)
      }{
         Whether the array is mapped for access through the IARY
         identifier.
      }
      \sstsubsection{
         STATUS = INTEGER (Given and Returned)
      }{
         The global status.
      }
   }
}
\sstroutine{
   ARY\_ISTMP
}{
   Determine if an array is temporary
}{
   \sstdescription{
      The routine returns a logical value indicating whether the
      specified array is temporary. Temporary arrays are deleted once
      the last identifier which refers to them is annulled.
   }
   \sstinvocation{
      CALL ARY\_ISTMP( IARY, TEMP, STATUS )
   }
   \sstarguments{
      \sstsubsection{
         IARY = INTEGER (Given)
      }{
         Array identifier.
      }
      \sstsubsection{
         TEMP = LOGICAL (Returned)
      }{
         Whether the array is temporary.
      }
      \sstsubsection{
         STATUS = INTEGER (Given and Returned)
      }{
         The global status.
      }
   }
}
\sstroutine{
   ARY\_MAP
}{
   Obtain mapped access to an array
}{
   \sstdescription{
      The routine obtains mapped access an array, returning a pointer
      to the mapped values and a count of the number of elements
      mapped.
   }
   \sstinvocation{
      CALL ARY\_MAP( IARY, TYPE, MMOD, PNTR, EL, STATUS )
   }
   \sstarguments{
      \sstsubsection{
         IARY = INTEGER (Given)
      }{
         Array identifier.
      }
      \sstsubsection{
         TYPE = CHARACTER $*$ ( $*$ ) (Given)
      }{
         The numerical data type required for access (e.g. {\tt '}\_REAL{\tt '}).
      }
      \sstsubsection{
         MMOD = CHARACTER $*$ ( $*$ ) (Given)
      }{
         The mapping mode for access to the array: {\tt '}READ{\tt '}, {\tt '}UPDATE{\tt '} or
         {\tt '}WRITE{\tt '}, with an optional initialisation mode {\tt '}/BAD{\tt '} or
         {\tt '}/ZERO{\tt '} appended.
      }
      \sstsubsection{
         PNTR = INTEGER (Returned)
      }{
         Pointer to the mapped values.
      }
      \sstsubsection{
         EL = INTEGER (Returned)
      }{
         Number of elements mapped.
      }
      \sstsubsection{
         STATUS = INTEGER (Given and Returned)
      }{
         The global status.
      }
   }
}
\sstroutine{
   ARY\_MAPZ
}{
   Obtain complex mapped access to an array
}{
   \sstdescription{
      The routine obtains complex mapped access to an array, returning
      pointers to the real and imaginary values and a count of the
      number of elements mapped.
   }
   \sstinvocation{
      CALL ARY\_MAPZ( IARY, TYPE, MMOD, RPNTR, IPNTR, EL, STATUS )
   }
   \sstarguments{
      \sstsubsection{
         IARY = INTEGER (Given)
      }{
         Array identifier.
      }
      \sstsubsection{
         TYPE = CHARACTER $*$ ( $*$ ) (Given)
      }{
         The numerical data type required for accessing the array (e.g.
         {\tt '}\_REAL{\tt '}).
      }
      \sstsubsection{
         MMOD = CHARACTER $*$ ( $*$ ) (Given)
      }{
         The mapping mode for access to the array: {\tt '}READ{\tt '}, {\tt '}UPDATE{\tt '} or
         {\tt '}WRITE{\tt '}, with an optional initialisation mode {\tt '}/BAD{\tt '} or
         {\tt '}/ZERO{\tt '} appended.
      }
      \sstsubsection{
         RPNTR = INTEGER (Returned)
      }{
         Pointer to the mapped real (i.e. non-imaginary) values.
      }
      \sstsubsection{
         IPNTR = INTEGER (Returned)
      }{
         Pointer to the mapped imaginary values.
      }
      \sstsubsection{
         EL = INTEGER (Returned)
      }{
         Number of elements mapped.
      }
      \sstsubsection{
         STATUS = INTEGER (Given and Returned)
      }{
         The global status.
      }
   }
}
\sstroutine{
   ARY\_MSG
}{
   Assign the name of an array to a message token
}{
   \sstdescription{
      The routine assigns the name of an array to a message token (in a
      form which a user will understand) for use in constructing
      messages with the MSG\_ and ERR\_ routines (see SUN/104).
   }
   \sstinvocation{
      CALL ARY\_MSG( TOKEN, IARY )
   }
   \sstarguments{
      \sstsubsection{
         TOKEN = CHARACTER $*$ ( $*$ ) (Given)
      }{
         Name of the message token.
      }
      \sstsubsection{
         IARY = INTEGER (Given)
      }{
         Array identifier.
      }
   }
   \sstnotes{
      \sstitemlist{

         \sstitem
         This routine has no STATUS argument and performs no error
         checking. If it should fail, then no assignment to the message
         token will be made and this will be apparent in the final
         message.
      }
   }
}
\sstroutine{
   ARY\_NDIM
}{
   Enquire the dimensionality of an array
}{
   \sstdescription{
      The routine determines the number of dimensions which an array
      has.
   }
   \sstinvocation{
      CALL ARY\_NDIM( IARY, NDIM, STATUS )
   }
   \sstarguments{
      \sstsubsection{
         IARY = INTEGER (Given)
      }{
         Array identifier.
      }
      \sstsubsection{
         NDIM = INTEGER (Returned)
      }{
         Number of array dimensions.
      }
      \sstsubsection{
         STATUS = INTEGER (Given and Returned)
      }{
         The global status.
      }
   }
}
\sstroutine{
   ARY\_NEW
}{
   Create a new simple array
}{
   \sstdescription{
      The routine creates a new simple array and returns an identifier
      for it. The array may subsequently be manipulated with the ARY\_
      routines.
   }
   \sstinvocation{
      CALL ARY\_NEW( FTYPE, NDIM, LBND, UBND, PLACE, IARY, STATUS )
   }
   \sstarguments{
      \sstsubsection{
         FTYPE = CHARACTER $*$ ( $*$ ) (Given)
      }{
         Full data type of the array.
      }
      \sstsubsection{
         NDIM = INTEGER (Given)
      }{
         Number of array dimensions.
      }
      \sstsubsection{
         LBND( NDIM ) = INTEGER (Given)
      }{
         Lower pixel-index bounds of the array.
      }
      \sstsubsection{
         UBND( NDIM ) = INTEGER (Given)
      }{
         Upper pixel-index bounds of the array.
      }
      \sstsubsection{
         PLACE = INTEGER (Given and Returned)
      }{
         An array placeholder (e.g. generated by the ARY\_PLACE routine)
         which indicates the position in the data system where the new
         array will reside. The placeholder is annulled by this
         routine, and a value of ARY\_\_NOPL will be returned (as defined
         in the include file ARY\_PAR).
      }
      \sstsubsection{
         IARY = INTEGER (Returned)
      }{
         Identifier for the new array.
      }
      \sstsubsection{
         STATUS = INTEGER (Given and Returned)
      }{
         The global status.
      }
   }
   \sstnotes{
      \sstitemlist{

         \sstitem
         If this routine is called with STATUS set, then a value of
         ARY\_\_NOID will be returned for the IARY argument, although no
         further processing will occur. The same value will also be
         returned if the routine should fail for any reason. In either
         event, the placeholder will still be annulled. The ARY\_\_NOID
         constant is defined in the include file ARY\_PAR.
      }
   }
}
\sstroutine{
   ARY\_NEWP
}{
   Create a new primitive array
}{
   \sstdescription{
      The routine creates a new primitive array and returns an
      identifier for it. The array may subsequently be manipulated with
      the ARY\_ routines.
   }
   \sstinvocation{
      CALL ARY\_NEWP( FTYPE, NDIM, UBND, PLACE, IARY, STATUS )
   }
   \sstarguments{
      \sstsubsection{
         FTYPE = CHARACTER $*$ ( $*$ ) (Given)
      }{
         Data type of the array (e.g. {\tt '}\_REAL{\tt '}). Note that complex types
         are not allowed for primitive arrays.
      }
      \sstsubsection{
         NDIM = INTEGER (Given)
      }{
         Number of array dimensions.
      }
      \sstsubsection{
         UBND( NDIM ) = INTEGER (Given)
      }{
         Upper pixel-index bounds of the array (the lower bound of each
         dimension is taken to be 1).
      }
      \sstsubsection{
         PLACE = INTEGER (Given and Returned)
      }{
         An array placeholder (e.g. generated by the ARY\_PLACE routine)
         which indicates the position in the data system where the new
         array will reside. The placeholder is annulled by this
         routine, and a value of ARY\_\_NOPL will be returned (as defined
         in the include file ARY\_PAR).
      }
      \sstsubsection{
         IARY = INTEGER (Returned)
      }{
         Identifier for the new array.
      }
      \sstsubsection{
         STATUS = INTEGER (Given and Returned)
      }{
         The global status.
      }
   }
   \sstnotes{
      \sstitemlist{

         \sstitem
         If this routine is called with STATUS set, then a value of
         ARY\_\_NOID will be returned for the IARY argument, although no
         further processing will occur. The same value will also be
         returned if the routine should fail for any reason. In either
         event, the placeholder will still be annulled. The ARY\_\_NOID
         constant is defined in the include file ARY\_PAR.
      }
   }
}
\sstroutine{
   ARY\_NOACC
}{
   Disable a specified type of access to an array
}{
   \sstdescription{
      The routine disables the specified type of access to an array, so
      that any subsequent attempt to access it in that way will fail.
      Access restrictions imposed on an array identifier by this
      routine will be propagated to any new identifiers derived from
      it, and cannot be revoked.
   }
   \sstinvocation{
      CALL ARY\_NOACC( ACCESS, IARY, STATUS )
   }
   \sstarguments{
      \sstsubsection{
         ACCESS = CHARACTER $*$ ( $*$ ) (Given)
      }{
         The type of access to be disabled: {\tt '}BOUNDS{\tt '}, {\tt '}DELETE{\tt '},
         {\tt '}MODIFY{\tt '}, {\tt '}SHIFT{\tt '}, {\tt '}TYPE{\tt '} or {\tt '}WRITE{\tt '}.
      }
      \sstsubsection{
         IARY = INTEGER (Given)
      }{
         Array identifier.
      }
      \sstsubsection{
         STATUS = INTEGER (Given and Returned)
      }{
         The global status.
      }
   }
   \sstnotes{
      Disabling each type of access imposes the following restrictions
      on an array:
      \sstitemlist{

         \sstitem
         {\tt '}BOUNDS{\tt '} prevents the pixel-index bounds of a base array from
         being altered.

         \sstitem
         {\tt '}DELETE{\tt '} prevents the array being deleted.

         \sstitem
         {\tt '}MODIFY{\tt '} prevents any form of modification to the array (i.e.
         it disables all the other access types).

         \sstitem
         {\tt '}SHIFT{\tt '} prevents pixel-index shifts from being applied to a
         base array.

         \sstitem
         {\tt '}TYPE{\tt '} prevents the data type of the array from being altered.

         \sstitem
         {\tt '}WRITE{\tt '} prevents new values from being written to the array,
         or the array{\tt '}s state from being reset.
      }
   }
}
\sstroutine{
   ARY\_OFFS
}{
   Obtain the pixel offset between two arrays
}{
   \sstdescription{
      The routine returns the pixel offset for each requested dimension
      between two arrays. These values are the offsets which should be
      added to the pixel indices of the first array to obtain the
      indices of the corresponding pixel in the second array.
   }
   \sstinvocation{
      CALL ARY\_OFFS( IARY1, IARY2, MXOFFS, OFFS, STATUS )
   }
   \sstarguments{
      \sstsubsection{
         IARY1 = INTEGER (Given)
      }{
         First array identifier.
      }
      \sstsubsection{
         IARY2 = INTEGER (Given)
      }{
         Second array identifier.
      }
      \sstsubsection{
         MXOFFS = INTEGER (Given)
      }{
         Maximum number of pixel offsets to return (i.e. the declared
         size of the OFFS argument).
      }
      \sstsubsection{
         OFFS( MXOFFS ) = INTEGER (Returned)
      }{
         Array of pixel offsets for each dimension.
      }
      \sstsubsection{
         STATUS = INTEGER (Given and Returned)
      }{
         The global status.
      }
   }
   \sstnotes{
      \sstitemlist{

         \sstitem
         The two array identifiers supplied need not refer to the same
         base array (although they may often do so). If they do not, then
         the offset between the pixels in each array is determined by
         matching the pixel indices of their respective base arrays.

         \sstitem
         Note that non-zero pixel offsets may exist even for dimensions
         which exceed the dimensionality of either of the two arrays
         supplied. The symbolic constant ARY\_\_MXDIM may be used to declare
         the size of the OFFS argument so that it will be able to hold the
         maximum number of non-zero offsets that this routine can return.
      }
   }
}
\sstroutine{
   ARY\_PLACE
}{
   Obtain an array placeholder
}{
   \sstdescription{
      The routine returns an array placeholder. A placeholder is used
      to identify a position in the underlying data system (HDS) and
      may be passed to other routines (e.g. ARY\_NEW) to indicate where
      a newly created array should be positioned.
   }
   \sstinvocation{
      CALL ARY\_PLACE( LOC, NAME, PLACE, STATUS )
   }
   \sstarguments{
      \sstsubsection{
         LOC = CHARACTER $*$ ( $*$ ) (Given)
      }{
         HDS locator to the structure to contain the new array.
      }
      \sstsubsection{
         NAME = CHARACTER $*$ ( $*$ ) (Given)
      }{
         Name of the new structure component (i.e. the array).
      }
      \sstsubsection{
         PLACE = INTEGER (Returned)
      }{
         Array placeholder identifying the nominated position in the
         data system.
      }
      \sstsubsection{
         STATUS = INTEGER (Given and Returned)
      }{
         The global status.
      }
   }
   \sstnotes{
      \sstitemlist{

         \sstitem
         Placeholders are intended only for local use within an
         application and only a limited number of them are available
         simultaneously. They are always annulled as soon as they are
         passed to another routine to create a new array, where they are
         effectively exchanged for an array identifier.

         \sstitem
         If this routine is called with STATUS set, then a value of
         ARY\_\_NOPL will be returned for the PLACE argument, although no
         further processing will occur. The same value will also be
         returned if the routine should fail for any reason. The ARY\_\_NOPL
         constant is defined in the include file ARY\_PAR.
      }
   }
}
\sstroutine{
   ARY\_RESET
}{
   Reset an array to an undefined state
}{
   \sstdescription{
      The routine resets an array so that its values become undefined.
      Its use is advisable before making format changes to an array if
      retention of the existing values is not required (e.g. before
      changing its data type with the ARY\_STYPE routine); this will
      avoid the cost of converting the existing values.
   }
   \sstinvocation{
      CALL ARY\_RESET( IARY, STATUS )
   }
   \sstarguments{
      \sstsubsection{
         IARY = INTEGER (Given)
      }{
         Array identifier.
      }
      \sstsubsection{
         STATUS = INTEGER (Given and Returned)
      }{
         The global status.
      }
   }
   \sstnotes{
      \sstitemlist{

         \sstitem
         This routine may only be used to reset the state of a base
         array. If an array section is supplied, then it will return
         without action. No error will result.

         \sstitem
         An array cannot be reset while it is mapped for access. This
         routine will fail if this is the case.
      }
   }
}
\sstroutine{
   ARY\_SAME
}{
   Enquire if two arrays are part of the same base array
}{
   \sstdescription{
      The routine determines whether two array identifiers refer to
      parts of the same base array.  If so, it also determines whether
      they intersect.
   }
   \sstinvocation{
      CALL ARY\_SAME( IARY1, IARY2, SAME, ISECT, STATUS )
   }
   \sstarguments{
      \sstsubsection{
         IARY1 = INTEGER (Given)
      }{
         Identifier for the first array (or array section).
      }
      \sstsubsection{
         IARY2 = INTEGER (Given)
      }{
         Identifier for the second array (or array section).
      }
      \sstsubsection{
         SAME = LOGICAL (Returned)
      }{
         Whether the identifiers refer to parts of the same base array.
      }
      \sstsubsection{
         ISECT = LOGICAL (Returned)
      }{
         Whether the arrays intersect.
      }
      \sstsubsection{
         STATUS = INTEGER (Given and Returned)
      }{
         The global status.
      }
   }
   \sstnotes{
      \sstitemlist{

         \sstitem
         Two arrays (or array sections) are counted as intersecting if
         (i) they both refer to the same base array and (ii) altering
         values in one of the arrays can result in the values in the other
         array changing in consequence.
      }
   }
}
\sstroutine{
   ARY\_SBAD
}{
   Set the bad-pixel flag for an array
}{
   \sstdescription{
      The routine sets the value of the bad-pixel flag for an array. A
      call to this routine with BAD set to .TRUE. declares that the
      specified array may contain bad pixel values for which checks
      must be made by algorithms which subsequently processes its
      values.  A call with BAD set to .FALSE. declares that there are
      definitely no bad values present and that subsequent checks for
      such values may be omitted.
   }
   \sstinvocation{
      CALL ARY\_SBAD( BAD, IARY, STATUS )
   }
   \sstarguments{
      \sstsubsection{
         BAD = LOGICAL (Given)
      }{
         Bad-pixel flag value to be set.
      }
      \sstsubsection{
         IARY = INTEGER (Given)
      }{
         Array identifier.
      }
      \sstsubsection{
         STATUS = INTEGER (Given and Returned)
      }{
         The global status.
      }
   }
   \sstnotes{
      \sstitemlist{

         \sstitem
         If the array is mapped for access when this routine is called,
         then the bad-pixel flag will be associated with the mapped
         values. This information will only be transferred to the actual
         data object when the array is unmapped (but only if it was mapped
         for UPDATE or WRITE access). The value transferred may be
         modified if conversion errors occur during the unmapping process.
      }
   }
}
\sstroutine{
   ARY\_SBND
}{
   Set new pixel-index bounds for an array
}{
   \sstdescription{
      The routine sets new pixel-index bounds for an array (or array
      section). The number of array dimensions may also be changed.  If
      a base array is specified, then a permanent change is made to the
      actual data object and this will be apparent through any other
      array identifiers which refer to it.  However, if an identifier
      for an array section is specified, then its bounds are altered
      without affecting other arrays.
   }
   \sstinvocation{
      CALL ARY\_SBND( NDIM, LBND, UBND, IARY, STATUS )
   }
   \sstarguments{
      \sstsubsection{
         NDIM = INTEGER (Given)
      }{
         New number of array dimensions.
      }
      \sstsubsection{
         LBND( NDIM ) = INTEGER (Given)
      }{
         New lower pixel-index bounds of the array.
      }
      \sstsubsection{
         UBND( NDIM ) = INTEGER (Given)
      }{
         New upper pixel-index bounds of the array,
      }
      \sstsubsection{
         IARY = INTEGER (Given)
      }{
         Array identifier.
      }
      \sstsubsection{
         STATUS = INTEGER (Given and Returned)
      }{
         The global status.
      }
   }
   \sstnotes{
      \sstitemlist{

         \sstitem
         The bounds of an array section cannot be altered while it is
         mapped for access through the identifier supplied to this
         routine.

         \sstitem
         The bounds of a base array cannot be altered while any part of
         it is mapped for access (i.e. even through another identifier).

         \sstitem
         The array{\tt '}s pixel values (if defined) will be retained for
         those pixels which lie within both the old and new bounds. Any
         pixels lying outside the new bounds will be lost (and cannot
         later be recovered by further changes to the array{\tt '}s bounds).
         Any new pixels introduced where the new bounds extend beyond the
         old ones will be assigned the {\tt "}bad{\tt "} value, and the subsequent
         value of the bad-pixel flag will reflect this.

         \sstitem
         If the bounds of a base array are to be altered and retention
         of the existing pixel values is not required, then a call to
         ARY\_RESET should be made before calling this routine. This will
         eliminate any processing which might otherwise be needed to
         retain the existing values. This step is not necessary with an
         array section, where no processing of pixel values takes place.
      }
   }
}
\sstroutine{
   ARY\_SECT
}{
   Create an array section
}{
   \sstdescription{
      The routine creates a new array section which refers to a
      selected region of an existing array (or array section). The
      section may be larger or smaller in extent than the original
      array.
   }
   \sstinvocation{
      CALL ARY\_SECT( IARY1, NDIM, LBND, UBND, IARY2, STATUS )
   }
   \sstarguments{
      \sstsubsection{
         IARY = INTEGER (Given)
      }{
         Identifier for the initial array.
      }
      \sstsubsection{
         NDIM = INTEGER (Given)
      }{
         Number of dimensions for new section.
      }
      \sstsubsection{
         LBND( NDIM ) = INTEGER (Given)
      }{
         Lower pixel-index bounds for the new section.
      }
      \sstsubsection{
         UBND( NDIM ) = INTEGER (Given)
      }{
         Upper pixel-index bounds for the new section.
      }
      \sstsubsection{
         IARY2 = INTEGER (Returned)
      }{
         Identifier for the new section.
      }
      \sstsubsection{
         STATUS = INTEGER (Given and Returned)
      }{
         The global status.
      }
   }
   \sstnotes{
      \sstitemlist{

         \sstitem
         The number of section dimensions need not match the number of
         dimensions in the initial array. Pixel-index bounds will be
         padded with 1{\tt '}s as necessary to identify the pixels to which the
         new section should refer.

         \sstitem
         Note that sections which extend beyond the pixel-index bounds
         of the initial array will be padded with bad pixels.

         \sstitem
         If this routine is called with STATUS set, then a value of
         ARY\_\_NOID will be returned for the IARY2 argument, although no
         further processing will occur. The same value will also be
         returned if the routine should fail for any reason. The ARY\_\_NOID
         constant is defined in the include file ARY\_PAR.
      }
   }
}
\sstroutine{
   ARY\_SHIFT
}{
   Apply pixel-index shifts to an array
}{
   \sstdescription{
      The routine applies pixel-index shifts to an array. An integer
      shift is applied to each dimension so that its pixel-index
      bounds, and the indices of each pixel, change by the amount of
      shift applied to the corresponding dimension. The array{\tt '}s pixels
      retain their values and none are lost.
   }
   \sstinvocation{
      CALL ARY\_SHIFT( NSHIFT, SHIFT, IARY, STATUS )
   }
   \sstarguments{
      \sstsubsection{
         NSHIFT = INTEGER (Given)
      }{
         Number of dimensions to which shifts are to be applied. This
         must not exceed the number of array dimensions. If fewer
         shifts are supplied than there are dimensions in the array,
         then the extra dimensions will not be shifted.
      }
      \sstsubsection{
         SHIFT( NSHIFT ) = INTEGER (Given)
      }{
         The pixel-index shifts to be applied to each dimension.
      }
      \sstsubsection{
         IARY = INTEGER (Given)
      }{
         Array identifier.
      }
      \sstsubsection{
         STATUS = INTEGER (Given and Returned)
      }{
         The global status.
      }
   }
   \sstnotes{
      \sstitemlist{

         \sstitem
         Pixel-index shifts applied to a base array will affect the
         appearance of that array as seen by all base-array identifiers
         associated with it. However, array sections derived from that
         base array will remain unchanged (as regards both pixel-indices
         and data content).

         \sstitem
         Pixel-index shifts cannot be applied to a base array while any
         part of it is mapped for access (i.e. even through another
         identifier).

         \sstitem
         Pixel-index shifts applied to an array section only affect
         that section itself, and have no effect on other array
         identifiers.

         \sstitem
         Pixel-index shifts cannot be applied to an array section while
         it is mapped for access through the identifier supplied to this
         routine.
      }
   }
}
\sstroutine{
   ARY\_SIZE
}{
   Determine the size of an array
}{
   \sstdescription{
      The routine returns the number of pixels in the array whose
      identifier is supplied (i.e. the product of its dimensions).
   }
   \sstinvocation{
      CALL ARY\_SIZE( IARY, NPIX, STATUS )
   }
   \sstarguments{
      \sstsubsection{
         IARY = INTEGER (Given)
      }{
         Array identifier.
      }
      \sstsubsection{
         NPIX = INTEGER (Returned)
      }{
         Number of pixels in the array.
      }
      \sstsubsection{
         STATUS = INTEGER (Given and Returned)
      }{
         The global status.
      }
   }
}
\sstroutine{
   ARY\_SSECT
}{
   Create a similar array section to an existing one
}{
   \sstdescription{
      The routine creates a new array section, using an existing
      section as a template. The new section bears the same
      relationship to its base array as the template section does to
      its own base array. Allowance is made for pixel-index shifts
      which may have been applied so that the pixel-indices of the new
      section match those of the template.  The number of dimensions of
      the input and template arrays may differ.
   }
   \sstinvocation{
      CALL ARY\_SSECT( IARY1, IARY2, IARY3, STATUS )
   }
   \sstarguments{
      \sstsubsection{
         IARY1 = INTEGER (Given)
      }{
         Identifier for the input array from which the section is to be
         derived. This may be a base array or an array section.
      }
      \sstsubsection{
         IARY2 = INTEGER (Given)
      }{
         Identifier for the template section (this may also be a base
         array or an array section).
      }
      \sstsubsection{
         IARY3 = INTEGER (Returned)
      }{
         Identifier for the new array section.
      }
      \sstsubsection{
         STATUS = INTEGER (Given and Returned)
      }{
         The global status.
      }
   }
   \sstnotes{
      \sstitemlist{

         \sstitem
         This routine normally generates an array section.  However, if
         both input arrays are base arrays with identical pixel-index
         bounds, then there is no need to create a section in order to
         access the required part of the first array. In this case a base
         array identifier will be returned instead.

         \sstitem
         The new section created by this routine will have the same
         number of dimensions as the array (or array section) from which
         it is derived. If the template (IARY2) array has fewer dimensions
         than this, then the bounds of any additional input dimensions are
         preserved unchanged in the new array. If the template (IARY2)
         array has more dimensions, then the excess ones are ignored.

         \sstitem
         This routine takes account of the regions of each base array
         to which the input array sections have access. It may therefore
         restrict the region accessible to the new section (and pad with
         {\tt "}bad{\tt "} pixels) so as not to grant access to regions of the base
         array which were not previously accessible through the input
         arrays.

         \sstitem
         If this routine is called with STATUS set, then a value of
         ARY\_\_NOID will be returned for the IARY3 argument, although no
         further processing will occur. The same value will also be
         returned if the routine should fail for any reason. The ARY\_\_NOID
         constant is defined in the include file ARY\_PAR.
      }
   }
}
\sstroutine{
   ARY\_STATE
}{
   Determine the state of an array (defined or undefined)
}{
   \sstdescription{
      The routine returns a logical value indicating whether an array{\tt '}s
      pixel values are currently defined.
   }
   \sstinvocation{
      CALL ARY\_STATE( IARY, STATE, STATUS )
   }
   \sstarguments{
      \sstsubsection{
         IARY = INTEGER (Given)
      }{
         Array identifier.
      }
      \sstsubsection{
         STATE = LOGICAL (Returned)
      }{
         Whether the array{\tt '}s pixel values are defined.
      }
      \sstsubsection{
         STATUS = INTEGER (Given and Returned)
      }{
         The global status.
      }
   }
}
\sstroutine{
   ARY\_STYPE
}{
   Set a new type for an array
}{
   \sstdescription{
      The routine sets a new full type for an array, causing its data
      storage type to be changed. If the array{\tt '}s pixel values are
      defined, then they will be converted from the old type to the new
      one.  If they are undefined, then no conversion will be
      necessary.  Subsequent enquiries will reflect the new type.
      Conversion may be performed between any types supported by the
      ARY\_ routines, including from a non-complex type to a complex
      type (and vice versa).
   }
   \sstinvocation{
      CALL ARY\_STYPE( FTYPE, IARY, STATUS )
   }
   \sstarguments{
      \sstsubsection{
         FTYPE = CHARACTER $*$ ( $*$ ) (Given)
      }{
         The new full type specification for the array (e.g.  {\tt '}\_REAL{\tt '}
         or {\tt '}COMPLEX\_INTEGER{\tt '}).
      }
      \sstsubsection{
         IARY = INTEGER (Given)
      }{
         Array identifier.
      }
      \sstsubsection{
         STATUS = INTEGER (Given and Returned)
      }{
         The global status.
      }
   }
   \sstnotes{
      \sstitemlist{

         \sstitem
         This routine may only be used to change the type of a base
         array. If it is called with an array which is not a base array,
         then it will return without action. No error will result.

         \sstitem
         An error will result if the array, or any part of it, is
         currently mapped for access (e.g. through another identifier).

         \sstitem
         If the type of an array is to be changed without its pixel
         values being retained, then a call to ARY\_RESET should be made
         beforehand. This will avoid the cost of converting all the
         values.
      }
   }
}
\sstroutine{
   ARY\_TEMP
}{
   Obtain a placeholder for a temporary array
}{
   \sstdescription{
      The routine returns an array placeholder which may be used to
      create a temporary array (i.e. one which will be deleted
      automatically once the last identifier associated with it is
      annulled). The placeholder returned by this routine may be passed
      to other routines (e.g. ARY\_NEW or ARY\_COPY) to produce a
      temporary array in the same way as a new permanent array would be
      created.
   }
   \sstinvocation{
      CALL ARY\_TEMP( PLACE, STATUS )
   }
   \sstarguments{
      \sstsubsection{
         PLACE = INTEGER (Returned)
      }{
         Placeholder for a temporary array.
      }
      \sstsubsection{
         STATUS = INTEGER (Given and Returned)
      }{
         The global status.
      }
   }
   \sstnotes{
      \sstitemlist{

         \sstitem
         Placeholders are intended only for local use within an
         application and only a limited number of them are available
         simultaneously. They are always annulled as soon as they are
         passed to another routine to create a new array, where they are
         effectively exchanged for an array identifier.

         \sstitem
         If this routine is called with STATUS set, then a value of
         ARY\_\_NOPL will be returned for the PLACE argument, although no
         further processing will occur. The same value will also be
         returned if the routine should fail for any reason. The ARY\_\_NOPL
         constant is defined in the include file ARY\_PAR.
      }
   }
}
\sstroutine{
   ARY\_TRACE
}{
   Set the internal ARY\_ system error-tracing flag
}{
   \sstdescription{
      The routine sets an internal flag in the ARY\_ system which
      enables or disables error-tracing messages. If this flag is set
      to .TRUE., then any error occurring within the ARY\_ system will
      be accompanied by error messages indicating which internal
      routines have exited prematurely as a result. If the flag is set
      to .FALSE., this internal diagnostic information will not appear
      and only standard error messages will be produced.
   }
   \sstinvocation{
      CALL ARY\_TRACE( NEWFLG, OLDFLG )
   }
   \sstarguments{
      \sstsubsection{
         NEWFLG = LOGICAL (Given)
      }{
         The new value to be set for the error-tracing flag.
      }
      \sstsubsection{
         OLDFLG = LOGICAL (Returned)
      }{
         The previous value of the flag.
      }
   }
   \sstnotes{
      \sstitemlist{

         \sstitem
         By default, the error tracing flag is set to .FALSE., so
         no internal diagnostic information will be produced.
      }
   }
}
\sstroutine{
   ARY\_TYPE
}{
   Obtain the numeric type of an array
}{
   \sstdescription{
      The routine returns the numeric type of an array as an upper-case
      character string (e.g. {\tt '}\_REAL{\tt '}).
   }
   \sstinvocation{
      CALL ARY\_TYPE( IARY, TYPE, STATUS )
   }
   \sstarguments{
      \sstsubsection{
         IARY = INTEGER (Given)
      }{
         Array identifier.
      }
      \sstsubsection{
         TYPE = CHARACTER $*$ ( $*$ ) (Returned)
      }{
         Numeric type of the array.
      }
      \sstsubsection{
         STATUS = INTEGER (Given and Returned)
      }{
         The global status.
      }
   }
   \sstnotes{
      \sstitemlist{

         \sstitem
         The symbolic constant ARY\_\_SZTYP may be used for declaring the
         length of a character variable which is to hold the numeric type
         of an array. This constant is defined in the include file
         ARY\_PAR.
      }
   }
}
\sstroutine{
   ARY\_UNMAP
}{
   Unmap an array
}{
   \sstdescription{
      The routine unmaps an array which has previously been mapped for
      READ, UPDATE or WRITE access.
   }
   \sstinvocation{
      CALL ARY\_UNMAP( IARY, STATUS )
   }
   \sstarguments{
      \sstsubsection{
         IARY = INTEGER (Given)
      }{
         Array identifier.
      }
      \sstsubsection{
         STATUS = INTEGER (Given and Returned)
      }{
         The global status.
      }
   }
   \sstnotes{
      \sstitemlist{

         \sstitem
         This routine attempts to execute even if STATUS is set on
         entry, although no further error report will be made if it
         subsequently fails under these circumstances.

         \sstitem
         An error will result if the array has not previously been
         mapped for access.
      }
   }
}
\sstroutine{
   ARY\_VALID
}{
   Determine whether an array identifier is valid
}{
   \sstdescription{
      Determine whether an array identifier is valid (i.e. associated
      with an array).
   }
   \sstinvocation{
      CALL ARY\_VALID( IARY, VALID, STATUS )
   }
   \sstarguments{
      \sstsubsection{
         IARY = INTEGER (Given)
      }{
         Identifier to be tested.
      }
      \sstsubsection{
         VALID = LOGICAL (Returned)
      }{
         Whether the identifier is valid.
      }
      \sstsubsection{
         STATUS = INTEGER (Given and Returned)
      }{
         The global status.
      }
   }
}
\sstroutine{
   ARY\_VERFY
}{
   Verify that an array{\tt '}s data structure is correctly constructed
}{
   \sstdescription{
      The routine checks that the data structure containing an array is
      correctly constructed and that the array{\tt '}s pixel values are
      defined. It also checks for the presence of any {\tt "}rogue{\tt "}
      components in the data structure. If an anomaly is found, then an
      error results. Otherwise, the routine returns without further
      action.
   }
   \sstinvocation{
      CALL ARY\_VERFY( IARY, STATUS )
   }
   \sstarguments{
      \sstsubsection{
         IARY = INTEGER (Given)
      }{
         Array identifier.
      }
      \sstsubsection{
         STATUS = INTEGER (Given and Returned)
      }{
         The global status.
      }
   }
}


\newpage
\section{\xlabel{changes_and_new_features_in_v11}%
Changes and new features in V1.1}
\label{changes_and_new_features_in_v11}

Only relatively minor changes have taken place since the previous
version (V1.0) of the ARY\_ system.  The most significant of these are
as follows:

\begin{enumerate}

\item An obscure bug resulting from an un-annulled HDS locator has been
fixed.  This could occasionally result in corrupted files if the
ARY\_STYPE routine was called repeatedly from successive invocations of
an application in an ADAM monolith.

\item Two new routines have been introduced, primarily to provide facilities 
required by the NDF\_ system:

\begin{itemize}

\item \noteroutine{ARY\_NDIM( IARY, NDIM, STATUS )}
                  {Enquire the dimensionality of an array}

\item \noteroutine{ARY\_OFFS( IARY1, IARY2, MXOFFS, OFFS, STATUS )}
                  {Obtain the pixel offset between two arrays}

\end{itemize}

\item A stand-alone (non-ADAM) version of the ARY\_ library has been
added and new linker options files have been provided to allow linking
with either version.

\item The encoding of ARY\_ system identifiers has been changed to
improve the chance of detecting erroneous identifier values.

\item Messages about array data structures now contain the full HDS object 
name, including the full container file name.

\item A few minor documentation errors have been corrected.

\end{enumerate}

No changes to existing applications should be required, neither should any 
re-compilation or re-linking be necessary.

% ? End of main text
\end{document}

