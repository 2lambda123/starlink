
\documentstyle{article} 
\pagestyle{myheadings}

%------------------------------------------------------------------------------
\newcommand{\irasdoccategory}  {IRAS90 Document}
\newcommand{\irasdocinitials}  {ID}
\newcommand{\irasdocnumber}    {6.5}
\newcommand{\irasdocauthors}   {D.S. Berry}
\newcommand{\irasdocdate}      {15th February 2008}
\newcommand{\irasdoctitle}     {IRQ: Handling of QUALITY within IRAS90}
%------------------------------------------------------------------------------

\newcommand{\irasdocname}{\irasdocinitials /\irasdocnumber}
\renewcommand{\_}{{\tt\char'137}}     % re-centres the underscore
\markright{\irasdocname}
\setlength{\textwidth}{160mm}
\setlength{\textheight}{240mm}
\setlength{\topmargin}{-5mm}
\setlength{\oddsidemargin}{0mm}
\setlength{\evensidemargin}{0mm}
\setlength{\parindent}{0mm}
\setlength{\parskip}{\medskipamount}
\setlength{\unitlength}{1mm}

%------------------------------------------------------------------------------
% Add any \newcommand or \newenvironment commands here
%------------------------------------------------------------------------------

\begin{document}
\thispagestyle{empty}
SCIENCE \& ENGINEERING RESEARCH COUNCIL \hfill \irasdocname\\
RUTHERFORD APPLETON LABORATORY\\
{\large\bf IRAS90\\}
{\large\bf \irasdoccategory\ \irasdocnumber}
\begin{flushright}
\irasdocauthors\\
\irasdocdate
\end{flushright}
\vspace{-4mm}
\rule{\textwidth}{0.5mm}
\vspace{5mm}
\begin{center}
{\Large\bf \irasdoctitle}
\end{center}
\vspace{5mm}
\setlength{\parskip}{0mm}
\tableofcontents
\setlength{\parskip}{\medskipamount}
\markright{\irasdocname}

\section {Introduction to QUALITY}
A QUALITY structure is one of the standard components of an NDF structure, and
is described fully in SUN/33. Briefly, if an NDF has a QUALITY component which
is in a {\em defined} state, then each pixel within the NDF DATA component has a
corresponding value in the QUALITY component. Currently, each QUALITY value
consists of an unsigned byte (i.e. 8 bits). In Fortran the least significant bit
is usually called bit 0 and the most significant bit is usually called bit 7.
Within the IRQ package the least significant bit is called bit 1 and the most
significant bit is called bit 8. Each bit within the QUALITY value can be used
to indicate if the corresponding pixel in the DATA component holds some specific
{\em quality}. For instance, bit 3 of the QUALITY component may be used to
indicate if any DATA pixels are saturated. A particular pixel in the QUALITY
component would have bit 3 set (i.e. equal to 1) if the corresponding DATA pixel
is saturated, or cleared (i.e. equal to 0) if the corresponding DATA pixel is
not saturated. 

Another option for flagging saturated data is to replace saturated DATA pixel
values with a ``bad'' (or ``magic'') value. This has the disadvantage that the
data is permanently destroyed by being flagged, and also there is no distinction
between data values which are set bad because of the fact they were saturated,
and pixels set bad for any other reason. 

SUN/33 doesn't specify how the facilities of the QUALITY component are to be 
used, and many possibilities exist. Obviously some coordination between 
applications is needed so that different applications interpret the QUALITY 
values in a consistent manner (i.e. using the above example, later applications
must know that bit 3 is a saturation flag). The IRQ package provides a set of
routines for doing this.

\section {Introduction to the facilities provided by the IRQ package}
\label {SEC:IRQ}
IRQ provides a system for handling Boolean qualities (i.e. qualities which are
either held or not held by each DATA pixel) on either VMS or UNIX systems. From
the point of view of an application, each defined quality is identified by a
{\em quality name} rather than by a bit number. Information about these quality
names is stored in an NDF extension, so that later applications can determine
which pixels within the NDF hold a given quality (or combination of qualities). 

Within IRQ, each defined quality name is usually associated with a bit in the
QUALITY array, and this bit is set if the corresponding DATA pixels are assigned
the specified quality. A typical application need know nothing about which
QUALITY bit is associated with which quality name. For instance, if {\em every}
pixel in an NDF holds a certain quality (or alternatively, if {\em no} pixels
hold the quality), then it is not necessary to reserve a bit in the QUALITY
array to represent the quality. Instead, a single Boolean scalar value can be
stored with the quality name in the NDF extension. This scalar is set to .TRUE.
if {\em all} pixels hold the quality, and .FALSE. if {\em no} pixels hold the
quality. In this way, the number of defined quality names can sometimes exceed
the number of bits in the QUALITY component. The handling of such situations is
done within IRQ and is completely invisible to the calling application. 

IRQ provides facilities to 
\begin{itemize}
\item Add quality name definitions (and associated descriptive comments) to an
NDF. There is a limit to the number of quality names which may be defined within
an NDF. The exact number depends on how many quality names require a QUALITY
bit, but it will always be at least 8. 
\item Remove quality name definitions from an NDF.
\item List all quality names defined within an NDF.
\item Assign a given quality to selected pixels.
\item Remove a given quality from selected pixels.
\item Set ``bad'' those pixels of a supplied array which hold a certain 
combination of qualities.
\end{itemize}

\section {A Set of Four Typical IRQ Applications}
The IRAS90 package contains a set of four typical IRQ applications, which give
an indication of the benefits which IRQ can provide for the user. The ATASK
documentation for these four applications is included in appendix
\ref{APP:APPS}, and they are briefly described in this section. 
\subsection{SETQUAL}
SETQUAL creates quality name definitions and stores them within a specified
extension of an NDF. It also assigns a specified quality to a sub-set of the
pixels with the NDF. In its simplest mode, the user provides an NDF, and a
string to use as a quality name (eg ``SATURATED''). If this quality name is not
already defined within the NDF, then it is added to the list of defined quality
names, together with a user-supplied comment describing the quality. The user
also provides another NDF to be used as a ``mask''. The ``bad'' pixels within
the mask NDF define the pixels which are to be assigned the given quality. For
instance, a user may have an image containing saturated pixels. If he wants
these pixels flagged without being permanently destroyed, he could proceed as
follows: 
\begin{enumerate}
\item Produce a mask NDF from the original NDF by setting all DATA pixels 
above the saturation value to the ``bad'' value. 
\item Run SETQUAL on the original NDF, giving some quality name such as
``SATURATED'', and specifying the mask created in the previous step. This leaves
the DATA component of the original NDF unchanged, but assigns the quality
``SATURATED'' to all the pixels which correspond to ``bad'' pixels in the mask. 
\end{enumerate}
Alternatively, instead of using a mask, the pixels to which the quality is
assigned may be specified by an explicit list of pixel indices stored in a
text file.

\subsection{REMQUAL}
The REMQUAL application removes quality name definitions from an NDF. This may be
necessary if there is no room for any more quality name definitions within an
NDF. In this case the user may choose to remove some unimportant quality name
definitions to make room for new, more important quality names. REMQUAL can also
remove {\em all} quality names information from an NDF. This can be useful if
for any reason the quality name information becomes corrupted. 

\subsection{SHOWQUAL}
The SHOWQUAL application displays all currently defined quality names within an
NDF together with the associated descriptive comments. Optionally, the number of
pixels which hold each quality can be displayed. 

\subsection{QUALTOBAD}
The QUALTOBAD application sets selected pixels within an NDF to the ``bad''
value on the basis of the pixel's quality. In effect, QUALTOBAD performs the
reverse operation of SETQUAL. For instance, using the example of saturated data
described above, the user may remove some varying background surface from his
original data, and then want to set pixels which were saturated in the original
data to the ``bad'' value . In this case he would run QUALTOBAD specifying the
background removed NDF as input, and specifying a {\em quality expression} of
``SATURATED''. This would cause all pixels which hold the quality ``SATURATED''
to be set bad in the output NDF. 

The ``quality expression'' can be more complex than a single quality name. In 
fact, a quality expression can consist of several quality names combined 
together using the usual operators of Boolean algebra. For instance, if the
quality expression ``SATURATED.AND..NOT.(SOURCE\_A .OR. SOURCE\_B)'' was given to 
QUALTOBAD, then pixels would be set ``bad'' only if they had the quality 
SATURATED, but did not have either of the qualities SOURCE\_A or SOURCE\_B.

\section{Quality Names}
A quality name must contain 15 or less characters. Any leading blanks are
removed from supplied quality names, and they are converted to upper case before
being stored in the NDF. Quality names may contain embedded blanks, but may not 
contain full-stop (".") characters. Certain names are reserved and may not be 
used. These are ``ANY'', ``IRQ\_BAD\_SLOT'' and ``IRQ\_FREE\_SLOT''.

\section{Quality Expressions}
A ``Quality Expression'' consists of a set of quality names combined together
using Boolean operators into a legal Boolean expression. See section \ref
{SEC:QEXP} for a description of the use of quality expressions within IRQ. In
the following, the symbols ``A'' and ``B'' are used to represent two qualities.
These can be considered as Boolean values; ``true'' if a pixel holds the
quality, and ``false'' otherwise. The supported Boolean operators are: 
\begin {description}
\item [.AND.] - The expression (A.AND.B) is true if and only if both A and B are
true. 
\item [.OR.] - The expression (A.OR.B) is true if and only if either A or B is
true. 
\item [.XOR.] - The expression (A.XOR.B) is true if and only if either A is
true and B is false, or A is false and B is true.
\item [.EQV.] - The expression (A.EQV.B) is true if and only if either A is
true and B is true, or A is false and B is false.
\item [.NOT.] - The expression (.NOT.A) is true if and only if A is false.
\end{description}
In addition to the above operators, the Boolean constants .FALSE. and .TRUE. can
be included within a quality expression. Expressions may contain several levels
of nested parentheses. 

The precedence of these operators decreases in the following order; .NOT.,
.AND., .OR., .XOR., .EQV. (the final two have equal precedence). In an
expression such as (A.XOR.B.EQV.C.XOR.D) in which all operators have equal
precedence, the evaluation proceeds from left to right., i.e. the expression is
evaluated as  (((A.XOR.B).EQV.C).XOR.D). If there is any doubt about the order 
in which an expression will be evaluated, parentheses should be used to ensure 
the required order of evaluation.

Some attempts are made to simplify a quality expression to reduce the run time
needed to evaluate the expression for every pixel. 

Quality expressions can be up to 254 characters long, and must not contain more 
than 40 symbols (Boolean operators, constants, or quality names).


\section {Using IRQ routines}
This section gives a brief outline of the IRQ routines which are available to 
perform some common tasks. The specific details required to use these routines 
are not included here but can be found in the subroutine specifications 
contained in appendix \ref {SEC:FULLSPEC}.

\subsection {Constants and Error Values}
\label {SEC:CON}
The IRQ package has associated with it various symbolic constants. These values
consist of a name of up to 5 characters prefixed by ``IRQ\_\_''  (note the {\em
double} underscore), and can be made available to an application by including
the following line at the start of the routines which uses them: 

\begin{verbatim}
      INCLUDE 'IRQ_PAR'
\end{verbatim}

This assumes that the IRAS90 package has been installed as part
of the UNIX Starlink Software Collection. If this is not the case, then the
alternative instructions given in appendix /ref {APP:NOSTAR} should be followed.

The values thus defined are described in the following sections, and also in the
subroutine specifications. Another set of symbolic constants is made available
by the statements 

\begin{verbatim}
      INCLUDE 'IRQ_ERR'
\end{verbatim}

These values have the same format of those contained in IRQ\_PAR, put define
various error conditions which can be generated within the IRQ package.
Applications can compare the $STATUS$ argument with these values to check for
specific error conditions. These values are described in appendix \ref
{APP:ERRORS}. 

\subsection{Initialising an NDF for use with IRQ}
Certain HDS structures must be created within an NDF before the NDF can be used
by IRQ. These structures hold information describing the currently defined
quality names within the NDF. Routine IRQ\_FIND can be used to see if such
structures exist within an NDF (see below). If no such structure yet exists
within an NDF, routine IRQ\_NEW must be called to create the structure. This
structure is held in a specified NDF extension. IRQ\_NEW returns an array of HDS
locators which must be passed to subsequent IRQ routines. One of these locators
points to a cloned copy of the NDF identifier. All access to the NDF by
subsequent IRQ routines is achieved through this cloned identifier. Once access
to the quality names information is no longer required, the resources used by
these HDS locators (including the cloned NDF identifier) should be annulled by
calling IRQ\_RLSE. Note, routines IRQ\_ANNUL and IRQ\_CLOSE play no part in
releasing resources used for accessing quality name information. These two
routines are only used for releasing resources used to store {\em compiled
quality expressions} (see section /ref {SEC:QEXP}). IRQ\_RLSE is the only
routine which needs to be called to release resources used for accessing {\em
quality names}. 

\subsection{Using previously initialised NDFs within IRQ}
If the NDF has been initialised for use by IRQ, then the structure holding the
quality names information must be found before other IRQ routines can be called.
Routine IRQ\_FIND does this. This routine looks through all the NDF extensions
until a suitable structure is found. If no such structure is found the status
value IRQ\_\_NOQNI is returned. If more than one extension contains such a
structure, then the status value IRQ\_\_MULT is returned. If the routine runs
successfully, then an array of HDS locators is returned similar to the array
returned by IRQ\_NEW. Once access to the quality names information is no longer
required, the resources used by these HDS locators (including the cloned
NDF identifier) should be annulled by calling IRQ\_RLSE. 

\subsection{Accessing the quality names information stored in an NDF}
To access the quality names information stored within an NDF, a set of HDS 
locators to the information must first be obtained by calling either IRQ\_NEW or 
IRQ\_FIND (see above). Once this has been done the information can be read, 
modified, or added to. IRQ\_ADDQN adds a new quality name definition to an NDF 
(so long as there is room for it within the structures and the QUALITY 
component). IRQ\_CHKQN checks to see if a given name is defined. IRQ\_GETQN 
searches for a specified quality name, and 
returns various items of information about it (such as which bit of the QUALITY 
component it is assigned to). IRQ\_NUMQN returns the number of defined quality 
names. IRQ\_NXTQN returns the next defined quality name. Repeated calls to 
IRQ\_NXTQN can be made to get a list of all the quality names defined within an 
NDF. IRQ\_REMQN removes the definition of a quality name from an NDF, so
long as the quality name has not been flagged as ``read-only'' (see
IRQ\_RWQN).

\subsection{Assigning and removing qualities to and from NDF pixels}
Once quality names have been defined, they can be assigned to selected pixels
within the NDF. There are two ways of specifying which pixels are ``selected''.
Routine IRQ\_SETQM requires a mask image to be provided in which the ``bad''
pixels define the pixels in the NDF DATA component to which the quality is to be
assigned. Routine IRQ\_SETQL requires a list of pixel indices to be provided
which defines the pixels in the NDF DATA component to which the quality is to be
assigned. The quality of the NDF pixels which are {\em not} selected is left
unchanged. Thus, if an unselected pixel already has the specified quality, these
routines will not {\em remove} the quality from those pixels. 

The routines IRQ\_RESQM and IRQ\_RESQL, are complimentary to IRQ\_SETQM and 
IRQ\_SETQL. Instead of ensuring that the selected pixels hold the specified
quality, these routines ensure that the selected pixels {\em do not} hold the 
specified quality. Again, the quality of unselected pixels is left unchanged.

Routine IRQ\_SETQ ensures that {\em all} pixels in an NDF have a specified
quality, and routine IRQ\_RESQ ensures that {\em no} pixels in an NDF have a
specified quality. 

The routine IRQ\_CNTQ will count the number of bits set in each bit plane in the
QUALITY component. 

\subsection{Finding NDF pixels which satisfy a quality expression}
\label {SEC:QEXP}
If a subset of pixels are to be operated on by an application, the application
will usually obtain a quality expression from the user which defines the pixels
to be operated on. That is to say, pixels which have qualities which do not
satisfy the quality expression would not usually be used by the application. The
quality expression is usually obtained from the user using the ADAM parameter
system. It is then passed to IRQ\_COMP to be compiled. If any problems are
detected with the expression (such as syntax errors, or undefined quality names)
IRQ\_COMP will report an error, and it is then up to the applications to decide 
what to do. Typically, it will flush the error, and reprompt the user for a new
quality expression. The routine IRQ\_SYNTX will check a quality expression for 
syntax errors without actually compiling it or checking that the referenced 
quality names are defined.

If IRQ\_COMP successfully compiles the quality expression, it returns an
identifier for the compiled quality expression. This identifier is actually a
pointer to a temporary HDS structure holding information about the quality
expression. The identifier is passed to IRQ\_SBAD which locates all NDF pixels
which have qualities which satisfy the quality expression. The corresponding
pixels in an array supplied to IRQ\_SBAD are set ``bad''. The other pixels are
left unchanged. Alternatively, pixels with qualities which {\em do not} satisfy
the quality expression can be set bad in the supplied array. Note, the array
supplied to IRQ\_SBAD must be the same shape and size as the NDF supplied to
IRQ\_FIND or IRQ\_NEW. If an NDF section was used, then the array supplied to
IRQ\_SBAD must be the same shape and size as the NDF section, {\em not} the base
NDF. 

Up to ten compiled quality expressions can be active at once. The identifiers
for each compiled quality expression should be annuled when it is no longer
needed by calling IRQ\_ANNUL. IRQ\_CLOSE should be called when all compiled 
quality expressions have been finished with. 

\section {Compiling and Linking with IRQ}
\label{SEC:LINK}
This section describes how to compile and link applications which use IRQ
subroutines, on both VMS and UNIX systems. It is assumed that the IRAS90 package
is installed as part of the Starlink Software Collection.

\subsection{VMS}
Each terminal session which is to include the compilation or linking of 
applications which use the IRQ package should start by issuing the commands:

\begin{verbatim}
$ ADAMSTART
$ ADAMDEV
$ IRAS90
$ IRAS90_DEV
\end{verbatim}

These commands set up logical names related to all the IRAS90
subsystems, including IRQ. 

To link a VMS ADAM application with the IRQ package, the linker options file
IRAS90\_LINK\_ADAM should be used. For example, to compile and link an ADAM
application called PROG with the IRQ library, the following commands should be
used: 

\begin{verbatim}
$ ADAMSTART
$ ADAMDEV
$ IRAS90
$ IRAS90_DEV
$ FORT PROG
$ ALINK PROG, IRAS90_LINK_ADAM/OPT 
\end{verbatim}

Stand-alone (i.e. non-ADAM) applications can be linked with the ``standalone''
version of IRQ. This version excludes the routines described in appendix
\ref{APP:ADAM}. To do this the link options file IRAS90\_LINK should be used
instead of IRAS90\_LINK\_ADAM. Thus to compile and link a stand-alone
application with the IRQ package, the following commands should be given: 

\begin{verbatim}
$ IRAS90
$ IRAS90_DEV
$ FORT PROG
$ LINK PROG, IRAS90_LINK/OPT 
\end{verbatim}

\subsection{UNIX}

Each terminal session which is to include the compilation or linking of 
applications which use any IRAS90 sub-package should do the following:

\begin{enumerate}
\item Execute the {\bf iras90} command. This sets up an alias for the 
{\bf iras90\_dev} command.

\item Execute the {\bf iras90\_dev} command.  This creates soft links
within the current directory to
all the iras90 include files, defines an environment variable IRAS90\_SRC
pointing to the IRAS90 source directory, adds the {\bf bin}
sub-directory of IRAS90\_SRC on to the end of the current value for the
environment variable PATH, and adds the {\bf lib} sub-directory of 
IRAS90\_SRC on to the end of the current value for the environment 
variable LD\_LIBRARY\_PATH. The command {\bf iras90\_dev star} will do
the same but will also create soft links to all starlink include files
(i.e. all files in the directory /star/include).

\item If a UNIX application accesses any include files (such as
IRQ\_PAR) then they should be specified (within the source file) in upper case 
without any directory
path. The I90\_PAR file (for instance) can be included using the statement

\verb+      INCLUDE 'I90_PAR'+\\

\item Soft links can be deleted (using the {\bf rm} command) when no longer 
needed.
The command {\bf iras90\_dev remove} will remove all IRAS90 soft links from
the current directory. The command {\bf iras90\_dev remove star} will
remove all IRAS90 and Starlink soft links from the current directory.
Note, soft links are only accessable in the directory from which the {bf ln} 
or {\bf iras90\_dev} command was issued. 
\item For ADAM applications the following {\bf alink} command should be used:

\verb+% alink prog.f -L$IRAS90_SRC/lib `irq_link_adam`+\\

where {\bf prog.f} is the fortran source file for the ATASK.
Note the use of opening apostrophies (`) instead of the more usual closing
apostrophy (') in the above {\bf alink} command.

\item For a stand-alone program the following {\bf f77} command should be used:

\verb+% f77 prog.f -o prog -L$IRAS90_SRC/lib `irq_link`+\\
\end{enumerate}


\appendix
\section {Routine Descriptions}
% Command for displaying routines in routine lists:
% =================================================

\newcommand{\noteroutine}[2]{{\small \bf #1} \\
                              \hspace*{3em} {\em #2} \\[1.5ex]}

\noteroutine{IRQ\_ADDQN(LOCS,QNAME,DEFLT,COMMNT,STATUS)}
   {Define a new quality name.}
\noteroutine{IRQ\_ANNUL(IDQ,STATUS)}
   {Annul an identifier for a compiled quality expression.}
\noteroutine{IRQ\_CHKQN(LOCS,QNAME,THERE,STATUS)}
   {Check that a specified quality name is defined.}
\noteroutine{IRQ\_CLOSE(STATUS)}
   {Close down the compiled quality expression identifier system.}
\noteroutine{IRQ\_CNTQ(LOCS,SIZE,SET,STATUS)}
   {Count the number of pixels with each bit set in the QUALITY component.}
\noteroutine{IRQ\_COMP(LOCS,SIZE,INFO,QEXP,UNDEF,NUNDEF,ERRPNT,IDQ,STATUS)}
   {Compile a quality expression.}
\noteroutine{IRQ\_FIND(INDF,LOCS,XNAME,STATUS)}
   {Find a structure containing quality names information.}
\noteroutine{IRQ\_GETQN(LOCS,QNAME,FIXED,VALUE,BIT,COMMNT,STATUS)}
   {Get information about a specified quality name.}
\noteroutine{IRQ\_NEW(INDF,XNAME,LOCS,STATUS)}
   {Create a structure to hold quality names information.}
\noteroutine{IRQ\_NUMQN(LOCS,NAMES,STATUS)}
   {Return the number of defined quality names.}
\noteroutine{IRQ\_NXTQN(LOCS,CONTXT,QNAME,FIXED,VALUE,BIT,COMMNT,DONE,STATUS)}
   {Return information about the next defined quality name.}
\noteroutine{IRQ\_REMQN(LOCS,QNAME,STATUS)}
   {Remove the definition of a quality name.}
\noteroutine{IRQ\_RESQ(LOCS,QNAME,STATUS)}
   {Ensure no pixels hold a specified quality.}
\noteroutine{IRQ\_RESQL(LOCS,LISTED,QNAME,NDIM,NCOORD,LIST,SET,STATUS)}
   {Ensure pixels selected by a list do not hold a specified quality.}
\noteroutine{IRQ\_RESQM(LOCS,BAD,QNAME,SIZE,MASK,SET,STATUS)}
   {Ensure pixels selected by a mask do not hold a specified quality.}
\noteroutine{IRQ\_RLSE(LOCS,STATUS)}
   {Release the resource used to locate quality name information.}
\noteroutine{IRQ\_RWQN( LOCS, QNAME, SET, NEWVAL, OLDVAL, STATUS )}
   {Get and/or set the read-only flag for a quality name.}
\noteroutine{IRQ\_SBAD(IDQ,HELD,SIZE,VEC,ALLBAD,NOBAD,STATUS)}
   {Set pixels bad which satisfy a given quality expression.}
\noteroutine{IRQ\_SETQ(LOCS,QNAME,STATUS)}
   {Ensure all pixels hold a specified quality.}
\noteroutine{IRQ\_SETQL(LOCS,LISTED,QNAME,NDIM,NCOORD,LIST,SET,STATUS)}
   {Ensure pixels selected by a list hold a specified quality.}
\noteroutine{IRQ\_SETQM(LOCS,BAD,QNAME,SIZE,MASK,SET,STATUS)}
   {Ensure pixels selected by a mask hold a specified quality.}
\noteroutine{IRQ\_SYNTX(QEXP,ERRPNT,STATUS)}
   {Check a quality expression for syntax errors.}

\section {Classified List}

\subsection{Gaining Access to Quality Name Information Within an NDF}
\noteroutine{IRQ\_FIND(INDF,LOCS,XNAME,STATUS)}
   {Find a structure containing quality names information.}
\noteroutine{IRQ\_NEW(INDF,XNAME,LOCS,STATUS)}
   {Create a structure to hold quality names information.}
\noteroutine{IRQ\_RLSE(LOCS,STATUS)}
   {Release the resource used to locate quality name information.}

\subsection{Storing, Retrieving and Deleting Quality Names}
\noteroutine{IRQ\_ADDQN(LOCS,QNAME,DEFLT,COMMNT,STATUS)}
   {Define a new quality name.}
\noteroutine{IRQ\_CHKQN(LOCS,QNAME,THERE,STATUS)}
   {Check that a specified quality name is defined.}
\noteroutine{IRQ\_GETQN(LOCS,QNAME,FIXED,VALUE,BIT,COMMNT,STATUS)}
   {Get information about a specified quality name.}
\noteroutine{IRQ\_NUMQN(LOCS,NAMES,STATUS)}
   {Return the number of defined quality names.}
\noteroutine{IRQ\_NXTQN(LOCS,CONTXT,QNAME,FIXED,VALUE,BIT,COMMNT,DONE,STATUS)}
   {Return information about the next defined quality name.}
\noteroutine{IRQ\_REMQN(LOCS,QNAME,STATUS)}
   {Remove the definition of a quality name.}
\noteroutine{IRQ\_RWQN( LOCS, QNAME, SET, NEWVAL, OLDVAL, STATUS )}
   {Get and/or set the read-only flag for a quality name.}

\subsection{Handling Quality Expressions}
\noteroutine{IRQ\_ANNUL(IDQ,STATUS)}
   {Annul an identifier for a compiled quality expression.}
\noteroutine{IRQ\_CLOSE(STATUS)}
   {Close down the compiled quality expression identifier system.}
\noteroutine{IRQ\_COMP(LOCS,SIZE,INFO,QEXP,UNDEF,NUNDEF,ERRPNT,IDQ,STATUS)}
   {Compile a quality expression.}
\noteroutine{IRQ\_SBAD(IDQ,HELD,SIZE,VEC,ALLBAD,NOBAD,STATUS)}
   {Set pixels bad which satisfy a given quality expression.}
\noteroutine{IRQ\_SYNTX(QEXP,ERRPNT,STATUS)}
   {Check a quality expression for syntax errors.}

\subsection{Assigning Qualities to Selected Pixels}
\noteroutine{IRQ\_RESQ(LOCS,QNAME,STATUS)}
   {Ensure no pixels hold a specified quality.}
\noteroutine{IRQ\_RESQL(LOCS,LISTED,QNAME,NDIM,NCOORD,LIST,SET,STATUS)}
   {Ensure pixels selected by a list do not hold a specified quality.}
\noteroutine{IRQ\_RESQM(LOCS,BAD,QNAME,SIZE,MASK,SET,STATUS)}
   {Ensure pixels selected by a mask do not hold a specified quality.}
\noteroutine{IRQ\_SETQ(LOCS,QNAME,STATUS)}
   {Ensure all pixels hold a specified quality.}
\noteroutine{IRQ\_SETQL(LOCS,LISTED,QNAME,NDIM,NCOORD,LIST,SET,STATUS)}
   {Ensure pixels selected by a list hold a specified quality.}
\noteroutine{IRQ\_SETQM(LOCS,BAD,QNAME,SIZE,MASK,SET,STATUS)}
   {Ensure pixels selected by a mask hold a specified quality.}

\subsection{Enquiring Pixel Quality}
\noteroutine{IRQ\_CNTQ(LOCS,SIZE,SET,STATUS)}
   {Count the number of pixels with each bit set in the QUALITY component.}

\section {Full Routine Specifications}
\label {SEC:FULLSPEC}
\newlength{\sstbannerlength}
\newlength{\sstcaptionlength}
\font\ssttt=cmtt10 scaled 1095
\newcommand{\sstroutine}[3]{
   \goodbreak
   \rule{\textwidth}{0.5mm}
   \vspace{-7ex}
   \newline
   \settowidth{\sstbannerlength}{{\Large {\bf #1}}}
   \setlength{\sstcaptionlength}{\textwidth}
   \addtolength{\sstbannerlength}{0.5em} 
   \addtolength{\sstcaptionlength}{-2.0\sstbannerlength}
   \addtolength{\sstcaptionlength}{-4.45pt}
   \parbox[t]{\sstbannerlength}{\flushleft{\Large {\bf #1}}}
   \parbox[t]{\sstcaptionlength}{\center{\Large #2}}
   \parbox[t]{\sstbannerlength}{\flushright{\Large {\bf #1}}}
   \begin{description}
      #3
   \end{description}
}

\newcommand{\sstdescription}[1]{\item[Description:] #1}

\newcommand{\sstusage}[1]{\item[Usage:] \mbox{} \\[1.3ex] {\ssttt #1}}

\newcommand{\sstinvocation}[1]{\item[Invocation:]\hspace{0.4em}{\tt #1}}

\newcommand{\sstarguments}[1]{
   \item[Arguments:] \mbox{} \\
   \vspace{-3.5ex}
   \begin{description}
      #1
   \end{description}
}

\newcommand{\sstreturnedvalue}[1]{
   \item[Returned Value:] \mbox{} \\
   \vspace{-3.5ex}
   \begin{description}
      #1
   \end{description}
}

\newcommand{\sstparameters}[1]{
   \item[Parameters:] \mbox{} \\
   \vspace{-3.5ex}
   \begin{description}
      #1
   \end{description}
}

\newcommand{\sstexamples}[1]{
   \item[Examples:] \mbox{} \\
   \vspace{-3.5ex}
   \begin{description}
      #1
   \end{description}
}

\newcommand{\sstsubsection}[1]{\item[{#1}] \mbox{} \\}

\newcommand{\sstexamplesubsection}[1]{\item[{\ssttt #1}] \mbox{} \\}

\newcommand{\sstnotes}[1]{\item[Notes:] \mbox{} \\[1.3ex] #1}

\newcommand{\sstdiytopic}[2]{\item[{\hspace{-0.35em}#1\hspace{-0.35em}:}] \mbox{} \\[1.3ex] #2}

\newcommand{\sstimplementationstatus}[1]{
   \item[{Implementation Status:}] \mbox{} \\[1.3ex] #1}

\newcommand{\sstbugs}[1]{\item[Bugs:] #1}

\newcommand{\sstitemlist}[1]{
  \mbox{} \\
  \vspace{-3.5ex}
  \begin{itemize}
     #1
  \end{itemize}
}

\newcommand{\sstitem}{\item}
\sstroutine{
   IRQ\_ADDQN
}{
   Define a new quality name
}{
   \sstdescription{
      This routine adds the quality name specified by QNAME to the NDF
      specified by LOCS. LOCS must previously have been assigned values
      by one of the routines IRQ\_FIND or IRQ\_NEW. If the quality name
      is already defined, an error is reported. Note, this routine does
      not reserve a bit in the QUALITY component for the new quality
      name, it mearly established a default value for the quality which
      will be used for all pixels in the NDF if no subsequent call to
      IRQ\_SETQC is made. Note, the string ANY cannot be used as a
      quality name. Also, quality names may not contain any full stops.

      An error is reported if only READ access is available to the NDF.
   }
   \sstinvocation{
      CALL IRQ\_ADDQN( LOCS, QNAME, DEFLT, COMMNT, STATUS )
   }
   \sstarguments{
      \sstsubsection{
         LOCS(5) = CHARACTER (Given)
      }{
         An array of 5 HDS locators. These locators identify the NDF
         and the associated quality name information.  They should have
         been obtained using routine IRQ\_FIND or routine IRQ\_NEW.
      }
      \sstsubsection{
         QNAME = CHARACTER (Given)
      }{
         The new quality name to store. The maximum length of this
         string is given by symbolic constant IRQ\_\_SZQNM which currently
         has the value 15. Leading spaces are ignored, and the stored
         name is converted to upper case.
      }
      \sstsubsection{
         DEFLT = LOGICAL (Given)
      }{
         If true, then by default all pixels are assumed to hold the
         quality specified by QNAME. If false, then it is assumed that
         no pixels hold the quality.
      }
      \sstsubsection{
         COMMNT = CHARACTER (Given)
      }{
         A descriptive comment to store with the quality name. The
         maximum length of this string is given by symbolic constant
         IRQ\_\_SZCOM, which currently has the value 50. Any characters
         beyond this length are ignored.
      }
      \sstsubsection{
         STATUS = INTEGER (Given and Returned)
      }{
         The global status.
      }
   }
}
\sstroutine{
   IRQ\_ANNUL
}{
   Release an IRQ identifier
}{
   \sstdescription{
      All internal resources used by the specified compiled quality
      expression identifier (created by IRQ\_COMP) are released.

      This routine attempts to execute even if STATUS is bad on entry,
      although no further error report will be made if it subsequently
      fails under these circumstances.
   }
   \sstinvocation{
      CALL IRQ\_ANNUL( IDQ, STATUS )
   }
   \sstarguments{
      \sstsubsection{
         IDQ = INTEGER (Given)
      }{
         An IRQ identifier for a compiled quality expression.
      }
      \sstsubsection{
         STATUS = INTEGER (Given and Returned)
      }{
         The global status.
      }
   }
}
\sstroutine{
   IRQ\_CHKQN
}{
   Check a specified quality name to see if it is defined
}{
   \sstdescription{
      This routine searches for a specified quality name in the quality
      name specified by LOCS. If it is found, THERE is returned true.
      Otherwise THERE is returned false.
   }
   \sstinvocation{
      CALL IRQ\_CHKQN( LOCS, QNAME, THERE, STATUS )
   }
   \sstarguments{
      \sstsubsection{
         LOCS(5) = CHARACTER (Given)
      }{
         An array of 5 HDS locators. These locators identify the NDF
         and the associated quality name information.  They should have
         been obtained using routine IRQ\_FIND or routine IRQ\_NEW.
      }
      \sstsubsection{
         QNAME = CHARACTER (Given)
      }{
         The quality name to search for. Leading blanks are ignored and
         the search is case-insensitive. The maximum allowed length for
         quality names is given by symbolic constant IRQ\_\_SZQNM which
         currently has the value of 15.
      }
      \sstsubsection{
         THERE = LOGICAL (Returned)
      }{
         If true, then the quality name is defined within the NDF
         specified by LOCS. If false, then the quality name is
         undefined.
      }
      \sstsubsection{
         STATUS = INTEGER (Given and Returned)
      }{
         The global status.
      }
   }
}
\sstroutine{
   IRQ\_CLOSE
}{
   Close down the IRQ identifier system
}{
   \sstdescription{
      This routine must be called once all use of compiled quality
      expression identifiers (as generated by IRQ\_COMP) has been
      completed.  All internal resources used by any such identifiers
      currently in use are released. Note, this routine does not
      release the locators created by IRQ\_NEW or IRQ\_FIND. IRQ\_RLSE
      must be called to release these locators.

      This routine attempts to execute even if STATUS is bad on entry,
      although no further error report will be made if it subsequently
      fails under these circumstances.
   }
   \sstinvocation{
      CALL IRQ\_CLOSE( STATUS )
   }
   \sstarguments{
      \sstsubsection{
         STATUS = INTEGER (Given and Returned)
      }{
         The global status.
      }
   }
}
\sstroutine{
   IRQ\_CNTQ
}{
   Count the number of pixels which are set in each bit-plane of the
   QUALITY component
}{
   \sstdescription{
      Each bit plane of the NDF QUALITY component corresponds to a
      different quality, described by a name stored in the quality
      names information structure in an NDF extension.  A pixel is set
      in a bit plane of the QUALITY component if the pixel has the
      quality associated with the bit plane. This routine counts the
      number of such pixels in each bit plane.

      Note, write or update access must be available for the NDF (as
      set up by routine NDF\_ASSOC for instance), and the QUALITY
      component of the NDF must not be mapped on entry to this routine.
   }
   \sstinvocation{
      CALL IRQ\_CNTQ( LOCS, SIZE, SET, STATUS )
   }
   \sstarguments{
      \sstsubsection{
         LOCS(5) = CHARACTER (Given)
      }{
         An array of 5 HDS locators. These locators identify the NDF
         and the associated quality name information.  They should have
         been obtained using routine IRQ\_FIND or routine IRQ\_NEW.
      }
      \sstsubsection{
         SIZE = INTEGER (Given)
      }{
         The number of bit planes for which a count of set pixels is
         required.
      }
      \sstsubsection{
         SET( SIZE ) = INTEGER (Returned)
      }{
         The number of pixels holding the corresponding quality in each
         of bit planes 1 to SIZE. The least significant bit is bit 1.
         If SIZE is larger than the number of bit planes in the QUALITY
         component, the unused elements are set to zero.
      }
      \sstsubsection{
         STATUS = INTEGER (Given and Returned)
      }{
         The global status.
      }
   }
}
\sstroutine{
   IRQ\_COMP
}{
   Compile a quality expression
}{
   \sstdescription{
      All the quality names referenced in the given quality expression
      (QEXP) are identified. If all quality names referenced in QEXP
      are defined within the NDF specified in LOCS, then the quality
      expression is `compiled{\tt '}, i.e.  converted into a form which can
      be used by IRQ\_SBAD. The compiled quality expression is
      identified by the returned IRQ identifier which should be
      released using IRQ\_ANNUL when no longer needed. If any error is
      reported, then IRQ is returned set to the value IRQ\_\_NOID.

      If any quality names referenced in the quality expression are not
      defined in the NDF specified by LOCS, they are returned in UNDEF,
      the number of such undefined quality names is returned in NUNDEF,
      an error is reported and STATUS is returned with value
      IRQ\_\_NOQNM. Additionally, if INFO is true, then a message is
      generated identifying each undefined quality name.

      If any of the STATUS values IRQ\_\_BADSY, IRQ\_\_MSOPT or IRQ\_\_MSOPD
      are returned (all of which correspond to various forms of syntax
      error in the quality expression, see ID6 appendix E), a pointer
      to the approximate position of the error within the quality
      expression is returned in ERRPNT.
   }
   \sstinvocation{
      CALL IRQ\_COMP( LOCS, SIZE, INFO, QEXP, UNDEF, NUNDEF, ERRPNT,
                     IDQ, STATUS )
   }
   \sstarguments{
      \sstsubsection{
         LOCS(5) = CHARACTER (Given)
      }{
         An array of 5 HDS locators. These locators identify the NDF
         and the associated quality name information.  They should have
         been obtained using routine IRQ\_FIND or routine IRQ\_NEW.
      }
      \sstsubsection{
         SIZE = INTEGER (Given)
      }{
         The size of the UNDEF array. This should be at least equal to
         the value of the symbolic constant IRQ\_\_QNREF
      }
      \sstsubsection{
         INFO = LOGICAL (Given)
      }{
         If set to .TRUE., then messages are produced identifying any
         undefined quality names.
      }
      \sstsubsection{
         QEXP = CHARACTER$*$($*$) (Given and Returned)
      }{
         A quality expression. See ID6 section 5 for details of the
         allowed formats for quality expressions. On exit, the string is
         converted to upper case and any leading blanks are removed.
      }
      \sstsubsection{
         UNDEF( SIZE ) = CHARACTER (Returned)
      }{
         An array holding any undefined quality names referenced in the
         quality expression. The array should have at least IRQ\_\_QNREF
         elements, each element being a string of length IRQ\_\_SZQNM.
      }
      \sstsubsection{
         NUNDEF = INTEGER (Returned)
      }{
         The number of undefined quality names referenced in the quality
         expression.
      }
      \sstsubsection{
         ERRPNT = INTEGER (Returned)
      }{
         If any of the STATUS values IRQ\_\_BADSY( {\tt "}Unrecognised logical
         operator or constant{\tt "}), IRQ\_\_MSOPT ( {\tt "}Missing operator{\tt "}) or
         IRQ\_\_MSOPD ( {\tt "}Missing operand{\tt "}) are returned, then ERRPNT
         returns the offset within the quality expression at which the
         error was detected. Note, the offset refers to the returned
         form of QEXP, not the given form. These will be different if
         the given form of QEXP has any leading blanks. An offset of
         zero is returned if none of the errors associated with the
         above STATUS values occur.
      }
      \sstsubsection{
         IDQ = INTEGER (Returned)
      }{
         An IRQ identifier for the compiled quality expression. This
         identifier can be passed to IRQ\_SBAD. This identifier should
         be annulled using routine IRQ\_ANNUL or IRQ\_CLOSE when it is no
         longer needed.  If an error is reported, then an invalid
         identifier (equal to IRQ\_\_NOID) is returned.
      }
      \sstsubsection{
         STATUS = INTEGER (Given and Returned)
      }{
         The global status.
      }
   }
}
\sstroutine{
   IRQ\_FIND
}{
   Find quality name information within an NDF
}{
   \sstdescription{
      A search is made through the extensions contained within the
      supplied NDF for an HDS structure containing quality name
      information. Such information is held in an HDS object named
      QUALITY\_NAMES (these objects can be created using IRQ\_NEW).  If
      no such object is found, an error is reported and the status
      IRQ\_\_NOQNI is returned (if more than one such object is found,
      an error is reported and the status IRQ\_\_MULT is returned).  The
      name of the NDF extension in which the object was found is
      returned in XNAME. An array of 5 HDS locators is returned which
      is needed when calling other IRQ routines. The first locator
      points to a temporary object which holds a cloned identifier for
      the NDF, the other 4 point to components of the QUALITY\_NAMES
      structure contained in the NDF.  IRQ\_RLSE should be called to
      annul these locators (and the NDF identifier) when no further
      access to the NDFs quality names information is required.

      The LOCS argument returned by this routine specifies the NDF
      which will be operated on by subsequent IRQ routines.
      Specifically, LOCS determines the bounds of the NDF. Care should
      therefore be taken that subsequent calls to IRQ routines refer to
      the NDF specified by the INDF argument to this routine, and not
      for instance to a section of the NDF which will in general have
      different bounds.
   }
   \sstinvocation{
      CALL IRQ\_FIND( INDF, LOCS, XNAME, STATUS )
   }
   \sstarguments{
      \sstsubsection{
         INDF = INTEGER (Given)
      }{
         The input NDF.
      }
      \sstsubsection{
         LOCS(5) = CHARACTER (Returned)
      }{
         A set of HDS locators as described above.  The character
         variables supplied for this argument should have a declared
         length equal to symbolic constant DAT\_\_SZLOC. These locator
         are annuled by calling IRQ\_RLSE.
      }
      \sstsubsection{
         XNAME = CHARACTER (Returned)
      }{
         The name of the NDF extension in which the quality name
         information was found. The character variable supplied for
         this argument should have a declared length equal to symbolic
         constant DAT\_\_SZNAM.
      }
      \sstsubsection{
         STATUS = INTEGER (Given and Returned)
      }{
         The global status.
      }
   }
}
\sstroutine{
   IRQ\_GETQN
}{
   Search for a specified quality name
}{
   \sstdescription{
      This routine searches for a specified quality name in the quality
      name specified by LOCS, and returns information related to the
      quality name. If the quality name is not defined then an error is
      reported and STATUS returned equal to IRQ\_\_NOQNM.
   }
   \sstinvocation{
      CALL IRQ\_GETQN( LOCS, QNAME, FIXED, VALUE, BIT, COMMNT, STATUS )
   }
   \sstarguments{
      \sstsubsection{
         LOCS(5) = CHARACTER (Given)
      }{
         An array of 5 HDS locators. These locators identify the NDF
         and the associated quality name information.  They should have
         been obtained using routine IRQ\_FIND or routine IRQ\_NEW.
      }
      \sstsubsection{
         QNAME = CHARACTER (Given)
      }{
         The quality name to search for. Leading blanks are ignored and
         the search is case-insensitive. The maximum allowed length for
         quality names is given by symbolic constant IRQ\_\_SZQNM which
         currently has the value of 15.
      }
      \sstsubsection{
         FIXED = LOGICAL (Returned)
      }{
         If true, then the quality is either held by all pixels, or by
         no pixels. In this case the quality does not have a
         corresponding bit in the QUALITY component. If false, then
         some pixels have the quality and some don{\tt '}t, as indicated by
         the corresponding bit in the QUALITY component.
      }
      \sstsubsection{
         VALUE = LOGICAL (Returned)
      }{
         If FIXED is true, then VALUE specifies whether all pixels hold
         the quality ( VALUE = .TRUE. ), or whether no pixels hold the
         quality ( VALUE = .FALSE. ). If FIXED is false, then VALUE is
         indeterminate.
      }
      \sstsubsection{
         BIT = INTEGER (Returned)
      }{
         If FIXED is false, then BIT holds the corresponding bit number
         in the QUALITY component. The least significant bit is called
         bit 1 (not bit 0). If FIXED is true, then BIT is
         indeterminate.
      }
      \sstsubsection{
         COMMNT = CHARACTER (Returned)
      }{
         The descriptive comment which was stored with the quality name.
         The supplied character variable should have a declared length
         given by symbolic constant IRQ\_\_SZCOM.
      }
      \sstsubsection{
         STATUS = INTEGER (Given and Returned)
      }{
         The global status.
      }
   }
}
\sstroutine{
   IRQ\_NEW
}{
   Create a new structure to hold quality name information within an
   NDF extension
}{
   \sstdescription{
      An HDS object (named QUALITY\_NAMES) is created to hold quality
      name information within the specified NDF extension. An error is
      reported if the NDF extension does not exist. If the extension does
      exist, an array of 5 HDS locators is returned which is needed
      when calling other IRQ routines. The first locator points to a
      temporary object which holds a cloned identifier for the NDF,
      the other 4 point to components of the QUALITY\_NAMES structure
      contained in the NDF. IRQ\_RLSE should be called to annul these
      locators (and the NDF identifier) when no further access to the
      NDFs quality names information is required.

      The QUALITY component of the NDF is reset to an undefined state by
      this routine.

      The LOCS argument returned by this routine specifies the NDF
      which will be operated on by subsequent IRQ routines.
      Specifically, LOCS determines the bounds of the NDF. Care should
      therefore be taken that subsequent calls to IRQ routines refer to
      the NDF specified by the INDF argument to this routine, and not
      for instance to a section of the NDF which will in general have
      different bounds.

      Note, an error is reported if only READ access is available to the
      NDF.
   }
   \sstinvocation{
      CALL IRQ\_NEW( INDF, XNAME, LOCS, STATUS )
   }
   \sstarguments{
      \sstsubsection{
         INDF = INTEGER (Given)
      }{
         The NDF identifier.
      }
      \sstsubsection{
         XNAME = CHARACTER (Given)
      }{
         The name of the NDF extension in which the quality name
         information is to be stored. If this extension does not exist
         then an error is reported.
      }
      \sstsubsection{
         LOCS(5) = CHARACTER (Returned)
      }{
         A set of HDS locators as described above.  The character
         variables supplied for this argument should have a declared
         length equal to symbolic constant DAT\_\_SZLOC. These locator
         are annuled by calling IRQ\_RLSE.
      }
      \sstsubsection{
         STATUS = INTEGER (Given and Returned)
      }{
         The global status.
      }
   }
}
\sstroutine{
   IRQ\_NUMQN
}{
   Return number of defined quality names
}{
   \sstdescription{
      The number of quality names defined in the NDF specified by
      LOCS is returned.
   }
   \sstinvocation{
      CALL IRQ\_NUMQN( LOCS, NAMES, STATUS )
   }
   \sstarguments{
      \sstsubsection{
         LOCS(5) = CHARACTER (Given)
      }{
         An array of 5 HDS locators. These locators identify the NDF
         and the associated quality name information.  They should have
         been obtained using routine IRQ\_FIND or routine IRQ\_NEW.
      }
      \sstsubsection{
         NAMES = INTEGER (Returned)
      }{
         The number of quality names defined in the structure located by
         LOCS.
      }
      \sstsubsection{
         STATUS = INTEGER (Given and Returned)
      }{
         The global status.
      }
   }
}
\sstroutine{
   IRQ\_NXTQN
}{
   Return the next quality name
}{
   \sstdescription{
      This routine returns the next quality name defined in the NDF
      specified by LOCS, together with supplementary information.  The
      next quality name is determined by the value of CONTXT. If CONTXT
      is zero on entry then the first quality name is returned.  On
      exit, CONTXT is set to a value which indicates where the next
      quality name is stored within the NDF. This value can be passed
      to a subsequent call to this routine to retrieve information
      about the next quality name.
   }
   \sstinvocation{
      CALL IRQ\_NXTQN( LOCS, CONTXT, QNAME, FIXED, VALUE, BIT,
                      COMMNT, DONE, STATUS )
   }
   \sstarguments{
      \sstsubsection{
         LOCS(5) = CHARACTER (Given)
      }{
         An array of 5 HDS locators. These locators identify the NDF
         and the associated quality name information.  They should have
         been obtained using routine IRQ\_FIND or routine IRQ\_NEW.
      }
      \sstsubsection{
         CONTXT = INTEGER (Given and Returned)
      }{
         The context of the current call. This should be set to zero
         before the first call to this routine, and then left unchanged
         between subsequent calls.
      }
      \sstsubsection{
         QNAME = CHARACTER (Returned)
      }{
         The next quality name. The character variable supplied for this
         argument should have a declared length equal to the symbolic
         constant IRQ\_\_SZQNM.
      }
      \sstsubsection{
         FIXED = LOGICAL (Returned)
      }{
         If true, then the quality is either held by all pixels, or by
         no pixels. In this case the quality does not have a
         corresponding bit in the QUALITY component. If false, then
         some pixels have the quality and some don{\tt '}t, as indicated by
         the corresponding bit in the QUALITY component.
      }
      \sstsubsection{
         VALUE = LOGICAL (Returned)
      }{
         If FIXED is true, then VALUE specifies whether all pixels hold
         the quality ( VALUE = .TRUE. ), or whether no pixels hold the
         quality ( VALUE = .FALSE. ). If FIXED is false, then VALUE is
         indeterminate.
      }
      \sstsubsection{
         BIT = INTEGER (Returned)
      }{
         If FIXED is false, then BIT holds the corresponding bit number
         in the QUALITY component. The least significant bit is called
         bit 1 (not bit 0). If FIXED is true, then BIT is
         indeterminate.
      }
      \sstsubsection{
         COMMNT = CHARACTER (Returned)
      }{
         The descriptive comment which was stored with the quality name.
         The supplied character variable should have a declared length
         given by symbolic constant IRQ\_\_SZCOM.
      }
      \sstsubsection{
         DONE = LOGICAL (Returned)
      }{
         Returned true if this routine is called when no more names
         remain to be returned.
      }
      \sstsubsection{
         STATUS = INTEGER (Given and Returned)
      }{
         The global status.
      }
   }
}
\sstroutine{
   IRQ\_REMQN
}{
   Remove the definition of a specified quality name
}{
   \sstdescription{
      The specified quality name is removed from the NDF specified
      by LOCS. Any associated bit in the QUALITY array is freed for
      future use. If the name is not defined an error is reported.
      A value of ANY for the quality names causes all defined quality
      names to be removed.

      Note, an error is reported if only read access is available to the
      NDF, or if the quality name has been flagged as read-only using
      routine IRQ\_RWQN.
   }
   \sstinvocation{
      CALL IRQ\_REMQN( LOCS, QNAME, STATUS )
   }
   \sstarguments{
      \sstsubsection{
         LOCS(5) = CHARACTER (Given)
      }{
         An array of 5 HDS locators. These locators identify the NDF
         and the associated quality name information.  They should have
         been obtained using routine IRQ\_FIND or routine IRQ\_NEW.
      }
      \sstsubsection{
         QNAME = CHARACTER (Given)
      }{
         The quality name to remove, or {\tt '}ANY{\tt '} if all quality names are
         to be removed.
      }
      \sstsubsection{
         STATUS = INTEGER (Given and Returned)
      }{
         The global status.
      }
   }
}
\sstroutine{
   IRQ\_RESQ
}{
   Remove a given quality from all pixels in the NDF
}{
   \sstdescription{
      The quality specified by QNAME is removed from all pixels in the
      NDF specified by LOCS (LOCS should be obtained either by calling
      IRQ\_FIND or IRQ\_NEW). An error is reported if the quality name is
      undefined within the NDF.

      Note, write or update access must be available for the NDF (as
      set up by routine NDF\_ASSOC for instance).
   }
   \sstinvocation{
      CALL IRQ\_RESQ( LOCS, QNAME, STATUS )
   }
   \sstarguments{
      \sstsubsection{
         LOCS(5) = CHARACTER (Given)
      }{
         An array of 5 HDS locators. These locators identify the NDF
         and the associated quality name information.  They should have
         been obtained using routine IRQ\_FIND or routine IRQ\_NEW.
      }
      \sstsubsection{
         QNAME = CHARACTER (Given)
      }{
         The quality name to be removed from all pixels in the NDF. This
         quality name must be defined in the NDF specified by LOC. Name
         definitions can be added to the NDF using routine IRQ\_ADDQN.
      }
      \sstsubsection{
         STATUS = INTEGER (Given and Returned)
      }{
         The global status.
      }
   }
}
\sstroutine{
   IRQ\_RESQL
}{
   Remove a quality from a list of pixels, leaving unlisted pixels
   unchanged
}{
   \sstdescription{
      The quality specified by QNAME is removed from all pixels
      included in (or, if LISTED is false, not included in) the
      supplied list of pixel indices. The quality of other pixels
      is left unaltered.  The quality name must be defined in the NDF
      specified by LOCS (LOCS should be obtained either by calling
      IRQ\_FIND or IRQ\_NEW). An error is reported if the quality name is
      undefined.

      Note, write or update access must be available for the NDF (as
      set up by routine NDF\_ASSOC for instance), and the QUALITY
      component must not be mapped on entry to this routine.
   }
   \sstinvocation{
      CALL IRQ\_RESQL( LOCS, LISTED, QNAME, NDIM, NCOORD, LIST, SET,
                      STATUS )
   }
   \sstarguments{
      \sstsubsection{
         LOCS(5) = CHARACTER (Given)
      }{
         An array of 5 HDS locators. These locators identify the NDF
         and the associated quality name information.  They should have
         been obtained using routine IRQ\_FIND or routine IRQ\_NEW.
      }
      \sstsubsection{
         LISTED = LOGICAL (Given)
      }{
         If true, then the quality is removed from all pixels included
         in the list given by LIST.  If false, then the quality is
         removed from all pixels not included in the list given by LIST.
      }
      \sstsubsection{
         QNAME = CHARACTER (Given)
      }{
         The quality name to be removed from the selected pixels. This
         quality name must be defined in the NDF specified by LOC. Name
         definitions can be added to the NDF using routine IRQ\_ADDQN.
      }
      \sstsubsection{
         NDIM = INTEGER (Given)
      }{
         The number of values required to specify a pixel position
         (i.e. the number of dimensions in the NDF).
      }
      \sstsubsection{
         NCOORD = INTEGER (Given)
      }{
         The number of pixels included in the input list.
      }
      \sstsubsection{
         LIST( NDIM, NCOORD ) = INTEGER (Given)
      }{
         The list of pixel indices. Any indices which lie outside the
         bounds of the NDF are ignored.
      }
      \sstsubsection{
         SET = INTEGER (Returned)
      }{
         The number of pixels which hold the quality.
      }
      \sstsubsection{
         STATUS = INTEGER (Given and Returned)
      }{
         The global status.
      }
   }
}
\sstroutine{
   IRQ\_RESQM
}{
   Remove a quality from pixels selected using a mask image, leaving
   unselected pixels unchanged
}{
   \sstdescription{
      The quality specified by QNAME is removed from all NDF pixels
      which either do (or, if BAD is false,  do not) correspond to `bad{\tt '}
      pixels in the input mask array.  The quality of all other pixels
      is left unchanged.  The quality name must be defined in the NDF
      specified by LOCS (LOCS should be obtained either by calling
      IRQ\_FIND or IRQ\_NEW). An error is reported if the quality name is
      undefined.

      Note, write or update access must be available for the NDF (as
      set up by routine NDF\_ASSOC for instance), and the QUALITY
      component of the NDF must not be mapped on entry to this routine.
   }
   \sstinvocation{
      CALL IRQ\_RESQM( LOCS, BAD, QNAME, SIZE, MASK, SET, STATUS )
   }
   \sstarguments{
      \sstsubsection{
         LOCS(5) = CHARACTER (Given)
      }{
         An array of 5 HDS locators. These locators identify the NDF
         and the associated quality name information.  They should have
         been obtained using routine IRQ\_FIND or routine IRQ\_NEW.
      }
      \sstsubsection{
         BAD = LOGICAL (Given)
      }{
         If true, then the quality is removed from all NDF pixels
         corresponding to `bad{\tt '} pixels in the mask.  If false, then the
         quality is removed from all NDF pixels corresponding to pixels
         which are not `bad{\tt '} in the mask.
      }
      \sstsubsection{
         QNAME = CHARACTER (Given)
      }{
         The quality name to be removed from the selected pixels. This
         quality name must be defined in the NDF specified by LOC. Name
         definitions can be added to the NDF using routine IRQ\_ADDQN.
      }
      \sstsubsection{
         SIZE = INTEGER (Given)
      }{
         The total number of pixels in the MASK array.
      }
      \sstsubsection{
         MASK( SIZE ) = REAL (Given)
      }{
         A vector which defines the pixels from which the quality
         specified by QNAME is to be removed.  It is assumed that this
         vector corresponds pixel-for-pixel with the vectorised NDF as
         supplied to routine IRQ\_FIND or IRQ\_NEW.
      }
      \sstsubsection{
         SET = INTEGER (Returned)
      }{
         The number of pixels in the NDF which hold the quality.
      }
      \sstsubsection{
         STATUS = INTEGER (Given and Returned)
      }{
         The global status.
      }
   }
}
\sstroutine{
   IRQ\_RLSE
}{
   Release a temporary structure created by IRQ\_NEW or IRQ\_FIND
}{
   \sstdescription{
      This routine releases the resources reserved by a call to IRQ\_NEW
      or IRQ\_FIND. The cloned NDF identifier held in LOCS(1) is
      annulled, and then all the five HDS locators in LOCS are annulled.
      If no defined quality names exist within the NDF, then the
      structure used to hold such names is deleted and the QUALITY
      component of the NDF is reset to an undefined state.

      Note, this routine attempts to execute even if STATUS is set on
      entry, although no further error report will be made if it
      subsequently fails under these circumstances.
   }
   \sstinvocation{
      CALL IRQ\_RLSE( LOCS, STATUS )
   }
   \sstarguments{
      \sstsubsection{
         LOCS(5) = CHARACTER (Given)
      }{
         An array of 5 HDS locators. These locators identify the NDF
         and the associated quality name information.  They should have
         been obtained using routine IRQ\_FIND or routine IRQ\_NEW.
      }
      \sstsubsection{
         STATUS = INTEGER (Given and Returned)
      }{
         The global status.
      }
   }
}
\sstroutine{
   IRQ\_RWQN
}{
   Get and/or set the read-only flag for a quality name
}{
   \sstdescription{
      This routine returns the current value of the read-only flag
      associated with a quality name, and optionally assigns a new
      value to the flag.

      If the read-only flag is set for a quality name, any attempt to
      remove the quality name using IRQ\_REMQN will result in an error
      being reported.
   }
   \sstinvocation{
      CALL IRQ\_RWQN( LOCS, QNAME, SET, NEWVAL, OLDVAL, STATUS )
   }
   \sstarguments{
      \sstsubsection{
         LOCS(5) = CHARACTER $*$ ( $*$ ) (Given)
      }{
         An array of 5 HDS locators. These locators identify the NDF
         and the associated quality name information.  They should have
         been obtained using routine IRQ\_FIND or routine IRQ\_NEW.
      }
      \sstsubsection{
         QNAME = CHARACTER $*$ ( $*$ ) (Given)
      }{
         The quality name to use. Leading blanks are ignored and
         the search is case-insensitive. The maximum allowed length for
         quality names is given by symbolic constant IRQ\_\_SZQNM which
         currently has the value of 15.
      }
      \sstsubsection{
         SET = LOGICAL (Given)
      }{
         If true, then the read-only flag for the quality name will be
         set to the value supplied in NEWVAL. Otherwise, the current
         value of the flag will be left unchanged.
      }
      \sstsubsection{
         NEWVAL = LOGICAL (Given)
      }{
         The new value for the read-only flag. Only accessed if SET is
         true.
      }
      \sstsubsection{
         OLDVAL = LOGICAL (Returned)
      }{
         The value of the read-only flag on entry to this routine. If the
         old value is of no interest, it is safe to supply the same
         variable for OLDVAL as for NEWVAL since OLDVAL is updated after
         NEWVAL is used.
      }
      \sstsubsection{
         STATUS = INTEGER (Given and Returned)
      }{
         The global status.
      }
   }
}
\sstroutine{
   IRQ\_SBAD
}{
   Set pixels `bad{\tt '} which satisfy a given quality expression
}{
   \sstdescription{
      IRQ\_COMP should be called before this routine to produce the
      compiled quality expression identified by IDQ. The QUALITY
      component of the NDF to which the quality expression refers (see
      IRQ\_COMP argument LOCS) is mapped as a one dimensional vector.
      The supplied array VEC must correspond pixel-for-pixel with the
      mapped QUALITY vector.  All pixels which hold a QUALITY
      satisfying the quality expression are found. If HELD is true,
      then the corresponding pixels in VEC are set to the `bad{\tt '} value
      (other pixels are left unaltered). If HELD is false, the
      corresponding pixels in VEC are left as they are, but all the
      other pixels in VEC are set to the `bad{\tt '} value. ALLBAD and NOBAD
      indicate if the output VEC values are either all bad or all good.

      Note, if the QUALITY component of the NDF is mapped for WRITE or
      UPDATE access on entry to this routine, an error is reported.
   }
   \sstinvocation{
      CALL IRQ\_SBAD( IDQ, HELD, SIZE, VEC, ALLBAD, NOBAD, STATUS )
   }
   \sstarguments{
      \sstsubsection{
         IDQ = INTEGER (Given)
      }{
         An identifier for a compiled quality expression, produced by
         routine IRQ\_COMP. This identifier determines the NDF to which
         the expression refers.
      }
      \sstsubsection{
         HELD = LOGICAL (Given)
      }{
         If true then those VEC pixels which hold a quality satisfying
         the supplied quality expression are set `bad{\tt '}. Otherwise, those
         pixels which don{\tt '}t hold such a quality are set `bad{\tt '}.
      }
      \sstsubsection{
         SIZE = INTEGER (Given)
      }{
         The total number of pixels in VEC. An error is reported if
         this is not the same as the total number of pixels in the NDF
         determined by IDQ.
      }
      \sstsubsection{
         VEC( SIZE ) = REAL (Given and Returned)
      }{
         The data to be set `bad{\tt '}, depending on the corresponding
         quality values stored in the NDF. It must be the same size as
         the NDF, and must correspond pixel-for-pixel with the
         vectorised NDF. Pixels which are not explicitly set `bad{\tt '} by
         this routine retain the values they had on entry.
      }
      \sstsubsection{
         ALLBAD = LOGICAL (Returned)
      }{
         Returned true if all pixels in VEC are returned with `bad{\tt '}
         values, and false if any returned pixel values are not `bad{\tt '}.
      }
      \sstsubsection{
         NOBAD = LOGICAL (Returned)
      }{
         Returned true if no pixels in VEC are returned with `bad{\tt '}
         values. False if any `bad{\tt '} pixel values are returned.
      }
      \sstsubsection{
         STATUS = INTEGER (Given and Returned)
      }{
         The global status.
      }
   }
}
\sstroutine{
   IRQ\_SETQ
}{
   Assign a given quality to all pixels in the NDF
}{
   \sstdescription{
      The quality specified by QNAME is assigned to all pixels in the
      NDF specified by LOCS (LOCS should be obtained either by calling
      IRQ\_FIND or IRQ\_NEW). An error is reported if the quality name is
      undefined within the NDF.

      Note, write or update access must be available for the NDF (as
      set up by routine NDF\_ASSOC for instance).
   }
   \sstinvocation{
      CALL IRQ\_SETQ( LOCS, QNAME, STATUS )
   }
   \sstarguments{
      \sstsubsection{
         LOCS(5) = CHARACTER (Given)
      }{
         An array of 5 HDS locators. These locators identify the NDF
         and the associated quality name information.  They should have
         been obtained using routine IRQ\_FIND or routine IRQ\_NEW.
      }
      \sstsubsection{
         QNAME = CHARACTER (Given)
      }{
         The quality name to assign to all pixels in the NDF. This
         quality name must be defined in the NDF specified by LOC. Name
         definitions can be added to the NDF using routine IRQ\_ADDQN.
      }
      \sstsubsection{
         STATUS = INTEGER (Given and Returned)
      }{
         The global status.
      }
   }
}
\sstroutine{
   IRQ\_SETQL
}{
   Assign a given quality to a list of pixels, leaving unlisted
   pixels unchanged
}{
   \sstdescription{
      The quality specified by QNAME is assigned to all pixels included
      (or, if LISTED is false, not included) in the supplied list of
      pixel indices. The quality of other pixels is left unaltered.
      The quality name must be defined in the NDF specified by LOCS
      (LOCS should be obtained either by calling IRQ\_FIND or IRQ\_NEW).
      An error is reported if the quality name is undefined.

      Note, write or update access must be available for the NDF (as
      set up by routine NDF\_ASSOC for instance), and the QUALITY
      component must not be mapped on entry to this routine.
   }
   \sstinvocation{
      CALL IRQ\_SETQL( LOCS, LISTED, QNAME, NDIM, NCOORD, LIST, SET,
                      STATUS )
   }
   \sstarguments{
      \sstsubsection{
         LOCS(5) = CHARACTER (Given)
      }{
         An array of 5 HDS locators. These locators identify the NDF
         and the associated quality name information.  They should have
         been obtained using routine IRQ\_FIND or routine IRQ\_NEW.
      }
      \sstsubsection{
         LISTED = LOGICAL (Given)
      }{
         If true, then the quality is assigned to all pixels included
         in the list given by LIST.  If false, then the quality is
         assigned to all pixels not included in the list given by LIST.
      }
      \sstsubsection{
         QNAME = CHARACTER (Given)
      }{
         The quality name to assign to the selected pixels. This quality
         name must be defined in the NDF specified by LOC. Name
         definitions can be added to the NDF using routine IRQ\_ADDQN.
      }
      \sstsubsection{
         NDIM = INTEGER (Given)
      }{
         The number of values required to specify a pixel position
         (i.e. the number of dimensions in the NDF).
      }
      \sstsubsection{
         NCOORD = INTEGER (Given)
      }{
         The number of pixels included in the input list.
      }
      \sstsubsection{
         LIST( NDIM, NCOORD ) = INTEGER (Given)
      }{
         The list of pixel indices. Any indices which lie outside the
         bounds of the NDF are ignored.
      }
      \sstsubsection{
         SET = INTEGER (Returned)
      }{
         The number of pixels in the NDF which hold the quality.
      }
      \sstsubsection{
         STATUS = INTEGER (Given and Returned)
      }{
         The global status.
      }
   }
}
\sstroutine{
   IRQ\_SETQM
}{
   Assign a quality to pixels selected using a mask image, leaving
   unselected pixels unchanged
}{
   \sstdescription{
      The quality specified by QNAME is assigned to all NDF pixels
      which either do (or, if BAD is false,  do not) correspond to `bad{\tt '}
      pixels in the input mask array.  The quality of all other pixels
      is left unchanged.  The quality name must be defined in the NDF
      specified by LOCS (LOCS should be obtained either by calling
      IRQ\_FIND or IRQ\_NEW). An error is reported if the quality name is
      undefined.

      Note, write or update access must be available for the NDF (as
      set up by routine NDF\_ASSOC for instance), and the QUALITY
      component of the NDF must not be mapped on entry to this routine.
   }
   \sstinvocation{
      CALL IRQ\_SETQM( LOCS, BAD, QNAME, SIZE, MASK, SET, STATUS )
   }
   \sstarguments{
      \sstsubsection{
         LOCS(5) = CHARACTER (Given)
      }{
         An array of 5 HDS locators. These locators identify the NDF
         and the associated quality name information.  They should have
         been obtained using routine IRQ\_FIND or routine IRQ\_NEW.
      }
      \sstsubsection{
         BAD = LOGICAL (Given)
      }{
         If true, then the quality is assigned to all NDF pixels
         corresponding to `bad{\tt '} pixels in the mask.  If false, then the
         quality is assigned to all NDF pixels corresponding to pixels
         which are not `bad{\tt '} in the mask.
      }
      \sstsubsection{
         QNAME = CHARACTER (Given)
      }{
         The quality name to assign to the selected pixels. This quality
         name must be defined in the NDF specified by LOC. Name
         definitions can be added to the NDF using routine IRQ\_ADDQN.
      }
      \sstsubsection{
         SIZE = INTEGER (Given)
      }{
         The total number of pixels in the MASK array.
      }
      \sstsubsection{
         MASK( SIZE ) = REAL (Given)
      }{
         A vector which defines the pixels to which the quality
         specified by QNAME is to be assigned.  It is assumed that this
         vector corresponds pixel-for-pixel with the vectorised NDF
         supplied to IRQ\_NEW or IRQ\_FIND.
      }
      \sstsubsection{
         SET = INTEGER (Returned)
      }{
         The number of pixels in the NDF which hold the quality.
      }
      \sstsubsection{
         STATUS = INTEGER (Given and Returned)
      }{
         The global status.
      }
   }
}
\sstroutine{
   IRQ\_SYNTX
}{
   Compile a quality expression
}{
   \sstdescription{
      The syntax of the supplied quality expression is checked, and an
      error is reported if a syntax error is detected.  If any of the
      STATUS values IRQ\_\_BADSY, IRQ\_\_MSOPT or IRQ\_\_MSOPD are returned
      (all of which correspond to various forms of syntax error in the
      quality expression, see ID6 appendix E), a pointer to the
      approximate position of the error within the quality expression
      is returned in ERRPNT.  Note, in order for a quality expression
      to compile succesfully (using IRQ\_COM), it must not only contain
      no syntax errors, but must also contain no undefined quality
      names. IRQ\_SYNTX cannot check for undefined quality names.
   }
   \sstinvocation{
      CALL IRQ\_SYNTX( QEXP, ERRPNT, STATUS )
   }
   \sstarguments{
      \sstsubsection{
         QEXP = CHARACTER$*$($*$) (Given and Returned)
      }{
         A quality expression. See ID6 section 5 for details of the
         allowed formats for quality expressions. On exit, the string is
         converted to upper case and any leading blanks are removed.
      }
      \sstsubsection{
         ERRPNT = INTEGER (Returned)
      }{
         If any of the STATUS values IRQ\_\_BADSY( {\tt "}Unrecognised logical
         operator or constant{\tt "}), IRQ\_\_MSOPT ( {\tt "}Missing operator{\tt "}) or
         IRQ\_\_MSOPD ( {\tt "}Missing operand{\tt "}) are returned, then ERRPNT
         returns the offset within the quality expression at which the
         error was detected. Note, the offset refers to the returned
         form of QEXP, not the given form. These will be different if
         the given form of QEXP has any leading blanks. An offset of
         zero is returned if none of the errors associated with the
         above STATUS values occur.
      }
      \sstsubsection{
         STATUS = INTEGER (Given and Returned)
      }{
         The global status.
      }
   }
}
\section {Templates for IRQ Routines Within the VAX LSE Editor}
The STARLSE package (see SUN/105) provides facilities for initialising the VAX
Language Sensitive Editor (LSE) to simplify the generation of Fortran
code conforming to the Starlink programming standard (see SGP/16). One of the
facilities provided by LSE is the automatic production of argument lists for
subroutine calls. Templates for all the subroutines in the IRQ package can be 
made available within LSE by performing the following steps (within LSE):
\begin{enumerate}
\item Issue the LSE command GOTO FILE/READ IRQ\_DIR:IRQ.LSE
\item Issue the LSE command DO
\item Move to a buffer holding a .FOR of a .GEN file in the usual way.
\item IRQ subroutine templates are then available by typing in the name of an 
IRQ subroutine (or an abbreviation) and expanding it (CTRL-E).
\item Help on the subroutine and its arguments can be obtained by placing the
cursor at a point in the buffer at which the subroutine name has been entered 
and pressing GOLD-PF2.
\end{enumerate}

\section {Using IRQ Without the Rest of IRAS90}
\label {APP:NOSTAR}
If the IRAS90 package has not been installed as part of the Starlink 
Software Collection, then some of the instructions given above need to be 
changed. In this appendix the term ``IRQ root directory'' is used to refer to
the directory containing the released IRQ files (on either VMS or UNIX).

\subsection{Compiling and Linking}
On VMS systems the IRAS90 start-up commands should be omitted and the 
development command IRQ\_DEV should be used in place of IRAS90\_DEV. The
linker options files IRQ\_LINK and IRQ\_LINK\_ADAM should be used instead
of IRAS90\_LINK and IRAS90\_LINK\_ADAM.

On UNIX systems, the {\bf iras90} startup command should be omitted and the
development command {\bf irq\_dev} should be used instead of {\bf iras90\_dev}. 
The link files {\bf irq\_link} and {\bf irq\_link\_adam} should be used instead
of {\bf iras90\_link} and {\bf iras90\_link\_adam}. 

\section {HDS Data Structures}
The IRQ package uses several different HDS data structures. These are described 
in this appendix. HDS {\em names} are indicated by being placed within
square brackets ({[} {]}), and HDS {\em types} are indicated by being placed 
within angled brackets ($<$ $>$).
\subsection{Quality names information stored in an NDF}
Information describing the quality names which are defined within an NDF is
stored in a structure called {[}{\bf QUALITY\_NAMES}{]}, with HDS {\em type} 
$<${\bf QUALITY\_NAMES}$>$. This structure can be stored in any extension 
within the NDF. The components of this structure are shown in table \ref 
{TAB:QN}.

\begin{table}[htb]
\centering
\caption{Components of a $<${\bf QUALITY\_NAMES}$>$ structure}
\label{TAB:QN}
\begin{tabular}{|l|l|l|}
\hline
Component Name & TYPE & Brief Description \\ \hline
{[}{\bf QUAL}{]} & $<${\bf IRQ\_QUAL}$>$ & Vector of quality name definitions\\
{[}{\bf LAST\_USED}{]} & $<${\bf \_INTEGER}$>$ & Highest used index within 
{[}{\bf QUAL}{]}\\
{[}{\bf NFREE}{]} & $<${\bf \_INTEGER}$>$ & No. of un-used cells within
{[}{\bf QUAL}{]}\\
{[}{\bf FREE}(NFREE){]} & $<${\bf \_INTEGER}$>$ & Vector holding indices of 
un-used cells in {[}{\bf QUAL}{]}\\
\hline
\end{tabular}
\end{table}

The {[}{\bf QUAL}{]} component is a vector in which each cell is a structure
holding various items of information needed to define a single quality name. The
array has an initial size of 8 but is increased if necessary.
There will usually be some unused cells within {[}{\bf QUAL}{]},
and the other components of the {[}{\bf QUALITY\_NAMES}{]} structure listed in
table \ref {TAB:QN} are used to locate these unused cells. In particular,
{[}{\bf FREE}{]} is a vector which holds the indices of all unused cells within
{[}{\bf QUAL}{]}. These indices may appear in any order within in {[}{\bf
FREE}{]}. 

The set of five HDS locators returned by routines IRQ\_NEW and IRQ\_FIND include
locators to each of the four components of the {[}{\bf QUALITY\_NAMES}{]}
structure, together with a locator for a temporary HDS structure holding a
single $<${\bf \_INTEGER}$>$ scalar used to store the cloned NDF identifier. 

The information describing each individual quality name is stored in a single
cell of the {[}{\bf QUAL}{]} vector. The components within each cell of this 
array are listed in table \ref {TAB:QUAL}.

\begin{table}[htb]
\centering
\caption{Components of a $<${\bf IRQ\_QUAL}$>$ structure}
\label{TAB:QUAL}
\begin{tabular}{|l|l|l|}
\hline
Component Name & TYPE & Brief Description \\ \hline
{[}{\bf NAME}{]} & $<${\bf \_CHAR}$>*$15 & A quality name \\
{[}{\bf FIXED}{]} & $<${\bf \_LOGICAL}$>$ & True if all pixels are in the same 
state\\
{[}{\bf VALUE}{]} & $<${\bf \_LOGICAL}$>$ & The state of all pixels, if fixed\\
{[}{\bf BIT}{]} & $<${\bf \_INTEGER}$>$ & QUALITY bit used to store this 
quality\\
{[}{\bf COMMENT}{]} & $<${\bf \_CHAR}$>*$50 & A descriptive comment for the 
quality\\
\hline
\end{tabular}
\end{table}

If {\em all} pixels hold a given quality, or if {\em no} pixels hold the
quality, then a true value is stored for {[}{\bf FIXED}{]}. A true value is
stored for {[}{\bf VALUE}{]} if {\em all} pixels hold the quality, and a false
value if {\em no} pixels hold the quality. In either case, {[}{\bf BIT}{]} is
ignored since no QUALITY bit needs to be reserved for the quality name, thus
allowing more than 8 quality names to be defined simultaneously. 

If some pixels {\em do} hold the quality but some {\em do not}, then a false 
value is stored for {[}{\bf FIXED}{]} and {[}{\bf VALUE}{]} is ignored. In this 
case, a QUALITY bit is reserved to represent the quality and its bit number (in 
the range 1 to 8) is stored in {[}{\bf BIT}{]}.

\subsection{Temporary structures used to hold compiled quality expressions}
When routine IRQ\_COMP is called to compile a quality expression, the resulting
information (known as a ``compiled quality expression'') is stored in a
temporary HDS structure. Up to ten compiled quality expressions can exist
simultaneously, each being stored in one cell of an array of temporary
structures. The identifier returned by IRQ\_COMP is just an index within this
array. Each cell of the array has an HDS {\em name} of {[}{\bf QEXP}{]} and an
HDS {\em type} of $<${\bf QEXP}$>$, and contains the components listed in table
\ref {TAB:QEXP}. 

\begin{table}[htb]
\centering
\caption{Components of a $<${\bf QEXP}$>$ structure}
\label{TAB:QEXP}
\begin{tabular}{|l|l|l|}
\hline
Component Name & TYPE & Brief Description \\ \hline
{[}{\bf MASKS}{]} & $<${\bf \_INTEGER}$>$ & A vector of bit masks\\
{[}{\bf OPCODE}{]} & $<${\bf \_INTEGER}$>$ & A vector of instruction codes\\
\hline
\end{tabular}
\end{table}

Each bit mask held in {[}{\bf MASKS}{]} specifies a set of QUALITY bits which
are to be tested as part of the evaluation of a quality expression performed by
routine IRQ\_SBAD. The instruction codes held in {[}{\bf OPCODE}{]} represent
the operations which must be performed on a ``First In - Last Out'' stack in
order to evaluate a quality expression. The sizes of theses vectors are held in
common. 

\section {Examples of Using IRQ}
\newcommand{\numcir}[1]{\mbox{\hspace{3ex}$\bigcirc$\hspace{-1.7ex}{\small #1}}}
\label{APP:EXAMS}
\subsection {Adding a new quality name}
This example shows a VMS code fragment which adds the quality name SATURATED to an
NDF and then assigns the quality to all pixels with value greater than 10.0.
An error is reported if the quality name is already in use.

\begin{quote}
\begin{tabbing} % Not clear why this is needed, since tabs are not used,
                % but the leading spaces get lost otherwise.

\verb#*  Include ADAM, IRAS90, IRQ and NDF symbolic constants.#\\
\verb#      INCLUDE 'SAE_PAR'                                #\\
\verb#      INCLUDE 'IRQ_PAR'                                #\\
\verb#      INCLUDE 'IRQ_ERR'                                #\\
\verb#      INCLUDE 'NDF_PAR'                                #\\
\verb#                                                       #\\
\verb#*  Declare local INTEGER variables.                    #\\
\verb#      INTEGER NDFIN, NDF2, NDIM, LBND(NDF__MXDIM), NEL,#\\
\verb#     :        UBND(NDF__MXDIM), PNT, PNT2, PLACE,      #\\
\verb#     :        STATUS                                   #\\
\verb#                                                       #\\
\verb#*  Declare local LOGICAL variables.                    #\\
\verb#      INTEGER FOUND                                    #\\
\verb#                                                       #\\
\verb#*  Declare local CHARACTER variables.                  #\\
\verb#      CHARACTER*(DAT__SZLOC) LOCS(5),XLOC              #\\
\verb#      CHARACTER*(DAT__SZNAM) XNAME                     #\\
\verb#                                                       #\\
\verb#*  Start an NDF context.                               #\\
\verb#      CALL NDF_BEGIN                                   #\\
\verb#                                                       #\\
\verb#*  Obtain an identifier for the input NDF.             #\\
\verb#      CALL NDF_ASSOC( 'IN', 'READ', NDFIN, STATUS )    #\\
\verb#                                                       #\\
\verb#*  Attempt to locate any existing quality name         #\\
\verb#*  information in the input NDF. If such information is#\\
\verb#*  found, LOCS is returned holding a set of five HDS   #\\
\verb#*  locators which identify the NDF and various items of#\\
\verb#*  quality information. XNAME is returned holding the  #\\
\verb#*  name of the NDF extension in which the information  #\\
\verb#*  was found. If no quality name information is found, #\\
\verb#*  then an error is reported.                          #\\
\verb#      CALL IRQ_FIND( NDFIN, LOCS, XNAME, STATUS )      #\\
\verb#                                                       #\\
\verb#*  If no quality name information was found, annul the #\\
\verb#*  error. New quality names information will be set up #\\
\verb#*  in the "IRAS" NDF extension.                        #\\
\verb#      IF( STATUS .EQ. IRQ__NOQNI ) THEN                #\\
\verb#         CALL ERR_ANNUL( STATUS )                      #\\
\verb#                                                       #\\
\verb#*  If the "IRAS" extension does not exist, create it.  #\\
\verb#         CALL NDF_XSTAT( NDFIN, 'IRAS', FOUND, STATUS )#\\
\verb#         IF( .NOT. FOUND ) THEN                        #\\
\verb#            CALL NDF_XNEW( NDFIN, 'IRAS', 'IRAS', 0, 0,#\\
\verb#     :                     XLOC, STATUS )              #\\
\verb#            CALL DAT_ANNUL( XLOC, STATUS )             #\\
\verb#         END IF                                        #\\
\verb#                                                       #\\
\verb#*  Create a new structure to hold quality information  #\\
\verb#*  in the "IRAS" NDF extension.                        #\\
\verb#         CALL IRQ_NEW( NDFIN, 'IRAS', LOCS, STATUS )   #\\
\verb#      END IF                                           #\\
\verb#                                                       #\\
\verb#*  Attempt to add the quality name "SATURATED" to the  #\\
\verb#*  NDF. If the name already exists an error will be    #\\
\verb#*  reported.                                           #\\
\verb#      CALL IRQ_ADDQN( LOCS, 'SATURATED', .FALSE.,      #\\
\verb#     :  'Pixels with value greater than 10.0', STATUS )#\\
\verb#                                                       #\\
\verb#*  Get a temporary NDF which is the same shape as the  #\\
\verb#*  input NDF.                                          #\\
\verb#      CALL NDF_BOUND( NDFIN, NDF__MXDIM, LBND, UBND,   #\\
\verb#     :                NDIM, STATUS )                   #\\
\verb#      CALL NDF_TEMP( PLACE, STATUS )                   #\\
\verb#      CALL NDF_NEW( '_REAL', NDIM, LBND, UBND, PLACE,  #\\
\verb#     :              NDF2, STATUS )                     #\\
\verb#                                                       #\\
\verb#*  Map the DATA array, initialising its contents to    #\\
\verb#*  zero.                                               #\\
\verb#      CALL NDF_MAP( NDF2, 'DATA', '_REAL',             #\\
\verb#     :              'WRITE/ZERO', PNT2, NEL, STATUS )  #\\
\verb#                                                       #\\
\verb#*  Map the DATA array of the input NDF.                #\\
\verb#      CALL NDF_MAP( NDFIN, 'DATA', '_REAL', 'READ',    #\\
\verb#     :              PNT, NEL, STATUS )                 #\\
\verb#                                                       #\\
\verb#*  Set pixels bad in the temporary array which         #\\
\verb#*  correspond to pixels greater than 10.0 in the input #\\
\verb#*  NDF.                                                #\\
\verb#      CALL MASKIT( %VAL(PNT), %VAL(PNT2), NEL, 10.0,   #\\
\verb#     :             STATUS )                            #\\
\verb#                                                       #\\
\verb#*  Assign the quality SATURATED to all the pixels which#\\
\verb#*  are bad in the temporary NDF.                       #\\
\verb#      CALL IRQ_SETQM( LOCS, .TRUE., 'SATURATED', NEL,  #\\
\verb#     :               %VAL( PNT2 ), STATUS )            #\\
\verb#                                                       #\\
\verb#*  Release the resources used by IRQ.                  #\\
\verb#      CALL IRQ_RLSE( LOCS, STATUS )                    #\\
\verb#                                                       #\\
\verb#*  End the NDF context.                                #\\
\verb#      CALL NDF_END( STATUS )                           #\\

\end{tabbing}
\end{quote}

\subsection {Finding pixels which satisfy a quality expression}
This example produces a copy of an input NDF in which all pixels which do not 
satisfy the quality expression ``.NOT.( BACKGROUND .OR. SATURATED ) '' are
set bad.


\begin{quote}
\begin{tabbing} 

\verb#*  Attempt to locate any quality name information in   #\\
\verb#*  the input NDF. If no quality name information is    #\\
\verb#*  found, then an error is reported.                   #\\
\verb#      CALL IRQ_FIND( NDFIN, LOCS, XNAME, STATUS )      #\numcir{1}\\
\verb#                                                       #\\
\verb#*  Attempt to compile the quality expression.          #\\
\verb#      QEXP = '.NOT. ( SATURATED .OR. BACKGROUND )'     #\\
\verb#      CALL IRQ_COMP( LOCS, IRQ__QNREF, .TRUE., QEXP,   #\numcir{2}\\
\verb#     :             UNDEF, NUNDEF, ERRPNT, IDQ, STATUS )#\\
\verb#                                                       #\\
\verb#*  Produce a temporary copy of the input NDF.          #\\
\verb#      CALL NDF_TEMP( PLACE, STATUS )                   #\\
\verb#      CALL NDF_COPY( NDFIN, PLACE, NDF2, STATUS )      #\\
\verb#                                                       #\\
\verb#*  Map the DATA array for UPDATE access.               #\\
\verb#      CALL NDF_MAP( NDF2, 'DATA', '_REAL', 'UPDATE',   #\\
\verb#     :              PNT2, NEL, STATUS )                #\\
\verb#                                                       #\\
\verb#*  Find all pixels which do not satisfy the quality    #\\
\verb#*  expression and set them bad in the NDF copy.        #\\
\verb#      CALL IRQ_SBAD( IDQ, .FALSE., NEL, %VAL( PNT2 ),  #\\
\verb#     :               ALLBAD, NOBAD, STATUS )           #\\
\verb#                                                       #\\
\verb#*  Unmap the NDF copy.                                 #\\
\verb#      CALL NDF_UNMAP( NDFIN, 'DATA', STATUS )          #\\
\verb#                                                       #\\
\verb#*  Close down the IRQ identifier system.               #\\
\verb#      CALL IRQ_CLOSE( STATUS )                         #\\
\verb#                                                       #\\
\verb#*  Release the resources used by IRQ.                  #\\
\verb#      CALL IRQ_RLSE( LOCS, STATUS )                    #\\

\end{tabbing}
\end{quote}

Programming notes:

\begin{enumerate}

\item NDFIN is the NDF identifier for the input NDF and should have been
obtained previously. If the NDF contains no quality name information, then
an error will be reported by IRQ\_FIND.

\item The quality expression must be assigned to a character variable, since the
call to IRQ\_COMP updates the expression by removing leading blanks, and
converting the expression to upper case. An access violation would result if a
literal string were supplied as an argument, instead of a character variable. If
either of the quality names ``SATURATED'' or ``BACKGROUND'' is not defined in
the input NDF, then an error will be reported by IRQ\_COMP, and the undefined
names will be returned in the UNDEF array. 

\end{enumerate}

\newpage
\section {Example IRQ applications}
\label {APP:APPS}
\sstroutine{
   QUALTOBAD
}{
   Set pixels bad on the basis of quality
}{
   \sstdescription{
      A quality name is a text string used to describe a certain quality
      held by selected pixels in an NDF (see application SETQUAL).
      Several such names can be defined within an NDF. This application
      selects pixels within the input NDF which have a given
      combination of qualities, and produces an output copy of the
      input NDF in which all such pixels are set to the `bad{\tt '} value in
      the DATA component (any VARIANCE values are left unchanged).

      The combination of qualities required of the pixels which are to
      be set bad in the output is specified by the `quality expression{\tt '}
      given by parameter QEXP. The simplest form of quality expression
      consists of a single quality name. A quality expression of
      {\tt "}SATURATED{\tt "} would cause all input pixels holding the quality
      defined by the quality name {\tt "}SATURATED{\tt "} to be set `bad{\tt '} in the
      output NDF.

      More complex quality expressions can be constructed by joining
      together several quality names in a logical expression using the
      logical operators .AND. .OR. .EQV. .NOT. .XOR. (the delimiting
      dots must be included). In addition quality expressions can
      contain the .TRUE. and .FALSE. logical constants. An example of
      such an expression could be SOURCE .AND. .NOT. SATURATED.
      Parentheses may be used as in a normal arithmetic expression, eg
      ( SOURCE\_A .OR. SOURCE\_B ) .AND. .NOT. SATURATED. If the given
      quality expression contains a syntax error, or refers to any
      quality names which are not defined in the input NDF, a message
      is given identifying the error, and the user is reprompted for a
      new expression.

      The quality expression {\tt "}ANY{\tt "} is used to specify all input pixels
      irrespective of quality (an unlikely requirement in this
      particular application, since it would cause all output pixels to
      be set `bad{\tt '}!).
   }
   \sstusage{
      QUALTOBAD IN OUT QEXP
   }
   \sstparameters{
      \sstsubsection{
         IN = NDF (Read)
      }{
         The input NDF.
      }
      \sstsubsection{
         OUT = NDF (Write)
      }{
         The output NDF. This is a copy of the input NDF. All pixels in
         the DATA component for which the corresponding QUALITY value
         satisfies the given quality expression are set to the `bad{\tt '}
         value. All quality information is propagated from the input to
         the output without change. VARIANCE is also left unchanged.
      }
      \sstsubsection{
         QEXP = LITERAL (Read)
      }{
         The quality expression. Pixels in the input NDF which have
         quality values matching this expression are set `bad{\tt '} in the
         output NDF. If the given quality expression contains a syntax
         error, or refers to any quality names which are not defined in
         the input NDF, a message is given identifying the error, and
         the user is reprompted for a new expression.
      }
   }
   \sstexamples{
      \sstexamplesubsection{
         QUALTOBAD M51 M51\_CLEAN {\tt "}SATURATED.OR.GLITCH{\tt "}
      }{
         This example produces an output NDF named M51\_CLEAN which is a
         copy of the input NDF named M51, except that those input pixels
         which hold either of the qualities {\tt "}SATURATED{\tt "} or {\tt "}GLITCH{\tt "} are
         set to `bad{\tt '}.
      }
   }
}
\newpage
\sstroutine{
   REMQUAL
}{
   Remove a specified quality definition from an NDF
}{
   \sstdescription{
      A quality name is a text string used to describe a certain quality
      held by selected pixels in an NDF (see application SETQUAL). This
      application allows the user to remove from an NDF any quality name
      definitions which are no longer required, thus releasing space for
      new quality names to be added using SETQUAL. In addition, this
      application can be used to remove all quality names information
      from an NDF if, for any reason, that information has become
      corrupted.
   }
   \sstusage{
      REMQUAL IN QNAMES
   }
   \sstparameters{
      \sstsubsection{
         IN = NDF (Read and Write)
      }{
         The input NDF in which the quality information is stored.
      }
      \sstsubsection{
         QNAMES( ) = LITERAL (Read)
      }{
         An array of up to 10 quality names to be removed from the
         input NDF.  If any quality name is not defined in the NDF,
         then a warning message is given but the application continues
         to remove any other specified quality names. If the string ANY
         is specified, then all defined quality names are removed. If
         no defined quality names remain, the structure used to store
         quality name information is deleted. This feature can be
         used to get rid of corrupted quality name information.
      }
   }
   \sstexamples{
      \sstexamplesubsection{
         REMQUAL M51 ANY
      }{
         This example will remove all defined quality names from the NDF
         named M51.
      }
   }
}
\newpage
\sstroutine{
   SETQUAL
}{
   Assign a specific quality to selected pixels of an NDF
}{
   \sstdescription{
      This applications associates a `quality name{\tt '}, with selected
      pixels in an NDF. Quality names are text strings which describe
      some quality of the pixels to which it is assigned, examples may
      be {\tt "}SATURATED{\tt "}, {\tt "}BACKGROUND{\tt "}, {\tt "}COSMIC RAY{\tt "}, etc. These quality
      names may be used by later applications to decide which pixels to
      operate on. For instance, a later application may be told to
      operate only on those pixels which have a quality satisfying the
      expression {\tt "}.NOT. SATURATED{\tt "}. The use of these quality names to
      flag conditions is non-destructive, i.e. the original data values
      are left unaltered. This is not true if the `bad value{\tt '} technique
      is used to flag such conditions, as the original data value is
      overwritten with the `bad{\tt '} value.

      The user specifies the quality name to be assigned to selected
      pixels within the NDF (see parameter QNAME). If this name is not
      currently defined within the NDF then a definition of the name is
      added to the NDF.

      The user selects the pixels which are to have their quality
      modified either by giving a `mask{\tt '} NDF, or a text file containing
      a list of pixel indices (see parameters SELECT, MASK and LIST).
      In the first case, the `bad{\tt '} pixels in the mask NDF define the
      selected pixels. In the second case, the pixel indices stored in
      the text file define the selected pixels.

      The exact operation to be performed on all the selected pixels is
      specified by parameter FUNCTION. Options exists to assign or
      remove the quality to or from all selected or un-selected pixels.
      When deciding which function is required, it must be remembered
      that some pixels may already hold the quality specified by QNAME
      (if the name was already defined within the NDF). Thus, for
      instance, assigning a quality to selected pixels does not of
      itself ensure that none of the un-selected pixel do not hold the
      quality as well. To do this, the quality must be removed from
      un-selected pixels as well as being assigned to the selected
      pixels.
   }
   \sstusage{
      SETQUAL IN QNAME [COMMENT] [MASK] [LIST]
   }
   \sstparameters{
      \sstsubsection{
         IN = NDF (Read and Write)
      }{
         The input NDF in which the quality information is to be stored.
      }
      \sstsubsection{
         QNAME = LITERAL (Read)
      }{
         A quality name of up to 15 characters. Leading spaces (but not
         embedded spaces) are ignored. The name is converted to upper
         case before being used. Quality names must not contain the {\tt "}.{\tt "}
         (full stop) character.  If the supplied name does not already
         exists in the NDF, then information defining the name is added
         to an NDF extension (see parameter XNAME).  There is a limit
         to the number of different quality names which can be stored
         in an NDF. If the limit would be exceeded by including the
         supplied name, an error is reported.  The user can then use
         applications SHOWQUAL and REMQUAL to remove quality names from
         the NDF, to make room for the new name.
      }
      \sstsubsection{
         COMMENT = LITERAL (Read)
      }{
         A comment to store with the quality name. This can be used as
         a reminder of the features shared by pixels which are assigned
         this quality. The comment can be up to 50 characters long. This
         parameter is only prompted for if the NDF does not already
         contain a definition of the quality name supplied by parameter
         QNAME.
      }
      \sstsubsection{
         MASK = NDF (Read)
      }{
         A mask NDF which can be used to define the pixels within the
         input NDF to be {\tt "}selected{\tt "} (see parameter FUNCTION). The mask
         should be aligned pixel-for-pixel with the input NDF.  Pixels
         which are bad (i.e. equal to the {\tt "}magic value{\tt "}) in the mask
         NDF are {\tt "}selected{\tt "}. The quality of any pixels in the input NDF
         which lie outside the bounds of the mask NDF is left
         unaltered.  This parameter is only prompted for if the
         parameter SELECT is given the value MASK.
      }
      \sstsubsection{
         LIST = FILENAME (Read)
      }{
         A text file which can be used to define the pixels within the
         input NDF to be {\tt "}selected{\tt "} (see parameter FUNCTION). Each
         record in the text file should contain the coordinates of a
         single pixel to be selected. Records beginning with a {\tt "}\#{\tt "}
         character are ignored. LIST is only prompted for if parameter
         SELECT is given the value LIST.
      }
      \sstsubsection{
         SELECT = LITERAL (Read)
      }{
         This parameter determines how the pixels are selected, and can
         take the values MASK or LIST (see parameters MASK and LIST).
         [MASK]
      }
      \sstsubsection{
         FUNCTION = LITERAL (Read)
      }{
         This parameter specifies what function is to be performed on
         the {\tt "}selected{\tt "} pixels (see parameters SELECT, MASK and LIST).
         It can take any of the values {\tt "}HS{\tt "}, {\tt "}HU{\tt "}, {\tt "}NS{\tt "}, {\tt "}NU{\tt "}, {\tt "}HS$+$NU{\tt "}
         or {\tt "}HU$+$NS{\tt "}

         HS - Ensure that the quality specified by QNAME is Held by the
              Selected pixels. The quality of un-selected pixels is left
              un-changed.

         HU - Ensure that the quality specified by QNAME is Held by the
              Un-selected pixels. The quality of selected pixels is left
              un-changed.

         NS - Ensure that the quality specified by QNAME is Not held by
              the Selected pixels. The quality of un-selected pixels is
              left un-changed.

         NU - Ensure that the quality specified by QNAME is Not held by
              the Un-selected pixels. The quality of selected pixels is
              left un-changed.

         HS$+$NU - Ensure that the quality specified by QNAME is Held by
              the Selected pixels and Not held by the Un-selected
              pixels.

         HU$+$NS - Ensure that the quality specified by QNAME is Held by
              the Un-selected pixels and Not held by the Selected
              pixels.
                                                           [HS$+$NU]
      }
      \sstsubsection{
         XNAME = LITERAL (Read)
      }{
         If no previous quality names have been stored in the input NDF
         then this parameter should be assigned the name of an NDF
         extension in which to store the supplied quality name. If
         necessary, the extension is created (see parameter XTYPE). If
         the input NDF already contains some quality names then the new
         quality name is put in the same extension as the other names.
         [QUALITY\_NAMES]
      }
      \sstsubsection{
         XTYPE = LITERAL (Read)
      }{
         If the input NDF contains no previous quality name
         information, and if the NDF extension specified by XNAME was
         not found in the input NDF, then a new scalar extension is
         created with name given by XNAME and HDS data type given by
         XTYPE. The run time default is to give the extension a type
         identical to its name.[]
      }
   }
   \sstexamples{
      \sstexamplesubsection{
         SETQUAL M51 SATURATED {\tt "}Saturated pixels in M51{\tt "} M51\_CUT
      }{
         This example creates a quality name {\tt "}SATURATED{\tt "} (described by
         the comment {\tt "}Saturated pixels in M51{\tt "}). The quality name is
         assigned to all pixels within the NDF named M51, which
         correspond to {\tt "}bad{\tt "} pixels in the NDF named M51\_CUT. The
         quality is removed from all other pixels.  Note, if the
         quality name {\tt "}SATURATED{\tt "} is already defined within M51, then
         this definition is used. In this case, a warning message is
         displayed, containing the descriptive comment which was stored
         with the name when it was originally defined.
      }
      \sstexamplesubsection{
         SETQUAL M51 SOURCE\_A SELECT=LIST LIST=SOURCE\_A.LIS FUNCTION=NU
      }{
         This example ensures that pixels within the NDF named M51
         which are not included in the list of pixel indices held in
         text file SOURCE\_A.LIS, do not have the quality {\tt "}SOURCE\_A{\tt "}.
         The quality of all other pixels is left unchanged.
      }
   }
   \sstnotes{
      \sstitemlist{

         \sstitem
         All the quality names which are currently defined within an
         NDF can be listed by application SHOWQUAL.  Quality name
         definitions can be removed from an NDF using application REMQUAL.
      }
   }
}
\newpage
\sstroutine{
   SHOWQUAL
}{
   Display the quality names defined in an NDF
}{
   \sstdescription{
      A quality name is a text string used to describe a certain quality
      held by selected pixels in an NDF (see application SETQUAL). This
      application displays a list of all the quality names currently
      defined within the input NDF, together with the descriptive
      comments which were stored with the quality name when the name
      was originally defined. An option exists for also displaying the
      number of pixels which hold each quality. Selecting this option
      adds considerably to the run time.
   }
   \sstusage{
      SHOWQUAL IN [COUNT]
   }
   \sstparameters{
      \sstsubsection{
         IN = NDF (Read)
      }{
         The input NDF.
      }
      \sstsubsection{
         COUNT = \_LOGICAL (Read)
      }{
         If true, then a count is made of all pixels in the input NDF
         which have each defined quality. These figures are displayed
         in parentheses between the quality name and associated
         comment.  This option adds significantly to the run time. [NO]
      }
   }
   \sstexamples{
      \sstexamplesubsection{
         SHOWQUAL M51 YES
      }{
         This example displays all the quality names currently defined
         for the NDF named M51, together with the number of pixels
         holding each quality.
      }
   }
}
\newpage

\section{Templates for IRQ Routines Within the VAX LSE Editor}
The STARLSE package (see SUN/105) provides facilities for initialising the VAX
Language Sensitive Editor (LSE) to simplify the generation of Fortran
code conforming to the Starlink programming standard (see SGP/16). One of the
facilities provided by LSE is the automatic production of argument lists for
subroutine calls. Templates for all the subroutines in the IRQ package can be 
made available within LSE by performing the following steps (within LSE):
\begin{enumerate}
\item Issue the LSE command GOTO FILE/READ IRQ\_DIR:IRQ.LSE
\item Issue the LSE command DO
\item Issue the LSE command DELETE BUFFER
\item Move to a buffer holding a .FOR of a .GEN file in the usual way.
\item IRQ subroutine templates are then available by typing in the name of an 
IRQ subroutine (or an abbreviation) and expanding it (CTRL-E).
\item Help on the subroutine and its arguments can be obtained by placing the
cursor at a point in the buffer at which the subroutine name has been entered 
and pressing GOLD-PF2.
\end{enumerate}

\section {Packages Called by IRQ}
IRQ\_ makes calls to the following packages:
\begin {description}
\item [CHR\_] - The CHR character handling package; see SUN/40.
\item [CMP\_] - HDS; see SUN/92.
\item [DAT\_] - HDS; see SUN/92.
\item [ERR\_] - The Starlink error reporting package; see SUN/104.
\item [MSG\_] - The Starlink message reporting package; see SUN/104.
\item [NDF\_] - The NDF access package; see SUN/33.
\item [VEC\_] - The PRIMDAT package; see SUN/39.
\end{description}

Access to these packages, together with packages called from within these 
packages, is necessary to use IRQ. 

\section {IRQ Error Codes}
\label {APP:ERRORS}
IRQ routines can return any STATUS value generated by the subroutine packages 
which it calls. In addition it can return the following IRQ-specific values 

\begin{description}
\item {\bf IRQ\_\_BADBT}\\ 
Bit value outside range [1,8] supplied.
\item {\bf IRQ\_\_BADDM}\\
Incorrect NDF dimensions supplied.
\item {\bf IRQ\_\_BADNM}\\ 
An illegal quality name has been given.
\item {\bf IRQ\_\_BADQN}\\ 
Incomplete QUALITY\_NAMES structure found.
\item {\bf IRQ\_\_BADSL}\\ 
Invalid slot number supplied.
\item {\bf IRQ\_\_BADST}\\ 
Incomplete slot structure found.
\item {\bf IRQ\_\_BADSY}\\ 
Unrecognised logical operator or constant in quality expression.
\item {\bf IRQ\_\_CMPLX}\\ 
Too many symbols in quality expression.
\item {\bf IRQ\_\_INCOM}\\ 
Supplied vector has different size to the NDF
\item {\bf IRQ\_\_INTER}\\ 
Internal IRQ error (report to maintainer of IRQ).
\item {\bf IRQ\_\_INVID}\\ 
Invalid IRQ identifier supplied.
\item {\bf IRQ\_\_IVNDF}\\ 
Invalid NDF identifier found.
\item {\bf IRQ\_\_LSHRT}\\ 
Character variable too short.
\item {\bf IRQ\_\_MSDOT}\\ 
Missing delimiter "." in quality expression.
\item {\bf IRQ\_\_MSOPD}\\ 
Missing operand in quality expression.
\item {\bf IRQ\_\_MSOPT}\\ 
Missing or invalid operator in quality expression.
\item {\bf IRQ\_\_MSPAR}\\ 
Unpaired parentheses in quality expression.
\item {\bf IRQ\_\_MULT }\\ 
More than one structure found holding quality names information.
\item {\bf IRQ\_\_NOMOR}\\
No identifiers left for compiled quality expressions.
\item {\bf IRQ\_\_NOOPS}\\ 
No operands can be found.
\item {\bf IRQ\_\_NOQNM}\\ 
A quality name could not be found.
\item {\bf IRQ\_\_NOQNI}\\ 
No quality names structure found in the NDF.
\item {\bf IRQ\_\_NOSPA}\\ 
Can't reclaim space in an instruction array.
\item {\bf IRQ\_\_NOWRT}\\ 
Write access to the NDF is unavailable.
\item {\bf IRQ\_\_QBAD }\\ 
BAD values exists in the QUALITY component.
\item {\bf IRQ\_\_QEXPL}\\ 
Too many characters in quality expression.
\item {\bf IRQ\_\_QIEXS}\\ 
Extension already contains a quality names structure.
\item {\bf IRQ\_\_QNEXS}\\ 
Quality name is already defined.
\item {\bf IRQ\_\_QLONG}\\ 
Quality name too long.
\item {\bf IRQ\_\_QREFS}\\ 
Too many quality names in quality expression.
\item {\bf IRQ\_\_QUNDF}\\ 
QUALITY component is undefined.
\item {\bf IRQ\_\_RDONL}\\ 
An attempt has been made to remove a read-only quality name.
\item {\bf IRQ\_\_STKOV}\\ 
Evaluation stack overflow.
\item {\bf IRQ\_\_STKUN}\\ 
Evaluation stack underflow.
\item {\bf IRQ\_\_XBITS}\\ 
No bits left in QUALITY component.
\end{description}

\section {Changes Introduced in the Current Version of this Document}
\label {SEC:CHANGES}

Changes introduced in version 5 of ID6:
\begin {enumerate}
\item Routine IRQ\_RWQN added.
\end {enumerate}

Changes introduced in version 4 of ID6:
\begin {enumerate}
\item Description of the use of IRQ on UNIX machines included.
\item STARLSE templates and help available for IRQ subroutines.
\item It is no longer necessary to include I90\_PAR before including
IRQ\_ERR.
\item The names of symbolic constants (both errors and parameters) which previously 
were longer than five characters (excluding the IRQ\_\_ prefix) have been
truncated to five characters. The actual {\em values} associated with each name 
have not changed.
\item The VMS version of IRQ is now released in the form of a sharable image 
rather than an object library. Applications linked with the sharable image have 
the advantage that they will not need to be {\em re}-linked when IRQ is upgraded 
in future. Note, the object library has been removed from the IRQ system.
\item The linking and development procedures have been modified to bring them 
into line with Starlink standards (see SSN/8).

\end {enumerate}

\end{document}
