%+
% SUN/203  DX - IBM Data Explorer for Data Visualisation
%
% A C Davenhall (Edinburgh)  27/10/95.
%
% Copyright 1997  Starlink, CCLRC.
%-

\documentstyle[twoside,11pt]{article}
\pagestyle{myheadings}
\newcounter{swapfoot}

%
% Set the DX and SX version numbers.

\newcommand{\DXversion}{3.1~}
\newcommand{\SXversion}{1.1~}

% -----------------------------------------------------------------------------
% ? Document identification
\newcommand{\stardoccategory}  {Starlink User Note}
\newcommand{\stardocinitials}  {SUN}
\newcommand{\stardocsource}    {sun\stardocnumber}
\newcommand{\stardocnumber}    {203.3}
\newcommand{\stardocauthors}   {D.S.~Berry, G.J.~Privett and
A.C.~Davenhall}
\newcommand{\stardocdate}      {15 September 1997}
\newcommand{\stardoctitle}     {SX \& DX --- IBM Data Explorer for \\
Data Visualisation}
% ? End of document identification
% -----------------------------------------------------------------------------

\newcommand{\stardocname}{\stardocinitials /\stardocnumber}
\markboth{\stardocname}{\stardocname}
\setlength{\textwidth}{160mm}
\setlength{\textheight}{230mm}
\setlength{\topmargin}{-2mm}
\setlength{\oddsidemargin}{0mm}
\setlength{\evensidemargin}{0mm}
\setlength{\parindent}{0mm}
\setlength{\parskip}{\medskipamount}
\setlength{\unitlength}{1mm}

% -----------------------------------------------------------------------------
%  Hypertext definitions.
%  ======================
%  These are used by the LaTeX2HTML translator in conjunction with star2html.

%  Comment.sty: version 2.0, 19 June 1992
%  Selectively in/exclude pieces of text.
%
%  Author
%    Victor Eijkhout                                      <eijkhout@cs.utk.edu>
%    Department of Computer Science
%    University Tennessee at Knoxville
%    104 Ayres Hall
%    Knoxville, TN 37996
%    USA

%  Do not remove the %begin{latexonly} and %end{latexonly} lines (used by
%  star2html to signify raw TeX that latex2html cannot process).
%begin{latexonly}
\makeatletter
\def\makeinnocent#1{\catcode`#1=12 }
\def\csarg#1#2{\expandafter#1\csname#2\endcsname}

\def\ThrowAwayComment#1{\begingroup
    \def\CurrentComment{#1}%
    \let\do\makeinnocent \dospecials
    \makeinnocent\^^L% and whatever other special cases
    \endlinechar`\^^M \catcode`\^^M=12 \xComment}
{\catcode`\^^M=12 \endlinechar=-1 %
 \gdef\xComment#1^^M{\def\test{#1}
      \csarg\ifx{PlainEnd\CurrentComment Test}\test
          \let\html@next\endgroup
      \else \csarg\ifx{LaLaEnd\CurrentComment Test}\test
            \edef\html@next{\endgroup\noexpand\end{\CurrentComment}}
      \else \let\html@next\xComment
      \fi \fi \html@next}
}
\makeatother

\def\includecomment
 #1{\expandafter\def\csname#1\endcsname{}%
    \expandafter\def\csname end#1\endcsname{}}
\def\excludecomment
 #1{\expandafter\def\csname#1\endcsname{\ThrowAwayComment{#1}}%
    {\escapechar=-1\relax
     \csarg\xdef{PlainEnd#1Test}{\string\\end#1}%
     \csarg\xdef{LaLaEnd#1Test}{\string\\end\string\{#1\string\}}%
    }}

%  Define environments that ignore their contents.
\excludecomment{comment}
\excludecomment{rawhtml}
\excludecomment{htmlonly}

%  Hypertext commands etc. This is a condensed version of the html.sty
%  file supplied with LaTeX2HTML by: Nikos Drakos <nikos@cbl.leeds.ac.uk> &
%  Jelle van Zeijl <jvzeijl@isou17.estec.esa.nl>. The LaTeX2HTML documentation
%  should be consulted about all commands (and the environments defined above)
%  except \xref and \xlabel which are Starlink specific.

\newcommand{\htmladdnormallinkfoot}[2]{#1\footnote{#2}}
\newcommand{\htmladdnormallink}[2]{#1}
\newcommand{\htmladdimg}[1]{}
\newenvironment{latexonly}{}{}
\newcommand{\hyperref}[4]{#2\ref{#4}#3}
\newcommand{\htmlref}[2]{#1}
\newcommand{\htmlimage}[1]{}
\newcommand{\htmladdtonavigation}[1]{}
\newcommand{\latexhtml}[2]{#1}
\newcommand{\html}[1]{}

%  Starlink cross-references and labels.
\newcommand{\xref}[3]{#1}
\newcommand{\xlabel}[1]{}

%  LaTeX2HTML symbol.
\newcommand{\latextohtml}{{\bf LaTeX}{2}{\tt{HTML}}}

%  Define command to re-centre underscore for Latex and leave as normal
%  for HTML (severe problems with \_ in tabbing environments and \_\_
%  generally otherwise).
\newcommand{\latex}[1]{#1}
\newcommand{\setunderscore}{\renewcommand{\_}{{\tt\symbol{95}}}}
\latex{\setunderscore}

% -----------------------------------------------------------------------------
%  Debugging.
%  =========
%  Remove % on the following to debug links in the HTML version using Latex.

% \newcommand{\hotlink}[2]{\fbox{\begin{tabular}[t]{@{}c@{}}#1\\\hline{\footnotesize #2}\end{tabular}}}
% \renewcommand{\htmladdnormallinkfoot}[2]{\hotlink{#1}{#2}}
% \renewcommand{\htmladdnormallink}[2]{\hotlink{#1}{#2}}
% \renewcommand{\hyperref}[4]{\hotlink{#1}{\S\ref{#4}}}
% \renewcommand{\htmlref}[2]{\hotlink{#1}{\S\ref{#2}}}
% \renewcommand{\xref}[3]{\hotlink{#1}{#2 -- #3}}
%end{latexonly}
% -----------------------------------------------------------------------------
% ? Document specific \newcommand or \newenvironment commands.

%
%  Commands and environments used to produce module documentation in the
%  same style as the DX User Reference document.

%\input{sx_coms.tex}
\newlength{\ModIndent}
\setlength{\ModIndent}{1.5in}

\newcommand{\myhspace}[1]{\hspace*{#1}}


%--------------------------------------------------------------------------
%  \Macro - displays heading (macro name and classification), the "Name"
%           section, and the "Syntax" section.
%
%            Args: 1 - Macro name
%                  2 - classification
%                  3 - Brief description
%                  4 - comma separated list of output parameter names
%                  5 - comma separated list of input parameter names

\newcommand{\Macro}[5]{
   \newpage
   \makebox[0.45\textwidth][l]{{\large\bf{#1}}\hspace{5mm}({\em macro})}
   \makebox[0.45\textwidth][r]{\normalsize\bf{#2}}\\
   \vspace*{5mm}\\
   {\normalsize\bf Name}\\
   \hspace*{\ModIndent}{\small{#1}} -- \parbox[t]{4in}{\small{#3}}\\
   \vspace*{5mm}\\
   {\normalsize\bf Syntax}\\
   \hspace*{\ModIndent}{\small{\bf {#4}} {\tt = {#1}}(\parbox[t]{3.2in}{{\bf{#5}});}}\\
}



%--------------------------------------------------------------------------
%  \Module - displays heading (module name and classification), the "Name"
%            section, and the "Syntax" section.
%
%            Args: 1 - Module name
%                  2 - classification
%                  3 - Brief description
%                  4 - comma separated list of output parameter names
%                  5 - comma separated list of input parameter names

\newcommand{\Module}[5]{
   \newpage
   \makebox[0.45\textwidth][l]{\large\bf{#1}}
   \makebox[0.45\textwidth][r]{\normalsize\bf{#2}}\\
   \vspace*{5mm}\\
   {\normalsize\bf Name}\\
   \hspace*{\ModIndent}{\small{#1}} -- \parbox[t]{4in}{\small{#3}}\\
   \vspace*{5mm}\\
   {\normalsize\bf Syntax}\\
   \hspace*{\ModIndent}{\small{\bf {#4}} {\tt = {#1}}(\parbox[t]{3.2in}{{\bf{#5}});}}\\
}




%--------------------------------------------------------------------------
%  \InputItem - Adds an input to the table in the "Inputs" section. This
%               command should be used within a "ModuleInputs" environment.
%
%            Args: 1 - parameter name
%                  2 - type
%                  3 - default
%                  4 - description
%                  5 - a font size command (eg \Large). If blank, then
%                      \footnotesize is used.

\newcommand{\InputItem}[5]{
   \parbox{0.65in}{\footnotesize{#5}\vspace*{1mm}\raggedright {\bf{#1}}\vspace*{1mm}}&
   \parbox{0.8in} {\footnotesize{#5}\vspace*{1mm}\raggedright {#2}\vspace*{1mm}}&
   \parbox{0.85in}{\footnotesize{#5}\vspace*{1mm}\raggedright {#3}\vspace*{1mm}}&
   \parbox{1.8in} {\footnotesize{#5}\vspace*{1mm}\raggedright {#4}\vspace*{1mm}}\\ \hline
}





%--------------------------------------------------------------------------
%  \begin{ModuleInputs} - Creates the table of module inputs in the
%  \end{ModuleInputs}     "Inputs" section. Add input parameters using the
%                         \InputItem command.
%

\newenvironment{ModuleInputs}{
    {\normalsize\bf Inputs}\\
    \hspace*{\ModIndent}
    \begin{tabular}{|l|l|l|l|} \hline
       \InputItem{Name}{\bf Type}{\bf Default}{\bf Description}{\small}
}{
    \end{tabular}
    \vspace*{3mm}
}




%--------------------------------------------------------------------------
%  \OutputItem - Adds an output to the table in the "Outputs" section. This
%               command should be used within a "ModuleOutputs" environment.
%
%            Args: 1 - parameter name
%                  2 - type
%                  3 - description
%                  4 - a font size command (eg \Large). If blank, then
%                      \footnotesize is used.

\newcommand{\OutputItem}[4]{
   \parbox{1.1in}{\footnotesize{#4}\vspace*{1mm}\raggedright {\bf{#1}}\vspace*{1mm}}&
   \parbox{1.1in}{\footnotesize{#4}\vspace*{1mm}\raggedright {#2}\vspace*{1mm}}&
   \parbox{2.05in}{\footnotesize{#4}\vspace*{1mm}\raggedright {#3}\vspace*{1mm}}\\ \hline
}





%--------------------------------------------------------------------------
%  \begin{ModuleOutputs} - Creates the table of module outputs in the
%  \end{ModuleOutputs}     "Outputs" section. Add output parameters using the
%                         \OutputItem command.
%

\newenvironment{ModuleOutputs}{
    {\normalsize\bf Outputs}\\
    \hspace*{\ModIndent}
    \begin{tabular}{|l|l|l|l|} \hline
       \OutputItem{Name}{\bf Type}{\bf Description}{\normalsize}
}{
    \end{tabular}
}




%--------------------------------------------------------------------------
%  \ModuleSection - displays plain text for the specified section.
%
%            Args: 1 - The section name
%            Args: 2 - The text for the section

\newcommand{\ModuleSection}[2]{
    \begin{list}{}{
                    \setlength{\rightmargin}{0in}
                    \setlength{\leftmargin}{\ModIndent}
                    \setlength{\labelwidth}{\ModIndent}
                    \setlength{\labelsep}{0in}
                  }
       \nopagebreak
       \item[{\normalsize\bf{#1}}\hfill] \makebox[1in]{} \par {\small{#2}}
    \end{list}
}




%--------------------------------------------------------------------------
%  \Example - displays a script language example
%
%            Args: 1 - The text of the example, split into lines using "\\"

\newcommand{\Example}[1]{
    \begin{list}{}{\setlength{\leftmargin}{5mm}}
    \item
    {\footnotesize
    {\tt{#1}}
    }
    \end{list}
}




%--------------------------------------------------------------------------
%  \Param - formats a parameter name
%
%            Args: 1 - parameter name

\newcommand{\Param}[1]{\hspace{0.5mm}{\bf{#1}}\hspace{0.5mm}}


%--------------------------------------------------------------------------
%  \Modnam - formats a module name
%
%            Args: 1 - module name

\newcommand{\Modnam}[1]{\verb+{#1}+}



%--------------------------------------------------------------------------
%
%  \DemoNet - Display a graphical representation of a section of a
%             demonstration network.
%
%            Args: 1 - Demo name
%                  2 - section number
%                  3 - Latex commands defining the network picture

\newcommand{\DemoNet}[3]{
\newpage
\begin{center}
{\large \bf Demonstration ``{#1}'' - section {#2}}
\end{center}
\nopagebreak

\vspace*{1cm}
\tiny
{#3}
\normalsize
}

%--------------------------------------------------------------------------
%
%  \DemoDesc - Display textual information describing a section of a
%              demonstration network.
%
%            Args: 1 - Demo name
%                  2 - section number
%                  3 - brief purpose
%                  4 - overview
%                  5 - List of \DemoMod commands describing the function
%                      of each module in the network section

\newcounter{modnum}
\newcommand{\DemoDesc}[5]{
\newpage
{\large \bf Demonstration ``{#1}'' - section {#2}}
{
\vspace*{5mm}
\begin{description}

\item [Purpose:]\mbox{}

\nopagebreak{#3}

\item [Description:]\mbox{}

\nopagebreak{#4}

\item [Modules:]\mbox{}

\nopagebreak
\begin{list}%
   {\arabic{modnum}\hfill--}{\usecounter{modnum}
                           \setlength{\leftmargin}{\labelsep}
                           \addtolength{\leftmargin}{\labelwidth}
                          }{#5}\end{list}

\end{description}
}
}

%--------------------------------------------------------------------------
%
%  \DemoMod - Format list text describing the purpose of a single module.
%
%            Args: 1 - Module name
%                  2 - purpose

\newcommand{\DemoMod}[2]{
\item \Modnam{{#1}}: {#2}
}

\newcommand{\DemoWidget}[1]{
   \mbox{$<$}{\small{\bf{#1}}}\mbox{$>$}
}
\newcommand{\Cpanel}[1]{ ``{\small {\sf {#1}}}'' }
\newcommand{\DemoPanel}[1]{\item [{\sf {#1}}]\mbox{}\par\vspace{-1mm}}

%=======================================================================
%
%  Modify the above command definitions when producing the html version of
%  the document.
\begin{htmlonly}

\newcommand{\myhspace}[1]{
\begin{rawhtml}
&#160 &#160 &#160
\end{rawhtml}
}

%--------------------------------------------------------------------------
%  \Macro - displays heading (macro name and classification), the "Name"
%           section, and the "Syntax" section.
%
%            Args: 1 - Macro name
%                  2 - classification
%                  3 - Brief description
%                  4 - comma separated list of output parameter names
%                  5 - comma separated list of input parameter names

\newcommand{\Macro}[5]{
   \subsection{\label{#1}\xlabel{#1}{#1} ~~~ {\em (a macro)}}
   \begin{description}
   \item [{\bf Name:}] {#1} -- {#3}
   \end{description}
   \begin{description}
   \item [{\bf Category:}] {#2}
   \end{description}
   \begin{description}
   \item [{\bf Syntax:}] {#4} = {#1}( {#5} );
   \end{description}
}



%--------------------------------------------------------------------------
%  \Module - displays heading (module name and classification), the "Name"
%            section, and the "Syntax" section.
%
%            Args: 1 - Module name
%                  2 - classification
%                  3 - Brief description
%                  4 - comma separated list of output parameter names
%                  5 - comma separated list of input parameter names

\newcommand{\Module}[5]{
   \subsection{\label{#1}\xlabel{#1}{#1}}
   \begin{description}
   \item [{\bf Name:}] {#1} -- {#3}
   \end{description}
   \begin{description}
   \item [{\bf Category:}] {#2}
   \end{description}
   \begin{description}
   \item [{\bf Syntax:}] {#4} = {#1}( {#5} );
   \end{description}
}




%--------------------------------------------------------------------------
%  \InputItem - Adds an input to the table in the "Inputs" section. This
%               command should be used within a "ModuleInputs" environment.
%
%            Args: 1 - parameter name
%                  2 - type
%                  3 - default
%                  4 - description
%                  5 - ignored

\newcommand{\InputItem}[5]{
   \item [{#1} = {#2} (default: {#3}) ] {#4}
}





%--------------------------------------------------------------------------
%  \begin{ModuleInputs} - Creates the table of module inputs in the
%  \end{ModuleInputs}     "Inputs" section. Add input parameters using the
%                         \InputItem command.
%

\newenvironment{ModuleInputs}{
    \begin{description}
    \item [{\bf Input Parameters:}]
    \begin{description}
}{
    \end{description}
    \end{description}
}




%--------------------------------------------------------------------------
%  \OutputItem - Adds an output to the table in the "Outputs" section. This
%               command should be used within a "ModuleOutputs" environment.
%
%            Args: 1 - parameter name
%                  2 - type
%                  3 - description
%                  4 - ignored

\newcommand{\OutputItem}[4]{
   \item [{#1} = {#2} ] {#3}
}





%--------------------------------------------------------------------------
%  \begin{ModuleOutputs} - Creates the table of module outputs in the
%  \end{ModuleOutputs}     "Outputs" section. Add output parameters using the
%                         \OutputItem command.
%

\newenvironment{ModuleOutputs}{
    \begin{description}
    \item [{\bf Output Parameters:}]
    \begin{description}
}{
    \end{description}
    \end{description}
}




%--------------------------------------------------------------------------
%  \ModuleSection - displays plain text for the specified section.
%
%            Args: 1 - The section name
%            Args: 2 - The text for the section

\newcommand{\ModuleSection}[2]{
    \begin{description}
    \item [{\bf {#1}:}] {#2}
    \end{description}
}




%--------------------------------------------------------------------------
%  \Example - displays a script language example
%
%            Args: 1 - The text of the example, split into lines using "\\"

\newcommand{\Example}[1]{
    \begin{quote}
    \begin{description}
    {\tt {#1}}
    \end{description}
    \end{quote}
}


%--------------------------------------------------------------------------
%  \Param - formats a parameter name
%
%            Args: 1 - parameter name

\newcommand{\Param}[1]{
{\bf{#1}}
}

%--------------------------------------------------------------------------
%  \Modnam - formats a module name
%
%            Args: 1 - module name

\newcommand{\Modnam}[1]{{\tt #1}}

%--------------------------------------------------------------------------
%
%  \DemoNet - Display a graphical representation of a section of a
%             demonstration network. Note, each network should be
%             held in two gifs called {demo name}_{section number}a.gif
%             and {demo name}_{section number}b.gif. These gifs are
%             displayed one on top of the other.
%
%            Args: 1 - Demo name
%                  2 - section number
%                  3 - Latex commands defining the network picture

\newcommand{\DemoNet}[3]{
\subsubsection{\large \bf Demonstration ``{#1}'' - section {#2}}
\vspace*{1cm}
\footnotesize
{#3}
\normalsize
}

%--------------------------------------------------------------------------
%
%  \DemoDesc - Display textual information describing a section of a
%              demonstration network.
%
%            Args: 1 - Demo name
%                  2 - section number
%                  3 - brief purpose
%                  4 - overview
%                  5 - List of \DemoMod commands describing the function
%                      of each module in the network section

\newcommand{\DemoDesc}[5]{
\vspace*{5mm}
\begin{description}

\item [Purpose:]
\nopagebreak{#3}
\begin{description}
\end{description}

\item [Description:]
\nopagebreak{#4}
\begin{description}
\end{description}

\item [Modules:]
\nopagebreak
\begin{enumerate}
{#5}
\end{enumerate}

\end{description}
}

%--------------------------------------------------------------------------
%
%  \DemoMod - Format list text describing the purpose of a single module.
%
%            Args: 1 - Module name
%                  2 - purpose

\newcommand{\DemoMod}[2]{
\item \Modnam{{\bf {#1}}}: {#2}
}

\newcommand{\DemoWidget}[1]{
   $<${\bf{#1}}$>$ }
}
\newcommand{\Cpanel}[1]{ {\sf ``{#1}''} }
\newcommand{\DemoPanel}[1]{\item [{\sf {#1}}]\mbox{}\par\vspace{-1mm}}

\end{htmlonly}

%+
%  Name:
%     SST.TEX

%  Purpose:
%     Define LaTeX commands for laying out Starlink routine descriptions.

%  Language:
%     LaTeX

%  Type of Module:
%     LaTeX data file.

%  Description:
%     This file defines LaTeX commands which allow routine documentation
%     produced by the SST application PROLAT to be processed by LaTeX and
%     by LaTeX2html. The contents of this file should be included in the
%     source prior to any statements that make of the sst commnds.

%  Notes:
%     The style file html.sty provided with LaTeX2html needs to be used.
%     This must be before this file.

%  Authors:
%     RFWS: R.F. Warren-Smith (STARLINK)
%     PDRAPER: P.W. Draper (Starlink - Durham University)

%  History:
%     10-SEP-1990 (RFWS):
%        Original version.
%     10-SEP-1990 (RFWS):
%        Added the implementation status section.
%     12-SEP-1990 (RFWS):
%        Added support for the usage section and adjusted various spacings.
%     8-DEC-1994 (PDRAPER):
%        Added support for simplified formatting using LaTeX2html.
%     {enter_further_changes_here}

%  Bugs:
%     {note_any_bugs_here}

%-

%  Define length variables.
\newlength{\sstbannerlength}
\newlength{\sstcaptionlength}
\newlength{\sstexampleslength}
\newlength{\sstexampleswidth}

%  Define a \tt font of the required size.
\latex{\newfont{\ssttt}{cmtt10 scaled 1095}}
\html{\newcommand{\ssttt}{\tt}}

%  Define a command to produce a routine header, including its name,
%  a purpose description and the rest of the routine's documentation.
\newcommand{\sstroutine}[3]{
   \goodbreak
   \rule{\textwidth}{0.5mm}
   \vspace{-7ex}
   \newline
   \settowidth{\sstbannerlength}{{\Large {\bf #1}}}
   \setlength{\sstcaptionlength}{\textwidth}
   \setlength{\sstexampleslength}{\textwidth}
   \addtolength{\sstbannerlength}{0.5em}
   \addtolength{\sstcaptionlength}{-2.0\sstbannerlength}
   \addtolength{\sstcaptionlength}{-5.0pt}
   \settowidth{\sstexampleswidth}{{\bf Examples:}}
   \addtolength{\sstexampleslength}{-\sstexampleswidth}
   \parbox[t]{\sstbannerlength}{\flushleft{\Large {\bf #1}}}
   \parbox[t]{\sstcaptionlength}{\center{\Large #2}}
   \parbox[t]{\sstbannerlength}{\flushright{\Large {\bf #1}}}
   \begin{description}
      #3
   \end{description}
}

%  Format the description section.
\newcommand{\sstdescription}[1]{\item[Description:] #1}

%  Format the usage section.
\newcommand{\sstusage}[1]{\item[Usage:] \mbox{}
\\[1.3ex]{\raggedright \ssttt #1}}

%  Format the invocation section.
\newcommand{\sstinvocation}[1]{\item[Invocation:]\hspace{0.4em}{\tt #1}}

%  Format the arguments section.
\newcommand{\sstarguments}[1]{
   \item[Arguments:] \mbox{} \\
   \vspace{-3.5ex}
   \begin{description}
      #1
   \end{description}
}

%  Format the returned value section (for a function).
\newcommand{\sstreturnedvalue}[1]{
   \item[Returned Value:] \mbox{} \\
   \vspace{-3.5ex}
   \begin{description}
      #1
   \end{description}
}

%  Format the parameters section (for an application).
\newcommand{\sstparameters}[1]{
   \item[Parameters:] \mbox{} \\
   \vspace{-3.5ex}
   \begin{description}
      #1
   \end{description}
}

%  Format the examples section.
\newcommand{\sstexamples}[1]{
   \item[Examples:] \mbox{} \\
   \vspace{-3.5ex}
   \begin{description}
      #1
   \end{description}
}

%  Define the format of a subsection in a normal section.
\newcommand{\sstsubsection}[1]{ \item[{#1}] \mbox{} \\}

%  Define the format of a subsection in the examples section.
\newcommand{\sstexamplesubsection}[2]{\sloppy
\item[\parbox{\sstexampleslength}{\ssttt #1}] \mbox{} \vspace{1.0ex}
\\ #2 }

%  Format the notes section.
\newcommand{\sstnotes}[1]{\item[Notes:] \mbox{} \\[1.3ex] #1}

%  Provide a general-purpose format for additional (DIY) sections.
\newcommand{\sstdiytopic}[2]{\item[{\hspace{-0.35em}#1\hspace{-0.35em}:}]
\mbox{} \\[1.3ex] #2}

%  Format the implementation status section.
\newcommand{\sstimplementationstatus}[1]{
   \item[{Implementation Status:}] \mbox{} \\[1.3ex] #1}

%  Format the bugs section.
\newcommand{\sstbugs}[1]{\item[Bugs:] #1}

%  Format a list of items while in paragraph mode.
\newcommand{\sstitemlist}[1]{
  \mbox{} \\
  \vspace{-3.5ex}
  \begin{itemize}
     #1
  \end{itemize}
}

%  Define the format of an item.
\newcommand{\sstitem}{\item}

%% Now define html equivalents of those already set. These are used by
%  latex2html and are defined in the html.sty files.
\begin{htmlonly}

%  sstroutine.
   \newcommand{\sstroutine}[3]{
      \subsection{#1\xlabel{#1}-\label{#1}#2}
      \begin{description}
         #3
      \end{description}
   }

%  sstdescription
   \newcommand{\sstdescription}[1]{\item[Description:]
      \begin{description}
         #1
      \end{description}
      \\
   }

%  sstusage
   \newcommand{\sstusage}[1]{\item[Usage:]
      \begin{description}
         {\ssttt #1}
      \end{description}
      \\
   }

%  sstinvocation
   \newcommand{\sstinvocation}[1]{\item[Invocation:]
      \begin{description}
         {\ssttt #1}
      \end{description}
      \\
   }

%  sstarguments
   \newcommand{\sstarguments}[1]{
      \item[Arguments:] \\
      \begin{description}
         #1
      \end{description}
      \\
   }

%  sstreturnedvalue
   \newcommand{\sstreturnedvalue}[1]{
      \item[Returned Value:] \\
      \begin{description}
         #1
      \end{description}
      \\
   }

%  sstparameters
   \newcommand{\sstparameters}[1]{
      \item[Parameters:] \\
      \begin{description}
         #1
      \end{description}
      \\
   }

%  sstexamples
   \newcommand{\sstexamples}[1]{
      \item[Examples:] \\
      \begin{description}
         #1
      \end{description}
      \\
   }

%  sstsubsection
   \newcommand{\sstsubsection}[1]{\item[{#1}]}

%  sstexamplesubsection
   \newcommand{\sstexamplesubsection}[2]{\item[{\ssttt #1}] #2}

%  sstnotes
   \newcommand{\sstnotes}[1]{\item[Notes:] #1 }

%  sstdiytopic
   \newcommand{\sstdiytopic}[2]{\item[{#1}] #2 }

%  sstimplementationstatus
   \newcommand{\sstimplementationstatus}[1]{
      \item[Implementation Status:] #1
   }

%  sstitemlist
   \newcommand{\sstitemlist}[1]{
      \begin{itemize}
         #1
      \end{itemize}
      \\
   }
%  sstitem
   \newcommand{\sstitem}{\item}

\end{htmlonly}

%  End of "sst.tex" layout definitions.
%.




% ? End of document specific commands
% -----------------------------------------------------------------------------
%  Title Page.
%  ===========
\renewcommand{\thepage}{\roman{page}}
\begin{document}
\thispagestyle{empty}

%  Latex document header.
%  ======================
\begin{latexonly}
   CCLRC / {\sc Rutherford Appleton Laboratory} \hfill {\bf \stardocname}\\
   {\large Particle Physics \& Astronomy Research Council}\\
   {\large Starlink Project\\}
   {\large \stardoccategory\ \stardocnumber}
   \begin{flushright}
   \stardocauthors\\
   \stardocdate
   \end{flushright}
   \vspace{-4mm}
   \rule{\textwidth}{0.5mm}
   \vspace{5mm}
   \begin{center}
   {\Huge\bf  \stardoctitle \\ [2.5ex]}
   \end{center}
   \vspace{5mm}

% ? Heading for abstract if used.
   \vspace{10mm}
   \begin{center}
      {\Large\bf Abstract}
   \end{center}
% ? End of heading for abstract.
\end{latexonly}

%  HTML documentation header.
%  ==========================
\begin{htmlonly}
   \xlabel{}
   \begin{rawhtml} <H1> \end{rawhtml}
      \stardoctitle\\
   \begin{rawhtml} </H1> \end{rawhtml}

% ? Add picture here if required.
% ? End of picture

   \begin{rawhtml} <P> <I> \end{rawhtml}
   \stardoccategory\ \stardocnumber \\
   \stardocauthors \\
   \stardocdate
   \begin{rawhtml} </I> </P> <H3> \end{rawhtml}
      \htmladdnormallink{CCLRC}{http://www.cclrc.ac.uk} /
      \htmladdnormallink{Rutherford Appleton Laboratory}
                        {http://www.cclrc.ac.uk/ral} \\
      \htmladdnormallink{Particle Physics \& Astronomy Research Council}
                        {http://www.pparc.ac.uk} \\
   \begin{rawhtml} </H3> <H2> \end{rawhtml}
      \htmladdnormallink{Starlink Project}{http://www.starlink.ac.uk/}
   \begin{rawhtml} </H2> \end{rawhtml}
   \htmladdnormallink{\htmladdimg{source.gif} Retrieve hardcopy}
      {http://www.starlink.ac.uk/cgi-bin/hcserver?\stardocsource}\\

%  HTML document table of contents.
%  ================================
%  Add table of contents header and a navigation button to return to this
%  point in the document (this should always go before the abstract \section).
  \label{stardoccontents}
  \begin{rawhtml}
    <HR>
    <H2>Contents</H2>
  \end{rawhtml}
  \htmladdtonavigation{\htmlref{\htmladdimg{contents_motif.gif}}
        {stardoccontents}}

\end{htmlonly}

% -----------------------------------------------------------------------------
% ? Document Abstract. (if used)
%   ==================

This manual documents the use of IBM DX (Data Explorer) on Starlink
systems. DX is a commercial scientific data visualisation package, suitable
for the visualisation and display of many sorts of astronomical data. It is
the package which Starlink recommends for the display of three-dimensional
scalar and vector data.  DX is not available at all Starlink sites; your
site manager should be able to advise on whether or not it is available at
your site.

This manual describes how to access DX and SX, the Starlink enhancements
to DX. It also documents these enhancements.

This edition of the manual applies to version \DXversion of DX and
describes version \SXversion of SX. If you have a later version of DX
it may be necessary to re-build SX. Your site manager should be able to
advise on this.

% ? End of document abstract
% -----------------------------------------------------------------------------
% ? Latex document Table of Contents (if used).
%  ===========================================
 \newpage
 \begin{latexonly}
   \setlength{\parskip}{0mm}
   \tableofcontents

   \newpage
   \listoftables
%   \listoffigures    % Comment out because there are no figures.

   \setlength{\parskip}{\medskipamount}
   \markboth{\stardocname}{\stardocname}
 \end{latexonly}
% ? End of Latex document table of contents
% -----------------------------------------------------------------------------

\vspace{2.5cm}

\subsection*{Revision history}

\begin{enumerate}

  \item 31st January 1996: Version 1 (DSB, GJP, ACD).

  \item 3rd March 1997: Version 2.  Minor changes (DSB, ACD).

  \item 15th September 1997: Version 3.  Revised the documentation
   for {\tt ndf2dx} to include the new parameter {\tt axes} (ACD).

\end{enumerate}

\vspace{2.5cm}
\copyright \underline{1997} Starlink, CCLRC


\cleardoublepage
\renewcommand{\thepage}{\arabic{page}}
\setcounter{page}{1}

\section{Introduction \xlabel{INTRO} }

This manual is a guide to the use of IBM DX (Data Explorer) on
Starlink systems. DX is a scientific visualisation package, suitable
for the visualisation and display of many sorts of astronomical data.
It is the package which Starlink recommends for the display of
three-dimensional scalar and vector data. It has particularly good
features for the display of three-dimensional vector data. DX is a
commercial package produced and sold by IBM. It is not available at all
Starlink sites. Your site manager should be able to advise on whether or
not it is available at your site. If it is not available locally and you
decide that you want to obtain a copy, Appendix~\ref{OBTAIN} outlines the
procedure.

This manual describes how to access DX and the Starlink enhancements to
it. It also documents these enhancements. The Starlink enhancements to
DX (collectively known as `SX') have a number of purposes:

\begin{itemize}

  \item they fill a few minor omissions in the functionality of basic
   DX,

  \item they provide easy-to-use functions to accomplish common tasks
   in a single step,

  \item they allow standard Starlink NDF data structures and data in
   other common astronomical formats to be imported into DX.

\end{itemize}

This edition of the manual applies to version \DXversion of DX and
describes version \SXversion of SX.  Normally SX will be installed at
Starlink sites which have bought DX and this is assumed in the rest of
this document.


\section{MPEG Animations \xlabel{MPEG} }

SX contains a module ({\tt{SXMakeMPEG}}) which allows individual images
displayed by DX to be encoded into an MPEG animation.  This module is
based on the Berkeley {\tt{mpeg\_encode}} program, which must be available
if the {\tt{SXMakeMPEG}} module is to be used.  If it is not currently
installed on your machine, it can be obtained from various ftp sites.  The
latest version  will be available at {\tt mm-ftp.cs.berkeley.edu} in
directory {\tt{/pub/multimedia/mpeg/encode}}.  However, obtaining software
from ftp sites in the USA is usually a very slow business.  For this
reason, pre-built versions of {\tt{mpeg\_encode}} have been created and
are available from ftp site {\tt{ast.man.ac.uk}} in directory
{\tt{outgoing/dsb}}. To install a copy, retrieve the file appropriate to
your operating system ({\tt{mpeg\_encode-osf.tar.Z}} or
{\tt{mpeg\_encode-sunos.tar.Z}}) and decompress it using
{\tt{uncompress}}.  The tar file contains the executable image and man
page.  These items should be extracted and copied to some suitable
locations such as {\tt{/usr/local/bin}} and {\tt{/usr/local/man/man1}}
respectively (your site manager will probably have to do this).  Also
available at {\tt{ast.man.ac.uk}} is the postscript file
{\tt{mpeg\_encode.ps}} which contains full documentation for
{\tt{mpeg\_encode}}.


\section{Starting DX \xlabel{START} }

No special privileges or quotas are required to run DX. DX (and, indeed,
visualisation software in general) tends to be quite profligate in its
use of computing resources. Files containing generated images can
require a significant amount of disk space, and in particular files
containing animations can be extremely large. Generating visualisations is
computationally intensive and can require substantial amounts of computer
memory. Therefore it is sensible to run DX on the computer with the
fastest processor and largest physical memory to which you have convenient
access.

DX is available for Digital Unix/alpha and Solaris/Sun.  Your site manager
should be able to advise which is available at your site.

To use DX you need a display capable of receiving X-output (typically an
X-terminal or a workstation console). DX will run on a black and white
display, but realistically you need a colour display\footnote{ For
example, with a black and white display the various icons for modules
in the visual programming editor are only partly visible.}.

To start DX, ensure that your display is configured to receive X-output
and then simply type:

\begin{quote}
{\tt dx}
\end{quote}

The following message should appear on your command terminal:

\begin{quote}
{\tt Starting DX user interface}
\end{quote}

and a new window (technically the canvas for the DX visual programming
editor) should appear. If DX fails to start properly then consult your
site manager who should be able to advise. The most likely reason is
that DX is not installed at your site. The Starlink extensions (SX)
should be automatically available when you start DX, and you can verify
this by clicking on `Import and Export' in the panel at the top left of
the visual programming editor window, and then looking for some modules
with names beginning with `SX' in the panel at the bottom left of the
visual programming editor window. If SX is not available then check that
your site manager has installed it.

\subsection{Starting DX without the Starlink enhancements}

It is possible to start DX without the Starlink enhancements, SX. Again
ensure that your terminal is configured to receive X-output. Then
type:

\begin{quote}
{\tt unalias \quad dx}
\end{quote}

Followed by:

\begin{quote}
{\tt dx}
\end{quote}

DX should now start, as above, but the Starlink enhancements will not
be available.  To re-instate the Starlink extensions, type the following
command before starting DX:

\begin{quote}
{\tt alias dx tcsh -c '"source \$SX\_DIR/dx.csh"'}
\end{quote}

\section{Learning to Use DX \xlabel{LEARN} }

DX is a powerful and flexible tool capable of generating sophisticated
visualisations of complex data. However, a necessary consequence of
this complexity is that it is non-trivial to use and it is necessary to
invest a certain amount of time to learn to use it effectively.
DX is essentially a tool which allows you to write programs or scripts
which generate some particular visualisation of a dataset. Though it is
possible to use pre-existing DX programs, most of the time you will use
DX to create, modify and use your own programs. DX programs can be
written using a text-based scripting language, which is not dissimilar
to conventional programming and scripting languages. However, this
scripting language is not the usual way to write DX programs, and it is
not discussed in this document. Rather, DX programs are usually written
using a `visual programming editor'. Icons representing modules to
perform some function (for example, reading a file, smoothing an image,
plotting an image etc.) are positioned on a canvas\footnote{A blank area
of a window used for constructing diagrams.} and joined by lines
representing the flow of data between modules. The assemblage so generated
performs the required visualisation (typically it will start by reading a
data file and end by generating an image). These assemblages are known as
`networks' or `visual programs'.  The Starlink enhancements to DX include
a number of networks for performing commonly required visualisations (see
Sections~\ref{SX} and \ref{DEMONET}) and make these tasks easier to perform.
Also a variety of documentation is available; it is summarised in
Table~\ref{DOCS}.


\begin{table}[htbp]

\begin{center}
\begin{tabular}{|l|} \hline
   \\
{\large\bf IBM Manuals}  \\
~~~ {\it QuickStart Guide}\cite{QUICKS} (a set of introductory tutorials) \\
~~~ {\it User's Guide}\cite{USERG} \\
~~~ {\it User's Reference}\cite{USERR} \\
~~~ {\it Programmer's Reference}\cite{PROGR} \\
   \\
{\large\bf Starlink Documentation} \\
~~~ {\bf SUN/203} {\it DX --- IBM Data Explorer for Data Visualisation}\,
(this document)  \\
~~~ {\bf SC/2} {\it The DX Cookbook}\cite{SC2}  \\
~~~ {\bf SG/8} {\it An Introduction to Visualisation Software for
     Astronomy}\cite{SG8}  \\
   \\
{\large\bf On-line sources of information} \\
~~~ A variety of on-line sources of information about DX are available;
 see SG/8 for a list. \\
   \\ \hline
\end{tabular}

\caption[Documentation and sources of information for DX.]
{Documentation and sources of information for DX. \label{DOCS} }
\end{center}

\vspace{5mm}

\end{table}

One possible route for learning to use DX is as follows.

\begin{enumerate}

  \item Part I of SG/8 \xref{{\it An Introduction to Visualisation
   Software for Astronomy}}{sg8}{}\cite{SG8}
   gives an overview of visualisation techniques, not
   tied to any particular package. If you have no prior knowledge of
   visualisation it may suggest techniques suitable for use with your
   data (DX supports most of the techniques described). Conversely,
   if you are familiar with visualisation and know the techniques that
   you want to apply then you may omit this step.

  \item SC/2 \xref{{\it The Starlink Data Explorer Cookbook}}{sc2}{}\cite{SC2}
   provides an introduction to DX and recipes for common tasks. You
   should read the introduction and try any recipes which seem similar
   to your requirements. This document may well provide all the
   information that you require.

  \item You can try the \htmladdnormallink{on-line
   tutorial}{http://www.phys.ocean.dal.ca/docs/DX_tutorial.html}
   available at the Department
   of Oceanography, University of Dalhousie (see SG/8\cite{SG8}).
   However, the speed of the transatlantic computer networks is such that
   the tutorial will often be un-usably slow (even on a Sunday afternoon).

  \item If none of the recipes in SC/2\cite{SC2} are suitable then try the
   IBM {\it QuickStart Guide}\cite{QUICKS}, again following examples which
   seem similar to what you want to do. See the first entry in
   Section~\ref{TIPS} below for a hint on running the examples in this
   guide.

  \item If necessary, you can consult the IBM {\it User's
   Guide}\cite{USERG} and {\it User's Reference}\cite{USERR} for detailed
   information.

  \item Finally, if you are experiencing difficulty using DX (and you
   are a Starlink user in the United Kingdom) then some advice and
   assistance is available from the Starlink Project. In the first
   instance contact David~Berry (e-mail: {\tt{dsb@ast.man.ac.uk}}).
   If he is unavailable then Clive Davenhall (e-mail: {\tt{acd@roe.ac.uk}})
   will act as a locum.

   An alternative source of expertise is the DX usenet news group:

  \begin{quote}
   {\tt comp.graphics.data-explorer}.
  \end{quote}

\end{enumerate}


\section{Useful Tips \label{TIPS} \xlabel{TIPS} }

This section lists various hints and tips which you may find useful when
using DX. Additional tips are available in the {\tt{README}} files
supplied with DX. In a standard installation these files are in directory
{\tt{/usr/lpp/dx}}. In a non-standard installation they will still be in
the {\tt{dx}} directory, but this directory will have a different location
in the host file system. The files available are:

\begin{description}

  \item[{\tt README}] contains system independent information,

  \item[{\tt README\_}{\it xxx}] contains system dependent information
   ({\it xxx} denotes a particular operating system, for example,
   {\tt alphax} for Digital Unix/alpha).

\end{description}

\begin{enumerate}

  \item There are a couple of cases where you may need to edit the
   tutorial files associated with the IBM {\it QuickStart
   Guide}\cite{QUICKS} before they will run. In a standard installation
   the tutorial files and associated data files are located in the
   following directories:

  \begin{tabular}{ll}
   tutorials:   &  {\tt /usr/lpp/dx/samples/tutorial} \\
   data files:  &  {\tt /usr/lpp/dx/samples/data} \\
  \end{tabular}

   In a non-standard installation DX will have the same directory tree
   beneath subdirectory {\tt dx}, but will have a different location
   in the host file system.

  \begin{itemize}

    \item The tutorial files assume that DX has been installed in the
     recommended location ({\tt{/usr/lpp}}, as above). If your system
     manager has installed it somewhere else you will need to edit
     the tutorial files so that they access data files in the correct
     location.

    \item A couple of tutorials refer to data file:

    \begin{tabular}{l}
     {\tt .../dx/samples/data/temp\_wind.header} \\
    \end{tabular}

     which does not exist. These tutorials should be edited to refer
     to file:

    \begin{tabular}{l}
     {\tt .../dx/samples/data/temp\_wind.general} \\
    \end{tabular}

  \end{itemize}

  \item The caps-lock key seems to alter the interpretation of keys
   which are not normally subject to the caps-lock setting. For example,
   you cannot delete a module icon from a network using
   {\tt{$<$control-delete$>$}} when the caps-lock key is on. {\it When using
   DX the caps-lock key should always be off.}

  \item When using DX on a Digital alpha, you may get an error message if you
   try to use the `Image' or `Display' modules, saying that `gethostname'
   has failed. This problem only seems to occur if your display (either
   workstation or X-terminal) has no alias in the {\tt /etc/hosts} file
   (or the equivalent NIS file). Just having an entry giving the numerical
   IP address of the display is insufficient; it must also have an alias.

  \item Sometimes when DX is executed on a Sun running Solaris it crashes
   with the error message:

  \begin{quote}
   {\tt cannot get shared memory segment}
  \end{quote}

   There are two possible causes of this problem.

  \begin{itemize}

    \item Under Solaris, DX uses shared memory in 256Mb segments.
     The default configuration for Solaris only allows 1Mb shared
     memory segments.  To allow DX to run, your site manager should
     log in as {\tt root}, and proceed as follows:

    \begin{enumerate}

      \item copy file {\tt /etc/system} to {\tt /etc/system.pre-dx}
       (that is, as a precaution, take a copy of the {\tt system} file
       before editing it),

      \item edit file {\tt /etc/system} and add the line:
      \begin{quote}
       {\tt set shmsys:shminfo\_shmmax = 0x10000000}
      \end{quote}

      \item reboot the system.

    \end{enumerate}

    \pagebreak[4]
    \item There is insufficient swap space available on the
     workstation\footnote{There are various Unix commands for showing
     the amount of swap space, but unfortunately they vary between
     Digital Unix/alpha and Solaris/Sun systems. Some of the possibilities
     are summarised in the following table:

    \begin{center}
    \begin{tabular}{rcc}
     Command         & Digital Unix/alpha & Solaris/Sun \\ \hline
     {\tt swap -s}   &                    & $\bullet$   \\
     {\tt swap -l}   &                    & $\bullet$   \\
     {\tt swapon -s} & $\bullet$          &             \\
     {\tt vmstat}    & $\bullet$          & $\bullet$   \\
    \end{tabular}
    \end{center}

     A bullet (`$\bullet$') indicates that the command may be available
     on a system of the given type. Of course, whether or not it is
     actually available on your system depends on how the system and
     your personal search path have been configured. If the command is not
     immediately available in your search path then try prefixing the
     command with the directory specification `{\tt /usr/sbin}'.

     Unfortunately the command which is available for both types of
     system, {\tt vmstat}, is the least useful because it does not
     directly show the amount of swap space available. Rather, it shows
     the number of free and used pages. Consult the appropriate {\tt man}
     pages for further details.

     \setcounter{swapfoot}{\value{footnote}}   }. You must either
     increase swap space or run DX with the {\tt -memory} option
     to reduce its use of memory.  For example, type {\tt dx -memory 32}
     to restrict DX to 32Mb of memory.

  \end{itemize}

  \item Sometimes DX generates the error message:

  \begin{quote}
   {\tt Out of memory}
  \end{quote}

   The solution is to explicitly  specify the (large) maximum amount of
   memory to be available for use when starting DX:

  \begin{quote}
   {\tt dx -memory} ~ {\it x}
  \end{quote}

   where {\it x}\, is the amount of memory to be used, in Mb. Thus, for
   example, to specify a maximum of 200Mb you would type:

  \begin{quote}
   {\tt dx -memory 200}
  \end{quote}

   The most suitable value depends on the capabilities and configuration
   of your system. The following (possibly contradictory) criteria
   apply:

  \begin{itemize}

    \item try to allocate at least two or three times the total size
     of your data set,

    \item but do not allocate more than the total amount of swap space
     on your machine\footnotemark[\value{swapfoot}],

    \item finally (but not least), other users will appreciate your
     setting a small value!

  \end{itemize}

   Appendix A, {\it Using Data Explorer: Some Useful Hints}, in the
   IBM {\it User's Guide}\cite{USERG} includes a section, {\it Memory
   Use}, which contains useful hints on reducing the amount of memory
   needed by DX.

  \item If the `Image' module runs without error but seems to produce no
   image (or if the image is too large or too small for the window), try
   resetting the image window by clicking anywhere inside it and then
   pressing {\tt $<$control-f$>$}. The displayed object should be
   re-centred and re-sized within the window.

   If the `Image' module repeatedly fails, generating X errors each time,
   then it may be necessary to log out and re-start the X window session.

  \item If you are running DX from an X-terminal it will sometimes
   crash with X-windows errors if you attempt to run a network with
   two image windows. The solution is to run such networks from a
   workstation console.

  \item If DX aborts for any reason (for example, you have typed
   {\tt $<$control-c$>$}, the computer crashing etc.) it sometimes leaves
   rogue processes running. These processes will run indefinitely and
   consume large amounts of CPU time.  You should check for them and then
   explicitly kill them. For example, on a Digital alpha you would check for
   them by typing something like:

  \begin{quote}
   {\tt ps aux | grep dx}
  \end{quote}

  \item Sometimes if your keyboard is not configured properly DX will
   generate a series of messages on start up as follows:

  \begin{quote}
   {\tt Could not convert X KEYSYM to a keycode}
  \end{quote}

   These messages can be averted by using {\tt xmodmap} to define a key
   for the required KEYSYM. For example, on a Digital alpha the following
   line could be edited into your {\tt .X11Startup} file:

  \begin{quote}
   {\tt xmodmap -e "keycode 80 = End"}
  \end{quote}

   which causes the F14 button on an N-108LK keyboard to generate the
   `End' KEYSYM. The tool {\tt xev} is useful for finding the key code
   associated with a given key on any keyboard.

  \item If {\tt $<$control-delete$>$} does not delete modules from a
   network (even when caps-lock is off), then create a file called
   {\tt .motifbind} in your home directory. It should contain the
   following line.

  \begin{quote}
   \verb-osfDelete                :<Key>Delete-
  \end{quote}

   The problem should not re-occur after you have logged out and
   logged in again.  Alternatively, to make the fix take effect
   immediately type:

  \begin{quote}
   {\tt \% xmbind .motifbind}
  \end{quote}

  \item If DX crashes with an error message something like:

  \begin{quote}
   {\tt sendsig: can't grow stack}
  \end{quote}

   a likely cause is that you were attempting to execute a network
   with cyclic connections (that is, modules which directly or
   indirectly depend on their own output). If you try to connect modules
   in this way DX will usually issue a warning and refuse to generate
   the connection. However, it sometimes fails to detect and trap
   more complex and indirect instances of cyclic connection.

\end{enumerate}


\section{Importing Data in Astronomical Formats
\label{IMPORTING} \xlabel{IMPORTING} }

There are two routes by which data in astronomical formats, such
as the Starlink Extensible N-Dimensional Data Format (NDF; see
\xref{SUN/33}{sun33}{}\cite{SUN33}), can be imported into DX. Either the
data can be converted to DX format prior to starting DX or alternatively
they can be accessed directly in their astronomical format from inside DX
using some of the Starlink enhancements to DX. The former route is
preferable if you plan to access the data more than once from within DX
(which will usually be the case) because the conversion will be done just
once; if you access the data in their original astronomical format from
within DX the conversion is done `on-the-fly' each time the data are read.
Also, in practice, the first approach is perhaps simpler.

\subsection{Conversion prior to using DX \label{PRIOR} }

The Starlink enhancements to DX include the application {\tt ndf2dx} (see
Appendix~\ref{NDF2DX}) for converting a standard Starlink NDF data file
into a format which can be accessed by DX. The output file generated is
in the native DX format and it can be imported directly into DX.

There are various options available for {\tt ndf2dx}
but usually a Starlink NDF can be
converted by simply typing:

\begin{quote}
{\tt \$SX\_DIR/ndf2dx} ~~ {\it ndf\_file} ~~ {\it dx\_file}
\end{quote}

where {\it ndf\_file}\, is the name of the input NDF file to be
converted and  {\it dx\_file}\, is the native DX format file created.
Note that the input file is not altered in any way by this process.
Also, by convention, the names of native DX format data files should end
with the suffix `{\tt .dx}'. The converted dataset may subsequently be
accessed from within a DX visual program using the standard module
`Import'.

The usual, or `native', method of storing an NDF dataset is as a file
formatted using the Starlink Hierarchical Data System (HDS). However,
in common with other Starlink applications, {\tt ndf2dx} can read
files in various other astronomical formats, such as FITS images, and
interpret them as NDF datasets. By default {\tt ndf2dx} will only
read HDS files. In order to configure it to recognise files in other
formats simply type:

\begin{quote}
{\tt convert}
\end{quote}

prior to using {\tt ndf2dx}. Then simply run {\tt ndf2dx} in the
usual way, supplying the name of the input file, be it a FITS image,
Figaro DST file or whatever, and any necessary conversions proceed
automatically and invisibly. Table~\ref{FORMATS} lists some of the
more common formats available. The set of data formats which is
recognised is configurable. The default configuration and how to specify
the configuration are both described in
\xref{SUN/55}{sun55}{}\cite{SUN55}.

\begin{table}[htbp]

\begin{center}
\begin{tabular}{rl}
Data Format                  & File Type \\ \hline
`Native' NDF (Starlink HDS)  & {\tt .sdf}  \\
FITS image                   & {\tt .fit}  \\
Figaro (version 2) DST       & {\tt .dst}  \\
GASP                         & {\tt .hdr}  \\
IRAF image                   & {\tt .imh}  \\
\end{tabular}

\caption[Valid data formats for input to {\tt ndf2dx}.]
{Valid data formats for input to {\tt ndf2dx}. The file type serves to
identify the format of a data file both to users and to {\tt ndf2dx}.
\label{FORMATS} }

\end{center}
\end{table}

\subsection{Conversion within DX}

An NDF data file can be accessed directly from within a DX visual
program by including SX macro {\tt SXReadNDF} (see Sections~\ref{SX} and
\ref{MODMAC}). The conversion is computed `on-the-fly' as the data are
read. Thus, if the data are to be accessed more than once (which will
usually be the case) it is more efficient to perform the conversion
prior to starting DX, as described in Section~\ref{PRIOR}, above.
If the conversion is done from within DX then the same set of
alternative formats which are automatically available for prior
conversions are also automatically available (see Table~\ref{FORMATS}).


\section{Importing Data in Scientific Formats \xlabel{IMPSCI} }

DX contains facilities for importing data in the following scientific
formats:

\begin{center}
\begin{tabular}{rcl}
{\bf CDF}    & -- & NASA Common Data Format,         \\
{\bf NetCDF} & -- & NASA Network Common Data Format, \\
{\bf HDF}    & -- & NCSA Hierarchical Data Format.   \\
\end{tabular}
\end{center}

CDF and NetCDF are widely used in solar-terrestrial physics, but little
used in other branches of astronomy. Importing data in these formats is
documented in Appendix~B, {\it Importing Data: File Formats}, in the IBM
{\it User's Guide}\cite{USERG}.


\section{Importing Data From Fortran Files \xlabel{IMPFOR} }

Before importing a data file into DX you must first create a general
format header file which describes the structure of the data file.
There are several ways of creating this header file.
\xref{SC/2 {\it The Starlink Data Explorer Cookbook}}{sc2}{}\cite{SC2}
gives a couple of examples. Section~4 of the IBM {\it
QuickStart Guide}\cite{QUICKS} contains further examples; in particular
all the keywords which can occur in the header file are described in
Section~4.3. Note that the name of the data file is one of the items of
information which must be included in the general format header.


\subsection{Formatted files}

Once a suitable general format header file has been created, a formatted
file written by a Fortran program can be imported into DX using the
standard module `Import'.

\subsection{Unformatted files}

Unfortunately DX cannot directly read an unformatted file written by
a Fortran program. Such files contain record control bytes which must
be  removed prior to importing the file into DX. However, the first
stage of importing such a file is still to create a general format
header file. The {\tt format} keyword should be either `{\tt ieee}'
or `{\tt binary}'.

Once the header file has been created there are two ways to proceed.

\pagebreak[4]
\begin{description}

  \item[First method] ~
  \begin{enumerate}

    \item Use {\tt SXUnfort} to remove the record control bytes prior
     to starting DX. From the Unix command line type:

    \begin{quote}
     {\tt \$SX\_DIR/SXUnfort} ~~ {\it input\_file} ~~ {\it output\_file}
    \end{quote}

     {\it input\_file}\, is a Fortran unformatted file and {\it
     output\_file}\, is the converted file.

    \item A general header file can then be created to describe the data.
     Note that the data file name entered into the general header
     file should be the name of the converted file ({\it{output\_file}},
     above).

    \item Import the converted file ({\it{output\_file}}) into DX using the
     standard module `Import'.

  \end{enumerate}

  \item[Second method] ~
  \newline Create a general header file, including the name of the raw
   Fortran data file, and then use the SX macro {\tt{SXReadFortran}}
   (see Sections~\ref{SX} and \ref{MODMAC}) to import the unformatted
   file directly into DX. The record control bytes are stripped
   `on-the-fly' each time the data are read. Thus, if the data are
   to be accessed more than once (which will usually be the case) it
   is more efficient to perform the conversion prior to starting DX,
   as described in the first method, above.

\end{description}

The procedure for importing Fortran data files is identical on both
Digital Unix/alpha and Solaris/Sun.  Of course, an unformatted file
written on a Digital alpha will differ from the corresponding file
written on a Sun because of the different byte order of the machines.
It is possible to input an unformatted file written on a Sun into DX
running on a Digital alpha, or vice versa, by using the `{\tt msb}' and
`{\tt lsb}' modifiers to the {\tt format} keyword in the general header
file\footnote{Strictly speaking it should be possible to input any
unformatted file written in IEEE floating point format, irrespective of
the type of machine that it was written on. Note, however, that Digital
VAXen and IBM mainframes do not use IEEE floating point format.}. See
section~4.3 of the IBM {\it QuickStart Guide}\cite{QUICKS} for details.


\section{Exporting NDFs From DX \xlabel{EXPNDF} }

DX is a visualisation package whose purpose is to display and examine
data. Usually you will use it to visualise the final set of data computed
at the end of a sequence of reductions and analyses or theoretical
calculations. Hence you are unlikely to want to convert a dataset in the
native DX format into the Starlink NDF format for further processing.
However, occasionally you may want to carry out such a conversion, and
it is possible in some cases. The native DX format is powerful, flexible
and general and it cannot be successfully converted to a Starlink NDF in
all cases.

To convert a native DX format file into a Starlink NDF use the `SXList'
module (see Sections~\ref{SX} and \ref{MODMAC}) to produce a text file
containing the required data. Then use KAPPA application {\tt trandat}
(see \xref{SUN/95}{sun95}{}\cite{SUN95}) or CONVERT application {\tt
ascii2ndf} (see \xref{SUN/55}{sun55}{}\cite{SUN55}) to convert the text
file to an NDF.


\section{SX -- The Starlink Enhancements to DX \label{SX} \xlabel{SX} }

This and subsequent sections assume that you have a basic familiarity
with DX. If you are in doubt then see, for example,
\xref{SC/2}{sc2}{}\cite{SC2}.

There are three types of Starlink enhancement to DX: modules, macros and
demonstration networks. The modules and macros fill a few omissions in
the functionality of basic DX. The demonstration networks are
easy-to-use examples of how to perform common tasks.

As a user you will not notice much difference between modules and
macros. Both perform some defined task and appear as an icon which may
be positioned within the canvas of the DX visual programming editor.
Technically the difference is that modules provide entirely new
functionality and are written in C, whereas macros are scripts invoking
several existing modules, though normally you will not be aware of these
details. There are, however, some slight differences in the way modules
and macros behave. For example, if you click on a macro and then select
the `Open Selected Macros' item from the `Windows' menu it will expand
into a diagram showing its constituent components. Also the way that
on-line help is provided is slightly different. Because of these
differences the type of each item (module or macro) is indicated in its
detailed documentation.

The following lists give one-line summaries for the modules and
macros and the demonstration networks. Subsequent sections document the
individual items in detail.

\subsection{Alphabetical list of modules and macros}

The Starlink DX modules and macros are summarised alphabetically below.
See Section~\ref{MODMAC} for a detailed description of each item.

\begin{description}

  \item[SXBin] -- bins points into a supplied grid.

  \item[SXBounds] -- finds the corners of a field's bounding box.

  \item[SXConstruct] -- constructs a regular field with regular
   connections.

  \item[SXDummy] -- a dummy module which does nothing.

  \item[SXEnum] -- enumerates the positions or connections in a field.

  \item[SXList] -- write values from a field or list to a text file.

  \item[SXMakeMpeg] -- constructs an MPEG animation.

  \item[SXPercents] -- searches a histogram of data values for requested
   percentiles.

  \item[SXPrint] -- creates a postscript version of a displayed image.

  \item[SXProfile] -- creates a one-dimensional profile through a field
   between two points.

  \item[SXRand] -- creates a set of random vectors.

  \item[SXReadFortran] -- reads a Fortran binary data file.

  \item[SXReadNDF] -- reads data from a Starlink NDF structure.

  \item[SXRegrid] -- samples a field at positions defined by another
   field.

  \item[SXSubset] -- creates a rectangular or arbitrary subset of a field.

  \item[SXVisible] -- identifies visible positions.

% \item[SXWriteNDF] -- writes a field to a Starlink NDF structure.

\end{description}

\subsection{Alphabetical list of demonstration networks}

The Starlink DX demonstration networks are summarised alphabetically below.
See Section~\ref{DEMONET} for a detailed description of each network.

\begin{description}

  \item[iso] -- displays iso-surfaces taken from a regular data grid.

  \item[scatter] -- displays scattered particle data. The data may be
   mapped on to a regular grid and written to a disk file.

  \item[slice] -- displays two-dimensional slices through a
   three-dimensional regular data grid.

  \item[stream] -- displays a three-dimensional vector field defined on a
   regular grid as a set of stream lines.

\end{description}


\section{Starlink DX Modules and Macros \label{MODMAC} \xlabel{MODMAC} }

This section presents a detailed description for each Starlink DX
module and macro. The entries are listed alphabetically. The format
in which the information for each module and macro is presented is
deliberately similar to that used for the standard modules in the IBM
{\it User's Reference}\cite{USERR}. The examples for several of the
modules and macros use example data files included in the standard
release of DX. If your copy of DX has been installed in a non-standard
location then see the first entry in Section~\ref{TIPS} for details of
how to find these files.

%=================================================================

%\input{sx_mods.tex}

%----  SXBin  ---------------------------------------------------------------

\Module{SXBin}
       {Realization}
       {bins points into a supplied grid}
       {output}
       {input, grid, type}

\begin{ModuleInputs}
\InputItem{input}
          {field or group}
          {none}
          {field or group with positions to bin}{}
\InputItem{grid}
          {field}
          {none}
          {grid to use as template}{}
\InputItem{type}
          {integer}
          {0}
          {type of output values required: \\
           \hspace*{2mm} 0 - mean data value in bin\\
           \hspace*{2mm} 1 - sum of data values in bin \\
           \hspace*{2mm} 2 - bin population}{}
\end{ModuleInputs}



\begin{ModuleOutputs}
\OutputItem{output}
           {field or group}
           {binned field}{}
\end{ModuleOutputs}


\ModuleSection{Description}{
The \htmlref{\Modnam{SXBin}}{SXBin} module bins the \Param{data} component of the
\Param{input} field into the bins defined by the ``connections'' component
of the \Param{grid} field. The input field can hold scattered or regularly
gridded points, but the ``data'' component must depend on ``positions''.
The \Param{grid} field must contain ``connections'' and ``positions''
components but need not contain a ``data'' component. The input ``data''
component must be either {\tt TYPE\_FLOAT} or {\tt TYPE\_DOUBLE}.

The ``data'' component in the \Param{output} field contains either the mean
or sum of the \Param{input} data values falling within each connection, or
the number of data values falling within each connection, as specified
by \Param{type}.

When binning a regular grid into another regular grid, beware of the
tendency to produce artificial large scale structure representing the
``beat frequency'' of the two grids.
}

\ModuleSection{Components}{
All components except the ``data'' component are copied from the \Param{grid}
field. The output ``data'' component added by this module depends on
``connections''. An ``invalid connections'' component is added if any output
data values could not be calculated (e.g. if the mean is required of an
empty bin).
}

\ModuleSection{Examples}{
This example bins the scattered data described in file ``{\tt CO2.general}'' onto a
regular grid, and displays it. \htmlref{\Modnam{SXBin}}{SXBin} is used to find the mean data
value in each grid connection:

\Example{
input = Import("/usr/lpp/dx/samples/data/CO2.general");\\
frame17 = Select(input,17);\\
camera = AutoCamera(frame17);\\
grid = SXConstruct(frame17,deltas=10);\\
bin = SXBin(frame17,grid);\\
coloured = AutoColor(bin);\\
Display(coloured,camera);\\
}

The next example produces a grid containing an estimate of the density of
the scattered points (i.e. the number of points per unit area). The
positions of the original scattered points are shown as dim grey
circles. \htmlref{\Modnam{SXBin}}{SXBin} finds the number of input positions in each bin,
\Modnam{Measure} finds the area of each bin, and \Modnam{Compute} divides the
counts by the areas to get the densities:

\Example{
input = Import("/usr/lpp/dx/samples/data/CO2.general");\\
frame17 = Select(input,17);\\
camera = AutoCamera(frame17);\\
glyphs=AutoGlyph(frame17,scale=0.1,ratio=1);\\
glyphs=Color(glyphs,"dim grey");\\
grid = Construct([-100,-170],deltas=[40,40],counts=[6,10]);\\
counts = SXBin(frame17,grid,type=2);\\
areas = Measure(counts,"element");\\
density = Compute("\$0/\$1",counts,areas);\\
coloured = AutoColor(density);\\
collected=Collect(coloured,glyphs);\\
Display(collected,camera);\\
}

}
\ModuleSection{See Also}{
\htmlref{\Modnam{SXRegrid}}{SXRegrid},
\Modnam{Map},
\Modnam{Construct},
\htmlref{\Modnam{SXConstruct}}{SXConstruct},
\Modnam{Measure}
}


%----  SXBounds  ---------------------------------------------------------------

\Macro{SXBounds}
      {Structuring}
      {Finds the corners of a field's bounding box}
      {lower, upper}
      {input}

\begin{ModuleInputs}
\InputItem{input}
          {field}
          {none}
          {input field}{}
\end{ModuleInputs}

\begin{ModuleOutputs}
\OutputItem{lower}
           {vector}
           {lower bounds}{}
\OutputItem{upper}
           {vector}
           {upper bounds}{}
\end{ModuleOutputs}


\ModuleSection{Description}{
The \htmlref{\Modnam{SXBounds}}{SXBounds} macro returns the upper and lower bounds of the box
enclosing the supplied \Param{field.}  The bounds are obtained from
the ``box'' component of the supplied field.
}


%----  SXConstruct ---------------------------------------------------------------

\Module{SXConstruct}
       {Realization}
       {constructs a regular field with regular connections}
       {output}
       {object, lower, upper, deltas, counts}

\begin{ModuleInputs}
\InputItem{object}
          {field}
          {no default}
          {object to define extent of new field}{}
\InputItem{lower}
          {vector}
          {no default}
          {explicit lower bounds of new field}{}
\InputItem{upper}
          {vector}
          {no default}
          {explicit upper bounds of new field}{}
\InputItem{deltas}
          {scalar or vector}
          {no default}
          {increment for each axis}{}
\InputItem{counts}
          {integer or vector}
          {no default}
          {number of positions along each axis}{}
\end{ModuleInputs}



\begin{ModuleOutputs}
\OutputItem{output}
           {field}
           {output field}{}
\end{ModuleOutputs}


\ModuleSection{Description}{
The \htmlref{\Modnam{SXConstruct}}{SXConstruct} module constructs a field with regular positions
and connections covering a volume with specified bounds. It is
similar to the standard \Modnam{Construct} module, but is somewhat easier to
use if a simple grid is required.

If \Param{object} is given, its bounds define the extent of the output
field. Otherwise, the vectors given for \Param{upper} and \Param{lower}
define the extent of the output field.

If \Param{deltas} is supplied, it defines the distances between adjacent
positions on each axis. It should be a vector with the same number
of dimensions as \Param{upper} and \Param{lower}, or a single value (in which
case the supplied value is used for all axes). The upper and lower
bounds are expanded if necessary until they span an integer number
of deltas.

If \Param{deltas} is not supplied, then \Param{counts} must be supplied and
should be an integer vector giving the number of positions on each
axis, or a single integer (in which case the same value is used for
all axes).
}

\ModuleSection{Components}{
The \Param{output} has ``positions'', ``connections'' and ``box'' components,
but no ``data'' component.
}

\ModuleSection{Examples}{
This example imports a scattered data set, extracts a single frame, uses \\
\htmlref{\Modnam{SXConstruct}}{SXConstruct} to make a grid covering the bounds of the imported data
set, with
increments of 10.0 along each axis, and then uses \htmlref{\Modnam{SXBin}}{SXBin} to find the
mean data value in each of the square connections of this new grid. The
resulting field is displayed:

\Example{
input = Import("/usr/lpp/dx/samples/data/CO2.general");\\
frame17 = Select(input,17);\\
newgrid = SXConstruct(frame17,deltas=10);\\
binned = SXBin(frame17,newgrid);\\
coloured = AutoColor(binned);\\
camera = AutoCamera(binned);\\
Display(coloured,camera);\\
}
}

\ModuleSection{See Also}{
\Modnam{Construct},
\Modnam{Grid}
}


%----  SXDummy ---------------------------------------------------------------

\Module{SXDummy}
       {Special}
       {a dummy module which does nothing}
       {}
       {field, group, vector, scalar, integer, vectorlist, scalarlist,
        integerlist, string, camera}

\begin{ModuleInputs}
\InputItem{field}
          {field}
          {none}
          {an input field}{}
\InputItem{group}
          {group}
          {none}
          {an input group}{}
\InputItem{vector}
          {vector}
          {none}
          {an input scalar}{}
\InputItem{scalar}
          {scalar}
          {none}
          {an input scalar}{}
\InputItem{integer}
          {integer}
          {none}
          {an input integer}{}
\InputItem{vectorlist}
          {vector list}
          {none}
          {an input vector list}{}
\InputItem{scalarlist}
          {scalar list}
          {none}
          {an input scalar list}{}
\InputItem{integerlist}
          {integer list}
          {none}
          {an input integer list}{}
\InputItem{string}
          {string}
          {none}
          {an input string}{}
\InputItem{camera}
          {camera}
          {none}
          {an input camera}{}
\end{ModuleInputs}


\ModuleSection{Description}{
The \htmlref{\Modnam{SXDummy}}{SXDummy} module does nothing. It is used merely to force inputs
to macros to be of a specified type. Connecting the output from an
\Modnam{Input} module to one of the inputs of \htmlref{\Modnam{SXDummy}}{SXDummy} forces the
\Modnam{Input} to be of the same type as the \htmlref{\Modnam{SXDummy}}{SXDummy} input to which it
is connected.
}




%----  SXEnum  ---------------------------------------------------------------

\Module{SXEnum}
       {Realization}
       {enumerates the positions or connections in a field}
       {output}
       {input, name, dep}

\begin{ModuleInputs}
\InputItem{input}
          {field or group}
          {none}
          {field or group to enumerate}{}
\InputItem{name}
          {string}
          {``{\tt data}''}
          { name of component to store the enumeration}{}
\InputItem{dep}
          {string}
          {``{\tt positions}''}
          {object to be enumerated; ``{\tt positions}'' or ``{\tt connections}''}{}
\end{ModuleInputs}

\begin{ModuleOutputs}
\OutputItem{output}
           {field or group}
           {enumerated field}{}
\end{ModuleOutputs}


\ModuleSection{Description}{
The \htmlref{\Modnam{SXEnum}}{SXEnum} module creates the component specified by \Param{name,}
and adds it to the \Param{output} field. The values in the new component
start at zero and increment by one for each position or connection in the
field.

The new component is in one-to-one correspondence with either the
``positions'' or ``connections'' component, dependent on \Param{dep.}
}

\ModuleSection{Components}{
Adds a component with the given \Param{name,} deleting any existing
component. All other components are copied from the \Param{input} field.
}

\ModuleSection{Examples}{
In this example, the 17th frame is extracted from the scattered data set
described in file ``{\tt CO2.general}'', and a field created holding
only those positions with offsets between 10 and 20:

\Example{
input = Import("/usr/lpp/dx/samples/data/CO2.general");\\
frame17 = Select(input,17);\\
enum = SXEnum(frame17,"index","positions");\\
marked = Mark(enum,"index");\\
included = Include(marked,10,20,1);\\
subset = Unmark(included,"index");\\
}

}



%----  SXList  ---------------------------------------------------------------

\Module{SXList}
       {Import and Export}
       {write values from a field or list to a text file}
       {nrec}
       {input, file, append, names}

\begin{ModuleInputs}
\InputItem{input}
          {field or value or value list or string or string list}
          {none}
          {input field or list}{}
\InputItem{file}
          {string}
          {(Message Window)}
          {name of the text file}{}
\InputItem{append}
          {integer}
          {0}
          {1 to append to existing file \\ 0 to create new file}{}
\InputItem{names}
          {string or string list}
          {``{\tt data}''}
          {the field components to list}{}
\end{ModuleInputs}

\begin{ModuleOutputs}
\OutputItem{nrec}
           {integer}
           {number of records written to the file}{}
\end{ModuleOutputs}

\ModuleSection{Description}{
The \htmlref{\Modnam{SXList}}{SXList} module writes numerical or string values to an output
text \Param{file.} The values may be obtained from a field, or they may be
specified explicitly as a list. Each record written to the \Param{file}
contains one item from the list, or one item from each of the
field components specified by \Param{names,} layed out in columns.

It is not an error to give a null \Param{input,} but nothing will be
written to the file.

If no value is supplied for \Param{file} then the list will be displayed
in the message window.

If \Param{append} is set to 1, the output will be appended to any
existing \Param{file} with the given name. A new file will be created if
none exists. If \Param{append} is set to 0, a new file is created every
time, potentially over-writing an existing file.
}

\ModuleSection{Examples}{
In this example, the ``data'' and ``positions'' components from a field are
written out to a text file called ``{\tt dump.dat}''. The file starts with a
header containing three comment lines:

\Example{
input = Import("/usr/lpp/dx/samples/data/CO2.general");\\
frame17 = Select(input,17);\\
SXList({"\#","\# Positions and data","\#"},"dump.dat",0);\\
SXList(frame17,"dump.dat",1,{"positions","data"});\\
}

}
\ModuleSection{See Also}{
\Modnam{Export},
\Modnam{Echo},
\Modnam{Print}
}

%----  SXMakeMpeg ---------------------------------------------------------------

\Macro{SXMakeMpeg}
      {Import and Export}
      {constructs an MPEG animation}
      {}
      {image, save, mpeg}

\begin{ModuleInputs}
\InputItem{image}
          {image}
          {none}
          {image holding the next frame of the animation}{}
\InputItem{save}
          {flag}
          {0}
          {0 to add another frame, 1 to save the frames to an MPEG}{}
\InputItem{mpeg}
          {string}
          {none}
          {name of an MPEG file, including a file suffix (e.g. ``.mpg'')}{}
\end{ModuleInputs}

\ModuleSection{Description}{
Each time the \htmlref{\Modnam{SXMakeMpeg}}{SXMakeMpeg} macro is activated with \Param{save} set
to 0, the supplied \Param{image} is converted into a ``{\tt PPM}'' file on
disk in the current directory, with a name like {\tt SX\_frame.n.ppm}
(where {\tt n} is a frame number). When \htmlref{\Modnam{SXMakeMpeg}}{SXMakeMpeg} is called with
\Param{save} set to 1, these {\tt PPM} files are encoded into an MPEG
animation with name given by \Param{mpeg}, using the public domain
Berkeley encoder ``{\tt mpeg\_encode}'', which should be installed on
your host system (it can be obtained from ftp site
{\tt mm-ftp.cs.berkeley.edu} in directory \newline
{\tt /pub/multimedia/mpeg/encode}, or from ast.man.ac.uk in directory
outgoing/dsb). The {\tt PPM} files and other intermediate files such
as the file holding the encoder parameter values, are deleted once the MPEG
has been created.

The input \Param{image} must be a rendered image, as created by the
\Modnam{Render} module.

Giving \Param{save} as 1, without supplying a value for \Param{mpeg}
causes the current {\tt PPM} frames to be deleted without attempting to
create an MPEG.

While {\tt mpeg\_encode} is running, it displays statistics and progress
messages in the DX message window. Note, it can take a long time to
encode an MPEG containing many frames, especially if each frame is large.
It is best to use small frames, partly to speed up the encoding, but also
because the resulting animation will play back faster, and occupy less
disk space.

Note, only one instance of the \htmlref{\Modnam{SXMakeMpeg}}{SXMakeMpeg} module can be included in a
DX network.

}

\ModuleSection{Examples}{
This example creates an MPEG from the first 3 frames of the data described in
file ``{\tt CO2.general}''.
The image size is reduced to 320 pixels by setting
the \Param{resolution} in \Modnam{AutoCamera} in order to speed up the display
of the movie.

\Example{
input = Import("/usr/lpp/dx/samples/data/CO2.general");\\
coloured = AutoColor(input);\\
cam = AutoCamera(input,resolution=320);\\
\\
frame = Select(coloured,1);\\
image = Render(frame,cam);\\
SXMakeMpeg(image);\\
\\
frame = Select(coloured,2);\\
image = Render(frame,cam);\\
SXMakeMpeg(image);\\
\\
frame = Select(coloured,3);\\
image = Render(frame,cam);\\
SXMakeMpeg(image);\\
\\
SXMakeMpeg(save=1,mpeg="CO2.mpg");\\
}

}

\ModuleSection{See Also}{
\Modnam{WriteImage}
}
%----  SXPercents ---------------------------------------------------------------

\Module{SXPercents}
       {Transformation}
       {searches a histogram of data values for requested percentiles}
       {outputlist, output}
       {input, percents}

\begin{ModuleInputs}
\InputItem{input}
          {field}
          {none}
          {field containing the histogram}{}
\InputItem{percents}
          {scalar list}
          {\{5.0,95.0\}}
          {the required percentiles}{}
\end{ModuleInputs}

\begin{ModuleOutputs}
\OutputItem{outputlist}
           {scalar list}
           {list containing the data values at all the requested percentiles}{}
\OutputItem{output}
           {scalar}
           {data value at a single requested percentile}{}
\OutputItem{...}
           {}
           {more single data values}{}
\end{ModuleOutputs}


\ModuleSection{Description}{
The \htmlref{\Modnam{SXPercents}}{SXPercents} module searches the histogram supplied by
\Param{input,} for the data values corresponding to the requested
percentiles. The \Param{input} histogram should have been created using
the \Modnam{Histogram} module.

Each value supplied in \Param{percents} should be a value in the range 0.0 to
100.0. For each value, $P$, the corresponding \Param{output} value is the
data value below which $P$ percent of the data values in the histogram
lie. Linear interpolation is performed to find this value.

The \Param{outputlist} parameter is a list containing all the individual
\Param{output} values.
}


\ModuleSection{Examples}{
This example displays the electron density field, with a colour
table which ignores the lowest and highest 2 percent of the data
values:

\Example{
electrondensity = Import("/usr/lpp/dx/samples/data/watermolecule");\\
histogram = Histogram(electrondensity)\\
plist, lo, hi = SXPercents(histogram,{2.0,98.0});\\
camera = AutoCamera(electrondensity);\\
coloured = AutoColor(electrondensity,min=lo,max=hi);\\
Display(coloured,camera);\\
}
}

\ModuleSection{See Also}{
\Modnam{Histogram},
\Modnam{Statistics}
}


%----  SXPrint ---------------------------------------------------------------

\Macro{SXPrint}
      {Import and Export}
      {creates a postscript version of a displayed image}
      {outlen}
      {object, camera, file, width, dpi, length, encapsulated, colour, portrait}

\begin{ModuleInputs}

\InputItem{object}
          {object}
          {none}
          {object to be rendered}{}
\InputItem{camera}
          {camera}
          {none}
          {camera used to display \Param{object}}{}
\InputItem{file}
          {string}
          {none}
          {output file name}{}
\InputItem{width}
          {scalar}
          {8}
          {width of printed image in inches}{}
\InputItem{dpi}
          {integer}
          {300}
          {dots per inch}{}
\InputItem{length}
          {value list}
          {1.0}
          {a length at old resolution}{}
\InputItem{encapsulated}
          {integer}
          {0}
          {produce an encapsulated file? (0=no, 1=yes)}{}
\InputItem{colour}
          {integer}
          {1}
          {produce a colour file? (0=no, 1=yes)}{}
\InputItem{portrait}
          {integer}
          {0}
          {use portrait mode? (0=no, 1=yes)}{}
\end{ModuleInputs}

\begin{ModuleOutputs}
\OutputItem{outlen}
           {value list}
           {\Param{length} at new resolution}{}
\end{ModuleOutputs}


\ModuleSection{Description}{
The \htmlref{\Modnam{SXPrint}}{SXPrint} macro renders the supplied object using the supplied
camera and creates a postscript file holding the image. The postscript
version is rendered at the resolution specified by \Param{dpi} to produce
a picture with the supplied \Param{width} (in inches).

The \Param{object} and \Param{camera} will usually be obtained from the
output parameters of the \Modnam{Image} module.

Note, the size and shape of some objects are specified in terms of pixels
(e.g. the objects created by \Modnam{Colorbar} and \Modnam{Caption}) and these
objects will {\em not} be re-rendered to the higher resolution. They will
thus appear smaller in the printed version of the image than on the screen.
To get round this problem, module \htmlref{\Modnam{SXPrint}}{SXPrint} can be used to calculate new sizes for such
objects which will cause them to retain their original relative sizes in
the printed version. To do this, assign the original object sizes (as
specified in the configuration box of the module which produces the object) to
\Param{length}, and do not assign any values to \Param{file} or
\Param{object} (all other parameters should be set to the values they will have when the
final postscript file is produced). The \Param{outlen} parameter will
then be returned holding the equivalent object sizes at the higher
resolution. These should then be passed on to \Modnam{Colorbar},
\Modnam{Caption}, etc, to produce objects of the correct size.
}

\ModuleSection{See Also}{
\Modnam{Image},
\Modnam{WriteImage}
}

%%----  SXProfile ---------------------------------------------------------------

\Macro{SXProfile}
      {Realization}
      {creates a 1-dimensional profile through a field between 2 points}
      {line, profile}
      {input, start, end, density}

\begin{ModuleInputs}
\InputItem{input}
          {field}
          {none}
          {input data field}{}
\InputItem{start}
          {vector}
          {lower bounds of input field}
          {starting point of line}{}
\InputItem{end}
          {vector}
          {upper bounds of input field}
          {ending point of line}{}
\InputItem{density}
          {integer}
          {30}
          {number of points along line}{}
\end{ModuleInputs}

\begin{ModuleOutputs}
\OutputItem{line}
           {geometry field}
           {geometry field describing the line}{}
\OutputItem{profile}
           {field}
           {1D field holding data values along the line}{}
\end{ModuleOutputs}


\ModuleSection{Description}{
The \htmlref{\Modnam{SXProfile}}{SXProfile} macro samples the \Param{input} field at regular
intervals along a straight line joining the \Param{start} and \Param{end}
positions. The number of samples is given by \Param{density}.

The sampled data values are returned in \Param{profile}, which has a
``positions'' component holding the distance from the \Param{start} to
each sample. This field can be converted to a 1-D plot using the \Modnam{Plot}
module, which can then be viewed using the \Modnam{Image} module. To obtain
a hard-copy, use the {\bf\tt Print Image} option in the {\bf\tt File} menu
of the \Modnam{Image} window. Superior plots can be obtained by dumping the
profile to an ASCII file using \htmlref{\Modnam{SXList}}{SXList}, and then using a plotting
program such as \Modnam{gnuplot}, \Modnam{sm} or \Modnam{Mongo} to produce the plot.
See SG/8 for an overview of available plotting packages.

Two vector-valued attributes named ``profile\_start'' and ``profile\_end''
are added to the output \Param{profile}, holding the \Param{start} and
\Param{end} vectors.

A geometry field describing the sampled positions in the original {\em n}-D
space is returned in \Param{line}. This could, for instance, be included
with a rendering of the original data to indicate the position of
the profile.
}

\ModuleSection{Examples}{

This example displays a profile through the {\tt watermolecule} field
containing 50 values sampled between positions \mbox{$[-1.0,-1.0,-1.0]$} and
\mbox{$[1.0,1.0,0.0]$}. The profile is converted into a plot using
\Modnam{Plot}. A title caption is created and collected together with the
plot, before displaying with \Modnam{Display}.

\Example{
electrondensity = Import("/usr/lpp/dx/samples/data/watermolecule");\\
line, profile = SXProfile(electrondensity,[-1.0,-1.0,-1.0],[1.0,1.0,0.0],50);\\
plot = Plot(profile,{"Distance along profile","Electron density"},\\
\hspace*{0.75in} labelscale=0.5,aspect=0.75,frame=2);\\
camera = AutoCamera(plot,width=3.5);\\
caption = Caption("Profile from [-1.0,-1.0,-1.0] to [1.0,1.0,0.0]",\\
\hspace*{0.75in} [0.7,0.99],height=18);\\
collected = Collect(plot,caption);\\
Display(collected,camera);\\
}

}

\ModuleSection{See Also}{
\Modnam{Plot},
\Modnam{Image},
\htmlref{\Modnam{SXList}}{SXList}
}

%----  SXRand ---------------------------------------------------------------

\Module{SXRand}
       {Realisation}
       {create a set of random vectors}
       {output}
       {nvec, ndim, dist, a, b}

\begin{ModuleInputs}
\InputItem{nvec}
          {integer or field or group}
          {1}
          {number of vectors required}{}
\InputItem{ndim}
          {integer}
          {1}
          {dimensionality for each vector}{}
\InputItem{dist}
          {integer}
          {1}
          {distribution: \parbox[t]{1in}{ 1 - uniform \\
                         2 - normal \\
                         3 - Poisson } }{}
\InputItem{a}
          {scalar}
          {0.0}
          {first parameter describing the distribution}{}
\InputItem{b}
          {scalar}
          {1.0}
          {second parameter describing the distribution}{}
\end{ModuleInputs}

\begin{ModuleOutputs}
\OutputItem{output}
           {vector list or field}
           {output vectors}{}
\end{ModuleOutputs}


\ModuleSection{Description}{
The \htmlref{\Modnam{SXRand}}{SXRand} module creates a set of random vectors.
Each vector has \Param{ndim} components, each chosen independently from the
distribution specified by \Param{dist.} The vectors can be returned as a list
of \Param{nvec} vectors (if \Param{nvec} is an integer), or as the ``data''
component of a field (if \Param{nvec} is a field).

If \Param{dist} is 1, then a uniform distribution between \Param{a} and \Param{b} is
used. If \Param{dist} is 2 then a normal distribution with mean \Param{a} and
standard deviation \Param{b} is used. If \Param{dist} is 3, then a Poisson
distribution is used with mean \Param{a.} Integer values are returned for
a Poisson distribution, and double precision values are returned for
uniform and normal distributions.
}



\ModuleSection{Components}{
If \Param{nvec} is a field, a ``data'' component is added to
the output field, replacing any existing ``data'' component. The output
data will be dependent on ``positions'' if the input field had no ``data''
component. Otherwise, it will have the same dependency as the input
field. Any components which are dependent on ``data'' are modified. All
other components are copied from the input field.
}

\ModuleSection{Examples}{
This example constructs a regular grid of 50 by 50 by 50 points containing
3-vectors at each position in which each component is uniformly distributed
between 0 and 1:

\Example{
newfield = Construct([0,0,0],[1,1,1],[50,50,50]);\\
rand = SXRand( newfield, 3 );\\
}

This example generates and displays a list of 5 samples from a normal
distribution of mean 10 and standard deviation 2:

\Example{
randvals = SXRand( 5, 1, 2, 10.0, 2.0 );\\
Echo( randvals );\\
}

}


%----  SXReadFortran ---------------------------------------------------------------

\Macro{SXReadFortran}
      {Import and Export}
      {reads a Fortran binary data file}
      {data}
      {header, variable, start, end, delta}

\begin{ModuleInputs}
\InputItem{header}
          {string}
          {none}
          {general format header file}{}
\InputItem{variable}
          {string or string list}
          {everything}
          {variables to be read}{}
\InputItem{start}
          {integer}
          {first frame}
          {starting data frame}{}
\InputItem{end}
          {integer}
          {last frame}
          {ending data frame}{}
\InputItem{delta}
          {integer}
          {1}
          {increment between frames}{}
\end{ModuleInputs}

\begin{ModuleOutputs}
\OutputItem{data}
           {object}
           {object containing requested variables}{}
\end{ModuleOutputs}


\ModuleSection{Description}{

The \htmlref{\Modnam{SXReadFortran}}{SXReadFortran} macro imports binary data files written by
Fortran unformatted {\tt WRITE} statements. It strips the Fortran
record control bytes out of the binary data before importing it. {\em
NB}, Fortran formatted files (i.e. text files) should be imported using
the \Modnam{Import} module.

The supplied value for \Param{header} should be a text file containing
a general format header describing the binary data. The actual data
should be in a separate file identified by the usual ``file'' keyword in
the header.

The conversion is done ``on-the-fly'' each time a binary file is
accessed. As an alternatively to using the \htmlref{\Modnam{SXReadFortran}}{SXReadFortran} macro, the
\htmlref{\Modnam{SXUnfort}}{SXUnfort} program can be used to perform the conversion once, after
which the \Modnam{import} module can be used to import the converted file. See
the prologue of \Modnam{\$SX\_DIR:SXUNfort.c} for more details.

The other parameters are equivalent to the corresponding parameters of the
\Modnam{Import} module (the \Param{format} parameter of the \Modnam{Import}
module is fixed at ``general'').

}

\ModuleSection{See Also}{
\Modnam{Import}
}


%----  SXReadNDF ---------------------------------------------------------------

\Macro{SXReadNDF}
      {Import and Export}
      {reads data from a Starlink NDF structure}
      {data}
      {ndf, positions}

\begin{ModuleInputs}
\InputItem{ndf}
          {string}
          {none}
          {the input NDF structure}{}
\InputItem{positions}
          {integer}
          {1}
          {data dependency: \\
          \hspace*{2mm} 0 - connections\\
          \hspace*{2mm} 1 - positions}{}
\end{ModuleInputs}

\begin{ModuleOutputs}
\OutputItem{data}
           {object}
           {data read from \Param{ndf}}{}
\end{ModuleOutputs}


\ModuleSection{Description}{
The \htmlref{\Modnam{SXReadNDF}}{SXReadNDF} macro imports data from a
Starlink NDF structure. It
uses the {\tt \$SX\_DIR/ndf2dx} application.

The dependency of the data values in the NDF is specified by the
\Param{positions} parameter.

The conversion is done ``on-the-fly'' each time an NDF is accessed. As an
alternatively to using the \htmlref{\Modnam{SXReadNDF}}{SXReadNDF} macro, the
{\tt ndf2dx}
application can be used to perform the conversion once, after which the
\Modnam{import} module can be used to import the converted file.

The facilities provided by the NDF library for transparently accessing
foreign data formats are available with this module. See Section 5.

}

\ModuleSection{See Also}{
\Modnam{Import}
%\htmlref{\Modnam{SXWriteNDF}}{SXWriteNDF}
}

%----  SXRegrid --------------------------------------------------------------------

\Module{SXRegrid}
       {Realization}
       {samples a field at positions defined by a another field}
       {output}
       {input, grid, nearest, radius, scale, exponent, coexp, type}

\begin{ModuleInputs}
\InputItem{input}
          {field or group}
          {none}
          {field or group with positions to regrid}{}
\InputItem{grid}
          {field}
          {none}
          {grid to use as template}{}
\InputItem{nearest}
          {integer or string}
          {1}
          {number of nearest neighbours to use, or ``{\tt infinity}''}{}
\InputItem{radius}
          {scalar or string}
          {``{\tt infinity}''}
          {radius from grid point to consider, or ``{\tt infinity}''}{}
\InputItem{scale}
          {scalar or vector or scalar list}
          {1.0}
          {scale lengths for weights}{}
\InputItem{exponent}
          {scalar}
          {1.0}
          {weighting exponent}{}
\InputItem{coexp}
          {scalar}
          {0.0}
          {exponential co-efficient for weights}{}
\InputItem{type}
          {integer}
          {0}
          {type of output values required: \\
           \hspace*{2mm} 0 - weighted mean \\
           \hspace*{2mm} 1 - weighted sum \\
           \hspace*{2mm} 2 - sum of weights}{}
\end{ModuleInputs}



\begin{ModuleOutputs}
\OutputItem{output}
           {field or group}
           {regridded field}{}
\end{ModuleOutputs}


\ModuleSection{Description}{
The \htmlref{\Modnam{SXRegrid}}{SXRegrid} module samples the ``data'' component of the \Param{input}
field at the positions held in the ``positions'' component of the
\Param{grid} field. It is similar to the standard \Modnam{Regrid} module, but
provides more versatility in assigning weights to each input position,
the option of returning the sums of the weights or the weighted sum
instead of the weighted mean, and seems to be much faster. Both
supplied fields can hold scattered or regularly gridded points, and
need not contain ``connections'' components. The ``data'' component in the
\Param{input} field must depend on ``positions''.

For each grid position, a set of near-by positions in the input
field are found (using \Param{nearest} and \Param{radius}). Each of these input
positions is given a weight dependent on its distance from the current
grid position. The output data value (defined at the grid position) can
be the weighted mean or weighted sum of these input data values, or
the sum of the weights (selected by \Param{type}).

The weight for each input position is of the form $(d/d_{0})^{p}$ where
$d$ is the distance from the current grid position to the current
input position, and $p$ is the value of \Param{exponent.} If a single value
is given for \Param{scale} then that value is used for the $d_{0}$ constant
for all the near-by input positions. If more than 1 value is given for
\Param{scale} then the first value is used for the closest input position,
the second value for the next closest, etc. The last supplied value is
used for any remaining input positions. A value of zero for \Param{scale}
causes the corresponding input position to be given zero weight.

If \Param{coexp} is not zero, then the above weights are modified to
become $\exp((d/d_{0})^{p})$.

If \Param{nearest} is given an integer value, it specifies $N$, the maximum
number of near-by input positions to use for each output position.
The $N$ input positions which are closest to the output position are
used. If the string ``{\tt infinity}'' is given, then all input positions
closer than the distance given by \Param{radius} are used. Using \Param{radius,}
you may specify a maximum radius (from the output position) within
which to find the near-by input positions. If the string ``{\tt infinity}''
is given for \Param{radius} then no limit is placed on the radius.

}

\ModuleSection{Components}{
All components except the ``data'' component are copied from the \Param{grid}
field. The output ``data'' component added by this module depends on
``positions''. An ``invalid positions'' component is added if any output
data values could not be calculated (e.g. if there are no near-by input
data values to define the weighted mean, or if the weights are too
large to be represented, or if the input grid position was invalid).
}

\ModuleSection{Examples}{
This example maps the scattered data described in file \newline
``{\tt CO2.general}'' onto a
regular grid and displays it. \htmlref{\Modnam{SXRegrid}}{SXRegrid} is used to find the data value
at the nearest input position to each grid position:

\Example{
input = Import("/usr/lpp/dx/samples/data/CO2.general");\\
frame17 = Select(input,17);\\
camera = AutoCamera(frame17);\\
grid = Construct([-100,-170],deltas=[10,10],counts=[19,34]);\\
regrid = SXRegrid(frame17,grid);\\
coloured = AutoColor(regrid);\\
Display(coloured,camera);\\
}

The next example produces a grid containing an estimate of the density of
the scattered points (i.e. the number of points per unit area). The
positions of the original scattered points are shown as dim grey
circles. \htmlref{\Modnam{SXRegrid}}{SXRegrid} finds the 5 closest input positions at each grid
position. Zero weight is given to the closest 3 positions. The fourth
position has a weight which is half the density of the points within the
circle passing through the fourth point (i.e. if the fourth point
is at a distance $D$ from the current grid position, there are 3 points
within a circle of radius $D$, so the density within that circle is
$3/(\pi.D^{2})$ ). The fifth position has a weight which is half the
density of the points within the circle passing through the fifth
point. The output data value is the sum of the weights (because
\Param{type} is set to 2), which is the mean of the densities within the
circles touching the fourth and fifth points:


\Example{
input = Import("/usr/lpp/dx/samples/data/CO2.general");\\
frame17 = Select(input,17);\\
camera = AutoCamera(frame17);\\
glyphs = AutoGlyph(frame17,scale=0.1,ratio=1);\\
glyphs = Color(glyphs,"dim grey");\\
grid = Construct([-100,-170],deltas=[10,10],counts=[19,34]);\\
density = SXRegrid(frame17,grid,nearest=5,scale=[0,0,0,0.691,0.798],\\
                 exponent=-2,type=2);\\
coloured = AutoColor(density);\\
collected = Collect(coloured,glyphs);\\
Display(collected,camera);\\
}

}
\ModuleSection{See Also}{
\htmlref{\Modnam{SXBin}}{SXBin},
\Modnam{ReGrid},
\Modnam{Map},
\Modnam{Construct},
\htmlref{\Modnam{SXConstruct}}{SXConstruct}
}


%----  SXSubset ---------------------------------------------------------------

\Macro{SXSubset}
      {Import and Export}
      {creates a rectangular or arbitrary subset of a field}
      {output}
      {input, object, lower, upper, expression}

\begin{ModuleInputs}
\InputItem{input}
          {field}
          {none}
          {the field to be subsetted}{}
\InputItem{object}
          {field}
          {object assigned to \Param{input}}
          {object to define corners of rectangular subset}{}
\InputItem{lower}
          {vector}
          {lower bounds of \Param{object}}
          {lower bounds of rectangular subset}{}
\InputItem{upper}
          {vector}
          {upper bounds of \Param{object}}
          {upper bounds of rectangular subset}{}
\InputItem{expression}
          {string}
          {none}
          {\Modnam{Compute} expression specifying an arbitrary subset}{}
\end{ModuleInputs}

\begin{ModuleOutputs}
\OutputItem{output}
           {field}
           {the subset}{}
\end{ModuleOutputs}


\ModuleSection{Description}{
The \htmlref{\Modnam{SXSubset}}{SXSubset} macro returns a subset of the \Param{input}
field, containing either the positions which fall within the rectangular
volume specified by \Param{object,} \Param{lower} and \Param{upper,} or the
positions for which the given \Param{expression} evaluates to a non-zero
answer. The \Param{input} field can have either scattered or gridded
positions, and need not have a ``connections'' component.

If a value is supplied for \Param{expression} then \Param{object,} \Param{lower} and
\Param{upper} are ignored. In this case, \Param{expression} should be a
string holding an arithmetic or logical expression suitable for use by
the \Modnam{Compute} module. It can contain references to the variables
``field'' ({\tt \$0} in the scripting language) and ``index'' ({\tt \$1} in
the scripting language) . The ``field'' variable refers to the ``positions''
component of the \Param{input} field, and so ``field.x'', ``field.y'' and
``field.z'' can be used to refer to the $(x,y,z)$ values at each
\Param{input} position. The ``index'' variable is a scalar with integer values
and refers to an enumeration of the \Param{input} ``positions'' component
(i.e. ``index'' is 0 at the first position, 1 at the second, 2 at the
second, etc ). {\em N.B.}, \Param{expression} should not refer to
components of ``field'' which do not exist (i.e. do not refer to
``field.z'' if \Param{input} is 2-dimensional).

If no \Param{expression} is supplied, \Param{lower} and \Param{upper} are
used to define a rectangular brick with the given bounds. \Param{lower}
and \Param{upper} default to the bounds of \Param{object.} {\em Note},
\Param{lower} and \Param{upper} should have the correct number of
components (i.e. 2 for a 2-dimensional \Param{input,} 3 for a 3-dimensional \Param{input})
}

\ModuleSection{Components}{
Modifies the ``data'', ``positions'' and ``connections'' and any components
which depend on ``positions'' or ``connections''. All other components
are propagated to the output.
}

\ModuleSection{Examples}{

This example displays the {\tt watermolecule} field between bounds of \\
\mbox{$[-1.0,-1.0,-1.0]$} and \mbox{$[1.0,1.0,0.0]$}. The \Modnam{ClipBox}
module could have been
used to do this, but \Modnam{ClipBox} does not actually remove
the excluded positions from the data set; it just tells the renderer not
to render them:

\Example{
electrondensity = Import("/usr/lpp/dx/samples/data/watermolecule");\\
subset = SXSubset(electrondensity,lower=[-1.0,-1.0,-1.0],upper=[1.0,1.0,0.0]);\\
volume = AutoColor(subset,min=0,max=0.5);\\
camera = AutoCamera(volume,"off-diagonal");\\
Display(volume,camera);\\
}

The next example creates a translucent isosurface of the {\tt watermolecule}
field, and then takes a subset of the field which matches the bounds of the
isosurface. The coloured volume and the isosurface are collected together
and displayed:

\Example{
electrondensity = Import("/usr/lpp/dx/samples/data/watermolecule");\\
isosurface = Isosurface(electrondensity,0.3);\\
isosurface = Color(isosurface,"grey",0.3);\\
subset = SXSubset(electrondensity,isosurface);\\
volume = AutoColor(subset,min=0,max=0.5);\\
both = Collect(volume,isosurface);\\
camera = AutoCamera(both,"off-diagonal");\\
Display(both,camera);\\
}

The next example, creates a spherical subset of the {\tt watermolecule}
field, centred on \mbox{$[1.0,0,-1]$}, of radius 0.5, and displays it face on. In the
\htmlref{\Modnam{SXSubset}}{SXSubset} expression, the squared radius of each position is
compared 0.25 (i.e. the square of 0.5):

\Example{
electrondensity = Import("/usr/lpp/dx/samples/data/watermolecule");\\
subset = SXSubset(electrondensity,expression="\$0.x**2+\$0.y**2+\$0.z**2<0.25");\\
volume = AutoColor(subset,min=0,max=0.5);\\
camera = AutoCamera(volume);\\
Display(volume,camera);\\
}

The next example, creates a subset of the scattered data described in
file ``{\tt CO2.general}'',
containing every second element, starting at element zero.
\htmlref{\Modnam{SXSubset}}{SXSubset} uses the modulus function in
\Modnam{Compute} (``{\tt \%}'') to
assign a value of 1 to the required positions, and zero to the other positions:

\Example{
data = Import("/usr/lpp/dx/samples/data/CO2.general");\\
subset = SXSubset(data,expression="(\$1+1)\%2");\\
glyphs = AutoGlyph(subset);\\
camera = AutoCamera(glyphs);\\
Display(glyphs,camera);\\
}

}

\ModuleSection{See Also}{
\Modnam{Compute},
\Modnam{Slab},
\Modnam{Slice},
\Modnam{ClipBox}
}

%----  SXVisible ---------------------------------------------------------------

\Macro{SXVisible}
      {Transformation}
      {identifies visible positions}
      {outfield, flags}
      {field, list, camera}

\begin{ModuleInputs}
\InputItem{field}
          {field}
          {none}
          {field containing positions to be checked}{}
\InputItem{list}
          {vector list}
          {none}
          {list of positions to be checked}{}
\InputItem{camera}
          {camera}
          {none}
          {camera used to view positions}{}
\end{ModuleInputs}

\begin{ModuleOutputs}
\OutputItem{outfield}
           {field}
           {field with invisible positions removed}{}
\OutputItem{flags}
           {byte list}
           {visibility flags for each position}{}
\end{ModuleOutputs}


\ModuleSection{Description}{
The \htmlref{\Modnam{SXVisible}}{SXVisible} macro determines which of the positions given by
\Param{field} or \Param{list} would be visible if viewed with the
projection defined by \Param{camera}. A list of bytes is returned in
\Param{flags} which corresponds one-for-one with the input positions.
This list contains 1 if the corresponding position is visible and 0
otherwise.

The input positions are defined by the ``positions'' component
of \Param{field} if \Param{field} is provided, or by \Param{list}
otherwise. A copy of the supplied \Param{field} (if any) is returned
in \Param{outfield} from which all invisible positions have been removed.

This module works with both perspective and orthographic cameras.
}

\ModuleSection{Components}{
All components of \Param{field} are propagated to the \Param{outfield}.
}


\ModuleSection{See Also}{
\Modnam{AutoCamera},
\Modnam{Image}
}

%----  SXWriteNDF ---------------------------------------------------------------

%\Macro{SXWriteNDF}
%      {Import and Export}
%      {writes a field to a Starlink NDF structure}
%      {}
%      {input, ndf}
%
%\begin{ModuleInputs}
%\InputItem{input}
%          {field}
%          {none}
%          {input field}{}
%\InputItem{ndf}
%          {string}
%          {none}
%          {the output NDF structure}{}
%\end{ModuleInputs}
%
%\ModuleSection{Description}{
%The \htmlref{\Modnam{SXWriteNDF}}{SXWriteNDF} macro exports a DX field to a Starlink NDF structure. It
%uses the {\tt dx2ndf} application in the Starlink {\tt CONVERT} package
%which should be installed on the host on which the DX executive is running.
%See SUN/55 for more details of the {\tt dx2ndf} command.
%
%The \Param{input} should be a single field (i.e. not a group containing
%fields).
%
%}
%
%\ModuleSection{See Also}{
%\Modnam{Export},
%\htmlref{\Modnam{SXReadNDF}}{SXReadNDF}
%}



%=================================================================


\newpage
\section{Starlink DX Demonstration Networks
\label{DEMONET} \xlabel{DEMONET} }

%=================================================================

%\input{sx_demos.tex}
Several demonstration networks are provided with SX which provide
ready-to-go facilities for generating simple visualisations, and also provide
a starting point for your own networks. The demonstrations available are:

\begin{quote}
\begin{description}

\item [\htmlref{{\bf iso}}{iso}] -- displays iso-surfaces taken from a regular data grid.
The network is in file \\
{\tt \$SX\_DIR/iso\_demo.net}.

\item [\htmlref{{\bf slice}}{slice}] -- displays 2-d slices through a 3-d
regular data grid. The network is in file \\
{\tt \$SX\_DIR/slice\_demo.net}.

\item [\htmlref{{\bf stream}}{stream}] -- displays a 3-d vector field defined on a
regular grid as a set of stream lines. The network is in file
{\tt \$SX\_DIR/stream\_demo.net}.

\item [\htmlref{{\bf scatter}}{scatter}] -- displays scattered particle data. The data may be
mapped on to a regular grid and written to a disk file.
The network is in file {\tt \$SX\_DIR/scatter\_demo.net}.

\end{description}
\end{quote}

Sample data files are also available:

\begin{quote}
\begin{description}

\item[{\tt /usr/lpp/dx/samples/data/storm\_data.dx}] -- contains vector and
scalar fields defined on a 3-d regular grid.

\item[{\tt /star/etc/sx/scattered\_data.1.dx}] -- contains vector and scalar
fields defined at scattered positions in 3-d space. If the
Starlink software collection is not rooted at {\tt /star}, then replace
{\tt /star} with the actual Starlink root directory. There are eight
other scattered data files in the same directory, with names in which the
``{\tt 1}'' in the above file name is replaced by an integer between 2
and 9.

\end{description}
\end{quote}

All these files are in native DX format.

\subsection{Running the Demonstration Networks}
\label{SEC:RUNNING_DEMO}

The demonstration networks can be read into DX like any other network,
using the ``Open Program...'' option in the ``File'' menu of the network
editor window. Alternatively, they can be executed more simply by
starting dx with the command:

\small
\begin{verbatim}
   % dx demo <name>
\end{verbatim}
\normalsize

where {\tt <name>} is the name of the demonstration to be run. If no
name is given, the user is given a list of available demonstrations and
prompted for a name before continuing. Using this access method, the
demonstration is presented as a finished application, with help provided
to guide the user through its use. The gory details of the network are
hidden away, and all the user sees are the ``front-end'' controls used to
control the behaviour of the network. Later sections of this document
describe the controls available for each network.

\subsubsection{Executing the network}
Each demonstration is basically a large program with several input
parameters, which runs from start to finish without intervention each
time it is executed, using the parameter values established by the user
before execution commenced\footnote{This execution model (which is common to all DX networks) is
somewhat different from many astronomical packages which allow parameter
values to be specified {\em after} execution has commenced. For instance,
a DX network will abort if the user has supplied a bad parameter value,
where as other packages may re-prompt the use for a new value in such
circumstances. Another more limiting feature is that DX networks cannot
suggest run-time dynamic defaults for parameter values which are based on
the results of some earlier processing of the data. It is sometimes
necessary to hold this execution model in mind in order to understand
the behaviour of the demonstration networks.}.
To use a demonstration, all parameters should be set to the
required values (or left at their default values), and the program should
then be executed once by selecting the ``Execute once'' item from the
``Execution'' menu in the Image window. While the program is
executing, the ``Execute'' menu in the Image window will be coloured
green. When execution ends (as indicated by the ``Execute'' menu
reverting to its usual black colour), the image window should be left
holding the required image as specified by the supplied parameter values.
You can then modify the parameter values, and re-execute the network
(using the ``Execution'' menu again) to create a modified display.

\subsubsection{Entering parameter values}
Parameter values are displayed and entered using several ``control
panels'', each panel containing controls related to a particular aspect
of the demonstration. To access these control panels, open the
``Windows'' menu on the menu-bar of the Image window, and then select the
``Open control panel by name'' item. This will display a list of the
control panels, and you can then open the panel by clicking on its name.
The panel will remain open until you close it using the \DemoWidget{Close}
button at the bottom of the panel. Each panel has on-line help information
describing its use which can be accessed by pressing the \DemoWidget{Help}
button at the bottom of the panel. Parameters can be set by selecting
items from menus, pressing buttons, typing values into a box, etc, in a
similar way to other Motif-based graphical user interfaces.

\begin{center}
{\em Always press return after typing a value into a data entry
box.}
\end{center}

\subsubsection{Specifying the input data file}
When run in this manner, the first thing which happens once the network
has been loaded is that a pop-up window appears explaining how to get
help on using the network. Click on the \DemoWidget{OK} button within the
pop-up window to get rid of this message. The next thing to do is to open
the \Cpanel{Select input file...} control panel as described above, and supply
an input file name. This can be done by typing the file name into the
data entry box (remembering to press return). Alternatively you
can browse through your file space by pressing the \DemoWidget{...} button at
the right hand end of the data entry box.

You must also specify the data format of the specified file using the
\DemoWidget{Data format} menu. The following formats are available:

\begin{description}
\item [DX native:] These are files created by DX or by the
``{\tt \$SX\_DIR/ndf2dx}'' application. They usually have a file extension of ``.dx''.
\item[DX general:] These are text files describing data typically produced by
a user's own programs. The data itself can be either ASCII or binary,
and may reside in another file
referenced from within the supplied description file. See the DX {\em User's Guide}
for details of the description file. Binary data written by a Fortran
program cannot be read using this format.
\item[DX general (Fortran binary):] These are identical to ``DX general'' files,
except that they refer to binary data created by a Fortran program, which
consequently contains record control bytes in addition to the data bytes. The
description file is identical to that for non-Fortran data.
\item[NDF:] These are standard Starlink NDF data structures.
\end{description}

Once you have specified the input file name and data format, you must
execute the network as described above. This produces an image using
default parameter settings. This is necessary because the network does
not know the permitted range of the parameter values until it has
examined the data.

\subsubsection{Producing sequences of images}
The demonstration networks can be re-executed automatically to produce a
sequence of images. This is controlled using the sequencer control panel
which can be opened by selecting ``Sequencer'' from the ``Execute'' menu in
the Image window. This panel contains controls similar to a standard
video recorder. To start the sequence press the play button (an
arrow pointing to the right). To interupt it, press the stop button (a
recessed square). The networks are written so that one or
more parameters are incremented automatically each time it is executed by the
sequencer. See the descriptions of the individual networks below for
details of which parameters are controlled by the sequencer.

\subsubsection{Changing your view of the displayed object and adding axes}
In addition to the demonstration control panels, the display can also be
modified using the ``Options'' menu situated on the menu-bar of
the Image window.  This menu allows control of (amongst other things) the
size and position of the image, and the position in space from which the
object is viewed. It also provides facilities for adding enumerated and
labelled 3-d axes to the image.

\begin{quote}
\begin{description}

\item [View control:] Select the ``View control...'' item from
  the ``Options'' menu. This produces a new window. You can
  press the ``Set view'' button to get a list of preset
  ``view points'' from which you can select, or you can press
  the ``Mode'' button, and then select ``Rotate'' to get a
  ``virtual 3-d tracker-ball'' which is controlled by
  pressing the left mouse button and moving the cursor over
  the image. You can also change from an {\em orthographic} view which is
  like viewing the object from a great distance with a powerful telescope,
  to a perspective view which is like being close up to the object.
  Note, the image is not re-drawn until you release the mouse button.

\item [Image size and position:] Select the ``View control...''
  item from the ``Options'' menu, and then press the
  ``Mode'' button. Then select the ``Pan/Zoom'' item from
  the displayed list. Now position the cursor at the place
  where you want the new image centre to be. To zoom in,
  press the left mouse button and drag it until the displayed
  box encloses the area you are interested in, and then
  release the button. The image is re-drawn with the
  selected area filling the screen. To zoom out, press
  the right-hand mouse button instead. When the button
  is released, the image is re-drawn with the whole
  screen compressed into the selected area.

\item [Axes control:] Select the ``Autoaxes...'' item from the
  ``Options'' menu, and press the ``Enabled'' button at the
  top-left of the window which is then popped up. You can
  also enter labels for the axes using the three data entry
  boxes just below the ``Enable'' button. Other aspects of
  the axes can be controlled by pressing the ``Expand''
  button at the bottom of the window. This displays more
  options which can then be set appropriately. Note, the
  axes do not appear until the program is re-executed, or
  the image window is reset by pressing $<${\tt control-F}$>$ (which
  also resets the view to a default off-diagonal view).

\item [Resetting the Image window:] The Image window retains
  its setting even if you select a new input file. Of
  course, the old settings may not be appropriate for the
  new data (e.g. the new data may cover a much larger
  volume and so may not all fit in the Image window).
  Pressing {\tt <control-F>} while in the Image window causes the
  view to reset to an off-diagonal display of size
  appropriate for the data.

\end{description}
\end{quote}

\subsubsection{Creating MPEG movies}
MPEG movies may be created by saving a series of displayed images and
then encoding them into an MPEG file. To do this, open the \Cpanel{MPEG
control...} control panel as described above, and press the
\DemoWidget{Save frames} button. Each subsequently displayed image will be
saved to disk as the next frame for the final movie. Remember to release
this button if you do not want to save the next image in this way.

Once all frames have been saved, release the \DemoWidget{Save frames} button,
enter a file name in the \DemoWidget{MPEG file name} data entry box
(remembering to press return), press the \DemoWidget{Create MPEG}
button, and re-execute the network. The saved frames will be encoded into
an MPEG file. Note, you must have the Berkeley {\tt mpeg\_encode} program
installed for this to work.

\subsection{``What do I do if...''}
\subsubsection{``...no image is displayed?''}
This could be due to the ``camera'' being too far away or pointing in
the wrong direction. Try resetting the camera by clicking on the title
bar of the Image window, and then pressing {\tt <control-F>}.

Sometimes there may be no data to display (for instance if you attempt to
start a stream line at a place where there is no valid data). In this
case, a window may pop up saying ``No data to display''. Try changing
the parameter values and re-executing the network.

Another possibility is that the data is too faint to see. Try adjusting
the colour controls for the network.

\subsubsection{``...a big window containing lots of little boxes appears?''}
If an error occurs while the network is executing (for instance if you
specify an input file which does not exist or an illegal \Modnam{Compute}
expression) then an error message is
written to the DX ``Message window'', and a window is opened showing the
network with the offending module highlighted in orange. This is probably
more than you really want! To get rid of the network editor window,
select the ``Close'' item from the ``File'' menu at the left hand end of
it's menu bar. You can also close the message window in a similar way.
You can then change the parameter values and re-execute the network.

\subsubsection{``...the whole thing seems to have locked solid?''}
You can use the ``Disconnect from server'' and ``Start server'' items in
the ``Connection'' menu of the Image window, to re-initialise the server
process which does the numerical data processing for DX.

\subsubsection{``...my sequence doesn't run?''}
It is sometimes necessary to press the stop button (a recessed square
icon) on the sequencer control panel, and then press the play button (an
arrow pointing to the right) several times to get the sequence to play
correctly.

\subsection{\label{iso}\xlabel{iso}Further details on using the {\bf iso} network}
This demonstration allows the user to import a data file, potentially
containing several different scalar or vector quantities defined on a
regular grid, and create a 3-d representation of a surface of
constant value (an ``iso-surface'') in any one of the quantities in the
file. Optionally, the colour at each point on the surface can be
determined by the value of another quantity in the same input file.
Alternatively, the surface can be given a uniform blue-grey colour. The
opacity of the surface can be varied from completely transparent, to
completely opaque. A sequence of different iso-surfaces (each
corresponding to a different constant value) can be created and displayed
automatically, and optionally saved in an MPEG animation file.

The following control panels are available:
\begin{description}

\DemoPanel{Select input file...}
This has been described in section \ref{SEC:RUNNING_DEMO}.

\DemoPanel{MPEG control...}
This has been described in section \ref{SEC:RUNNING_DEMO}.

\DemoPanel{Select field to iso-surface...}
Press the \DemoWidget{Field to define iso-surface} button to see a menu of all the
fields (i.e. defined quantities) in the input file. Select the field
which you want to use to define the shape of the iso-surface.
If the input file only contains a single un-named field, then it will be
indicated in the menu by the base file name.

You can display the common logarithm of the data instead of the actual
data by pressing the \DemoWidget{Take common log of data} button. The
shape of the iso-surfaces will not change, but linearly spaced sequences
of iso-surfaces will become logarithmically spaced.  Negative or zero
data values are excluded from the display.

\DemoPanel{Set iso-surface values...}
The default data value for the iso-surface is the median value of the
selected field. This can be over-ridden by entering a value in the
\DemoWidget{Iso value} box, and then pressing the \DemoWidget{Value selection
method} button and selecting ``User-supplied value'' from the menu.

Alternatively, an equally spaced sequence of values can be used in
succession. The maximum and minimum values are entered in \DemoWidget{Max. iso
value} and \DemoWidget{Min. iso value} and the number of steps is entered in
\DemoWidget{No. of values}. Executing the network will cause the Sequencer
control panel to appear.

\DemoPanel{Surface colouring...}
If you want the colour of the iso-surface to represent the data value in
a second field (i.e. another quantity in the data file), then press the \DemoWidget{Use
another field to colour the surface?} button,
and select the field by pressing the \DemoWidget{Field to use to colour
surface} button and selecting from the menu of available fields.

If the \DemoWidget{Use another field to colour the surface?} button is {\em
not} pressed, then the surface will have a uniform blue-grey colour.

The opacity value entered in the \DemoWidget{Surface opacity} box applies to
both sorts of colouring. A value of zero produces a completely clear
(i.e. invisible) surface, and a value of one produces a completely opaque
surface.

Press the \DemoWidget{Take common log of data} button to use the logarithm of
the data values to determine the colour. Zero or negative data values
produce holes in the surface.

\end{description}


\subsection{\label{slice}\xlabel{slice}Further details on using the {\bf slice} network}
This demonstration allows the user to import a data file, potentially
containing several different scalar or vector quantities defined on a
regular grid, and create a representation of a 2-d slice
through the grid perpendicular to one of the axes. The colour at each
point on the slice is determined by the magnitude of the data value, or
optionally of the common logarithm of the data value. The orientation
and position of the slice can be set by the user, and a bounding box can
be displayed encompassing the whole volume. A sequence of equally spaced
slices can be created and displayed automatically. Images can be saved
and encoded into an MPEG animation.

The following control panels are available:
\begin{description}

\DemoPanel{Select input file...}
This has been described in section \ref{SEC:RUNNING_DEMO}.

\DemoPanel{MPEG control...}
This has been described in section \ref{SEC:RUNNING_DEMO}.

\DemoPanel{Select field to display...}
Press the \DemoWidget{Field to display} button to see a menu of all the
fields (i.e. defined quantities) in the input file. Select the field
which you want to display.
If the input file only contains a single un-named field, then it will be
indicated in the menu by the base file name.

You can display the common logarithm of the data instead of the actual
data by pressing the \DemoWidget{Take common log of data} button.
Negative or zero data values are excluded from the display.

\DemoPanel{Set slice position...}

The \DemoWidget{Show bounding box} button determines if a bounding box should be
displayed with the slice. The bounding box encloses the entire data set.

The \DemoWidget{Axis name} button allows you to select the orientation of the
slice. The displayed slice will be perpendicular to the selected axis.

The position at which the slice cuts the axis can be determined in four
ways, selected using the \DemoWidget{Position selection method} menu:

\begin{description}

\item [Default position:] A default position is used. This is the mid point
of the selected axis.

\item [User-supplied position:] The user-supplied position is used. This
should be entered in the \DemoWidget{Axis value} box.

\item [Sequence of positions:] An equally spaced sequence of positions is used
in succession. The maximum and minimum  values are entered in \DemoWidget{Max.
axis value} and \DemoWidget{Min. axis value}, and the number of steps is entered
in \DemoWidget{No. of values}.
Executing the network will cause the Sequencer control panel to appear.

\item [Image window probe:] A cursor is used to set the position. Open the
view control dialogue box by clicking on ``View Control'' in the
``Options'' menu of the Image window.  Then click on ``Mode'' and select
``Cursors'' from the displayed list of options.  Position the arrow cursor
somewhere over the displayed image and double click. This will produce a
small square cursor (or {\em probe}) which can be dragged around by pointing
at it and holding the left mouse button down.

The probe is constrained to move parallel to an axis, and the projection
of the probe position on to each face of the enclosing box is displayed
in order to ease the problem of positioning the probe in 3 dimensions.
Once the probe is positioned, release the left mouse button and execute
the network.

\end{description}

\DemoPanel{Slice colouring...}
The slice will be displayed with a colour table going from blue at low
data values to red at high data values. The data values corresponding to
pure red and blue can either be entered by the user, or default values
can be used. The method is determined by the setting of the
\DemoWidget{Method} button. Any data values which fall outside the given
limits will produce holes in the slice (i.e. the background colour,
black by default, will show through). The default values are the maximum
and minimum data values on the slice. Thus the default scaling will
change as the slice is moved through the data volume.

This panel also allows you to change the size of the labels on the colour
bar displayed at the right hand edge of the window. The default size is
multiplied by the supplied factor.  The default size is dependent on the
distance between tick marks and so will in general alter if the data
range covered by the colour bar is changed.

\end{description}

\subsection{\label{stream}\xlabel{stream}Further details on using the {\bf stream} network}
This demonstration allows the user to import a data file,
potentially containing several different quantities defined
on a regular grid, and create a representation of a vector
field using stream lines.

The starting positions of the stream lines can be given
explicitly by the user, using cursor or keyboard.
Alternatively, stream lines can be started at each point on
a regular grid covering a specified volume. These
positions may then be modified using an arbitrary vector
algebra expression, involving the original positions and
the frame number in a sequence of frames. Thus
sequences of images can be made in which the stream line
starting positions move through the data.

The stream lines may be coloured to represent the
magnitude of the field, or of another field in the same
data file.
An image may also contain an iso-surface defined by a
separate field in the input data file, and a bounding box
encompassing the whole volume.
The point from which the objects are viewed can be automatically rotated
in 3-D between successive frames to give a continuous rotation.
Any set of displayed images can be saved and encoded
into an MPEG animation.

The following control panels are available:
\begin{description}

\DemoPanel{Select input file...}
This has been described in section \ref{SEC:RUNNING_DEMO}.

\DemoPanel{MPEG control...}
This has been described in section \ref{SEC:RUNNING_DEMO}.

\DemoPanel{Streamline positions...}
This panel controls the number of stream lines displayed, their starting
positions and maximum lengths. The stream lines are defined by the field
selected with the \DemoWidget{Field to define stream lines} button, and this field
must contain a vector quantity.

The stream line starting positions can be specified in three ways, as
selected by the \DemoWidget{Method} button:

\begin{description}

\item [Keyboard:]
The positions are entered using the controls at the lower left of the
panel, under the heading ``Keyboard positions''.  To add a new position,
enter the $x$, $y$ and $z$ co-ordinates in the three slider boxes at the
bottom, and then press \DemoWidget{Add}. To delete a position, click on the
line holding the position in the panel just above the \DemoWidget{Add} and
\DemoWidget{Delete} buttons, and then press \DemoWidget{Delete}.

\item [Grid:]
An evenly spaced grid of positions is used as defined by the controls at
the lower right of the panel. The volume enclosing the grid is defined
with \DemoWidget{Grid lower bounds} and \DemoWidget{Grid upper bounds}, and the
number of positions along each axis is defined with \DemoWidget{No. of grid
points}.

\item [Probes:]
Positions are given using the cursor to control probes in the image
window. Press ``View control'' in the ``Options'' menu of the Image window.
Then select ``Cursors'' using the ``Mode'' button in the view control
dialogue box. Position the arrow cursor somewhere over the image and
double click on the left button. This should produce a small square
probe which can be dragged around the 3-d volume represented by the
displayed image. To do this point at the probe with the cursor, press the
left mouse button and at the same time move the mouse. Note, you can only
move the probe parallel to one of the axes at any one time. Spots are
shown on a bounding box indicating the projection of the probe position
onto the faces of the bounding box. Many probes may be introduced into
the volume in this way. To delete a probe, point at it with the cursor
and double click on the left mouse button again. When all the required
probes have been positioned, re-execute the network to create the
corresponding stream lines.

\end{description}

Once the positions have been defined by one of the above methods, they
can be modified by entering an algebraic expression into the
\DemoWidget{Expression} box. This is an expression suitable for use by the
\Modnam{Compute} module (see the DX {\em User's Reference Guide} for details), in which
the symbol $p$ represents the original positions. For instance the
expression ``$p+[0,1,0]$'' increments all the positions by 1 in
the $y$
direction. The default expression ``$p$'' leaves the positions unchanged.

The symbol $i$ can be used to represent the frame number within a
sequence, starting at 1.  For instance, the expression
``$p+[i-1,0,0]$''
increments the stream line starting positions by 1 along the $x$ axis for
each frame in the sequence. The number of frames in the sequence is set
in the \DemoWidget{No. of frames} box. A sequence of more than one frame can
be controlled using the Sequencer control panel, which should
automatically appear if \DemoWidget{No. of frames} is larger than 1. A
sequence of 1 frame isn't really a sequence at all and is created as
normal using the ``Execute'' menu.

Stream lines end when they pass outside the bounding box, reach a point
where the velocity has zero magnitude, or exceed the ``time'' given in the
\DemoWidget{Max. time} box. This presumes that the vector represents a
velocity, and that a point moving with unit speed (i.e. of unit vector
magnitude) moves a unit distance (as defined by the spatial co-ordinate
system of the data) in unit time.

The units of time are defined by the data
(i.e. a point with unit vector magnitude moves a unit distance in unit time).

\DemoPanel{Display control...}
A yellow box is displayed which encloses the entire data volume if the
\DemoWidget{Display bounding box} button is pressed.

The size of the colour bar annotation can be changed by entering a value
in the \DemoWidget{Scale for colour bar labels} box. The default size is multiplied
by this factor. The default size depends on the distance between tick
marks and so varies with the data range. Note, a colour bar is only
displayed if the stream lines are coloured to show some data value (see
the \Cpanel{Streamline colours...} control panel).

The stream lines can be shown as thin lines, tubes or ribbons, and the
\DemoWidget{Rendering style} button is used to select the method to use. Ribbons
can show twist in the vector field.

\DemoPanel{Streamline colours...}
If the \DemoWidget{Ignore colour data} button is pressed, the stream lines are coloured
a uniform yellow colour irrespective of the settings of the other
controls on this panel. Otherwise the quantity selected using the \DemoWidget{Field
holding colour data} button is used to determine the colour at each point
of the streamline.

If the \DemoWidget{Use default scaling for colour data} button is pressed, the colour
scaling is set so that the minimum data value along any of the displayed
stream lines is shown as blue, and the maximum value is shown as red.

If the \DemoWidget{Use default scaling for colour data} button is {\em not}
lit,
the data values entered by the user in the two boxes at the bottom of the
panel are used to define blue and red. Any data values outside this range
will be shown as gaps in the stream lines.

If the specified field is a vector quantity then the magnitude of the
vector is used to determine the colour.  The common logarithm of the data
is used to define the colour instead of the data itself if the \DemoWidget{Use
common log of colour data} button is pressed.

\DemoPanel{Iso-surface control...}
If the \DemoWidget{Display iso-surface} button is pressed, an iso-surface is
displayed together with the stream lines. The iso-surface shows a surface
of constant value in the quantity selected using the \DemoWidget{Field to define
iso-surface} button, which may be a scalar or vector quantity.

If the \DemoWidget{Use default iso-surface value} button is pressed, the median
data value is used to define the iso-surface. Otherwise, the data value
entered in the \DemoWidget{Iso-surface data value} box is used. For vector fields
this should be the vector magnitude on the surface.

The surface has a uniform blue-grey colour, and it's opacity can be set
using the \DemoWidget{Iso-surface opacity} data entry box. A value of 0.0
produces a completely clear (i.e. invisible) surface, and a value 1.0
produces a completely opaque surface.

\DemoPanel{Auto rotation...}
If the \DemoWidget{Enable auto-rotation} button is pressed, each displayed image
is rotated in 3-D by the number of degrees given in the \DemoWidget{Rotation
speed} box relative to the previous image. If the \DemoWidget{Reset} button
is pressed the view point resets to the original view point.

The rotation is around a great circle centred on the to-point of the
current camera. This is usually the centre of the object, but this can be
changed using the view control facilities of the Image window.

\end{description}


\subsection{\label{scatter}\xlabel{scatter}Further details on using the {\bf scatter} network}
This demonstration allows the user to import and examine scattered
particle data. Each particle is defined by a position in 3-d
space and any number of additional data quantities, which may be vectors
or scalars.

Images are created which allow the user to view an arbitrary subset of
the particles from any point in space. Each particle may be presented by
a {\em glyph} consisting of a simple point, or a sphere (rendered in
3-D),
or a 3-d rocket-like object (useful for vector quantities). The size and shape of
each sphere or rocket can be determined by any arbitrary combination of
the particle's data values, its position, its distance from the camera,
etc.  The colour used to represent each particle can also be determined
independently in a similar arbitrary manner. Objects used to represent
particles may be made partially transparent so that crowded areas
appear brighter than sparsely populated areas.

A series of data sets can be imported and displayed automatically, and
selected particles can be followed from one frame to the next.
The scattered data can be binned or interpolated onto a regular grid
which can then be saved on disk for visualisation using other demonstration
networks.
The point from which the objects are viewed can be automatically rotated
in 3-D between successive frames to give a continuous rotation.
Any set of displayed images can be saved and encoded into an MPEG
animation.
A bounding box can optionally be displayed.

\begin{center}
{ \em Note, the view control facilities of the Image window are not
available with this network unless the camera update mode is set to
manual on the \Cpanel{Camera control...} panel. }
\end{center}

The following control panels are available:
\begin{description}

\DemoPanel{Select input file...}

This has been described in section \ref{SEC:RUNNING_DEMO}. In addition,
if you want to look at a sequence of files, you can include the string
``\%d'' in the file name. This string will be replaced by the current
frame number returned by the sequencer. This number starts at 1 and
increases up to the value given in the \DemoWidget{Max. frame number} data
entry box. To start the sequence playing, select ``Sequencer'' from the
``Execute'' menu on the Image window. This will pop up a new window
containing controls for the sequencer. These controls are similar to
those for a standard video recorder. Pressing the play button
(an arrow pointing to the right) will start the sequence.

\DemoPanel{MPEG control...}
This has been described in section \ref{SEC:RUNNING_DEMO}.

\DemoPanel{Camera control...}
This panel controls the projection between 3-d space and the 2-d screen.
This projection is determined by the current {\em camera}. The camera can be
controlled in several ways, determined by the setting of the \DemoWidget{Camera
update mode} button:

\begin{description}

\item [Auto:]
This is the default. The camera is chosen automatically to suit the
displayed object. This will be an off-diagonal (if the data is 3-d)
orthographic view from a distance such that the object fills most of the
screen. Each time the object changes, the camera is updated
automatically. Note, in this mode the view control facilities provided
by the Image window are inoperative.

\item[Manual:]
The camera is not updated automatically, but retains its parameters even
if the object is changed.  This mode allows the user to change the camera
manually using the view control facilities of the Image window.

\item[Freeze:]
The camera is frozen in its current state. The view remains the same even
if the viewed object changes, and the view control facilities of the
image window are inoperative.

\item[Auto-rotate:]
The view point is rotated in 3-D by the number of degrees given in the
\DemoWidget{Rotation speed} data entry box each time the network is executed, but
the camera is otherwise frozen. The view control facilities of the
Image window are inoperative.

\end{description}

\DemoPanel{Define data quantities...}
Several aspects of the displayed image (such as the colour and size of
the glyphs used to represent each particle) can be controlled by the
values stored in the imported data file. Up to three different quantities
can be used in this way, and these quantities are referred to as $a$, $b$
and $c$ within the demonstration. The correspondence between these names
and the quantities in the data file can be specified using the buttons on
this control panel.

\DemoPanel{Select positions to display...}
Particles may be excluded from the displayed image by entering a string
in the \DemoWidget{Selection expression} data entry box (remember to press
return after you have typed in the string). This string is an
algebraic expression which is passed to the \Modnam{Compute} module for
evaluation (see the DX {\em User's Reference Guide} for a description of the
syntax). It is evaluated for each particle, and only those particle for
which the resulting value is greater than zero are displayed.

The expression may refer to any of the following variables:

\begin{description}

\item[$a$, $b$, $c$ ] --
These are the imported data values specified on the \Cpanel{Define data
quantities...} control panel. They may be vector or scalar.

\item[$p$ ] --
This is the position of the current particle, padded with trailing zeros
if necessary to make it 3-d. It is a 3-d vector.

\item[$v$ ] --
This is the position in 3-d space from which the object is being viewed.
It is a 3-d vector.

\item[$f$ ] --
This is the current frame number determined by the sequencer. It is a scalar.

\item[$e$ ] --
This is the offset of the current particle from the start of the list of
particles. The first particle in the data file has $e=0$, the second has
$e=1$, etc. It is a scalar.

\end{description}

Here are some  example expressions:

\begin{quote}
\begin{description}

\item [``$1.0$''] -- gives every particle the value of 1.0.
Since this is greater than zero, every particle is displayed. This is the
default.

\item [``$0.0$'' ] -- causes no particles to be displayed.

\item [``$a-b$'' ] -- causes particles to be displayed only if the value of the $a$
quantity is greater than the $b$ quantity. These
quantities are defined on the \Cpanel{Define data quantities...} control panel.

\item [``$e<3$'' ] -- causes only the first 3 particles to be displayed.

\item [``$p.x>10$'' ] -- causes only particles with $x$ co-ordinates larger than 10 to be
displayed.

\item [``$mag(p-v)>10$ {\scriptsize \&\&} $a<0$'' ] -- displays particles which are further than 10 units
away from the observers view point (in the spatial co-ordinate system of
the data), and which also have an $a$ value less than zero.

\end{description}
\end{quote}

Particles may also be selected for display using the view control
facilities of the Image window. To do this you must first select manual
camera update mode on the \Cpanel{Camera control...} control panel. The
view point may then be adjusted manually by panning, zooming, navigating
through the data space, etc, until the view contains only the positions
in which you are interested. Then press \DemoWidget{Exclude non-visible
particles} and re-execute the network. This will exclude all particles
which do not currently fall within the bounds of the image window. You
will normally then press the \DemoWidget{Freeze particle selection} in order
to stop particles being re-selected as the view changes. You can then
change the view-point, and only the selected particles will remain in the
image.

Pressing the \DemoWidget{Freeze particle selection} button causes the current
selection of particles to be frozen. This is useful, for instance, if you
want to follow selected particle through a sequence of input files. If
you don't freeze the selection, different particles may be displayed from
each input file (for instance if the selection expression depends on the
data quantities in the file).

Pressing the \DemoWidget{Show no. of selected particles} button causes a window to
pop up telling you the number of selected particles.

\DemoPanel{Display control...}
This panel controls several extra objects which can be included in the
display along with the particle data.

The \DemoWidget{Bounding box} button determines if a bounding box is to be
displayed. If it is set to ``Current'' then a box which just encompasses
the currently selected positions is displayed. This box is updated as the
positions change. If it is set to ``Freeze'' then the box is not updated as
the positions change. If it is set to ``None'' then no bounding box is
displayed.

Selecting ``Yes'' from the \DemoWidget{Show colour bar} menu causes a colour
bar to be included showing the colour value associated with each
colour (as set by the \Cpanel{Colour control...} panel).
The scale of the labels on the colour bar may be changed by entering a
value in the \DemoWidget{Colour bar label scale} box.

Pressing the \DemoWidget{Show file name caption} button causes a caption
showing the current input file can be added to the display.

\DemoPanel{Colour control...}
The colour used to represent each selected particle is determined by a
two-stage process:

\begin{enumerate}

\item a scalar value known as the {\em colour value} is associated with each
particle.

\item these colour values are mapped onto actual colours using one of two
standard colour look-up tables (one gives a grey-scale image and the
other gives a colour image).

\end{enumerate}

In addition, the objects used to represent each particle may be made
partially transparent by setting an opacity value of less than 1.0. This
gives some idea of the density of particles (crowded areas with many
overlapping particles will look more opaque than sparsely populated
areas).

The colour value for each particle is determined by a text string entered
in the \DemoWidget{Colour value expression} data entry box which is an
algebraic expression passed to the \Modnam{Compute} module for
evaluation. It is evaluated for each particle, and gives the colour
value. If the supplied expression produces vector values, then the
modulus of the vector is used as the colour value. See the description of
the \Cpanel{Select positions to display...} control panel for the
variables which may be included in this expression.

Here are some  example expressions:

\begin{quote}
\begin{description}

\item [``$1.0$'' ] -- causes all particles to have a constant colour value of 1.0. This
will result in all particles having the same colour (as determined by
the colour look-up table). This is the default.

\item  [``$a$'' ] -- causes each particle's colour value to be equal to the $a$
 quantity defined on the \Cpanel{Define data quantities...} control panel.

\item [``$sin(a*b)$'' ] -- causes each particle's colour value to be the sine of the
product of the $a$ and $b$ quantities as defined in the \Cpanel{Define data
quantities...} control panel.

\item [``$1.0/mag(p-v)**2$'' ] -- causes each particle's colour value to be inversely
proportional to the square of the distance to the observer. This is
useful for getting an idea of the global structure of the data set,
especially when combined with the rotation facilities of the
\Cpanel{Camera control...} control panel and the ``View control'' options
of the Image window.

\end{description}
\end{quote}

In addition, a constant colour value may be assigned to all currently
visible particles (i.e. particles which are within the bounds of the
Image window) by pressing the \DemoWidget{Use specified constant colour value for
visible particles} button. The value to use is entered in the \DemoWidget{Colour
value for visible particles} data entry box. All particles which are
outside the bounds of the Image window retain the colour values
established by the \DemoWidget{Colour value expression} box. This facility can be
useful for following selected particles between successive frames in a
sequence. You would typically display the first frame, and then use the
``View control'' options of the Image window to adjust the view so that
only the particles of interest are visible. You then press the \DemoWidget{Use
specified constant colour value for visible particles} button, and
re-execute the network. This will assign the given colour value to the
visible particles. You would normally then press the \DemoWidget{Freeze colour
values} button (so that these colour values are not updated each time the
view changes) and then change the view to show a larger volume.

If the \DemoWidget{Show colour value statistics} button is pressed, a pop-up window is
created showing the mean, standard deviation, maximum, minimum and
median colour values for the current frame.

Having assigned a colour value to each particle, these are mapped into
actual colours using either a grey-scale or colour look-up table, as
selected by the \DemoWidget{Colour table} button.  The grey-scale table
uses black to represent the minimum colour value, and white for the maximum
colour value, with shades of grey in-between. The colour table uses blue
and red instead of black and white, with smoothly varying colours in-between.
A bar giving a key to the colour used for each colour value can be
displayed using the \Cpanel{Display control...} control panel. The
maximum and minimum colour values used by the table can either be
specified by the user (in which case the colour table remains fixed for
all frames), or default values can be used which covers the entire range
of the colour values of the current frame (in this case the colour table
may change between frames). This choice is made using the
\DemoWidget{Colour table range} button.  In ``User-supplied'' mode, the
user enters values into the \DemoWidget{Colour table minimum} and
\DemoWidget{Colour table maximum} data entry boxes.

\DemoPanel{Glyph control...}
Each particle may be represented by either a point, or a {\em glyph}, as
selected by the \DemoWidget{Marker type} button.  Glyphs are 3-d objects
which are rendered to show light and shade, and which get smaller as they
recede from the observer as a normal 3-d object would do (if a
perspective camera is being used). Spheres are used
to represent a scalar value at a given position, and rocket-like
structures are used to represent a vector value at a given position.
Points are single screen pixels which do not have a 3-d appearance.

The size of the glyph for each particle is defined using the string
entered in the \DemoWidget{Glyph size expression} data entry box. This string
is an algebraic expression which is passed to the \Modnam{Compute} module
for evaluation. It is evaluated for each particle and gives the size of
the associated glyph. If the expression evaluates to a scalar quantity
then spherical glyphs are used with radius given by the expression. If
the expression evaluates to a vector quantity then rocket glyphs are
used with length given by the expression. See the description of the
\Cpanel{Select positions to display...} control panel for the variables which
may be included in this expression.

Here are some  example expressions:

\begin{quote}
\begin{description}

\item [``$1.0$'' ] -- causes all particles to be displayed with the same sized spherical glyph.
This is the default.

\item [``$a$'' ] -- causes all particles to be displayed as a glyph with size (and direction
if $a$ is a vector) given by the $a$ quantity.

\item [``\mbox{$[a,b,c]$}'' ] -- causes particles to be represented as
rocket glyphs with
direction and length determined by the three quantities $a$, $b$ and $c$
which should be scalar values.

\end{description}
\end{quote}

The resulting glyph sizes are measured in the data's spatial co-ordinate
system, and may be modified by a constant scale factor by entering the
factor in the \DemoWidget{Glyph scale factor} data entry box, and ensuring
that the \DemoWidget{Use default glyph scale} button is not pressed. If this
button {\em is} pressed, the value in the \DemoWidget{Glyph scale factor} box is
ignored and a default value is used which attempts to avoid either very
large or very small glyphs. The thickness of rocket glyphs can be
modified by entering a factor in the \DemoWidget{Glyph thickness} box.

A constant glyph size can be assigned to all particles which are
currently within the bounds of the Image window by entering the value in
the \DemoWidget{Constant size for visible glyphs} box, pressing the
\DemoWidget{Use constant size for visible glyphs} button and re-executing the
network. You can press the \DemoWidget{Freeze glyph sizes} button if you want
to retain the current glyph sizes between successive frames.

\DemoPanel{Export gridded data...}
If the \DemoWidget{Save gridded data} button is pressed, then the data specified by the
\DemoWidget{Data to grid} button is converted to a regular grid, and exported to a
disk file in native DX format. The button is automatically reset.

The name of the disk file is given in the \DemoWidget{Output filename} box. The
supplied string may contain a single occurrence of the sub-string ``\%d''
which will be replaced with the current frame number (see the \Cpanel{Select
input file...} control panel).

The \DemoWidget{Data to grid} button determines which data is exported.
Selecting ``Selected data'' causes only the positions selected using the
\Cpanel{Select positions to display...} control panel to be exported.
Selecting ``Input data'' causes the entire input data file to be exported.

The grid onto which the scattered data is mapped encompasses the exported
data, and each cell is cubic with dimension given by \DemoWidget{Grid cell
dimension}. If zero is given for the grid cell dimension, then a
replacement value is chosen which gives the required number of cells
along the $x$ axis, as entered in the \DemoWidget{X axis cell count} box.

The data values stored at each output grid point are determined by the
setting of the \DemoWidget{Data to grid} button and the \DemoWidget{Output data}
button:
\begin{itemize}

\item If \DemoWidget{Output data} is set to ``Density values'', then each output
grid point holds a single scalar value which is an estimate of the
density of particles in the neighbourhood of the grid point.

\item If \DemoWidget{Output data} is set to ``Mean data values'' and
\DemoWidget{Data to grid} is set to ``Selected data'', then each grid point contains
a single data value which is an estimate of the mean glyph size value
of the particles in the neighbourhood of the grid point (see control
panel \Cpanel{Glyph control...}).

\item If  \DemoWidget{Output data} is set to ``Mean data values'' and
\DemoWidget{Data to grid} is set to ``Input data'', then each grid point has the
same number of data values as each input particle, and each output data
value is an estimate of the mean of the corresponding input data value in
the neighbourhood of the output grid point.

\end{itemize}

There are two ways of obtaining these estimates, selected by the
\DemoWidget{Gridding method} button:

\begin{description}

\item [Sampling:]
The input data is interpolated at the positions of the output grid
points. For each output grid point, the nearest $N$ input particles are
identified (where $N$ is given by the value in the \DemoWidget{No. of
neighbours to use} box). The output data values are either the mean of
these input data values (if \DemoWidget{Output data} is set to ``Mean data
values''), or an estimate of the density of particles in the neighbourhood
(if \DemoWidget{Output data} is set to ``Density values''). The output data are
{\em position-dependent}. Note, this method is very slow. Progress reports
are given in the DX message window showing how far the re-gridding has to
go.

\item [Binning:]
The input particles are binned into the output grid. Each output grid
cell contains either the mean of the particle data values in the cell, or
the density of particles in the cell (dependent on the setting of
\DemoWidget{Output data}). The output data are {\em connection-dependent}.

\end{description}

\end{description}

\subsection{A detailed look inside the {\bf iso} demonstration network}
The SX demonstration networks attempt to be reasonably general in terms of
the data they can be used with and the options they provide. This
necessarily causes them to be more complex than networks written to
perform specific visualisations of particular data files. For instance,
displaying a simple iso-surface can be done using just three modules
(\Modnam{Import}, \Modnam{Isosurface}, and \Modnam{Image}), whereas the ``iso''
demonstration contains 108 modules! Generality is bought at the expense
of much greater complexity, and this complexity results in there being
many more things to go wrong! It is usually a good idea to make your
networks as specific as possible.

By their nature, networks are difficult to structure and document. In an
attempt to ease this problem, each network has been divided into several
sections, which (to a greater or lesser degree) perform separate,
well-defined tasks. If you examine the networks using DX, you will find
these sections placed side by side, working from left to right in the
network editor window.

To illustrate the way that these networks are written, the ``iso''
network is described in detail in the following pages. Each section is
presented graphically, with a description of what the section does on
the following page.

\DemoNet{iso}{1}{
\begin{picture}(65,130)
\put(32,118){\makebox(3,3){1}}
\put(0,101){\makebox(3,3){2}}
\put(0,84) {\makebox(3,3){3}}
\put(23,67) {\makebox(3,3){4}}
\put(26,47){\makebox(3,3){5}}
\put(25,29){\makebox(3,3){6}}
\put(0,17){\makebox(3,3){7}}
\put(15,23){\makebox(3,3){8}}
\put(43,17){\makebox(3,3){9}}
\put(0,140) {\special{psfile=sun203_iso_1.ps}}
\end{picture}}

\DemoDesc{iso}{1}{Issue initial help and obtain the input file.}{
If the network is executed before an input file name has been given,
a pop-up window containing help information is displayed, and the rest of the
network is prevented from executing. If an input file name {\em has} been
given, the full file specification and the file base name are passed on to
the rest of the network, which then executes normally.
}{
\DemoMod{FileSelector}{This module returns the file name entered by the
user in the ``Input file name:'' widget in the ``Select input file...''
control panel. The left output is the full file specification and the
right output is the file base name. If no file has been given, then a
{\tt NULL} value is returned for both outputs.}

\DemoMod{Inquire}{Uses the ``is null'' enquiry to see if a {\tt NULL}
file specification was returned by module 1 (\Modnam{FileSelector}). The output
is set to 1 if this is the case (i.e. no input file was specified), and zero
otherwise.}

\DemoMod{Compute}{Adds 1 onto the output from module 2 (\Modnam{Inquire}) putting
it in the range [1,2], suitable for use by module 4 (\Modnam{Route}).}

\DemoMod{Route}{Routes execution to the left output if module 3 (\Modnam{Compute})
outputs 1, or to the right output if it outputs 2. The file
specification obtained by module 1 (\Modnam{FileSelector}) is passed on to
the selected output. The result of this is that if a file name has been
given, then it is passed on to module 8 (\Modnam{\em filespec}). If no file
name has been given, then a {\tt NULL} value is passed on to module 5
(\Modnam{Format}). None of the modules which are dependent on the un-selected
output are executed, with the exception of any \Modnam{Collect} modules
(which are always executed).}

\DemoMod{Format}{Creates a string holding the help message. This module
is only executed if module 3 (\Modnam{Compute}) outputs 2 (i.e. if no data file
has been given). The {\tt NULL} value routed to its right input is used only
to trigger execution of the module. It does not form part of the string, which
is permanently assigned to its left input (the \Param{template}
parameter) using the
module's configuration dialogue box.}

\DemoMod{Message}{Creates a pop-up window containing the string output by
module 5 (\Modnam{Format}). Execution of the network stops when this has
been done (except for any \Modnam{Collect} modules in other sections).}

\DemoMod{\em kill}{ This is a \Modnam{Transmitter} module to which the name
``{\em kill}'' has been given using its configuration dialogue box. It
transmits the value returned by module 3 (\Modnam{Compute}) to corresponding
\Modnam{Receiver} modules in other sections of the network. This value is
used to explicitly stop execution of (or ``kill'') modules down-stream of
any \Modnam{Collect} modules, if no input file has been given. This is
necessary because the \Modnam{Collect} module is always executed, even if none
of its inputs have been executed.}

\DemoMod{\em filespec}{ This is a \Modnam{Transmitter} module to which the
name ``{\em filespec}'' has been given. It transmits the full file
specification supplied by the user to corresponding \Modnam{Receiver}
modules in other sections of the network.}

\DemoMod{\em filename}{ This is a \Modnam{Transmitter} module which transmits
the base file name to corresponding \Modnam{Receiver} modules in other sections
of the network.}
}

\DemoNet{iso}{2}{
\begin{picture}(100,170)
\put(0,158){\makebox(3,3){1}}
\put(40,158){\makebox(3,3){2}}
\put(81,158){\makebox(3,3){3}}
\put(17,137){\makebox(3,3){4}}
\put(37,137){\makebox(3,3){5}}
\put(58,137){\makebox(3,3){6}}
\put(81,137){\makebox(3,3){7}}
\put(30,108){\makebox(3,3){8}}
\put(4,86){\makebox(3,3){9}}
\put(50,91){\makebox(3,3){10}}
\put(70,93){\makebox(3,3){11}}
\put(53,75){\makebox(3,3){12}}
\put(3,63){\makebox(3,3){13}}
\put(32,70){\makebox(3,3){14}}
\put(80,59){\makebox(3,3){15}}
\put(49,47){\makebox(3,3){16}}
\put(14,32){\makebox(3,3){17}}
\put(14,5){\makebox(3,3){18}}

\put(0,180) {\special{psfile=sun203_iso_2.ps}}
\end{picture}
}
\DemoDesc{iso}{2}{Import the data and ensure it is a group structure.}{
The data in the input file is imported, using a module appropriate for the
file's format. The imported data may be structured in several
ways. For instance, it may contain several quantities (or ``fields'' in
DX parlance) in a ``group'' structure, or it may contain just a single
quantity. If it contains only a single quantity, the field may or may not
have an associated name. The modules down-stream of module 8
(\Modnam{Switch}) ensure that the data is passed on to the rest of the
network in a consistent way so that further checks on the data structure
can be avoided. If the input file contains only a single field, a new
empty group structure is created and the field is put in it. If the field
has no name it is given a name (the base file name). The rest of the
network can then assume that the data is a group of named fields.
}{
\DemoMod{Selector}{This module returns an integer in the range [1,4]
identifying the data format selected by the user using the ``Data format''
widget in the ``Select input file...'' control panel.}

\DemoMod{\em filespec}{A \Modnam{Receiver} module which picks up and returns
the ``{\em filespec}'' value transmitted by section 1 of the network.
This value is the full specification of the input data file.}

\DemoMod{\em filename}{Returns the base file name transmitted by section 1 of
the network.}

\DemoMod{Import}{Imports the data assuming it is in DX ``native'' format.}

\DemoMod{Import}{Imports the data assuming it is in DX ``general'' format.}

\DemoMod{SXReadFortran}{Imports the data assuming it is binary data
created by a Fortran program, in DX ``general'' format.}

\DemoMod{SXReadNDF}{Imports the data assuming it is a Starlink NDF.}

\DemoMod{Switch}{Routes the output from the required data reader through
to the rest of the network. Un-required inputs are not calculated.}

\DemoMod{Inquire}{Inquires if the input data is a group
of fields.}

\DemoMod{Inquire}{Inquires if the input data (assuming it is a
single field) has a name.}

\DemoMod{Attribute}{Obtains the name of the input field
(assuming it has one).}

\DemoMod{Compute}{Adds 1 to  the output from module 10 (\Modnam{Inquire})
so that it is in the range [1,2] and can thus be used to select one of the
two inputs to module 15 (\Modnam{Switch}).}

\DemoMod{Compute}{Adds 1 to  the output from module 9 (\Modnam{Inquire})
so that it is in the range [1,2] and can thus be used to select one of the
two inputs to module 17 (\Modnam{Switch}).}

\DemoMod{Collect}{Creates an empty group to which the input field
can be appended.}

\DemoMod{Switch}{Passes through the name of the input field (if it has
one), or the base name of the input file otherwise.}

\DemoMod{Append}{Puts the input field into a group, giving it the
name passed on by module 15 (\Modnam{Switch}).}

\DemoMod{Switch}{If the input data is a group of fields, it is passed
through to the output. If the input data is a single field, the newly created
group containing the input data is passed through to the output.}

\DemoMod{\em input\_data}{Transmits the input data in the format assumed by the rest
of the network.}

}



\DemoNet{iso}{3}{
\begin{picture}(80,200)
\put(0,220) {\special{psfile=sun203_iso_3.ps}}
\put(26,195){\makebox(3,3){1}}
\put(50,177){\makebox(3,3){2}}
\put(29,160){\makebox(3,3){3}}
\put(44,144){\makebox(3,3){4}}
\put(58,144){\makebox(3,3){5}}
\put(51,120){\makebox(3,3){6}}
\put(2,103){\makebox(3,3){7}}
\put(63,103){\makebox(3,3){8}}
\put(18,72){\makebox(3,3){9}}
\put(32,53){\makebox(3,3){10}}
\put(53,53){\makebox(3,3){11}}
\put(66,53){\makebox(3,3){12}}
\put(16,29){\makebox(3,3){13}}
\put(1,16){\makebox(3,3){14}}
\put(38,11){\makebox(3,3){15}}
\put(52,15){\makebox(3,3){16}}
\end{picture}
}
\DemoDesc{iso}{3}{Create the iso-surface.}{
The user specifies which of the fields in the input file is to be used to
define the iso-surface. This field is extracted, and optionally replaced
by a field holding the common logarithm of the data values.
The iso-surface is then created and passed on to later
sections of the network.
}{
\DemoMod{\em input\_data}{Returns the input data transmitted by section 2
of the network. This will always be a group of named fields. }

\DemoMod{Selector}{Returns the name of the field to be used to define the
iso-surface. This is specified by the user on the ``Select field to
iso-surface'' control panel. The module automatically reads the names
of all the fields in the input data group and supplies them as options on
the control panel.}

\DemoMod{Select}{Extracts the selected field from the input data
group.}

\DemoMod{Include}{Removes all zero or negative data values from
the selected field, so that the logarithm of the data can safely be taken.}

\DemoMod{Format}{Creates a string of the form ``{\tt Log(<name>)}''
where {\tt <name>} is the name of the field.}

\DemoMod{Compute}{Takes the common logarithm of the data.}

\DemoMod{Toggle}{Returns 1 if the user has pressed the ``Take common
log of data'' button on the ``Select field to iso-surface'' control
panel, and 2 otherwise.}

\DemoMod{Options}{Assigns the string created by module 5
(\Modnam{Format}) as the new name of the field.}

\DemoMod{Switch}{Passes on either the original data, or the logarithm of
the data, as determined by module 7 (\Modnam{Toggle}).}

\DemoMod{\em iso\_value}{Receives the data value which is to
be used to define the iso-surface. This is transmitted by section 6 of
the network.}

\DemoMod{Gradient}{Calculates the gradient of the data. \Modnam{IsoSurface} has an option to calculate
the gradient itself. But it is more efficient to calculate it once
using the \Modnam{Gradient} module, than to allow \Modnam{IsoSurface} to
re-calculate it every time a new iso-surface is required.}

\DemoMod{Attribute}{Obtains the name of the data field. This may have
been modified to indicate that the logarithm of the data has been taken.}

\DemoMod{IsoSurface}{Creates the iso-surface structure. Its inputs are
the data array, the data value to define the iso-surface, and the
gradient of the data array.}

\DemoMod{\em iso\_field}{Transmits the field which was used to define the
iso-surface for later use.}

\DemoMod{\em isosurface}{Transmits the iso-surface to the rest of the
network.}

\DemoMod{\em iso\_name}{Transmits the name of the field used to define
the iso-surface for later use.}

}




\DemoNet{iso}{4}{
\begin{picture}(100,220)
\put(0,240) {\special{psfile=sun203_iso_4.ps}}
\put(0,213){\makebox(3,3){1}}
\put(38,213){\makebox(3,3){2}}
\put(46,197){\makebox(3,3){3}}
\put(22,181){\makebox(3,3){4}}
\put(48,172){\makebox(3,3){5}}
\put(58,172){\makebox(3,3){6}}
\put(50,155){\makebox(3,3){7}}
\put(18,129){\makebox(3,3){8}}
\put(45,133){\makebox(3,3){9}}
\put(36,113){\makebox(3,3){10}}
\put(2,93){\makebox(3,3){11}}
\put(25,98){\makebox(3,3){12}}
\put(45,93){\makebox(3,3){13}}
\put(58,93){\makebox(3,3){14}}
\put(28,77){\makebox(3,3){15}}
\put(56,73){\makebox(3,3){16}}
\put(24,56){\makebox(3,3){17}}
\put(56,56){\makebox(3,3){18}}
\put(95,56){\makebox(3,3){19}}
\put(4,36){\makebox(3,3){20}}
\put(34,36){\makebox(3,3){21}}
\put(78,36){\makebox(3,3){22}}
\put(10,20){\makebox(3,3){23}}
\put(5,4){\makebox(3,3){24}}
\end{picture}
}

\DemoDesc{iso}{4}{Colour the iso-surface created by section 3.}{

This section colours the iso-surface using the data value (or optionally
the logarithm of the data value) in another specified field to determine
the colour at each point. The iso-surface is also given a constant opacity.
If the user requests a constant colour for the iso-surface, then the output
from this section (i.e. from module 24) is not needed, and so none of this
section is executed (except for those parts which are needed to execute the
\Modnam{System} module 22). }{

\DemoMod{\em isosurface}{Receives the geometry field describing the
iso-surface, transmitted by section 3.}

\DemoMod{\em input\_data}{Receives the input data group, transmitted by
section 2.}

\DemoMod{Selector}{Obtains the name of the field to be used to colour the
iso-surface. This is specified by the user on the ``Surface colouring...''
control panel.}

\DemoMod{Select}{Extracts the selected field from the input data group.}

\DemoMod{Include}{Removes all zero or negative data values from
the selected field.}

\DemoMod{Format}{Creates a string of the form ``{\tt Log(<field name>)}''.}

\DemoMod{Compute}{Takes the common logarithm of the data.}

\DemoMod{Toggle}{Returns 1 if the user has pressed the ``Take common
log of data'' button on the ``Surface colouring...'' control panel, and 2
otherwise.}

\DemoMod{Options}{Assigns the string created by module 6
(\Modnam{Format}) as the new name of the field.}

\DemoMod{Switch}{Passes on either the original data, or the logarithm of
the data.}

\DemoMod{Map}{For each position on the iso-surface, the data value at
the corresponding position in the field passed on by module 10
(\Modnam{Switch}) is found. These data values, together with the iso-surface
positions, are passed on to module 20 (\Modnam{Color}).}

\DemoMod{Attribute}{The name of the field defining the iso-surface colour
is obtained.}

\DemoMod{Scalar}{Obtains the minimum data value to be displayed, as
entered by the user in the ``Surface colouring...'' control panel. The
value is constrained to be in the range of the field passed on by module
10 (\Modnam{Switch}).}

\DemoMod{Scalar}{Obtains the maximum data value to be displayed, as
entered by the user in the ``Surface colouring...'' control panel.}

\DemoMod{Compute}{Returns the minimum of the values passed on by modules
13 and 14 (\Modnam{Scalar}). This is done in case the user gave the values
the wrong way round. The returned value is displayed as blue.}

\DemoMod{Compute}{Returns the maximum of the values passed on by modules
13 and 14 (\Modnam{Scalar}). The returned value is displayed as red.}

\DemoMod{\em opacity}{Receives the opacity to be given to the
iso-surface, transmitted by section 5.}

\DemoMod{Colormap}{Creates a colour map which maps the supplied minimum
and maximum data values on to blue and red.}

\DemoMod{Format}{Formats a blank string. See module 22 (\Modnam{System}).}

\DemoMod{Color}{Associates a colour and opacity with each of the mapped data
values on the iso-surface.}

\DemoMod{ColorBar}{Creates a vertical coloured bar annotated with associated
data values, and labelled with the name of the colour field passed on by
module 12 (\Modnam{Attribute}).}

\DemoMod{System}{Executes the blank string passed on by module 19
(\Modnam{Format}) as an operating system command. A blank command does
nothing of course, so you may wonder why this is done. In fact it is a
trick which can be used to ensure that certain modules get executed even
if their outputs are not required. DX assumes that all operating system
commands executed using a \Modnam{System} module, may potentially have an
effect on the final visualisation, and therefore \Modnam{System} modules
are always executed if possible. This means that all modules involved in
creating the input command string are also executed. In the current case,
this includes the two \Modnam{Scalar} modules 13 and 14 which are used to obtain
the data limits for colouring the iso-surface. These modules only allow the
user to enter values which are in the range of their input data, {\em
i.e.} the data passed on by module 10 (\Modnam{Switch}). These limits are
re-calculated every time the modules are executed. Thus, the inclusion of
the \Modnam{System} module (22) ensures that these limits are always
re-calculated, and are thus always consistent with the data selected
using module 3 (\Modnam{Selector}). If this were not done, the user would be
free to enter inappropriate values if the colouring field had been
changed while displaying an un-coloured iso-surface.}

\DemoMod{Collect}{Collects the coloured iso-surface and the colour bar
into a single object which can be displayed.}

\DemoMod{\em coloured\_iso}{Transmits the annotated coloured iso-surface
on to the rest of the network.}

}








\DemoNet{iso}{5}{
\begin{picture}(112,187)
\put(0,187) {\special{psfile=sun203_iso_5.ps}}
\put(11,163){\makebox(3,3){1}}
\put(43,170){\makebox(3,3){2}}
\put(71,170){\makebox(3,3){3}}
\put(91,163){\makebox(3,3){4}}
\put(5,130){\makebox(3,3){5}}
\put(44,130){\makebox(3,3){6}}
\put(69,135){\makebox(3,3){7}}
\put(69,117){\makebox(3,3){8}}
\put(23,112){\makebox(3,3){9}}
\put(24,93){\makebox(3,3){10}}
\put(29,75){\makebox(3,3){11}}
\put(24,61){\makebox(3,3){12}}
\put(90,51){\makebox(3,3){13}}
\put(25,42){\makebox(3,3){14}}
\put(30,23){\makebox(3,3){15}}
\put(16,5){\makebox(3,3){16}}
\put(57,5){\makebox(3,3){17}}
\end{picture}
}

\DemoDesc{iso}{5}{Add a caption and display the iso-surface}{
This section selects whether to display the iso-surface with variable or
constant colour, adds a caption to the display, and displays it. It also
sets the opacity of the surface.
}{

\DemoMod{Toggle}{Returns 1 if the user has requested that the iso-surface
should be coloured using a second field, and 2 if the iso-surface is to
have a constant colour. This is determined by the state of a button on the
``Surface colouring...'' control panel.}

\DemoMod{\em coloured\_iso}{Receives the coloured iso-surface transmitted
by section 4.}

\DemoMod{\em isosurface}{Receives the original iso-surface transmitted
by section 3, which has a constant blue-grey colour.}

\DemoMod{Scalar}{Obtains the opacity for the iso-surface. Zero produces a
completely clear surface, and 1.0 produces a completely opaque
surface. This is determined by the ``Surface Opacity'' widget on the
``Surface colouring...'' control panel.}

\DemoMod{\em iso\_name}{Receives the name of the field used to define the shape
of the iso-surface, transmitted by section 3.}

\DemoMod{\em iso\_value}{Receives the data value used to define the shape
of the iso-surface, transmitted by section 6.}

\DemoMod{Color}{Assigns the given opacity to the iso-surface. The original
colour (blue-grey) is unchanged.}

\DemoMod{Switch}{Passes on the iso-surface with the colouring determined
by module 1 (\Modnam{Toggle}), either variable or constant. Modules
connected to the un-required input are not executed.}

\DemoMod{Format}{Formats a string identifying the quantity determining
the iso-surface shape, and its value.}

\DemoMod{Caption}{Creates a graphical caption containing the string
created by module 9 (\Modnam{Format}). This caption is ``fixed'' to the
screen (i.e. it does not move as the viewing position is moved or
rotated).}

\DemoMod{Collect}{The caption and the iso-surface are collected into a
single object. Note, this module will execute even if the modules
creating its inputs have not been executed. This happens if the user
attempts to execute the network without first specifying an input file
(see modules 4 and 7 in section 1).}

\DemoMod{\em kill}{Receives a flag from section 1 indicating if the execution
of the rest of the network should be prevented, because of the user not
giving an input file.}

\DemoMod{\em opacity}{Transmits the requested opacity to section 4.}

\DemoMod{Route}{Execution is routed to module 15 (\Modnam{Image}) and
subsequent modules so long as the user has given an input file. Otherwise,
none of the remaining modules are executed.}

\DemoMod{Image}{The iso-surface is displayed (at long last!).}

\DemoMod{\em render}{The final renderable object is transmitted to
section 7, which is responsible for creating movies.}

\DemoMod{\em camera}{The current camera structure is transmitted to section 7.
The camera structure contains details of how the object is projected onto
the screen (viewing position and direction, screen resolution, etc). It
reflects the use of the ``View Control'' options item in the \Modnam{Image}
window.}

}




\DemoNet{iso}{6}{
\begin{picture}(120,121)
\put(0,131) {\special{psfile=sun203_iso_6.ps}}
\put(42,111){\makebox(3,3){1}}
\put(27,90){\makebox(3,3){2}}
\put(57,97){\makebox(3,3){3}}
\put(74,97){\makebox(3,3){4}}
\put(84,97){\makebox(3,3){5}}
\put(95,90){\makebox(3,3){6}}
\put(97,74){\makebox(3,3){7}}
\put(16,57){\makebox(3,3){8}}
\put(79,58){\makebox(3,3){9}}
\put(45,39){\makebox(3,3){10}}
\put(63,39){\makebox(3,3){11}}
\put(99,33){\makebox(3,3){12}}
\put(1,25){\makebox(3,3){13}}
\put(36,8){\makebox(3,3){14}}
\put(66,17){\makebox(3,3){15}}
\end{picture}
}

\DemoDesc{iso}{6}{Obtain the data value for the iso-surface}{
This section determines the data value to define the iso-surface. It can
either be the median of the data, a user-supplied value, or an element
from a sequence of values.
}{

\DemoMod{\em iso\_field}{Receives the field data used to create the
iso-surface, transmitted by section 3. This could potentially be the
logarithm of the data imported from the input file.}

\DemoMod{Histogram}{Calculates the median of the data values in the
field used to create the iso-surface.}

\DemoMod{Scalar}{Obtains the user-supplied data value which is to be used
to defined the iso-surface. This is entered in the ``Set iso-surface
values...'' control panel. The supplied value is constrained to lie
within the data range of the specified field (as received by module 1).}

\DemoMod{Scalar}{Obtains the starting data value if a sequence of
iso-surfaces are to be displayed. This is constrained to lie within the
data range of the input field. It is entered in the ``Set iso-surface
values...'' control panel.}

\DemoMod{Scalar}{Obtains the finishing data value if a sequence of
iso-surfaces are to be displayed. This is constrained to lie within the
data range of the input field. It is entered in the ``Set iso-surface
values...'' control panel.}

\DemoMod{Integer}{Obtains the number of iso-surfaces to display in a
sequence. The data values for each are evenly spread between the values
obtained by modules 4 and 5 (\Modnam{Scalar}).}

\DemoMod{Sequencer}{The sequencer causes the entire network to be
repeatedly re-executed (not just this section). A new iso-surface is
displayed each time. The number of re-executions is given by module 6
(\Modnam{Integer}). The output is an integer index identifying the current
execution. The sequence of re-executions is initiated by pressing the
``Play'' button in the Sequencer control panel.}

\DemoMod{Selector}{Obtains an integer identifying the method to use to select
the iso-surface data value, as specified by the user using the ``Value
selection method'' button on the ``Set Iso-surface values...'' control panel.
A value of 1 is returned if the default (median) value is to be used as
found by module 2 (\Modnam{Histogram}). A value of 2 is returned if the
user-supplied value obtained by module 3 (\Modnam{Scalar}) is to be used. A
value of 3 is returned if the sequence of iso-surfaces defined by modules
4, 5 and 6 is to be displayed.}

\DemoMod{Compute}{Calculates the data value for the current iso-surface
in the sequence.}

\DemoMod{Compute}{Outputs 1 if the value obtained by module 8
(\Modnam{Selector}) is 3 (i.e. a sequence of iso-surfaces is to be
displayed), or zero otherwise.}

\DemoMod{Switch}{Selects the current iso-surface value on the basis of
the value returned by module 8 (\Modnam{Selector}). }

\DemoMod{Format}{Formats a blank string. Note, the blank string is
permanently assigned to the left hand input tab using the module's
configuration dialogue box. The other input tabs are only connected in
order to trigger execution of the module. The values they provide are not
actually used.}

\DemoMod{ManageSequencer}{Ensures that the sequencer control panel is
visible if the user wants to display a sequence of iso-surfaces, and closes
it otherwise. Note, the user could do this manually by selecting the
``Sequencer'' item from the ``Execute'' menu. Including this module
removes the need for user-intervention.}

\DemoMod{\em iso\_value}{Transmits the data value required for the
current iso-surface.}

\DemoMod{System}{Executes the blank string passed on by module 12
(\Modnam{Format}) as an operating system command. This is done to ensure that
modules 2, 3, 4, 5 and 6 are always executed and thus kept consistent with any
change in input data. This is necessary because module 11 (\Modnam{Switch}) would
otherwise suppress the execution of some of these modules. See module 22 in
section 4 for further discussion of this ``trick''.}

}



\DemoNet{iso}{7}{
\begin{picture}(66,118)
\put(0,130) {\special{psfile=sun203_iso_7.ps}}
\put(2,108){\makebox(3,3){1}}
\put(35,108){\makebox(3,3){2}}
\put(10,87){\makebox(3,3){3}}
\put(26,66){\makebox(3,3){4}}
\put(41,66){\makebox(3,3){5}}
\put(25,44){\makebox(3,3){6}}
\put(32,23){\makebox(3,3){7}}
\put(53,23){\makebox(3,3){8}}
\put(22,6){\makebox(3,3){9}}
\end{picture}
}

\DemoDesc{iso}{7}{Gather the frames for a movie and encode them into an MPEG file.}{
The displayed image is re-created incorporating any axes, rotation,
shift, etc, introduced by the use of the \Modnam{Image} window controls.
This image is saved on disk as a ``PPM'' file by the \htmlref{\Modnam{SXMakeMpeg}}{SXMakeMpeg}
macro if the user has enabled the saving of images. When all images have
been saved, they are encoded into an MPEG file using the Berkeley
``{\tt mpeg\_encode}'' software, which should be installed on the host
system.
}{

\DemoMod{\em render}{Receives the displayed object, transmitted by
section 5.}

\DemoMod{\em camera}{Receives information describing the position from
which the displayed object is viewed, transmitted by section 5.}

\DemoMod{Render}{Projects the object received by module 1 (\Modnam{\em render})
into a 2-d image, viewed in the way described by the camera
received by module 2 (\Modnam{\em camera}). This image will be a copy of the
image displayed in the \Modnam{Image} window.}

\DemoMod{Toggle}{Obtains the setting of the ``Save frames'' button on the
``MPEG control'' control panel. It returns 1 if the button is pressed
(i.e. if the current displayed image is to be saved so that it can become
the next frame in an MPEG movie), or zero otherwise.}

\DemoMod{Reset}{Obtains the setting of the ``Create MPEG'' button on the
``MPEG control'' control panel. It returns 1 if the button is pressed
(i.e. if the current collection of saved frames is to be encoded into an
MPEG file), or zero otherwise. After execution of the module, the button
is automatically reset to its ``released'' position. }

\DemoMod{Compute}{Returns 1 if either the ``Save frames'' or the ``Create
MPEG'' button is pressed, and zero if neither of them are pressed.}

\DemoMod{Route}{Causes execution to be routed to module 9
(\htmlref{\Modnam{SXMakeMpeg}}{SXMakeMpeg}) only if one of the two buttons have been pressed, i.e.
if module 6 (\Modnam{Compute}) outputs 1. Otherwise, module 9 is not
executed.}

\DemoMod{String}{Obtains the name of the file in which to store the
encoded MPEG movie. This is supplied by the user on the ``MPEG control''
control panel.}

\DemoMod{SXMakeMpeg}{This either saves the current image as a ``PPM''
file on disk, or encodes the current collection of PPM files into an MPEG
movie. It is a macro, and expects the Berkeley MPEG encoder
(``{\tt mpeg\_encode}'') to be available on the host system.}

}



%=================================================================


\newpage
\section{Acknowledgements \xlabel{ACK} }

Ewan~Brown explained the intricacies of the various Unix commands for
displaying the amount of swap space available. Alan Chipperfield advised
on the configuration and operation of the Starlink CONVERT package.
Mike~Lawden  made several useful comments on a draft version of the
document.

\addcontentsline{toc}{section}{References}
\begin{thebibliography}{99}

  \bibitem{QUICKS} {\it IBM Visualization Data Explorer: QuickStart
   Guide}, Second Edition, 1995, document reference number:
   SC34-3262-01.

  \bibitem{USERG} {\it IBM Visualization Data Explorer: User's Guide},
   Sixth Edition, 1995, document reference number: SC38-0496-05.

  \bibitem{USERR} {\it IBM Visualization Data Explorer: User's Reference},
   Third Edition, September 1995, document reference number:
   SC38-0486-02.

  \bibitem{PROGR} {\it IBM Visualization Data Explorer: Programmer's
   Reference}, Sixth Edition, 1995, document reference number:
   SC38-0497-5.

  \bibitem{SC2} SC/2.3 {\it The DX Cookbook}\, by A.C.~Davenhall,
   1 October 1997 (Starlink).

  \bibitem{SG8} SG/8.2 {\it An Introduction to Visualisation Software
   for Astronomy}\, by A.C.~Davenhall, 4th March 1997 (Starlink).

  \bibitem{SUN33} SUN/33.4 {\it NDF Routines for Accessing the
   Extensible N-Dimensional Data Format}\, by R.F.~Warren-Smith,
   21 March 1995 (Starlink).

  \bibitem{SUN55} SUN/55.9 {\it CONVERT A Format-conversion Package}\,
   by M.J.~Currie, G.J.~Privett and A.J.~Chipperfield, 7 December 1995
   (Starlink).

  \bibitem{SUN95} SUN/95.9 {\it KAPPA --- Kernel Application Package}\,
   by M.J.~Currie, 9 December 1995 (Starlink).

\end{thebibliography}

\newpage
\appendix
\section{The ndf2dx Conversion Utility \label{NDF2DX} \xlabel{NDF2DX} }

This appendix describes {\tt ndf2dx} in detail.

\sstroutine{
   NDF2DX
}{
   Convert a Starlink NDF structure into a DX native data file
}{
   \sstdescription{
      NDF2DX converts a 1, 2 or 3 dimensional Starlink NDF
      structure to a native format DX data file containing a single
      field with the following array components:
      \begin{description}
      \item [{\tt data}] -- This is copied from the DATA component of the
      NDF. It is given a ``dep'' attribute indicating whether the values
      in it are in one-to-one correspondence with the {\tt positions}
      component, or the {\tt connections} component (see parameter DEP). Bad
      values are
      replaced by zero, and identified in the {\tt invalid positions} or
      {\tt invalid connections} components. The created component is
      scalar and of type {\tt int}, {\tt float} or {\tt double},
      depending on the data type of the NDF.
      \item [{\tt positions}] -- This is derived from the AXIS
      component of the NDF and is stored as a product of regular arrays
      if possible. Default values are used if the AXIS
      structure is undefined, in which case each pixel centre is given the
      corresponding pixel co-ordinates. If the user indicates that
      the data are positions-dependent (see parameter DEP), then the
      {\tt positions} component holds the co-ordinates of the
      NDF pixel centres. If the data are connections-dependent, then the
      {\tt positions} component holds the co-ordinates of the corners of
      the NDF pixels. All positions are stored as {\tt float} values.
      \item [{\tt connections}] -- This component identifies adjacent
      positions which can be connected together to form cubic cells. If the
      data are connections-dependent then these cells correspond to NDF pixels.
      Otherwise they form a grid in which each cell corner occurs at an NDF
      pixel centre.
      \item [{\tt variance}] -- This is copied from the VARIANCE
      component of the NDF. It is given the same ``dep'' attribute as the
      {\tt data} component. It is not created if the VARIANCE component
      is undefined, or if the PVAR parameter is set false. Bad values are
      replaced by zero and marked in the appropriate {\tt invalid ...}
      component.
      \item [{\tt invalid positions}] -- This is a list of indices into
      the {\tt positions} component, identifying positions which have bad
      DATA or VARIANCE values. It is of type {\tt integer}. It is not
      created if there are no bad values or if parameter PBAD is given
      the value FALSE or if the data are connections dependent (see
      parameter DEP).
      \item [{\tt invalid connections}] -- This is a list of indices into
      the {\tt connections} component, identifying cells which have bad
      DATA or VARIANCE values. It is of type {\tt integer}. It is not
      created if there are no bad values or if parameter PBAD is given
      the value FALSE or if the data are positions dependent (see
      parameter DEP).
      \end{description}

      In addition, the field is given the following attributes:
      \begin{description}
      \item [{\tt name}] -- This is a string object which is set equal to
      the TITLE component of the NDF (or ``$<$undefined$>$'') if there is no TITLE
      component.
      \item [{\tt axis labels}] -- This is a string list containing one
      string for each axis. Each string consists of the corresponding NDF AXIS
      LABEL component followed by the axis UNITS in parentheses. This
      attribute is used by the DX \Modnam{AutoAxes} module.
      \item [{\tt ndf\_label}] -- A string attribute holding the NDF LABEL
      component, or ``$<$undefined$>$'' if there is no LABEL component.
      \item [{\tt ndf\_units}] -- A string attribute holding the NDF UNITS
      component, or ``$<$undefined$>$'' if there is no UNITS component.
      \end{description}


   }
   \sstusage{
      \$SX\_DEV/ndf2dx in out bmode dep pvar pbad axes
   }
   \sstparameters{
      \sstsubsection{
         IN = NDF (Read)
      }{
         The input NDF.
      }
      \sstsubsection{
         OUT = FILENAME (Write)
      }{
         The output file, including ``{\tt .dx}'' suffix if required. Existing files
         with the given name will be over-written.
      }
      \sstsubsection{
         BMODE = LOGICAL (Read)
      }{
         If YES, then the output file will hold data arrays in binary
         form, otherwise they will be in text form. [YES]
      }
      \sstsubsection{
         DEP = LITERAL (Read)
      }{
         The data dependency. This can be either ``connections'' or
         ``positions'' (case insensitive, unambiguous abbreviations are allowed)
         Positions-dependent data values are considered to be instantaneous
         samples from a smoothly varying field, taken at the NDF pixel
         centres. Connections-dependent data values are considered to be
         values which describe the whole pixel volume (for instance, the
         data values may be connection dependent if they represent the
         integral of a field over the pixel volume). DX interpolates
         smoothly in a position-dependent data array to find data values
         at un-tabulated positions, but uses the value of the cell in
         which the position lies if the data are connection dependent.
         See the DX {\em User's Guide} Section 2.1 for more information.
         [POSITIONS]
       }
      \sstsubsection{
         PVAR = LOGICAL (Read)
      }{
         If TRUE then any VARIANCE component in the input NDF will be
         copied to a {\tt variance} component in the output DX field. [YES]
      }
      \sstsubsection{
         PBAD = LOGICAL (Read)
      }{
         If TRUE then an {\tt invalid connections} or {\tt invalid
         positions} component will be added to the output field identifying
         any bad DATA or VARIANCE values in the input NDF. [YES]
      }
      \sstsubsection{
         AXES = LOGICAL (Read)
      }{
         If TRUE then any axes components in the NDF are copied to
         the output file.  If FALSE then any axes components are not
         copied and the output file will have the default units of
         pixels. [YES]
      }

   }
   \sstexamples{
      \sstexamplesubsection{
         ndf2dx jet jet.dx
      }{
         This example converts the NDF in file {\tt jet.sdf} into a
         native DX structure in file {\tt jet.dx}.
      }
   }

   \sstimplementationstatus{
      \sstitemlist{
         \sstitem
         This routine cannot currently process data values which are of
         type COMPLEX.
      }
   }


}


\section{Obtaining a Copy of DX \label{OBTAIN} \xlabel{OBTAIN} }

DX is a commercial product produced and sold by IBM. There is no CHEST
agreement for it, though there is an educational discount. As a guide
the following prices applied at the time of writing (November 1995):

\begin{center}
\begin{tabular}{rcc}
                    &  Full List Price &  Educational Price \\ \hline
Node locked licence &  \pounds 4700    & \pounds 940   \\
Floating licence    &  \pounds 5895    & \pounds 1179  \\
\end{tabular}
\end{center}

These prices are exclusive of VAT. A node-locked licence allows any
number of concurrent invocations of DX on a single workstation (though
most Starlink machines probably could not cope with multiple
invocations). A floating licence allows only a single invocation at any
instant, but it may be on any machine at your site. The licences permit
you to run the version that you have bought in perpetuity. However to
obtain the next version you would need to buy an upgrade, which costs
about \pounds 200.

Normally sites will obtain a copy by including an item for DX in their
annual bid submitted to Starlink. If the bid is successful a copy of
DX will be bought. Alternatively, of course, a site could simply
purchase a copy using local funds. In either case the person at Starlink
to contact in the first instance is C.A.~Clayton (e-mail {\tt
cac@star.rl.ac.uk}) who will supply the necessary details.


\typeout{  }
\typeout{*****************************************************}
\typeout{  }
\typeout{Reminder: run this document through Latex twice}
\typeout{to resolve the references.}
\typeout{  }
\typeout{*****************************************************}
\typeout{  }

\end{document}
