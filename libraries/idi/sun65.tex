\documentstyle[11pt]{article}
\pagestyle{myheadings}

% -----------------------------------------------------------------------------
\newcommand{\stardoccategory}  {Starlink User Note}
\newcommand{\stardocinitials}  {SUN}
\newcommand{\stardocnumber}    {65.8}
\newcommand{\stardocsource}    {sun65.8}
\newcommand{\stardocauthors}   {Nicholas Eaton}
\newcommand{\stardocdate}      {20 July 1995}
\newcommand{\stardoctitle}     {IDI \\[\latex{1.0ex}]
                                Image Display Interface\\[\latex{2.5ex}]
                                Programmer's Guide}
% -----------------------------------------------------------------------------

\newcommand{\stardocname}{\stardocinitials /\stardocnumber}
\markright{\stardocname}
\setlength{\textwidth}{160mm}
\setlength{\textheight}{230mm}
\setlength{\topmargin}{-2mm}
\setlength{\oddsidemargin}{0mm}
\setlength{\evensidemargin}{0mm}
\setlength{\parindent}{0mm}
\setlength{\parskip}{\medskipamount}
\setlength{\unitlength}{1mm}

% -----------------------------------------------------------------------------
% Hypertext definitions.
% These are used by the LaTeX2HTML translator in conjuction with star2html.

% Comment.sty: version 2.0, 19 June 1992
% Selectively in/exclude pieces of text.
%
% Author
%    Victor Eijkhout                                      <eijkhout@cs.utk.edu>
%    Department of Computer Science
%    University Tennessee at Knoxville
%    104 Ayres Hall
%    Knoxville, TN 37996
%    USA

%  Do not remove the %begin{latexonly} and %end{latexonly} lines (used by
%  star2html to signify raw TeX that latex2html cannot process).
%begin{latexonly}
\makeatletter
\def\makeinnocent#1{\catcode`#1=12 }
\def\csarg#1#2{\expandafter#1\csname#2\endcsname}

\def\ThrowAwayComment#1{\begingroup
    \def\CurrentComment{#1}%
    \let\do\makeinnocent \dospecials
    \makeinnocent\^^L% and whatever other special cases
    \endlinechar`\^^M \catcode`\^^M=12 \xComment}
{\catcode`\^^M=12 \endlinechar=-1 %
 \gdef\xComment#1^^M{\def\test{#1}
      \csarg\ifx{PlainEnd\CurrentComment Test}\test
          \let\html@next\endgroup
      \else \csarg\ifx{LaLaEnd\CurrentComment Test}\test
            \edef\html@next{\endgroup\noexpand\end{\CurrentComment}}
      \else \let\html@next\xComment
      \fi \fi \html@next}
}
\makeatother

\def\includecomment
 #1{\expandafter\def\csname#1\endcsname{}%
    \expandafter\def\csname end#1\endcsname{}}
\def\excludecomment
 #1{\expandafter\def\csname#1\endcsname{\ThrowAwayComment{#1}}%
    {\escapechar=-1\relax
     \csarg\xdef{PlainEnd#1Test}{\string\\end#1}%
     \csarg\xdef{LaLaEnd#1Test}{\string\\end\string\{#1\string\}}%
    }}

%  Define environments that ignore their contents.
\excludecomment{comment}
\excludecomment{rawhtml}
\excludecomment{htmlonly}

%  Hypertext commands etc. This is a condensed version of the html.sty
%  file supplied with LaTeX2HTML by: Nikos Drakos <nikos@cbl.leeds.ac.uk> &
%  Jelle van Zeijl <jvzeijl@isou17.estec.esa.nl>. The LaTeX2HTML documentation
%  should be consulted about all commands (and the environments defined above)
%  except \xref and \xlabel which are Starlink specific.

\newcommand{\htmladdnormallinkfoot}[2]{#1\footnote{#2}}
\newcommand{\htmladdnormallink}[2]{#1}
\newcommand{\htmladdimg}[1]{}
\newenvironment{latexonly}{}{}
\newcommand{\hyperref}[4]{#2\ref{#4}#3}
\newcommand{\htmlref}[2]{#1}
\newcommand{\htmlimage}[1]{}
\newcommand{\htmladdtonavigation}[1]{}
\newcommand{\latexhtml}[2]{#1}
\newcommand{\html}[1]{}

% Starlink cross-references and labels.
\newcommand{\xref}[3]{#1}
\newcommand{\xlabel}[1]{}

%  LaTeX2HTML symbol.
\newcommand{\latextohtml}{{\bf LaTeX}{2}{\tt{HTML}}}

%  Define command to recentre underscore for Latex and leave as normal
%  for HTML (severe problems with \_ in tabbing environments and \_\_
%  generally otherwise).
\newcommand{\latex}[1]{#1}
\newcommand{\setunderscore}{\renewcommand{\_}{{\tt\symbol{95}}}}
\latex{\setunderscore}

% -----------------------------------------------------------------------------
%  Debugging.
%  =========
%  Un-comment the following to debug links in the HTML version using Latex.

% \newcommand{\hotlink}[2]{\fbox{\begin{tabular}[t]{@{}c@{}}#1\\\hline{\footnotesize #2}\end{tabular}}}
% \renewcommand{\htmladdnormallinkfoot}[2]{\hotlink{#1}{#2}}
% \renewcommand{\htmladdnormallink}[2]{\hotlink{#1}{#2}}
% \renewcommand{\hyperref}[4]{\hotlink{#1}{\S\ref{#4}}}
% \renewcommand{\htmlref}[2]{\hotlink{#1}{\S\ref{#2}}}
% \renewcommand{\xref}[3]{\hotlink{#1}{#2 -- #3}}
%end{latexonly}
% -----------------------------------------------------------------------------
% Add any document specific \newcommand or \newenvironment commands here

\newcommand{\routinehead}[1]{\vspace{\bigskipamount}{\large\bf#1}}
\newenvironment{routinelist}{\begin{list}{}{\setlength{\leftmargin}{2cm}
                             \setlength{\parsep}{\smallskipamount}}}{\end{list}}
\newcommand{\routine}[2]{\item\hspace{-1cm}#1#2\\}
\newcommand{\lroutine}[2]{\item\hspace{-1cm}#1#2\\}

\begin{htmlonly}
  \newcommand{\lroutine}[2]{\item \htmlref{#1}{#1}#2\\}
  \newcommand{\routinehead}[1]{\subsection{#1}}
  \newcommand{\routine}[2]{\item #1#2\\}
\end{htmlonly}

% -----------------------------------------------------------------------------
%  Title Page.
%  ===========
\begin{document}
\thispagestyle{empty}

%  Latex document header.
\begin{latexonly}
   CCLRC / {\sc Rutherford Appleton Laboratory} \hfill {\bf \stardocname}\\
   {\large Particle Physics \& Astronomy Research Council}\\
   {\large Starlink Project\\}
   {\large \stardoccategory\ \stardocnumber}
   \begin{flushright}
   \stardocauthors\\
   \stardocdate
   \end{flushright}
   \vspace{-4mm}
   \rule{\textwidth}{0.5mm}
   \vspace{5mm}
   \begin{center}
   {\Large\bf \stardoctitle}
   \end{center}
   \vspace{5mm}

%  Add heading for abstract if used.
%   \vspace{10mm}
%   \begin{center}
%      {\Large\bf Description}
%   \end{center}
\end{latexonly}

%  HTML documentation header.
\begin{htmlonly}
   \xlabel{}
   \begin{rawhtml} <H1> \end{rawhtml}
      \stardoctitle
   \begin{rawhtml} </H1> \end{rawhtml}

%  Add picture here if required.

   \begin{rawhtml} <P> <I> \end{rawhtml}
   \stardoccategory\ \stardocnumber \\
   \stardocauthors \\
   \stardocdate
   \begin{rawhtml} </I> </P> <H3> \end{rawhtml}
      \htmladdnormallink{CCLRC}{http://www.clrc.ac.uk} /
      \htmladdnormallink{Rutherford Appleton Laboratory}
                        {http://www.clrc.ac.uk/ral} \\
      Particle Physics \& Astronomy Research Council \\
   \begin{rawhtml} </H3> <H2> \end{rawhtml}
      \htmladdnormallink{Starlink Project}{http://www.starlink.ac.uk/}
   \begin{rawhtml} </H2> \end{rawhtml}
   \htmladdnormallink{\htmladdimg{source.gif} Retrieve hardcopy}
      {http://www.starlink.ac.uk/cgi-bin/hcserver?\stardocsource}\\

%  Start new section for abstract if used.
%  \section{\xlabel{abstract}Abstract}

\end{htmlonly}

% -----------------------------------------------------------------------------
%  Document Abstract. (if used)
%  ==================

The Image Display Interface (IDI) is an international standard for
displaying astronomical data on an image display.
The specification of the interface is given in Terrett et~al,
1988\footnote{D.L.Terrett, P.M.B.Shames, R.J.Hanisch, R.Albrecht,
K.Banse, F.Pasian, M.Pucillo and P.Santin., 1988,
An image display interface for astronomical image processing,
Astron.Astrophys.Suppl.Ser., {\bf 76}, 263-304.}
This guide is an introduction to
programming with IDI, and describes the details of the current implementation.
The user should refer to the specification document, for details of the
subroutine calls.

% -----------------------------------------------------------------------------
%  Table of Contents. (if used)
%  ==================
%  The commands used here disable the creation of a "second" table of
%  contents and create a button in the navigation panel to get to the
%  postion of the \label{stardoccontents} command. If this isn't the
%  behaviour you require remove the htmlonly environment (and its
%  contents) and the \begin{latexonly} and \end{latexonly}.
%
\begin{htmlonly}
   \htmladdtonavigation{\htmlref{\htmladdimg{contents_motif.gif}}
                                            {stardoccontents}}
  \label{stardoccontents}
  \begin{rawhtml}
     <HR>
     <H2>Contents</H2>
  \end{rawhtml}
\end{htmlonly}
\begin{latexonly}
   \setlength{\parskip}{0mm}
   \tableofcontents
   \setlength{\parskip}{\medskipamount}
   \markright{\stardocname}
\end{latexonly}
% -----------------------------------------------------------------------------

\newpage

\section{Why Use IDI?}

IDI is intended for those users (applications) that need to be able to
manipulate images on an image display to a greater extent than is
available with \xref{GKS}{sun83}{} and its offspring (e.g.
\xref{PGPLOT}{sun15}{}). GKS allows an image to
be displayed, its look-up table changed and a cursor to be moved over it.
The most obvious advantage of IDI over GKS is that it allows the image
to be scrolled and zoomed. Other differences that users may wish to exploit
is the ability to blink images and to read back a representation of
the whole display, which can then be used to obtain a hardcopy of it.

IDI allows these functions to be programmed in a device independent way.
Once IDI has been implemented for one device then the same program can
run on the different devices just by supplying the appropriate device name
to {\bf Open Display}, with the proviso that not all the routines have
been implemented on all devices.

This does not mean that IDI supercedes GKS. IDI was written to offer
features not available in GKS. It does not have the sophistication of
GKS for producing vector (line) plots or character annotation. IDI
does have routines to draw lines and plot text, but these are primitive
and offer the user little control over how the result will appear, in
terms of character sizes, style of line widths etc.

The major strength of IDI is the ability to perform many types of
interaction using the mouse. This can be programmed to move the cursor,
or the memories, rotate the look-up table, zoom up and down, blink the
memories, or perform even more complex tasks by passing control back
to the calling program.

It should be noted that it is not possible to mix calls from GKS and IDI
on the same display; the two packages use completely different models
of the display. An application could, however, use the packages one after
the other by utilising the \xref{Applications Graphics
Interface}{sun48}{} (SUN/48) to mediate between them.


\section{Summary of IDI calls}

The following list gives all the top-level routines in the interface that
have been implement by Starlink
An error status will be returned if a routine that has not been implemented
is called. The routine arguments are defined in the specification
document.

\routinehead{Control}
\begin{routinelist}
\routine{IIDOPN}{( devnam, dispid, status )}
              {Open Display}
\routine{IIDCLO}{( dispid, status )}
               {Close Display}
\routine{IIDRST}{( dispid, status )}
               {Reset Display}
\routine{IIDUPD}{( dispid, status )}
               {Update Display}
\routine{IIDERR}{( status, messag, meslen )}
               {Get Error}
\end{routinelist}
\routinehead{Configuration}
\begin{routinelist}
\routine{IIDQDV}{( dispid, nconf, xsize, ysize, depth, nvlut, nitt, ncurs,
                        status )}
               {Query Device Characteristics}
\routine{IIDQCI}{( dispid, capid, narr, outarr, nout, status )}
               {Query Capabilities Integer}
\routine{IIDQCR}{( dispid, capid, narr, outarr, nout, status )}
               {Query Capabilities Real}
\routine{IIDQDC}{( dispid, nconf, memtyp, nmemax, modcon, memid, memsix,
                        memsiy,
               memdep, ittdep, nmem, status )}
               {Query Defined Configuration}
\routine{IIDSEL}{( dispid, nconf, status )}
               {Select Configuration}
\end{routinelist}
\routinehead{Memories}
\begin{routinelist}
\routine{IIMSMV}{( dispid, memid, nmem, lvis, status )}
               {Set Memory Visibility}
\routine{IIZWSC}{( dispid, memid, nmem, xoff, yoff, status )}
               {Write Memory Scroll}
\routine{IIZWZM}{( dispid, memid, nmem, zoomf, status )}
               {Write Memory Zoom}
\routine{IIZRSZ}{( dispid, memid, xoff, yoff, zoomf, status )}
               {Read Memory Scroll and Zoom}
\routine{IIMSLT}{( dispid, memid, lutnum, ittnum, status )}
               {Select Memory Look-up Tables}
\routine{IIMWMY}{( dispid, memid, image, npix, depth, pack, xstart, ystart,
                        status )}
               {Write Memory}
\routine{IIMCMY}{( dispid, memid, nmem, back, status )}
               {Clear Memory}
\routine{IIMRMY}{( dispid, memid, npix, xstart, ystart, depth, pack, itton,
                        image, status )}
               {Read Memory}
\routine{IIMSTW}{( dispid, memid, direcn, xsize, ysize, depth, xoff, yoff,
                        status )}
               {Set Transfer Window}
\end{routinelist}
\routinehead{Graphics}
\begin{routinelist}
\routine{IIGPLY}{( dispid, memid, x, y, nxy, color, lstyle, status )}
               {Polyline}
\routine{IIGTXT}{( dispid, memid, text, xpos, ypos, tpath, tangle, color,
                        tsize, status )}
               {Plot Text}
\end{routinelist}
\routinehead{Look-up Table}
\begin{routinelist}
\routine{IILWIT}{( dispid, memid, ittnum, start, nent, itt, status )}
               {Write Intensity Transformation Table}
\routine{IILRIT}{( dispid, memid, ittnum, start, nent, itt, status )}
               {Read Intensity Transformation Table}
\routine{IILWLT}{( dispid, lutnum, start, nent, vlut, status )}
               {Write Video Look-up Table}
\routine{IILRLT}{( dispid, lutnum, start, nent, vlut, status )}
               {Read Video Look-up Table}
\end{routinelist}
\routinehead{Zoom and Pan}
\begin{routinelist}
\routine{IIZWZP}{( dispid, xoff, yoff, zoomf, status )}
               {Write Display Zoom and Pan}
\routine{IIZRZP}{( dispid, xoff, yoff, zoomf, status )}
               {Read Display Zoom and Pan}
\end{routinelist}
\routinehead{Cursor}
\begin{routinelist}
\routine{IICINC}{( dispid, memid, numcur, shape, color, xc, yc, status )}
               {Initialize Cursor}
\routine{IICSCV}{( dispid, numcur, lvis, status )}
               {Set Cursor Visibility}
\routine{IICRCP}{( dispid, inmid, numcur, xc, yc, outmid, status )}
               {Read Cursor Position}
\routine{IICWCP}{( dispid, memid, numcur, xc, yc, status )}
               {Write Cursor Position}
\end{routinelist}
\routinehead{Region of Interest}
\begin{routinelist}
\routine{IIRINR}{( dispid, memid, roicol, xmin, ymin, xmax, ymax, roiid,
                        status )}
               {Initialize Rectangular Region of Interest}
\routine{IIRSRV}{( dispid, roiid, lvis, status )}
               {Set Visibility Rectangular Region of Interest}
\routine{IIRRRI}{( dispid, inmid, roiid, xmin, ymin, xmax, ymax, outmid,
                        status )}
               {Read Rectangular Region of Interest}
\routine{IIRWRI}{( dispid, memid, roiid, xmin, ymin, xmax, ymax, status )}
               {Write Rectangular Region of Interest}
\end{routinelist}
\routinehead{Interaction}
\begin{routinelist}
\routine{IIIENI}{( dispid, intty, intid, objty, objid, intop, extrn,
                        status )}
               {Enable Interaction}
\routine{IIIEIW}{( dispid, trigs, status )}
               {Execute Interaction and Wait}
\routine{IIISTI}{( dispid, status )}
               {Stop Interactive Input}
\routine{IIIQID}{( dispid, intty, intid, messag, meslen, status )}
               {Query Interactor Description}
\routine{IIIGLD}{( dispid, locnum, dx. dy, status )}
               {Get Locator Displacement}
\end{routinelist}
\routinehead{Miscellaneous routines}
\begin{routinelist}
\routine{IIDSNP}{( dispid, cmode, npix, xstart, ystart, depth, pack, image,
                        status )}
               {Create Snapshot}
\routine{IILSBV}{( dispid, memid, lvis, status )}
               {Set Intensity Bar Visibility}
\end{routinelist}
\routinehead{Workstation interface}
\begin{routinelist}
\routine{IIDENC}{( dispid, status )}
               {Enable Configuration}
\routine{IIDAMY}{( dispid, xsize, ysize, depth, memtyp, memid, status )}
               {Allocate Memory}
\routine{IIDSTC}{( dispid, nconf, status )}
               {Stop Configuration}
\routine{IIDRLC}{( dispid, nconf, status )}
               {Release Configuration}
\end{routinelist}
\routinehead{ADAM interface routines}
\begin{routinelist}
\lroutine{IDI\_ANNUL}{( dispid, status )}
               {Annul display in the ADAM environment}
\lroutine{IDI\_ASSOC}{( pname, acmode, dispid, status )}
               {Associate display in the ADAM environment}
\lroutine{IDI\_CANCL}{( pname, status )}
               {Cancel display parameter in the ADAM environment}
\end{routinelist}

\section{Using the IDI Library}

\subsection{Control}

\subsubsection{Opening and Closing IDI}

All programs using IDI to perform graphics I/O on a device have to begin and
end the IDI sections with calls to {\bf Open Display} and {\bf Close Display}.
If the program is an ADAM task then these two calls have to be replaced by
appropriate calls to \htmlref{IDI\_ASSOC}{IDI_ASSOC} and
\htmlref{IDI\_CANCL}{IDI_CANCL} or \htmlref{IDI\_ANNUL}{IDI_ANNUL}
\latex{(see Appendix~\ref{se:apg} for more details on these functions)}.

More than one device can be open at a time, each needing separate calls
to {\bf Open Display} and {\bf Close Display}. The devices are selected
by the unique identifier returned from {\bf Open Display}, and these
identifiers should not be tampered with. Device selection is achieved by
passing the relevant device identifier to subsequent IDI routines.

By default IDI does not reset the display when the device is opened.
This enables applications using IDI to manipulate pictures drawn by
other applications. If an application requires the device to be in its
default state then an explicit call to {\bf Reset Display} is required.

As an example consider an application that is drawing a picture from
afresh, and therefore wants the device reset at the beginning. If the
application is an ADAM task IDI is opened using
\begin{small}
\begin{verbatim}
      CALL IDI_ASSOC( 'DEVICE', 'WRITE', ID, STATUS )
\end{verbatim}
\end{small}
where 'DEVICE' is the name of a parameter in the interface file
through which the name of the device is obtained. The access mode
is set to 'WRITE' which forces a reset of the device, the other
possibilities 'READ' and 'UPDATE' leave the device as it was found.
The third argument is the display identifier which is returned from
the routine, and is used in subsequent IDI calls to direct operations
to this device. The status argument is an integer which indicates if
any problems have occurred.

In a non-ADAM application the same effect is achieved with
\begin{small}
\begin{verbatim}
      CALL IIDOPN( DNAME, ID, STATUS )
      CALL IIDRST( ID, STATUS )
\end{verbatim}
\end{small}
where DNAME is a string containing the name of the device, corresponding
to an entry in the file gns\_idinames. The display identifier returned
from {\bf Open Display} is passed to {\bf Reset Display} to indicate
that this is the device to reset.

\subsubsection{Buffering}

For efficiency IDI can save up its output and send it to the graphics
device in batches. There may be occasions, therefore, when the device does
not have the complete picture displayed. To force the program to flush the
output buffer a call to {\bf Update Display} is required:
\begin{small}
\begin{verbatim}
      CALL IIDUPD( ID, STATUS )
\end{verbatim}
\end{small}

Quite often IDI will automatically flush the output buffer, for example
when the buffer is full, when the program is reading from the display,
or when using the mouse. The output buffer is also flushed when IDI is closed
down so that the picture is guaranteed to be complete after a call to
{\bf Close Display}.

\subsubsection{Error messages}

IDI returns error status values when it cannot complete a requested
function satisfactorily. These status values can be converted into
error messages by calling {\bf Get Error}:
\begin{small}
\begin{verbatim}
      CALL IIDERR( STATUS, MESSAG, MESLEN )
\end{verbatim}
\end{small}
The first argument is the integer status value returned from an IDI
routine, and the string containing the error message is returned in
the second argument. If the status value is not defined then a blank
string is returned from this routine.

IDI does not use the usual Starlink concept of inherited status. The
status value passed to each routine is initialised to IDI\_\_OK (=~0)
at the beginning of each routine. This puts the onus on the applications
programmer to provide sufficient status checks to locate the source of
any problem. If an application is using an
inherited status scheme it is probably best to pass a local status
variable to the IDI routines, and if an error occurs copy this to the
inherited status of the calling routine, or deal with the error in some
other way.

\subsection{Memories}

IDI uses memories as its drawing board. The memories can be considered to
be rectangular areas of pixels held internally in the display device.
The manner in which these memories appear on the screen is also controlled
by IDI.
An image is displayed on a device by loading the pixel values into a memory.
The pixel values are an array of unsigned integers in the range 0 to
$2^{depth}-1$ where $depth$ is the memory depth (bits per pixel). Any
values that exceed the maximum are truncated by discarding the most
significant bits. The memory can be displayed
at different positions or with different zoom factors, and the colour
representation of the image can be altered by changing the display path
(see below).

Images are loaded into a memory using {\bf Write Memory} and they can be
read back from a memory using {\bf Read Memory}. Images do not have to
fill the whole memory and the selection of a section of memory to write
to is achieved using {\bf Set Transfer Window}. A transfer window is a
rectangular sub-section of memory and the process of writing or reading
an image employs the transfer window to define which part of the
memory is to be used. If required the image
loading can begin in the middle of a row. A memory is cleared by
filling the memory with a constant value (usually zero) using the
routine {\bf Clear Memory}.

Writing an image into memory with {\bf Write Memory} does not
guarantee that the image will appear on the display. Although this
may happen with some devices it should not be relied upon, and a
call to {\bf Set Memory Visibility} should be used to ensure
consistent behaviour.

Suppose an application wants to display an image, having NX and NY
pixels along each axis, in the bottom left corner of the memory.
The application has already checked that the memory is big
enough to accept the image, but the size and shape of the image may
not match the size and shape of the memory. A transfer window is
therefore set up to match the size of the image and located at the
bottom left of the memory.
\begin{small}
\begin{verbatim}
      DIRECN = 0
      XOFF = 0
      YOFF = 0
      CALL IIMSTW( ID, MEMID, DIRECN, NX, NY, DEPTH, XOFF, YOFF, STATUS )
\end{verbatim}
\end{small}
The memory identifier (MEMID) is used to select one memory from those
available on the current configuration. A list of available memory
identifiers can be obtained from {\bf Query Defined Configuration}
as shown below. The load direction (DIRECN=0) indicates that the image
is to be loaded from bottom to top. In this case the size of the transfer
window is made to match the size of the image (NX, NY). The X and Y
offsets (XOFF, YOFF) indicate where the origin of the transfer window
is with respect to the origin of the memory. The depth
argument (DEPTH) specifies the expected number of significant bits
for each pixel in the input data stream (e.g. data in the range 0 to
255 is stored in eight bits). If this does not exactly match the
depth of the memory then data truncation or padding will occur.

The image is injected into the pixel memory using {\bf Write Memory}.
\begin{small}
\begin{verbatim}
      XSTART = 0
      YSTART = 0
      NPIX = NX * NY
      CALL IIMWMY( ID, MEMID, PIXDAT, NPIX, DEPTH, PACK,
     :             XSTART, YSTART, STATUS )
\end{verbatim}
\end{small}
The pixel values are stored in the integer input data array (PIXDAT)
as a contiguous stream, and the number of pixels to draw is given by
NPIX. The data depth (DEPTH) and packing factor (PACK) indicate how
individual pixel values are stored in the integer elements of the
input array. For instance if the integer word is 32 bits long and
the pixel values only have 8 significant bits, then it is possible
to store four pixel values in each integer (DEPTH=8, PACK=4).
Although this type of packing will save space it will probably not
improve plotting times, and so there is no advantage in programs
changing the packing of the data themselves. Of course if the data
is stored externally in packed form then this feature enables the data
to be displayed without having to unpack it first. The start position
(XSTART, YSTART) indicates where, with respect to the origin of the
transfer window, the first pixel in the data stream is to appear.
Subsequent pixels are placed to the right of this one until the edge
of the transfer window is reached when a new row of pixels is started.
If the size of the transfer window does not match the size of the
image then the line breaks will not appear in the correct place and
the image will appear skewed.

The final stage of displaying the image is to make the memory visible.
This is done with the routine {\bf Set Memory Visibility}.
\begin{small}
\begin{verbatim}
      NMEM = 1
      VIS = 1
      CALL IIMSMV( ID, MEMID, NMEM, VIS, STATUS )
\end{verbatim}
\end{small}
This routine takes an array of memory identifiers as its input, so
that more than one memory can be made visible at once, but this example
only uses the one memory. The visibility (VIS) is a logical value which
takes the value TRUE to make the memory visible, and FALSE to make
the memory invisible. The IDI specification defines these logical
values to be integers having the value zero (FALSE) and non-zero
(TRUE).

The display path essentially chooses which intensity transformation
table and which look-up table (discussed below) will be used to display
the image. The intensity transformation table and look-up table converts
the integer values stored in the memory into colours on the screen.
If more than one of these tables is available then the bindings
of the tables to the memories can be changed using
{\bf Select Memory Look-up Tables}.

The position and zoom factor of the memory on the screen can be altered
using the routines {\bf Write Memory Scroll} and {\bf Write Memory Zoom}.
These routines are non-interactive and the amount of scroll and zoom
is set by the input arguments.
A zoom factor of 0 means no zoom, a zoom factor of 1 means zoom once
(double the size) etc. Negative zooms may be supported on some
devices and a zoom factor of -1 means unzoom once (half the size) etc.
Scroll offsets are given in pixel units with positive values moving
the image to the right and up. Like {\bf Set Memory Visibility}
these two routines take a list of memories as their input so that
more than one memory can be operated on with one call, if required.
A program can inquire the current scroll and zoom settings with
{\bf Read Memory Scroll and Zoom}. This routine can be used
after an interactive operation to inquire what the final values of
any scroll and zoom are.

Two memories can be blinked (alternately displayed) using the routine
{\bf Blink Memories}. When this routine is called the blinking begins
immediately and continues until the right hand mouse button is pressed.
The speed of the blink is initially set by the input argument to the
routine, but can be subsequently changed using the left and centre
mouse buttons. See the section in the Ikon implementation notes for
more details.

\subsection{Configuration}

The device configuration defines the various capabilities of the device,
such as the number of memories, their sizes and their depths, the
number of look-up tables etc. A device may have more than one
configuration, for example a device with a fixed address space
may have one configuration having one memory which fills the available
space and another configuration that has several smaller memories
allocating the space between them. There are a number of inquiry routines
that allow the program to find out various parameters of a given
configuration. The inquiries can be used to find out about configurations
other than the current one to see if one may be more suitable for the
program's requirement. {\bf Query Defined Configuration} gives details
of the requested configuration.

If there is more than one configuration available on a device then the
routine {\bf Select Configuration} will implement the chosen one.

{\bf Query Device Characteristics} gives details of the basic set up
of the device, such as the size of the display in pixels, the number
of configurations available, the number of look-up tables and the number
of cursors. More detailed inquiries of the capabilities and current
settings for the device can be obtained with {\bf Query Capabilities
Integer} and {\bf Query Capabilities Real}.
\hyperref{This appendix}{Appendix~}{}{se:qcn} gives details of the
available capabilities. Note that
some of the capabilities are arrays of numbers and so the array arguments
should be dimensioned large enough to accommodate the returns, otherwise the
information returned will be incomplete. A global variable IDI\_\_MAXCP
has been supplied which can be used to dimension the array and is
as large as any capability on any implemented device. This is defined
in the include file IDI\_PAR. Examples of using
{\bf Query Capabilities Integer} can be found in the
\hyperref{test program}{in appendix~}{}{se:exp}.

As an example of searching for a suitable configuration consider an
application that requires a minimum size (MEMINX, MEMINY) for an
image memory. The first call is to {\bf Query Device Characteristics}
to establish, amongst other things, the number of configurations
available which is returned in the second argument.
\begin{small}
\begin{verbatim}
      CALL IIDQDV( ID, NUMCON, DXSIZE, DYSIZE, DDEPTH, NLUT, NITT,
     :             NCURS, STATUS )
\end{verbatim}
\end{small}
As before the first argument is the display identifier returned from
{\bf Open Display}. The other arguments return the display size and
depth, the number of look-up tables, the number of intensity
transformation tables and the number of cursors.

The application then examines each configuration in turn using
{\bf Query Defined Configuration} to establish the size of the memories,
returned in the seventh and eighth arguments.
\begin{small}
\begin{verbatim}
*   Want information on image memories, so memory type = 1
      MEMTYP = 1

*   Examine each configuration in turn, starting from configuration 0
      NCONF = 0
  10  CONTINUE
      IF ( NCONF .LT. NUMCON ) THEN
         CALL IIDQDC( ID, NCONF, MEMTYP, IMAXCP, MODCON, MEMIDS,
     :                MEMSIX, MEMSIY, MEMDEP, ITTDEP, NMEM, STATUS )

*   Examine each memory in turn
         M = 0
  20     CONTINUE
         IF ( M .LT. NMEM ) THEN
            M = M + 1

*   Check the memory size against the required one
            IF ( ( MEMSIX( M ) .GE. MEMINX ) .AND.
     :           ( MEMSIY( M ) .GE. MEMINY ) ) GOTO 30
            GOTO 20
         ENDIF

*   Increment the configuration loop counter
         NCONF = NCONF + 1
         GOTO 10
      ENDIF

*   If this point is reached then have not found a suitable memory
*   so abort the program
      GOTO 99

*   A suitable memory has been found with configuration number NCONF
  30  CONTINUE
\end{verbatim}
\end{small}

The memory type argument (MEMTYP) is used to indicate which type of memory
the routine {\bf Query Defined Configuration} will return information on.
The choices are image memories (1), text memories (2) or graphics
memories (4).

As with most IDI identifiers the configuration identifiers (NCONF)
start from zero and go up to one less than the number of configurations
(NUMCON-1).

The information is returned from {\bf Query Defined Configuration} in
integer arrays having one entry per memory in the configuration. The
size of the arrays can be dimensioned using the global variable
(IDI\_\_MAXCP) since memory lists form part of the capability inquiries.
The actual number of memories in the configuration is given by the output
argument (NMEM). Each memory is examined in turn to see if it fulfills
the size criterion.

The output list of memory identifiers (MEMIDS) contains the values to
be used to identify a particular memory for subsequent IDI routines,
in the way that the display identifier selects a particular device.
The other arguments returned from {\bf Query Defined Configuration}
are the memory depths, the intensity transformation table depths and
the configuration mode (MODCON) which indicates if the device is set
up for monochrome (0), pseudo-colour (1) or true-colour (2) display.

The final stage of the example is to remember which memory was chosen
and to load the appropriate configuration into the device.
\begin{small}
\begin{verbatim}
*   A suitable memory has been found with configuration number NCONF
  30  CONTINUE

*   Obtain the memory identifier of the chosen memory
      MEMID = MEMIDS( M )

*   Select the appropriate configuration
      CALL IIDSEL( ID, NCONF, STATUS )
\end{verbatim}
\end{small}

\subsection{Graphics}

IDI offers two primitive graphics functions to allow for annotation of an
image. The routine {\bf Polyline} will draw a line connecting a sequence
of given points. The routine {\bf Plot Text} will draw text, in one of
up to four sizes, at a given position and at a given angle. Depending on
the capabilities of a device either of these
routines may overwrite the pixel values of the given memories, and any
existing data at these locations will be lost.
The appearance of the text (font style and size in pixels) can differ
from device to device, because the IDI specification does not define
its own fonts and allows implementors to make use of any hardware
facilities. The call to {\bf Plot Text} is as follows:
\begin{small}
\begin{verbatim}
      CALL IIGTXT( ID, MEMID, STRING, XPOS, YPOS, TPATH, TANGLE,
     :             COLOR, TSIZE, STATUS )
\end{verbatim}
\end{small}
The text to be drawn is passed in the third argument (STRING). The
position of the text is given by the next two arguments (XPOS, YPOS)
which define the offset of the bottom left corner of the first
character with respect to the memory origin. The text path (TPATH)
defines if the text is to run horizontally, vertically or back to front,
and the orientation of the string (TANGLE) turns the string the given
number of degrees clockwise. Some implementations may only offer the
default values (0) for both these arguments with the result that the
text will only appear horizontally running from left to right.
Depending on the type of memory the colour argument (COLOR) is either
one of the set of predefined values or is an index into the current
look-up table. The text size (TSIZE) is defined imprecisely and takes
one of four values, normal (0), large (1), very large (2) or small (3).
The values other than the default (0) may not be implemented.

The routine {\bf Create Snapshot} will return a picture of the display
screen in an integer array. It differs from {\bf Read Memory} in that it
does not just return the pixel values from one memory, but takes account
of all memories visible on the screen, and at each pixel location returns
the value from the memory which is visible at that point. It works on the
principle of what you see is what you get, and thus the look-up tables
currently displayed are used to transform the pixel values into a value
representing the colour. For pseudo-colour devices the displayed colour
is transformed onto a single scale using the following fractions of each
primary colour $ 0.30 * Red + 0.59 * Green + 0.11 * Blue $.
The call to {\bf Create Snapshot} is as follows:
\begin{small}
\begin{verbatim}
      CALL IIDSNP( ID, CMODE, NPIX, XSTART, YSTART, DEPTH, PACK,
     :             IMDAT, STATUS )
\end{verbatim}
\end{small}
The colour mode (CMODE) is zero for a pseudo-colour or monochrome device.
The position arguments (XSTART, YSTART) define the start position
with respect to the origin of the current transfer window.
The given number of pixels (NPIX) are read out sequentially
from the current transfer window, with line breaks occurring at the
right hand edge of the window.
The data depth (DEPTH) and packing factor (PACK) indicate how
the pixel values are to be stored in the integer elements of the output
array (IMDAT).

\subsection{Look-up Table}

A pixel value is converted into a spot of colour on the screen in two
stages. The pixel value is used as an index into the intensity
transformation array. This table has one entry for each possible
pixel value, with the pixel values in the range 0 to $2^{depth}-1$
where $depth$ is the memory depth (bits per pixel). Each entry in
this table contains an index into another array, the look-up table,
which gives the proportion of red, green and blue used to make the
spot of colour. If the number of entries in the look-up table matches
the number of entries in the intensity transformation table, then the
default intensity transformation table contains a linear mapping, which
means that the pixel values point directly into the look-up table. If the
tables are of different length, then the default mapping will scale
linearly between the two.

The default look-up table is a linear grey scale, but this can be changed
by sending a new look-up table with {\bf Write Video Look-up Table}.
To preserve device independence the look-up table entries are stored
in a real array with values ranging from 0.0 to 1.0, where the latter
indicates the maximum intensity on the device. As an example consider
changing one of the screen colours to magenta. The look-up table array
is most conveniently defined as a 2-dimensional real array having 3
elements in its first dimension and the number of colour entries in the
second.
\begin{small}
\begin{verbatim}
*   Dimension the array of colours, in this example length = 1
      REAL VLUT( 3, 1 )

*   Define the number of entries and the red, green and blue intensities
      NENT = 1
      VLUT( 1, NENT ) = 1.0
      VLUT( 2, NENT ) = 0.0
      VLUT( 3, NENT ) = 1.0

*   Load the colour into the look-up table
      CALL IILWLT( ID, LUTNUM, START, NENT, VLUT, STATUS )
\end{verbatim}
\end{small}

The look-up table number (LUTNUM) defines which of the available
look-up tables to use. The number of available look-up tables can be
inquired using {\bf Query Device Characteristics} or {\bf Query
Capabilities Integer}. The look-up table
is updated starting at the entry defined by the third argument
(START), which in this example corresponds to the index of the
one colour to change, and continues until the given number of
colours (NENT) has been loaded.

Any memory that uses this look-up table will be updated when the new
look-up table is sent. The bindings of memories to look-up tables can
be changed with {\bf Select Memory Look-up Tables}.

The look-up tables can be read back using {\bf Read Video Look-up Table}.

\subsection{Zoom and Pan}

All the memories on the screen can be zoomed and panned together using
{\bf Write Display Zoom and Pan}. This saves the effort of having to set
the zoom and scrolls for all the memories individually using {\bf Write
Memory Scroll} and {\bf Write Memory Zoom}. These display settings can be
read back using {\bf Read Display Zoom and Pan}.
A zoom factor of 0 means no zoom, a zoom factor of 1 means zoom once
(double the size) etc. Negative zooms may be supported on some
devices and a zoom factor of -1 means unzoom once (half the size) etc.
Scroll offsets are given in pixel units with positive values moving
the image to the right and up.

\subsection{Interaction}

The interactions possible with IDI give the user the greatest scope for
making use of the image display in an interactive application. The
device which controls the interactions is known as the {\it interactor}
and corresponds to a mouse or trackerball, or some similar object.
An interactor such as a mouse or trackerball is considered to have two
components; the part which controls the two-dimensional movement is
known as the {\it locator}, and the buttons are known as {\it triggers}.

All interactions are controlled by two basic routines. {\bf Enable
Interaction} sets up one interaction per call. If more than one
interaction is required then this routine has to be called again with
the appropriate arguments. The routine {\bf Execute Interaction and
Wait} executes all the current interactions and continues until an exit
trigger is pressed. An exit trigger is one of the interactor buttons that
is defined in {\bf Enable Interaction} to stop that particular interaction.
The exit trigger is therefore a stop button which disables the interaction
and returns control to the program.
If more than one interaction is defined, for example scrolling and zooming
the memory, then it is usual to define the same trigger to be the exit
trigger for all interactions.

It is possible to set up more than one interaction to be simultaneously
using the same interactor component, for example the locator (mouse)
could be used to scroll the memory and rotate the look-up table at the
same time, although the effect would be pretty nauseous.

The interactions are defined by two components, the first defines the
action to be done, such as move or zoom, and the second defines the object
that this action is to be done to, such as cursor or memory.
The \htmlref{interaction cross reference table}{interaction-cross-ref}
\latex{(see Appendix~\ref{se:xin})}
shows which actions on which objects are possible for a given
device. Clearly some of the possibilities are meaningless, such as rotate
display or zoom look-up table, and any attempt to perform such interactions
will result in an error.

Examples of setting up interactions are shown in the following section on
regions of interest and in the
\hyperref{example program}{example program (appendix~}{}{se:exp}).

The routine {\bf Stop Interactive Input} will clear out all current
interactions previously set up with {Enable Interaction}. If more than
one interactive session is required in a single application, for
example getting a user to change the shape of a region of interest,
and then getting the user to move the fixed region of interest around
the screen, then {\bf Stop Interactive Input} should be called between
each.

Messages indicating what the user has to do to manage an interaction
can be constructed from the output of the routine
{\bf Query Interactor Description}. This will return a string that describes
how the given interaction is to be performed, such as `move mouse', or
`press centre button'. This can be output by the program to give
instructions to the user on how to achieve the particular interaction.
Examples of the use of this routine can be found in the
\hyperref{example program}{example program in appendix~}{}{se:exp}.

More complex interactions than those already offered can be programmed
using the {\it application specific code} mode. Normally the program control
would remain inside {\bf Execute Interaction and Wait} until an exit
trigger is fired. In the {\it application specific} mode the routine
{\bf Execute Interaction and Wait} returns control to the calling program,
where other tasks can be performed. The program then calls the routine
{\bf Execute Interaction and Wait} again, and this loop is repeated until
an exit trigger is fired. The routine {\bf Get Locator Displacement} can
be used inside the loop to inquire if the locator has moved,
and thus control the programmed interaction. A skeleton example of
such an interaction follows.
\begin{small}
\begin{verbatim}
*   Set up an application specific interaction to read the locator position
*   Set up the mouse ( interactor type = 0, interactor id = 0 ) to have
*   no visible effect ( object type = 0, object id = 0 ) on application
*   specific code ( interactive operation = 0 ). End the interaction by
*   pressing the right hand button ( exit trigger number = 2 ).
      INTTY = 0
      INTID = 0
      OBJTY = 0
      OBJID = 0
      INTOP = 0
      EXTRN = 2
      CALL IIIENI( ID, INTTY, INTID, OBJTY, OBJID, INTOP, EXTRN,
     :             STATUS )

*   Loop ( in a non-FORTRAN77 way ) until the exit trigger is fired
      DO WHILE ( TRIGS( EXTRN + 1 ) .EQ. 0 )

*   Enable the interaction
         CALL IIIEIW( ID, TRIGS, STATUS )

*   Inquire the locator displacement
         CALL IIIGLD( ID, INTID, DX, DY, STATUS )

*   Do some application specific functions
         ...

      ENDDO
\end{verbatim}
\end{small}

Note in the loop test the exit trigger number is incremented because
the trigger numbers start from zero whereas, by default, the FORTRAN
array TRIGS has one as its first index. Put another way the exit
trigger number of two corresponds to the third entry in the TRIGS array.
The displacements returned from {\bf Get Locator Displacement} give the
shift in the locator position (in pixels) since the last call to this
routine, and these can then be used in the application specific section.

The routine {\bf Execute Interaction and Wait} requires one of
its arguments to be a logical array whose length matches the number of
triggers on the device. The number of triggers can be obtained via the
routine {\bf Query Capabilities Integer}, but standard FORTRAN~77 does not
allow dynamic length arrays to be defined. Thus a general purpose
application needs to know what is the maximum number of triggers it
will have to cope with so it can define the array accordingly. A global
parameter IDI\_\_MAXTR has therefore been supplied which can be used to
dimension the array, and this is defined in the include file IDI\_PAR.
The following code segment shows how to define the trigger array in a
FORTRAN program
\begin{small}
\begin{verbatim}
*  Global Constants:
      INCLUDE 'IDI_PAR'
*  Local variables:
      INTEGER TRIGS( IDI__MAXTR )
\end{verbatim}
\end{small}

\subsection{Cursor}

The cursor is usually controlled by the interactor (mouse) but there are
several routines to give basic cursor control from a program. {\bf Set Cursor
Visibility} switches the cursor on or off and {\bf Write Cursor Position}
repositions the cursor on the display. The cursor position can be read
back using the routine {\bf Read Cursor Position}.

The routine {\bf Initialize Cursor} can be used to change the shape of the
cursor to one of the predefined shapes. If more than one cursor colour is
supported then this can also be changed using {\bf Initialize Cursor}.

\subsection{Region of Interest}

A region of interest is a rectangular area on the screen indicated by a
rubber-band box. In most respects it is similar in action to a cursor,
but instead of indicating a point it defines an area. The routine
{\bf Initialize Rectangular Region of Interest} is required to set up a
region of interest, and this returns an identifier which is used by the
other routines to access the region. {\bf Set Visibility Rectangular
Region of Interest} switches the box on or off and {\bf Write Rectangular
Region of Interest} repositions the box on the display. The position of
the region can be read back using the routine {\bf Read Rectangular
Region of Interest}

A region of interest can also be controlled using the interactor (mouse).
The {\it object identifier} field of {\bf Enable Interaction} should
contain the relevant ROI identifier returned by {\bf Initialize Rectangular
Region of Interest}. A region can either be moved using an {\it interactive
operation}~=~1, or its shape can be changed using an {\it interactive
operation}~=~7. When moving a region interactively, the size of the area is
fixed and the interaction moves the box over the display.

When modifying a region of interest interactively, a rubber-band box
is displayed which has one corner anchored and the other corner under
the control of the mouse. The active corner is indicated by a small
crosshair cursor. By moving the active corner the size and shape of the
box is changed. To allow complete control a trigger is set up which
switches the active corner to the opposite side. Any region can then be
defined by first moving the locator to position one corner as required,
switching the active corner, and then moving the locator again to position
the opposite corner. This type of operation requires two interactions to be
set up. The first defines that the mouse is to move the active
corner of the rubber-band box, the second defines a button which will
be used to switch the active corner from its current position to the
opposite corner of the rubber-band box.
The following code segment shows how this works.
\begin{small}
\begin{verbatim}
*   Set up the mouse ( interactor type = 0, interactor id = 0 ) to
*   control the ROI ( object type = 4, object id = ROIID ) by modifying
*   it ( interactive operation = 7 ). End the interaction by pressing
*   the right hand button ( exit trigger number = 2 ).
      INTTY = 0
      INTID = 0
      OBJTY = 4
      OBJID = ROIID
      INTOP = 7
      EXTRN = 2
      CALL IIIENI( ID, INTTY, INTID, OBJTY, OBJID, INTOP, EXTRN,
     :             STATUS )

*   Define the left-hand trigger ( interactor type = 5, interactor id = 0 )
*   to toggle the ROI active corner ( object type = 4, object id = ROIID )
*   while modifying it ( interactive operation = 7 ). End the interaction
*   by pressing the right hand button ( exit trigger number = 2 ).
      INTTY = 5
      INTID = 0
      OBJTY = 4
      OBJID = ROIID
      INTOP = 7
      EXTRN = 2
      CALL IIIENI( ID, INTTY, INTID, OBJTY, OBJID, INTOP, EXTRN,
     :             STATUS )

*   Enable the interaction
      CALL IIIEIW( ID, TRIGS, STATUS )

*   Stop the interaction
      CALL IIISTI( ID, STATUS )
\end{verbatim}
\end{small}

\subsection{Workstation Interface}

When using a workstation as an image display device it is possible to
dynamically allocate memory. This is useful if, for instance, the
available memory is not large enough to accommodate an image. Consider
the earlier example in which the existing configuration was searched
for a memory of a minimum size (MEMINX, MEMINY). In that case the
program aborted if a suitable memory could not be found. Using the
workstation interface a suitably sized memory can be requested. An
application should first inquire if the workstation interface is
supported using {\bf Query Capabilities Integer}
\begin{small}
\begin{verbatim}
*   Inquire if the workstation interface is supported
      NARR = 1
      CALL IIDQCI( ID, ISDYNC, NARR, YESNO, NVAL, STATUS )

*   Abort if it is not supported
      IF ( YESNO .EQ. 0 ) GOTO 99
\end{verbatim}
\end{small}
The global parameter ISDYNC defines the capability number used to inquire
if the dynamic configuration is implemented (see
\hyperref{this appendix}{appendix~}{}{se:qcn}).

The memory allocation is begun with a call to {\bf Enable Configuration}
which initialises the process. The memory requirements are passed to
{\bf Allocate Memory} and then the process is ended with a call to
{\bf Stop Configuration} which returns a {\it configuration number} to
be used to access the new memory. The new configuration has to be
selected with {\bf Select Configuration} before the new memory can
be used. The following code section shows this in action.
\begin{small}
\begin{verbatim}
*   Begin the configuration process
      CALL IIDENC( ID, STATUS )

*   Request an image memory (MEMTYP=1) of the required size
      MEMTYP = 1
      CALL IIDAMY( ID, MEMINX, MEMINY, DDEPTH, MEMTYP, IMEMID, STATUS )

*   Check that the request has been satisfied
      IF ( STATUS .NE. IDI__OK ) GOTO 99

*   End the configuration process
      CALL IIDSTC( ID, ICONF, STATUS )

*   Select this configuration
      CALL IIDSEL( ID, ICONF, STATUS )
\end{verbatim}
\end{small}

\newpage
\section{Linking with IDI}
If any of the idi include files are required, create a soft link to them
with the command
\begin{small}
\begin{verbatim}
      idi_dev
\end{verbatim}
\end{small}
and use an include statement of the form:
\begin{small}
\begin{verbatim}
      INCLUDE 'IDI_PAR'
\end{verbatim}
\end{small}

A standalone program can be linked by specifying `idi\_link` on the
compiler command line. Thus to compile and link a standalone application
called `prog' the following could be used
\begin{small}
\begin{verbatim}
      % f77 prog.f -o prog `idi_link`
\end{verbatim}
\end{small}
(note the use of the backward quotes).

For programs in the ADAM environment the compile and link command is
\begin{small}
\begin{verbatim}
      $ alink prog.f `idi_link_adam`
\end{verbatim}
\end{small}

\section{Device Names}

The choice of display device used by an IDI program is selected by
passing the device name to {\bf Open Display} or by defining the name
through the parameter system if
\htmlref{IDI\_ASSOC}{IDI_ASSOC} is used in an ADAM task.

IDI uses the \xref{Graphics Name Service}{sun57}{}
(SUN/57) to translate the given
device names. Acceptable device names can be found by running the example
program {\tt gnsrun\_idi} (normally found in {\tt /star/bin/examples}).


\section{Errors}

The IDI routines return a non-zero status value if the routine was
unable to perform its task properly; a successful operation will return
a zero value. Note that the standard IDI routines (those beginning with
the two character prefix II) do not use inherited status and reset the
value of the status argument to IDI\_\_OK (=~0) at the beginning of each
routine. Therefore the status value should be frequently checked if the
application is to trap any errors.

The routine IIDERR can be used to obtain an error message from an
IDI status value. The FORTRAN error definitions are declared in the
file IDI\_ERR.FOR. The C error definitions are declared in
the file idi\_err.htxt.

\section{Acknowledgements}

The X-windows IDI driver is largely based on software kindly supplied
by Paolo Santin of the Trieste Astronomical Observatory.

\newpage
\appendix
\section{Query capabilities names}
\label{se:qcn}

The table gives the types of capabilities that can be determined via calls
to {\bf Query Capabilities Integer} or {\bf Query Capabilities Real}.
The first column lists the capability. The second column gives the
integer code to be used in the call. Alternatively by including the
FORTRAN file 'IDI\_PAR' in a program the codes can be accessed using
the mnemonics given in the third column. The final column describes
the type and size of the return arguments. An entry of n
signifies a single integer value and an entry of n() implies an
array of integer values. Values r and r() signify single and multiple
real values, and l and l() imply single and multiple logical values.
The logical values are represented by integers with zero meaning 'false'
and any other value meaning 'true'.
The global constant IDI\_\_MAXCP can be used to dimension the arrays passed
to the query routines; this is defined in the file 'IDI\_PAR'.

\begin{quote}
\begin{tabular}{llll}
Capability & Integer Code & Mnemonic & Output array \\
\\
implementation level & 1 & IIMPLE & n \\
\\
number of available configurations & 10 & INCON & n \\
selected configuration number & 11 & ICONS & n \\
physical size of display (x,y) & 12 & ISIZED & n(2) \\
display depth (number of bits in DAC) & 13 & IDISDE & n \\
maximum depth of LUTs & 14 & ILUTDE & n \\
number of LUTs & 15 & INLUT & n \\
number of ITTs per image memory & 16 & INITT & n \\
zoom range (min,max) & 17 & IZOOMR & n(2) \\
number of available VLUT colours & 18 & INLUTC & n \\
\\
memory visible & 20 & IMEMVI & l() \\
list of memories in currently selected & 21 & IMEMS & n() \\
\hfill configuration \\
depth of memories (in bits) & 22 & IMEMDE & n() \\
list of current LUT bindings & 23 & ILUTBI & n() \\
list of current display path bindings & 24 & IDISBI & n() \\
list of current ITT bindings & 25 & IITTBI & n() \\
\\
maximum dimension of transfer window & 30 & ITWMDX & n() \\
\hfill in x \\
maximum dimension of transfer window & 31 & ITWMDY & n() \\
\hfill in y \\
transfer window x sizes & 32 & ITWSIX & n() \\
transfer window y sizes & 33 & ITWSIY & n() \\
transfer window x offsets & 34 & ITWOFX & n() \\
transfer window y offsets & 35 & ITWOFY & n() \\
depth of transfer windows (in bits) & 36 & ITWDE & n() \\
\end{tabular}
\end{quote}

\begin{quote}
\begin{tabular}{llll}
Capability & Integer Code & Mnemonic & Output array \\
\\
number of available device cursors & 40 & INCUR & n \\
array of cursor shapes & 41 & IACUSH & n() \\
number of cursor shapes & 42 & INCUSH & n \\
list of current cursor bindings & 43 & ICURBI & n() \\
list of current cursor shapes & 44 & ICURSH & n() \\
list of current cursor colours & 45 & ICURCO & n() \\
list of current cursor visibilities & 46 & ICURVI & l() \\
\\
number of locators & 50 & INLOC & n \\
number of real evaluators & 51 & INREVL & n \\
number of integer evaluators & 52 & INIEVL & n \\
number of logical evaluators (switches) & 53 & INLEVL & n \\
number of character evaluators & 54 & INCEVL & n \\
number of triggers & 55 & INTRIG & n \\
\\
ROI implemented & 60 & ISROI & l \\
number of device ROIs & 61 & INROI & n \\
list of current ROI bindings & 62 & IROIBI & n() \\
list of current ROI marker colours & 63 & IROICO & n() \\
list of current ROI visibilities & 64 & IROIVI & l() \\
\\
blink implemented & 70 & ISBLI & l \\
blink period for each memory & 71 & RBLIP & r() \\
trigger number to increase blink rate & 77 & IBLIIT & n \\
trigger number to decrease blink rate & 78 & IBLIDT & n \\
trigger number to stop blink & 79 & IBLIST & n \\
\\
split screen implemented & 80 & ISSS & l \\
split screen x memory offsets & 81 & ISSOFX & n() \\
split screen y memory offsets & 82 & ISSOFY & n() \\
split screen (x,y) address & 83 & ISSXY & n(2) \\
split screen enabled & 84 & ISSSON & l \\
\\
intensity bar implemented & 90 & ISBAR & l \\
intensity bar visible & 91 & IBARVI & l() \\
\\
snapshot implemented & 100 & ISSNAP & l \\
escape function implemented & 101 & ISESC & l \\
diagnostic routine implemented & 102 & ISDIAG & l \\
dynamic configuration implemented & 103 & ISDYNC & l \\
\end{tabular}
\end{quote}

\newpage
\section{ADAM Programmer's Guide to the IDI package}
\label{se:apg}

\subsection{Introduction}

This section describes the use of the Image Display Interface (IDI) in
ADAM application programs. It is expected that the reader is familiar with
programming for ADAM and with IDI.

The image display interface (IDI) is fully described in the specification
document, and the details of the Starlink implementation are given in
the preceding sections. This section will therefore only deal with issues
relating to the use of IDI in ADAM application programs.

All IDI routines may be used in ADAM applications with the exception of
the following routines, which must {\bf never} be called in ADAM programs

\hspace*{20mm}\begin{minipage}{140mm}

\noindent
IIDOPN - {\bf Open Display}

\noindent
IIDCLO - {\bf Close Display}

\end{minipage}

The function of IIDOPN is performed by the environment routine
\htmlref{IDI\_ASSOC}{IDI_ASSOC}.
The function of IIDCLO is performed by the environment routine
\htmlref{IDI\_CANCL}{IDI_CANCL},
or \htmlref{IDI\_ANNUL}{IDI_ANNUL}.

Note that these routines, unlike the core IDI routines, use the principle
of inherited status. The routine IDI\_ASSOC will only execute if the
status is equal to zero on entry. The routines IDI\_ANNUL and IDI\_CANCL
will execute if the status is non-zero on entry, so that devices can be
closed in the event of unrelated errors, and will return the error
status unchanged.

\subsection{The IDI Parameter Routines}

As with all other packages in the Software Environment, the only access to
objects outside of application programs is via Program Parameters. IDI has
three subroutines (the IDI parameter routines) that provide the
necessary interaction with the outside world. They are:

\begin{routinelist}
\lroutine{IDI\_ANNUL}{( dispid, status )}
{Associate an IDI device with a parameter}
\lroutine{IDI\_ASSOC}{( pname, acmode, dispid, status )}
{Release an IDI device}
\lroutine{IDI\_CANCL}{( pname, status )}
{Release an IDI device and cancel the parameter}
\end{routinelist}

Here is a skeletal example of a program using IDI
\begin{small}
\begin{verbatim}
      SUBROUTINE IDI_TEST ( STATUS )

      INCLUDE 'IDI_ERR'
      INTEGER ID, STATUS

*   Open a display device for IDI and reset it
      CALL IDI_ASSOC( 'DEVICE', 'WRITE', ID, STATUS )
      IF ( STATUS .EQ. IDI__OK ) THEN

*   Perform IDI functions on this display
         ...

*   Release device
         CALL IDI_ANNUL( ID, STATUS )
      ENDIF
\end{verbatim}
\end{small}

IDI\_ASSOC should be the first IDI routine to be called in the application.
It obtains the name of the graphics device to be used (via the parameter
system) and opens this device for subsequent IDI operations.

The first argument of IDI\_ASSOC is a Program Parameter (which should be
defined to be a device parameter in the Interface Module for the
application).

The second argument is the access mode required. This can be one of

\hspace*{20mm}\begin{minipage}{140mm}

\noindent
'READ' The application is only going to 'read' from the device, (i.e.
perform cursor or similar operations), without altering the display
appearance. The display is not cleared or reset when the device is
opened.

'WRITE' The application is going to 'write' on the device creating
a new screen display. The display is cleared and reset when it
is opened.

'UPDATE' The application will modify the device, taking the existing
display and changing it. The display is not cleared or reset when the
device is opened.

\end{minipage}

The third argument is the display identifier returned to the application.
This must be passed to all subsequent IDI routines which are to be used
on this device.

The fourth argument is the usual status value.

When the application has finished using the device, the display is closed
and the identifier annulled using the IDI\_ANNUL, or IDI\_CANCL routines.

\subsection{Interface File}

A simple interface file for the above example would be:

\begin{small}
\begin{verbatim}
interface IDI_TEST
   parameter DEVICE
      position 1
      ptype  'DEVICE'
      keyword 'DEVICE'
      access 'READ'
      vpath 'PROMPT'
      default IKON
      prompt 'Display device '
   endparameter
endinterface
\end{verbatim}
\end{small}

\subsection{Reference Section}

The description of each of the application level subroutines in the package
now follows

\vspace{10mm}
\parbox{160mm}{

\rule{160mm}{0.5mm}

\hspace*{10mm}\parbox{140mm}{
{\bf IDI\_ANNUL\label{IDI_ANNUL}\xlabel{IDI_ANNUL}} \\
Annul IDI display identifier and close the device.}

\rule{160mm}{0.1mm}

\hspace*{10mm}\parbox{140mm}{
The device associated with the display identifier is closed and the
identifier is annulled. No further IDI routines can be called with this
display identifier.}

\rule{160mm}{0.1mm}

\hspace*{10mm}\parbox{140mm}{
CALL IDI\_ANNUL~(~DISPID, STATUS~)}

\hspace*{10mm}\parbox{140mm}{
DISPID = INTEGER \hspace{10mm} (~given~)}

\hspace*{30mm}\parbox{120mm}{
A variable containing the display identifier.}

\hspace*{10mm}\parbox{140mm}{
STATUS = INTEGER}

\hspace*{30mm}\parbox{120mm}{
Variable to contain the status. This routine is executed regardless of the
given value of status. If its input value is not IDI\_\_OK then it is left
unchanged by this routine, even if it fails to complete. It its input value
is IDI\_\_OK and this routine fails, then the value is changed to an
appropriate error number.}

\rule{160mm}{0.5mm}
%}

%\parbox{160mm}{

\rule{160mm}{0.5mm}

\hspace*{10mm}\parbox{140mm}{
{\bf IDI\_ASSOC\label{IDI_ASSOC}\xlabel{IDI_ASSOC}} \\
Open a graphics device and return a display identifier.}

\rule{160mm}{0.1mm}

\hspace*{10mm}\parbox{140mm}{
Associate a graphics device with the specified parameter and return a
display identifier to reference this device.}

\rule{160mm}{0.1mm}

\hspace*{10mm}\parbox{140mm}{
CALL IDI\_ASSOC~(~PNAME, ACMODE, DISPID, STATUS~)}

\hspace*{10mm}\parbox{140mm}{
PNAME = CHARACTER~*~(~*~) \hspace{10mm} (~given~)}

\hspace*{30mm}\parbox{120mm}{
String specifying the name of the device parameter.}

\hspace*{10mm}\parbox{140mm}{
ACMODE = CHARACTER~*~(~*~) \hspace{10mm} (~given~)}

\hspace*{30mm}\parbox{120mm}{
String specifying the access mode: 'READ', 'WRITE' or 'UPDATE' as
appropriate.}

\hspace*{10mm}\parbox{140mm}{
DISPID = INTEGER \hspace{10mm} (~returned~)}

\hspace*{30mm}\parbox{120mm}{
A variable containing the display identifier.}

\hspace*{10mm}\parbox{140mm}{
STATUS = INTEGER}

\hspace*{30mm}\parbox{120mm}{
Variable to contain the status. If this variable is not IDI\_\_OK on input,
then the routine will return without action. If this routine fails to
complete, this variable will be set to an appropriate error number.}

\rule{160mm}{0.5mm}
%}

%\parbox{160mm}{

\rule{160mm}{0.5mm}

\hspace*{10mm}\parbox{140mm}{
{\bf IDI\_CANCL\label{IDI_CANCL}\xlabel{IDI_CANCL}} \\
Close an IDI display and break the parameter association.}

\rule{160mm}{0.1mm}

\hspace*{10mm}\parbox{140mm}{
The device associated with the parameter is closed and the association
between the graphics device and the specified device parameter is broken.}

\rule{160mm}{0.1mm}

\hspace*{10mm}\parbox{140mm}{
CALL IDI\_CANCL~(~PNAME, STATUS~)}

\hspace*{10mm}\parbox{140mm}{
PNAME = CHARACTER~*~(~*~) \hspace{10mm} (~given~)}

\hspace*{30mm}\parbox{120mm}{
String specifying the name of the device parameter which has previously
been associated with a device using IDI\_ASSOC.}

\hspace*{10mm}\parbox{140mm}{
STATUS = INTEGER}

\hspace*{30mm}\parbox{120mm}{
Variable to contain the status. This routine is executed regardless of the
given value of status. If its input value is not IDI\_\_OK then it is left
unchanged by this routine, even if it fails to complete. If its input value
is IDI\_\_OK and this routine fails, then the value is changed to an
appropriate error number.}

\rule{160mm}{0.5mm}
}


\newpage
\section{Example Program}
\label{se:exp}

This example program will read in a data file and display it in the
middle of the screen. A cursor will be displayed which can be
moved about the screen with the mouse, and the image can be zoomed
and unzoomed using the left hand and centre mouse buttons. The
interaction is terminated by pressing the right hand mouse button, and
the cursor position relative to the memory origin is output.

The input data file should contain integers between 0 and 255.

\begin{small}
\begin{verbatim}
      SUBROUTINE IDI_TEST ( STATUS )
*+
*  Name:
*     IDI_TEST
*  Purpose:
*     TEST IDI
*  Language:
*     FORTRAN
*  Type of Module:
*     ADAM A-task
*  Arguments:
*     STATUS = INTEGER (Given and Returned)
*        The global status.
*  Authors:
*     NE: Nick Eaton  (Durham University)
*  History:
*     May 1989 (NE):
*        Original version
*     March 1991 (NE):
*        Update to use NDFs and work with X-windows interface
*-
*  Type definitions:
      IMPLICIT NONE              ! No implicit typing
*  Global Constants:
      INCLUDE 'IDI_ERR'          ! IDI error codes
      INCLUDE 'IDI_PAR'          ! IDI global constants
*  Status:
      INTEGER STATUS             ! Global status
*  Local variables:
      CHARACTER * 64 TEXT
      INTEGER TRIGS( IDI__MAXTR )
      INTEGER DEPTH, DIRECN, DSIZE(2), EXTRN, ID, INDF, INTID, INTOP,
     :        INTTY, IPIN, LBND(2), LTEXT, MEMID, NARR, NCHAR, NDATA,
     :        NDIM, NEL, NUMCUR, NVAL, NX, NY, OBJID, OBJTY, OUTMID,
     :        PACK, UBND(2), XC, XOFF, XSTART, YC, YOFF, YSTART
*.

*   Check inherited global status.
      IF ( STATUS .NE. IDI__OK ) GOTO 99

*   Get the picture to plot. Should be 2-dimensional.
      CALL NDF_BEGIN
      CALL NDF_ASSOC( 'IN', 'READ', INDF, STATUS )
      CALL NDF_MAP( INDF, 'DATA', '_INTEGER', 'READ', IPIN, NEL, STATUS)
      CALL NDF_BOUND( INDF, 2, LBND, UBND, NDIM, STATUS )
      IF ( ( STATUS .NE. IDI__OK ) .OR. ( NDIM .GT. 2 ) ) GOTO 99

*   Calculate image size
      NX = UBND( 1 ) - LBND( 1 ) + 1
      NY = UBND( 2 ) - LBND( 2 ) + 1
      NDATA = NX * NY

*   Open IDI for a device obtained through the parameter system
*   Force a reset by using 'WRITE' mode
      CALL IDI_ASSOC( 'DEVICE', 'WRITE', ID, STATUS )
      IF ( STATUS .NE. IDI__OK ) GOTO 99

*   Inquire the physical size of the display using Query Capabilities
*   Integer code 12 ( ISIZED ) returns the screen size in pixels
      NARR = 2
      CALL IIDQCI( ID, ISIZED, NARR, DSIZE, NVAL, STATUS )

*   Check the image size against the display size
      IF ( ( NX .GT. DSIZE( 1 ) ) .OR. ( NY .GT. DSIZE( 2 ) ) ) GOTO 99

*   Inquire the depth of the first (default) memory using Query Capabilities
*   Integer code 22 ( IMEMDE ) returns the memory depth
      NARR = 1
      CALL IIDQCI( ID, IMEMDE, NARR, DEPTH, NVAL, STATUS )

*   Set up the transfer window to be the same size as the input image
*   and to be centered in the middle of the screen; XOFF, YOFF define
*   the position of the lower left corner of the transfer window. Plot
*   the image in the base memory ( 0 ) and loading from bottom to top ( 0 )
      MEMID = 0
      DIRECN = 0
      XOFF = MAX( 0, ( DSIZE( 1 ) - NX ) / 2 )
      YOFF = MAX( 0, ( DSIZE( 2 ) - NY ) / 2 )
      CALL IIMSTW( ID, MEMID, DIRECN, NX, NY, DEPTH, XOFF, YOFF,
     :             STATUS )

*   Write the image to the screen starting at position 0, 0 in the
*   transfer window. The data is taken from the lowest byte of each
*   integer word in the data array ( defined by PACK = 1, DEPTH = 8 ).
      PACK = 1
      XSTART = 0
      YSTART = 0
      CALL IIMWMY( ID, MEMID, %VAL( IPIN ), NDATA, DEPTH, PACK,
     :             XSTART, YSTART, STATUS )

*   Abort if an error has occurred
      IF ( STATUS .NE. IDI__OK ) THEN

*   Obtain a meaningful IDI error message
         CALL IIDERR( STATUS, TEXT, NCHAR )

*   Output this message
         CALL ERR_REP( 'IDI_WMY', TEXT, STATUS )
         GOTO 99
      ENDIF

*   Display the memory by setting its visibility to .TRUE.
      CALL IIMSMV( ID, MEMID, 1, 1, STATUS )

*   Set up interactions to move the cursor and zoom the memory
*   Set up the mouse ( interactor type = 0, interactor id = 0 ) to
*   control the cursor ( object type = 1, object id = 0 ) by moving
*   it ( interactive operation = 1 ). End the interaction by pressing
*   the right hand button ( exit trigger number = 2 ).
      INTTY = 0
      INTID = 0
      OBJTY = 1
      OBJID = 0
      INTOP = 1
      EXTRN = 2
      CALL IIIENI( ID, INTTY, INTID, OBJTY, OBJID, INTOP, EXTRN,
     :             STATUS )

*   Inquire the interactive operation and instruct the user
      CALL IIIQID( ID, INTTY, INTID, TEXT, LTEXT, STATUS )
      TEXT( LTEXT + 1 : ) = ' to control cursor'
      CALL MSG_OUT( ' ', TEXT, STATUS )

*   Set up the left hand button ( interactor type = 5, interactor id = 0 )
*   to control the memory ( object type = 5, object id = 0 ) by increasing
*   the zoom ( interactive operation = 3 ). End the interaction by pressing
*   the right hand button ( exit trigger number = 2 ).
      INTTY = 5
      INTID = 0
      OBJTY = 5
      OBJID = MEMID
      INTOP = 3
      EXTRN = 2
      CALL IIIENI( ID, INTTY, INTID, OBJTY, OBJID, INTOP, EXTRN,
     :             STATUS )

*   Inquire the interactive operation and instruct the user
      CALL IIIQID( ID, INTTY, INTID, TEXT, LTEXT, STATUS )
      TEXT( LTEXT + 1 : ) = ' to increase zoom'
      CALL MSG_OUT( ' ', TEXT, STATUS )

*   Set up the centre button ( interactor type = 5, interactor id = 1 )
*   to control the memory ( object type = 5, object id = 0 ) by decreasing
*   the zoom ( interactive operation = 4 ). End the interaction by pressing
*   the right hand button ( exit trigger number = 2 ).
      INTTY = 5
      INTID = 1
      OBJTY = 5
      OBJID = MEMID
      INTOP = 4
      EXTRN = 2
      CALL IIIENI( ID, INTTY, INTID, OBJTY, OBJID, INTOP, EXTRN,
     :             STATUS )

*   Inquire the interactive operation and instruct the user
      CALL IIIQID( ID, INTTY, INTID, TEXT, LTEXT, STATUS )
      TEXT( LTEXT + 1 : ) = ' to decrease zoom'
      CALL MSG_OUT( ' ', TEXT, STATUS )

*   Inquire the exit trigger operation and instruct the user
      CALL IIIQID( ID, INTTY, EXTRN, TEXT, LTEXT, STATUS )
      TEXT( LTEXT + 1 : ) = ' to exit'
      CALL MSG_OUT( ' ', TEXT, STATUS )

*   Move the cursor to the middle of the screen. A memory id = -1 sets
*   the cursor position relative to the screen origin.
      XC = DSIZE( 1 ) / 2
      YC = DSIZE( 2 ) / 2
      NUMCUR = 0
      CALL IICINC( ID, -1, NUMCUR, 1, 2, XC, YC, STATUS )

*   Display the cursor by setting its visibility to .TRUE.
      CALL IICSCV( ID, NUMCUR, 1, STATUS )

*   Execute the interactions.
      CALL IIIEIW( ID, TRIGS, STATUS )

*   Read the cursor position relative to the memory origin.
      CALL IICRCP( ID, MEMID, NUMCUR, XC, YC, OUTMID, STATUS )

*   Output the cursor position.
      CALL MSG_SETI( 'XC', XC )
      CALL MSG_SETI( 'YC', YC )
      CALL MSG_OUT( ' ', 'Cursor position is ^XC , ^YC', STATUS )

*   Undisplay the cursor by setting its visibility to .FALSE.
      CALL IICSCV( ID, NUMCUR, 0, STATUS )

  99  CONTINUE

*   Close down IDI using the parameter system
      CALL IDI_CANCL( 'DEVICE', STATUS )

*   Close the data file
      CALL NDF_END( STATUS )

      END
\end{verbatim}
\end{small}

\newpage
\begin{small}
\begin{verbatim}
interface IDI_TEST
   parameter IN
      position 1
      type 'NDF'
      keyword 'IN'
      access 'READ'
      vpath 'PROMPT'
      prompt 'NDF file containing image '
   endparameter
   parameter DEVICE
      position 2
      ptype  'DEVICE'
      keyword 'DEVICE'
      access 'READ'
      vpath 'PROMPT'
      default XWINDOWS
      prompt 'Display device '
   endparameter
endinterface
\end{verbatim}
\end{small}

\newpage
\section{X-WINDOWS implementation notes}
\label{se:xin}

\subsection{GWM interface}

IDI uses the \xref{Graphics Window Manager}{sun57}{} to look after the creation
and management of the display window. When IDI opens (IIDOPN or IDI\_ASSOC)
it first looks for an existing GWM window with the given name, and if
it finds one it attaches to that, otherwise it creates a new window with
the default characteristics. The window pixmap, which is used by GWM to
store the picture, is then read by IDI and stored in an internal memory.
The purpose of this is to enable existing displays to be manipulated by
IDI, for example a picture drawn with GKS can be scrolled using IDI.
In an ADAM application 'READ' or 'UPDATE' mode should be used in
\htmlref{IDI\_ASSOC}{IDI_ASSOC} if an existing display is to be manipulated.

If an application wants to start with a clean sheet then the display
should be reset. In an ADAM application this is done by specifying
'WRITE' mode in IDI\_ASSOC. In a non-ADAM application the routine
{\bf Reset Display} should be called immediately after {\bf Open
Display}.

\subsection{Memories}

When an X-window device is opened one memory is made available
(identifier 0) which corresponds to the window pixmap. The size of the
memory is equal to the size of the pixmap, which can be different to
the size of the window.

\subsection{Overlays}

If a GWM window contains an overlay plane then this can be accessed as
if it were a separate memory (identifier 1). The overlay memory depth
is only one bit but it can be accessed like any other image memory.
It can be written to with {\bf Write Memory}, {\bf Polyline} or
{\bf Plot Text}. It can be scrolled and zoomed separately to the base
memory. It has a separate look-up table (identifier 1) to the base
plane. Although the memory is only one bit deep the overlay LUT has as
many entries as the base LUT, however every entry contains the same
overlay colour. If the overlay colour is to be changed then the new
colour should be sent to all the pens in the overlay LUT.

\subsection{Writing to the memories}

Most workstations running IDI will have an 8 bit display, but the
full 8 bit capability of the device is not normally available. The
X-windows driver will reserve a number of pens for its exclusive use
and GWM will not normally allocate all the available pens to one
window. The look-up table length is adjusted by IDI to match the
number of pens allocated to the GWM window. The default intensity
transformation table is adjusted, so that data values going from 0 to
255, map linearly onto the look-up table going from 0 to the allocated
length. For instance, if the look-up table contains 64 entries then a data
value of 0 will be assigned to pen 0 and a data value of 255 will be assigned
to pen 63, with intermediate values scaled appropriately.
The number of pens allocated to the window can be inquired using
{\bf Query Capabilities Integer} with capability number 18 (INLUTC),
or can be inquired from GWM using \xref{GWM\_GETCI}{sun130}{GWM_GETCI}.

The input to the image display routines {\bf Write Memory}, {\bf Read Memory}
is an array of integers, and the arguments {\it packing factor} and
{\it data depth} control how the data to be displayed is packed into a
single integer word.

\subsection{Zoom}

The X driver does not support hardware zooming and so the zooming has
been implemented in software. The current implementation allows zooming
by factors of up to 32 times. The zoom factors go from -31 to 31, with
0 as no zoom. Negative zooming (zoom factors less than 0) is not
currently supported for {\it interactive operations}.

\subsection{Look-up tables}

Each window has one look up table (LUT) and one intensity transformation
table (ITT) per memory. The intensity transformation table is used to
translate pixel numbers into entries in the look up table.

\subsection{Interactions\label{interactions-cross-ref}}

The X-windows interface supports two {\it interactor types}; {\it locators}
and {\it triggers}. Two types of {\it locator} are supported: the mouse
{\it interactor identifier}~=~0 and the keyboard arrows
{\it interactor identifier}~=~1. The action of the keyboard arrows can be
speeded up by simultaneously depressing the SHIFT key. When using the keyboard
as an interactor the window focus has to be shifted to the display window
by clicking in the window or on the title bar.

The number of available triggers can be inquired using capability code
55 (INTRIG) in {\bf Query Capabilities Integer}.
Triggers can be either the mouse buttons (identifiers 0, 1, 2), or keys on
the keyboard (identifiers 3 to 54). The table lists the current set
of triggers and the corresponding identifiers.

\begin{small}
\begin{center}
\begin{tabular}{|r|p{9em}||r|p{9em}||r|p{9em}|}
\hline
Id & Trigger & Id & Trigger & Id & Trigger \\
\hline
0 & Mouse button left   & 20 &  G key & 40 & 0 key \\
1 & Mouse button centre & 21 &  H key & 41 & 1 key \\
2 & Mouse button right  & 22 &  I key & 42 & 2 key \\
3 & Up arrow            & 23 &  J key & 43 & 3 key \\
4 & Up arrow shifted    & 24 &  K key & 44 & 4 key \\
5 & Left arrow          & 25 &  L key & 45 & 5 key \\
6 & Left arrow shifted  & 26 &  M key & 46 & 6 key \\
7 & Right arrow         & 27 &  N key & 47 & 7 key \\
8 & Right arrow shifted & 28 &  O key & 48 & 8 key \\
9 & Down arrow          & 29 &  P key & 49 & 9 key \\
10 & Down arrow shifted  & 30 & Q key & 50 & Period key \\
11 & Keypad PF4          & 31 & R key & 51 & Plus key \\
12 & Space bar           & 32 & S key & 52 & Minus key \\
13 & Return key          & 33 & T key & 53 & Shift key left \\
14 & A key               & 34 & U key & 54 & Shift key right \\
15 & B key               & 35 & V key & & \\
16 & C key               & 36 & W key & & \\
17 & D key               & 37 & X key & & \\
18 & E key               & 38 & Y key & & \\
19 & F key               & 39 & Z key & & \\
\hline
\end{tabular}
\end{center}
\end{small}

A user can be informed of the appropriate trigger by obtaining a description
from the routine {\bf Query Interactor Description}.

At present only a few of all the possible interactions have been implemented.
A non-zero error status will be returned from {\bf Execute Interaction and
Wait} if the interaction has not been implemented.
The table shows the implemented interactions with a cross (~+~).

\begin{small}
\begin{center}
\begin{tabular}{|p{11em}||p{3.0em}||p{3.5em}||p{2.5em}|
                |p{3.5em}||p{3.5em}||p{3.7em}||p{3.5em}|}
\hline
\multicolumn{8}{|c|}{X-windows Interaction Cross Reference} \\ \hline \hline
Description & No effect & Cursor & ITT & VLUT & ROI & Memory & Display \\
\hline
Application specific code & + & . & . & . & . & . & . \\
Move object               & + & + & . & . & + & + & + \\
Rotate object             & + & . & . & + & . & . & . \\
Increase zoom             & + & . & . & . & . & + & + \\
Reduce zoom               & + & . & . & . & . & + & + \\
Set zoom to normal        & + & . & . & . & . & + & + \\
Blink object              & + & . & . & . & . & . & . \\
Modify object             & + & . & . & . & + & . & . \\
\hline
\end{tabular}
\end{center}
\end{small}

Scrolling a memory or the display with the mouse begins as soon as the
hardware pointer enters the display window, and the memory moves as
if it were attached to the hardware pointer, however scrolling does
not happen if the mouse is moved too quickly.

With the X-windows interface it is possible to display a fixed cursor
and to scroll and zoom the memory under it, although the default hardware
cursor should not be used in this case. A fixed crosshair cursor or region
of interest is made visible at the required position, and an interaction
to scroll the memory is set up. The example program can easily be converted
to perform this operation. To do this replace the cursor scroll interaction
with a memory scroll interaction. Thus
\begin{small}
\begin{verbatim}
      INTTY = 0
      INTID = 0
      OBJTY = 1
      OBJID = 0
      INTOP = 1
      EXTRN = 2
      CALL IIIENI( ID, INTTY, INTID, OBJTY, OBJID, INTOP, EXTRN,
     :             STATUS )
\end{verbatim}
\end{small}
becomes
\begin{small}
\begin{verbatim}
      INTTY = 0
      INTID = 0
      OBJTY = 5
      OBJID = MEMID
      INTOP = 1
      EXTRN = 2
      CALL IIIENI( ID, INTTY, INTID, OBJTY, OBJID, INTOP, EXTRN,
     :             STATUS )
\end{verbatim}
\end{small}

When rotating the look-up table the action is controlled by the left
and right movement of the mouse, or with the left and right keyboard
arrows.

\subsection{Cursors}

The default cursor ({\it cursor shape}~=~0) is the hardware pointer. When
it is activated, and in the display window, it is shown by an open cross.

The cross hair cursor ({\it cursor shape}~=~1), cross cursor ({\it cursor
shape}~=~2) and regions of interest (ROI) are drawn onto the screen using
an exclusive OR (XOR).
When using the mouse to control the movement of one of these cursors the
hardware pointer has to be brought close to the centre of the crosshairs.
When close enough the crosshair cursor becomes attached to the hardware
pointer and will remain attached as long as the mouse is moved slowly.
If the mouse is jerked quickly the hardware pointer and crosshair
will part. When using a region of interest the hardware pointer has
to be brought close to the active corner.

\subsection{Text}

Text is drawn with one of the preset device fonts. The text sizes small,
normal and large are distinct, but the very large text size is the same
as the large size. Only the default text path and text orientation are
supported.

\subsection{Deficiencies}

\begin{itemize}
\item
Cursors or regions of interest drawn with the XOR function may sometimes
be hard to see if many of the allocated pens used to draw an image contain
similar colours. Installing a more varied look-up table should make the
cursors more visible.

\item
When scrolling a memory, text or polylines drawn by the current application
will not be refreshed if that part of the memory is scrolled out of the
window and then scrolled back, unless the window is completely refreshed by,
for example, zooming.
\end{itemize}

\end{document}

