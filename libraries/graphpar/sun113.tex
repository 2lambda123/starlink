\documentclass[twoside,11pt]{article}

% ? Specify used packages
% \usepackage{graphicx}        %  Use this one for final production.
% \usepackage[draft]{graphicx} %  Use this one for drafting.
% ? End of specify used packages

\pagestyle{myheadings}

% -----------------------------------------------------------------------------
% ? Document identification
% Fixed part
\newcommand{\stardoccategory}  {Starlink User Note}
\newcommand{\stardocinitials}  {SUN}
\newcommand{\stardocsource}    {sun\stardocnumber}
\newcommand{\stardoccopyright}
{Copyright \copyright\ 2000 Council for the Central Laboratory of the Research Councils}

% Variable part - replace [xxx] as appropriate.
\newcommand{\stardocnumber}    {113.5}
\newcommand{\stardocauthors}   {P M Allan\\A J Chipperfield\\D L Terrett}
\newcommand{\stardocdate}      {19 July 1995}
\newcommand{\stardoctitle}     {ADAM Graphics Programmer's Guide}
\newcommand{\stardocversion}   {1.1-6}
\newcommand{\stardocmanual}    {Programmers' Manual}
\newcommand{\stardocabstract}  {[Text of abstract]}
% ? End of document identification
% -----------------------------------------------------------------------------

% +
%  Name:
%     sun.tex
%
%  Purpose:
%     Template for Starlink User Note (SUN) documents.
%     Refer to SUN/199
%
%  Authors:
%     AJC: A.J.Chipperfield (Starlink, RAL)
%     BLY: M.J.Bly (Starlink, RAL)
%     PWD: Peter W. Draper (Starlink, Durham University)
%
%  History:
%     17-JAN-1996 (AJC):
%        Original with hypertext macros, based on MDL plain originals.
%     16-JUN-1997 (BLY):
%        Adapted for LaTeX2e.
%        Added picture commands.
%     13-AUG-1998 (PWD):
%        Converted for use with LaTeX2HTML version 98.2 and
%        Star2HTML version 1.3.
%      1-FEB-2000 (AJC):
%        Add Copyright statement in LaTeX
%     {Add further history here}
%
% -

\newcommand{\stardocname}{\stardocinitials /\stardocnumber}
\markboth{\stardocname}{\stardocname}
\setlength{\textwidth}{160mm}
\setlength{\textheight}{230mm}
\setlength{\topmargin}{-2mm}
\setlength{\oddsidemargin}{0mm}
\setlength{\evensidemargin}{0mm}
\setlength{\parindent}{0mm}
\setlength{\parskip}{\medskipamount}
\setlength{\unitlength}{1mm}

% -----------------------------------------------------------------------------
%  Hypertext definitions.
%  ======================
%  These are used by the LaTeX2HTML translator in conjunction with star2html.

%  Comment.sty: version 2.0, 19 June 1992
%  Selectively in/exclude pieces of text.
%
%  Author
%    Victor Eijkhout                                      <eijkhout@cs.utk.edu>
%    Department of Computer Science
%    University Tennessee at Knoxville
%    104 Ayres Hall
%    Knoxville, TN 37996
%    USA

%  Do not remove the %begin{latexonly} and %end{latexonly} lines (used by
%  LaTeX2HTML to signify text it shouldn't process).
%begin{latexonly}
\makeatletter
\def\makeinnocent#1{\catcode`#1=12 }
\def\csarg#1#2{\expandafter#1\csname#2\endcsname}

\def\ThrowAwayComment#1{\begingroup
    \def\CurrentComment{#1}%
    \let\do\makeinnocent \dospecials
    \makeinnocent\^^L% and whatever other special cases
    \endlinechar`\^^M \catcode`\^^M=12 \xComment}
{\catcode`\^^M=12 \endlinechar=-1 %
 \gdef\xComment#1^^M{\def\test{#1}
      \csarg\ifx{PlainEnd\CurrentComment Test}\test
          \let\html@next\endgroup
      \else \csarg\ifx{LaLaEnd\CurrentComment Test}\test
            \edef\html@next{\endgroup\noexpand\end{\CurrentComment}}
      \else \let\html@next\xComment
      \fi \fi \html@next}
}
\makeatother

\def\includecomment
 #1{\expandafter\def\csname#1\endcsname{}%
    \expandafter\def\csname end#1\endcsname{}}
\def\excludecomment
 #1{\expandafter\def\csname#1\endcsname{\ThrowAwayComment{#1}}%
    {\escapechar=-1\relax
     \csarg\xdef{PlainEnd#1Test}{\string\\end#1}%
     \csarg\xdef{LaLaEnd#1Test}{\string\\end\string\{#1\string\}}%
    }}

%  Define environments that ignore their contents.
\excludecomment{comment}
\excludecomment{rawhtml}
\excludecomment{htmlonly}

%  Hypertext commands etc. This is a condensed version of the html.sty
%  file supplied with LaTeX2HTML by: Nikos Drakos <nikos@cbl.leeds.ac.uk> &
%  Jelle van Zeijl <jvzeijl@isou17.estec.esa.nl>. The LaTeX2HTML documentation
%  should be consulted about all commands (and the environments defined above)
%  except \xref and \xlabel which are Starlink specific.

\newcommand{\htmladdnormallinkfoot}[2]{#1\footnote{#2}}
\newcommand{\htmladdnormallink}[2]{#1}
\newcommand{\htmladdimg}[1]{}
\newcommand{\hyperref}[4]{#2\ref{#4}#3}
\newcommand{\htmlref}[2]{#1}
\newcommand{\htmlimage}[1]{}
\newcommand{\htmladdtonavigation}[1]{}

\newenvironment{latexonly}{}{}
\newcommand{\latex}[1]{#1}
\newcommand{\html}[1]{}
\newcommand{\latexhtml}[2]{#1}
\newcommand{\HTMLcode}[2][]{}

%  Starlink cross-references and labels.
\newcommand{\xref}[3]{#1}
\newcommand{\xlabel}[1]{}

%  LaTeX2HTML symbol.
\newcommand{\latextohtml}{\LaTeX2\texttt{HTML}}

%  Define command to re-centre underscore for Latex and leave as normal
%  for HTML (severe problems with \_ in tabbing environments and \_\_
%  generally otherwise).
\renewcommand{\_}{\texttt{\symbol{95}}}

% -----------------------------------------------------------------------------
%  Debugging.
%  =========
%  Remove % on the following to debug links in the HTML version using Latex.

% \newcommand{\hotlink}[2]{\fbox{\begin{tabular}[t]{@{}c@{}}#1\\\hline{\footnotesize #2}\end{tabular}}}
% \renewcommand{\htmladdnormallinkfoot}[2]{\hotlink{#1}{#2}}
% \renewcommand{\htmladdnormallink}[2]{\hotlink{#1}{#2}}
% \renewcommand{\hyperref}[4]{\hotlink{#1}{\S\ref{#4}}}
% \renewcommand{\htmlref}[2]{\hotlink{#1}{\S\ref{#2}}}
% \renewcommand{\xref}[3]{\hotlink{#1}{#2 -- #3}}
%end{latexonly}
% -----------------------------------------------------------------------------
% ? Document specific \newcommand or \newenvironment commands.

\newcommand{\aster}{$*$}
\begin{htmlonly}
  \newcommand{\aster}{*}
\end{htmlonly}

%+
%  Name:
%     SST.TEX

%  Purpose:
%     Define LaTeX commands for laying out Starlink routine descriptions.

%  Language:
%     LaTeX

%  Type of Module:
%     LaTeX data file.

%  Description:
%     This file defines LaTeX commands which allow routine documentation
%     produced by the SST application PROLAT to be processed by LaTeX and
%     by LaTeX2html. The contents of this file should be included in the
%     source prior to any statements that make of the sst commnds.

%  Notes:
%     The style file html.sty provided with LaTeX2html needs to be used.
%     This must be before this file.

%  Authors:
%     RFWS: R.F. Warren-Smith (STARLINK)
%     PDRAPER: P.W. Draper (Starlink - Durham University)

%  History:
%     10-SEP-1990 (RFWS):
%        Original version.
%     10-SEP-1990 (RFWS):
%        Added the implementation status section.
%     12-SEP-1990 (RFWS):
%        Added support for the usage section and adjusted various spacings.
%     8-DEC-1994 (PDRAPER):
%        Added support for simplified formatting using LaTeX2html.
%     {enter_further_changes_here}

%  Bugs:
%     {note_any_bugs_here}

%-

%  Define length variables.
\newlength{\sstbannerlength}
\newlength{\sstcaptionlength}
\newlength{\sstexampleslength}
\newlength{\sstexampleswidth}

%  Define a \tt font of the required size.
\latex{\newfont{\ssttt}{cmtt10 scaled 1095}}
\html{\newcommand{\ssttt}{\tt}}

%  Define a command to produce a routine header, including its name,
%  a purpose description and the rest of the routine's documentation.
\newcommand{\sstroutine}[3]{
   \goodbreak
   \rule{\textwidth}{0.5mm}
   \vspace{-7ex}
   \newline
   \settowidth{\sstbannerlength}{{\Large {\bf #1}}}
   \setlength{\sstcaptionlength}{\textwidth}
   \setlength{\sstexampleslength}{\textwidth}
   \addtolength{\sstbannerlength}{0.5em}
   \addtolength{\sstcaptionlength}{-2.0\sstbannerlength}
   \addtolength{\sstcaptionlength}{-5.0pt}
   \settowidth{\sstexampleswidth}{{\bf Examples:}}
   \addtolength{\sstexampleslength}{-\sstexampleswidth}
   \parbox[t]{\sstbannerlength}{\flushleft{\Large {\bf #1}}}
   \parbox[t]{\sstcaptionlength}{\center{\Large #2}}
   \parbox[t]{\sstbannerlength}{\flushright{\Large {\bf #1}}}
   \begin{description}
      #3
   \end{description}
}

%  Format the description section.
\newcommand{\sstdescription}[1]{\item[Description:] #1}

%  Format the usage section.
\newcommand{\sstusage}[1]{\item[Usage:] \mbox{}
\\[1.3ex]{\raggedright \ssttt #1}}

%  Format the invocation section.
\newcommand{\sstinvocation}[1]{\item[Invocation:]\hspace{0.4em}{\tt #1}}

%  Format the arguments section.
\newcommand{\sstarguments}[1]{
   \item[Arguments:] \mbox{} \\
   \vspace{-3.5ex}
   \begin{description}
      #1
   \end{description}
}

%  Format the returned value section (for a function).
\newcommand{\sstreturnedvalue}[1]{
   \item[Returned Value:] \mbox{} \\
   \vspace{-3.5ex}
   \begin{description}
      #1
   \end{description}
}

%  Format the parameters section (for an application).
\newcommand{\sstparameters}[1]{
   \item[Parameters:] \mbox{} \\
   \vspace{-3.5ex}
   \begin{description}
      #1
   \end{description}
}

%  Format the examples section.
\newcommand{\sstexamples}[1]{
   \item[Examples:] \mbox{} \\
   \vspace{-3.5ex}
   \begin{description}
      #1
   \end{description}
}

%  Define the format of a subsection in a normal section.
\newcommand{\sstsubsection}[1]{ \item[{#1}] \mbox{} \\}

%  Define the format of a subsection in the examples section.
\newcommand{\sstexamplesubsection}[2]{\sloppy
\item[\parbox{\sstexampleslength}{\ssttt #1}] \mbox{} \vspace{1.0ex}
\\ #2 }

%  Format the notes section.
\newcommand{\sstnotes}[1]{\item[Notes:] \mbox{} \\[1.3ex] #1}

%  Provide a general-purpose format for additional (DIY) sections.
\newcommand{\sstdiytopic}[2]{\item[{\hspace{-0.35em}#1\hspace{-0.35em}:}]
\mbox{} \\[1.3ex] #2}

%  Format the implementation status section.
\newcommand{\sstimplementationstatus}[1]{
   \item[{Implementation Status:}] \mbox{} \\[1.3ex] #1}

%  Format the bugs section.
\newcommand{\sstbugs}[1]{\item[Bugs:] #1}

%  Format a list of items while in paragraph mode.
\newcommand{\sstitemlist}[1]{
  \mbox{} \\
  \vspace{-3.5ex}
  \begin{itemize}
     #1
  \end{itemize}
}

%  Define the format of an item.
\newcommand{\sstitem}{\item}

%% Now define html equivalents of those already set. These are used by
%  latex2html and are defined in the html.sty files.
\begin{htmlonly}

%  sstroutine.
   \newcommand{\sstroutine}[3]{
      \subsection{#1\xlabel{#1}-\label{#1}#2}
      \begin{description}
         #3
      \end{description}
   }

%  sstdescription
   \newcommand{\sstdescription}[1]{\item[Description:]
      \begin{description}
         #1
      \end{description}
      \\
   }

%  sstusage
   \newcommand{\sstusage}[1]{\item[Usage:]
      \begin{description}
         {\ssttt #1}
      \end{description}
      \\
   }

%  sstinvocation
   \newcommand{\sstinvocation}[1]{\item[Invocation:]
      \begin{description}
         {\ssttt #1}
      \end{description}
      \\
   }

%  sstarguments
   \newcommand{\sstarguments}[1]{
      \item[Arguments:] \\
      \begin{description}
         #1
      \end{description}
      \\
   }

%  sstreturnedvalue
   \newcommand{\sstreturnedvalue}[1]{
      \item[Returned Value:] \\
      \begin{description}
         #1
      \end{description}
      \\
   }

%  sstparameters
   \newcommand{\sstparameters}[1]{
      \item[Parameters:] \\
      \begin{description}
         #1
      \end{description}
      \\
   }

%  sstexamples
   \newcommand{\sstexamples}[1]{
      \item[Examples:] \\
      \begin{description}
         #1
      \end{description}
      \\
   }

%  sstsubsection
   \newcommand{\sstsubsection}[1]{\item[{#1}]}

%  sstexamplesubsection
   \newcommand{\sstexamplesubsection}[2]{\item[{\ssttt #1}] #2}

%  sstnotes
   \newcommand{\sstnotes}[1]{\item[Notes:] #1 }

%  sstdiytopic
   \newcommand{\sstdiytopic}[2]{\item[{#1}] #2 }

%  sstimplementationstatus
   \newcommand{\sstimplementationstatus}[1]{
      \item[Implementation Status:] #1
   }

%  sstitemlist
   \newcommand{\sstitemlist}[1]{
      \begin{itemize}
         #1
      \end{itemize}
      \\
   }
%  sstitem
   \newcommand{\sstitem}{\item}

\end{htmlonly}

%  End of "sst.tex" layout definitions.
%.

% ? End of document specific commands
% -----------------------------------------------------------------------------
%  Title Page.
%  ===========
\renewcommand{\thepage}{\roman{page}}
\begin{document}
\thispagestyle{empty}

%  Latex document header.
%  ======================
\begin{latexonly}
   CCLRC / \textsc{Rutherford Appleton Laboratory} \hfill \textbf{\stardocname}\\
   {\large Particle Physics \& Astronomy Research Council}\\
   {\large Starlink Project\\}
   {\large \stardoccategory\ \stardocnumber}
   \begin{flushright}
   \stardocauthors\\
   \stardocdate
   \end{flushright}
   \vspace{-4mm}
   \rule{\textwidth}{0.5mm}
   \vspace{5mm}
   \begin{center}
   {\Huge\textbf{\stardoctitle \\ [2.5ex]}}
   {\LARGE\textbf{\stardocversion \\ [4ex]}}
   {\Huge\textbf{\stardocmanual}}
   \end{center}
   \vspace{5mm}

% ? Add picture here if required for the LaTeX version.
%   e.g. \includegraphics[scale=0.3]{filename.ps}
% ? End of picture

% ? Heading for abstract if used.
%   \vspace{10mm}
%   \begin{center}
%      {\Large\textbf{Abstract}}
%   \end{center}
% ? End of heading for abstract.
\end{latexonly}

%  HTML documentation header.
%  ==========================
\begin{htmlonly}
   \xlabel{}
   \begin{rawhtml} <H1> \end{rawhtml}
      \stardoctitle\\
      \stardocversion\\
      \stardocmanual
   \begin{rawhtml} </H1> <HR> \end{rawhtml}

% ? Add picture here if required for the hypertext version.
%   e.g. \includegraphics[scale=0.7]{filename.ps}
% ? End of picture

   \begin{rawhtml} <P> <I> \end{rawhtml}
   \stardoccategory\ \stardocnumber \\
   \stardocauthors \\
   \stardocdate
   \begin{rawhtml} </I> </P> <H3> \end{rawhtml}
      \htmladdnormallink{CCLRC / Rutherford Appleton Laboratory}
                        {http://www.cclrc.ac.uk} \\
      \htmladdnormallink{Particle Physics \& Astronomy Research Council}
                        {http://www.pparc.ac.uk} \\
   \begin{rawhtml} </H3> <H2> \end{rawhtml}
      \htmladdnormallink{Starlink Project}{http://www.starlink.ac.uk/}
   \begin{rawhtml} </H2> \end{rawhtml}
   \htmladdnormallink{\htmladdimg{source.gif} Retrieve hardcopy}
      {http://www.starlink.ac.uk/cgi-bin/hcserver?\stardocsource}\\

%  HTML document table of contents.
%  ================================
%  Add table of contents header and a navigation button to return to this
%  point in the document (this should always go before the abstract \section).
  \label{stardoccontents}
  \begin{rawhtml}
    <HR>
    <H2>Contents</H2>
  \end{rawhtml}
  \htmladdtonavigation{\htmlref{\htmladdimg{contents_motif.gif}}
        {stardoccontents}}

% ? New section for abstract if used.
%  \section{\xlabel{abstract}Abstract}
% ? End of new section for abstract
\end{htmlonly}

% -----------------------------------------------------------------------------
% ? Document Abstract. (if used)
%  ==================
%\stardocabstract
% ? End of document abstract

% -----------------------------------------------------------------------------
% ? Latex Copyright Statement
%  =========================
\begin{latexonly}
\newpage
\vspace*{\fill}
\stardoccopyright
\end{latexonly}
% ? End of Latex copyright statement

% -----------------------------------------------------------------------------
% ? Latex document Table of Contents (if used).
%  ===========================================
  \newpage
  \begin{latexonly}
    \setlength{\parskip}{0mm}
    \tableofcontents
    \setlength{\parskip}{\medskipamount}
    \markboth{\stardocname}{\stardocname}
  \end{latexonly}
% ? End of Latex document table of contents
% -----------------------------------------------------------------------------

\cleardoublepage
\renewcommand{\thepage}{\arabic{page}}
\setcounter{page}{1}

\section{Introduction}

This guide outlines the options available to a programmer writing an ADAM
applications program who wishes to incorporate graphics into the program and
describes the routines that handle the necessary interaction with the ADAM
parameter system when a graphics device is opened for a GKS based graphics
package.

Where a suite of graphics programs are being written or where a program is to
interact with graphics produced by another package (e.g.\ KAPPA), the
\xref{Applications Graphics Interface}{sun48}{}
(AGI---SUN/48) should be used. AGI provides
facilities for storing and restoring information about pictures such as extent
and world coordinates. AGI is layered on top of the routines described in
this document.

\section{Application}

For most ADAM applications,
\xref{PGPLOT}{sun15}{} (SUN/15) is the graphics package of choice;
it provides an easy to use yet powerful interface for plotting 2D axes, contour
maps, grey-scale, colour images and many other styles of plot. The
\xref{NCAR package}{sun88}{}
(SUN/88) provides similar facilities and some additional styles of plot
(e.g.\ stream-line diagrams) and is in general more powerful (for example, NCAR
can draw an X,Y plot with different scales on the right and left hand axes) but
is somewhat harder to use. NCAR also contains a 3D drawing package. It
is possible to combine NCAR graphics with calls to SGS and GKS by using
the \xref{Starlink Extensions to NCAR}{sun90}{} (SNX---SUN/90) package.

The NAG graphics package can also be used but it is forbidden for
software that is to become part of the Starlink software collection because
its availability on all hardware platforms is not guaranteed
(particularly in its single precision form) and non-Starlink sites may
not have a licence to use it.

NCAR and the NAG graphics do not have their own mechanisms for opening graphics
devices and SGS or GKS (see below) must be used to open a device before
plotting can begin.

For line graphics which requires a style not catered for by PGPLOT or NCAR, the
\xref{Simple Graphics System}{sun85}{}
(SGS) should be used (SUN/85). This package is ``lower
level'' than PGPLOT or NCAR (there are no facilities for drawing axes for
example) but it does have excellent facilities for manipulating transformations
and ``organising'' the display surface. SGS programs are also permitted to make
direct calls to GKS (with a few documented restrictions) so the full power of
GKS is available if SGS's facilities are not adequate.

Programming purely with GKS is not advisable because getting even simple things
done (e.g.\ opening a workstation) can be very long-winded and more easily
achieved with SGS. Furthermore, achieving device independence is far from
straight-forward; all the necessary facilities are available but making proper
use of them requires an in-depth knowledge of GKS. Again, this is better left
to SGS or higher level packages.

Programs requiring intimate interaction with an image display device should use
the \xref{Image Display Interface}{sun65}{}
(IDI---SUN/65) which is not based on GKS and offers
access to facilities that are outside the GKS model of a graphics device.
Currently the only device supported by IDI is
X-windows. The routines for opening and closing an IDI device in an
ADAM program are described in SUN/65.


\subsection{PGPLOT}

All PGPLOT routines may be used in ADAM applications with the exception of the
following routines, which must {\em never} be called in ADAM programs :
\begin{itemize}
\item \xref{PGBEG}{sun15}{PGBEG}
\item \xref{PGEND}{sun15}{PGEND}
\item \xref{PGASK}{sun15}{PGASK}
\end{itemize}
The functions of {\tt PGBEG} (and the equivalent obsolete routine {\tt
PGBEGIN}) are performed by the environment routine
\htmlref{{\tt PGP\_ASSOC}}{PGP_ASSOC}. The
environment routines \htmlref{{\tt PGP\_ANNUL}}{PGP_ANNUL},
\htmlref{{\tt PGP\_CANCL}}{PGP_CANCL} and
\htmlref{{\tt PGP\_DEACT}}{PGP_DEACT} all call {\tt PGEND},
but should be used as
described \hyperref{here}{in section~}{}{pgplot} in order to be compatible
with any future
enhancements to PGPLOT which might separate the action of closing a
workstation and closing down PGPLOT itself.

{\tt PGASK} controls whether PGPLOT prompts the user for confirmation before
clearing
the screen of a graphics device; if used in an ADAM program PGPLOT will bypass
the parameter system and attempt to read directly from the users terminal. So
the ``prompt state'' must never be changed from its
default state ({\tt{'OFF'}} in
ADAM programs). If prompting is required it is up to the application to keep
track of when
\xref{{\tt PGPAGE}}{sun15}{PGPAGE}
will clear the screen and issue its own prompt.

\subsection{SGS}

All SGS routines may be used in ADAM applications with the exception of the
following routines, which must {\em never} be called in ADAM programs :
\begin{itemize}
\item \xref{SGS\_INIT}{sun85}{SGS_INIT}
\item \xref{SGS\_OPEN}{sun85}{SGS_OPEN}
\item \xref{SGS\_OPNWK}{sun85}{SGS_OPNWK}
\item \xref{SGS\_CLOSE}{sun85}{SGS_CLOSE}
\end{itemize}
The functions of {\tt SGS\_INIT}, {\tt SGS\_OPEN} and {\tt SGS\_OPNWK} are
performed by the environment routine \htmlref{{\tt SGS\_ASSOC}}{SGS_ASSOC}.
The environment routines \htmlref{{\tt SGS\_ANNUL}}{SGS_ANNUL}
and \htmlref{{\tt SGS\_CANCL}}{SGS_CANCL} dispose of
resources used in SGS programs.  {\tt SGS\_CLOSE} will be called by the
environment level SGS de-activation routine
\htmlref{{\tt SGS\_DEACT}}{SGS_DEACT}.


\subsection{GKS}

Subroutines in the `environment level' of the ADAM GKS system have names of
the form GKS\_x whereas true GKS routines have names of the form Gx.

All GKS routines may be used in ADAM applications with the exception of the
following routines, which must {\em never} be called in ADAM applications:

\begin{itemize}
\item GOPKS
\item GOPWK
\item GCLWK
\item GCLKS
\end{itemize}

The functions of {\tt GOPKS} and {\tt GOPWK} are performed by the environment
level routine \htmlref{{\tt GKS\_ASSOC}}{GKS_ASSOC}. The function of
{\tt GCLWK} is performed by the routines
\htmlref{{\tt GKS\_ANNUL}}{GKS_ANNUL} or
\htmlref{{\tt GKS\_CANCL}}{GKS_CANCL}. The function of {\tt GCLKS}
is performed by the routine \htmlref{{\tt GKS\_DEACT}}{GKS_DEACT}.

\section{Environment Level Routines}

As with all other packages in ADAM, the only access to objects outside of
application programs is via ADAM program parameters. The connection between
graphics devices and the application program is controlled by means of a few
environment level routines.

\subsection{SGS routines\xlabel{sgs}}

SGS has four environment level subroutines (the SGS `parameter' or SGSPAR
routines) which provide the necessary interaction with the outside world.
They are :

\begin{center}
\begin{tabular}{||l|l||} \hline
Subroutine & \multicolumn{1}{c||}{Function} \\ \hline
SGS\_ASSOC  & Associate a graphics workstation with a parameter and open it.\\
SGS\_ANNUL  & Close a graphics workstation without cancelling the parameter.\\
SGS\_CANCL  & Close a graphics workstation and cancel the parameter.\\
SGS\_DEACT  & De-activate ADAM SGS.\\ \hline
\end{tabular}
\end{center}

Here is a skeletal example of a program using SGS :
\begin{quote}
\begin{verbatim}
      SUBROUTINE SGSTEST( STATUS )
      ..
      INTEGER STATUS
      INTEGER ZONE

      ..
*    Obtain Zone on a graphics workstation
      CALL SGS_ASSOC( 'DEVICE', 'WRITE', ZONE, STATUS )
      IF( STATUS .EQ. SAI__OK ) THEN

         ..

*       Perform graphics operations on the zone

         ..

*       Release Zone
         CALL SGS_ANNUL( ZONE, STATUS )
      ENDIF

      ..

      CALL SGS_DEACT( STATUS )

\end{verbatim}
\end{quote}

\htmlref{{\tt SGS\_ASSOC}}{SGS_ASSOC}
should be the first SGS routine to be called in the application.
It obtains (via the parameter system) the name of the graphics workstation
to be used, creates an initial SGS zone on the workstation, and returns an
SGS zone identifier which can be used in subsequent SGS subroutine calls.

The first argument of {\tt SGS\_ASSOC} is an ADAM program parameter.
It should be
defined to be a graphics device parameter in the interface module for the
application (see the
\hyperref{example interface file}{example interface file
in section}{}{ifl}). The value of
this parameter should be the name of a workstation as defined by GNS (see
\xref{SUN/57}{sun57}{GKSWorkstationNames}).

The second argument is the access mode required. This can be one of :
\begin{description}
\item[{\tt 'READ'}] The application is only going to `read' from the workstation
({\em i.e.}\ perform cursor or similar operations).
Screens will not be cleared when vdu workstations are opened.
\item[{\tt 'WRITE'}] The application is going to `write' on the workstation,
{\em i.e.}\ actually do some graphics.
Workstations are completely initialized when they are opened.
\item[{\tt 'UPDATE'}] The application will modify the graphics on the
workstation.
Screens will not be cleared when vdu workstations are opened.
\end{description}

Note that the facility to prevent screen clearing is specific to RAL GKS and
is not implemented for all workstations.

The third argument is the SGS zone identifier returned to the application.

The fourth argument is the usual status value. It follows the ADAM error
strategy as described in \xref{SUN/104}{sun104}{}.

When the application has finished using the workstation, it should be closed
using \htmlref{{\tt SGS\_CANCL}}{SGS_CANCL} unless it is
required to keep the parameter active
(to update a global parameter for example), in which case
\htmlref{{\tt SGS\_ANNUL}}{SGS_ANNUL}
should be used.

When the application has finished using SGS, it should be de-activated by
calling \htmlref{{\tt SGS\_DEACT}}{SGS_DEACT}.

\subsection{PGPLOT routines\label{pgplot}\xlabel{pgplot}}

PGPLOT has four environment level subroutines (the PGPLOT `parameter' or PGPPAR
routines) which provide the necessary interaction with the outside world.
They are :

\begin{center}
\begin{tabular}{||l|l||} \hline
Subroutine & \multicolumn{1}{c||}{Function} \\ \hline
PGP\_ASSOC  & Associate a graphics workstation with a parameter and open it.\\
PGP\_ANNUL  & Close a graphics workstation without cancelling the parameter.\\
PGP\_CANCL  & Close a graphics workstation and cancel the parameter.\\
PGP\_DEACT  & De-activate ADAM PGPLOT.\\ \hline
\end{tabular}
\end{center}

Here is a skeletal example of a program using PGPLOT :
\begin{quote}
\begin{verbatim}
      SUBROUTINE PGPTEST( STATUS )
      ..
      INTEGER STATUS
      INTEGER UNIT

      ..
*    Obtain Zone on a graphics workstation
      CALL PGP_ASSOC( 'DEVICE', 'WRITE', 1, 1, UNIT, STATUS )
      IF( STATUS .EQ. SAI__OK ) THEN

         ..

*       Perform graphics operations on the device

         ..

*       Release Device
         CALL PGP_ANNUL( UNIT, STATUS )
      ENDIF

      ..

      CALL PGP_DEACT( STATUS )

\end{verbatim}
\end{quote}

Note the close similarity with the SGS skeleton program.
\htmlref{{\tt PGP\_ASSOC}}{PGP_ASSOC} should be
the first PGPLOT routine called in the application. It obtains (via the
parameter system) the name of the graphics workstation  to be used, opens the
workstation, and returns a PGPLOT unit number. PGPLOT only supports one
graphics device open at one time so this unit number is always returned as
one\footnote{\xref{{\tt PGBEG}}{sun15}{PGBEG} has a similar redundant argument so that if, one day,
PGPLOT is extended to support multiple devices, existing programs would not
have to be changed.}.

The first, second and last arguments of {\tt PGP\_ASSOC} are the same as the
corresponding arguments of {\tt SGS\_ASSOC}.

The third and fourth arguments are the number of sub-plots per page in X and Y
(c.f.\ \xref{{\tt PGBEG}}{sun15}{PGBEG}).

The fifth argument is the unit identifier returned to the application.

When the application has finished using the workstation, it should be closed
using \htmlref{{\tt PGP\_CANCL}}{PGP_CANCL}
unless it is required to keep the parameter active
(to update a global parameter for example), in which case
\htmlref{{\tt PGP\_ANNUL}}{PGP_ANNUL} should be used.

When the application has finished using PGPLOT, it should be de-activated by
calling \htmlref{{\tt PGP\_DEACT}}{PGP_DEACT}.

\subsection{GKS routines\xlabel{gks}}

GKS has five environment level subroutines (the GKS `parameter' or GKSPAR
routines). Four provide the necessary interaction with the outside world
and one gives access to the GKS internal status.
They are :

\begin{center}
\begin{tabular}{||l|l||} \hline
Subroutine & \multicolumn{1}{c||}{Function} \\ \hline
GKS\_ASSOC  & Associate a graphics workstation with a parameter and open it.\\
GKS\_ANNUL  & Close a graphics workstation without cancelling the parameter.\\
GKS\_CANCL  & Close a graphics workstation and cancel the parameter.\\
GKS\_DEACT  & De-activate ADAM GKS.\\
GKS\_GSTAT  & Inquire whether GKS has reported an error
(see Section \ref{errhnd}). \\ \hline
\end{tabular}
\end{center}

Here is a skeletal example of a program using GKS :
\begin{quote}
\begin{verbatim}
      SUBROUTINE GKSTEST( STATUS )
      ..
      INTEGER STATUS
      INTEGER WKID

      ..
*    Obtain Workstation
      CALL GKS_ASSOC( 'DEVICE', 'READ', WKID, STATUS )
      IF( STATUS .EQ. SAI__OK ) THEN

         ..

*    Perform graphics operations on the workstation

         ..

*       Release Workstation
         CALL GKS_ANNUL( WKID, STATUS )
      ENDIF

      ..

      CALL GKS_DEACT( STATUS )
\end{verbatim}
\end{quote}

Note the close similarity with the SGS skeleton program.
\htmlref{{\tt GKS\_ASSOC}}{GKS_ASSOC} should be
the first GKS routine to be called in the application. It obtains (via the
parameter system) the name of the graphics workstation to be used, opens it,
and returns a GKS workstation identifier which can be used in subsequent GKS
subroutine calls.

The notes in the previous section on the arguments of {\tt SGS\_ASSOC}
 apply equally
well to {\tt GKS\_ASSOC} with the exception that the third argument is a GKS
workstation identifier rather than an SGS zone identifier.

The workstation is closed using
\htmlref{{\tt GKS\_ANNUL}}{GKS_ANNUL} or
\htmlref{{\tt GKS\_CANCL}}{GKS_CANCL}
when the application has finished using it.

When the application has finished using GKS, it should be de-activated by
calling \htmlref{{\tt GKS\_DEACT}}{GKS_DEACT}.

\section{Interface Files}
\label{ifl}
A simple interface file for the above SGS example would be:
\begin{quote}
\begin{verbatim}
## SGSTEST - Test some SGS, PGPLOT or GKS graphics functions

interface SGSTEST

   parameter      DEVICE
      ptype       'DEVICE'
      position    1
      type        'GRAPHICS'
      access      'read'
      vpath       'prompt'
      help        'Name of workstation to be used'
      default     xwindows
   endparameter

endinterface
\end{verbatim}
\end{quote}

The same interface file can be used with the PGPTEST and GKSTEST programs if
the line saying `interface SGSTEST' is replaced by the line `interface
PGPTEST' or `interface GKSTEST.

\section{Error Handling}
\label{errhnd}
The PGPLOT, SGS and GKS parameter routines conform to the Starlink error
reporting and inherited status strategy described in \xref{SUN/104}{sun104}{}.
Error reports will generally be of the form:
\begin{quote}
\begin{verbatim}
ROUTINE_NAME: Text
\end{verbatim}
\end{quote}
but where there routine name is not useful, it will be omitted.

If it is required to test for particular status values, symbolic names should
be used. They can be defined by including in the application, the statements:

\begin{quote}
\begin{verbatim}
INCLUDE 'SAE_PAR'   ! To define SAI__OK etc.
INCLUDE 'SGS_ERR'   ! To define the SGS__ error values
\end{verbatim}
\end{quote}

for SGS program, or the statements:

\begin{quote}
\begin{verbatim}
INCLUDE 'SAE_PAR'   ! To define SAI__OK etc.
INCLUDE 'PGP_ERR'   ! To define the GKS__ error values
\end{verbatim}
\end{quote}

for PGPLOT programs\begin{quote}

\begin{verbatim}
INCLUDE 'SAE_PAR'   ! To define SAI__OK etc.
INCLUDE 'GKS_ERR'   ! To define the GKS__ error values
\end{verbatim}
\end{quote}

for GKS programs. It is of course possible to include PGPLOT, SGS and GKS
status values in a single program.

The symbolic names are listed in \hyperref{here}{Section~}{}{errs}.

A substitute GKS error handling routine is linked with ADAM applications so
that error messages are reported in conformance with the Starlink error
reporting strategy, however, GKS stand-alone routines do not operate the
inherited status strategy. Thus programs which use GKS must be very careful
about handling error conditions if they are to avoid generating streams
of unhelpful error messages.

Subroutine \htmlref{{\tt GKS\_GSTAT}}{GKS_GSTAT}
 has been provided by the environment to help with this problem.
If this routine is called after every non-environment GKS routine
(or sequence of routine calls) then the
net effect is as if the GKS routine(s) did have an inherited status argument.

Note that for stand alone GKS routines which do have a status argument, the
environment status variable should {\em not} be used. Use a local variable,
ignore its value and use {\tt GKS\_GSTAT} to get an environment status value
for the
routine. Liberal use of IF .. THEN .. ELSE .. ENDIF structures testing the
environment status value is recommended to prevent invalid output from GKS and
a large number of error messages.

Note also that the GKS enquiry routines do not report an error when they fail,
they just return a status value (this is to allow the GKS error handling
routine to make enquires about the state of GKS when handling an error without
fear of triggering another error report and hence being called recursively).
This means that an application that detects an error from an enquiry routine
must compose and report a suitable message as well as setting the ADAM status

Example :\nopagebreak
\begin{quote}
\begin{verbatim}

      SUBROUTINE GKSTEST(STATUS)

* Test GKS in ADAM.
* Draw a box and write some text in the box.

      INCLUDE 'SAE_PAR'
      INCLUDE 'GKS_PAR'
      INTEGER STATUS
      INTEGER WKID
      INTEGER N
      PARAMETER (N = 5)
      REAL X(N), Y(N)
      DATA X/0.0,0.0,1.0,1.0,0.0/,
     :     Y/0.0,1.0,1.0,0.0,0.0/

* Check STATUS on entry.
      IF ( STATUS .NE. SAI__OK ) RETURN

* Open GKS, open and activate workstation.
      CALL GKS_ASSOC( 'DEVICE', 'WRITE', WKID, STATUS )
      IF ( STATUS .EQ. SAI__OK ) THEN
* End of standard opening sequence.
*---------------------------------------------------------------------
* Draw a box and write in it.
*   Polyline.
         CALL GPL( N, X, Y )
*   Set text alignment.
         CALL GSTXAL( GACENT, GACENT )
*   Set character height.
         CALL GSCHH( 0.04 )
*   Text.
         CALL GTX( 0.5, 0.5, 'Successful test of GKS 7.2' )
*---------------------------------------------------------------------
* Check for error in GKS call sequence.
         CALL GKS_GSTAT( STATUS )

* Cancel parameter and annul the workstation id.
         CALL GKS_CANCL( 'DEVICE', STATUS )
      ENDIF

      CALL GKS_DEACT( STATUS )
      END
\end{verbatim}
\end{quote}


\section{Compiling and Linking}

Graphics programs are linked by including {\tt `pgp\_link\_adam`},
{\tt `sgs\_link\_adam`} or {\tt `gks\_link\_adam`} as appropriate on
the alink command line.

\section{Status Values}
\label{errs}
The following status values may be returned by the SGS parameter
routines.

\begin{tabular}{ll}
SGS\_\_ILLAC     & Illegal access mode \\
SGS\_\_TOOZD     & No more available zone descriptors \\
SGS\_\_UNKPA     & Parameter not found \\
\end{tabular}

The following status values may be returned by the PGPLOT parameter
routines.

\begin{tabular}{ll}
PGP\_\_TOOZD     & No more available unit descriptors \\
PGP\_\_ILLAC     & Illegal access mode \\
PGP\_\_UNKPA     & Parameter not found \\
PGP\_\_ISACT     & Parameter currently active \\
PGP\_\_ERROR     & An error has been reported from GKS itself \\
PGP\_\_DVERR     & Workstation name not defined by GNS
\end{tabular}

The following status values may be returned by the GKS parameter
routines.

\begin{tabular}{ll}
GKS\_\_TOOZD     & No more available graphics descriptors \\
GKS\_\_ILLAC     & Illegal access mode \\
GKS\_\_UNKPA     & Parameter not associated with workstation \\
GKS\_\_ISACT     & Parameter currently active \\
GKS\_\_ERROR     & An error has been reported from GKS itself \\
GKS\_\_DVERR     & Workstation name not defined by GNS
\end{tabular}
\newpage

\section{Routine descriptions}

\begin{small}
\sstroutine{
   GKS\_ANNUL
}{
   Close a graphics workstation without cancelling the parameter
}{
   \sstdescription{
      De-activate and close the graphics workstation whose Workstation
      Identifier was obtained using \htmlref{GKS\_ASSOC}{GKS_ASSOC}, and
      annul the
      Workstation Identifier. Do not cancel the associated parameter.
   }
   \sstinvocation{
      CALL GKS\_ANNUL( WKID, STATUS )
   }
   \sstarguments{
      \sstsubsection{
         WKID = INTEGER (Given)
      }{
         A variable containing the Workstation Identifier.
      }
      \sstsubsection{
         STATUS = INTEGER (Given and returned)
      }{
         The global status.
      }
   }
   \sstnotes{
      On entry, the STATUS variable holds the global status value.
      If the given value of STATUS is SAI\_\_OK and the routine fails to
      complete, STATUS will be set to an appropriate error number.
      If the given value of STATUS is not SAI\_\_OK, then the routine
      will still attempt to execute and will return with STATUS
      unchanged.
   }
}
\sstroutine{
   GKS\_ASSOC
}{
   Associate a graphics workstation with a parameter, and open it
}{
   \sstdescription{
      Associate a graphics workstation with the specified Graphics
      Device Parameter and return a GKS Workstation Identifier to
      reference it.
      In GKS terms, the following actions occur --
         If GKS is not already open, it is opened and, if
         the workstation associated with the device parameter is
         not already open, that too is opened and activated.
   }
   \sstinvocation{
      CALL GKS\_ASSOC ( PNAME, MODE, WKID, STATUS )
   }
   \sstarguments{
      \sstsubsection{
         PNAME = CHARACTER\aster(\aster) (Given)
      }{
         Expression specifying the name of a Graphics Device
         Parameter.
      }
      \sstsubsection{
         MODE = CHARACTER\aster(\aster) (Given)
      }{
         Expression specifying the access mode -  {\tt '}READ{\tt '}, {\tt '}WRITE{\tt '}
         or {\tt '}UPDATE{\tt '}, as appropriate.
      }
      \sstsubsection{
         WKID = INTEGER (Returned)
      }{
         A Variable to contain the Workstation Identifier.
      }
      \sstsubsection{
         STATUS = INTEGER (Given and returned)
      }{
         The global status.
      }
   }
   \sstnotes{
      The facility whereby update mode does not to clear the
      workstation is specific to RAL GKS and is not implemented for all
      workstations.
   }
}
\sstroutine{
   GKS\_CANCL
}{
   Close graphics workstation and cancel the parameter
}{
   \sstdescription{
      De-activate and close the graphics workstation associated with
      the specified Graphics Device Parameter and cancel the parameter.
      The workstation must have been opened using
      \htmlref{GKS\_ASSOC}{GKS_ASSOC}.
   }
   \sstinvocation{
      CALL GKS\_CANCL ( PNAME, STATUS )
   }
   \sstarguments{
      \sstsubsection{
         PNAME = CHARACTER\aster(\aster) (Given)
      }{
         Expression specifying the name of a Graphics Device
         Parameter.
      }
      \sstsubsection{
         STATUS = INTEGER (Given and returned)
      }{
         The global status.
      }
   }
   \sstnotes{
      On entry, the STATUS variable holds the global status value.
      If the given value of STATUS is SAI\_\_OK and the routine fails to
      complete, STATUS will be set to an appropriate error number.
      If the given value of STATUS is not SAI\_\_OK, then the routine
      will still attempt to execute and will return with STATUS
      unchanged.
   }
}
\sstroutine{
   GKS\_DEACT
}{
   De-activate ADAM GKS
}{
   \sstdescription{
      De-activate ADAM GKS after use by an application.
   }
   \sstinvocation{
      CALL GKS\_DEACT ( STATUS )
   }
   \sstarguments{
      \sstsubsection{
         STATUS = INTEGER (Given and returned)
      }{
         The global status.
      }
   }
   \sstnotes{
      On entry, the STATUS variable holds the global status value.
      If the given value of STATUS is SAI\_\_OK and the routine fails to
      complete, STATUS will be set to an appropriate error number.
      If the given value of STATUS is not SAI\_\_OK, then the routine
      will still attempt to execute and will return with STATUS
      unchanged.
   }
}
\sstroutine{
   GKS\_GSTAT
}{
   Inquire whether GKS has reported an error
}{
   \sstdescription{
      Check whether a GKS error has been reported since the last time
      this routine was called. If it has, set STATUS to GKS\_\_ERROR, and
      cancel the GKS error state.
   }
   \sstinvocation{
      CALL GKS\_GSTAT ( STATUS )
   }
   \sstarguments{
      \sstsubsection{
         STATUS = INTEGER (Given and returned)
      }{
         The global status.
      }
   }
   \sstnotes{
      The global status will be set to GKS\_\_ERROR if GKS has reported
      an error; otherwise it will be unchanged.  If the given value of
      STATUS is not SAI\_\_OK, then the routine will return without
      action.
   }
}

\sstroutine{
   PGP\_ANNUL
}{
   Close a graphics workstation without cancelling the parameter
}{
   \sstdescription{
      Close PGPLOT, and annul the unit identifier.
      Do not cancel the associated parameter.
   }
   \sstinvocation{
      CALL PGP\_ANNUL ( UNIT, STATUS )
   }
   \sstarguments{
      \sstsubsection{
         UNIT = INTEGER (Given and returned)
      }{
         A variable containing the PGPLOT unit number
      }
      \sstsubsection{
         STATUS = INTEGER (Given and returned)
      }{
         The global status.
      }
   }
   \sstnotes{
      On entry, the STATUS variable holds the global status value.
      If the given value of STATUS is SAI\_\_OK and the routine fails to
      complete, STATUS will be set to an appropriate error number.
      If the given value of STATUS is not SAI\_\_OK, then the routine
      will still attempt to execute and will return with STATUS
      unchanged.
   }
}
\sstroutine{
   PGP\_ASSOC
}{
   Associate a graphics workstation with a parameter, and open it
}{
   \sstdescription{
      Associate a graphics workstation with the specified Graphics
      Device Parameter and return an PGP unit identifier
   }
   \sstinvocation{
      CALL PGP\_ASSOC ( PNAME, MODE, NX, NY, UNIT, STATUS )
   }
   \sstarguments{
      \sstsubsection{
         PNAME = CHARACTER\aster(\aster) (Given)
      }{
         Expression specifying the name of a graphics parameter.
      }
      \sstsubsection{
         MODE = CHARACTER\aster(\aster) (Given)
      }{
         Expression specifying the access mode.
         If it is {\tt '}READ{\tt '} or {\tt '}UPDATE{\tt '}, a vdu workstation screen
         will not be cleared when the workstation is opened.
         If it is {\tt '}WRITE{\tt '}, the screen will be cleared as usual.
      }
      \sstsubsection{
         NX = INTEGER (Given)
      }{
         Number of sub-plots in X.
      }
      \sstsubsection{
         NY = INTEGER (Given)
      }{
         Number of sub-plots in Y.
      }
      \sstsubsection{
         UNIT = INTEGER (Returned)
      }{
         A Variable to contain the unit identifier.
      }
      \sstsubsection{
         STATUS = INTEGER (Given and returned)
      }{
         The global status.
      }
   }
   \sstnotes{
      PGPLOT currently only supports one workstation so the unit is
      always returned as one.
   }
}
\sstroutine{
   PGP\_CANCL
}{
   Close a graphics workstation and cancel the parameter
}{
   \sstdescription{
      Close the graphics workstation associated with the specified graphics
      device parameter and cancel the parameter. The workstation must have
      been opened using \htmlref{PGP\_ASSOC}{PGP_ASSOC}.
   }
   \sstinvocation{
      CALL PGP\_CANCL ( PNAME, STATUS )
   }
   \sstarguments{
      \sstsubsection{
         PNAME = CHARACTER\aster(\aster) (Given)
      }{
         Expression specifying the name of a Graphics Device
         Parameter.
      }
      \sstsubsection{
         STATUS = INTEGER (Given and returned)
      }{
         The global status.
      }
   }
   \sstnotes{
      On entry, the STATUS variable holds the global status value.
      If the given value of STATUS is SAI\_\_OK and the routine fails to
      complete, STATUS will be set to an appropriate error number.
      If the given value of STATUS is not SAI\_\_OK, then the routine
      will still attempt to execute and will return with STATUS
      unchanged.
   }
}
\sstroutine{
   PGP\_DEACT
}{
   De-activate ADAM PGPLOT
}{
   \sstdescription{
      De-activate ADAM PGPLOT after use by an application.
   }
   \sstinvocation{
      CALL PGP\_DEACT ( STATUS )
   }
   \sstarguments{
      \sstsubsection{
         STATUS = INTEGER (Given and returned)
      }{
         The global status.
      }
   }
   \sstnotes{
      On entry, the STATUS variable holds the global status value.
      If the given value of STATUS is SAI\_\_OK and the routine fails to
      complete, STATUS will be set to an appropriate error number.
      If the given value of STATUS is not SAI\_\_OK, then the routine
      will still attempt to execute and will return with STATUS
      unchanged.
   }
}

\sstroutine{
   SGS\_ANNUL
}{
   Close a graphics workstation without cancelling the parameter
}{
   \sstdescription{
      De-activate and close the graphics workstation whose base
      zone (ZONE) was obtained using \htmlref{SGS\_ASSOC}{SGS_ASSOC},
      and annul the zone identifier.
      Do not cancel the associated parameter.
   }
   \sstinvocation{
      CALL SGS\_ANNUL ( ZONE, STATUS )
   }
   \sstarguments{
      \sstsubsection{
         ZONE = INTEGER (Given and returned)
      }{
         A variable containing the base zone identifier.
      }
      \sstsubsection{
         STATUS = INTEGER (Given and returned)
      }{
         The global status.
      }
   }
   \sstnotes{
      On entry, the STATUS variable holds the global status value.
      If the given value of STATUS is SAI\_\_OK and the routine fails to
      complete, STATUS will be set to an appropriate error number.
      If the given value of STATUS is not SAI\_\_OK, then the routine
      will still attempt to execute and will return with STATUS
      unchanged.
   }
}
\sstroutine{
   SGS\_ASSOC
}{
   Associate a graphics workstation with a parameter, and open it
}{
   \sstdescription{
      Associate a graphics workstation with the specified Graphics
      Device Parameter and return an SGS zone identifier to reference
      the base zone of the workstation.
   }
   \sstinvocation{
      CALL SGS\_ASSOC ( PNAME, MODE, ZONE, STATUS )
   }
   \sstarguments{
      \sstsubsection{
         PNAME = CHARACTER\aster(\aster) (Given)
      }{
         Expression specifying the name of a graphics parameter.
      }
      \sstsubsection{
         MODE = CHARACTER\aster(\aster) (Given)
      }{
         Expression specifying the access mode.
         If it is {\tt '}READ{\tt '} or {\tt '}UPDATE{\tt '}, a vdu workstation screen
         will not be cleared when the workstation is opened.
         If it is {\tt '}WRITE{\tt '}, the screen will be cleared as usual.
      }
      \sstsubsection{
         ZONE = INTEGER (Returned)
      }{
         A Variable to contain the Zone identifier.
      }
      \sstsubsection{
         STATUS = INTEGER (Given and returned)
      }{
         The global status.
      }
   }
   \sstnotes{
      The facility whereby update mode does not to clear the
      workstation is specific to RAL GKS and is not implemented for all
      workstations.
   }
}
\sstroutine{
   SGS\_CANCL
}{
   Close a graphics workstation and cancel the parameter
}{
   \sstdescription{
      De-activate and close the graphics workstation associated with
      the specified graphics device parameter and cancel the parameter.
      The workstation must have been opened using
      \htmlref{SGS\_ASSOC}{SGS_ASSOC}.
   }
   \sstinvocation{
      CALL SGS\_CANCL ( PNAME, STATUS )
   }
   \sstarguments{
      \sstsubsection{
         PNAME = CHARACTER\aster(\aster) (Given)
      }{
         Expression specifying the name of a Graphics Device
         Parameter.
      }
      \sstsubsection{
         STATUS = INTEGER (Given and returned)
      }{
         The global status.
      }
   }
   \sstnotes{
      On entry, the STATUS variable holds the global status value.
      If the given value of STATUS is SAI\_\_OK and the routine fails to
      complete, STATUS will be set to an appropriate error number.
      If the given value of STATUS is not SAI\_\_OK, then the routine
      will still attempt to execute and will return with STATUS
      unchanged.
   }
}
\sstroutine{
   SGS\_DEACT
}{
   De-activate ADAM SGS
}{
   \sstdescription{
      De-activate ADAM SGS after use by an application.
   }
   \sstinvocation{
      CALL SGS\_DEACT ( STATUS )
   }
   \sstarguments{
      \sstsubsection{
         STATUS = INTEGER (Given and returned)
      }{
         The global status.
      }
   }
   \sstnotes{
      On entry, the STATUS variable holds the global status value.
      If the given value of STATUS is SAI\_\_OK and the routine fails to
      complete, STATUS will be set to an appropriate error number.
      If the given value of STATUS is not SAI\_\_OK, then the routine
      will still attempt to execute and will return with STATUS
      unchanged.
   }
}
\end{small}
\end{document}
