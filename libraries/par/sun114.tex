\documentclass[twoside,11pt]{article}

% ? Specify used packages
% \usepackage{graphicx}        %  Use this one for final production.
% \usepackage[draft]{graphicx} %  Use this one for drafting.
% ? End of specify used packages

\pagestyle{myheadings}

% -----------------------------------------------------------------------------
% ? Document identification
% Fixed part
\newcommand{\stardoccategory}  {Starlink User Note}
\newcommand{\stardocinitials}  {SUN}
\newcommand{\stardocsource}    {sun\stardocnumber}

% Variable part - replace [xxx] as appropriate.
\newcommand{\stardocnumber}    {114.4}
\newcommand{\stardocauthors}   {Malcolm J. Currie\\
                                Alan J. Chipperfield}
\newcommand{\stardocdate}      {1999 September 24}
\newcommand{\stardoctitle}     {PAR \\ [\latex{1ex}]
                                Interface to the ADAM
                                Parameter System}
\newcommand{\stardocversion}   {Version 2.3}
\newcommand{\stardocmanual}    {Programmer's Manual}
\newcommand{\stardocabstract}  {PAR is a library of Fortran subroutines that
provides convenient mechanisms for applications to exchange information with
the outside world, through input-output channels called {\em parameters\/}.
Parameters enable a
user to control an application's behaviour.  PAR supports numeric,
character, and logical {\em parameters}, and is currently implemented
only on top of the {\footnotesize ADAM} parameter system.
\par
The PAR library permits parameter values to be obtained, without or with
a variety of constraints.  Results may be put into parameters to
be passed onto other applications.  Other facilities include setting a prompt
string, and suggested defaults.
\par
This document also introduces a preliminary C interface for the PAR library
\dash\ this may be subject to change in the light of experience.

}
% ? End of document identification
% -----------------------------------------------------------------------------

% +
%  Name:
%     sun.tex
%
%  Purpose:
%     Template for Starlink User Note (SUN) documents.
%     Refer to SUN/199
%
%  Authors:
%     AJC: A.J.Chipperfield (Starlink, RAL)
%     BLY: M.J.Bly (Starlink, RAL)
%
%  History:
%     17-JAN-1996 (AJC):
%        Original with hypertext macros, based on MDL plain originals.
%     16-JUN-1997 (BLY):
%        Adapted for LaTeX2e.
%        Added picture commands.
%     {Add further history here}
%
% -

\newcommand{\stardocname}{\stardocinitials /\stardocnumber}
\markboth{\stardocname}{\stardocname}
\setlength{\textwidth}{160mm}
\setlength{\textheight}{230mm}
\setlength{\topmargin}{-2mm}
\setlength{\oddsidemargin}{0mm}
\setlength{\evensidemargin}{0mm}
\setlength{\parindent}{0mm}
\setlength{\parskip}{\medskipamount}
\setlength{\unitlength}{1mm}

% -----------------------------------------------------------------------------
%  Hypertext definitions.
%  ======================
%  These are used by the LaTeX2HTML translator in conjunction with star2html.

%  Comment.sty: version 2.0, 19 June 1992
%  Selectively in/exclude pieces of text.
%
%  Author
%    Victor Eijkhout                                      <eijkhout@cs.utk.edu>
%    Department of Computer Science
%    University Tennessee at Knoxville
%    104 Ayres Hall
%    Knoxville, TN 37996
%    USA

%  Do not remove the %begin{latexonly} and %end{latexonly} lines (used by
%  star2html to signify raw TeX that latex2html cannot process).
%begin{latexonly}
\makeatletter
\def\makeinnocent#1{\catcode`#1=12 }
\def\csarg#1#2{\expandafter#1\csname#2\endcsname}

\def\ThrowAwayComment#1{\begingroup
    \def\CurrentComment{#1}%
    \let\do\makeinnocent \dospecials
    \makeinnocent\^^L% and whatever other special cases
    \endlinechar`\^^M \catcode`\^^M=12 \xComment}
{\catcode`\^^M=12 \endlinechar=-1 %
 \gdef\xComment#1^^M{\def\test{#1}
      \csarg\ifx{PlainEnd\CurrentComment Test}\test
          \let\html@next\endgroup
      \else \csarg\ifx{LaLaEnd\CurrentComment Test}\test
            \edef\html@next{\endgroup\noexpand\end{\CurrentComment}}
      \else \let\html@next\xComment
      \fi \fi \html@next}
}
\makeatother

\def\includecomment
 #1{\expandafter\def\csname#1\endcsname{}%
    \expandafter\def\csname end#1\endcsname{}}
\def\excludecomment
 #1{\expandafter\def\csname#1\endcsname{\ThrowAwayComment{#1}}%
    {\escapechar=-1\relax
     \csarg\xdef{PlainEnd#1Test}{\string\\end#1}%
     \csarg\xdef{LaLaEnd#1Test}{\string\\end\string\{#1\string\}}%
    }}

%  Define environments that ignore their contents.
\excludecomment{comment}
\excludecomment{rawhtml}
\excludecomment{htmlonly}

%  Hypertext commands etc. This is a condensed version of the html.sty
%  file supplied with LaTeX2HTML by: Nikos Drakos <nikos@cbl.leeds.ac.uk> &
%  Jelle van Zeijl <jvzeijl@isou17.estec.esa.nl>. The LaTeX2HTML documentation
%  should be consulted about all commands (and the environments defined above)
%  except \xref and \xlabel which are Starlink specific.

\newcommand{\htmladdnormallinkfoot}[2]{#1\footnote{#2}}
\newcommand{\htmladdnormallink}[2]{#1}
\newcommand{\htmladdimg}[1]{}
\newenvironment{latexonly}{}{}
\newcommand{\hyperref}[4]{#2\ref{#4}#3}
\newcommand{\htmlref}[2]{#1}
\newcommand{\htmlimage}[1]{}
\newcommand{\htmladdtonavigation}[1]{}
\newcommand{\latexhtml}[2]{#1}
\newcommand{\html}[1]{}

%  Starlink cross-references and labels.
\newcommand{\xref}[3]{#1}
\newcommand{\xlabel}[1]{}

%  LaTeX2HTML symbol.
\newcommand{\latextohtml}{{\bf LaTeX}{2}{\tt{HTML}}}

%  Define command to re-centre underscore for Latex and leave as normal
%  for HTML (severe problems with \_ in tabbing environments and \_\_
%  generally otherwise).
\newcommand{\latex}[1]{#1}
\newcommand{\setunderscore}{\renewcommand{\_}{{\tt\symbol{95}}}}
\latex{\setunderscore}

% -----------------------------------------------------------------------------
%  Debugging.
%  =========
%  Remove % on the following to debug links in the HTML version using Latex.

% \newcommand{\hotlink}[2]{\fbox{\begin{tabular}[t]{@{}c@{}}#1\\\hline{\footnotesize #2}\end{tabular}}}
% \renewcommand{\htmladdnormallinkfoot}[2]{\hotlink{#1}{#2}}
% \renewcommand{\htmladdnormallink}[2]{\hotlink{#1}{#2}}
% \renewcommand{\hyperref}[4]{\hotlink{#1}{\S\ref{#4}}}
% \renewcommand{\htmlref}[2]{\hotlink{#1}{\S\ref{#2}}}
% \renewcommand{\xref}[3]{\hotlink{#1}{#2 -- #3}}
%end{latexonly}
% -----------------------------------------------------------------------------
% ? Document specific \newcommand or \newenvironment commands.
\newcommand{\dash}{--}
\newcommand{\listline}{\hspace{1pt}\\}
\newcommand{\listova}[1]{}
\newcommand{\listovb}[1]{\textbf{#1}\\}
\begin{htmlonly}
  \newcommand{\dash}{-}
  \newcommand{\listline}{}
  \newcommand{\listova}[1]{#1}
  \newcommand{\listovb}[1]{}
\end{htmlonly}

%+
%  Name:
%     SST.TEX

%  Purpose:
%     Define LaTeX commands for laying out Starlink routine descriptions.

%  Language:
%     LaTeX

%  Type of Module:
%     LaTeX data file.

%  Description:
%     This file defines LaTeX commands which allow routine documentation
%     produced by the SST application PROLAT to be processed by LaTeX and
%     by LaTeX2html. The contents of this file should be included in the
%     source prior to any statements that make of the sst commnds.

%  Notes:
%     The style file html.sty provided with LaTeX2html needs to be used.
%     This must be before this file.

%  Authors:
%     RFWS: R.F. Warren-Smith (STARLINK)
%     PDRAPER: P.W. Draper (Starlink - Durham University)

%  History:
%     10-SEP-1990 (RFWS):
%        Original version.
%     10-SEP-1990 (RFWS):
%        Added the implementation status section.
%     12-SEP-1990 (RFWS):
%        Added support for the usage section and adjusted various spacings.
%     8-DEC-1994 (PDRAPER):
%        Added support for simplified formatting using LaTeX2html.
%     {enter_further_changes_here}

%  Bugs:
%     {note_any_bugs_here}

%-

%  Define length variables.
\newlength{\sstbannerlength}
\newlength{\sstcaptionlength}
\newlength{\sstexampleslength}
\newlength{\sstexampleswidth}

%  Define a \tt font of the required size.
\latex{\newfont{\ssttt}{cmtt10 scaled 1095}}
\html{\newcommand{\ssttt}{\tt}}

%  Define a command to produce a routine header, including its name,
%  a purpose description and the rest of the routine's documentation.
\newcommand{\sstroutine}[3]{
   \goodbreak
   \rule{\textwidth}{0.5mm}
   \vspace{-7ex}
   \newline
   \settowidth{\sstbannerlength}{{\Large {\bf #1}}}
   \setlength{\sstcaptionlength}{\textwidth}
   \setlength{\sstexampleslength}{\textwidth}
   \addtolength{\sstbannerlength}{0.5em}
   \addtolength{\sstcaptionlength}{-2.0\sstbannerlength}
   \addtolength{\sstcaptionlength}{-5.0pt}
   \settowidth{\sstexampleswidth}{{\bf Examples:}}
   \addtolength{\sstexampleslength}{-\sstexampleswidth}
   \parbox[t]{\sstbannerlength}{\flushleft{\Large {\bf #1}}}
   \parbox[t]{\sstcaptionlength}{\center{\Large #2}}
   \parbox[t]{\sstbannerlength}{\flushright{\Large {\bf #1}}}
   \begin{description}
      #3
   \end{description}
}

%  Format the description section.
\newcommand{\sstdescription}[1]{\item[Description:] #1}

%  Format the usage section.
\newcommand{\sstusage}[1]{\item[Usage:] \mbox{}
\\[1.3ex]{\raggedright \ssttt #1}}

%  Format the invocation section.
\newcommand{\sstinvocation}[1]{\item[Invocation:]\hspace{0.4em}{\tt #1}}

%  Format the arguments section.
\newcommand{\sstarguments}[1]{
   \item[Arguments:] \mbox{} \\
   \vspace{-3.5ex}
   \begin{description}
      #1
   \end{description}
}

%  Format the returned value section (for a function).
\newcommand{\sstreturnedvalue}[1]{
   \item[Returned Value:] \mbox{} \\
   \vspace{-3.5ex}
   \begin{description}
      #1
   \end{description}
}

%  Format the parameters section (for an application).
\newcommand{\sstparameters}[1]{
   \item[Parameters:] \mbox{} \\
   \vspace{-3.5ex}
   \begin{description}
      #1
   \end{description}
}

%  Format the examples section.
\newcommand{\sstexamples}[1]{
   \item[Examples:] \mbox{} \\
   \vspace{-3.5ex}
   \begin{description}
      #1
   \end{description}
}

%  Define the format of a subsection in a normal section.
\newcommand{\sstsubsection}[1]{ \item[{#1}] \mbox{} \\}

%  Define the format of a subsection in the examples section.
\newcommand{\sstexamplesubsection}[2]{\sloppy
\item[\parbox{\sstexampleslength}{\ssttt #1}] \mbox{} \vspace{1.0ex}
\\ #2 }

%  Format the notes section.
\newcommand{\sstnotes}[1]{\item[Notes:] \mbox{} \\[1.3ex] #1}

%  Provide a general-purpose format for additional (DIY) sections.
\newcommand{\sstdiytopic}[2]{\item[{\hspace{-0.35em}#1\hspace{-0.35em}:}]
\mbox{} \\[1.3ex] #2}

%  Format the implementation status section.
\newcommand{\sstimplementationstatus}[1]{
   \item[{Implementation Status:}] \mbox{} \\[1.3ex] #1}

%  Format the bugs section.
\newcommand{\sstbugs}[1]{\item[Bugs:] #1}

%  Format a list of items while in paragraph mode.
\newcommand{\sstitemlist}[1]{
  \mbox{} \\
  \vspace{-3.5ex}
  \begin{itemize}
     #1
  \end{itemize}
}

%  Define the format of an item.
\newcommand{\sstitem}{\item}

%% Now define html equivalents of those already set. These are used by
%  latex2html and are defined in the html.sty files.
\begin{htmlonly}

%  sstroutine.
   \newcommand{\sstroutine}[3]{
      \subsection{#1\xlabel{#1}-\label{#1}#2}
      \begin{description}
         #3
      \end{description}
   }

%  sstdescription
   \newcommand{\sstdescription}[1]{\item[Description:]
      \begin{description}
         #1
      \end{description}
      \\
   }

%  sstusage
   \newcommand{\sstusage}[1]{\item[Usage:]
      \begin{description}
         {\ssttt #1}
      \end{description}
      \\
   }

%  sstinvocation
   \newcommand{\sstinvocation}[1]{\item[Invocation:]
      \begin{description}
         {\ssttt #1}
      \end{description}
      \\
   }

%  sstarguments
   \newcommand{\sstarguments}[1]{
      \item[Arguments:] \\
      \begin{description}
         #1
      \end{description}
      \\
   }

%  sstreturnedvalue
   \newcommand{\sstreturnedvalue}[1]{
      \item[Returned Value:] \\
      \begin{description}
         #1
      \end{description}
      \\
   }

%  sstparameters
   \newcommand{\sstparameters}[1]{
      \item[Parameters:] \\
      \begin{description}
         #1
      \end{description}
      \\
   }

%  sstexamples
   \newcommand{\sstexamples}[1]{
      \item[Examples:] \\
      \begin{description}
         #1
      \end{description}
      \\
   }

%  sstsubsection
   \newcommand{\sstsubsection}[1]{\item[{#1}]}

%  sstexamplesubsection
   \newcommand{\sstexamplesubsection}[2]{\item[{\ssttt #1}] #2}

%  sstnotes
   \newcommand{\sstnotes}[1]{\item[Notes:] #1 }

%  sstdiytopic
   \newcommand{\sstdiytopic}[2]{\item[{#1}] #2 }

%  sstimplementationstatus
   \newcommand{\sstimplementationstatus}[1]{
      \item[Implementation Status:] #1
   }

%  sstitemlist
   \newcommand{\sstitemlist}[1]{
      \begin{itemize}
         #1
      \end{itemize}
      \\
   }
%  sstitem
   \newcommand{\sstitem}{\item}

\end{htmlonly}

%  End of "sst.tex" layout definitions.
%.



% ? End of document specific commands
% -----------------------------------------------------------------------------
%  Title Page.
%  ===========
\renewcommand{\thepage}{\roman{page}}
\begin{document}
\thispagestyle{empty}

%  Latex document header.
%  ======================
\begin{latexonly}
   CCLRC / {\sc Rutherford Appleton Laboratory} \hfill {\bf \stardocname}\\
   {\large Particle Physics \& Astronomy Research Council}\\
   {\large Starlink Project\\}
   {\large \stardoccategory\ \stardocnumber}
   \begin{flushright}
   \stardocauthors\\
   \stardocdate
   \end{flushright}
   \vspace{-4mm}
   \rule{\textwidth}{0.5mm}
   \vspace{5mm}
   \begin{center}
   {\Huge\bf  \stardoctitle \\ [2.5ex]}
   {\LARGE\bf \stardocversion \\ [4ex]}
   {\Huge\bf  \stardocmanual}
   \end{center}
   \vspace{5mm}

% ? Add picture here if required for the LaTeX version.
%   e.g. \includegraphics[scale=0.3]{filename.ps}
% ? End of picture

% ? Heading for abstract if used.
   \vspace{10mm}
   \begin{center}
      {\Large\bf Abstract}
   \end{center}
% ? End of heading for abstract.
\end{latexonly}

%  HTML documentation header.
%  ==========================
\begin{htmlonly}
   \xlabel{}
   \begin{rawhtml} <H1> \end{rawhtml}
      \stardoctitle\\
      \stardocversion\\
      \stardocmanual
   \begin{rawhtml} </H1> \end{rawhtml}

% ? Add picture here if required for the hypertext version.
%   e.g. \includegraphics[scale=0.7]{filename.ps}
% ? End of picture

   \begin{rawhtml} <P> <I> \end{rawhtml}
   \stardoccategory\ \stardocnumber \\
   \stardocauthors \\
   \stardocdate
   \begin{rawhtml} </I> </P> <H3> \end{rawhtml}
      \htmladdnormallink{CCLRC}{http://www.cclrc.ac.uk} /
      \htmladdnormallink{Rutherford Appleton Laboratory}
                        {http://www.cclrc.ac.uk/ral} \\
      \htmladdnormallink{Particle Physics \& Astronomy Research Council}
                        {http://www.pparc.ac.uk} \\
   \begin{rawhtml} </H3> <H2> \end{rawhtml}
      \htmladdnormallink{Starlink Project}{http://www.starlink.ac.uk/}
   \begin{rawhtml} </H2> \end{rawhtml}
   \htmladdnormallink{\htmladdimg{source.gif} Retrieve hardcopy}
      {http://www.starlink.ac.uk/cgi-bin/hcserver?\stardocsource}\\

%  HTML document table of contents.
%  ================================
%  Add table of contents header and a navigation button to return to this
%  point in the document (this should always go before the abstract \section).
  \label{stardoccontents}
  \begin{rawhtml}
    <HR>
    <H2>Contents</H2>
  \end{rawhtml}
  \htmladdtonavigation{\htmlref{\htmladdimg{contents_motif.gif}}
        {stardoccontents}}

% ? New section for abstract if used.
  \section{\xlabel{abstract}Abstract}
% ? End of new section for abstract
\end{htmlonly}

% -----------------------------------------------------------------------------
% ? Document Abstract. (if used)
%  ==================
\stardocabstract
% ? End of document abstract
% -----------------------------------------------------------------------------
% ? Latex document Table of Contents (if used).
%  ===========================================
  \newpage
  \begin{latexonly}
    \setlength{\parskip}{0mm}
    \tableofcontents
    \setlength{\parskip}{\medskipamount}
    \markboth{\stardocname}{\stardocname}
  \end{latexonly}
% ? End of Latex document table of contents
% -----------------------------------------------------------------------------
\cleardoublepage
\renewcommand{\thepage}{\arabic{page}}
\setcounter{page}{1}

\section{\xlabel{introduction}Introduction}

Consider an application that subtracts four from every element of an
array.  This has a limited use \dash\ just one invocation.  However, it would
be more general and far more useful if we could subtract an arbitrary
value from an arbitrary array.  These arbitrary values are called {\em
parameters\/} and through them we may qualify the behaviour of an
application.  In a sense they are like input-output channels, though
instead of using logical-unit numbers they have memorable names.

PAR is a library of subroutines that makes it easier to import and
export numeric, character, and logical parameters between applications
and the outside world.  In a contouring application, for example, these
three types of parameters might be used to find the contour heights, the
title for the graph, and whether or not the contours are to be
annotated. Currently, PAR is only available for {\footnotesize ADAM}
applications.  However, this does mean that an application using PAR
will be able to take advantage of features in the {\footnotesize ADAM}
parameter system, such as a variety of defaulting mechanisms, a special
null value, online help, and automatic type conversion.  These make the
application easier to use, more powerful, and more flexible.

PAR enables parameter values to be obtained, with or without
constraints.  Supported constraints are:
\begin{itemize}
\item even, and odd integers;
\item within or beyond a range of values; and
\item selection from a menu of options,
whose contents may be abbreviated.
\end{itemize}
There are facilities for setting the prompt string, suggested-default
values, and minimum and maximum permitted values. There is a routine
that lets an application obtain new values repeatedly through the same
parameter.  Results may be put into parameters to be used in procedures
and/or passed on to other applications.  Finally, PAR can inquire the
{\em state\/} of a parameter.

This document explains what a parameter is and the states it may reside
in; it describes the usage of the PAR library with examples, progressing
from the basic routines to the more sophisticated ones; and gives the
linking procedures.  There is a reference section containing full
details of each subroutine, and classified lists of the routines.

As a consequence of PAR only being implemented using the {\footnotesize
ADAM} parameter system (SUBPAR), the programmer must also write an {\it
interface file} for each application.  This controls the behaviour of
the parameter, for example, whether it is defaulted or prompted for,
where a suggested default should be obtained from, what is the default
value, and where help is found.
\xref{SUN/115}{sun115}{} \dash\ the {\it Interface Module
Reference Manual\/} \dash\ describes the syntax and facilities of interface
files, and is required reading for those using PAR.  Note that the
examples in this document assume the {\footnotesize ADAM} implementation.

For the same reason, many error messages you see will be prefixed by 'SUBPAR:',
indicating that they emanate from the SUBPAR library.

\section{\xlabel{what_is_a_parameter}What is a Parameter?}

From the programmer's point of view, a PAR parameter is a communication
channel between an application and the outside world.  In a traditional
application you would probably use Fortran input-output statements to
prompt the user, or read from a text file for the additional data needed
by your programme.  The concept of a PAR parameter is similar to this,
but instead of passing the information through a logical-unit number,
PAR uses a named communication channel.  This has advantages over normal
methods.  A name is more memorable both to the user and to the
programmer, especially if it describes the parameter's function.  PAR
also allows more sophisticated ways of controlling the behaviour of an
application than the traditional Fortran I/O, and performs data-type
conversion automatically.

\section{\xlabel{parameter_states_}Parameter States} \label{se:states}

A parameter has a number of states.  Initially, a parameter has no
value, and is said to be in the {\em ground\/} state.  When the
parameter has been given a value, the parameter moves to the {\em
active\/} state.  A parameter acquires a value in the first instance by
looking for one supplied on the command line.  Failing that, the
parameter system will attempt to get a value, for example by prompting
or adopting a fixed or dynamic default.  If an application obtains a
value from a parameter already in the active state, the existing value
for that parameter is returned.  To obtain a further value the
application must first {\em cancel\/} the parameter and move it to the
{\em cancelled\/} state.  When the application next gets a value for
that parameter (within the same invocation), the parameter returns to
the active state.  When an application completes the parameter returns
to the ground state. PAR also permits users of your application to enter
a null value for a parameter (a {\tt !} in the {\footnotesize ADAM}
parameter system).  This puts the parameter into the {\em null\/} state,
and can be used to select a special value or control the application in
a special way.

\section{\xlabel{error_handling}Error Handling}
\label{se:error}

Like other Starlink subroutine libraries, PAR adheres throughout to the
error-handling strategy described in
\xref{SUN/104}{sun104}{}.
Subroutines have an
integer final argument called STATUS that is an inherited status, and
will return without action unless the status is set to the value
SAI\_\_OK when they are invoked.  (SAI\_\_OK is a symbolic constant
defined in the include file SAE\_PAR.)  This strategy greatly simplifies
the coding since you can make a series of subroutine calls without
having to check STATUS after each call; if an error occurs the
subsequent routines do nothing. \footnote{PAR\_CANCL and PAR\_UNSET are
exceptions to this behaviour.  Since they are freeing resources they
attempt to work regardless of the inherited status.}  Where necessary,
PAR makes error reports through the ERR\_ routines in the manner
described in
\xref{SUN/104}{sun104}{}, and therefore sets STATUS to a value other than
SAI\_\_OK.


\section{\xlabel{basic_routines}Basic Routines}

\subsection{\xlabel{getting_a_parameters_value}Getting a Parameter's Value}

The way an application obtains a value from a parameter is to {\it
get\/} the value.  In simple terms, a get operation is the equivalent of
a Fortran read, though it can do much more.  In this example,

\begin{verbatim}
          CALL PAR_GET0L( 'FLAG', VALUE, STATUS )
\end{verbatim}

gets a logical value from the parameter FLAG and stores it in the
logical variable VALUE.  The data type is logical because of the L
suffix in the routine's name.  If VALUE were a character variable, you
would use a similar routine, but this time with a C suffix as shown below.

\begin{verbatim}
          CALL PAR_GET0C( 'STRING', VALUE, STATUS )
\end{verbatim}

There are similar routines for the other primitive data types, each with
an appropriate suffix: D for double precision, I for integer and R for
real.  By convention where a routine has variants for different data
types it has an ``x'' suffix, so in this case the routines are known
collectively as PAR\_GET0x.  Note that there are no routines for
obtaining byte and word values.  These must be obtained using the
integer version of the routine.  The variable STATUS is the inherited
status value adhering to the rules in
\xref{SUN/104}{sun104}{},
and summarised in
Section~\ref{se:error}.  The 0 in the routine name indicates that the
routine passes a scalar (zero dimensions).  Likewise a routine with a 1
in its name passes a vector, and with an N it passes an $n$-dimensional
array.

There is a maximum length for the parameter name (set by {\footnotesize
ADAM}) of fifteen characters, but this is quite adequate except
perhaps for lexicographers.

The actual value of the parameter supplied to the application at run
time need not necessarily have the same data type as you have requested
in your application. The parameter system will handle conversions
between the type of the value and the type required by the
application.  So you only need to concern yourself with what your
application requires.

What does the user see?  This will depend on the implementation, but
suppose the user is prompted for a value.  If you had in your code

\begin{verbatim}
          CALL PAR_GET0R( 'THRESHOLD', THRESH, STATUS )
\end{verbatim}

The user might see

\begin{verbatim}
    THRESHOLD - Give the intensity threshold >
\end{verbatim}

with the parameter name first, followed by a prompt string briefly
describing the function of the parameter.  (In the {\footnotesize ADAM}
implementation this string is usually defined within the interface file,
but there is a PAR routine for setting it within the application
itself.)  Suppose the user enters 32768 in response to the above prompt.
The next time the user ran the application there could be a suggested
default of the last-used value like this,

\begin{verbatim}
    THRESHOLD - Give the intensity threshold /32768/ >
\end{verbatim}

which could be accepted merely by pressing the return key.

In the {\footnotesize ADAM} implementation the PAR\_GET calls will
(re-)prompt whenever the user gives too many values or an invalid value.
Invalid values include supplying a character string where a number is
required, and being outside the interface file {\em range\/} or
{\em in\/} constraints.

\subsection{\xlabel{getting_an_array_of_values}Getting an Array of Values}

In addition to obtaining a scalar value, there are routines for getting
arrays of values.  At this point it is appropriate to say a few words
about the shape of a parameter.  A parameter's shape may be scalar,
vector, or $n$-dimensional.  When you get values from a parameter, the
parameter acquires the shape of the values supplied by the user.
Therefore, you could use the same parameter to obtain a scalar and a
cube.  How much is the user restricted by the parameter shape?  When
getting an array the application specifies the maximum number of
dimensions and the maximum sizes along each dimension; the values
supplied must not exceed these maxima, but there may be fewer.  Now we
return to the descriptions of the basic PAR\_ routines.

Here is an example to get a vector of real values.

\begin{verbatim}
          REAL VALUES( 4 )

              :       :       :

          CALL PAR_GET1R( 'HEIGHTS', 4, VALUES, ACTVAL, STATUS )
\end{verbatim}

It enables all or part of a four-element array to be obtained and
stored in the real variable VALUES.  ACTVAL is the actual number of
values obtained from parameter HEIGHTS.  Again there are versions of
this routine for the different data types listed earlier.

In the above example the user might enter

\begin{verbatim}
          [1,2.5D0,3.456]
\end{verbatim}

where the brackets indicate an array and the commas
separate the values.\footnote{In the ADAM implementation the user can
omit the brackets in response to a prompt.}  For this input, ACTVAL
becomes 3.

A parameter's values may be stored as an $n$-dimensional array.
The following example shows a 3-dimensional mask of integers
being obtained from parameter MASK and being stored in an array called
VALUES.  MAXD gives the maximum dimensions of the array.  The array
itself must be large enough to hold the values.  ACTD receives the
actual dimensions of the array as specified by the user.

\begin{verbatim}
          INTEGER VALUES( 3, 2, 2 )

              :       :       :

          MAXD( 1 ) = 3
          MAXD( 2 ) = 2
          MAXD( 3 ) = 2
          CALL PAR_GETNI( 'MASK', 3, MAXD, VALUES, ACTD, STATUS )
\end{verbatim}

This routine is unlikely to be in common usage because of the
impracticality of the user of your application entering more than the
smallest $n$-dimensional arrays of parameters, and having to know the
parameter's shape.  To illustrate the former point, in the above
example the user might enter
\begin{verbatim}
          [[[1,2,3],[4,5,6]],[[7,8,9],[10,11,12]]]
\end{verbatim}

where the nested brackets delimit the dimensions, and the commas
separate the values within a dimension.  Thus there are three values
along the first dimension; each of these triples is bracketed in pairs
along the second dimension; and this is repeated for each element along
the third dimension, within the outermost brackets.  In other words the
first dimension increases fastest, followed by the second, and so
on \dash\ the Fortran order.  Thus, in the example, VALUES(3,1,2) is 9.

Usually, it will suffice to get a vector of values disregarding
the dimensionality of the parameter.

\begin{verbatim}
          REAL VALUES( 12 )

              :       :       :

          CALL PAR_GETVR( 'MASK', 12, VALUES, ACTVAL, STATUS )
\end{verbatim}

This has the same arguments as PAR\_GET1x we saw above, though the actual
number of array dimensions of the parameter need not be one.

There may be occasions when you want an exact number of values.
Below, we obtain two character variables from parameter 'CONSTELLATION'.
When too few values are supplied to the parameter, PAR\_EXACx
prompts for the remainder of the values.

\begin{verbatim}
          CHARACTER * ( 3 ) CONSTE( 2 )

              :       :       :

          CALL PAR_EXACC( 'CONSTELLATION', 2, CONSTE, STATUS )
\end{verbatim}

An example of what the user sees is given in Section~\ref{se:avl}.

\subsection{\xlabel{putting_a_parameters_value}Putting a Parameter's Value}

The opposite to {\em get\/} is {\em put}.  Using traditional Fortran
input-output, the equivalent of putting a value is to write the value to
a file.  In addition to getting parameter values, PAR can also put
values into parameters.  Suppose we have an application that computes
statistical parameters for a data array.  We should like to store these
statistics in parameters, so that they may be passed to a later
application.  Thus we could have a number of puts \dash\ one per statistic.
Here we just put the standard deviation value in the parameter SIGMA.

\begin{verbatim}
          CALL PAR_PUT0D( 'SIGMA', STDDEV, STATUS )
\end{verbatim}

The value is written to a {\em parameter file\/} associated with the
application.  This file contains not only any values you have put there,
but also the last-used values of all the parameters of that
application.\footnote{A parameter file is implementation specific,
however it is expected to be part of any implementation.  In the ADAM
parameter system the parameter file is an HDS file named after the
application and located in the ADAM\_USER directory. So for an application
called STATISTICS, the above parameter value could be `read' by a
parameter in a different application by the user giving the HDS object
name with an {\tt @} prefix, namely {\tt @ADAM\_USER:STATISTICS.SIGMA}
for VMS and {\tt @\$ADAM\_USER/statistics.SIGMA} for UNIX.}

If we had the values corresponding to the quartiles and median of a
data array we could use a routine that handles a 1-dimensional array.
\begin{verbatim}
          DOUBLE PRECISION QRTILE( 3 )

              :       :       :

          CALL PAR_PUT1D( 'QUARTILES', 3, QRTILE, STATUS )
\end{verbatim}

When you put an array, the subroutine arguments specify both the
dimensionality and size of that array, and the size of an `object' into
which the array is to be put.\footnote{In the ADAM implementation this
is usually a component of the parameter file \dash\ in this case an object of
the required size will be created.  However, if indirection to a named
HDS file is used the shape specified in the PAR routine must match the
object's shape.}

Analogous to the PAR\_GETNx and PAR\_GETVx routines for getting arrays
of values, there are equivalent generic routines, PAR\_PUTNx and
PAR\_PUTVx, for putting values in parameters.  Using PAR\_PUTNx in one
application and accessing the array via PAR\_GETNx without prompting is
the most likely way to handle $n$-dimensional arrays of parameter
values.

See
\xref{SUN/115}{sun115}{parameter_specification_for_output_parameters}
\latex{Appendix~C} for details of the specifications needed within
the interface file to deal with scalar and array output parameters.

\subsection{\xlabel{cancelling_a_parameter}Cancelling a Parameter}

Cancelling a parameter means that values associated with it are lost.
It is most-commonly used for processing in a loop.

If we have a loop in which we wish to get new values for a parameter and
do some calculations in each iteration, we must cancel any parameters
obtained in the loop, otherwise the first value will be used repeatedly,
since the parameter is in the active state.

Here is an illustration.  ITER calculations are performed, using
the value of parameter REJECT.

\begin{verbatim}
          DO I = 1, ITER
             CALL PAR_GET0D( 'REJECT', ABC, STATUS )

                 < perform calculation involving variable ABC >

             CALL PAR_CANCL( 'REJECT', STATUS )
          END DO
\end{verbatim}

\subsection{\xlabel{setting_the_dynamic_prompt}Setting the Dynamic Prompt}

In the ADAM implementation, the prompt a user of an application sees
usually originates from the interface file, and hence is fixed.
However, it is possible to create a prompt string dynamically from
within the application. For example,

\begin{verbatim}
          CALL PAR_PROMT( 'OK', 'The file '// NAME( :NC ) /
         :                / ' is to be erased. OK ?', STATUS )
\end{verbatim}

will change the prompt string for parameter OK, so that the named
file given by the variable NAME, may be substituted at run time.

\subsection{\xlabel{setting_the_dynamic_default}Setting the Dynamic Default}

When the parameter system prompts for a parameter, there is usually a
suggested default given between {\tt / /} delimiters, so that the user
can just press {\tt <CR>} to use the suggestion, or edit the suggestion.
The suggested value can originate from a number of places.  This
includes from within an application, by setting a {\em dynamic\/}
default.

In the following example the dynamic default for the scalar parameter
SHADE depends on the value of variable COLOUR.  If COLOUR is one, the
dynamic default for parameter SHADE will be $-$1.0, and 0.0 otherwise.

\begin{verbatim}
    *  Set the default shading (parameter SHADE) depending on the colour.
          IF ( COLOUR .EQ. 1 ) THEN
             CALL PAR_DEF0R( 'SHADE', -1.0, STATUS )
          ELSE
             CALL PAR_DEF0R( 'SHADE', 0.0, STATUS )
          END IF

    *  Obtain the shading.
          CALL PAR_GET0R( 'SHADE', SHADE, STATUS )
\end{verbatim}


Note that in {\footnotesize ADAM} applications the {\em ppath\/} in the
interface file must begin {\tt DYNAMIC} to ensure that the user of your
application will be prompted with the dynamic default (see
\xref{SUN/115}{sun115}{the_ppath_field}).

The {\footnotesize ADAM} parameter system also allows the dynamic
default to be used as the parameter value without prompting, through
the interface file {\em vpath\/} (see
\xref{SUN/115}{sun115}{the_vpath_field}).


There are also routines that set the dynamic default for a vector
parameter (PAR\_DEF1x) and an array parameter (PAR\_DEFNx).   Again
there are versions of this routine for the different data types listed
earlier.

\subsection{\xlabel{setting_lower_and_upper_limits}Setting Lower and Upper Limits}

It is a common requirement to restrict a parameter's value to a specific
range, or to be above some lower limit or less than some upper limit.
For example, the order of a polynomial would have to be positive, but
less than some maximum value that the polynomial-fitting software can
handle.  Whilst this can be done using the {\em range\/} field in the
interface file, PAR offers dynamic control through two
routines \dash\ PAR\_MAXx to set the maximum value, and PAR\_MINx to set the
minimum value for the parameter.  Note that there are no logical-type
versions of these routines.

Here is an example of setting the minimum value.

\begin{verbatim}
    *  The number of histogram bins must be positive.
          CALL PAR_MINI( 'NBINS', 1, STATUS )
          CALL PAR_GET0I( 'NBINS', NBINS, STATUS )
\end{verbatim}

We can set an upper limit too.
\begin{verbatim}
    *  The number of histogram bins must be in the range 10 to 1000
    *  inclusive.
          CALL PAR_MINI( 'NBINS', 10, STATUS )
          CALL PAR_MAXI( 'NBINS', 1000, STATUS )
          CALL PAR_GET0I( 'NBINS', NBINS, STATUS )
\end{verbatim}

If the user gives a value outside the legitimate range, an error report
appears, and the user is prompted for an acceptable value. In the case
of character parameters, the parameter's value follows the minimum value
and precedes the maximum value in the ASCII collating list. \footnote{In
the ADAM implementation the dynamic minimum and maximum must lie within
the limits set by the range field in the interface file.  If not,
the get routine returns with an error status.  The allowed values reside
in the intersection set of the two ranges.  Users can also specify MIN
or MAX for a parameter value to assign the minimum or maximum value to
that parameter.}

If the minimum value is greater than the maximum at the time you
get a parameter value, this instructs PAR not to permit values between
the minimum and maximum.  So the maximum is always the value immediately
below the range to be excluded, and the minimum is the value immediately
above.  In case that is not clear here is an illustration.

\begin{verbatim}
    *  Choose an integer excluding -1, 0, 1, and 2.
          CALL PAR_MINI( 'IVALUE', 3, STATUS )
          CALL PAR_MAXI( 'IVALUE', -2, STATUS )
          CALL PAR_GET0I( 'IVALUE', NUMBER, STATUS )
\end{verbatim}

\subsection{\xlabel{cancelling_dynamic_values}Cancelling Dynamic Values}

\subsubsection{Range}

Having set a minimum or a maximum value, you can alter its value by a
further call to PAR\_MINx or PAR\_MAXx.  Like a parameter itself, these
control values remain in effect until the end of an application, or
until they are cancelled.  Therefore, should you no longer want either
or both of the maximum and minimum limits, you must cancel them with
PAR\_UNSET.  This routine's second argument is a comma-separated list
which selects the values to reset.

In the following example parameter NX is obtained twice, first using a
range, and then with no maximum limit and a different minimum.

\begin{verbatim}
    *  Get the integer value between 1 and 360.
          CALL PAR_MINI( 'NX', 1, STATUS )
          CALL PAR_MAXI( 'NX', 360, STATUS )
          CALL PAR_GET0I( 'NX', NX, STATUS )

    *  Cancel the parameter, as a minimum or maximum value cannot be set
    *  when the parameter is active, and because we want to obtain
    *  another value.
          CALL PAR_CANCL( 'NX', STATUS )

    *  Set a new minimum value.
          CALL PAR_MINI( 'NX', NX, STATUS )

    *  Cancel the maximum value.
          CALL PAR_UNSET( 'NX', 'MAX', STATUS )

    *  Obtain a new value, with a new minimum and no maximum.
          CALL PAR_GET0I( 'NX', NX2, STATUS )
\end{verbatim}

See Appendix~\ref{ap:ref} for details of the options for the second
argument.

\subsubsection{Dynamic Defaults}

PAR\_UNSET need not be called to change the value of a default; a call
to one of the PAR\_DEF routines will suffice.  However, if you wish to
no longer have a dynamic default for parameter NX say, you will need to
make the following call.

\begin{verbatim}
          CALL PAR_UNSET( 'NX', 'DEFAULT', STATUS )
\end{verbatim}

\subsection{\xlabel{inquiring_the_state_of_a_parameter}Inquiring the State of a Parameter}

PAR provides a routine for inquiring the state of a parameter.
One practical use for this is to determine whether a parameter is
specified on the command line or not, and hence to affect the behaviour
of the application.  For example, you have an application that loops for
efficient and easy interactive use, but in a procedure or batch mode you
want the application to process just one set of data.

\begin{verbatim}
          INCLUDE 'PAR_PAR'        ! PAR constants
          INTEGER STATE

              :       :       :

          CALL PAR_STATE( 'INIT', STATE, STATUS )

          LOOP = .TRUE.
          DO WHILE ( LOOP .AND STATUS .EQ. SAI__OK )
             CALL PAR_GET0R( 'INIT', START, STATUS )

                 < perform calculation >

             LOOP = STATE .NE. PAR__ACTIVE
             IF ( LOOP ) CALL PAR_CANCL( 'INIT', STATUS )
          END DO
\end{verbatim}

INIT is a parameter required for each calculation.  It is obtained
within a code loop.  If INIT is specified on the command-line, INIT is
in the active state before the call to PAR\_GET0R.  So a logical
expression involving the state decides whether there is but one cycle
around the loop or many.  The include file PAR\_PAR contains the
definitions of each of the states returned by PAR\_STATE. The next
section describes how we might end the loop in the interactive case.

\section{\xlabel{abort_and_null}Abort and Null}

There are two special status values that can be returned by the parameter
system when getting parameter values.  They are PAR\_\_ABORT and
PAR\_\_NULL, and are defined in the include file PAR\_ERR.

\subsection{\xlabel{abort}Abort}

PAR\_\_ABORT is returned if a user enters {\tt !!} in response to a
prompt.  In this event you should ensure that your application exits
immediately.  In practice this means that a) the user is not be prompted
for any further parameters, though you do not have to check for an abort
status after every PAR\_GET call \dash\ the inherited status will look after
that \dash\ and b) the application exits cleanly, freeing any resources used.
There must be a status check before any of the parameter values are
used, in case they have not been obtained successfully.

For example
\begin{verbatim}
          INCLUDE 'PAR_ERR'       ! PAR error constants

              :       :       :

          CALL PAR_GET0I( 'STEP', NSTEP, STATUS )
          CALL PAR_GET1R( 'HEIGHTS', 50, HEITS, NUMHTS, STATUS )
          CALL PAR_GET0L( 'COLOUR', COLOUR, STATUS )
          IF ( STATUS .EQ. PAR__ABORT ) THEN
             < perform any tidying operations >
             GOTO 999
          END IF

             < perform calculations >

      999 CONTINUE
          END

\end{verbatim}
performs three parameter get calls, before testing for an
abort; if an abort status is found, the routine exits.  The status check
against PAR\_\_ABORT is unusual, and it is included to illustrate the
symbolic constant.  In normal circumstances you would test whether
STATUS and SAI\_\_OK were not equal, since it checks whether any error
has occurred.

\subsection{\xlabel{null}Null}

PAR\_\_NULL indicates that no value is available, and may be interpreted
according to circumstances.  This normally occurs when a user assigns a
value of {\tt !} to a parameter.

Null can be used in a variety of circumstances.  These include to
end a loop, to indicate that an optional file is not required, and to
force a default value to be used.

Here is an example of handling null as a valid value for a parameter.

\begin{verbatim}
          CALL ERR_MARK
          CALL PAR_GET0R( 'WEIGHT', WEIGHT, STATUS )
          IF ( STATUS .EQ. PAR__NULL ) THEN
             WEIGHT = 1.0
             CALL ERR_ANNUL( STATUS )
          END IF
          CALL ERR_RLSE
\end{verbatim}

When the PAR\_GET0R returns with the null status, the variable WEIGHT is
set to 1.0.  This particular instance has limited value, but the
constant could be dynamic, depending on the values of other parameters
or data read from files.  The call to ERR\_ANNUL restores STATUS to
SAI\_\_OK, and also removes an error message that would otherwise be
reported later (usually when the application completes).  Since we do
not want to lose any earlier error reports, ERR\_MARK and ERR\_RLSE
bracket this fragment of code to set up and release an error context.
See
\xref{SUN/104}{sun104}{}
for further details.

Here null ends a loop, say to plot an histogram with different numbers
of bins.
\begin{verbatim}
      100 CONTINUE
          CALL ERR_MARK
          CALL PAR_GET0I( 'NBINS', NBIN, STATUS )
          IF ( STATUS .EQ. PAR__NULL ) THEN
             CALL ERR_ANNUL( STATUS )
             CALL ERR_RLSE
             GOTO 200
          END IF
          CALL ERR_RLSE

              < Perform calculations >

          GOTO 100

    *   Come here at the end of the calculations.
      200 CONTINUE

\end{verbatim}

\section{\xlabel{extended_routines}Extended Routines}
\label{se:extended}

In addition to the basic routines there are some packaged facilities for
common sequences of calls.  Most deal with dynamic control of acceptable
values when getting the parameter.  Some are needed because the {\em
in\/} choices and {\em range\/} limits imposed by the {\footnotesize
ADAM} interface module are static, and therefore cannot be adjusted
depending on the values of other parameters or data.  Note that in the
extended routines that follow, all call a PAR\_GET$n$x routine which
first checks that each supplied value satisfies these interface-file
constraints, and prompts for a new value if it does not.  Only after
these tests are made will each value be compared with the constraints in
the extended routines.  Therefore, you are recommended not to use the in
and range fields for parameters obtained using the extended routines.

Here is an example of handling your own constraints to illustrate why it
is more convenient to use the packaged routines, where applicable, and
also to demonstrate how you might write your own extended routine should
you need one. It is a little contrived, since the PAR library already
provides the most-common combinations.

Here we obtain from parameter NAME an integer that is not exactly
divisible by 3, set the dynamic default to 1, and interpret null to
mean set the value to 3.  See
\xref{SUN/104}{sun104}{} for details of the MSG\_ calls.

\begin{verbatim}
          LOGICAL NOTOK
          INTEGER VALUE

              :       :       :

    *  Set the dynamic default.
          CALL PAR_DEF0I( 'NAME', 1, STATUS )

    *  Start a new error context.
          CALL ERR_MARK

    *  Loop to obtain the value of the parameter.
    *  ==========================================

    *  Initialise NOTOK to start off the loop and indicate that a
    *  satisfactory value has yet to be obtained.
          NOTOK = .TRUE.

      100 CONTINUE

    *  The loop will keep going as long as a suitable value has not be read
    *  and there is no error.
             IF ( .NOT. NOTOK .OR. ( STATUS .NE. SAI__OK ) ) GOTO 120

    *  Get a value from the parameter.
             CALL PAR_GET0I( 'NAME', VALUE, STATUS )

    *  Check for an error before using the value.
             IF ( STATUS .EQ. SAI__OK ) THEN

    *  Test the value against the constraints.  Here it is just to see
    *  if the value is divisible by 3.  You can replace this expression
    *  with more complicated constraints.
                NOTOK = MOD( VALUE, 3 ) .EQ. 0

    *  The value is not within the constraints, so report as an error,
    *  including full information using tokens.  You would substitute a
    *  routine name for fac_xxxxx.
                IF ( NOTOK ) THEN
                   STATUS = SAI__ERROR
                   CALL MSG_SETC( 'PARAM, 'NAME' )
                   CALL MSG_SETI( 'VALUE', VALUE )

                   CALL ERR_REP( 'fac_xxxxx_OUTR',
         :           '^VALUE is not permitted for ^PARAM.  Please give '/
         :           /'an integer not exactly divisible by 3.', STATUS )

    *  The error is flushed so the user can see it immediately.
                   CALL ERR_FLUSH( STATUS )

    *  Cancel the parameter to enable a retry to get a value satisfying
    *  the constraint.
                   CALL PAR_CANCL( 'NAME', STATUS )
                END IF

    *  Use the default value following an error.
             ELSE

    *  Annul a null error to prevent an error report about null appearing.
    *  Create a message informing the user of what has happened.
                IF ( STATUS .EQ. PAR__NULL ) THEN
                   CALL ERR_ANNUL( STATUS )

    *  If MSG verbose output is requested, inform the user what has happened.
    *  You would substitute a routine name for fac_xxxxx.
                   CALL MSG_SETC( 'PARAM', 'NAME' )
                   CALL MSG_OUTIF( MSG__VERB, 'fac_xxxxx_DEFA',
         :           'A value of 3 has been adopted '/
         :           /'for parameter ^PARAM.', STATUS )
                END IF

    *  Set the returned value to the special case.
                VALUE = 3

    *  Terminate the loop.
                NOTOK = .FALSE.
             END IF

    *  Go to the head of the loop.
             GOTO 100

    *  Come here when the loop has been exited.  This includes when
    *  an error status was returned by the PAR_GET0I routine.
      120 CONTINUE

    *  Release the new error context.
          CALL ERR_RLSE
\end{verbatim}

To set another constraint you have to modify the logical expression
for variable NOTOK, and revise the ERR\_REP error report, possibly
with more tokens.  Suppose variables VMIN and VMAX define a range
of permitted values, which is inclusive when VMAX is the larger of
the two and exclusive when VMAX is less than VMIN.  The constraint
expression could modified to the following.

\begin{verbatim}
    *  Check that the value is within the specified include or exclude
    *  range, and not divisible by 3.
            IF ( VMIN .GT. VMAX ) THEN
               NOTOK = ( ( VALUE .LT. VMIN ) .AND. ( VALUE .GT. VMAX ) )
     :                 .OR. ( MOD( VALUE, 3 ) .EQ. 0 )
            ELSE
               NOTOK = ( VALUE .LT. VMIN ) .OR. ( VALUE .GT. VMAX ) .OR.
     :                 ( MOD( VALUE, 3 ) .NE. 0 )
            END IF
\end{verbatim}

and the error report would look something like this

\begin{verbatim}
    *  The value is not within the constraints, so report as an error,
    *  including full information using tokens.
                IF ( NOTOK ) THEN
                   STATUS = PAR__ERROR
                   CALL MSG_SETC( 'PARAM', 'NAME' )
                   CALL MSG_SETI( 'VALUE', VALUE )
                   CALL MSG_SETI( 'MIN', VMIN )
                   CALL MSG_SETI( 'MAX', VMAX )
                   IF ( VMIN .GT. VMAX ) THEN
                      CALL MSG_SETC( 'XCLD', 'outside' )
                   ELSE
                      CALL MSG_SETC( 'XCLD', 'in' )
                   END IF

                   CALL ERR_REP( 'fac_xxxxx_OUTR',
     :               '^VALUE is not permitted for ^PARAM.  Please give '/
     :               /'an integer ^XCLD the range ^MIN to ^MAX, and not '/
     :               /'exactly divisible by 3.', STATUS )
\end{verbatim}

In practice you would probably define a logical variable outside the loop
to indicate whether the range was inclusive or exclusive.\footnote{In
the ADAM implementation users of your application may wish to take
advantage of the MAX/MIN facility.  If this is so you will need to add
calls to PAR\_MAXI and PAR\_MINI in the previous example.}

Having looked `behind the scenes', we can now look at what the PAR\_ extended
routines offer.

\subsection{\xlabel{scalar_values_between_limits}Scalar Values between Limits}
\label{se:range}

A parameter value may be forced to lie within a range using a call to
the generic routine PAR\_GDR0x.  The name is derived from {\em G\/}et
with a {\em D\/}efault and {\em R\/}ange.  For example,

\begin{verbatim}
          CALL PAR_GDR0R( 'SCALE', 1.0, 0.0, 2.0, .FALSE., SCAFAC, STATUS )
\end{verbatim}

gets the value for the real parameter SCALE using a dynamic default of
1.0, stores the value in the variable SCAFAC, and ensures that the value
lies in the range 0.0\dash2.0.  If a supplied value is out of this range,
PAR\_GDR0R reports the acceptable limits and prompts the user for
another value. (This applies to all the routines with constraints
described in Section~\ref{se:extended}.)  Thus the user would see
something like the following.

\begin{verbatim}
    SCALE - Scale factor /1.0/ > 3.5
    !! SUBPAR: 3.5 is greater than the MAXIMUM value 2.
    SCALE - Scale factor /1.0/ >
\end{verbatim}

The fifth argument (NULL) will normally be false.  When it is true, it
instructs the PAR routine to return the dynamic-default value whenever a
null value is supplied, and then to annul the error status. If the
\xref{MSG filtering level}{sun104}{conditional_message_reporting}
\latex{ (see SUN/104)} is set to `verbose', an informational message will be
output.

In the above case if NULL were made .TRUE., and the user entered the null
symbol whilst in `verbose' mode, the dialogue would be as follows:

\begin{verbatim}
    SCALE - Scale factor /1.0/ > !
    !! A value of 1 has been adopted for parameter SCALE.
\end{verbatim}

NULL should only be set true when the dynamic default will {\em always\/} give
reasonable behaviour in the application.

For all but the last example in Section~\ref{se:extended}, the null flag
is set to .FALSE., and will not be mentioned again until then.

Swapping the limits lets you exclude values in the range.  Therefore,

\begin{verbatim}
          CALL PAR_GDR0R( 'SCALE', 1.0, 2.0, 0.0, .FALSE., SCAFAC, STATUS )
\end{verbatim}

would permit a value of 3.5, unlike before, as well as 0.0 and 2.0.
However, 1.5 would not now be acceptable.

\begin{verbatim}
    SCALE - Scale factor > 1.5
    !! SUBPAR: 1.5 is in the excluded MIN/MAX range between 0 and 2.
    SCALE - Scale factor >
\end{verbatim}

Notice that the suggested default has disappeared.\footnote{This assumes
that the ppath in the ADAM interface file starts with {\tt 'DYNAMIC'}.}
This occurred because 1.0 violates the range constraint, and it is a
feature of the PAR\_ extended routines.  If you do not want a dynamic
default, set the dynamic-default argument to a value that violates the
constraints.  The range-exclusion feature is present in all the PAR\_
extended routines that have a range constraint.

Not surprisingly, there are only versions of the range-constraint
routines for integer, real, or double-precision parameters.   These have
the usual I, R, and D suffices respectively.  The range limits have the
same data type as the value.

\subsection{\xlabel{arrays_of_values_between_limits}Arrays of Values between Limits}
\label{se:avl}

There are similar generic routines for obtaining an array of values
between limits.  The first permits up to a given number of values to be
obtained.  We modify the example from the previous section to illustrate
this routine.

\begin{verbatim}
          REAL SCAFAC( 10 )

              :       :       :

          CALL PAR_GDRVR( 'SCALE', 10, 0.0, 2.0, .FALSE., SCAFAC,
         :                NVAL, STATUS )
\end{verbatim}

This time up to ten values may be obtained from parameter SCALE, and
stored in the array SCAFAC.  All the values must lie in the range
0.0\dash2.0.  Since the number of values is not fixed, there is no
dynamic-default argument.  PAR\_DEF1x can be called prior to PAR\_GDRVx,
if desired.

If you want an exact number of values, there is a routine to do this
for you.  For example, suppose that you want red, green, and blue
intensities to define a colour, then the following code would obtain
exactly three normalised intensities \dash\ one for each colour \dash\ between
zero and one.

\begin{verbatim}
    *  Get an RGB colour.  The default is yellow.
          DEFAUL( 1 ) = 1.0
          DEFAUL( 2 ) = 1.0
          DEFAUL( 3 ) = 0.0
          CALL PAR_GDR1R( 'RGB', 3, DEFAUL, 0.0, 1.0, .FALSE., RGB, STATUS )
\end{verbatim}

The default must be an array.  Should you not want a dynamic default in
this case, just set the first element of DEFAUL to be negative or
greater than one.

From the user's perspective, instructions will be given if additional
values are required.

\begin{verbatim}
    RGB - Red, green, and blue intensities /[1.0,1.0,0.0]/ > 1.0,0,0,0.5
    !! SUBPAR: No more than 3 elements are allowed for parameter RGB.
    RGB - Red, green, and blue intensities /[1.0,1.0,0.0]/ > 1.0,0.0
    !! 1 more value is still needed.
    RGB - Red, green, and blue intensities /[1.0,1.0,0.0]/ > 1.0,0.0,0.0
    !! SUBPAR: No more than 1 element is allowed for parameter RGB.
    RGB - Red, green, and blue intensities /[1.0,1.0,0.0]/ > 0.0
\end{verbatim}

First the user gives too many values, and is told how many to give.
Next too few are supplied, so PAR\_GDR1R asks for another.  Again
the user is not paying attention and gives the whole RGB value instead
of just the outstanding blue intensity.  Finally, the user gives the
last value, yielding an RGB of 1.0,0.0,0.0 or the colour red.

Another variation of the theme is when you want to have an exact number
of values, and each value is constrained to its own range.
For instance, suppose that you wanted to obtain the
two-dimensional co-ordinates of a point within a rectangle, whose bounds
are known.  The following code could obtain the required values, where
LBND and UBND define the lower and upper co-ordinates of the rectangle.

\begin{verbatim}
          DOUBLE PRECISION DEFAUL( 2 ), POS( 2 )
          DOUBLE PRECISION LBND( 2 ), UBND( 2 )

              :       :       :

    *  Set the limits of the area in which the point must lie.
          LBND( 1 ) = 0.0D0
          UBND( 1 ) = 50.0D0
          LBND( 2 ) = -20.0D0
          UBND( 2 ) = 10.0D0

    *  Get the co-ordinates of the point within the rectangle.  There
    *  is no dynamic default because the default violates the range
    *  constraint.
          DEFAUL( 1 ) = -1.0D0
          CALL PAR_GRM1D( 'POSITION', 2, DEFAUL, LBND, UBND,
         :                .FALSE., POS, STATUS )
\end{verbatim}

PAR\_GRM1x will inform the user of any value(s) that violate a range,
and prompts for new values.  The name is derived from {\em G\/}et with
{\em R\/}anges for {\em M\/}ultiple dimensions.\footnote{In retrospect
the name probably should have been PAR\_GDM1x for {\em G\/}et with
{\em D\/}efaults and {\em M\/}ultiple ranges, but the existing name is
already in use in applications.}

If you want to constrain each value of an array to its own range, but do
not require an exact number of values, there is even a PAR routine to do
that.  PAR\_GRMVx returns up to some maximum number of values of values,
each within a defined range.  In the following example an application
needs integer compression factors (COMPRS) along each of NDIM
dimensions.  These must be positive and no greater than the size of the
array along each dimension, and are set by the CMPMIN and CMPMAX arrays.
ACTVAL returns the actual number values obtained from parameter
COMPRESS, and in this example, it is used to set no compression for the
higher dimensions.

\begin{verbatim}
    *  Set the acceptable range of values from no compression to compress
    *  to a single element in a dimension.
          DO I = 1, NDIM
             CMPMIN( I ) = 1
             CMPMAX( I ) = DIMS( I )
          END DO

    *  Get the compression factors.
          CALL PAR_GRMVI( 'COMPRESS', NDIM, CMPMIN, CMPMAX, COMPRS, ACTVAL,
         :                STATUS )

    *  Should less values be supplied than the number of dimensions, do
    *  not compress the higher dimensions.
          IF ( ACTVAL .LT. NDIM ) THEN
             DO I = ACTVAL + 1, NDIM
                COMPRS( I ) = 1
             END DO
          END IF
\end{verbatim}

\subsection{\xlabel{parity_constraint}Parity Constraint}

There are occasions when you need an odd or even integer.  For example,
the size of a smoothing kernel must be odd.  PAR provides two routines
with these functions.  Thus,

\begin{verbatim}
          CALL PAR_GODD( 'BOX', 5, 3, 11, .FALSE., BOXSIZ, STATUS )
\end{verbatim}

obtains an odd integer between 3 and 11 from parameter BOX, and stores
it in variable BOXSIZ.  The dynamic default is 5.  Similarly,

\begin{verbatim}
          CALL PAR_GEVEN( 'OFFSET', 2, -2, 10, .FALSE., OFFS, STATUS )
\end{verbatim}
gets a even value in the range $-$2 to 10 from parameter OFFSET, with a
dynamic default of 2.  Zero counts as an even number, and so it would be
acceptable here.

\subsection{\xlabel{menu_constraint}Menu Constraint}

A common requirement is to select an option from a menu.  PAR\_CHOIC
offers this functionality.   You provide a list of options separated by
commas.  When the user selects a choice not in this menu, an error
report appears that lists the available options, and the user is
prompted. \footnote{Remember that in the ADAM implementation the PAR\_GET0C
call made by PAR\_CHOIC will first test the obtained value against the
range or in fields in the interface file.  If the supplied value is
unacceptable, the user will be prompted by PAR\_GET0C.  Only once these
constraints are passed will the value be tested against the menu.}
PAR\_CHOIC permits the user of your application to use an unambiguous
abbreviation.  However, the unshortened value is returned.  The value is
also in uppercase, though the list of options need not be so.

\begin{verbatim}
          CHARACTER * 10 FUNCT, OPTDEF
          CHARACTER * 72 OPLIST

              :       :       :

          OPLIST = 'Exit,Device,Histogram,List,Peep,'/
         :         /'Region,Save,Slice,Statistics,Value'
          OPTDEF = 'Region'
          CALL PAR_CHOIC( 'OPTION', OPTDEF, OPLIST, .FALSE., FUNCT, STATUS )
\end{verbatim}

So in the above example there are ten options available for parameter
OPTION.  The dynamic default is 'Region'.  If you assign the second
argument with a value not in the main list, such as a blank string, it
instructs PAR\_CHOIC not to set a dynamic default. Note that
PAR\_CHOIC only returns character values.

This is what the user might see for the above example.
\begin{verbatim}
    OPTION - Inspection option /'Region'/ > View
    !! The choice View is not in the menu.  The options are
    !     Exit,Device,Histogram,List,Peep,Region,Save,Slice,Statistics,Value.
    !! Invalid selection for parameter OPTION.
    OPTION - Inspection option /'Region'/ > S
    !! The choice S is ambiguous.  The options are
    !     Exit,Device,Histogram,List,Peep,Region,Save,Slice,Statistics,Value.
    !! Invalid selection for parameter OPTION.
    OPTION - Inspection option /'Region'/ > lust
    Selected the nearest match "LIST" for parameter OPTION.
\end{verbatim}

The first value is unacceptable as it is not in the menu.  The second is
ambiguous because there are several options beginning with an ess. Had
the user entered {\tt sl} say, the abbreviation would have selected a
value of 'SLICE'. The final value appears not to be in the list of
choices, but PAR\_CHOIC allows the user one typing mistake, and so the
user actually selects option 'LIST'. In this case, a warning message is output
unless the
\xref{MSG filtering level}{sun104}{conditional_message_reporting}
\latex{ (see SUN/104)} is set to `quiet'.

There is a similar routine \dash\ PAR\_CHOIV \dash\ to get a vector of character
values from a menu.

\subsection{\xlabel{combined_constraints}Combined Constraints}

Should you wish to combine the constraints of a numeric value within a
range, and a list of options, there are PAR routines to do it for you.
Here is an illustration to show how this might be profitably used.  The
application from which the following extract is taken wants to replace
certain array values with a constant; this can either be a numeric value
or set to the bad-pixel value.

\begin{verbatim}
          CHARACTER * ( VAL__SZR ) CNEWLO, THLDEF

              :       :       :

    *  Get the replacement value.
          THLDEF = '0.0'
          CALL PAR_MIX0R( 'NEWLO', THLDEF, VAL__MINR, VAL__MAXR,
         :                'Bad', .FALSE., CNEWLO, STATUS )

          IF ( STATUS .EQ. SAI__OK ) THEN

    *  It may be the bad-pixel value.
             IF ( CNEWLO .EQ. 'BAD' ) THEN
                NEWLO = VAL__BADR
             ELSE

    *  Convert the output numeric string to its numeric value.
                CALL CHR_CTOR( CNEWLO, NEWLO, STATUS )
             END IF
          END IF
\end{verbatim}

This obtains from parameter NEWLO a value which is either 'BAD' or a
real value.  The value is returned in variable CNEWLO.  VAL\_\_MINR and
VAL\_\_MAXR are the minimum and maximum values, and VAL\_\_BADR is the
bad value, for the real data type, and VAL\_\_SZR is the maximum number
of characters needed to store a real value; all symbolic constants are
defined in the PRM\_PAR include file (
\xref{SUN/39}{sun39}{}).
 The dynamic default is
'0.0'. Remember that the dynamic default and returned value are
strings.  Therefore, you must first test the returned value for being any
of the items on the menu, before converting it to a number.  Since the
returned value may be undefined following an error, it is prudent to
check the status before using the value.

Should the user give an unacceptable value, PAR\_MIX0x informs the user
of the error and lists the available options, and then invites the user
to supply another value.

Here is a more subtle example.  This only has numbers in the menu, but
it is advantageous when only certain numeric values are acceptable.
\begin{verbatim}
    *  Get the plate number.
          CALL PAR_MIX0I( 'PLATE', ' ', 101, 1500, '5,11,23,47,49',
         :                .FALSE., CPLATE, STATUS )
          IF ( STATUS .EQ. SAI__OK )
         :  CALL CHR_CTOI( CPLATE, PLATNO, STATUS )
\end{verbatim}

Here PAR\_MIX0I obtains an `integer' that is either 5, 11, 23, 47, 49,
or in the range 101\dash1500, and returns it in the {\em character\/}
variable CPLATE.  CHR\_CTOI converts the string into a true integer
value, PLATNO.  Since the second argument of PAR\_MIX0I is a blank
string, there is no dynamic default.  To produce a suggested default,
the second argument would have to be one of the menu options, or satisfy
the range constraint (when converted to an integer). If we had swapped
the range limits, PAR\_MIX0I would allow all values not in the range
102\dash1499.

\subsection{\xlabel{logical_value}Logical Value}

PAR\_GTD0L obtains a scalar logical value, with a dynamic default
defined, and has the capability of handling a null status.  Thus the
following obtains a value from the parameter SWITCH and stores it in the
variable called POWER.  The dynamic default is .TRUE..

\begin{verbatim}
           NULL = .FALSE.
           CALL PAR_GTD0L( 'SWITCH', .TRUE., NULL, POWER, STATUS )
\end{verbatim}

The third argument is the same as we met in Section~\ref{se:range}. If
we reverse the polarity of NULL, PAR\_GTD0L will assign the dynamic
default to POWER whenever the parameter is in the null state, and
returns with STATUS set to SAI\_\_OK.  To reiterate NULL=.TRUE. should
only be used when the dynamic default will {\em always\/} give
reasonable behaviour in the application.  This is highly likely for a
logical value.

\section{\xlabel{other_types_of_parameters}Other Types of Parameters}

Besides the numeric, character, and logical parameters we have seen so
far, there is another class of parameter to which some of the PAR
routines may be applied.  In {\footnotesize ADAM}, these are parameters
that specify the name of an HDS or Fortran data file, a magnetic-tape or
graphics device.

The values are obtained from such parameters by calling special routines
within the particular subroutine library that connect to the
{\footnotesize ADAM} parameter system.  The routines are usually called
{\it fac}\_ASSOC, where {\it fac\/} is the facility name of the package,
{\it e.g.}\ NDF\_ASSOC accesses an NDF whose name is found via a
parameter, and SGS\_ASSOC obtains the name of an SGS graphics device.
There are also {\it fac}\_CANCL routines that free other resources
opened by the corresponding {\it fac}\_ASSOC routine, and therefore you
should not use PAR\_CANCL to cancel parameters obtained by using an {\it
fac}\_ASSOC routine. However, it is possible and quite legitimate to
call PAR\_PROMT to set a prompt string for this class of parameter, or
to call PAR\_DEF0C to set the dynamic default, or to find the state of a
parameter using PAR\_STATE.

\section{\xlabel{compilation_and_linking}Compilation and Linking}

\subsection{\xlabel{unix}UNIX}

To compile and link a UNIX {\footnotesize ADAM} application called prog
with the PAR library, you should use the following command.

\begin{verbatim}
    alink prog.f
\end{verbatim}

\subsection{\xlabel{vms}VMS}

PAR is presently only available to {\footnotesize ADAM} programmers.
Only the normal ADAM startup commands need be issued to access routines
in the PAR library.  These are

\begin{verbatim}
     $ ADAMSTART
     $ ADAMDEV
     $ PAR_DEV
\end{verbatim}

and they will ensure that all necessary definitions are made for
compilation and linking.

PAR is distributed as a shareable image, and it is included in the
{\footnotesize ADAM} ALINK command.  So to link an {\footnotesize ADAM}
A-task called PROG you just enter

\begin{verbatim}
     $ FORTRAN PROG
     $ ALINK PROG
\end{verbatim}

\section{\xlabel{acknowledgements}Acknowledgements}

Dennis Kelly and Alan Chipperfield wrote the original simple PAR library
and the {\footnotesize ADAM} parameter system.  Some of the extended
routines are partly based on AIF routines written by Dave Baines and
Steven Beard at ROE.  Thanks to Alan Chipperfield for explaining the
workings of the {\footnotesize ADAM} parameter system to the author, and
for a critical reading of this document.  Rodney Warren-Smith also
suggested many improvements.

\newpage
\appendix
\small

\section{\xlabel{alphabetical_list_of_routines}Alphabetical List of Routines}
The argument lists of the following routines, together with on-line
help information, are available within the Starlink language-sensitive
editor STARLSE (
\xref{SUN/105}{sun105}{}).
\begin{description}
\item [\htmlref{PAR\_CANCL}{PAR_CANCL}
( PARAM, STATUS )] \listline
\textit{Cancel a parameter}
\item [\htmlref{PAR\_CHOIC}{PAR_CHOIC}
( PARAM, DEFAUL, OPTS, NULL, VALUE, STATUS )] \listline
\textit{Obtain from a parameter a character value selected from a menu
            of options}
\item [\htmlref{PAR\_CHOIV}{PAR_CHOIV}
( PARAM, MAXVAL, OPTS, VALUES, ACTVAL, STATUS )] \listline
\textit{Obtain from a parameter a list of character values selected from
            a menu of options}
\item [\htmlref{PAR\_DEF0x}{PAR_DEF0x}
( PARAM, VALUE, STATUS )] \listline
\textit{Set a scalar dynamic default parameter value}
\item [\htmlref{PAR\_DEF1x}{PAR_DEF1x}
( PARAM, NVAL VALUES, STATUS )] \listline
\textit{Set a vector of values as the dynamic default for a parameter}
\item [\htmlref{PAR\_DEFNx}{PAR_DEFNx}
( PARAM, NDIM, MAXD, VALUES, ACTD, STATUS )] \listline
\textit{Set an array of values as the dynamic default for a parameter}
\item [\htmlref{PAR\_EXACx}{PAR_EXACx}
( PARAM, NVALS, VALUES, STATUS )] \listline
\textit{Obtain an exact number of values from a parameter}
\item [\htmlref{PAR\_GDR0x}{PAR_GDR0x}
( PARAM, DEFAUL, VMIN, VMAX, NULL, VALUE, STATUS )] \listline
\textit{Obtain a scalar value within a given range from a parameter}
\item [\htmlref{PAR\_GDR1x}{PAR_GDR1x}
( PARAM, NVALS, DEFAUL, VMIN, VMAX, NULL, VALUES, \listova{STATUS )}] \listline
\listovb{STATUS )}
\textit{Obtain an exact number of values within a given range from a parameter}
\item [\htmlref{PAR\_GDRVx}{PAR_GDRVx}
( PARAM, MAXVAL, VMIN, VMAX, VALUES, ACTVAL, STATUS )] \listline
\textit{Obtain a vector of values within a given range from a parameter}
\item [\htmlref{PAR\_GET0x}{PAR_GET0x}
( PARAM, VALUE, STATUS )] \listline
\textit{Obtain a scalar value from a parameter}
\item [\htmlref{PAR\_GET1x}{PAR_GET1x}
( PARAM, MAXVAL, VALUES, ACTVAL, STATUS )] \listline
\textit{Obtain a vector of values from a parameter}
\item [\htmlref{PAR\_GETNx}{PAR_GETNx}
( PARAM, NDIM, MAXD, VALUES, ACTD, STATUS )] \listline
\textit{Obtain an array parameter value}
\item [\htmlref{PAR\_GETVx}{PAR_GETVx}
( PARAM, MAXVAL, VALUES, ACTVAL, STATUS )] \listline
\textit{Obtain a vector of values from a parameter regardless of its shape}
\item [\htmlref{PAR\_GEVEN}{PAR_GEVEN}
( PARAM, DEFAUL, VMIN, VMAX, NULL, VALUE, STATUS )] \listline
\textit{Obtain an even integer value from a parameter}
\item [\htmlref{PAR\_GODD}{PAR_GODD}
( PARAM, DEFAUL, VMIN, VMAX, NULL, VALUE, STATUS )] \listline
\textit{Obtain an odd integer value from a parameter}
\item [\htmlref{PAR\_GRM1x}{PAR_GRM1x}
( PARAM, NVALS, DEFAUL, VMIN, VMAX, NULL, VALUES, \listova{STATUS )}] \listline
\listovb{STATUS )}
\textit{Obtain from a parameter an exact number of values each within a
            given range}
\item [\htmlref{PAR\_GRMVx}{PAR_GRMVx}
( PARAM, MAXVAL, VMIN, VMAX, VALUES, ACTVAL, STATUS )] \listline
\textit{Obtain from a parameter a vector of values each within a given
            range}
\item [\htmlref{PAR\_GTD0L}{PAR_GTD0L}
( PARAM, DEFAUL, NULL, VALUE, STATUS )] \listline
\textit{Obtain a logical value from a parameter with a dynamic default}
\item [\htmlref{PAR\_MAXx}{PAR_MAXx}
( PARAM, VALUE, STATUS )] \listline
\textit{Set a maximum value for a parameter}
\item [\htmlref{PAR\_MINx}{PAR_MINx}
( PARAM, VALUE, STATUS )] \listline
\textit{Set a minimum value for a parameter}
\item [\htmlref{PAR\_MIX0x}{PAR_MIX0x}
( PARAM, DEFAUL, VMIN, VMAX, OPTS, NULL, VALUE, STATUS )] \listline
\textit{Obtain from a parameter a character value either from a menu of
            options or as a numeric value within a given range}
\item [\htmlref{PAR\_MIXVx}{PAR_MIXVx}
( PARAM, MAXVAL, VMIN, VMAX, OPTS, VALUES, ACTVAL, \listova{STATUS )}] \listline
\listovb{STATUS )}
\textit{Obtain from a parameter, character values either selected from a
            menu of options or as numeric values within a given range}
\item [\htmlref{PAR\_PROMT}{PAR_PROMT}
( PARAM, PROMPT, STATUS )] \listline
\textit{Set a new prompt string for a parameter}
\item [\htmlref{PAR\_PUT0x}{PAR_PUT0x}
( PARAM, VALUE, STATUS )] \listline
\textit{Put a scalar value into a parameter}
\item [\htmlref{PAR\_PUT1x}{PAR_PUT1x}
( PARAM, NVAL, VALUES, STATUS )] \listline
\textit{Put a vector of values into a parameter}
\item [\htmlref{PAR\_PUTNx}{PAR_PUTNx}
( PARAM, NDIM, MAXD, VALUES, ACTD, STATUS )] \listline
\textit{Put an array of values into a parameter}
\item [\htmlref{PAR\_PUTVx}{PAR_PUTVx}
( PARAM, NVAL, VALUES, STATUS )] \listline
\textit{Put an array of values into a parameter as if the parameter were a vector}
\item [\htmlref{PAR\_STATE}{PAR_STATE}
( PARAM, STATE, STATUS )] \listline
\textit{Inquire the state of a parameter}
\item [\htmlref{PAR\_UNSET}{PAR_UNSET}
( PARAM, WHICH, STATUS )] \listline
\textit{Cancel various parameter control values.}
\end{description}

\section{\xlabel{classified_list_of_routines}Classified List of Routines}
\subsection{\xlabel{getting_values_from_a_parameter_without_constraints}Getting Values from a Parameter without Constraints}
\begin{description}
\item [\htmlref{PAR\_GET0x}{PAR_GET0x}
( PARAM, VALUE, STATUS )] \listline
\textit{Obtain a scalar value from a parameter}
\item [\htmlref{PAR\_GET1x}{PAR_GET1x}
( PARAM, MAXVAL, VALUES, ACTVAL, STATUS )] \listline
\textit{Obtain a vector of values from a parameter}
\item [\htmlref{PAR\_GETNx}{PAR_GETNx}
( PARAM, NDIM, MAXD, VALUES, ACTD, STATUS )] \listline
\textit{Obtain an array parameter value}
\item [\htmlref{PAR\_GETVx}{PAR_GETVx}
( PARAM, MAXVAL, VALUES, ACTVAL, STATUS )] \listline
\textit{Obtain a vector of values from a parameter regardless of its shape}
\end{description}

\subsection{\xlabel{getting_parameter_values_from_a_menu}Getting Parameter Values from a Menu}

\begin{description}
\item [\htmlref{PAR\_CHOIC}{PAR_CHOIC}
( PARAM, DEFAUL, OPTS, NULL, VALUE, STATUS )] \listline
\textit{Obtain from a parameter a character value selected from a menu
            of options}
\item [\htmlref{PAR\_CHOIV}{PAR_CHOIV}
( PARAM, MAXVAL, OPTS, VALUES, ACTVAL, STATUS )] \listline
\textit{Obtain from a parameter a list of character values selected from
            a menu of options}
\item [\htmlref{PAR\_MIX0x}{PAR_MIX0x}
( PARAM, DEFAUL, VMIN, VMAX, OPTS, NULL, VALUE, STATUS )] \listline
\textit{Obtain from a parameter a character value either from a menu of
             options or as a numeric value within a given range}
\item [\htmlref{PAR\_MIXVx}{PAR_MIXVx}
( PARAM, MAXVAL, VMIN, VMAX, OPTS, VALUES, ACTVAL, \listova{STATUS )}] \listline
\listovb{STATUS )}
\textit{Obtain from a parameter, character values either selected from a
             menu of options or as numeric values within a given range}
\end{description}

\subsection{\xlabel{getting_values_within_ranges}Getting Values within Ranges}

\begin{description}
\item [\htmlref{PAR\_GDR0x}{PAR_GDR0x}
( PARAM, DEFAUL, VMIN, VMAX, NULL, VALUE, STATUS )] \listline
\textit{Obtain a scalar value within a given range from a parameter}
\item [\htmlref{PAR\_GDR1x}{PAR_GDR1x}
( PARAM, NVALS, DEFAUL, VMIN, VMAX, NULL, VALUES, \listova{STATUS )}] \listline
\listovb{STATUS )}
\textit{Obtain an exact number of values within a given range from a parameter}
\item [\htmlref{PAR\_GDRVx}{PAR_GDRVx}
( PARAM, MAXVAL, VMIN, VMAX, VALUES, ACTVAL, STATUS )] \listline
\textit{Obtain a vector of values within a given range from a parameter}
\item [\htmlref{PAR\_GEVEN}{PAR_GEVEN}
( PARAM, DEFAUL, VMIN, VMAX, NULL, VALUE, STATUS )] \listline
\textit{Obtain an even integer value from a parameter}
\item [\htmlref{PAR\_GODD}{PAR_GODD}
( PARAM, DEFAUL, VMIN, VMAX, NULL, VALUE, STATUS )] \listline
\textit{Obtain an odd integer value from a parameter}
\item [\htmlref{PAR\_GRM1x}{PAR_GRM1x}
( PARAM, NVALS, DEFAUL, VMIN, VMAX, NULL, VALUES, \listova{STATUS )}] \listline
\listovb{STATUS )}
\textit{Obtain from a parameter an exact number of values each within a
             given range}
\item [\htmlref{PAR\_GRMVx}{PAR_GRMVx}
( PARAM, MAXVAL, VMIN, VMAX, VALUES, ACTVAL, STATUS )] \listline
\textit{Obtain from a parameter a vector of values each within a given
             range}
\item [\htmlref{PAR\_MIX0x}{PAR_MIX0x}
( PARAM, DEFAUL, VMIN, VMAX, OPTS, NULL, VALUE, STATUS )] \listline
\textit{Obtain from a parameter a character value either from a menu of
             options or as a numeric value within a given range}
\item [\htmlref{PAR\_MIXVx}{PAR_MIXVx}
( PARAM, MAXVAL, VMIN, VMAX, OPTS, VALUES, ACTVAL, \listova{STATUS )}] \listline
\listovb{STATUS )}
\textit{Obtain from a parameter, character values either selected from a
             menu of options or as numeric values within a given range}
\end{description}


\subsection{\xlabel{putting_values_into_parameters}Putting Values into Parameters}

\begin{description}
\item [\htmlref{PAR\_PUT0x}{PAR_PUT0x}
( PARAM, VALUE, STATUS )] \listline
\textit{Put a scalar value into a parameter}
\item [\htmlref{PAR\_PUT1x}{PAR_PUT1x}
( PARAM, NVAL, VALUES, STATUS )] \listline
\textit{Put a vector of values into a parameter}
\item [\htmlref{PAR\_PUTNx}{PAR_PUTNx}
( PARAM, NDIM, MAXD, VALUES, ACTD, STATUS )] \listline
\textit{Put an array of values into a parameter}
\item [\htmlref{PAR\_PUTVx}{PAR_PUTVx}
( PARAM, NVAL, VALUES, STATUS )] \listline
\textit{Put an array of values into a parameter as if the parameter were a
            vector}
\end{description}

\subsection{\xlabel{miscellaneous}Miscellaneous}

\begin{description}
\item [\htmlref{PAR\_CANCL}{PAR_CANCL}
( PARAM, STATUS )] \listline
\textit{Cancel a parameter}
\item [\htmlref{PAR\_GTD0L}{PAR_GTD0L}
( PARAM, DEFAUL, NULL, VALUE, STATUS )] \listline
\textit{Obtain a logical value from a parameter with a dynamic default}
\item [\htmlref{PAR\_MAXx}{PAR_MAXx}
( PARAM, VALUE, STATUS )] \listline
\textit{Set a maximum value for a parameter}
\item [\htmlref{PAR\_MINx}{PAR_MINx}
( PARAM, VALUE, STATUS )] \listline
\textit{Set a minimum value for a parameter}
\item [\htmlref{PAR\_PROMT}{PAR_PROMT}
( PARAM, PROMPT, STATUS )] \listline
\textit{Set a new prompt string for a parameter}
\item [\htmlref{PAR\_UNSET}{PAR_UNSET}
( PARAM, WHICH, STATUS )] \listline
\textit{Cancel various parameter control values.}
\end{description}

\subsection{\xlabel{inquiry_routines}Inquiry Routines}

\begin{description}
\item [\htmlref{PAR\_STATE}{PAR_STATE}
( PARAM, STATE, STATUS )] \listline
\textit{Inquire the state of a parameter}
\end{description}

\section{\xlabel{error_codes}Error Codes}

If a PAR routine completes its task successfully, it will return with
status SAI\_\_OK.  If it detects an error condition, PAR defines the
following error codes.

\begin{center}
\begin{tabular}{|ll|}
\hline
Symbolic name & Meaning \\ \hline
PAR\_\_ACINV & Invalid access mode for the parameter \\
PAR\_\_AMBIG & Ambiguous option list \\
PAR\_\_ERROR & Some undefined error \\
PAR\_\_NULL & Null parameter value \\ \hline
\end{tabular}
\end{center}

\newpage
\section{\xlabel{correspondence_between_par_and_subpar_states}Correspondence between PAR and SUBPAR States}

PAR and SUBPAR states do not have a one-to-one correspondence.  Each
SUBPAR state is equated to the appropriate generic PAR states.  The
former version of PAR\_STATE routine returned SUBPAR states, rather than
the PAR states as happens now.  Existing applications which compare the
returned state with any of SUBPAR\_\_GROUND, SUBPAR\_\_ACTIVE,
SUBPAR\_\_CANCEL, and SUBPAR\_\_NULL (defined in SUBPAR\_PAR) will
continue to function, since the PAR equivalents have the same numerical
values.

\begin{table}[h]
\caption{The PAR states.}
\begin{center}
\begin{tabular}{|l|l|l|}
\hline
PAR Symbolic name & Meaning & SUBPAR states \\ \hline
PAR\_\_GROUND & The parameter has never had a value.    & SUBPAR\_\_GROUND \\
              &                                         & SUBPAR\_\_EOL \\
              &                                         & SUBPAR\_\_RESET \\
              &                                         & SUBPAR\_\_ACCEPT \\
              &                                         & SUBPAR\_\_RESACC \\
              &                                         & SUBPAR\_\_FPROMPT \\
              &                                         & SUBPAR\_\_RESPROM \\
              &                                         & SUBPAR\_\_ACCPR \\
              &                                         & SUBPAR\_\_RESACCPR \\
PAR\_\_ACTIVE & The parameter has a value.              & SUBPAR\_\_ACTIVE \\
              &                                         & SUBPAR\_\_MAX \\
              &                                         & SUBPAR\_\_MIN \\
PAR\_\_CANCEL & The parameter value has been cancelled. & SUBPAR\_\_CANCEL \\
PAR\_\_NULLST & The parameter is null.                  & SUBPAR\_\_NULL \\ \hline
\end{tabular}
\end{center}
\end{table}

\section{\xlabel{the_c_interface}The C Interface}
A preliminary C interface is provided for trial purposes. It may be subject to
change in the light of experience.

The interface obeys the rules defined in (PRO)LUN/10.

Briefly, the function name is generated from the Fortran subroutine name by
forcing the name to lower case apart from the first character following any
underscores, which is forced to upper case. Underscores are then removed.

For example: The C interface function for `PAR\_GET0R' is `parGet0r'.

Arguments are provided in the same order as for the Fortran routine with the
exception that CHARACTER arrays and returned CHARACTER strings
have an additional argument of type \texttt{int} (passed by value) immediately
following them to specify a maximum length for the output string(s) including
the terminating null for which space must be allowed.

There is a fixed relationship between the type of the Fortran
argument and the type of the argument supplied to the C function \dash\ it is
as follows:
\begin{center}
\begin{tabular}{|l|l|}
\hline
Fortran type & C type \\
\hline
INTEGER & int \\
REAL & float \\
REAL*8 & double \\
DOUBLE PRECISION & double \\
LOGICAL & int \\
CHARACTER & char \\
FUNCTION & \textit{type} (*\textit{name})() \\
SUBROUTINE & void (*\textit{name})() \\
\hline
\end{tabular}
\end{center}

Apart from any argument named `status', given-only scalar arguments (not
including character strings) are passed by value.
All others are passed by pointer.

Arrays must be passed with the elements stored in the order required by Fortran.

All necessary constants and function prototypes can be defined by:
\begin{quote} \begin{verbatim}
#include "par.h"
\end{verbatim} \end{quote}

\section{\xlabel{c_interface_function_prototypes}The C Interface Function Prototypes}
Where \textit{T} is one of d, i, l, r and \textit{TYPE} is the corresponding C
type, the function prototypes are:
\begin{flushleft}
\begin{description}
\item[void parCancl]
\textbf{( const~char~\texttt{*}param, int~\texttt{*}status );}\\
\textit{Cancel a parameter}
\item[void parChoic]
\textbf{( const~char~\texttt{*}param, const~char~\texttt{*}defaul,
               const~char~\texttt{*}opts, int~null, char~\texttt{*}value,
               int~value\_length, int~\texttt{*}status );} \\
\textit{Obtain from a parameter a character value selected from a menu
            of options}
\item[void parChoiv]
\textbf{( const~char~\texttt{*}param, int~maxval, const~char~\texttt{*}opts,
               char~\texttt{*}const~\texttt{*}values, int~values\_length,
               int~\texttt{*}actval, int~\texttt{*}status );} \\
\textit{Obtain from a parameter a list of character values selected from
            a menu of options}
\item[void parDef0c]
\textbf{( const~char~\texttt{*}param, const~char~\texttt{*}value,
          int~\texttt{*}status );}\\
\textit{Set a scalar character dynamic default parameter value}
\item[void parDef0\textit{T}]
\textbf{( const~char~\texttt{*}param, \textit{TYPE}~value,
          int~\texttt{*}status );}\\
\textit{Set a scalar dynamic default parameter value}
\item[void parDef1c]
\textbf{( const~char~\texttt{*}param, int~nval,
          char~\texttt{*}const~\texttt{*}values, int~values\_length,
          int~\texttt{*}status );} \\
\textit{Set a vector of character values as the dynamic default for a parameter}
\item[void parDef1\textit{T}]
\textbf{( const~char~\texttt{*}param, int~nval,
          const~\textit{TYPE}~\texttt{*}values, int~\texttt{*}status );}\\
\textit{Set a vector of values as the dynamic default for a parameter}
\item[void parDefnc]
\textbf{( const~char~\texttt{*}param, int~ndim, const~int~\texttt{*}maxd,
               char~\texttt{*}const~\texttt{*}values, int~values\_length, const~int~\texttt{*}actd,
               int~\texttt{*}status );} \\
\textit{Set an array of character values as the dynamic default for a parameter}
\item[void parDefn\textit{T}]
\textbf{( const~char~\texttt{*}param, int~ndim, const~int~\texttt{*}maxd,
               const~\textit{TYPE}~\texttt{*}values, const~int~\texttt{*}actd,
               int~\texttt{*}status );} \\
\textit{Set an array of values as the dynamic default for a parameter}
\item[void parExacc]
\textbf{( const~char~\texttt{*}param, int~nvals, char~\texttt{*}const~\texttt{*}values,
               int~values\_length, int~\texttt{*}status );} \\
\textit{Obtain an exact number of character values from a parameter}
\item[void parExac\textit{T}]
\textbf{( const~char~\texttt{*}param, int~nvals, \textit{TYPE}~\texttt{*}values, int~\texttt{*}status );}\\
\textit{Obtain an exact number of values from a parameter}
\item[void parGdr0\textit{T}]
\textbf{( const~char~\texttt{*}param, \textit{TYPE}~defaul, \textit{TYPE}~vmin,
               \textit{TYPE}~vmax, int~null, \textit{TYPE}~\texttt{*}value,
               int~\texttt{*}status );} \\
\textit{Obtain a scalar value within a given range from a parameter}
\item[void parGdr1\textit{T}]
\textbf{( const~char~\texttt{*}param, int~nvals,
               \textit{TYPE}~\texttt{*}defaul, \textit{TYPE}~vmin,
               \textit{TYPE}~vmax, int~null, \textit{TYPE}~\texttt{*}values,
               int~\texttt{*}status );} \\
\textit{Obtain an exact number of values within a given range from a parameter}
\item[void parGdrv\textit{T}]
\textbf{( const~char~\texttt{*}param, int~maxval, \textit{TYPE}~vmin,
               \textit{TYPE}~vmax, \textit{TYPE}~\texttt{*}values,
               int~\texttt{*}actval, int~\texttt{*}status );} \\
\textit{Obtain a vector of values within a given range from a parameter}
\item[void parGet0c]
\textbf{( const~char~\texttt{*}param, char~\texttt{*}value,
          int~value\_length, int~\texttt{*}status );}\\
\textit{Obtain a scalar character value from a parameter}
\item[void parGet0\textit{T}]
\textbf{( const~char~\texttt{*}param,
          \textit{TYPE}~\texttt{*}value, int~\texttt{*}status );}\\
\textit{Obtain a scalar value from a parameter}
\item[void parGet1c]
\textbf{( const~char~\texttt{*}param, int~maxval,
               char~\texttt{*}const~\texttt{*}values, int~values\_length,
               int~\texttt{*}actval, int~\texttt{*}status );} \\
\textit{Obtain a vector of character values from a parameter}
\item[void parGet1\textit{T}]
\textbf{( const~char~\texttt{*}param, int~maxval,
               \textit{TYPE}~\texttt{*}values, int~\texttt{*}actval,
               int~\texttt{*}status );} \\
\textit{Obtain a vector of values from a parameter}
\item[void parGetnc]
\textbf{( const~char~\texttt{*}param, int~ndim, const~int~\texttt{*}maxd,
               char~\texttt{*}const~\texttt{*}values, int~values\_length,
               int~\texttt{*}actd, int~\texttt{*}status );} \\
\textit{Obtain a character array parameter value}
\item[void parGetn\textit{T}]
\textbf{( const~char~\texttt{*}param, int~ndim, const~int~\texttt{*}maxd,
               \textit{TYPE}~\texttt{*}values, int~\texttt{*}actd,
               int~\texttt{*}status );} \\
\textit{Obtain an array parameter value}
\item[void parGetvc]
\textbf{( const~char~\texttt{*}param, int~maxval,
               char~\texttt{*}const~\texttt{*}values, int~values\_length,
               int~\texttt{*}actval, int~\texttt{*}status );} \\
\textit{Obtain a vector of character values from a parameter regardless of its
               shape}
\item[void parGetv\textit{T}]
\textbf{( const~char~\texttt{*}param, int~maxval,
               \textit{TYPE}~\texttt{*}values, int~\texttt{*}actval,
               int~\texttt{*}status );} \\
\textit{Obtain a vector of values from a parameter regardless of its shape}
\item[void parGeven]
\textbf{( const~char~\texttt{*}param, int~defaul, int~vmin, int~vmax,
               int~null, int~\texttt{*}value, int~\texttt{*}status );} \\
\textit{Obtain an even integer value from a parameter}
\item[void parGodd]
\textbf{( const~char~\texttt{*}param, int~defaul, int~vmin, int~vmax,
              int~null, int~\texttt{*}value, int~\texttt{*}status );} \\
\textit{Obtain an odd integer value from a parameter}
\item[void parGrm1\textit{T}]
\textbf{( const~char~\texttt{*}param, int~nvals,
               const~\textit{TYPE}~\texttt{*}defaul,
               const~\textit{TYPE}~\texttt{*}vmin,
               const~\textit{TYPE}~\texttt{*}vmax, int~null,
               \textit{TYPE}~\texttt{*}values, int~\texttt{*}status );} \\
\textit{Obtain from a parameter an exact number of values each within a
            given range}
\item[void parGrmv\textit{T}]
\textbf{( const~char~\texttt{*}param, int~maxval,
               const~\textit{TYPE}~\texttt{*}vmin,
               const~\textit{TYPE}~\texttt{*}vmax,
               const~\textit{TYPE}~\texttt{*}values, int~\texttt{*}actval,
               int~\texttt{*}status );} \\
\textit{Obtain from a parameter a vector of values each within a given
            range}
\item[void parGtd0l]
\textbf{( const~char~\texttt{*}param, int~defaul, int~null,
          int~\texttt{*}value, int~\texttt{*}status );}\\
\textit{Obtain a logical value from a parameter with a dynamic default}
\item[void parMaxc]
\textbf{( const~char~\texttt{*}param, const~char~\texttt{*}value,
          int~\texttt{*}status );}\\
\textit{Set a maximum character value for a parameter}
\item[void parMax\textit{T}]
\textbf{( const~char~\texttt{*}param, \textit{TYPE}~value,
          int~\texttt{*}status );}\\
\textit{Set a maximum value for a parameter}
\item[void parMinc]
\textbf{( const~char~\texttt{*}param, const~char~\texttt{*}value,
          int~\texttt{*}status );}\\
\textit{Set a minimum character value for a parameter}
\item[void parMin\textit{T}]
\textbf{( const~char~\texttt{*}param, \textit{TYPE}~value,
          int~\texttt{*}status );}\\
\textit{Set a minimum value for a parameter}
\item[void parMix0\textit{T}]
\textbf{( const~char~\texttt{*}param, const~char~\texttt{*}defaul,
               \textit{TYPE}~vmin, \textit{TYPE}~vmax,
               const~char~\texttt{*}opts, int~null, char~\texttt{*}value,
               int~value\_length, int~\texttt{*}status );} \\
\textit{Obtain from a parameter a character value either from a menu of
            options or as a numeric value within a given range}
\item[void parMixv\textit{T}]
\textbf{( const~char~\texttt{*}param, int~maxval, \textit{TYPE}~vmin,
               \textit{TYPE}~vmax, const~char~\texttt{*}opts,
               char~\texttt{*}const~\texttt{*}values, int~values\_length,
               int~\texttt{*}actval, int~\texttt{*}status );} \\
\textit{Obtain from a parameter, character values either selected from a
            menu of options or as numeric values within a given range}
\item[void parPromt]
\textbf{( const~char~\texttt{*}param, const~char~\texttt{*}prompt,
          int~\texttt{*}status );}\\
\textit{Set a new prompt string for a parameter}
\item[void parPut0c]
\textbf{( const~char~\texttt{*}param, const~char~\texttt{*}value,
          int~\texttt{*}status );}\\
\textit{Put a scalar character value into a parameter}
\item[void parPut0\textit{T}]
\textbf{( const~char~\texttt{*}param, \textit{TYPE}~value,
          int~\texttt{*}status );}\\
\textit{Put a scalar value into a parameter}
\item[void parPut1c]
\textbf{( const~char~\texttt{*}param, int~nval,
               char~\texttt{*}const~\texttt{*}values,  int~values\_length,
               int~\texttt{*}status );} \\
\textit{Put a vector of character values into a parameter}
\item[void parPut1\textit{T}]
\textbf{( const~char~\texttt{*}param, int~nval,
          const~\textit{TYPE}~\texttt{*}values, int~\texttt{*}status );}\\
\textit{Put a vector of values into a parameter}
\item[void parPutnc]
\textbf{( const~char~\texttt{*}param, int~ndim, const~int~\texttt{*}maxd,
               char~\texttt{*}const~\texttt{*}values, int~values\_length,
               const~int~\texttt{*}actd, int~\texttt{*}status );} \\
\textit{Put an array of character values into a parameter}
\item[void parPutn\textit{T}]
\textbf{( const~char~\texttt{*}param, int~ndim, const~int~\texttt{*}maxd,
               const~\textit{TYPE}~\texttt{*}values, const~int~\texttt{*}actd,
               int~\texttt{*}status );} \\
\textit{Put an array of values into a parameter}
\item[void parPutvc]
\textbf{( const~char~\texttt{*}param, int~nval,
               char~\texttt{*}const~\texttt{*}values, int~values\_length,
               int~\texttt{*}status );} \\
\textit{Put an array of character values into a parameter as if the parameter
            were a vector}
\item[void parPutv\textit{T}]
\textbf{( const~char~\texttt{*}param, int~nval,
          const~\textit{TYPE}~\texttt{*}values, int~\texttt{*}status );}\\
\textit{Put an array of values into a parameter as if the parameter were a
            vector}
\item[void parState]
\textbf{( const~char~\texttt{*}param, int~\texttt{*}state,
          int~\texttt{*}status );}\\
\textit{Inquire the state of a parameter}
\item[void parUnset]
\textbf{( const~char~\texttt{*}param, const~char~\texttt{*}which,
          int~\texttt{*}status );}\\
\textit{Cancel various parameter control values.}
\end{description}
\end{flushleft}

\newpage
\section{\xlabel{reference_manual}Reference Manual}
\label{ap:ref}

\sstroutine{
   PAR\_CANCL
}{
   Cancels a parameter
}{
   \sstdescription{
      The named parameter is cancelled. A subsequent attempt to get a value
      for the parameter will result in a new value being obtained by the
      underlying parameter system.
   }
   \sstinvocation{
      CALL PAR\_CANCL( PARAM, STATUS )
   }
   \sstarguments{
      \sstsubsection{
         PARAM = CHARACTER$*$($*$) (Given)
      }{
         The name of the parameter to be cancelled.
      }
      \sstsubsection{
         STATUS = INTEGER (Given and Returned)
      }{
         The global status.
      }
   }
   \sstnotes{
      The routine attempts to execute regardless of the value of
      STATUS.  If the import value is not SAI\_\_OK, then it is left
      unchanged, even if the routine fails to complete.  If the STATUS
      is SAI\_\_OK on entry and the routine fails to complete, STATUS
      will be set to an appropriate error number, and there will one or
      more additional error reports.
   }
}

\sstroutine{
   PAR\_CHOIC
}{
   Obtains from a parameter a character value selected from a menu
   of options
}{
   \sstdescription{
      This routine obtains a scalar character value from a parameter.
      The value must be one of a supplied list of acceptable values,
      and can be an abbreviation provided it is unambiguous.  A dynamic
      default may be suggested.
   }
   \sstinvocation{
      CALL PAR\_CHOIC( PARAM, DEFAUL, OPTS, NULL, VALUE, STATUS )
   }
   \sstarguments{
      \sstsubsection{
         PARAM = CHARACTER $*$ ( $*$ ) (Given)
      }{
         The name of the parameter.
      }
      \sstsubsection{
         DEFAUL = CHARACTER $*$ ( $*$ ) (Given)
      }{
         The suggested default value for the parameter.  No default
         will be suggested when DEFAUL is not one of the options defined
         by OPTS.  A status of PAR\_\_AMBIG is returned if the default is
         ambiguous, {\it i.e.}\ an abbreviation of more than one of the
         options.
      }
      \sstsubsection{
         OPTS = CHARACTER $*$ ( $*$ ) (Given)
      }{
         The list of acceptable options for the value obtained from the
         parameter.  Items should be separated by commas.  The list is
         case-insensitive.
      }
      \sstsubsection{
         NULL = LOGICAL (Given)
      }{
         NULL controls the behaviour of this routine when the parameter
         is in the null state.  If NULL is .FALSE., this routine
         returns with STATUS=PAR\_\_NULL.  If NULL is .TRUE., the
         returned VALUE takes the value of DEFAUL and, if the
\xref{MSG filtering level}{sun104}{conditional_message_reporting}
\latex{ (see SUN/104)} is `verbose',
         a message informs the user of the value used for
         the parameter. The routine then returns with STATUS=SAI\_\_OK.
         This feature is intended for convenient handling of null values.
         NULL should only be set to .TRUE. when the value of DEFAUL will
         always give a reasonable value for the parameter.
      }
      \sstsubsection{
         VALUE = CHARACTER $*$ ( $*$ ) (Returned)
      }{
         The selected option from the list.  It is in uppercase and
         in full, even if an abbreviation has been given for the actual
         parameter.  If STATUS is returned not equal to SAI\_\_OK, VALUE
         takes the value of DEFAUL.
      }
      \sstsubsection{
         STATUS = INTEGER (Given and Returned)
      }{
         The global status.
      }
   }
   \sstnotes{
      The search for a match of the obtained character value with an
      item in the menu adheres to the following rules.
      \begin{itemize}
         \item All comparisons are performed in uppercase.  Leading blanks
         are ignored.
         \item A match is found when the value equals the full name of an
         option.  This enables an option to be the prefix of another
         item without it being regarded as ambiguous.  For example,
         {\tt "}10,100,200{\tt "} would be an acceptable list of options.
         \item If there is no exact match, an abbreviation is acceptable.
         A comparison is made of the value with each option for the
         number of characters in the value.  The option that best fits
         the value is declared a match, subject to two provisos.
         Firstly, there must be no more than one character different
         between the value and the start of the option.  (This allows
         for a mistyped character.)  Secondly, there must be only one
         best-fitting option.  Whenever these criteria are not
         satisfied, the user is told of the error, and is presented
         with the list of options, before being prompted for a new value
         If a nearest match is selected, the user is informed unless the
\xref{MSG filtering level}{sun104}{conditional_message_reporting}
\latex{ (see SUN/104)} is `quiet'.
         \end{itemize}
   }
}

\sstroutine{
   PAR\_CHOIV
}{
   Obtains from a parameter a list of character values selected from
   a menu of options
}{
   \sstdescription{
      This routine obtains a vector of character values from a
      parameter.  Each value must be one of a supplied list of
      acceptable values, and can be an abbreviation provided it is
      unambiguous.
   }
   \sstinvocation{
      CALL PAR\_CHOIV( PARAM, MAXVAL, OPTS, VALUES, ACTVAL, STATUS )
   }
   \sstarguments{
      \sstsubsection{
         PARAM = CHARACTER $*$ ( $*$ ) (Given)
      }{
         The name of the parameter.
      }
      \sstsubsection{
         MAXVAL = INTEGER (Given)
      }{
         The maximum number of values required.  A PAR\_\_ERROR status is
         returned when the number of values requested is less than one.
      }
      \sstsubsection{
         OPTS = CHARACTER $*$ ( $*$ ) (Given)
      }{
         The list of acceptable options for the values obtained from the
         parameter.  Items should be separated by commas.  The list is
         case-insensitive.
      }
      \sstsubsection{
         VALUES( MAXVAL ) = CHARACTER $*$ ( $*$ ) (Returned)
      }{
         The selected options from the list in the order supplied to the
         parameter.  They are in uppercase and in full, even if an
         abbreviation has been given for the actual parameter.
      }
      \sstsubsection{
         ACTVAL = INTEGER (Returned)
      }{
         The actual number of values obtained.
      }
      \sstsubsection{
         STATUS = INTEGER (Given and Returned)
      }{
         The global status.
      }
   }
   \sstnotes{
      The search for a match of each obtained character value with an
      item in the menu adheres to the following rules.
      \begin{itemize}
      \item  All comparisons are performed in uppercase.  Leading blanks
         are ignored.
      \item  A match is found when the value equals the full name of an
         option.  This enables an option to be the prefix of another
         item without it being regarded as ambiguous.  For example,
         {\tt "}10,100,200{\tt "} would be an acceptable list of options.
         \item If there is no exact match, an abbreviation is acceptable.
         A comparison is made of the value with each option for the
         number of characters in the value.  The option that best fits
         the value is declared a match, subject to two provisos.
         Firstly, there must be no more than one character different
         between the value and the start of the option.  (This allows
         for a mistyped character.)  Secondly, there must be only one
         best-fitting option.  Whenever these criteria are not
         satisfied, the user is told of the error, and is presented
         with the list of options, before being prompted for a new value
         If a nearest match is selected, the user is informed unless the
\xref{MSG filtering level}{sun104}{conditional_message_reporting}
\latex{ (see SUN/104)} is `quiet'.
         \end{itemize}
   }
}

\sstroutine{
   PAR\_DEF0x
}{
   Sets a scalar dynamic default parameter value
}{
   \sstdescription{
      This routine sets a scalar as the dynamic default value for a
      parameter. The dynamic default may be used as the parameter value
      by means of appropriate specifications in the interface file.

      If the declared parameter type differs from the type of the
      value supplied, then conversion is performed.
   }
   \sstinvocation{
      CALL PAR\_DEF0x( PARAM, VALUE, STATUS )
   }
   \sstarguments{
      \sstsubsection{
         PARAM = CHARACTER $*$ ( $*$ ) (Given)
      }{
         The name of the parameter.
      }
      \sstsubsection{
         VALUE = ? (Given)
      }{
         The dynamic default value for the parameter.
      }
      \sstsubsection{
         STATUS = INTEGER (Given and Returned)
      }{
         The global status.
      }
   }
   \sstnotes{
      \sstitemlist{
         \sstitem There is a routine for each of the data types character,
         double precision, integer, logical, and real: replace {\tt "x"} in the
         routine name by C, D, I, L, or R respectively as appropriate.  The
         VALUE argument must have the corresponding data type.
      }
   }
}

\sstroutine{
   PAR\_DEF1x
}{
   Sets a vector of values as the dynamic default for a parameter
}{
   \sstdescription{
      This routine sets a 1-dimensional array of values as the dynamic
      default for a parameter. The dynamic default may be used as the
      parameter value by means of appropriate specifications in the
      interface file.

      If the declared parameter type differs from the type of the
      array supplied, then conversion is performed.
   }
   \sstinvocation{
      CALL PAR\_DEF1x( PARAM, NVAL, VALUES, STATUS )
   }
   \sstarguments{
      \sstsubsection{
         PARAM = CHARACTER $*$ ( $*$ ) (Given)
      }{
         The name of a parameter of primitive type.
      }
      \sstsubsection{
         NVAL = INTEGER (Given)
      }{
         The number of default values.
      }
      \sstsubsection{
         VALUES( NVAL ) = ? (Given)
      }{
         The array to contain the default values.
      }
      \sstsubsection{
         STATUS = INTEGER (Given and Returned)
      }{
         The global status.
      }
   }
   \sstnotes{
      \sstitemlist{

         \sstitem
         There is a routine for each of the data types character,
         double precision, integer, logical, and real: replace {\tt "x"} in the
         routine name by C, D, I, L, or R respectively as appropriate.  The
         VALUES argument must have the corresponding data type.
      }
   }
}

\sstroutine{
   PAR\_DEFNx
}{
   Sets an array of values as the dynamic default for a parameter
}{
   \sstdescription{
      This routine sets an array of values as the dynamic default for
      a parameter. The dynamic default may be used as the parameter
      value by means of appropriate specifications in the interface
      file.

      If the declared parameter type differs from the type of the
      array supplied, then conversion is performed.
   }
   \sstinvocation{
      CALL PAR\_DEFNx( PARAM, NDIM, MAXD, VALUES, ACTD, STATUS )
   }
   \sstarguments{
      \sstsubsection{
         PARAM = CHARACTER$*$($*$) (Given)
      }{
         The name of the parameter.
      }
      \sstsubsection{
         NDIM = INTEGER (Given)
      }{
         The number of dimensions of the values array.
      }
      \sstsubsection{
         MAXD( NDIM ) = INTEGER (Given)
      }{
         The dimensions of the values' array.
      }
      \sstsubsection{
         VALUES( $*$ ) = ? (Given)
      }{
         The default values, given in Fortran order.
      }
      \sstsubsection{
         ACTD( NDIM ) = INTEGER (Given)
      }{
         The dimensions of the dynamic default object to be created.
      }
      \sstsubsection{
         STATUS = INTEGER (Given and Returned)
      }{
         The global status.
      }
   }
   \sstnotes{
      \sstitemlist{

         \sstitem
         There is a routine for each of the data types character,
         double precision, integer, logical, and real: replace {\tt "x"} in the
         routine name by C, D, I, L, or R respectively as appropriate.  The
         VALUES argument must have the corresponding data type.

         \sstitem
         The current implementation of the underlying parameter system,
         SUBPAR, creates an $n$-dimensional HDS object, containing the
         specified values. The ACTD argument gives the dimensions of
         the object to be created.  If the dynamic default is used as the
         suggested value in a prompt, the name of this object, rather than
         its contents, is offered.
      }
   }
}

\sstroutine{
   PAR\_EXACx
}{
   Obtains an exact number of values from a parameter
}{
   \sstdescription{
      This routine obtains an exact number of values from a parameter
      and stores them in a vector.  If the number of values obtained
      is less than that requested, a further get or gets will occur
      for the remaining values until the exact number required is
      obtained (or an error occurs).  The routine reports how many
      additional values are required prior to these further reads.
      Should too many values be entered an error results and a further
      read is attempted.
   }
   \sstinvocation{
      CALL PAR\_EXACx( PARAM, NVALS, VALUES, STATUS )
   }
   \sstarguments{
      \sstsubsection{
         PARAM = CHARACTER $*$ ( $*$ ) (Given)
      }{
         The name of the parameter.
      }
      \sstsubsection{
         NVALS = INTEGER (Given)
      }{
         The number of values needed.  Values will be requested until
         exactly this number of values (no more and no less) has been
         obtained.
      }
      \sstsubsection{
         VALUES( NVALS ) = ? (Returned)
      }{
         The values obtained from the parameter system.  They will only
         be valid if STATUS is not set to an error value.
      }
      \sstsubsection{
         STATUS = INTEGER (Given and Returned)
      }{
         The global status.
      }
   }
   \sstnotes{
      \sstitemlist{

         \sstitem
         There is a routine for each of the data types character,
         double precision, integer, logical, and real: replace {\tt "x"} in the
         routine name by C, D, I, L, or R respectively as appropriate.  The
         VALUES argument must have the corresponding data type.

         \sstitem
         If more than one attempt is made is to obtain the values, the
         current value of the parameter will be only that read at the
         last get, and not the full array of values.

         \sstitem
         A PAR\_\_ERROR status is returned when the number of values
         requested is not positive.
      }
   }
}

\sstroutine{
   PAR\_GDR0x
}{
   Obtains a scalar value within a given range from a parameter
}{
   \sstdescription{
      This routine obtains from a parameter a scalar value that lies
      within a supplied range of acceptable values.  A dynamic default
      may be defined.
   }
   \sstinvocation{
      CALL PAR\_GDR0x( PARAM, DEFAUL, VMIN, VMAX, NULL, VALUE, STATUS )
   }
   \sstarguments{
      \sstsubsection{
         PARAM = CHARACTER $*$ ( $*$ ) (Given)
      }{
         The name of the parameter.
      }
      \sstsubsection{
         DEFAUL = ? (Given)
      }{
         The suggested-default value for the parameter.  No default
         will be suggested when DEFAUL is not within the range of
         acceptable values defined by VMIN and VMAX.
      }
      \sstsubsection{
         VMIN = ? (Given)
      }{
         The value immediately above a range wherein the obtained
         value cannot lie.  Thus if VMAX is greater than VMIN, VMIN
         is the minimum allowed for the obtained value.  However,
         should VMAX be less than VMIN, all values are acceptable
         except those between VMAX and VMIN exclusive.
      }
      \sstsubsection{
         VMAX = ? (Given)
      }{
         The value immediately below a range wherein the obtained
         value cannot lie.  Thus if VMAX is greater than VMIN, VMAX
         is the maximum allowed for the obtained value.  However,
         should VMAX be less than VMIN, all values are acceptable
         except those between VMAX and VMIN exclusive.
      }
      \sstsubsection{
         NULL = LOGICAL (Given)
      }{
         NULL controls the behaviour of this routine when the parameter
         is in the null state.  If NULL is .FALSE., this routine
         returns with STATUS=PAR\_\_NULL.  If NULL is .TRUE., the
         returned VALUE takes the value of DEFAUL and, if the
\xref{MSG filtering level}{sun104}{conditional_message_reporting}
\latex{ (see SUN/104)} is `verbose',
         a message informs the user of the value used for
         the parameter. The routine then returns with STATUS=SAI\_\_OK.
         This feature is intended for convenient handling of null values.
         NULL should only be set to .TRUE. when the value of DEFAUL will
         always give a reasonable value for the parameter.
      }
      \sstsubsection{
         VALUE  = ? (Returned)
      }{
         The value associated with the parameter.  If STATUS is returned
         not equal to SAI\_\_OK, VALUE takes the value of DEFAUL.
      }
      \sstsubsection{
         STATUS = INTEGER (Given and Returned)
      }{
         The global status.
      }
   }
   \sstnotes{
      \sstitemlist{

         \sstitem
         There is a routine for each of the data types double precision,
         integer, and real: replace {\tt "x"} in the routine name by
         D, I, or R respectively as appropriate.  The DEFAUL, VMIN,
         VMAX, and VALUE arguments all must have the corresponding
         data type.

         \sstitem
         If the value violates the constraint, the user is informed of
         the constraint and prompted for another value.
      }
   }
}

\sstroutine{
   PAR\_GDR1x
}{
   Obtains an exact number of values within a given range from a
   parameter
}{
   \sstdescription{
      This routine obtains an exact number of values from a parameter.
      all of which must be within a supplied range of acceptable
      values.  Dynamic defaults may be defined.
   }
   \sstinvocation{
      CALL PAR\_GDR1x( PARAM, NVALS, DEFAUL, VMIN, VMAX, NULL, VALUES,
                      STATUS )
   }
   \sstarguments{
      \sstsubsection{
         PARAM = CHARACTER $*$ ( $*$ ) (Given)
      }{
         The name of the parameter.
      }
      \sstsubsection{
         NVALS = INTEGER (Given)
      }{
         The number of values needed.  Values will be requested until
         exactly this number (no more and no less) has been obtained.
      }
      \sstsubsection{
         DEFAUL( NVALS ) = ? (Given)
      }{
         The suggested-default values for the parameter.  No default
         will be suggested when any of the DEFAUL elements is not
         within the range of acceptable values defined by VMIN and
         VMAX.
      }
      \sstsubsection{
         VMIN = ? (Given)
      }{
         The value immediately above a range wherein the obtained
         values cannot lie.  Thus if VMAX is greater than VMIN, VMIN
         is the minimum allowed for the obtained values.  However,
         should VMAX be less than VMIN, all values are acceptable
         except those between VMAX and VMIN exclusive.
      }
      \sstsubsection{
         VMAX = ? (Given)
      }{
         The value immediately below a range wherein the obtained
         values cannot lie.  Thus if VMAX is greater than VMIN, VMAX
         is the maximum allowed for the obtained values.  However,
         should VMAX be less than VMIN, all values are acceptable
         except those between VMAX and VMIN exclusive.
      }
      \sstsubsection{
         NULL = LOGICAL (Given)
      }{
         NULL controls the behaviour of this routine when the parameter
         is in the null state.  If NULL is .FALSE., this routine
         returns with STATUS=PAR\_\_NULL.  If NULL is .TRUE., the
         returned VALUE takes the value of DEFAUL and, if the
\xref{MSG filtering level}{sun104}{conditional_message_reporting}
\latex{ (see SUN/104)} is `verbose',
         a message informs the user of the value used for
         the parameter. The routine then returns with STATUS=SAI\_\_OK.
         This feature is intended for convenient handling of null values.
         NULL should only be set to .TRUE. when the value of DEFAUL will
         always give a reasonable value for the parameter.
      }
      \sstsubsection{
         VALUES( NVALS ) = ? (Returned)
      }{
         The values associated with the parameter.  If STATUS is
         returned not equal to SAI\_\_OK, VALUE takes the values of
         DEFAUL.
      }
      \sstsubsection{
         STATUS = INTEGER (Given and Returned)
      }{
         The global status.
      }
   }
   \sstnotes{
      \sstitemlist{

         \sstitem
         There is a routine for each of the data types double precision,
         integer, and real: replace {\tt "x"} in the routine name by
         D, I, or R respectively as appropriate.  The DEFAUL, VMIN, VMAX,
         and VALUES arguments all must have the corresponding data type.

         \sstitem
         If any of the values violates the constraint, the user is
         informed of the constraint and prompted for another vector of
         values.  This is not achieved through the MIN/MAX system.
      }
   }
}

\sstroutine{
   PAR\_GDRVx
}{
   Obtains a vector of values within a given range from a parameter
}{
   \sstdescription{
      This routine obtains up to a given number of values from a
      parameter.  All the values must be within a supplied range of
      acceptable values.
   }
   \sstinvocation{
      CALL PAR\_GDRVx( PARAM, MAXVAL, VMIN, VMAX, VALUES, ACTVAL,
                      STATUS )
   }
   \sstarguments{
      \sstsubsection{
         PARAM = CHARACTER $*$ ( $*$ ) (Given)
      }{
         The name of the parameter.
      }
      \sstsubsection{
         MAXVAL = INTEGER (Given)
      }{
         The maximum number of values required.  A PAR\_\_ERROR status is
         returned when the number of values requested is less than one.
      }
      \sstsubsection{
         VMIN = ? (Given)
      }{
         The value immediately above a range wherein the obtained
         values cannot lie.  Thus if VMAX is greater than VMIN, VMIN
         is the minimum allowed for the obtained values.  However,
         should VMAX be less than VMIN, all values are acceptable
         except those between VMAX and VMIN exclusive.
      }
      \sstsubsection{
         VMAX = ? (Given)
      }{
         The value immediately below a range wherein the obtained
         values cannot lie.  Thus if VMAX is greater than VMIN, VMAX
         is the maximum allowed for the obtained values.  However,
         should VMAX be less than VMIN, all values are acceptable
         except those between VMAX and VMIN exclusive.
      }
      \sstsubsection{
         VALUES( MAXVAL ) = ? (Returned)
      }{
         The values associated with the parameter.  They will only be
         valid if STATUS is not set to an error value.
      }
      \sstsubsection{
         ACTVAL = INTEGER (Returned)
      }{
         The actual number of values obtained.
      }
      \sstsubsection{
         STATUS = INTEGER (Given and Returned)
      }{
         The global status.
      }
   }
   \sstnotes{
      \sstitemlist{

         \sstitem
         There is a routine for each of the data types double precision,
         integer, and real: replace {\tt "x"} in the routine name by D, I, or R
         respectively as appropriate.  The VMIN, VMAX, and VALUES arguments
         all must have the corresponding data type.

         \sstitem
         Should too many values be read, the parameter system will
         repeat the get in order to obtain a permitted number of values.

         \sstitem
         If any of the values violates the constraint, the user is
         informed of the constraint and prompted for another vector of
         values.  This is not achieved through the MIN/MAX system.
      }
   }
}

\sstroutine{
   PAR\_GET0x
}{
   Obtains a scalar value from a parameter
}{
   \sstdescription{
      This routine obtains a scalar value from a parameter.  If it is
      necessary, the value is converted to the required type.
   }
   \sstinvocation{
      CALL PAR\_GET0x( PARAM, VALUE, STATUS )
   }
   \sstarguments{
      \sstsubsection{
         PARAM = CHARACTER $*$ ( $*$ ) (Given)
      }{
         The parameter name.
      }
      \sstsubsection{
         VALUE = ? (Returned)
      }{
         The parameter value.
      }
      \sstsubsection{
         STATUS = INTEGER (Given and Returned)
      }{
         The global status.
      }
   }
   \sstnotes{
      \sstitemlist{

         \sstitem
         There is a routine for each of the data types character,
         double precision, integer, logical, and real: replace {\tt "x"} in the
         routine name by C, D, I, L, or R respectively as appropriate.  The
         VALUE argument must have the corresponding data type.

         \sstitem
         Note that a scalar (0-dimensional) parameter is different from
         a vector (1-dimensional) parameter containing a single value.
      }
   }
}

\sstroutine{
   PAR\_GET1x
}{
   Obtains a vector of values from a parameter
}{
   \sstdescription{
      This routine obtains a vector of values from a parameter. If it is
      necessary, the values are converted to the required type.
   }
   \sstinvocation{
      CALL PAR\_GET1x( PARAM, MAXVAL, VALUES, ACTVAL, STATUS )
   }
   \sstarguments{
      \sstsubsection{
         PARAM = CHARACTER $*$ ( $*$ ) (Given)
      }{
         The parameter name.
      }
      \sstsubsection{
         MAXVAL = INTEGER (Given)
      }{
         The maximum number of values that can be obtained.
      }
      \sstsubsection{
         VALUES( MAXVAL ) = ? (Returned)
      }{
         The array to receive the values associated with the parameter.
      }
      \sstsubsection{
         ACTVAL = INTEGER (Returned)
      }{
         The actual number of values obtained.
      }
      \sstsubsection{
         STATUS = INTEGER (Given and Returned)
      }{
         The global status.
      }
   }
   \sstnotes{
      \sstitemlist{

         \sstitem
         There is a routine for each of the data types character,
         double precision, integer, logical, and real: replace {\tt "x"} in the
         routine name by C, D, I, L, or R respectively as appropriate.  The
         VALUES argument must have the corresponding data type.

         \sstitem
         Note that this routine will accept a scalar value, returning
         a single-element vector.
      }
   }
}

\sstroutine{
   PAR\_GETNx
}{
   Obtains an array parameter value
}{
   \sstdescription{
      This routine obtains an $n$-dimensional array of values from a
      parameter.  If necessary, the values are converted to the
      required type.
   }
   \sstinvocation{
      CALL PAR\_GETNx( PARAM, NDIM, MAXD, VALUES, ACTD, STATUS )
   }
   \sstarguments{
      \sstsubsection{
         PARAM = CHARACTER $*$ ( $*$ ) (Given)
      }{
         The parameter name.
      }
      \sstsubsection{
         NDIM = INTEGER (Given)
      }{
         The number of dimensions of the values array.
         This must match the number of dimensions of the parameter.
      }
      \sstsubsection{
         MAXD( NDIM ) = INTEGER (Given)
      }{
         Array specifying the maximum dimensions of the array to be
         read. These may not be smaller than the dimensions of the
         actual parameter nor greater than the dimensions of the VALUES
         array.
      }
      \sstsubsection{
         VALUES( $*$ ) = ? (Returned)
      }{
         The values obtained from the parameter.  These are in Fortran
         order.
      }
      \sstsubsection{
         ACTD( NDIM ) = INTEGER (Returned)
      }{
         The actual dimensions of the array.  Unused dimensions are set
         to 1.
      }
      \sstsubsection{
         STATUS = INTEGER
      }{
         The global status.
      }
   }
   \sstnotes{
      \sstitemlist{

         \sstitem
         There is a routine for each of the data types character,
         double precision, integer, logical, and real: replace {\tt "x"} in the
         routine name by C, D, I, L, or R respectively as appropriate.  The
         VALUES argument must have the corresponding data type.

         \sstitem
         Note that this routine will accept a scalar value, returning
         a single-element array with the specified number of dimensions.
      }
   }
}

\sstroutine{
   PAR\_GETVx
}{
   Obtains a vector of values from a parameter regardless of the
   its shape
}{
   \sstdescription{
      This routine obtains an array of values from a parameter
      as if the parameter were vectorized ({\it i.e.}\ regardless of its
      dimensionality).  If necessary, the values are converted to the
      required type.
   }
   \sstinvocation{
      CALL PAR\_GETVx( PARAM, MAXVAL, VALUES, ACTVAL, STATUS )
   }
   \sstarguments{
      \sstsubsection{
         PARAM = CHARACTER $*$ ( $*$ ) (Given)
      }{
         The parameter name.
      }
      \sstsubsection{
         MAXVAL = INTEGER (Given)
      }{
         The maximum number of values that can be held in the values
         array.
      }
      \sstsubsection{
         VALUES( MAXVAL ) = ? (Returned)
      }{
         Array to receive the values associated with the object.
      }
      \sstsubsection{
         ACTVAL = INTEGER (Returned)
      }{
         The actual number of values obtained.
      }
      \sstsubsection{
         STATUS = INTEGER (Given and Returned)
      }{
         The global status.
      }
   }
   \sstnotes{
      \sstitemlist{

         \sstitem
         There is a routine for each of the data types character,
         double precision, integer, logical, and real: replace {\tt "x"} in the
         routine name by C, D, I, L, or R respectively as appropriate.  The
         VALUES argument must have the corresponding data type.

         \sstitem
         Note that this routine will accept a scalar value, returning
         a single-element vector.
      }
   }
}

\sstroutine{
   PAR\_GEVEN
}{
   Obtains an even integer value from a parameter
}{
   \sstdescription{
      This routine obtains a scalar integer value from a parameter.
      This value must be even and within a supplied range of acceptable
      values.  A dynamic default may be defined.
   }
   \sstinvocation{
      CALL PAR\_GEVEN( PARAM, DEFAUL, VMIN, VMAX, NULL, VALUE, STATUS )
   }
   \sstarguments{
      \sstsubsection{
         PARAM = CHARACTER $*$ ( $*$ ) (Given)
      }{
         The name of the parameter.
      }
      \sstsubsection{
         DEFAUL = INTEGER (Given)
      }{
         The suggested-default value for the parameter.  No default
         will be suggested when DEFAUL is not within the range of
         acceptable values defined by VMIN and VMAX, or DEFAUL is odd.
      }
      \sstsubsection{
         VMIN = INTEGER (Given)
      }{
         The value immediately above a range wherein the obtained
         value cannot lie.  Thus if VMAX is greater than VMIN, VMIN
         is the minimum allowed for the obtained value.  However,
         should VMAX be less than VMIN, all values are acceptable
         except those between VMAX and VMIN exclusive.
      }
      \sstsubsection{
         VMAX = INTEGER (Given)
      }{
         The value immediately below a range wherein the obtained
         value cannot lie.  Thus if VMAX is greater than VMIN, VMAX
         is the maximum allowed for the obtained value.  However,
         should VMAX be less than VMIN, all values are acceptable
         except those between VMAX and VMIN exclusive.
      }
      \sstsubsection{
         NULL = LOGICAL (Given)
      }{
         NULL controls the behaviour of this routine when the parameter
         is in the null state.  If NULL is .FALSE., this routine
         returns with STATUS=PAR\_\_NULL.  If NULL is .TRUE., the
         returned VALUE takes the value of DEFAUL and, if the
\xref{MSG filtering level}{sun104}{conditional_message_reporting}
\latex{ (see SUN/104)} is `verbose',
         a message informs the user of the value used for
         the parameter. The routine then returns with STATUS=SAI\_\_OK.
         This feature is intended for convenient handling of null values.
         NULL should only be set to .TRUE. when the value of DEFAUL will
         always give a reasonable value for the parameter.
      }
      \sstsubsection{
         VALUE  = INTEGER (Returned)
      }{
         The value associated with the parameter.  If STATUS is returned
         not equal to SAI\_\_OK, VALUE takes the value of DEFAUL.
      }
      \sstsubsection{
         STATUS = INTEGER (Given and Returned)
      }{
         The global status.
      }
   }
   \sstnotes{
      \sstitemlist{

         \sstitem
         Zero is deemed to be even.

         \sstitem
         If the value violates the constraint, the user is informed of
         the constraint and prompted for another value.
      }
   }
}

\sstroutine{
   PAR\_GODD
}{
   Obtains an odd integer value from a parameter
}{
   \sstdescription{
      This routine obtains a scalar integer value from a parameter.
      This value must be odd and within a supplied range of acceptable
      values.  A dynamic default may be defined.
   }
   \sstinvocation{
      CALL PAR\_GODD( PARAM, DEFAUL, VMIN, VMAX, NULL, VALUE, STATUS )
   }
   \sstarguments{
      \sstsubsection{
         PARAM = CHARACTER $*$ ( $*$ ) (Given)
      }{
         The name of the parameter.
      }
      \sstsubsection{
         DEFAUL = INTEGER (Given)
      }{
         The suggested-default value for the parameter.  No default
         will be suggested when DEFAUL is not within the range of
         acceptable values defined by VMIN and VMAX, or DEFAUL is odd.
      }
      \sstsubsection{
         VMIN = INTEGER (Given)
      }{
         The value immediately above a range wherein the obtained
         value cannot lie.  Thus if VMAX is greater than VMIN, VMIN
         is the minimum allowed for the obtained value.  However,
         should VMAX be less than VMIN, all values are acceptable
         except those between VMAX and VMIN exclusive.
      }
      \sstsubsection{
         VMAX = INTEGER (Given)
      }{
         The value immediately below a range wherein the obtained
         value cannot lie.  Thus if VMAX is greater than VMIN, VMAX
         is the maximum allowed for the obtained value.  However,
         should VMAX be less than VMIN, all values are acceptable
         except those between VMAX and VMIN exclusive.
      }
      \sstsubsection{
         NULL = LOGICAL (Given)
      }{
         NULL controls the behaviour of this routine when the parameter
         is in the null state.  If NULL is .FALSE., this routine
         returns with STATUS=PAR\_\_NULL.  If NULL is .TRUE., the
         returned VALUE takes the value of DEFAUL and, if the
\xref{MSG filtering level}{sun104}{conditional_message_reporting}
\latex{ (see SUN/104)} is `verbose',
         a message informs the user of the value used for
         the parameter. The routine then returns with STATUS=SAI\_\_OK.
         This feature is intended for convenient handling of null values.
         NULL should only be set to .TRUE. when the value of DEFAUL will
         always give a reasonable value for the parameter.
      }
      \sstsubsection{
         VALUE  = INTEGER (Returned)
      }{
         The value associated with the parameter.  If STATUS is returned
         not equal to SAI\_\_OK, VALUE takes the value of DEFAUL.
      }
      \sstsubsection{
         STATUS = INTEGER (Given and Returned)
      }{
         The global status.
      }
   }
   \sstnotes{
      \sstitemlist{

         \sstitem
         Zero is deemed to be even.

         \sstitem
         If the value violates the constraint, the user is informed of
         the constraint and prompted for another value.
      }
   }
}

\sstroutine{
   PAR\_GRM1x
}{
   Obtains from a parameter an exact number of values each within a
   given range
}{
   \sstdescription{
      This routine obtains an exact number of values from a parameter.
      Each value must be within its own range of acceptable values
      supplied to the routine.  Dynamic defaults may be defined.

      This routine is particularly useful for obtaining co-ordinate
      information, where each co-ordinate has different bounds, and
      a series of calls to PAR\_GDR0x would not permit an arbitrary
      number of dimensions.
   }
   \sstinvocation{
      CALL PAR\_GRM1x( PARAM, NVALS, DEFAUL, VMIN, VMAX, NULL, VALUES,
                      STATUS )
   }
   \sstarguments{
      \sstsubsection{
         PARAM = CHARACTER $*$ ( $*$ ) (Given)
      }{
         The name of the parameter.
      }
      \sstsubsection{
         NVALS = INTEGER (Given)
      }{
         The number of values needed.  Values will be requested until
         exactly this number (no more and no less) has been obtained.
      }
      \sstsubsection{
         DEFAUL( NVALS ) = ? (Given)
      }{
         The suggested-default values for the parameter.  No default
         will be suggested when any of the DEFAUL elements is not
         within the range of acceptable values defined by VMIN and
         VMAX for that value.
      }
      \sstsubsection{
         VMIN( NVALS ) = ? (Given)
      }{
         The values immediately above a range wherein each obtained
         value cannot lie.  Thus if VMAX is greater than VMIN, VMIN
         is the minimum allowed for the corresponding obtained value.
         However, should VMAX be less than VMIN, all values are
         acceptable except those between VMAX and VMIN exclusive.
      }
      \sstsubsection{
         VMAX( NVALS ) = ? (Given)
      }{
         The values immediately below a range wherein each obtained
         value cannot lie.  Thus if VMAX is greater than VMIN, VMAX
         is the maximum allowed for the corresponding obtained value.
         However, should VMAX be less than VMIN, all values are
         acceptable except those between VMAX and VMIN exclusive.
      }
      \sstsubsection{
         NULL = LOGICAL (Given)
      }{
         NULL controls the behaviour of this routine when the parameter
         is in the null state.  If NULL is .FALSE., this routine
         returns with STATUS=PAR\_\_NULL.  If NULL is .TRUE., the
         returned VALUE takes the value of DEFAUL and, if the
\xref{MSG filtering level}{sun104}{conditional_message_reporting}
\latex{ (see SUN/104)} is `verbose',
         a message informs the user of the value used for
         the parameter. The routine then returns with STATUS=SAI\_\_OK.
         This feature is intended for convenient handling of null values.
         NULL should only be set to .TRUE. when the value of DEFAUL will
         always give a reasonable value for the parameter.
      }
      \sstsubsection{
         VALUES( NVALS ) = ? (Returned)
      }{
         The values associated with the parameter.  If STATUS is
         returned not equal to SAI\_\_OK, VALUE takes the values of
         DEFAUL.
      }
      \sstsubsection{
         STATUS = INTEGER (Given and Returned)
      }{
         The global status.
      }
   }
   \sstnotes{
      \sstitemlist{

         \sstitem
         There is a routine for each of the data types double precision,
         integer, and real: replace {\tt "x"} in the routine name by D, I, or R
         respectively as appropriate.  The DEFAUL, VMIN, VMAX, and VALUES
         arguments must have the corresponding data type.

         \sstitem
         If any of the values violates the constraints, the user is
         informed of the constraints and prompted for another vector of
         values.  This is not achieved through the MIN/MAX system.
      }
   }
}

\sstroutine{
   PAR\_GRMVx
}{
   Obtains from a parameter a vector of values each within a given
   range
}{
   \sstdescription{
      This routine obtains from a parameter up to a given number of
      values.  Each value must be within its own range of acceptable
      values supplied to the routine.

      This routine is particularly useful for obtaining values that
      apply to $n$-dimensional array where each value is constrained by
      the size or bounds of the array, and where the number of values
      need not equal $n$.  For example, the size of a smoothing kernel
      could be defined by one value that applies to all dimensions, or
      as individual sizes along each dimension.
   }
   \sstinvocation{
      CALL PAR\_GRMVx( PARAM, MAXVAL, VMIN, VMAX, VALUES, ACTVAL,
                      STATUS )
   }
   \sstarguments{
      \sstsubsection{
         PARAM = CHARACTER $*$ ( $*$ ) (Given)
      }{
         The name of the parameter.
      }
      \sstsubsection{
         MAXVAL = INTEGER (Given)
      }{
         The maximum number of values required.  A PAR\_\_ERROR status is
         returned when the number of values requested is less than one.
      }
      \sstsubsection{
         VMIN( MAXVAL ) = ? (Given)
      }{
        The values immediately above a range wherein each obtained
         value cannot lie.  Thus if VMAX is greater than VMIN, VMIN
         is the minimum allowed for the corresponding obtained value.
         However, should VMAX be less than VMIN, all values are
         acceptable except those between VMAX and VMIN exclusive.
      }
      \sstsubsection{
         VMAX( MAXVAL ) = ? (Given)
      }{
         The values immediately below a range wherein each obtained
         value cannot lie.  Thus if VMAX is greater than VMIN, VMAX
         is the maximum allowed for the corresponding obtained value.
         However, should VMAX be less than VMIN, all values are
         acceptable except those between VMAX and VMIN exclusive.
      }
      \sstsubsection{
         VALUES( MAXVAL ) = ? (Returned)
      }{
         The values associated with the parameter.  They will only be
         valid if STATUS is not set to an error value.
      }
      \sstsubsection{
         ACTVAL = INTEGER (Returned)
      }{
         The actual number of values obtained.
      }
      \sstsubsection{
         STATUS = INTEGER (Given and Returned)
      }{
         The global status.
      }
   }
   \sstnotes{
      \sstitemlist{

         \sstitem
         There is a routine for each of the data types double precision,
         integer, and real: replace {\tt "x"} in the routine name by D, I, or R
         respectively as appropriate.  The VMIN, VMAX, and VALUES
         arguments must have the corresponding data type.

         \sstitem
         Should too many values be obtained, the parameter system will
         repeat the get in order to obtain a permitted number of values.

         \sstitem
         If any of the values violates the constraints, the user is
         informed of the constraints and prompted for another vector of
         values.  This is not achieved through the MIN/MAX system.
      }
   }
}

\sstroutine{
   PAR\_GTD0L
}{
   Obtains a logical value from a parameter with a dynamic
   default
}{
   \sstdescription{
      This routine obtains a scalar logical value from a parameter.
      A dynamic default is defined.
   }
   \sstinvocation{
      CALL PAR\_GTD0L( PARAM, DEFAUL, NULL, VALUE, STATUS )
   }
   \sstarguments{
      \sstsubsection{
         PARAM = CHARACTER $*$ ( $*$ ) (Given)
      }{
         The name of the parameter.
      }
      \sstsubsection{
         DEFAUL = LOGICAL (Given)
      }{
         The suggested default value for the parameter.
      }
      \sstsubsection{
         NULL = LOGICAL (Given)
      }{
         NULL controls the behaviour of this routine when the parameter
         is in the null state.  If NULL is .FALSE., this routine
         returns with STATUS=PAR\_\_NULL.  If NULL is .TRUE., the
         returned VALUE takes the value of DEFAUL and, if the
\xref{MSG filtering level}{sun104}{conditional_message_reporting}
\latex{ (see SUN/104)} is `verbose',
         a message informs the user of the value used for
         the parameter. The routine then returns with STATUS=SAI\_\_OK.
         This feature is intended for convenient handling of null values.
         NULL should only be set to .TRUE. when the value of DEFAUL will
         always give a reasonable value for the parameter.
      }
      \sstsubsection{
         VALUE  = LOGICAL (Returned)
      }{
         The value associated with the parameter.  It will only be
         valid if STATUS is not set to an error value.
      }
      \sstsubsection{
         STATUS = INTEGER (Given and Returned)
      }{
         The global status.
      }
   }
}

\sstroutine{
   PAR\_MAXx
}{
   Sets a maximum value for a parameter
}{
   \sstdescription{
      This routine sets a maximum value for the specified parameter.
      The value will be used as an upper limit for any value
      subsequently obtained for the parameter.

      If the routine fails, any existing maximum value will be unset.
   }
   \sstinvocation{
      CALL PAR\_MAXx( PARAM, VALUE, STATUS )
   }
   \sstarguments{
      \sstsubsection{
         PARAM = CHARACTER$*$($*$) (Given)
      }{
         The parameter name.
      }
      \sstsubsection{
         VALUE = ? (Given)
      }{
         The value to be set as the maximum.  It must not be outside
         any RANGE specified for the parameter in the interface file.
      }
      \sstsubsection{
         STATUS = INTEGER (Given and Returned)
      }{
         The global status.
      }
   }
   \sstnotes{
      \sstitemlist{

         \sstitem
         There is a routine for each of the data types character,
         integer, real, and double precision: replace {\tt "x"} in the routine
         name by C, I, R, or D respectively as appropriate.  The VALUE
         argument must have the corresponding data type.  If the parameter
         has a different type, the maximum value will be converted to the
         type of the parameter which must be character, double precision,
         integer, or real.

         \sstitem
         If a minimum value has been set (using PAR\_MINx) that is
         greater than the maximum at the time the parameter value is
         obtained, only values between the limits will not be permitted.

         \sstitem
         The maximum value set by this routine overrides any upper
         RANGE value which may have been specified in the interface file.
         The specified value must not be outside any RANGE values.  The
         maximum value may also be selected as the parameter value \dash\ again
         in preference to any upper RANGE value \dash\ by specifying MAX as the
         parameter value on the command line or in response to a prompt.
      }
   }
}

\sstroutine{
   PAR\_MINx
}{
   Sets a minimum value for a parameter
}{
   \sstdescription{
      This routine sets a minimum value for the specified parameter.
      The value will be used as a lower limit for any value
      subsequently obtained for the parameter.

      If the routine fails, any existing minimum value will be unset.
   }
   \sstinvocation{
      CALL PAR\_MINx( PARAM, VALUE, STATUS )
   }
   \sstarguments{
      \sstsubsection{
         PARAM = CHARACTER $*$ ( $*$ ) (Given)
      }{
         The parameter name.
      }
      \sstsubsection{
         VALUE = ? (Given)
      }{
         The value to be set as the minimum. It must not be outside
         any RANGE specified for the parameter in the interface file.
      }
      \sstsubsection{
         STATUS = INTEGER (Given and Returned)
      }{
         The global status.
      }
   }
   \sstnotes{
      \sstitemlist{

         \sstitem
         There is a routine for each of the data types character,
         integer, real, and double precision: replace {\tt "x"} in the routine
         name by C, I, R, or D respectively as appropriate.  The VALUE
         argument must have the corresponding data type.  If the parameter
         has a different type, the minimum value will be converted to the
         type of the parameter which must be character, double precision,
         integer, or real.

         \sstitem
         If a maximum value has been set (using PAR\_MAXx) that is less
         than the minimum at the time the parameter value is obtained,
         only values between the limits will not be permitted.

         \sstitem
         The minimum value set by this routine overrides any lower
          RANGE value which may have been specified in the interface file.
         The specified value must not be outside any RANGE values.  The
         minimum value may also be selected as the parameter value \dash\ again
         in preference to any lower RANGE value \dash\ by specifying MIN as the
         parameter value on the command line or in response to a prompt.
      }
   }
}

\sstroutine{
   PAR\_MIX0x
}{
   Obtains from a parameter a character value from either a menu of
   options or within a numeric range
}{
   \sstdescription{
      This routine obtains a scalar character value from a parameter.
      The value must be either:
      \begin{itemize}
         \item one of a supplied list of acceptable values, with
            unambiguous abbreviations accepted; or

         \item a numeric character string equivalent to a number, and the
            number must lie within a supplied range of acceptable
            values.
      \end{itemize}

      A dynamic default may be suggested.
   }
   \sstinvocation{
      CALL PAR\_MIX0x( PARAM, DEFAUL, VMIN, VMAX, OPTS, NULL, VALUE,
                      STATUS )
   }
   \sstarguments{
      \sstsubsection{
         PARAM = CHARACTER $*$ ( $*$ ) (Given)
      }{
         The name of the parameter.
      }
      \sstsubsection{
         DEFAUL = CHARACTER $*$ ( $*$ ) (Given)
      }{
         The suggested-default value for the parameter.  No default
         will be suggested when DEFAUL is not numeric within the range
         of acceptable values defined by VMIN and VMAX, and not one of
         the options defined by OPTS.  A status of PAR\_\_AMBIG is
         returned if the default is ambiguous, {\it i.e.}\ an abbreviation of
         more than one of the options.
      }
      \sstsubsection{
         VMIN = ? (Given)
      }{
        The value immediately above a range wherein the obtained
         numeric value cannot lie.  Thus if VMAX is greater than VMIN,
         VMIN is the minimum numeric value allowed for the obtained
         value.  However, should VMAX be less than VMIN, all numeric
         values are acceptable except those between VMAX and VMIN
         exclusive.
      }
      \sstsubsection{
         VMAX = ? (Given)
      }{
         The value immediately below a range wherein the obtained
         numeric value cannot lie.  Thus if VMAX is greater than VMIN,
         VMAX is the maximum numeric value allowed for the obtained
         value.  However, should VMAX be less than VMIN, all numeric
         values are acceptable except those between VMAX and VMIN
         exclusive.
      }
      \sstsubsection{
         OPTS = CHARACTER $*$ ( $*$ ) (Given)
      }{
         The list of acceptable options for the value obtained from the
         parameter.  Items should be separated by commas.  The list is
         case-insensitive.
      }
      \sstsubsection{
         NULL = LOGICAL (Given)
      }{
         NULL controls the behaviour of this routine when the parameter
         is in the null state.  If NULL is .FALSE., this routine
         returns with STATUS=PAR\_\_NULL.  If NULL is .TRUE., the
         returned VALUE takes the value of DEFAUL and, if the
\xref{MSG filtering level}{sun104}{conditional_message_reporting}
\latex{ (see SUN/104)} is `verbose',
         a message informs the user of the value used for
         the parameter. The routine then returns with STATUS=SAI\_\_OK.
         This feature is intended for convenient handling of null values.
         NULL should only be set to .TRUE. when the value of DEFAUL will
         always give a reasonable value for the parameter.
      }
      \sstsubsection{
         VALUE = CHARACTER $*$ ( $*$ ) (Returned)
      }{
         The selected option from the list or the character form of the
         numeric value.  The former is in uppercase and in full, even
         if an abbreviation has been given for the actual parameter.
         If STATUS is returned not equal to SAI\_\_OK, VALUE takes the
         value of DEFAUL.
      }
      \sstsubsection{
         STATUS = INTEGER (Given and Returned)
      }{
         The global status.
      }
   }
   \sstnotes{
      \sstitemlist{

         \sstitem
         There is a routine for each of the data types double precision,
         integer, and real: replace {\tt "x"} in the routine name by D, I, or R
         respectively as appropriate.  The VMIN and VMAX arguments must
         have the corresponding data type.

         \sstitem
         There are two stages to identify or validate the character
         value obtained from the parameter.

      In the first the value is converted to the data type specified by
      the {\tt "x"} in the routine name.  If this is successful, the derived
      numeric value is compared with the range of acceptable values
      defined by VMIN and VMAX.  A value satisfying these constraints
      is returned in VALUE and the routine exits.

      The second stage searches for a match of the character value with
      an item in the menu.  This step adheres to the following rules.
         \begin{itemize}
         \item  The value is converted to the data type specified by the
         {\tt "x"} in the routine name.  If this is successful, the numeric
         value is compared with the range of acceptable values defined
         by VMIN and VMAX.  A value satisfying these constraints is
         returned and the matching process terminates.
         \item  All comparisons are performed in uppercase.  Leading blanks
         are ignored.
         \item  A match is found when the value equals the full name of an
         option.  This enables an option to be the prefix of another
         item without it being regarded as ambiguous.  For example,
         {\tt "}10,100,200{\tt "} would be an acceptable list of options.
         \item If there is no exact match, an abbreviation is acceptable.
         A comparison is made of the value with each option for the
         number of characters in the value.  The option that best fits
         the value is declared a match, subject to two provisos.
         Firstly, there must be no more than one character different
         between the value and the start of the option.  (This allows
         for a mistyped character.)  Secondly, there must be only one
         best-fitting option.  Whenever these criteria are not
         satisfied, the user is told of the error, and is presented
         with the list of options, before being prompted for a new value
         If a nearest match is selected, the user is informed unless the
\xref{MSG filtering level}{sun104}{conditional_message_reporting}
\latex{ (see SUN/104)} is `quiet'.
         \end{itemize}
      }
   }
}

\sstroutine{
   PAR\_MIXVx
}{
   Obtains from a parameter character values from either a menu of
   options or within a numeric range
}{
   \sstdescription{
      This routine obtains a vector of character values from a
      parameter.  Each value must be either:
         \begin{itemize}

         \item one of a supplied list of acceptable values, with
            unambiguous abbreviations accepted; or

         \item  a numeric character string equivalent to a number, and the
            number must lie within a supplied range of acceptable
            values.
         \end{itemize}
   }
   \sstinvocation{
      CALL PAR\_MIXVx( PARAM, MAXVAL, VMIN, VMAX, OPTS, VALUES, ACTVAL,
                      STATUS )
   }
   \sstarguments{
      \sstsubsection{
         PARAM = CHARACTER $*$ ( $*$ ) (Given)
      }{
         The name of the parameter.
      }
      \sstsubsection{
         MAXVAL = INTEGER (Given)
      }{
         The maximum number of values required.  A PAR\_\_ERROR status is
         returned when the number of values requested is less than one.
      }
      \sstsubsection{
         VMIN = ? (Given)
      }{
         The value immediately above a range wherein the obtained
         numeric values cannot lie.  Thus if VMAX is greater than VMIN,
         VMIN is the minimum numeric value allowed for the obtained
         values.  However, should VMAX be less than VMIN, all numeric
         values are acceptable except those between VMAX and VMIN
         exclusive.
      }
      \sstsubsection{
         VMAX = ? (Given)
      }{
         The value immediately below a range wherein the obtained
         numeric values cannot lie.  Thus if VMAX is greater than VMIN,
         VMAX is the maximum numeric value allowed for the obtained
         values.  However, should VMAX be less than VMIN, all numeric
         values are acceptable except those between VMAX and VMIN
         exclusive.
      }
      \sstsubsection{
         OPTS = CHARACTER $*$ ( $*$ ) (Given)
      }{
         The list of acceptable options for each value obtained from the
         parameter.  Items should be separated by commas.  The list is
         case-insensitive.
      }
      \sstsubsection{
         VALUES( MAXVAL ) = CHARACTER $*$ ( $*$ ) (Returned)
      }{
         The selected values that are either options from the list or
         the character form of numeric values that satisfy the range
         constraint.  The former values are in uppercase and in full,
         even if an abbreviation has been given for the actual
         parameter.  Note that all values must satisfy the constraints.
         The values will only be valid if STATUS is not set to an error
         value.
      }
      \sstsubsection{
         STATUS = INTEGER (Given and Returned)
      }{
         The global status.
      }
   }
   \sstnotes{
      \sstitemlist{

         \sstitem
         There is a routine for each of the data types double precision,
         integer, and real: replace {\tt "x"} in the routine name by D, I, or R
         respectively as appropriate.  The VMIN and VMAX arguments must
         have the corresponding data type.

         \sstitem
         There are two stages to identify or validate each character
         value obtained from the parameter.

      In the first the value is converted to the data type specified by
      the {\tt "x"} in the routine name.  If this is successful, the derived
      numeric value is compared with the range of acceptable values
      defined by VMIN and VMAX.  A value satisfying these constraints
      is returned in the VALUES.

      The second stage searches for a match of the character value with
      an item in the menu.  This step adheres to the following rules.
         \begin{itemize}
         \item  The value is converted to the data type specified by the
         {\tt "x"} in the routine name.  If this is successful, the numeric
         value is compared with the range of acceptable values defined
         by VMIN and VMAX.  A value satisfying these constraints is
         returned and the matching process terminates.
         \item  All comparisons are performed in uppercase.  Leading blanks
         are ignored.
         \item  A match is found when the value equals the full name of an
         option.  This enables an option to be the prefix of another
         item without it being regarded as ambiguous.  For example,
         {\tt "}10,100,200{\tt "} would be an acceptable list of options.
         \item If there is no exact match, an abbreviation is acceptable.
         A comparison is made of the value with each option for the
         number of characters in the value.  The option that best fits
         the value is declared a match, subject to two provisos.
         Firstly, there must be no more than one character different
         between the value and the start of the option.  (This allows
         for a mistyped character.)  Secondly, there must be only one
         best-fitting option.  Whenever these criteria are not
         satisfied, the user is told of the error, and is presented
         with the list of options, before being prompted for a new value
         If a nearest match is selected, the user is informed unless the
\xref{MSG filtering level}{sun104}{conditional_message_reporting}
\latex{ (see SUN/104)} is `quiet'.
         \end{itemize}

      This routine exits when all the values satisfy the criteria.
      }
   }
}

\sstroutine{
   PAR\_PROMT
}{
   Sets a new prompt string for a parameter
}{
   \sstdescription{
      Replace the prompt string for the indicated parameter by the
      given string.
   }
   \sstinvocation{
      CALL PAR\_PROMT( PARAM, PROMPT, STATUS )
   }
   \sstarguments{
      \sstsubsection{
         PARAM = CHARACTER $*$ ( $*$ ) (Given)
      }{
         The parameter name.
      }
      \sstsubsection{
         PROMPT = CHARACTER $*$ ( $*$ ) (Given)
      }{
         The new prompt string.
      }
      \sstsubsection{
         STATUS = INTEGER
      }{
         The global status.
      }
   }
}

\sstroutine{
   PAR\_PUT0x
}{
   Puts a scalar value into a parameter
}{
   \sstdescription{
      This routine puts a scalar value into a parameter.  If necessary,
      the specified value is converted to the type of the parameter.
   }
   \sstinvocation{
      CALL PAR\_PUT0x( PARAM, VALUE, STATUS )
   }
   \sstarguments{
      \sstsubsection{
         PARAM = CHARACTER $*$ ( $*$ ) (Given)
      }{
         The parameter name.
      }
      \sstsubsection{
         VALUE = ? (Given)
      }{
         The value to be put into the parameter.
      }
      \sstsubsection{
         STATUS = INTEGER (Given and Returned)
      }{
         The global status.
      }
   }
   \sstnotes{
      \sstitemlist{

         \sstitem
         There is a routine for each of the data types character,
         double precision, integer, logical, and real: replace {\tt "x"} in the
         routine name by C, D, I, L, or R respectively as appropriate.  The
         VALUE argument must have the corresponding data type.

         \sstitem
         A scalar (0-dimensional) parameter is different from a vector
         (1-dimensional) parameter containing a single value.

         \sstitem
         In order to obtain a storage object for the parameter, the
         current implementation of the underlying ADAM parameter system
         will proceed in the same way as it does for input parameters.
         This can result in users being prompted for `a value'. This
         behaviour, and how to avoid it, is discussed further in the
         Interface Module Reference Manual
         (\xref{SUN/115}{sun115}{parameter_specification_for_output_parameters}).

         \sstitem
         Limit checks for IN, RANGE, MIN/MAX are not applied.
      }
   }
}

\sstroutine{
   PAR\_PUT1x
}{
   Puts a vector of values into a parameter
}{
   \sstdescription{
      This routine puts a 1-dimensional array of values into a
      parameter.  If necessary, the specified array is converted to
      the type of the parameter.
   }
   \sstinvocation{
      CALL PAR\_PUT1x( PARAM, NVAL, VALUES, STATUS )
   }
   \sstarguments{
      \sstsubsection{
         PARAM = CHARACTER $*$ ( $*$ ) (Given)
      }{
         The parameter name.
      }
      \sstsubsection{
         NVAL = INTEGER (Given)
      }{
         The number of values that are to be put into the parameter.
      }
      \sstsubsection{
         VALUES( NVAL ) = ? (Given)
      }{
         The array of values to be put into the parameter.
      }
      \sstsubsection{
         STATUS = INTEGER (Given and Returned)
      }{
         The global status.
      }
   }
   \sstnotes{
      \sstitemlist{

         \sstitem
         There is a routine for each of the data types character,
         double precision, integer, logical, and real: replace {\tt "x"} in the
         routine name by C, D, I, L, or R respectively as appropriate.  The
         VALUES argument must have the corresponding data type.

         \sstitem
         A scalar (0-dimensional) parameter is different from a vector
         (1-dimensional) parameter containing a single value.

         \sstitem
         In order to obtain a storage object for the parameter, the
         current implementation of the underlying ADAM parameter system
         will proceed in the same way as it does for input parameters.
         This can result in users being prompted for `a value'. This
         behaviour, and how to avoid it, is discussed further in the
         Interface Module Reference Manual
         (\xref{SUN/115}{sun115}{parameter_specification_for_output_parameters}).

         \sstitem
         Limit checks for IN, RANGE, MIN/MAX are not applied.
      }
   }
}

\sstroutine{
   PAR\_PUTNx
}{
   Puts an array of values into a parameter
}{
   \sstdescription{
      This routine puts an $n$-dimensional array of values into a
      parameter.  If necessary, the specified array is converted to
      the type of the parameter.
   }
   \sstinvocation{
      CALL PAR\_PUTNx ( PARAM, NDIM, MAXD, VALUES, ACTD, STATUS )
   }
   \sstarguments{
      \sstsubsection{
         PARAM = CHARACTER $*$ ( $*$ ) (Given)
      }{
         The parameter name.
      }
      \sstsubsection{
         NDIM = INTEGER (Given)
      }{
         The number of dimensions of the values array.  This must match
         the number of dimensions of the object.
      }
      \sstsubsection{
         MAXD( NDIM ) = INTEGER (Given)
      }{
         The array specifying the dimensions of the array to be put.
         These may not be greater than the actual dimensions of the
         parameter (ACTD) nor those of the VALUES array.
      }
      \sstsubsection{
         VALUES( $*$ ) = ? (Given)
      }{
         The array of values to be put into the parameter.  These must
         be in Fortran order.
      }
      \sstsubsection{
         ACTD( NDIM ) = INTEGER (Given)
      }{
         The dimensions of the parameter storage to be created. These are
         unlikely to be different from MAXD.

      }
      \sstsubsection{
         STATUS = INTEGER (Given and Returned)
      }{
         The global status.
      }
   }
   \sstnotes{
      \sstitemlist{

         \sstitem
         There is a routine for each of the data types character,
         double precision, integer, logical, and real: replace {\tt "x"} in the
         routine name by C, D, I, L, or R respectively as appropriate.  The
         VALUES argument must have the corresponding data type.

         \sstitem
         In order to obtain a storage object for the parameter, the
         current implementation of the underlying ADAM parameter system
         will proceed in the same way as it does for input parameters.
         This can result in users being prompted for `a value'. This
         behaviour, and how to avoid it, is discussed further in the
         Interface Module Reference Manual
         (\xref{SUN/115}{sun115}{parameter_specification_for_output_parameters}).

         \sstitem
         Limit checks for IN, RANGE, MIN/MAX are not applied.
      }
   }
}

\sstroutine{
   PAR\_PUTVx
}{
   Puts an array of values into a parameter as if the parameter were
   a vector
}{
   \sstdescription{
      This routine puts a 1-dimensional array of primitive values into
      a parameter as it if the parameter were vectorized ({\it i.e.}\
      regardless of its actual dimensionality).  If necessary, the
      specified array is converted to the type of the parameter.
   }
   \sstinvocation{
      CALL PAR\_PUTVx( PARAM, NVAL, VALUES, STATUS )
   }
   \sstarguments{
      \sstsubsection{
         PARAM = CHARACTER $*$ ( $*$ ) (Given)
      }{
         The parameter name.
      }
      \sstsubsection{
         NVAL = INTEGER (Given)
      }{
         The number of values that are to be put into the parameter.
         It must match the actual parameter's size.
      }
      \sstsubsection{
         VALUES( NVAL ) = ? (Given)
      }{
         The values to be put into the parameter.
      }
      \sstsubsection{
         STATUS = INTEGER (Given and Returned)
      }{
         The global status.
      }
   }
   \sstnotes{
      \sstitemlist{

         \sstitem
         There is a routine for each of the data types character,
         double precision, integer, logical, and real: replace {\tt "x"} in the
         routine name by C, D, I, L, or R respectively as appropriate.  The
         VALUES argument must have the corresponding data type.

         \sstitem
         In order to obtain a storage object for the parameter, the
         current implementation of the underlying ADAM parameter system
         will proceed in the same way as it does for input parameters.
         This can result in users being prompted for `a value'. This
         behaviour, and how to avoid it, is discussed further in the
         Interface Module Reference Manual
         (\xref{SUN/115}{sun115}{parameter_specification_for_output_parameters}).

         \sstitem
         Limit checks for IN, RANGE, MIN/MAX are not applied.
      }
   }
}

\sstroutine{
   PAR\_STATE
}{
   Inquires the state of a parameter
}{
   \sstdescription{
      This routine returns the current state of the indicated parameter.
      The possible states are GROUND, ACTIVE, CANCELLED and ANNULLED.
   }
   \sstinvocation{
      CALL PAR\_STATE( PARAM, STATE, STATUS )
   }
   \sstarguments{
      \sstsubsection{
         PARAM = CHARACTER $*$ ( $*$ ) (Given)
      }{
         The parameter name.
      }
      \sstsubsection{
         STATE = INTEGER (Returned)
      }{
         The current state value of the parameter.
      }
      \sstsubsection{
         STATUS = INTEGER (Given and Returned)
      }{
         The global status.
      }
   }
   \sstnotes{
      The symbolic names for these state values are as follows.
      PAR\_\_GROUND is the ground state, PAR\_\_ACTIVE is the active
      state, PAR\_\_CANCEL is the cancelled state, and PAR\_\_NULLST is
      the null state.  These are defined in the Fortran INCLUDE file
      'PAR\_PAR'.
   }
}

\sstroutine{
   PAR\_UNSET
}{
   Cancels various parameter control values.
}{
   \sstdescription{
      This routine cancels one or more control values of a parameter.
      These are currently the parameter's dynamic default value (set by
      PAR\_DEFnx or DAT\_DEF), its minimum value (set by PAR\_MINx), its
      maximum value (set by PAR\_MAXx), or its prompt string (set by
      PAR\_PROMT).

      The routine will operate regardless of the given STATUS value and will
      not report or set STATUS if the specified values have not been set or
      are already cancelled.
   }
   \sstinvocation{
      CALL PAR\_UNSET( PARAM, WHICH, STATUS )
   }
   \sstarguments{
      \sstsubsection{
         PARAM = CHARACTER$*$($*$) (Given)
      }{
         The name of the parameter.
      }
      \sstsubsection{
         WHICH = CHARACTER$*$($*$) (Given)
      }{
         A comma-separated list of the control values to be cancelled,
         selected from the following options:
      \begin{itemize}
            \item 'DEFAULT' to cancel the dynamic default,
            \item 'MAXIMUM' to cancel the maximum value, and
            \item 'MINIMUM' to cancel the minimum value.
      \end{itemize}
      Unambiguous abbreviations are permitted.
      }
      \sstsubsection{
         STATUS = INTEGER (Unused)
      }{
         The global status.  The routine is executed regardless of the
         import value of STATUS.  If the import value is not SAI\_\_OK,
         then it is left unchanged, even if the routine fails to
         complete.  If the import value is SAI\_\_OK on entry and the
         routine fails to complete, STATUS will be set to an
         appropriate value.
      }
   }
}

\end{document}
