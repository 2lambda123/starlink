\documentstyle[11pt]{article}
\pagestyle{myheadings}

% -----------------------------------------------------------------------------
% ? Document identification
\newcommand{\stardoccategory}  {Starlink System Note}
\newcommand{\stardocinitials}  {SSN}
\newcommand{\stardocsource}    {ssn20.3}
\newcommand{\stardocnumber}    {20.3}
\newcommand{\stardocauthors}   {R.F.Warren-Smith \& D.S.Berry}
\newcommand{\stardocdate}      {17th July 2000}
\newcommand{\stardoctitle}     {Adding Format Conversion\\
                                Facilities to the NDF\\
                                Data Access Library}
\newcommand{\stardocversion}   {Version 1.6}
\newcommand{\stardocmanual}    {Developer's Guide}
% ? End of document identification
% -----------------------------------------------------------------------------

\newcommand{\stardocname}{\stardocinitials /\stardocnumber}
\markright{\stardocname}
\setlength{\textwidth}{160mm}
\setlength{\textheight}{230mm}
\setlength{\topmargin}{-2mm}
\setlength{\oddsidemargin}{0mm}
\setlength{\evensidemargin}{0mm}
\setlength{\parindent}{0mm}
\setlength{\parskip}{\medskipamount}
\setlength{\unitlength}{1mm}

% -----------------------------------------------------------------------------
%  Hypertext definitions.
%  ======================
%  These are used by the LaTeX2HTML translator in conjunction with star2html.

%  Comment.sty: version 2.0, 19 June 1992
%  Selectively in/exclude pieces of text.
%
%  Author
%    Victor Eijkhout                                      <eijkhout@cs.utk.edu>
%    Department of Computer Science
%    University Tennessee at Knoxville
%    104 Ayres Hall
%    Knoxville, TN 37996
%    USA

%  Do not remove the %begin{latexonly} and %end{latexonly} lines (used by
%  star2html to signify raw TeX that latex2html cannot process).
%begin{latexonly}
\makeatletter
\def\makeinnocent#1{\catcode`#1=12 }
\def\csarg#1#2{\expandafter#1\csname#2\endcsname}

\def\ThrowAwayComment#1{\begingroup
    \def\CurrentComment{#1}%
    \let\do\makeinnocent \dospecials
    \makeinnocent\^^L% and whatever other special cases
    \endlinechar`\^^M \catcode`\^^M=12 \xComment}
{\catcode`\^^M=12 \endlinechar=-1 %
 \gdef\xComment#1^^M{\def\test{#1}
      \csarg\ifx{PlainEnd\CurrentComment Test}\test
          \let\html@next\endgroup
      \else \csarg\ifx{LaLaEnd\CurrentComment Test}\test
            \edef\html@next{\endgroup\noexpand\end{\CurrentComment}}
      \else \let\html@next\xComment
      \fi \fi \html@next}
}
\makeatother

\def\includecomment
 #1{\expandafter\def\csname#1\endcsname{}%
    \expandafter\def\csname end#1\endcsname{}}
\def\excludecomment
 #1{\expandafter\def\csname#1\endcsname{\ThrowAwayComment{#1}}%
    {\escapechar=-1\relax
     \csarg\xdef{PlainEnd#1Test}{\string\\end#1}%
     \csarg\xdef{LaLaEnd#1Test}{\string\\end\string\{#1\string\}}%
    }}

%  Define environments that ignore their contents.
\excludecomment{comment}
\excludecomment{rawhtml}
\excludecomment{htmlonly}

%  Hypertext commands etc. This is a condensed version of the html.sty
%  file supplied with LaTeX2HTML by: Nikos Drakos <nikos@cbl.leeds.ac.uk> &
%  Jelle van Zeijl <jvzeijl@isou17.estec.esa.nl>. The LaTeX2HTML documentation
%  should be consulted about all commands (and the environments defined above)
%  except \xref and \xlabel which are Starlink specific.

\newcommand{\htmladdnormallinkfoot}[2]{#1\footnote{#2}}
\newcommand{\htmladdnormallink}[2]{#1}
\newcommand{\htmladdimg}[1]{}
\newenvironment{latexonly}{}{}
\newcommand{\hyperref}[4]{#2\ref{#4}#3}
\newcommand{\htmlref}[2]{#1}
\newcommand{\htmlimage}[1]{}
\newcommand{\htmladdtonavigation}[1]{}
\newcommand{\latexhtml}[2]{#1}
\newcommand{\html}[1]{}

%  Starlink cross-references and labels.
\newcommand{\xref}[3]{#1}
\newcommand{\xlabel}[1]{}

%  LaTeX2HTML symbol.
\newcommand{\latextohtml}{{\bf LaTeX}{2}{\tt{HTML}}}

%  Define command to re-centre underscore for Latex and leave as normal
%  for HTML (severe problems with \_ in tabbing environments and \_\_
%  generally otherwise).
\newcommand{\latex}[1]{#1}
\newcommand{\setunderscore}{\renewcommand{\_}{{\tt\symbol{95}}}}
\latex{\setunderscore}

% -----------------------------------------------------------------------------
%  Debugging.
%  =========
%  Remove % on  the following to debug links in the HTML version using Latex.

% \newcommand{\hotlink}[2]{\fbox{\begin{tabular}[t]{@{}c@{}}#1\\\hline{\footnotesize #2}\end{tabular}}}
% \renewcommand{\htmladdnormallinkfoot}[2]{\hotlink{#1}{#2}}
% \renewcommand{\htmladdnormallink}[2]{\hotlink{#1}{#2}}
% \renewcommand{\hyperref}[4]{\hotlink{#1}{\S\ref{#4}}}
% \renewcommand{\htmlref}[2]{\hotlink{#1}{\S\ref{#2}}}
% \renewcommand{\xref}[3]{\hotlink{#1}{#2 -- #3}}
%end{latexonly}
% -----------------------------------------------------------------------------
% ? Document specific \newcommand or \newenvironment commands.
\newcommand{\file}[1]{{\tt{#1}}}
\newcommand{\st}[1]{{\em{#1}}}

\newcommand{\fitsurl}[0]{http://www.gsfc.nasa.gov/astro/fits/fits_home.html}

% ? End of document specific commands
% -----------------------------------------------------------------------------
%  Title Page.
%  ===========
\renewcommand{\thepage}{\roman{page}}
\begin{document}
\thispagestyle{empty}

%  Latex document header.
%  ======================
\begin{latexonly}
   CCLRC / {\sc Rutherford Appleton Laboratory} \hfill {\bf \stardocname}\\
   {\large Particle Physics \& Astronomy Research Council}\\
   {\large Starlink Project\\}
   {\large \stardoccategory\ \stardocnumber}
   \begin{flushright}
   \stardocauthors\\
   \stardocdate
   \end{flushright}
   \vspace{-4mm}
   \rule{\textwidth}{0.5mm}
   \vspace{5mm}
   \begin{center}
   {\Huge\bf  \stardoctitle \\ [2.5ex]}
   {\LARGE\bf \stardocversion \\ [4ex]}
   {\Huge\bf  \stardocmanual}
   \end{center}
   \vspace{15mm}

% ? Heading for abstract if used.
   \vspace{10mm}
   \begin{center}
      {\Large\bf Description}
   \end{center}
% ? End of heading for abstract.
\end{latexonly}

%  HTML documentation header.
%  ==========================
\begin{htmlonly}
   \xlabel{}
   \begin{rawhtml} <H1> \end{rawhtml}
      \stardoctitle
   \begin{rawhtml} </H1> \end{rawhtml}

% ? Add picture here if required.
% ? End of picture

   \begin{rawhtml} <P> <I> \end{rawhtml}
   \stardoccategory\ \stardocnumber \\
   \stardocauthors \\
   \stardocdate
   \begin{rawhtml} </I> </P> <H3> \end{rawhtml}
      \htmladdnormallink{CCLRC}{http://www.cclrc.ac.uk} /
      \htmladdnormallink{Rutherford Appleton Laboratory}
                        {http://www.cclrc.ac.uk/ral} \\
      \htmladdnormallink{Particle Physics \& Astronomy Research Council}
                        {http://www.pparc.ac.uk} \\
   \begin{rawhtml} </H3> <H2> \end{rawhtml}
      \htmladdnormallink{Starlink Project}{http://www.starlink.ac.uk/}
   \begin{rawhtml} </H2> \end{rawhtml}
   \htmladdnormallink{\htmladdimg{source.gif} Retrieve hardcopy}
      {http://www.starlink.ac.uk/cgi-bin/hcserver?\stardocsource}\\

%  HTML document table of contents.
%  ================================
%  Add table of contents header and a navigation button to return to this
%  point in the document (this should always go before the abstract \section).
  \label{stardoccontents}
  \begin{rawhtml}
    <HR>
    <H2>Contents</H2>
  \end{rawhtml}
  \htmladdtonavigation{\htmlref{\htmladdimg{contents_motif.gif}}
        {stardoccontents}}

% ? New section for abstract if used.
  \section{\xlabel{description}\xlabel{abstract}Description}
% ? End of new section for abstract

\end{htmlonly}

% -----------------------------------------------------------------------------
% ? Document Abstract. (if used)
%  ==================
The NDF (\st{Extensible N-Dimensional Data Format\/}) data access
library (\xref{SUN/33}{sun33}{}) provides a programming interface and
a data model for astronomical applications software. It is based on a
flexible underlying data format HDS (\xref{SUN/92}{sun92}{}) and a set
of conventions for structuring data within HDS.

This document describes facilities for extending the range of data
formats accessible via the NDF library, to include any arbitrary
``foreign'' format for which a conversion utility can be defined.
This gives NDF-based applications access to a potentially wide range
of data formatting possibilities, including data compression.

The intended readership of this document includes:
\begin{enumerate}
\item Developers working with data formats and associated conversion
utilities,
\item Programmers who anticipate using the NDF library to access foreign data,
\item Knowledgeable users who need to access data stored in new
(\st{e.g.}\ personal) formats.
\end{enumerate}
% ? End of document abstract
% -----------------------------------------------------------------------------
% ? Latex document Table of Contents (if used).
%  ===========================================
 \newpage
 \begin{latexonly}
   \null\vspace {5mm}
   \begin {center}
   \rule{80mm}{0.5mm} \\ [1ex]
   {\Large\bf \stardoctitle \\ [2.5ex]
              \stardocversion} \\ [2ex]
   \rule{80mm}{0.5mm}
   \end{center}
   \setlength{\parskip}{0mm}
   \tableofcontents
   \setlength{\parskip}{\medskipamount}
   \markright{\stardocname}
 \end{latexonly}

% ? End of Latex document table of contents
% -----------------------------------------------------------------------------
\newpage
\renewcommand{\thepage}{\arabic{page}}
\setcounter{page}{1}
\begin{latexonly}
   \null\vspace {5mm}
   \begin {center}
   \rule{80mm}{0.5mm} \\ [1ex]
   {\Large\bf \stardoctitle \\ [2.5ex]
              \stardocversion} \\ [2ex]
   \rule{80mm}{0.5mm}
   \end{center}
   \vspace{30mm}
\end{latexonly}

\section{INTRODUCTION}

One of the best things about standards is the large number you have to
choose from. ``Standard'' ways of storing data proliferate in all
areas of computing, and astronomy is no exception.

Given this, there is a natural desire to write software that can cope
with more than one data format, but this can be a major undertaking.
This document describes features provided by the NDF library
(\xref{SUN/33}{sun33}{}) that help to make it a little easier.

\subsection{\xlabel{philosophy}Philosophy}

There are two main ways of writing software that can read and write
multiple data formats. Perhaps the most obvious is to incorporate a
knowledge of the data model used by each format into the data access
library and to have it make the appropriate calls (\st{e.g.}\ to
different lower-level data access libraries) according to the format
being used. This approach is generally quite efficient, but it often
presents serious difficulties in practice.

The main problem is that the data access software rapidly becomes
extremely complex.  Usually, a single person will maintain the data
access library used by a suite of applications so, if multiple data
formats are to be supported, that person must become expert in all the
formats required. Since the resulting library will be the only means
of access to these formats, it must be very sophisticated and
anticipate every requirement (even if many of the features are, in
fact, never used). Given the number and complexity of formats in use,
the range of data models they present, and the rate at which they
change, this is a near impossible task. It is not surprising,
therefore, that few systems attempt to support more than a couple of
formats in this way.

An alternative approach is to interpose format conversion software
between the original data and the application. This is potentially
less efficient, but modern computing equipment makes this less of a
problem than it once was. The great advantage is that it decouples the
problem of format conversion from that of data access. It also splits
the provision of software for accessing each different data format
into a series of separate tasks.  This makes it possible to support a
wide variety of formats relatively easily.

\subsection{\xlabel{the_format_conversion_approach}The Format Conversion Approach}

The NDF data access library follows this latter approach by allowing
format conversion utilities to be added to it, thereby allowing it to
access a range of ``foreign'' data (\st{i.e.}\ data which are not
stored in the native NDF format). This has the following advantages:

\begin{itemize}
\item The full range of normal NDF data access operations can be
supported -- reading, writing, updating, deleting, reshaping, \st{etc.}

\item Format conversion utilities can be added to already-built
software. Thus you can add the ability to read new data formats to
standard applications, without having to re-build them.

\item Format conversion utilities can be written and added by anyone,
so the problem of understanding and accessing a range of different
formats can be shared. Particular problems can be tackled by whoever
best understands them. This makes the NDF library capable of accessing
data in formats completely unknown to its original author.

\item Because it is easy to add new format conversion utilities, they
do not always need to be very sophisticated. Instead, they can be
invented or adapted to tackle new problems as they arise.  Many of the
difficulties encountered when converting a complicated data format
into another one with a different data model can be ignored, unless
they happen to be relevant to the situation at hand.

\end{itemize}

The main reason that this approach can be used is because of the
relatively sophisticated formatting possibilities and data model
presented by the NDF library, with its in-built extensibility.  This
makes it possible to convert foreign data into NDF format and back
again without losing information, while the opposite process is not
always possible.

\subsection{\xlabel{how_format_conversion_operates}How Format Conversion Operates}

To illustrate how this system works, suppose an application which uses
the NDF library wants to access an existing NDF data structure, but
the person running it only has data available in a foreign format.
The following outlines the sequence of operations that might occur:

\begin{enumerate}

\item The NDF library will first obtain the name of the dataset to be
accessed in the normal way, \st{e.g.}\ by prompting (alternatively,
the application could obtain the name and pass it to the library, but
the two methods are equivalent here).

\item The library will check whether the data are stored in native NDF
format.  If so, it will access them directly. If not, it will next
identify which foreign format they are stored in. This is done by
inspecting the extension on the file name (for instance, the file name
\file{mydata.fit} might designate a dataset stored in
\htmladdnormallink{FITS}{\fitsurl} format).

\item The library will then look to see if a format conversion utility
has been defined to convert from the foreign format into native NDF
format. Assuming one has, it will invoke it, causing the data to be
converted and written into a scratch object in native NDF format.

\item The scratch object will then be accessed as normal. The
application need not know that it hasn't been given a normal NDF.  In
addition, all references which the application makes to the dataset
name will use the original (foreign) file name, so the user usually
need not be aware that conversion has occurred either.

\item When the dataset is released by the application, the scratch
object will be deleted (in fact, this is optional -- see
\S\ref{sect:keep}). If it has been modified, a format conversion
utility will be sought, and invoked, to perform back-conversion of the
modified data before this occurs.

\end{enumerate}

A rather similar sequence of events might occur when creating a new
dataset (\st{e.g.}\ as output from an application), except that the
format conversion stage on input would not be required.

These steps are an exact analogue of the conversions that the
\xref{NDF}{sun33}{} and \xref{HDS}{sun92}{} libraries perform
transparently whenever an application attempts to (\st{e.g.}\/)
access an integer data array as floating point, or to read data
previously written on a machine which uses a different number
representation. The only difference is that format conversion
utilities are not a permanent part of the data access software, but
are invoked as separate processes which communicate through files
rather than via memory.\footnote{With the file caching available on
modern operating systems, this distinction is actually rather
blurred.} This makes it possible to add and remove them as required.

\section{SETTING UP FOR FORMAT CONVERSION}

\subsection{\xlabel{name_your_formats}\label{sect:inputformats}Name Your Formats}

The first step in setting up the NDF library to access foreign data
formats is to define a name for each foreign format to be recognised,
and to associate a file extension with each of these names. The file
extension will be used to determine which format a file is written in.

This is done by defining the environment variable called
NDF\_FORMATS\_IN to contain a format list, such as the following:

\begin{quote}
\begin{small}
\begin{verbatim}
setenv NDF_FORMATS_IN 'FITS(.fit),FIGARO(.dst),IRAF(.imh)'
\end{verbatim}
\end{small}
\end{quote}

This is a comma-separated list of format specifications, where each
specification consists of a format name
(\st{e.g.}\ \htmladdnormallink{FITS}{\fitsurl}) with an associated
file extension (\st{e.g.}\ `.fit') in parentheses.

This list serves two purposes. First, it defines the set of formats
and associated file extensions to be recognised when accessing
input\footnote{Strictly speaking, NDF\_FORMATS\_IN defines the formats
recognised when accessing \st{pre-existing} datasets. Although it is
possible to update and write to such datasets, it is nevertheless
convenient to refer to them as ``input'' datasets.} datasets. This
means, for instance, that if a dataset name such as:

\begin{quote}
\file{run66.fit}
\end{quote}

were given to the NDF library, it would recognise it as a FITS format
file and try to carry out the appropriate conversion.

The list also defines a search order for foreign data formats. This
means that if the dataset name supplied had been simply:

\begin{quote}
\file{run66}
\end{quote}

then the NDF library would first look for a native format NDF with
this name (\st{i.e.}\ in the file \file{run66.sdf}). If this was not
found, it would then look for a file called \file{run66.fit}, then
\file{run66.dst} and then \file{run66.imh}, stopping when the first
one was found and associating the appropriate data format with it. If
none of the files existed, a ``file not found'' error would result.

Note that the ability to select sections from pre-existing NDF
datasets (see
\xref{SUN/33}{sun33}{using_subscripts_to_access_ndf_sections}) is also
available when accessing foreign data files, so that entering:

\begin{quote}
\file{run66.fit(100.0$\sim$50.0)}
\end{quote}

or

\begin{quote}
\file{run66(100.0:200.0,10:512)}
\end{quote}

would result in the same actions as above to locate a suitable file
and to convert its format, with the required section then being
extracted from the converted NDF and passed to the application.

\subsection{\xlabel{rules_and_regulations}\label{sect:rules}Rules and Regulations}

You may define up to 50 foreign formats to be recognised in this way,
and may give them any names and file extensions you like (apart from
the format name NDF and the file extension `.sdf' which are reserved
for the native NDF format). Format names are not case sensitive,
although file extensions are if that makes sense for the host file
system (\st{e.g.}\ they are case sensitive on UNIX).  File extensions
should always begin with a `.' and appear in parentheses following the
associated format name.  There is no individual limit on the length of
a format name or file extension, but the entire format list is limited
to 1024 characters.

Note that the same foreign format name and/or file extension may
appear more than once in the NDF\_FORMATS\_IN list. The first
occurrence takes precedence when searching for files. Thus, you could
associate different file extensions with the same format name to
define synonyms for file extensions.

\subsection{\xlabel{defining_conversion_commands}\label{sect:conversioncommands}Defining Conversion Commands}

For each foreign format which appears in the NDF\_FORMATS\_IN list,
you should also provide commands to perform the necessary format
conversions to and/or from the native NDF format. These commands are
also defined by means of environment variables.

Taking the \htmladdnormallink{FITS}{\fitsurl} format (above) as an
example, this means defining up to two commands -- one for converting
from FITS format to NDF format and the other for converting back
again, such as the following:

\begin{quote}
\begin{small}
\begin{verbatim}
setenv NDF_FROM_FITS 'fitsin in=^dir^name^type out=^ndf'
setenv NDF_TO_FITS   'fitsout in=^ndf out=^dir^name^type'
\end{verbatim}
\end{small}
\end{quote}

Here, the names of two environment variables have been formed by
prefixing `NDF\_FROM\_' and `NDF\_TO\_' to the foreign format name (in
upper case) and each of these variables has been set to contain a
command which performs the appropriate format conversion (in this case
by invoking two conversion utilities called ``fitsin'' and
``fitsout'', which we assume to exist).

Ideally, you would define both of these commands. However, if you only
want to support conversion in one direction, then it is quite
acceptable to omit either of them. The commands are only accessed when
the occasion to use them arises, so no error will result if they are
omitted but never used.

When needed, the conversion commands you define will be interpreted
(in a separate process) by a command interpreter appropriate to the
host operating system.\footnote{On UNIX, this will be the ``sh''
(Bourne) shell.} The commands are actually invoked by passing them to
the C run time library ``system'' function, and they may therefore use
any components of the environment which are inherited through that
interface. Typically this means that such things as the default
directory and environment variables are available to these commands.

Before the commands are invoked, the NDF library will perform token
substitution on them, in order to insert the names of the actual
datasets to be processed.  The tokens used to represent these datasets
are, in fact, \xref{\st{message tokens}}{sun104}{msg} -- identical to
those used by the MERS and EMS libraries (\xref{SUN/104}{sun104}{} and
\xref{SSN/4}{ssn4}{}) and commonly used when reporting errors and
other messages from within applications.  They are used in conversion
commands in exactly the same way (they appear in the example commands
above prefixed with the `\verb#^#' substitution character), and the
NDF library defines a set of them for this purpose, as follows:

\begin{center}
\begin{tabular}{|l|l|}
\hline
{\bf Token} & {\bf Value}\\
\hline\hline
{\bf dir}   & Directory in which the foreign file resides\\
{\bf name}  & Foreign file name (without directory or extension)\\
{\bf type}  & Foreign file extension (with leading `.')\\
{\bf vers}  & Foreign file version number (blank if not supported)\\
{\bf fxs}   & Foreign extension specifier (see \S\ref{sect:fxs} )\\
{\bf fxscl} & Clean version of {\bf fxs} (all non-alphanumeric characters
replaced by underscores)\\
{\bf fmt}   & Foreign format name (upper case)\\
{\bf ndf}   & Full name of the native NDF format copy of the dataset\\
\hline
\end{tabular}
\end{center}

Note that the EMS library, which performs substitution of these
tokens, imposes a limit of 200 characters on the resulting command. If
long file names are in use this may present a problem unless the
conversion command itself is short. Fortunately, this can always be
arranged by wrapping it up in a simple script if necessary.

\subsection{\xlabel{accessing_sub-structures_within_foreign_data_files}\label{sect:fxs}Accessing Sub-structures Within Foreign Data Files}
The native HDS format allows multiple NDFs to be stored within a single disk
file, and some foreign data formats provide somewhat similar facilities.
As a concrete example, the FITS format allows images to be stored within
{\em image extensions}, so a single FITS file may contain several images,
each of which can be thought of as a foreign format NDF. When an
NDF application is run, a specific NDF within such a FITS file can be
selected by appending a {\em foreign extension specifier} (FXS) to the
end of the file name. A foreign extension specifier consists of a string
delimited by matching square brackets. The string identifies a sub-structure
within the specified file, using syntax specific to the data format. So,
for instance, the second image extension within a FITS file called {\tt
m51.fit} could be specified using the string ``{\tt m51.fit[2]}''. Here, the
sub-string ``{\tt [2]}'' forms the foreign extension specifier, and uses the
syntax expected by the CONVERT application FITS2NDF.

The foreign extension specifier is made available to external commands
using a message token called {\bf fxs}. Since this will certainly include
square brackets (and possibly other non-alphanumeric characters), it
cannot safely be included directly within the name of a file. You may
want to do this for instance, when setting up the NDF\_KEEP\_  or
NDF\_TEMP\_ environment variables. For this reason, a ``cleaned'' version
of the foreign extension specifier is also available, in a message token
called {\bf fxscl}. This is equal to {\bf fxs} except that all
non-alphanumeric characters are replaced by underscores.

\begin{center}
{\em Note, currently the NDF library only allows foreign extension specifiers
to be given when accessing existing NDFs for read-only access. An error
will be reported if an FXS is included in the name of an NDF to be
created, or an existing NDF for which update or write access is required.}
\end{center}

\subsection{\xlabel{writing_format_conversion_utilities}\label{sect:conversionutilities}Writing Format Conversion Utilities}

In the previous section, the utilities ``fitsin'' and ``fitsout'' were
presumed to exist to perform the necessary conversions. For commonly
encountered formats, this is likely to be the case, and the CONVERT
package (\xref{SUN/55}{sun55}{}) and other likely sources of
conversion utilities should be investigated before embarking on
writing your own.  Don't forget that you can often adapt existing
utilities (including those provided by the operating system) by
combining them into a suitable script.

If you do need to write your own format conversion utilities from
scratch, then the rules that apply are very few. It should obviously
be possible to execute your utility by invoking a suitable command
which includes the names of the input and output datasets.  Your
utility will also need to be able to interpret the NDF name it
receives. This means that if you are writing a program, it should
probably use the \xref{NDF library}{sun33}{} to access the NDF data
(rather than, say, \xref{HDS}{sun92}{}, which cannot necessarily
interpret the compound data structure names that will
occur).\footnote{But also see \S\ref{sect:whichndf} for ways of
avoiding this restriction.}  For a template example of a conversion
utility that reads data from unformatted Fortran files, see
\xref{SUN/33}{sun33}{READIMG}.

As far as possible, the NDF library will attempt to ensure that the
output dataset to be written by a conversion command does not already
exist, by deleting it first if necessary (your conversion utility
should then create it). However, it may not always be wise to depend
on this. In particular, recovery from error conditions (such as failed
conversions) is likely to be more robust if conversion commands are
able to cope when their output datasets already exist.

Unless you are debugging, you should also arrange for conversion
utilities not to write to the standard output channel, as such output
will otherwise appear whenever a conversion occurs. This is not
normally wanted.

Beyond this, you have complete freedom to define and implement the
conversion you want to perform. This may have whatever side effects
you choose, so long as it results in the production of the requested
output dataset, leaves its input dataset intact and returns an
appropriate status value to the NDF library (see \S\ref{sect:errors}
for a discussion of error handling in conversion commands).

\subsection{\xlabel{defining_output_formats}\label{sect:definingoutput}Defining Output Formats}

As you might expect, you define the formats for output\footnote{As
before, we really mean \st{new} datasets here (because you could
write output to a pre-existing dataset, which is covered by the
NDF\_FORMATS\_IN list), but thinking of them as ``output'' datasets is
more convenient.} datasets in rather the same way as for input
datasets (\S\ref{sect:inputformats}), by means of a search
list. However, the way this list is used is slightly different in this
case.

The output format list is found by translating the environment
variable NDF\_FORMATS\_OUT, which might typically have a definition
such as:

\begin{quote}
\begin{small}
\begin{verbatim}
setenv NDF_FORMATS_OUT '.,FITS(.fit),FIGARO(.dst),IRAF(.imh)'
\end{verbatim}
\end{small}
\end{quote}

Ignoring, for the moment, the `.' at the start, this list defines the
names of foreign data formats which are to be recognised when creating
new datasets, and associates a file extension with each one. The
syntax and restrictions are identical to the NDF\_FORMATS\_IN list
(see \S\ref{sect:rules}).

There is no requirement for the output formats to be the same as those
used for input although, for obvious reasons, they will often be
so. You could, however, give your formats different names or file
extensions in the output list if you wanted.

The NDF library uses the same commands to perform format conversion
for output datasets as for input ones (see
\S\ref{sect:conversioncommands}), so the names of output formats
should be chosen to select the environment variable containing the
appropriate command. Note, however, that the ``NDF\_FROM\_\ldots''
command will not be used in the case of output datasets.

\subsection{\xlabel{specifying_an_output_format}Specifying an Output Format}

With the output format list above, the following could
be given to the NDF library when it is expecting the name of a new
dataset:

\begin{quote}
\file{newfile.fit}
\end{quote}

and it would recognise this as a request to write the new file in
\htmladdnormallink{FITS}{\fitsurl} format (performing the appropriate
conversion when necessary).

If the name supplied were simply:

\begin{quote}
\file{newfile}
\end{quote}

(\st{i.e.}\ if no file extension is specified), then the \st{first}
format appearing in the output format list would be used. This is
where the `.' in the earlier example (\S\ref{sect:definingoutput})
comes in, as it stands for the native NDF format. Hence, a native
format NDF would be written in this case. This is normally the
required behaviour, so having `.' at the start of the format list is
recommended.

However, if you wanted to work predominantly with a foreign format
(say you were using NDF applications with another package which could
not access NDF data directly), then you could put that format at the
start of the output format list. For example:

\begin{quote}
\begin{small}
\begin{verbatim}
setenv NDF_FORMATS_OUT 'IRAF(.imh),FITS(.fit)'
\end{verbatim}
\end{small}
\end{quote}

would cause all output files to be written in \xref{IRAF}{sun179}{}
format and to have a file extension of `.imh' by default. You could
still specify FITS format explicitly by giving a file extension of
`.fit'.

\subsection{\xlabel{propagating_data_formats}Propagating Data Formats}

The output format may also be determined according to the format of a
related input dataset. This is achieved by putting the ``wild-card''
character `$*$' at the start of the output format list, as follows:

\begin{quote}
\begin{small}
\begin{verbatim}
setenv NDF_FORMATS_OUT '*,.,FIGARO(.dst),FITS(.fit),IRAF(.imh)
\end{verbatim}
\end{small}
\end{quote}

This affects new datasets which are created as a result of the
\st{propagation} of information from an existing dataset
(\st{e.g.}\ by applications calling the routines
\xref{NDF\_PROP}{sun33}{NDF_PROP} or
\xref{NDF\_SCOPY}{sun33}{NDF_SCOPY}, as described in
\xref{SUN/33}{sun33}{component_propagation}) and for which no explicit
output file extension is given. In this instance, not only will data
values be propagated, but so also will the dataset format.  Thus, if a
\xref{FIGARO}{sun86}{} format dataset (with file extension `.dst') had
been accessed for input and propagated to create an output dataset,
then a similar FIGARO format dataset would be created (also with a
`.dst' file extension) unless an explicit output file extension were
given.\footnote{Note that in this case the \st{input} format
description is being used to create the \st{output} dataset, so it
would not strictly be necessary for the FIGARO format to appear in the
NDF\_FORMATS\_OUT list.}

Note that output datasets which are created by applications
without propagation from an existing dataset do not inherit any format
information. In this case, any `$*$' in the format list is ignored and
the normal rules apply (in the example above, a native format NDF
dataset would be created instead).

\subsection{\xlabel{resolving_naming_ambiguities}Resolving Naming Ambiguities}

Unfortunately, because the `.' (dot) character is used both to
separate a file extension from its file name and also to separate
fields in an NDF (or \xref{HDS}{sun92}{}) object name, ambiguities can
sometimes arise. For example, if the dataset name:

\begin{quote}
\file{datafile.fit}
\end{quote}

is supplied, it might mean a foreign
(\htmladdnormallink{FITS}{\fitsurl}) data file with a `.fit'
extension, or it might identify an NDF structure called \file{FIT}
residing within an HDS file called \file{datafile.sdf}.

In such cases, the NDF library always uses the former
interpretation. That is, it attempts to access (or create) a foreign
format file whenever a file extension appears to be present and
corresponds with a known foreign data format. For example, if `.xyz'
is a recognised foreign file extension, then:

\begin{quote}
\file{myfile.xyz}
\end{quote}
and
\begin{quote}
\file{my.file.xyz}
\end{quote}

are both references to foreign format files rather than HDS objects
(although they may not necessarily be valid file names on all operating
systems). Conversely:

\begin{quote}
\file{yourfile.abc}
\end{quote}

Would not identify a foreign file if `.abc' is not a recognised
foreign file extension.

On UNIX, where the file system is case sensitive, it is possible to
circumvent this behaviour by exploiting the case insensitivity of HDS
component names. For instance, if `.img' (lower case) is a recognised
foreign file extension, then the dataset name:

\begin{quote}
\file{anyfile.IMG}
\end{quote}

with `.IMG' in upper case, refers to a native format NDF (an object
called \file{IMG} contained within the HDS file \file{anyfile.sdf}).
To leave this possibility open, it is recommended that foreign file
extensions should always contain at least one lower case character.

\subsection{\xlabel{example_setting_up_a_new_format}Example: Setting Up a New Format}

The following example shows the C~shell commands that might be used on
a UNIX system to give NDF-based applications access to a new data
format. Typically, commands such as these would appear in a startup
file, perhaps packaged as part of a ``driver'' that could be installed
to give access to the data format in question:

\begin{quote}
\begin{small}
\begin{verbatim}
#  Ensure that the new format and its file extension are recognised on
#  input.
      if ($?NDF_FORMATS_IN) then
         setenv NDF_FORMATS_IN $NDF_FORMATS_IN',NEW(.new)'
      else
         setenv NDF_FORMATS_IN 'NEW(.new)'
      endif

#  Similarly, ensure they are recognised on output.
      if ($?NDF_FORMATS_OUT) then
         setenv NDF_FORMATS_OUT $NDF_FORMATS_OUT',NEW(.new)'
      else
         setenv NDF_FORMATS_OUT '.,NEW(.new)'
      endif

#  Define commands to convert from the new format to NDF format and
#  vice versa.
      setenv NDF_FROM_NEW 'new2ndf in='\'^dir^name^type\'' out='\'^ndf\'
      setenv NDF_TO_NEW   'ndf2new in='\'^ndf\'' out='\'^dir^name^type\'
\end{verbatim}
\end{small}
\end{quote}

This example illustrates a couple of points which were not addressed
earlier:

\begin{enumerate}
\item We first check to see if the NDF\_FORMATS\_IN and
NDF\_FORMATS\_OUT environment variables are already defined. If they
are, we can append our new format description to them so as not to
disturb any definitions already in use. Otherwise we must set them up
from scratch.

\item The environment variable definitions have been written so that
single quote characters appear around the names of datasets. For
example, the translation of the environment variable NDF\_FROM\_NEW
would be:
\begin{quote}
\begin{small}
\begin{verbatim}
new2ndf in='^dir^name^type' out='^ndf'
\end{verbatim}
\end{small}
\end{quote}
Although the syntax needed is a bit messy, this does mean that any
special characters that appear in dataset names will be handled
correctly (\st{i.e.}\ literally), and not expanded by the shell that
interprets the command.
\end{enumerate}

\section{ADDITIONAL FACILITIES}

\subsection{\xlabel{explicit_deletion_commands}Explicit Deletion Commands}

When accessing files containing foreign format data, the NDF library
will, on occasion, have to delete them (for instance, the routine
\xref{NDF\_DELET}{sun33}{NDF_DELET} might have been called by an
application).  Normally, this causes no problem, as a named file can
easily be deleted when necessary.  With some formats, however, this is
not so simple. For example, data written in \xref{IRAF}{sun179}{}
format will normally reside in two associated files -- although the
NDF library can delete the one it knows about, the other one would
remain in existence.

To overcome this and other similar problems, it is possible to define
an explicit deletion command for any foreign format which needs
special treatment. If one is defined, it will over-ride any attempt by
the NDF library to delete files which it knows are written in that
format.

Taking the IRAF format as an example, the command would be defined via
the environment variable NDF\_DEL\_IRAF in the same way as when
defining format conversion commands. For example:

\begin{quote}
\begin{small}
\begin{verbatim}
setenv NDF_DEL_IRAF 'rm -f ^dir^name.imh ^dir^name.pix'
\end{verbatim}
\end{small}
\end{quote}

would unsure that both files associated with the dataset (with
extensions `.imh' and `.pix') are deleted when necessary.

The deletion command is invoked in the normal way, by passing it to
the C run time library ``system'' function, having first performed
message token substitution on it (see
\S\ref{sect:conversioncommands}). In this case, the NDF library
defines the following tokens for use in the command:

\begin{center}
\begin{tabular}{|l|l|}
\hline
{\bf Token} & {\bf Value}\\
\hline\hline
{\bf dir}  & Directory in which the foreign file resides\\
{\bf name} & Foreign file name (without directory or extension)\\
{\bf type} & Foreign file extension (with leading `.')\\
{\bf vers} & Foreign file version number (blank if not supported)\\
{\bf fxs}  & Foreign extension specifier (see \S\ref{sect:fxs} )\\
{\bf fxscl} & Clean version of {\bf fxs} (all non-alphanumeric characters
replaced by underscores)\\
{\bf fmt}  & Foreign format name (upper case)\\
\hline
\end{tabular}
\end{center}

\subsection{\xlabel{retaining_converted_data}\label{sect:keep}Retaining Converted Data}

Normally, when a foreign format dataset is converted to or from the
native NDF format, the native copy of the data will be held in
temporary file space (it will usually be written to a data structure
held in the standard \xref{HDS}{sun92}{} \xref{scratch
file}{sun92}{scratch_files}), and this copy will be deleted when no
longer required. Normally, this occurs when the dataset is released by
the application.

Sometimes, however, it may be more convenient to retain the converted
copy. For example, if you have foreign format data but plan to run
several NDF-based applications on it, then retaining the native NDF
copy the first time it is converted will save you having to re-convert
the data on subsequent occasions.

To do this, the NDF library `KEEP' tuning parameter should be set to
1. This can be done by calling the \xref{NDF\_TUNE}{sun33}{NDF_TUNE}
routine from within an application, but can also be done by setting
the environment variable NDF\_KEEP to `1' outside the application (see
\xref{SUN/33}{sun33}{tuning_the_ndf_system}), for example:

\begin{quote}
\begin{small}
\begin{verbatim}
setenv NDF_KEEP 1
\end{verbatim}
\end{small}
\end{quote}

If this is done, then subsequent access to a foreign format dataset
(say \file{galaxy.fit}) will create a corresponding native format NDF
copy of the data in the default directory (in this case in a file
called \file{galaxy.sdf}). This will be retained, and will then be
accessed the next time \file{galaxy} is specified as a dataset name
(remember, if no file extension is given, there is always an implicit
search for a native format dataset before looking for a foreign one).

The `KEEP' tuning parameter may be changed at any time, so control
over individual datasets is possible if it is set from within an
application. The value used will be that in effect when the dataset is
first accessed.

\subsection{\xlabel{specifying_where_the_native_ndf_is_stored}\label{sect:whichndf}Specifying Where the Native NDF is Stored}

If the native NDF copy of a foreign dataset is not being kept, then
the NDF library will, by default, store it within the
\xref{HDS}{sun92}{} \xref{scratch file}{sun92}{scratch_files}, as
described earlier. This is generally most efficient. However, not all
conversion utilities will necessarily be able to access such an NDF,
particularly if they know nothing of the NDF or HDS data formats
themselves. This would be the case with a general purpose data
compression utility, for instance.

The NDF library therefore allows you to specify where the native NDF
copy of the data should be stored. This is done by defining
environment variables containing message tokens that evaluate to the
name you want this NDF to have.

Up to two such environment variables may be defined for each foreign
format. Their names are generated by prefixing `NDF\_KEEP\_' and
`NDF\_TEMP\_' to the foreign format name (in upper case), and they
correspond to the two cases (a) where the native copy of the NDF is
being kept, and (b) where it is not. The two cases are handled
separately, but the message tokens available are the same in both
instances, as follows:

\begin{center}
\begin{tabular}{|l|l|}
\hline
{\bf Token} & {\bf Value}\\
\hline\hline
{\bf dir}  & Directory in which the foreign file resides\\
{\bf name} & Foreign file name (without directory or extension)\\
{\bf type} & Foreign file extension (with leading `.')\\
{\bf vers} & Foreign file version number (blank if not supported)\\
{\bf fxs}  & Foreign extension specifier (see \S\ref{sect:fxs} )\\
{\bf fxscl} & Clean version of {\bf fxs} (all non-alphanumeric characters
replaced by underscores)\\
{\bf fmt}  & Foreign format name (upper case)\\
\hline
\end{tabular}
\end{center}

For example, if we were defining a COMPRESSED format and wanted to
ensure that the native NDF data was always stored in its own file in
the default directory, so that the UNIX ``compress'' utility could
access it, we might use:

\begin{quote}
\begin{small}
\begin{verbatim}
setenv NDF_KEEP_COMPRESSED ^name
setenv NDF_TEMP_COMPRESSED tmp_^name
\end{verbatim}
\end{small}
\end{quote}

Then, whatever the setting of the `KEEP' tuning parameter, the name of
an NDF in the default directory would be generated, so the HDS scratch
file would never be used to hold the NDF copy of data from a
COMPRESSED dataset.  Note that specifying an explicit NDF name in this
way does not affect whether the native NDF copy is deleted when the
dataset is released. This is still determined by the `KEEP' tuning
parameter (see \S\ref{sect:keep}).

If, in the above example, the foreign dataset were called
\file{/home/me/data/nebula.sdf.Z} (and the `KEEP' tuning parameter was
not set), then the NDF name would be \file{tmp\_nebula} and this name
would be passed (as the value of the `\verb#^#ndf' message token) to
any conversion commands that needed to be invoked. The NDF itself
would reside in an HDS file called \file{tmp\_nebula.sdf} (the `.sdf'
extension being added automatically by HDS).

Note that the value given for the NDF\_KEEP\_COMPRESSED environment
variable above is, in fact, the same as its default.  You should
generally be wary of setting this to anything except its default value
because the user of an application might well be confused if he sets
the NDF\_KEEP environment variable to specify that the NDF should be
kept, but it ends up with an unexpected name. This facility does,
however, give control over which directory is used to store the file.

\subsection{\xlabel{efficiency_considerations}Efficiency Considerations}

When deciding where to store the native NDF format copy of a dataset,
it is wise to specify a location on a local file system wherever
possible. This is, of course, always good practice where large
datasets are concerned, as access to remote files is usually far less
efficient and can generate considerable network traffic that may
interfere with other people's work.

With NDF format conversion facilities, local file access is even more
important. This is because the temporary datasets involved are always
read immediately after being written, and very frequently deleted
immediately after that. In this situation, an operating system with
good file caching will often not actually write the data to a local
file at all, but merely copy it to and from memory. This is far faster
than waiting for actual data transfer to take place, which is what
will normally happen if remote files are involved.

For this reason, you are recommended to configure format conversion
software so that temporary datasets are stored in the user's default
directory, in the expectation that this directory, at least, will be
chosen sensibly and reside on a local file system. Users may, however,
still need to be reminded of the need for this (\st{e.g.}\ in
documentation). You may also need to explain how to change this
behaviour if, for example, access to larger amounts of space for
temporary files becomes necessary.

Note that, by default, temporary NDF datasets are stored in the
standard \xref{HDS}{sun92}{} scratch file, which resides in a
directory specified by the \xref{HDS\_SCRATCH}{sun92}{scratch_files}
environment variable. If this variable is not explicitly set, the
user's default directory is used.

\subsection{\xlabel{example_data_compression}\label{sect:datacompression}Example: Data Compression}

To illustrate the above, the following is a complete example of the
C~shell commands that might be used to allow access to compressed NDF
data files (with file extension `.sdf.Z') on UNIX systems:

\begin{quote}
\begin{small}
\begin{verbatim}
#  Define the COMPRESSED format to be recognised on input, with file
#  extension `.sdf.Z'.
      if ($?NDF_FORMATS_IN) then
         setenv NDF_FORMATS_IN $NDF_FORMATS_IN',COMPRESSED(.sdf.Z)'
      else
         setenv NDF_FORMATS_IN 'COMPRESSED(.sdf.Z)'
      endif

#  Similarly, recognise it on output.
      if ($?NDF_FORMATS_OUT) then
         setenv NDF_FORMATS_OUT $NDF_FORMATS_OUT',COMPRESSED(.sdf.Z)'
      else
         setenv NDF_FORMATS_OUT '.,COMPRESSED(.sdf.Z)'
      endif

#  Store the uncompressed data in the default directory.
      setenv NDF_KEEP_COMPRESSED ^name
      setenv NDF_TEMP_COMPRESSED tmp_^name

#  Use the "uncompress" and "compress" utilities to convert the data.
      setenv NDF_FROM_COMPRESSED 'uncompress -c -f ^dir^name^type >^ndf.sdf'
      setenv NDF_TO_COMPRESSED 'compress -c -f ^ndf.sdf >^dir^name^type;:'

#  Suppress processing of extension information for compressed data.
      setenv NDF_XTN_COMPRESSED ''
\end{verbatim}
\end{small}
\end{quote}

Note that the ``compress'' command has been followed by a null ``:''
command which does nothing. This is because an error status may be
returned if the file being compressed does not get any smaller, so the
``:'' command ensures that the invoking NDF application always
receives a success status. In a production system, more secure error
handling than this would probably be required.

For an explanation of the final definition of the NDF\_XTN\_COMPRESSED
environment variable, you should refer ahead to
\S\ref{sect:extensionlist}.

({\bf{Warning:}} \st{Users should be warned that it is unwise to
archive compressed data unless thay are sure that the necessary
decompression software will be available to them in future, possibly
on different hardware and/or operating system platforms.})

\subsection{\xlabel{handling_errors_in_conversion_commands}\label{sect:errors}Handling Errors in Conversion Commands}

When a command associated with access to foreign data completes, the
NDF library checks to determine if it was successful.

It first looks at the status value returned by the C ``system'' call
which invoked the command. These status values are operating-system
dependent but, on most systems, there is provision for the command to
return either a ``success'' or an ``error'' status to the command
interpreter and for the invoking application to receive this.  If the
NDF library does not receive a success status back, it deduces that
the command has failed and generates an appropriate error report. The
NDF\_ routine that was invoked then returns to the application with
its STATUS argument set and the application would probably then abort
and display the error message.

For commands that invoke conversion utilities (associated with either
of the environment variables NDF\_FROM\_\ldots\ or NDF\_TO\_\ldots),
the NDF library will also check to see that the output dataset from
the conversion operation has been created. Where this dataset is a
native format NDF, it will be opened to check that it contains a valid
NDF data structure. Any problem will again result in an error report
from the invoking application.

When writing data access commands or conversion utilities, the
recommended course of action if an error occurs is for diagnostic
error information to be written to the standard error channel, and for
the invoked command to return with an ``error'' status value
appropriate to the command interpreter in use.

It is generally wise to avoid having the conversion utility re-prompt
for new input, as this can be confusing for the user who may not be
aware that conversion is taking place. This can normally be arranged
by appropriately redirecting the standard input and/or output channels
to a null device so that the command will abort and control will
return to the NDF library if an attempt is made to prompt for (or
read) new input.

\subsection{\xlabel{avoiding_unwanted_recursion}Avoiding Unwanted Recursion}

When writing format conversion utilities, it is often convenient to
use the NDF library to access the native NDF format version of the
data (see \S\ref{sect:conversionutilities}). However, you should bear
in mind that the NDF library's ability to invoke format conversion
commands will still be active unless you take action to switch it
off. This means that unwanted recursion is possible if a conversion
utility accesses a foreign dataset that in turn causes a further
conversion utility to be invoked, and so on\ldots

In practice, this is unlikely to be a problem if care is taken to
ensure that NDF datasets are never stored in objects whose names might
be mistaken for foreign format data files. If it does prove necessary
to suppress unwanted format conversion, however, this can be achieved
by setting the NDF\_ library's \xref{DOCVT tuning
parameter}{sun33}{tuning_the_ndf_system} to zero.  This will have the
effect of disabling recognition of foreign data files by the
conversion utility.

One way of doing this is by setting the environment variable
NDF\_DOCVT to 0 as part of the format conversion command, immediately
before the conversion utility itself is invoked. Alternatively, the
conversion utility may call the \xref{NDF\_TUNE}{sun33}{NDF_TUNE}
routine itself in order to control recognition of foreign data
formats. The latter approach allows individual control over each
dataset accessed by the utility if necessary.

\subsection{\xlabel{debugging_conversion_commands}Debugging Conversion Commands}

Normally, all foreign data access commands invoked by the NDF library
execute silently, unless an error occurs or a command writes
information to standard output (this should normally be avoided). To
assist in debugging, however, the NDF library provides a tuning
parameter `SHCVT'. This can be used to make it display all commands
before they are executed but after message token substitution has
taken place.

To enable this feature, the `SHCVT' tuning parameter should be set to
1. This can be done from within an application by calling the
\xref{NDF\_TUNE}{sun33}{NDF_TUNE} routine (see
\xref{SUN/33}{sun33}{tuning_the_ndf_system}), or from outside the
application by setting the NDF\_SHCVT environment variable, as
follows:

\begin{quote}
\begin{small}
\begin{verbatim}
setenv NDF_SHCVT 1
\end{verbatim}
\end{small}
\end{quote}

\section{DATA EXTENSIBILITY}

\subsection{\xlabel{general_principles_of_extensibility}General Principles}

A notable feature of the NDF data format is its extensibility, which
is achieved by means of independent
\st{extensions\footnotemark}\footnotetext{Be careful to distinguish an
``extension'' to an NDF data structure (which is an addition of extra
data to the file) from the ``file extension'' (which is the end part
of the file name, such as `.sdf', used to identify the file's
format).}\ to the format, which can be defined and added to suit the
needs of individual software authors.  A key distinction between these
extensions and the other contents of an NDF dataset is that the
meaning and processing rules for data held in extensions are generally
unknown to writers of format conversion utilities, whereas the
standard components of an NDF have well-defined and universal meanings
(see \xref{SUN/33}{sun33}{extensibility}).

This has important implications. It means, for instance, that it is
relatively straightforward to write a general purpose utility to
change (say) \xref{IRAF}{sun179}{} format into NDF format, so long as
only standard NDF components need to be considered. However, if the
receiving NDF application is equipped to handle data in its own NDF
extension, then converting that additional data
(\st{i.e.}\ extracting it from the IRAF file and putting it into the
NDF extension) will require specialist knowledge, and so cannot be
expected of a general purpose utility.

What is required is for conversion utilities to be extensible in the
same way as the NDF datasets themselves. A standard utility could then
be used to convert the bulk of the data, and a more specialised
utility could simply add the extension information to the converted
dataset.

As will be explained below, the NDF library supports this concept, but
there still remains one problem. If, for example, you had written a
software package and an associated utility that extracted specialist
extension information from IRAF datasets, you would probably not want
to repeat this work for every other possible data format that might
come along in future -- you would surely prefer to use the same
specialist utility to access a whole range of foreign formats. This is
where the NDF's \htmlref{\st{FITS extension}}{sect:fitsextension}
comes in.

\subsection{\xlabel{the_fits_extension}\label{sect:fitsextension}The FITS Extension}
An important feature of the well-known
\htmladdnormallink{FITS}{\fitsurl}
\footnote{FITS stands for Flexible Image Transport System.} data
format (which was originally designed as a convenient container for
the interchange of astronomical images between sites) is its ``FITS
header''. This, in essence, is a sequence of character strings each of
which contains the name of a keyword, an associated value and
(optionally) a comment.

Although rather few of the keywords that appear in a FITS header have
standardised meanings, the freedom that this gives makes it a
convenient place to store information about which the reader or writer
may have little knowledge.  A special NDF extension mirroring the
properties of a FITS header can therefore provide a useful ``airlock''
or ``staging post'' for interchanging specialist information between
general purpose conversion utilities (for which the information is
meaningless) and specialist utilities (for which it has meaning).

To satisfy this requirement a FITS extension, equipped to hold FITS
header information, may be added to an NDF. By convention, it consists
of a 1-dimensional (\xref{HDS}{sun92}{}) array of \_CHAR$*$80
character strings which holds a sequence of header records according
to FITS formatting rules (including the final `END' record).

\subsection{\xlabel{extension_import_and_export_operations}Extension Import and Export Operations}

To illustrate the function that the \htmlref{FITS
extension}{sect:fitsextension} performs, consider the following
sequence of events in which an \xref{IRAF}{sun179}{} format file is
read by an application that expects to find a \xref{CCDPACK}{sun139}{}
extension present:

\begin{enumerate}
\item Having detected that it needs to convert the data format, the
NDF library first invokes a general purpose conversion command, as
defined in the NDF\_FROM\_IRAF environment variable (see
\S\ref{sect:conversioncommands}). This, in turn, invokes a conversion
utility which creates the NDF data structure and fills in all the
relevant standard components using information obtained from the IRAF
dataset.

\item The same utility then assembles all ancillary information that
it doesn't recognise (essentially the contents of the IRAF header
file) and writes it to a FITS extension, which it creates in the new
NDF. It then terminates.

\item The NDF library then invokes a specialist utility, written by
the designer of the CCDPACK extension. This inspects the FITS header
(and other standard components of the NDF if necessary) and transfers
the information it recognises into the CCDPACK extension, which it
creates. It too, then terminates.

\item Other specialist utilities may then be invoked, if required, to
create further extensions using information in the FITS header.

\item Finally, the original application regains control and accesses
the NDF dataset that has been built.
\end{enumerate}

When writing to a foreign dataset, the sequence of events is broadly
similar, except that the specialist utilities are invoked first
(before the general purpose one) and transfer information from their
relevant extensions {\bf into} the FITS extension. The general purpose
conversion utility then transfers the contents of the FITS extension
to the foreign dataset as part of its conversion task.

The processes of (a) creating a specialist extension from information
stored in the FITS extension and (b) writing specialist extension
information back into the FITS extension are referred to as
\st{importing} and \st{exporting} the extension information.

Using this scheme, utilities that import and export extension
information will, in many circumstances, be able to rely entirely on
the contents of the FITS extension and need not access the foreign
data file at all. This relieves their authors of the need to
understand the foreign format, beyond knowing what FITS keywords will
be used to store the information of interest.  Import and export
utilities are therefore easily re-used when new formats are
encountered.  Indeed, since FITS keywords are so widely used, there
will often be conventions in place that make even a change of keywords
unnecessary when adding a new format.

The following sections now describe the stages involved in setting up
import and export utilities to make use of this scheme.

\subsection{\xlabel{defining_the_extension_list}\label{sect:extensionlist}Defining the Extension List}

It is first necessary to define the set of specialist NDF extensions
that should be recognised. This is normally done via the environment
variable NDF\_XTN, as follows:

\begin{quote}
\begin{small}
\begin{verbatim}
setenv NDF_XTN CCDPACK,IRAS90
\end{verbatim}
\end{small}
\end{quote}

This is simply a comma-separated list of extension names conforming to
the naming conventions described in
\xref{SUN/33}{sun33}{extension_names_and_software_packages}. It
applies to all foreign format datasets, unless overridden. The order
in which extensions occur in this list determines the order in which
they will be imported. They will be exported in the reverse order.

On occasion, it may be necessary to use a different list of NDF
extensions for a particular foreign format. Most commonly, this
involves simply using an empty list for formats that do not require
any extension handling (data compression of ordinary NDF data files
would be an example -- see \S\ref{sect:datacompression}). To specify a
separate extension list for a particular foreign format, an
environment variable is used whose name is constructed by prefixing
`NDF\_XTN\_' to the format name (in upper case). For example:

\begin{quote}
\begin{small}
\begin{verbatim}
setenv NDF_XTN_COMPRESSED ''
\end{verbatim}
\end{small}
\end{quote}

would over-ride the normal extension list with an empty one so that no
extension handling would occur when COMPRESSED format data is
accessed.

\subsection{\xlabel{extension_import_and_export_commands}Extension Import and Export Commands}

The commands that perform import and export of extension data are
defined in the usual way via environment variables whose names are
formed by prefixing `NDF\_IMP\_' or `NDF\_EXP\_' to the extension name
(in upper case), for example:

\begin{quote}
\begin{small}
\begin{verbatim}
setenv NDF_IMP_CCDPACK 'impccd ndf=^ndf'
setenv NDF_EXP_CCDPACK 'expccd ndf=^ndf'
\end{verbatim}
\end{small}
\end{quote}

Here, the extension name is \xref{CCDPACK}{sun139}{} (and should
appear in the NDF\_XTN list) and the ``impccd'' and ``expccd''
utilities are assumed to have been written to import and export
information for this extension.

The commands are invoked after message token substitution has taken
place, as described in \S\ref{sect:conversioncommands}. In this case,
the set of tokens defined for use is as follows:

\begin{center}
\begin{tabular}{|l|l|}
\hline
{\bf Token} & {\bf Value}\\
\hline\hline
{\bf dir}  & Directory in which the foreign file resides\\
{\bf name} & Foreign file name (without directory or extension)\\
{\bf type} & Foreign file extension (with leading `.')\\
{\bf vers} & Foreign file version number (blank if not supported)\\
{\bf fxs}  & Foreign extension specifier (see \S\ref{sect:fxs} )\\
{\bf fxscl} & Clean version of {\bf fxs} (all non-alphanumeric characters
replaced by underscores)\\
{\bf fmt}  & Foreign format name (upper case)\\
{\bf ndf}  & Full name of the native NDF format copy of the dataset\\
{\bf xtn}  & Name of the NDF extension (upper case)\\
\hline
\end{tabular}
\end{center}

As explained earlier, the foreign format file should not normally be
accessed by import and export utilities unless that is unavoidable, so
one set of import and export commands will normally suffice for
accessing a whole range of foreign formats.

In special cases, however, where techniques specific to a particular
format are needed, an alternative set of commands may be defined to
apply to that format alone.  This is done via environment variables
whose names are constructed by appending an underscore and the foreign
format name to the usual names shown above. For instance, when
importing and exporting CCDPACK extension information to
\xref{FIGARO}{sun86}{} files, one might want to use:

\begin{quote}
\begin{small}
\begin{verbatim}
setenv NDF_IMP_CCDPACK_FIGARO 'impccd ndf=^ndf file=^dir^name^type'
setenv NDF_EXP_CCDPACK_FIGARO 'expccd ndf=^ndf file=^dir^name^type'
\end{verbatim}
\end{small}
\end{quote}

If these variables were defined, they would over-ride any defined
without the `\_FIGARO' suffix when accessing that particular format.
Any special techniques can therefore be restricted to those formats
that require them.

\subsection{\xlabel{writing_import_and_export_utilities}Writing Import and Export Utilities}

Before writing your own import and export utilities, you should
consider using standard ones that already exist. For example, the
KAPPA package (\xref{SUN/95}{sun95}{}) contains a general purpose
``\xref{fitsimp}{sun95}{FITSIMP}'' command that can be used to build a
specialist NDF extension by importing information from a \htmlref{FITS
extension}{sect:fitsextension}. It is driven by a keyword translation
table stored in a text file, so can easily be adapted for different
needs.  For example, it might be used in an NDF import command as
follows:

\begin{quote}
\begin{small}
\begin{verbatim}
setenv NDF_IMP_MINE 'fitsimp ndf=^ndf xname=MINE table=$HOME/mine.imp'
\end{verbatim}
\end{small}
\end{quote}

Here, \file{mine.imp} is the table that drives the importation
process. This could be different for each format if necessary. An
equivalent extension export utility ``\xref{fitsexp}{sun95}{FITSEXP}''
is also available.

If you find that you must write your own software for this purpose,
then the IMG library (\xref{SUN/160}{sun160}{}) provides a convenient
programming interface for accessing items of NDF extension information
(including individual items within the FITS extension) and should make
most import and export utilities straightforward to write. With a
little more effort, you can, of course, also use the
\xref{NDF}{sun33}{} and \xref{HDS}{sun92}{} libraries, which allow you
to construct any form of extension you want.

\subsection{\xlabel{example_setting_up_an_extension}Example: Setting Up an Extension}

The following example shows typical C~shell commands that might be used
to allow NDF applications to handle specialist extension information
derived from foreign format datasets. Normally such commands would
form part of the startup sequence for the package that utilised the
extension.

\begin{quote}
\begin{small}
\begin{verbatim}
#  Append the CCDPACK extension to the list of extensions to be
#  handled.
      if ($?NDF_XTN) then
         setenv NDF_XTN $NDF_XTN,CCDPACK
      else
         setenv NDF_XTN CCDPACK
      endif

#  Define commands for importing and exporting CCDPACK extension
#  information.
      setenv NDF_IMP_CCDPACK 'ccdimp ndf=^ndf table=$CCDPACK_DIR/^fmt.imp'
      setenv NDF_EXP_CCDPACK 'ccdexp ndf=^ndf table=$CCDPACK_DIR/^fmt.exp'
\end{verbatim}
\end{small}
\end{quote}

Note that we have specified keyword translation tables here (for use
in the import and export commands) which depend on the foreign data
format being accessed. This would be necessary if, for instance, data
in different formats were to follow different conventions about how
its header information is stored, so that different
\htmladdnormallink{FITS}{\fitsurl} keywords were used in the
\htmlref{FITS extension}{sect:fitsextension} as a result. By
concentrating this information in a table, it becomes easy to change
and users can even have their own versions if necessary.

Applications which process the converted data need only deal with the
validated information stored within their own private extension. They
are therefore insulated from details of the conversion process and any
need to change in order to access new data formats in future. The use
of a private extension also protects them from the possibility of
other software inadvertently corrupting their private data.

\newpage
\appendix

\section{ENVIRONMENT VARIABLES AND TOKENS}

\subsection{\xlabel{summary_of_environment_variables_used}Summary of Environment Variables Used}

The following list summarises the environment variables used by the
NDF library to control access to foreign format data.  The symbols
$<$FMT$>$ and $<$XTN$>$ are used to represent the names of foreign
data formats and NDF extensions respectively.  All environment
variable names should be entirely in upper case.

\begin{description}

\item[NDF\_FORMATS\_IN]\mbox{}\\
Comma-separated list of format names and associated file extensions,
used when accessing pre-existing datasets. (Mandatory if pre-existing
foreign datasets are to be accessed.)

\item[NDF\_FORMATS\_OUT]\mbox{}\\
Comma-separated list of format names and associated file extensions,
used when accessing new datasets. (Mandatory if new foreign datasets
are to be accessed other than by propagation of format information
from a pre-existing dataset.)

\item[NDF\_FROM\_$<$FMT$>$]\mbox{}\\
Command to convert a foreign format dataset into an NDF. (Mandatory if
datasets in that format are to be read.)

\item[NDF\_TO\_$<$FMT$>$]\mbox{}\\
Command to convert an NDF into a foreign format dataset. (Mandatory if
datasets in that format are to be written or modified.)

\item[NDF\_DEL\_$<$FMT$>$]\mbox{}\\
Command to delete a foreign format dataset. (Optional.)

\item[NDF\_KEEP\_$<$FMT$>$]\mbox{}\\
Name of the NDF to be used to hold a native format copy of a foreign
dataset in cases where the NDF is to be retained for future
use. (Optional.)

\item[NDF\_TEMP\_$<$FMT$>$]\mbox{}\\
Name of the NDF to be used to hold a native format copy of a foreign
dataset in cases where the NDF is temporary and will be deleted when
the dataset is released. (Optional.)

\item[NDF\_XTN]\mbox{}\\
Comma-separated list of NDF extensions to be recognised when accessing
foreign datasets. (Optionally required if extension information is to
be handled.)

\item[NDF\_XTN\_$<$XTN$>$]\mbox{}\\
Comma-separated list of NDF extensions to be recognised when accessing
datasets in a particular foreign format. (Optionally overrides the
value of NDF\_XTN for that format.)

\item[NDF\_IMP\_$<$XTN$>$]\mbox{}\\
Command to import information into an NDF extension. (Optionally
required to allow reading of extension information from foreign
datasets.)

\item[NDF\_IMP\_$<$XTN$>$\_$<$FMT$>$]\mbox{}\\
Command to import information into an NDF extension when accessing
datasets in a particular foreign format. (Optionally overrides the
value of NDF\_IMP\_$<$XTN$>$ for that format.)

\item[NDF\_EXP\_$<$XTN$>$]\mbox{}\\
Command to export information from an NDF extension. (Optionally
required to allow writing or modification of extension information in
foreign datasets.)

\item[NDF\_EXP\_$<$XTN$>$\_$<$FMT$>$]\mbox{}\\
Command to export information from an NDF extension when accessing
datasets in a particular foreign format. (Optionally overrides the
value of NDF\_EXP\_$<$XTN$>$ for that format.)

\end{description}

\subsection{\xlabel{summary_of_message_tokens_used}Summary of Message Tokens Used}

The following table summarises all the message tokens used by the NDF
library in format conversion commands, \st{etc:}

\begin{center}
\begin{tabular}{|l|l|}
\hline
{\bf Token} & {\bf Value}\\
\hline\hline
{\bf dir}  & Directory in which the foreign file resides\\
{\bf name} & Foreign file name (without directory or extension)\\
{\bf type} & Foreign file extension (with leading `.')\\
{\bf vers} & Foreign file version number (blank if not supported)\\
{\bf fxs}  & Foreign extension specifier (see \S\ref{sect:fxs} )\\
{\bf fxscl} & Clean version of {\bf fxs} (all non-alphanumeric characters
replaced by underscores)\\
{\bf fmt}  & Foreign format name (upper case)\\
{\bf ndf}  & Full name of the native NDF format copy of the dataset\\
{\bf xtn}  & Name of the NDF extension (upper case)\\
\hline
\end{tabular}
\end{center}

Each environment variable which is used to control access to foreign
format data, and which permits substitution of the above message
tokens, is of the form:

\begin{quote}
NDF\_$<$CTX$>$\_\ldots
\end{quote}

where $<$CTX$>$ is the name of a ``context'' within which message
token substitution takes place. The following table shows which tokens
are defined for use within each context:

\newcommand{\yes}[0]{$\surd$}
\newcommand{\no}[0]{$\times$}
\begin{center}
\begin{tabular}{r|l|ccccccccc|}
\multicolumn{2}{c}{}&\multicolumn{9}{c}{\st{Token}}\\
\cline{3-11}
\multicolumn{2}{c|}{}&{\bf dir}&{\bf name}&{\bf type}&{\bf vers}
&{\bf fxs}&{\bf fxscl}&{\bf fmt}&{\bf ndf}&{\bf xtn}\\
\cline{2-11}
&{\bf FROM} & \yes & \yes & \yes & \yes & \yes & \yes & \yes & \yes & \no \\
&{\bf TO}   & \yes & \yes & \yes & \yes & \yes & \yes & \yes & \yes & \no \\
&{\bf DEL}  & \yes & \yes & \yes & \yes & \yes & \yes & \yes & \no  & \no \\
\st{Context} &{\bf KEEP} & \yes & \yes & \yes & \yes & \yes & \yes & \yes & \no  & \no \\
&{\bf TEMP} & \yes & \yes & \yes & \yes & \yes & \yes & \yes & \no & \no \\
&{\bf IMP}  & \yes & \yes & \yes & \yes & \yes & \yes & \yes & \yes & \yes \\
&{\bf EXP}  & \yes & \yes & \yes & \yes & \yes & \yes & \yes & \yes & \yes \\
\cline{2-11}
\end{tabular}
\end{center}

\end{document}
