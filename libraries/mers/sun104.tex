\documentclass[twoside,11pt]{article}

% ? Specify used packages
% \usepackage{graphicx}        %  Use this one for final production.
% \usepackage[draft]{graphicx} %  Use this one for drafting.
% ? End of specify used packages

\pagestyle{myheadings}

% -----------------------------------------------------------------------------
% ? Document identification
% Fixed part
\newcommand{\stardoccategory}  {Starlink User Note}
\newcommand{\stardocinitials}  {SUN}
\newcommand{\stardocsource}    {sun\stardocnumber}

% Variable part - replace [xxx] as appropriate.
\newcommand{\stardocnumber}    {104.11}
\newcommand{\stardocauthors}   {P C T Rees \\
                                A J Chipperfield}
\newcommand{\stardocdate}      {22 October 2001}
\newcommand{\stardoctitle}     {MERS (MSG and ERR)\\ [1ex]
                                Message and Error Reporting Systems}
\newcommand{\stardocversion}   {Version 1.7}
\newcommand{\stardocmanual}    {Programmer's Manual}
\newcommand{\stardocabstract}  {
This document describes two Fortran subroutine libraries, MSG and ERR, which 
can be used to provide informational text to the user from any application 
program.
It also describes a preliminary C interface for the two libraries --
this may be subject to change in the light of experience.
\par
The Message Reporting System, MSG, is used for reporting non-error information
and the Error Reporting System, ERR, is used specifically for reporting error
messages.
The conventions for reporting errors from Starlink software are also
discussed in detail.
\par
This document is recommended reading for anyone writing applications
software for use on Starlink.
}
% ? End of document identification
% -----------------------------------------------------------------------------

% +
%  Name:
%     sun.tex
%
%  Purpose:
%     Template for Starlink User Note (SUN) documents.
%     Refer to SUN/199
%
%  Authors:
%     AJC: A.J.Chipperfield (Starlink, RAL)
%     BLY: M.J.Bly (Starlink, RAL)
%     PWD: Peter W. Draper (Starlink, Durham University)
%
%  History:
%     17-JAN-1996 (AJC):
%        Original with hypertext macros, based on MDL plain originals.
%     16-JUN-1997 (BLY):
%        Adapted for LaTeX2e.
%        Added picture commands.
%     13-AUG-1998 (PWD):
%        Converted for use with LaTeX2HTML version 98.2 and
%        Star2HTML version 1.3.
%     {Add further history here}
%
% -

\newcommand{\stardocname}{\stardocinitials /\stardocnumber}
\markboth{\stardocname}{\stardocname}
\setlength{\textwidth}{160mm}
\setlength{\textheight}{230mm}
\setlength{\topmargin}{-2mm}
\setlength{\oddsidemargin}{0mm}
\setlength{\evensidemargin}{0mm}
\setlength{\parindent}{0mm}
\setlength{\parskip}{\medskipamount}
\setlength{\unitlength}{1mm}

% -----------------------------------------------------------------------------
%  Hypertext definitions.
%  ======================
%  These are used by the LaTeX2HTML translator in conjunction with star2html.

%  Comment.sty: version 2.0, 19 June 1992
%  Selectively in/exclude pieces of text.
%
%  Author
%    Victor Eijkhout                                      <eijkhout@cs.utk.edu>
%    Department of Computer Science
%    University Tennessee at Knoxville
%    104 Ayres Hall
%    Knoxville, TN 37996
%    USA

%  Do not remove the %begin{latexonly} and %end{latexonly} lines (used by 
%  LaTeX2HTML to signify text it shouldn't process).
%begin{latexonly}
\makeatletter
\def\makeinnocent#1{\catcode`#1=12 }
\def\csarg#1#2{\expandafter#1\csname#2\endcsname}

\def\ThrowAwayComment#1{\begingroup
    \def\CurrentComment{#1}%
    \let\do\makeinnocent \dospecials
    \makeinnocent\^^L% and whatever other special cases
    \endlinechar`\^^M \catcode`\^^M=12 \xComment}
{\catcode`\^^M=12 \endlinechar=-1 %
 \gdef\xComment#1^^M{\def\test{#1}
      \csarg\ifx{PlainEnd\CurrentComment Test}\test
          \let\html@next\endgroup
      \else \csarg\ifx{LaLaEnd\CurrentComment Test}\test
            \edef\html@next{\endgroup\noexpand\end{\CurrentComment}}
      \else \let\html@next\xComment
      \fi \fi \html@next}
}
\makeatother

\def\includecomment
 #1{\expandafter\def\csname#1\endcsname{}%
    \expandafter\def\csname end#1\endcsname{}}
\def\excludecomment
 #1{\expandafter\def\csname#1\endcsname{\ThrowAwayComment{#1}}%
    {\escapechar=-1\relax
     \csarg\xdef{PlainEnd#1Test}{\string\\end#1}%
     \csarg\xdef{LaLaEnd#1Test}{\string\\end\string\{#1\string\}}%
    }}

%  Define environments that ignore their contents.
\excludecomment{comment}
\excludecomment{rawhtml}
\excludecomment{htmlonly}

%  Hypertext commands etc. This is a condensed version of the html.sty
%  file supplied with LaTeX2HTML by: Nikos Drakos <nikos@cbl.leeds.ac.uk> &
%  Jelle van Zeijl <jvzeijl@isou17.estec.esa.nl>. The LaTeX2HTML documentation
%  should be consulted about all commands (and the environments defined above)
%  except \xref and \xlabel which are Starlink specific.

\newcommand{\htmladdnormallinkfoot}[2]{#1\footnote{#2}}
\newcommand{\htmladdnormallink}[2]{#1}
\newcommand{\htmladdimg}[1]{}
\newcommand{\hyperref}[4]{#2\ref{#4}#3}
\newcommand{\htmlref}[2]{#1}
\newcommand{\htmlimage}[1]{}
\newcommand{\htmladdtonavigation}[1]{}

\newenvironment{latexonly}{}{}
\newcommand{\latex}[1]{#1}
\newcommand{\html}[1]{}
\newcommand{\latexhtml}[2]{#1}
\newcommand{\HTMLcode}[2][]{}

%  Starlink cross-references and labels.
\newcommand{\xref}[3]{#1}
\newcommand{\xlabel}[1]{}

%  LaTeX2HTML symbol.
\newcommand{\latextohtml}{\LaTeX2\texttt{HTML}}

%  Define command to re-centre underscore for Latex and leave as normal
%  for HTML (severe problems with \_ in tabbing environments and \_\_
%  generally otherwise).
\renewcommand{\_}{\texttt{\symbol{95}}}

% -----------------------------------------------------------------------------
%  Debugging.
%  =========
%  Remove % on the following to debug links in the HTML version using Latex.

% \newcommand{\hotlink}[2]{\fbox{\begin{tabular}[t]{@{}c@{}}#1\\\hline{\footnotesize #2}\end{tabular}}}
% \renewcommand{\htmladdnormallinkfoot}[2]{\hotlink{#1}{#2}}
% \renewcommand{\htmladdnormallink}[2]{\hotlink{#1}{#2}}
% \renewcommand{\hyperref}[4]{\hotlink{#1}{\S\ref{#4}}}
% \renewcommand{\htmlref}[2]{\hotlink{#1}{\S\ref{#2}}}
% \renewcommand{\xref}[3]{\hotlink{#1}{#2 -- #3}}
%end{latexonly}
% -----------------------------------------------------------------------------
% ? Document specific \newcommand or \newenvironment commands.
% Add any document specific \newcommand or \newenvironment commands here
\newcommand{\dash}{--}
\begin{htmlonly}
  \newcommand{\dash}{-}
\end{htmlonly}
\newcommand{\listline}{\hspace{1pt}\\}
\begin{htmlonly}
  \newcommand{\listline}{}
\end{htmlonly}
%+
%  Name:
%     SST.TEX

%  Purpose:
%     Define LaTeX commands for laying out Starlink routine descriptions.

%  Language:
%     LaTeX

%  Type of Module:
%     LaTeX data file.

%  Description:
%     This file defines LaTeX commands which allow routine documentation
%     produced by the SST application PROLAT to be processed by LaTeX and
%     by LaTeX2html. The contents of this file should be included in the
%     source prior to any statements that make of the sst commnds.

%  Notes:
%     The style file html.sty provided with LaTeX2html needs to be used.
%     This must be before this file.

%  Authors:
%     RFWS: R.F. Warren-Smith (STARLINK)
%     PDRAPER: P.W. Draper (Starlink - Durham University)

%  History:
%     10-SEP-1990 (RFWS):
%        Original version.
%     10-SEP-1990 (RFWS):
%        Added the implementation status section.
%     12-SEP-1990 (RFWS):
%        Added support for the usage section and adjusted various spacings.
%     8-DEC-1994 (PDRAPER):
%        Added support for simplified formatting using LaTeX2html.
%     {enter_further_changes_here}

%  Bugs:
%     {note_any_bugs_here}

%-

%  Define length variables.
\newlength{\sstbannerlength}
\newlength{\sstcaptionlength}
\newlength{\sstexampleslength}
\newlength{\sstexampleswidth}

%  Define a \tt font of the required size.
\latex{\newfont{\ssttt}{cmtt10 scaled 1095}}
\html{\newcommand{\ssttt}{\tt}}

%  Define a command to produce a routine header, including its name,
%  a purpose description and the rest of the routine's documentation.
\newcommand{\sstroutine}[3]{
   \goodbreak
   \rule{\textwidth}{0.5mm}
   \vspace{-7ex}
   \newline
   \settowidth{\sstbannerlength}{{\Large {\bf #1}}}
   \setlength{\sstcaptionlength}{\textwidth}
   \setlength{\sstexampleslength}{\textwidth}
   \addtolength{\sstbannerlength}{0.5em}
   \addtolength{\sstcaptionlength}{-2.0\sstbannerlength}
   \addtolength{\sstcaptionlength}{-5.0pt}
   \settowidth{\sstexampleswidth}{{\bf Examples:}}
   \addtolength{\sstexampleslength}{-\sstexampleswidth}
   \parbox[t]{\sstbannerlength}{\flushleft{\Large {\bf #1}}}
   \parbox[t]{\sstcaptionlength}{\center{\Large #2}}
   \parbox[t]{\sstbannerlength}{\flushright{\Large {\bf #1}}}
   \begin{description}
      #3
   \end{description}
}
\newcommand{\sstsubroutine}[3]{\sstroutine{#1}{#2}{#3}}

%  Format the description section.
\newcommand{\sstdescription}[1]{\item[Description:] #1}

%  Format the usage section.
\newcommand{\sstusage}[1]{\item[Usage:] \mbox{}
\\[1.3ex]{\raggedright \ssttt #1}}

%  Format the invocation section.
\newcommand{\sstinvocation}[1]{\item[Invocation:]\hspace{0.4em}{\tt #1}}

%  Format the arguments section.
\newcommand{\sstarguments}[1]{
   \item[Arguments:] \mbox{} \\
   \vspace{-3.5ex}
   \begin{description}
      #1
   \end{description}
}

%  Format the returned value section (for a function).
\newcommand{\sstreturnedvalue}[1]{
   \item[Returned Value:] \mbox{} \\
   \vspace{-3.5ex}
   \begin{description}
      #1
   \end{description}
}

%  Format the parameters section (for an application).
\newcommand{\sstparameters}[1]{
   \item[Parameters:] \mbox{} \\
   \vspace{-3.5ex}
   \begin{description}
      #1
   \end{description}
}

%  Format the examples section.
\newcommand{\sstexamples}[1]{
   \item[Examples:] \mbox{} \\
   \vspace{-3.5ex}
   \begin{description}
      #1
   \end{description}
}

%  Define the format of a subsection in a normal section.
\newcommand{\sstsubsection}[1]{ \item[{#1}] \mbox{} \\}

%  Define the format of a subsection in the examples section.
\newcommand{\sstexamplesubsection}[2]{\sloppy
\item[\parbox{\sstexampleslength}{\ssttt #1}] \mbox{} \vspace{1.0ex}
\\ #2 }

%  Format the notes section.
\newcommand{\sstnotes}[1]{\item[Notes:] \mbox{} \\[1.3ex] #1}

%  Provide a general-purpose format for additional (DIY) sections.
\newcommand{\sstdiytopic}[2]{\item[{\hspace{-0.35em}#1\hspace{-0.35em}:}]
\mbox{} \\[1.3ex] #2}

%  Format the implementation status section.
\newcommand{\sstimplementationstatus}[1]{
   \item[{Implementation Status:}] \mbox{} \\[1.3ex] #1}

%  Format the bugs section.
\newcommand{\sstbugs}[1]{\item[Bugs:] #1}

%  Format a list of items while in paragraph mode.
\newcommand{\sstitemlist}[1]{
  \mbox{} \\
  \vspace{-3.5ex}
  \begin{itemize}
     #1
  \end{itemize}
}

%  Define the format of an item.
\newcommand{\sstitem}{\item}

%% Now define html equivalents of those already set. These are used by
%  latex2html and are defined in the html.sty files.
\begin{htmlonly}

%  sstroutine.
   \newcommand{\sstroutine}[3]{
      \subsection{#1\xlabel{#1}-\label{#1}#2}
      \begin{description}
         #3
      \end{description}
   }

%  sstsubroutine.
%  Same as sstroutine but at subsubsection level
   \newcommand{\sstsubroutine}[3]{
      \subsubsection{#1\xlabel{#1}-\label{#1}#2}
      \begin{description}
         #3
      \end{description}
   }

%  sstdescription
   \newcommand{\sstdescription}[1]{\item[Description:]
      \begin{description}
         #1
      \end{description}
      \\
   }

%  sstusage
   \newcommand{\sstusage}[1]{\item[Usage:]
      \begin{description}
         {\ssttt #1}
      \end{description}
      \\
   }

%  sstinvocation
   \newcommand{\sstinvocation}[1]{\item[Invocation:]
      \begin{description}
         {\ssttt #1}
      \end{description}
      \\
   }

%  sstarguments
   \newcommand{\sstarguments}[1]{
      \item[Arguments:] \\
      \begin{description}
         #1
      \end{description}
      \\
   }

%  sstreturnedvalue
   \newcommand{\sstreturnedvalue}[1]{
      \item[Returned Value:] \\
      \begin{description}
         #1
      \end{description}
      \\
   }

%  sstparameters
   \newcommand{\sstparameters}[1]{
      \item[Parameters:] \\
      \begin{description}
         #1
      \end{description}
      \\
   }

%  sstexamples
   \newcommand{\sstexamples}[1]{
      \item[Examples:] \\
      \begin{description}
         #1
      \end{description}
      \\
   }

%  sstsubsection
   \newcommand{\sstsubsection}[1]{\item[{#1}]}

%  sstexamplesubsection
   \newcommand{\sstexamplesubsection}[2]{\item[{\ssttt #1}] #2}

%  sstnotes
   \newcommand{\sstnotes}[1]{\item[Notes:] #1 }

%  sstdiytopic
   \newcommand{\sstdiytopic}[2]{\item[{#1}] #2 }

%  sstimplementationstatus
   \newcommand{\sstimplementationstatus}[1]{
      \item[Implementation Status:] #1
   }

%  sstitemlist
   \newcommand{\sstitemlist}[1]{
      \begin{itemize}
         #1
      \end{itemize}
      \\
   }
%  sstitem
   \newcommand{\sstitem}{\item}

\end{htmlonly}

%  End of "sst.tex" layout definitions.
%.
% ? End of document specific commands
% -----------------------------------------------------------------------------
%  Title Page.
%  ===========
\renewcommand{\thepage}{\roman{page}}
\begin{document}
\thispagestyle{empty}

%  Latex document header.
%  ======================
\begin{latexonly}
   CCLRC / \textsc{Rutherford Appleton Laboratory} \hfill \textbf{\stardocname}\\
   {\large Particle Physics \& Astronomy Research Council}\\
   {\large Starlink Project\\}
   {\large \stardoccategory\ \stardocnumber}
   \begin{flushright}
   \stardocauthors\\
   \stardocdate
   \end{flushright}
   \vspace{-4mm}
   \rule{\textwidth}{0.5mm}
   \vspace{5mm}
   \begin{center}
   {\Huge\textbf{\stardoctitle \\ [2.5ex]}}
   {\LARGE\textbf{\stardocversion \\ [4ex]}}
   {\Huge\textbf{\stardocmanual}}
   \end{center}
   \vspace{5mm}

% ? Add picture here if required for the LaTeX version.
%   e.g. \includegraphics[scale=0.3]{filename.ps}
% ? End of picture

% ? Heading for abstract if used.
   \vspace{10mm}
   \begin{center}
      {\Large\textbf{Abstract}}
   \end{center}
% ? End of heading for abstract.
\end{latexonly}

%  HTML documentation header.
%  ==========================
\begin{htmlonly}
   \xlabel{}
   \begin{rawhtml} <H1 ALIGN=CENTER> \end{rawhtml}
      \stardoctitle\\
      \stardocversion\\
      \stardocmanual
   \begin{rawhtml} </H1> <HR> \end{rawhtml}

% ? Add picture here if required for the hypertext version.
%   e.g. \includegraphics[scale=0.7]{filename.ps}
% ? End of picture

   \begin{rawhtml} <P> <I> \end{rawhtml}
   \stardoccategory\ \stardocnumber \\
   \stardocauthors \\
   \stardocdate
   \begin{rawhtml} </I> </P> <H3> \end{rawhtml}
      \htmladdnormallink{CCLRC}{http://www.cclrc.ac.uk} /
      \htmladdnormallink{Rutherford Appleton Laboratory}
                        {http://www.cclrc.ac.uk/ral} \\
      \htmladdnormallink{Particle Physics \& Astronomy Research Council}
                        {http://www.pparc.ac.uk} \\
   \begin{rawhtml} </H3> <H2> \end{rawhtml}
      \htmladdnormallink{Starlink Project}{http://www.starlink.rl.ac.uk/}
   \begin{rawhtml} </H2> \end{rawhtml}
   \htmladdnormallink{\htmladdimg{source.gif} Retrieve hardcopy}
      {http://www.starlink.rl.ac.uk/cgi-bin/hcserver?\stardocsource}\\

%  HTML document table of contents. 
%  ================================
%  Add table of contents header and a navigation button to return to this 
%  point in the document (this should always go before the abstract \section). 
  \label{stardoccontents}
  \begin{rawhtml} 
    <HR>
    <H2>Contents</H2>
  \end{rawhtml}
  \htmladdtonavigation{\htmlref{\htmladdimg{contents_motif.gif}}
        {stardoccontents}}

% ? New section for abstract if used.
  \section{\xlabel{abstract}Abstract}
% ? End of new section for abstract
\end{htmlonly}

% -----------------------------------------------------------------------------
% ? Document Abstract. (if used)
%  ==================
\stardocabstract
% ? End of document abstract
% -----------------------------------------------------------------------------
% ? Latex document Table of Contents (if used).
%  ===========================================
  \newpage
  \begin{latexonly}
    \setlength{\parskip}{0mm}
    \tableofcontents
    \setlength{\parskip}{\medskipamount}
    \markboth{\stardocname}{\stardocname}
  \end{latexonly}
% ? End of Latex document table of contents
% -----------------------------------------------------------------------------
\cleardoublepage
\renewcommand{\thepage}{\arabic{page}}
\setcounter{page}{1}

% ? Main text

\section{\xlabel{introduction}Introduction}

There is a general need for application programs on Starlink to provide the
user with informative textual messages about:

\begin {itemize}
\item What they do -- for example, during long operations it is helpful if the
user is kept informed of what a program is doing.

\item What results have been obtained -- for example, the notification of  the
final results from a procedure, or of some intermediate results that would help
the user respond to further prompts. 

\item What errors have occurred -- for example, errors which lead to the user
being prompted to provide more sensible input to a program, or fatal
errors which cause an application to stop.
\end {itemize}

This document describes two subroutine libraries which can be used to provide
informational text to the user from an application program.
The two packages are:

\begin {description}
\item [MSG] Message Reporting System, used for reporting non-error information. 
\item [ERR] Error Reporting System, used specifically for reporting error
messages.
\end {description}

This document is aimed at all programmers who are writing applications software
on Starlink, either for use within the 
\xref{ADAM environment}{sg4}{}\latex{ (see SG/4)}
or stand-alone.
The major part of this document concerns the use of the stand-alone version of
the Message and Error Reporting System subroutines.
Its use in ADAM applications is essentially the same but there are some
\htmlref
{additional features}{adam_sect}\latex{ which are described in Appendix \ref{adam_sect}}.

\section{\xlabel{msg_message_reporting_system}MSG -- Message Reporting System \xlabel{msg}}

\subsection{\xlabel{overview}Overview}

The most obvious way of producing informational messages from within 
Fortran application programs is through formatted WRITE and PRINT 
statements.
However:
\begin{itemize}
\item It is generally considered a good idea in a large system to direct
all output through a single routine to improve portability and to make
re-direction of output easier.
\item Some environments, such as ADAM, require output via a `user-interface'
program and not direct to the terminal.
\item It is sometimes difficult to format numerical output in its most
concise form within textual messages.
To do this in-line each time a message is sent to the user would be very
inconvenient and justifies the provision of a dedicated set of subroutines. 
\end{itemize} 
These considerations have led to the design and implementation of a set of
subroutines which form the Message Reporting System.
The Message System subroutines have names of the form

\begin {quote}
\begin {small}
\begin{verbatim}
MSG_name
\end{verbatim}
\end {small}
\end {quote}

where \texttt{name} indicates what the subroutine does.


\subsection{\xlabel{reporting_messages}Reporting messages}

The primary message reporting subroutine is MSG\_OUT. 
It has a calling sequence of the form:

\begin {small}
\begin{verbatim}
      CALL MSG_OUT( PARAM, TEXT, STATUS )
\end{verbatim}
\end {small}

where the argument PARAM is a character string giving the name of the message,
TEXT is a character string giving the message text, and STATUS is the
integer subroutine status value. 
MSG\_OUT sends the message string, TEXT, to the standard output stream which
will normally be the user's terminal.

The subroutine MSG\_OUT uses the Starlink convention of 
\htmlref{inherited status}{inherited_status}. 
This means that calls to MSG\_OUT will not output the message unless the 
given value of STATUS is equal to SAI\_\_OK.
If an error is encountered within the subroutine, then STATUS is returned
set to an error value. 
(The global constant SAI\_\_OK is defined in the include file SAE\_PAR. 
\latex{The use of this global constant and of inherited status are discussed in detail
in \S\ref{inherited_status}.}) 

The maximum length for an output message is 300 characters.
If it exceeds this length, then the message is truncated with
an ellipsis, \textit{i.e.}\ ``\ldots'', but no error will result.

By default, messages are split so that output lines do not exceed 
79 characters -- the split is made on word boundaries if possible. 
The maximum output line size can be altered using
\htmlref{tuning parameters}{tuning}\latex{ (see \S\ref{tuning})}.

It is recommended that,
within the application, the message name, PARAM, should be a unique identifier 
for the message string, TEXT. 
However, the message name serves no useful purpose within a stand-alone 
application and is often given as a blank string.
In ADAM applications the message name has a
\htmlref
{specific purpose}{mespar_sect}\latex{ which is discussed in detail in Appendix \ref{adam_msg}}.

Here is an example of using MSG\_OUT:
\begin {small}
\begin{verbatim}
      CALL MSG_OUT( 'EXAMPLE_MSGOUT', 'An example of MSG_OUT.', STATUS )
\end{verbatim}
\end {small}
It is sometimes useful to intersperse blank lines amongst lines of textual
output for clarity.
This could be done using calls to MSG\_OUT, \textit{e.g.}
\begin {small}
\begin{verbatim}
      CALL MSG_OUT( 'MSG_BLANK', ' ', STATUS )
\end{verbatim}
\end {small}
For convenience, the subroutine MSG\_BLANK has been provided for this 
purpose, \textit{e.g.}
\begin {small}
\begin{verbatim}
      CALL MSG_BLANK( STATUS )
\end{verbatim}
\end {small}
The status argument in MSG\_BLANK behaves in the same way as the status 
argument in MSG\_OUT.


\subsection{\xlabel{conditional_message_reporting}Conditional message reporting}
It is sometimes useful to have varying levels of message output which 
may be controlled by the user of an application.
Instances where this facility might be of use are:
\begin {itemize}
\item the ability to switch on informational messages for novice users
which can later be switched off as the user becomes familiar with 
the application;
\item the ability to switch off unnecessary informational messages when
an application is run in batch mode, or within a command procedure;
\item the ability to output detailed information from an application on 
request, say from within an iterative procedure.
\end {itemize}
Conditional message output is achieved explicitly in the Message Reporting
System using the subroutine MSG\_OUTIF to assign a ``priority'' to the
message, \textit{e.g.}
\begin {small}
\begin{verbatim}
      CALL MSG_OUTIF( MSG__NORM, ' ', 'A conditional message', STATUS )
\end{verbatim}
\end {small}
Here, the first argument is the ``priority'' associated with the 
message and can be any one of three levels which are 
represented by symbolic constants defined in the include file 
\htmlref
{MSG\_PAR}{compiling_and_linking}\latex{ (see \S\ref{compiling_and_linking})}:
\begin {quote}
\begin {description}
\item [MSG\_\_QUIET] -- quiet mode, high priority;
\item [MSG\_\_NORM] -- normal mode, normal priority (default);
\item [MSG\_\_VERB] -- verbose mode, low priority.
\item [MSG\_\_DEBUG] -- debug mode, lowest priority
\end {description}
\end {quote}
Whether or not the message will be output depends upon the ``conditional
message output filter'' which may be set using the subroutine MSG\_IFSET.
\textit{e.g.}
\begin {small}
\begin{verbatim}
      CALL MSG_IFSET( MSG__QUIET, STATUS )
\end{verbatim}
\end {small}
The first argument of MSG\_IFSET is the required conditional output filter
level -- it may take the same values as the message priority; by default it
is set to MSG\_\_NORM.
The current conditional output filtering level may be inquired using
subroutine MSG\_IFLEV.

See also
\htmlref{MSG\_TUNE}{MSG_TUNE}
which allows the filter level to be set by an environment variable, and
\htmlref{MSG\_IFGET}{MSG_IFGET}
which allows ADAM programs to obtain the filter level from an ADAM parameter.

The action of MSG\_OUTIF resulting from each of the defined priority values is 
as follows:
\begin {quote}
\begin {description}
\item [MSG\_\_QUIET] -- always output given the message, regardless of output
filter setting;
\item [MSG\_\_NORM] -- output the given message if the current output filter is
set to either MSG\_\_NORM, MSG\_\_VERB or MSG\_\_DEBUG;
\item [MSG\_\_VERB] -- output the given message only if the current output
filter is set to MSG\_\_VERB or MSG\_\_DEBUG.
\item [MSG\_\_VERB] -- output the given message only if the current output
filter is set to MSG\_\_DEBUG.

\end {description}
\end {quote}
In this scheme, messages given the priority MSG\_\_QUIET could never be 
turned off and would therefore be the most most important messages being
output by an application.

Here is an example of how conditional message output might be used 
in an application using interactive graphics with differing levels of 
informational messages to match how familiar the user is with the application:
\begin {small}
\begin{verbatim}
*  Use the cursor to enter the approximate positions of stars on the 
*  displayed image to be fitted.
      CALL MSG_OUT( ' ', 'Use the cursor to enter star positions', 
     :             STATUS )

*  Explain the positioning of the cursor.
      CALL MSG_OUTIF( MSG__NORM, ' ', 
     :               'The graphics cursor should be positioned ' //
     :               'close to the centre of each star image', STATUS )

*  Explain the cursor keys to new user.
      CALL MSG_OUTIF( MSG__VERB, ' ', 'Cursor keys: 1 add entry', STATUS )
      CALL MSG_OUTIF( MSG__VERB, ' ', '             2 reject last entry', 
     :               STATUS )
      CALL MSG_OUTIF( MSG__VERB, ' ', '             3 entry complete', 
     :               STATUS )
\end{verbatim}
\end {small}

\subsection{\xlabel{conditional_message_reporting_and_msg_out}Conditional Message Reporting and MSG\_OUT}

Although conditional message reporting may be handled explicitly by calling
MSG\_OUTIF, calls to MSG\_OUT and MSG\_BLANK also have conditional message
reporting built in.
Their output has priority MSG\_\_NORM associated with it and will therefore
only be output when the conditional output filter is set to MSG\_\_NORM (the 
default) or MSG\_\_VERB.

\subsection{\xlabel{message_tokens}Message tokens}
In the previous examples of MSG\_OUT and MSG\_OUTIF, the message text is 
``constant'' in the sense that it does not refer to any variable items, 
\textit{e.g.}\ file names or numeric values. 
However, very often, applications need to include the values of Fortran 
variables within output messages. 
This is done in the Message System using tokens embedded within the message
text.
For example, a program which measures the intensity of an emission line in
a spectrum can output its result by:

\begin {small}
\begin{verbatim}
      CALL MSG_SETR( 'FLUX', FLUX )
      CALL MSG_OUT( 'EXAMPLE_RESULT', 
     :             'Emission flux is ^FLUX (erg/cm2/A/s).',
     :             STATUS )
\end{verbatim}
\end {small}

Here, the subroutine MSG\_SETR is called to define a token named ``FLUX''
and assign to it the value of the REAL variable FLUX encoded as a character 
string.
The token name, immediately preceded by the up-arrow, ``$^\wedge$'', escape 
character is then included in the given message text. 
As the given text is processed, the token is expanded to the string assigned to
the token.

For example, If the variable FLUX in this example has the value 2.4, then
the message output to the terminal would be:

\begin {quote}
\begin {small}
\begin{verbatim}
Emission flux is 2.4 (erg/cm2/A/s).
\end{verbatim}
\end {small}
\end {quote}

There is a set of MSG\_SET\textit{x} subroutines, one subroutine for each of five
standard Fortran 77 data types (the Fortran type COMPLEX has not been provided
for).
Here, \textit{x} corresponds to the Fortran data type of the value to be assigned
to the named message token:

\begin {center}
\begin {tabular}{||c|c||}
\hline
\textit{x} & \textit{Fortran Type}\\
\hline
D & DOUBLE PRECISION\\
R & REAL\\
I & INTEGER\\
L & LOGICAL\\
C & CHARACTER\\
\hline
\end {tabular}
\end {center}

In each case, the calling sequence is of the form:

\begin {small}
\begin{verbatim}
      CALL MSG_SETx( TOKEN, VALUE )
\end{verbatim}
\end {small}

where TOKEN is a character string giving the name chosen by the user and VALUE
is a variable or constant of the appropriate type.
The numeric subroutines, MSG\_SETD, MSG\_SETR and MSG\_SETI, adopt the most
concise format that will represent the value by removing trailing zeros,
leading and trailing blanks, and by avoiding the use of exponential notation
unless it is necessary. 
MSG\_SETL uses \texttt{TRUE} or \texttt{FALSE} according to the value it is
given. MSG\_SETC removes trailing blanks from the character string; leading
blanks are not removed.

An additional feature of the MSG\_SET\textit{x} routines is that calls to these 
routines using an existing token will result in the value being appended 
to the previously assigned token string.
Here is the previous example written to exploit this feature of the 
MSG\_SET\textit{x} routines:

\begin {small}
\begin{verbatim}
*  Local Constants:
      CHARACTER FUNITS * 12
      PARAMETER( FUNITS = ' erg/cm2/A/s' )

      ...

      CALL MSG_SETR( 'FLUX', FLUX )
      CALL MSG_SETC( 'FLUX', FUNITS )
      CALL MSG_OUT( 'EXAMPLE_RESULT', 
     :             'Emission flux is ^FLUX.', STATUS )
\end{verbatim}
\end {small}

where the CHARACTER variable FUNITS has been assigned the value of an 
appropriate unit of flux (\textit{e.g.}\ erg/cm2/A/s) earlier in the program.
Note that repeated calls to the MSG\_SET\textit{x} routines will append values 
to the token string with no separator, hence a leading space in the FUNITS 
string is needed to separate the flux value and its units in the expanded 
message.

The text string associated with a message token (\textit{i.e.}\ the token value)
may be up to 200 characters long.
Token names may be up to 15 characters long and should be valid names:
\textit{i.e.}\ they should begin with an alphabetic character and continue with
alphanumeric or underscore characters.
A maximum of 64 uniquely named message tokens may be included in any output
message. 

No message tokens are defined initially and
after each call to MSG\_OUT, MSG\_OUTIF or MSG\_LOAD (or 
\htmlref{a corresponding ERR 
routine}{report_sect}\latex{, see \S\ref{report_sect}}) 
all existing tokens are left undefined. 

\subsection{\xlabel{formatted_tokens}Formatted tokens}
The MSG\_SET\textit{x} subroutines encode numeric values in the most concise 
format that will describe the supplied value. 
This is normally what is wanted for a simple message. 
Sometimes, however, a specific format is required; for example, the message
may form part of a table. 
More precise numeric formats could be achieved using the Fortran WRITE
statement with the Message System, \textit{e.g.}

\begin {small}
\begin{verbatim}
*  Local Constants:
      CHARACTER * 12 FUNITS
      PARAMETER( FUNITS = ' erg/cm2/A/s' )

*  Local Variables:
      CHARACTER * 12 VALUE

      ...

*  Output the flux value.
      WRITE( VALUE, '( 1E12.5 )' ) FLUX
      CALL MSG_SETC( 'FLUX', VALUE )
      CALL MSG_SETC( 'FLUX', FUNITS )
      CALL MSG_OUT( 'EXAMPLE_RESULT', 
     :             'Emission flux is ^FLUX.', STATUS )
\end{verbatim}
\end {small}

which would produce the message:

\begin {quote}
\begin {small}
\begin{verbatim}
Emission flux is  2.40000E+00 (erg/cm2/A/s).
\end{verbatim}
\end {small}
\end {quote}

In this case, the first call to MSG\_SETC assigns the supplied character 
string VALUE to the named token ``FLUX''.

Since this sequence of a formatted internal WRITE followed by a call to
MSG\_SETC is of general use, it is provided in a set of subroutines of the
form:

\begin {small}
\begin{verbatim}
      CALL MSG_FMTx( TOKEN, FORMAT, VALUE )
\end{verbatim}
\end {small}

where FORMAT is a valid Fortran 77 format string which can be used to
encode the supplied value, VALUE.
As for MSG\_SET\textit{x}, \textit{x} corresponds to one of the five standard
Fortran data types -- \textit{D}, \textit{R}, \textit{I}, \textit{L} and \textit{C}.

Using MSG\_FMT\textit{x}, the example given above can be performed by:

\begin {small}
\begin{verbatim}
*  Output the flux value.
      CALL MSG_FMTR( 'FLUX', '1E12.5', FLUX )
      CALL MSG_SETC( 'FLUX', FUNITS )
      CALL MSG_OUT( 'EXAMPLE_RESULT', 
     :             'Emission flux is ^FLUX.', STATUS )
\end{verbatim}
\end {small}

The use of the MSG\_FMT\textit{x} routines along with their ability to append 
values of any type to existing tokens is a very powerful tool for constructing 
tabular output from applications software.


\subsection{\xlabel{including_escape_characters_in_messages}Including escape characters in messages}
Sometimes it is necessary to include the message token escape character,
``$^\wedge$'', literally in a message. 
When the message token escape character is immediately followed by a blank
space, or is at the end of the MSG\_OUT text, it is included literally. 
If this is not the case, then it can be included literally by duplicating it.
So, for example:

\begin {small}
\begin{verbatim}
      CALL MSG_SETC( 'TOKEN', 'message token' )
      CALL MSG_OUT( 'EXAM_UPARROW', 
     :             'Up-arrow, ^^, is the ^TOKEN escape character.', 
     :             STATUS )
\end{verbatim}
\end {small}

would produce the message:

\begin {quote}
\begin {small}
\begin{verbatim}
Up-arrow, ^, is the message token escape character.
\end{verbatim}
\end {small}
\end {quote}

Escape characters and token names will also be output literally if they 
appear within the value assigned to a message token; \textit{i.e.}\ message token 
substitution is not recursive.
This means that if the message system is to be used to output the value of a 
character variable, the contents of which are unknown and may therefore 
include escape characters, the value should first be assigned to a message 
token.
Thus,

\begin {small}
\begin {verbatim}
      CALL MSG_SETC( 'TEXT', VALUE )
      CALL MSG_OUT( 'EXAMPLE_OK', '^TEXT', STATUS )
\end{verbatim}
\end {small}

will output the contents of VALUE literally, whereas

\begin {small}
\begin{verbatim}
      CALL MSG_OUT( 'EXAMPLE_BAD', VALUE, STATUS )
\end{verbatim}
\end {small}

might not produce the desired result.
This consideration is particularly important when outputting text values 
such as file names 
\htmlref{within the ADAM environment}{adam_msg},
where a number of additional escape characters are 
defined\latex{ (see Appendix \ref{adam_msg})}.


\subsection{\xlabel{intercepting_messages}Intercepting messages \label{inter_sect}}

It may sometimes be convenient within an application to write the text of a 
message, complete with decoded message tokens, to a character variable 
instead of the standard output stream.
The Message System provides subroutine MSG\_LOAD to do this.
MSG\_LOAD has the calling sequence:

\begin {small}
\begin{verbatim}
      CALL MSG_LOAD( PARAM, TEXT, OPSTR, OPLEN, STATUS )
\end{verbatim}
\end {small}

Here, the arguments PARAM, TEXT and STATUS are identical to those for MSG\_OUT.
The behaviour of MSG\_LOAD is also the same as MSG\_OUT except that, instead of
sending the expanded message text to the standard output stream, MSG\_LOAD
returns it in the character variable OPSTR (regardless the output filtering
level).
OPLEN returns the length of the message in OPSTR.
If the message text is longer than the declared length of OPSTR, then the
message is truncated with an ellipsis, \textit{i.e.}\ ``\ldots'', but no error 
results.

The symbolic constant MSG\_\_SZMSG is provided for defining the length of 
character variables which are to hold such messages.
This constant is defined in the include file 
\htmlref
{MSG\_PAR}{compiling_and_linking}\latex{ (see \S\ref{compiling_and_linking})}.


\subsection{\xlabel{renewing_annulled_message_tokens}Renewing annulled message tokens}

Each call to MSG\_OUT, MSG\_OUTIF or MSG\_LOAD will annul any defined message 
tokens, regardless of the success of the call.
This feature of the Message Reporting System ensures that message token names
can be re-used with safety in a series of calls to, say, MSG\_OUT (\textit{e.g.}\ 
in order to output a table of values line by line).
However, under certain circumstances it is useful to be able to restore, or 
renew, the values of any message tokens set prior to the call to the output
routine. This can be done using the subroutine MSG\_RENEW.
In order to be effective, MSG\_RENEW must be called after the call
which annulled the required message tokens and prior to any further message
token  definitions (\textit{e.g.}\ using the MSG\_SET\textit{x} and MSG\_FMT\textit{x} 
routines). 
If MSG\_RENEW is called with existing message tokens defined, no action is
taken.

Here is an example of the use of MSG\_RENEW where a table of values is being
output to the user and to a log file:

\begin {small}
\begin{verbatim}
*  Loop to output the table.
      DO 10 I = 1, NROWS

*     Set a token for each column.
         CALL MSG_FMTI( 'COL1', '1X, I10', I )
         CALL MSG_FMTI( 'COL2', 'I7', HDNUMB( I ) )
         CALL MSG_FMTR( 'COL3', 'F10.5', X( I ) )
         CALL MSG_FMTR( 'COL4', 'F10.5', Y( I ) )
         CALL MSG_FMTR( 'COL5', 'F10.5', Z( I ) )

*     Output a row of the table to the user.
         CALL MSG_OUT( ' ', '^COL1 ^COL2 ^COL3 ^COL4 ^COL5', STATUS )

*     Renew the token values.
         CALL MSG_RENEW

*     Build a string with the row of data.
         CALL MSG_LOAD( ' ', '^COL1 ^COL2 ^COL3 ^COL4 ^COL5', OPSTR,
     :                 OPLEN, STATUS )

*     Write the string to a file.
         WRITE( OPFILE, '( 1X, A )', IOSTAT = IOSTAT ) OPSTR( 1 : OPLEN )
 10   CONTINUE
\end{verbatim}
\end {small}


\subsection{\xlabel{resilience}Resilience}

Other than MSG\_OUT, MSG\_OUTIF, MSG\_BLANK and MSG\_LOAD, the Message System 
subroutines do not use a STATUS argument. 
This is because they are intended to be very robust.
In order to construct a message for output to the user, they will attempt to 
recover from any internal failure. 
The STATUS argument in the ``output'' routines conforms to the Starlink
convention for 
\htmlref
{inherited status}{inherited_status}\latex{ (see \S\ref{inherited_status})}.
This means that an application can contain sequences of MSG\_SET\textit{x}, 
MSG\_FMT\textit{x} and MSG\_OUT calls, and only needs to check the status at the 
end.

There are two kinds of ``failure'' that can occur within the Message
System:

\begin {itemize}
\item Message Construction -- Any tokens which cannot be evaluated while
constructing the output message are indicated in the message text by using a
special syntax. 
This syntax is illustrated for the text resulting from the call:

\begin {small}
\begin{verbatim}
      CALL MSG_OUT( 'EXAMPLE_FLUX', 
     :             'Emission flux is ^FLUX (erg/cm2/s).', STATUS )
\end{verbatim}
\end {small}

If the token ``FLUX'' is not defined, then this call will produce the text:

\begin {quote}
\begin {small}
\begin{verbatim}
Emission flux is ^<FLUX> (erg/cm2/s).
\end{verbatim}
\end {small}
\end {quote}

There are several reasons why a token may be undefined:

\begin {itemize}
\item It has not been defined using MSG\_SET\textit{x} \textit{etc}.

\item It has been annulled by a previous call to MSG\_OUT, MSG\_OUTIF 
or MSG\_LOAD (or a corresponding ERR routine).

\item An attempt has been made to define more than 64 message tokens.

\item An error has been made in a call to one of the MSG\_FMT\textit{x} 
subroutines. 
This error could be either a syntax error in the FORMAT argument, or the
result of specifying a field width which is too small (referred to 
as an ``output conversion error'' in Fortran). 
\end {itemize}

Errors in message construction which result from undefined tokens are \emph{not}
considered fatal and STATUS is not set as a result.

\item Message Output -- If an error occurs when
MSG\_OUT, MSG\_OUTIF or MSG\_BLANK attempt to output the message, then the 
STATUS argument will be set to an error value and an  error message will be 
\htmlref{reported}{report_sect}\latex{ (see \S\ref{report_sect})}.
Errors writing the message text to the standard output stream may occur for a
number of reasons -- the seriousness of such an error is dependent upon where 
the message was intended to go:

\begin {itemize}
\item Command terminal or VAX/VMS batch log file -- 
In this case the likelihood of being able to inform the user about the error
is small.
Generally, the application has little chance of continuing successfully under
these circumstances.

\item Output file -- 
The message output may have been re-directed to an output file for the
duration of a particular application. 
Under these circumstances, the user can probably be informed about the error
and the application may be able to continue.
\end {itemize}
\end {itemize}


\section{\xlabel{err_error_reporting_system}ERR -- Error Reporting System \xlabel{err}}

\subsection{\xlabel{overview}Overview}

Although the Message Reporting System could be used for reporting errors, there
are a number of considerations which demand that separate facilities are
available for this:

\begin {itemize}
\item The inherited status scheme is used by the Message System, and so the 
MSG output subroutines will not execute if STATUS is set to an error value. 
Consequently, the Message System cannot be used to report information about
an earlier error which has resulted in STATUS being set. 

\item In a program or package consisting of many levels of subroutines, each
routine which has something informative to say about the error should be
able to contribute to the information that the user receives.
This includes:

\begin {itemize}
\item The subroutine which first detects the error, as this will probably
have access to specific information which is hidden from higher level
routines. 

\item The chain of subroutines between the main program and the routine in
which the error originated.
Some of these will usually be able to report on the context in which the error
occurred, and so add relevant information which is not available to routines 
at lower levels. 
\end {itemize}

This can lead to several error reports arising from a single failure. 

\item It is not always necessary for an error report to reach the user. 
For example, a high-level subroutine or the main program, may decide that it
can handle an error detected at a lower level safely without informing the
user. 
In this case, it is necessary for error reports associated with that
error to be discarded, and this can only happen if the output of error
messages to the user is deferred. 

\end {itemize}

These considerations have led to the design and implementation of a set of
subroutines which form the Error Reporting System.
The subroutines have names of the form:

\begin {quote}
\begin {small}
\begin{verbatim}
ERR_name
\end{verbatim}
\end {small}
\end {quote}

where \texttt{name} indicates what the subroutine does.
These subroutines work in conjunction with the Message System and allow error
messages to incorporate message tokens. 


\subsection{\xlabel{inherited_status_checking}Inherited status checking
\label{inherited_status} \xlabel{inherited_status}} 
The recommended method of indicating when errors have occurred within
Starlink software is to use an integer status value in each subroutine
argument list. 
This inherited status argument, say STATUS, should always be the last
argument and every subroutine should check its value on entry. 
The principle is as follows:

\begin {itemize}
\item The subroutine returns without action if the given value of STATUS is 
not SAI\_\_OK. 

\item The subroutine leaves STATUS unchanged if it completes successfully.

\item The subroutine sets STATUS to an appropriate error value and reports an
error message if it fails to complete successfully. 
\end {itemize}

Here is an example of the use of inherited status within a simple subroutine:

\begin {small}
\begin{verbatim}
      SUBROUTINE ROUTN( VALUE, STATUS )
      
*  Define the SAI__OK global constant.
      INCLUDE 'SAE_PAR'
      INTEGER STATUS

      ...

*  Check the inherited global status.
      IF ( STATUS .NE. SAI__OK ) RETURN

      <application code>

      END
\end{verbatim}
\end {small}

If an error occurs within the ``application code'' of such a subroutine, then
STATUS is set to a value which is not SAI\_\_OK, an error is reported (see
below) and the subroutine aborts. 

Note that it is often useful to use a status argument and inherited status
checking in subroutines which ``cannot fail''. 
This prevents them executing, possibly producing a run-time error, if
their arguments contain rubbish after a previous error.
Every piece of software that calls such a routine is then saved from making
an extra status check. 
Furthermore, if the routine is later upgraded it may acquire the potential
to fail, and so a status argument will subsequently be required.
If a status argument is included initially, existing code which calls 
the routine will not need to be changed (see 
\htmlref{further discussion}{convert_sect}
of this\latex{ in \S\ref{convert_sect}}).

\subsection{\xlabel{setting_and_defining_status_values}Setting and defining status values \label{def_sect}}

The use of the global constants SAI\_\_OK and SAI\_\_ERROR for setting
status values is recommended in general applications. 
These global constants may be defined in each subroutine by including the
file SAE\_PAR at the beginning of the subroutine, prior to the declaration
of any subroutine arguments or local variables. 
When writing subroutine libraries, however, it is useful to have a larger 
number of globally unique error codes available and to define symbolic
constants for these in a separate include 
file.
The naming convention:

\begin {quote}
\begin {small}
\begin{verbatim}
fac__ecode
\end{verbatim}
\end {small}
\end {quote}

should be used for the names of error codes defined in this way; where 
\texttt{fac} is the three-character facility prefix and \texttt{ecode} is up 
to five alphanumeric characters of error code name. 
\emph{Note the double underscore used in this naming convention.}
The include file should be referred to by the name fac\_ERR, \textit{e.g.}

\begin {small}
\begin{verbatim}
      INCLUDE 'SGS_ERR'
\end{verbatim}
\end {small}

where the facility name is \texttt{SGS}, the Starlink Simple Graphics System,
in this case. 
These symbolic constants should be defined at the beginning of every
subroutine which requires them, prior to the declaration of any subroutine
arguments or local variables. 

The purpose of error codes is to enable the status argument to indicate that an 
error has occurred by having a value which is not equal to SAI\_\_OK.
By using a set of predefined error codes the calling module is 
able to test the returned status to distinguish between error conditions 
which may require different action.
It is not generally necessary to define a very large number of error codes
which would allow a unique value to be used every time an error report is
made. 
It is sufficient to be able to distinguish the important classes of error which
may occur. 
Examples of existing software can be consulted as a guide in this matter.

The Starlink utility
\xref{MESSGEN}{sun185}{} \latex{ (see SUN/185)} 
should be used on UNIX to generate 
a set of globally unique error codes for a package. It may be used to create
the Fortran include file and/or a C header file defining symbolic names
for the error codes, and/or the ``facility error message file'', which
can be used to 
\htmlref{associate a simple message}{fio_sect}
with each error code\latex{ (see \S\ref{fio_sect})}.
There is 
\htmlref{an alternative}{stat_sect}
but compatible method of calculating the set of error codes for a 
package\latex{ described in Appendix \ref{stat_sect}}.

Software from outside a package which defines a set of error codes may
use that package's codes to test for specific error conditions arising within
that package.
However, with the exception of the SAI\_\_ codes, it should \emph{not} assign
these values to the status argument.
To do so could cause confusion about which package detected the error. 

\subsection{\xlabel{reporting_errors}Reporting errors \label{report_sect}}

The subroutine used to report errors is ERR\_REP. 
It has a calling sequence of the form

\begin {small}
\begin{verbatim}
      CALL ERR_REP( PARAM, TEXT, STATUS )
\end{verbatim}
\end {small}

Here, the argument PARAM is the error message name, TEXT is the error message
text and STATUS is the inherited status.
These arguments are broadly similar to those used in the Message System
subroutine MSG\_OUT. 

The error message name PARAM should be a globally unique identifier for the
error report.
It is recommended that it has the form:

\begin {quote}
\begin {small}
\begin{verbatim}
routn_message
\end{verbatim}
\end {small}
\end {quote}

in the general case of subroutines within an application, or:

\begin {quote}
\begin {small}
\begin{verbatim}
fac_routn_message
\end{verbatim}
\end {small}
\end {quote}

in the case of routines within a subroutine library.
In the former case, \texttt{routn} is the name of the application routine from
which ERR\_REP is being called and \texttt{message} is a sequence of characters
uniquely identifying the error report within that subroutine. 
In the latter case, \texttt{fac\_routn} is the full name of the subroutine from
which ERR\_REP is being called (see the 
\xref{Starlink Application Programming Standard}{sgp16}{}
\latex{, SGP/16,} for a discussion of the recommended subroutine naming
convention), and \texttt{message} is a sequence of characters unique within 
that subroutine.
These naming conventions are designed to ensure that each individual error
report made within a complete software system has a unique error name
associated with it. 

Here is a simple example of error reporting where part of the application
code of the previous example detects an invalid value of some kind, sets
STATUS, reports the error and then aborts: 

\begin {small}
\begin{verbatim}
      IF ( <value invalid> ) THEN
         STATUS = SAI__ERROR
         CALL ERR_REP( 'ROUTN_BADV', 'Value is invalid.', STATUS )
         GO TO 999
      END IF

      ...

 999  CONTINUE
      END
\end{verbatim}
\end {small}

In the event of an invalid value, the Error System would produce a message 
like:
\begin{small}
\begin{verbatim}
      !! Value is invalid.
\end{verbatim}
\end{small}
Note that when the message is output to the user, the Error System precedes 
the given text with 
\htmlref{exclamation marks}{excl_sect}.
\latex{For more information on this, see \S\ref{excl_sect}.}

The sequence of three operations:

\begin {quote}
\begin {enumerate}
\item Set STATUS to an error value.
\item Report an error.
\item Abort.
\end {enumerate}
\end {quote}

is the standard response to an error condition and should be adopted by all
software which uses the Error System. 

Note that the behaviour of the STATUS argument in ERR\_REP differs somewhat 
from that in MSG\_OUT in that ERR\_REP will execute regardless of the input 
value of STATUS.
Although the Starlink convention is for subroutines not to execute if their
status argument indicates a previous error, the Error System subroutines
obviously cannot behave in this way if their purpose is to report these 
errors.

On exit from ERR\_REP the value of STATUS remains unchanged, with three
exceptions:

\begin{itemize}
\item If ERR\_REP is called with STATUS set to SAI\_\_OK -- in this case an
additional error message 
to this effect is stacked for output to the user and STATUS is returned set
to ERR\_\_BADOK.
\item If ERR\_REP is unable to output the error message -- in this case
STATUS is returned set to ERR\_\_OPTER.
\item If an internal fault occurs. Currently there are no other faults
which will return an error status but the possibility should be allowed for.
\end{itemize}


\subsection{\xlabel{message_tokens_in_error_messages}Message tokens in error messages \label{etok_sect}}

Message tokens can be used in the error text presented to ERR\_REP
in the same manner as their use in calls to MSG\_OUT, MSG\_OUTIF and 
MSG\_LOAD.
Here is an example where two values, LOWER and UPPER, are in conflict: 

\begin {small}
\begin{verbatim}
*  Check if LOWER and UPPER are in conflict.
      IF ( LOWER .GT. UPPER ) THEN

*     Construct and report the error message.
         STATUS = SAI__ERROR
         CALL MSG_SETI( 'LO', LOWER )
         CALL MSG_SETI( 'UP', UPPER )
         CALL ERR_REP( 'BOUND_ERR', 
     :                'LOWER(^LO) is greater than UPPER(^UP).', STATUS )
         GO TO 999
      END IF
\end{verbatim}
\end {small}

If the value of LOWER is 50 and the value of UPPER is 10, then the user
might receive a message like:

\begin {quote}
\begin {small}
\begin{verbatim}
!! LOWER(50) is greater than UPPER(10).
\end{verbatim}
\end {small}
\end {quote}

After a call to ERR\_REP, all message tokens are left undefined.


\subsection{\xlabel{when_to_report_an_error}When to report an error \label{when_sect}}

In the following example, part of an application makes a series of subroutine
calls:

\begin {small}
\begin{verbatim}
      CALL ROUTN1( A, B, STATUS )
      CALL ROUTN2( C, STATUS )
      CALL ROUTN3( T, Z, STATUS )

*  Check the global status.
      IF ( STATUS .NE. SAI__OK ) GO TO 999

      ...

999   CONTINUE
      END
\end{verbatim}
\end {small}

Each of these subroutines uses the inherited status strategy and makes 
error reports by calling ERR\_REP.
If an error occurs within any of the subroutines, STATUS will be set 
to an error value by that routine and inherited status checking by all
subsequent routines will cause them not to execute. 
Thus, it becomes unnecessary to check for an error after each subroutine
call, and a single check at the end of a sequence of calls is all that is
required to correctly handle any error condition that may arise. 
Because an error report will already have been made by the subroutine that
failed, it is usually sufficient simply to abort if an error arises in a
sequence of subroutine calls.

It is important to distinguish the case where a called subroutine sets STATUS 
and makes its own error report, as above, from the case where STATUS is set 
explicitly as a result of a directly detected error, as in the previous
example.
If the error reporting strategy is to function correctly, then responsibility 
for reporting the error must lie with the routine which modifies the status 
argument.
The golden rule is therefore:

\begin {quote}
\emph{If STATUS is explicitly set to an error value, then an accompanying
call to ERR\_REP \emph{must} be made.} 
\end {quote}

Unless there are good documented reasons why this cannot be done,
subroutines which return a status value and do not make an accompanying
error report should be regarded as containing a bug\footnote{For historical
reasons there are still some routines in ADAM which
set a status value without making an accompanying error report -- these are
gradually being corrected. 
If such a routine is used before it has been corrected, then the strategy 
outlined here is recommended.
It is advisable not to complicate new code by attempting to make an error
report on behalf of the faulty subroutine.
If it is appropriate, please ensure that the relevant support person is made
aware of the problem.}.

\subsection{\xlabel{the_content_of_error_messages}The content of error messages}

The purpose of an error message is to be informative and it should
therefore provide as much relevant information about the context of the
error as possible.
It must also avoid the danger of being misleading, or of containing
too much irrelevant information which might be confusing to a user. 
Particular care is necessary when reporting errors from within subroutines
which might be called by a wide variety of software.
Such reports must not make unjustified assumptions about what sort of 
application might be calling them.
For example, in a routine that adds two arrays, the report:

\begin {quote}
\begin {small}
\begin{verbatim}
!! Error adding two arrays.
\end{verbatim}
\end {small}
\end {quote}

would be preferable to:

\begin {quote}
\begin {small}
\begin{verbatim}
!! Error adding two images.
\end{verbatim}
\end {small}
\end {quote}

if the same routine could be called to add two spectra!

The name of the routine which called ERR\_REP to make an error report can
often be a vital piece of information when trying to understand what went
wrong. 
However, the error report is intended for the user, not the programmer, and so
the name of an obscure internal routine is more likely to confuse than to
clarify the nature of the error.
A good rule of thumb is to include the names of routines in error reports
only if those names also appear in documentation -- so that the function
they perform can be discovered without delving into the code.
An example of this appears in the next section.


\subsection{\xlabel{adding_contextual_information}Adding contextual information}

Instead of simply aborting when a status value is set by a called
subroutine, it is also possible for an application to add further information
about the circumstances surrounding the error. 
In the following example, an application makes several calls to a subroutine
which might return an error status value. 
In each case, it reports a further error message so that it is clear which 
operation was being performed when the lower-level error occurred:

\begin {small}
\begin{verbatim}
*  Smooth the sky values.
      CALL SMOOTH( NX, NY, SKY, STATUS )
      IF ( STATUS .NE. SAI__OK ) THEN
         CALL ERR_REP( 'SKYOFF_SKY',
     :                'SKYOFF: Failed to smooth sky values.', STATUS )
         GO TO 999
      END IF

*  Smooth the object values.
      CALL SMOOTH( NX, NY, OBJECT, STATUS )
      IF ( STATUS .NE. SAI__OK ) THEN
         CALL ERR_REP( 'SKYOFF_OBJ',
     :                'SKYOFF: Failed to smooth object values.', 
     :                STATUS )
         GO TO 999
      END IF

      ...

999   CONTINUE
      END
\end{verbatim}
\end {small}

Notice how an additional error report is made in each case, but because
the original status value contains information about the precise nature of 
the error which occurred within the subroutine SMOOTH, it is left unchanged. 

If the first call to subroutine SMOOTH were to fail, say because it could not 
find any valid pixels in the image it was smoothing, then the error message 
the user would receive might be:

\begin {quote}
\begin {small}
\begin{verbatim}
!! Image contains no valid pixels to smooth.
!  SKYOFF: Error smoothing sky image.
\end{verbatim}
\end {small}
\end {quote}

The first part of this message originates from within the subroutine SMOOTH, 
while the second part qualifies the earlier report, making it clear how the 
error has arisen.
Since SKYOFF is the name of an application known to the user, it has been 
included in the contextual error message.

This technique can often be very useful in simplifying error diagnosis, but
it should not be overdone; the practice of reporting errors at \emph{every}
level in a program hierarchy tends to produce a flood of redundant messages.
As an example of good practice for a subroutine library, an error report made 
when an error is first detected, followed by a further contextual error report 
from the ``top-level'' routine which the user actually called, normally
suffices to produce very helpful error messages. 


\subsection{\xlabel{deferred_error_reporting}Deferred error reporting \label{errdef_sect}}

The action of the subroutine ERR\_REP is to report an error to the 
Error System but the Error System has the capacity to defer the output of
that message to the user.
This allows the final delivery of error messages to be controlled within
applications software, and this control is achieved using the subroutines
ERR\_MARK, ERR\_RLSE, ERR\_FLUSH and ERR\_ANNUL. 
This section describes the function of these subroutines and how they are
used.

Subroutine ERR\_MARK has the effect of ensuring that all subsequent error 
messages are deferred by the Error System and stored in an ``error table''
instead of being delivered immediately to the user.
ERR\_MARK also starts a new ``error context'' which has its own table of error
messages and message tokens which are independent of those in the previous
error context.
A return to the previous context can later be made by calling ERR\_RLSE.
When ERR\_RLSE is called, the new error context created by ERR\_MARK ceases
to exist and any error messages stored in it are transferred to the
previous context. 
Calls to ERR\_MARK and ERR\_RLSE can be nested if required but should always 
occur in matching pairs.
In this way, no existing error messages can be lost through the deferral 
mechanism.

The system starts at base-level context (level 1) -- at this level, error 
messages are output to the user immediately.
If a call to ERR\_RLSE returns the system to base-level context, any messages
still stored in the error table will be automatically delivered to the user.

The purpose of deferred error reporting can be illustrated by the following
example.
Consider a subroutine, say HELPER, which detects an error during execution. 
The subroutine HELPER reports the error that has occurred, giving as much
contextual information about the error as it can.
It also returns an error status value, enabling the software that called it
to react to the failure appropriately. 
However, what may be considered an ``error'' at the level of subroutine
HELPER, \textit{e.g.}\ an ``end of file'' condition, may be considered by the
calling module to be a case which can be handled without informing the user,
\textit{e.g.}\ by simply terminating its input sequence. 
Thus, although the subroutine HELPER will always report the error condition,
it is not always necessary for the associated error message to reach the
user. 
The deferral of error reporting enables application programs to handle such 
error conditions internally.

Here is a schematic example of what subroutine HELPER might look like:

\begin {small}
\begin{verbatim}
      SUBROUTINE HELPER( LINE, STATUS )

      ...

*  Check if a Fortran I/O error has occurred.
      IF ( IOSTAT .NE. 0 ) THEN

*     Set STATUS and report the error.
         IF ( IOSTAT .LT. 0 ) THEN

*        Report an end-of-file error.
            STATUS = <end-of-file error code>
            CALL ERR_REP( 'HELPER_FIOER', 
     :         'Fortran I/O error: end of input file reached', STATUS )
         ELSE

*        Report a Fortran I/O error.
            STATUS = SAI__ERROR
            CALL ERR_REP( 'HELPER_FIOER', 
     :         'Fortran I/O error encountered during data input',
     :         STATUS )
         END IF

*     Abort.
         GO TO 999
      END IF

      ...

 999  CONTINUE
      END
\end{verbatim}
\end {small}

Suppose HELPER is called and reports an error, returning with STATUS set.
At this point, the error message may, or may not, have been received by the
user -- this will depend on the environment in which the routine is running,
and on whether the software which called HELPER took any action to defer the
error report. 
HELPER itself does not need to take action (indeed it should \emph{not} take 
action) to ensure delivery of the message to the user; its responsibility 
ends when it aborts, and responsibility for handling the error condition then
passes to the software which called it. 

Now suppose that the subroutine HELPED calls HELPER and wishes to defer any
messages from HELPER so that it can decide how to handle error conditions
itself, rather than troubling the user with spurious messages. 
It can do this by calling the routine ERR\_MARK before it calls HELPER.

The operation of error message deferral can be illustrated by a simple
example:

\begin {small}
\begin{verbatim}
      SUBROUTINE HELPED( STATUS )

      ...

*  Create a new error context.
      CALL ERR_MARK

      <any error messages from HELPER are now deferred>

      CALL HELPER( LINE, STATUS )

      ...
\end{verbatim}
\end {small}

By calling ERR\_MARK before calling HELPER, subroutine HELPED ensures that any
error  messages reported by HELPER are deferred, \textit{i.e.}\ held in the error table.
HELPED can then handle the error condition itself in one of two ways:

\begin {itemize}
\item By calling ERR\_ANNUL( STATUS ), which ``annuls'' the error, deleting 
any deferred error messages in the current context and resetting STATUS to 
SAI\_\_OK.
This effectively causes the error condition to be ignored.
For instance, it might be used if an ``end of file'' condition was expected, 
but was to be ignored and some appropriate action taken instead.
(A call to ERR\_REP could also be used after ERR\_ANNUL to replace the 
initial error condition with another more appropriate one, although this is
not often done.) 

\item By calling ERR\_FLUSH( STATUS ), which ``flushes out'' the error, 
sending any deferred error messages in the current context to the user and
resetting STATUS to SAI\_\_OK. 
This notifies the user that a problem has occurred, but allows the 
application to continue anyway.
For instance, it might be used if a series of files were being read: if one
of these files could not be accessed, then the user could be informed of
this by calling ERR\_FLUSH before going on to process the next file. 
\end{itemize}

Here is the previous example, elaborated to demonstrate the use of ERR\_ANNUL.
It shows how an ``end of file'' condition from HELPER might be detected, 
annulled, and stored by HELPED in a logical variable EOF for later use:

\begin {small}
\begin{verbatim}
*  Initialise end-of-file flag, EOF.
      EOF = .FALSE.

*  Create a new error context.
      CALL ERR_MARK

*  Read line of data.
      CALL HELPER( LINE, STATUS )

*  Trap end-of-file error status and annul any reported error messages
*  for the current error context.
      IF ( STATUS .EQ. <end-of-file error status> ) THEN
         CALL ERR_ANNUL( STATUS )
         EOF = .TRUE.
      END IF

*  Release the current error context.
      CALL ERR_RLSE

*  Abort application on error.
      IF ( STATUS .NE. SAI__OK ) GO TO 999

      ...

999   CONTINUE
      END
\end{verbatim}
\end {small}

Note that the routine chooses only to handle ``end of file'' error
conditions; any other error condition will not be annulled and will
subsequently cause an abort when STATUS is checked after the call to
ERR\_RLSE. 

Here is an example showing
how both ERR\_FLUSH and ERR\_ANNUL may be used during the process of
acquiring a value from the user via a call to the subroutine RDPAR: 

\begin {small}
\begin{verbatim}
*  Create a new error context.
      CALL ERR_MARK

*  Loop to get DSCALE parameter value.
      DO WHILE ( .TRUE. )
         CALL RDPAR( 'DSCALE', DSCALE, STATUS )

*     Check the returned global status.
         IF ( STATUS .EQ. SAI__OK ) THEN

*        Success, so continue with the application.
            GO TO 10
         ELSE IF ( STATUS .EQ. <abort status> ) THEN

*        User wanted to abort, so abort the application.
            CALL ERR_RLSE
            GO TO 999
         ELSE IF ( STATUS .EQ. <null status> ) THEN

*        User entered "null", so annul the error and supply a default.
            CALL ERR_ANNUL( STATUS )
            DSCALE = 1.0
            GO TO 10
         ELSE

*        An error has occurred, so ensure the user knows about it before
*        trying again.
            CALL ERR_FLUSH( STATUS )
            CALL CNPAR( 'DSCALE', STATUS )
         END IF
      END DO

 10   CONTINUE

*  Release the current error context.
      CALL ERR_RLSE

      ...

 999  CONTINUE
      END
\end{verbatim}
\end {small}

Note how ERR\_FLUSH is used to ensure that any error messages are output to
the user before trying again to get a new value. 
In effect, it passes responsibility for the error condition to the user.
This interactive situation is typical of how ERR\_FLUSH should 
be used; it is not needed very often during normal error reporting, and it 
is certainly \emph{not} required as a regular means of ensuring the delivery
of error messages following calls to ERR\_REP --
this should be left to the Error System itself when returned to the base-level
context.

Note that if ERR\_FLUSH cannot output the error message to the user, then it
will return the error status ERR\_\_OPTER.
This allows critical applications to attempt to recover in the event of the
failure of the Error System.

Finally, as a safety feature, if ERR\_FLUSH is called when no errors have
been reported, it outputs the message 

\begin {quote}
\begin {small}
\begin{verbatim}
!! No error to report (improper use of EMS).
\end{verbatim}
\end {small}
\end {quote}

This is to highlight problems where the inherited status has been set by 
some item of software, but no accompanying error message has been reported. 

\subsection{\xlabel{error_table_limits}Error table limits}
The error table can contain up to 32 error messages, normally reported at
different levels within the hierarchy of a structured program. 
If an attempt is made to defer the reporting of more than 32 error messages, 
then the last reported error message will be replaced by the message:

\begin{quote}
\begin {small}
\begin{verbatim}
!! Error message stack overflow (EMS fault).
\end{verbatim}
\end {small}
\end{quote}

There are up to 256 context levels available in the Error System, the 
initial (base-level) error context level being 1.
The current error context level may be inquired using a call to ERR\_LEVEL.
If an attempt is made to mark a context level beyond 256, the error message:
\begin{quote}
\begin {small}
\begin{verbatim}
!! Error context stack overflow (EMS fault).
\end{verbatim}
\end {small}
\end{quote}
is placed on the error stack at context level 256 and any subsequent error
reports will be placed at context level 256.
A bug report should be made if either of the ``EMS fault'' error messages are
reported from software.


\subsection{\xlabel{format_of_delivered_messages}Format of delivered messages \label{excl_sect}}
When messages are delivered to the user, the Error System prefixes the given
text with exclamation marks to call attention to the message and to 
distinguish between error messages and normal informational messages output 
using MSG.
When a sequence of deferred messages is flushed, the first will be prefixed
by `\texttt{!!\ }' and the remainder by `\texttt{!\ \ }'. 

By default, messages are split so that output lines do not exceed 
79 characters -- the split is made on word boundaries if possible. 
The maximum output line size can be altered using
\htmlref{tuning parameters}{tuning}\latex{ (see \S\ref{tuning})}.
If a message has to be split for delivery by the Error System, text on 
continuation lines is indented by three spaces, \textit{e.g.}
\begin {small}
\begin{verbatim}
!! The first line of an error message ...
!     and its continuation onto another line.
!  A second contextual error message.
\end{verbatim}
\end {small}

\subsection{\xlabel{routines_which_perform_cleaningup_operations}Routines which perform ``cleaning-up'' operations}

If a subroutine performs ``cleaning-up'' operations which must execute even
if the inherited status has been set, then a different sequence of status 
checking must usually be performed.
The deferral of error messages may also be involved.

Normally, the effect required is that a cleaning-up routine called with 
its status argument set to SAI\_\_OK will behave like any other routine, 
setting the status value and reporting an error if it fails.
However, if the value of status has been set to an error condition because of 
a previous error, it must still attempt to execute, even if there is a
good chance that it will not succeed.
In this latter case, an error report is not normally required from the
cleaning-up routine. 

The following is a typical example:

\begin {small}
\begin{verbatim}
      CALL ALLOC( NBYTES, PNTR, STATUS )

      <application code>

      CALL DEALL( NBYTES, PNTR, STATUS )
\end{verbatim}
\end {small}

Here, ALLOC allocates some memory for use by the ``application code'' and 
DEALL de-allocates it at the end.
The following error conditions may arise:

\begin{itemize}
\item DEALL fails -- In this case we want to receive an error message from
DEALL saying why this happened. 

\item The application code fails - In this case STATUS will be set to an
error value, but DEALL must still execute in order to recover the allocated
memory. 

\item ALLOC fails -- In this case STATUS will be set to an error value, and
DEALL will attempt to execute, but will also fail because there is no memory
to deallocate. 
In this case, we normally only want to receive an error message from ALLOC.
\end {itemize}

The solution is to write DEALL so that it saves the value of STATUS on entry
and restores it again on exit. 
To preserve the associated error messages, calls to ERR\_MARK and ERR\_RLSE
are also required. 
For example:


\begin {small}
\begin{verbatim}
      SUBROUTINE DEALL( NBYTES, PNTR, STATUS )

      ...

*  Save the initial status value and set a new value for this routine.
      ISTAT = STATUS
      STATUS = SAI__OK

*  Create a new error context.
      CALL ERR_MARK

      <clean-up code>

*  If the initial status was bad, then ignore all internal errors.
      IF ( ISTAT .NE. SAI__OK ) THEN
         CALL ERR_ANNUL( STATUS )
         STATUS = ISTAT
      END IF

*  Release the current error context.
      CALL ERR_RLSE

      END
\end{verbatim}
\end {small}

Note how a new error context is used to constrain ERR\_ANNUL to annulling
only errors arising from the ``clean-up code'', and not the pre-existing
error condition which is to be preserved. 

Two routines are provided to ``wrap up'' these clean-up calls: ERR\_BEGIN and
ERR\_END, which begin and end what is effectively a new error reporting
environment.
ERR\_BEGIN will begin a new error context and return the 
status value set to SAI\_\_OK. 
A call to ERR\_END will annul the current error context if the previous
context contains undelivered error messages.
It will then release the current error context.
ERR\_END returns the status of the last reported error message pending
delivery to the user after the current error context has been released.
If there are no error messages pending output, then the status is
returned set to SAI\_\_OK.
This behaviour is exactly that represented by the code in the previous example.
Here is the previous example re-written using calls to ERR\_BEGIN and
ERR\_END:

\begin {small}
\begin{verbatim}
      SUBROUTINE DEALL( NBYTES, PNTR, STATUS )

      ...

*  Begin a new error reporting environment.
      CALL ERR_BEGIN( STATUS )

      <clean-up code>

*  End the current error reporting environment.
      CALL ERR_END( STATUS )

      END
\end{verbatim}
\end {small}

Like ERR\_MARK and ERR\_RLSE, ERR\_BEGIN and ERR\_END should always occur in
pairs and can be nested if required.
If ERR\_BEGIN is called with STATUS set to an error value, then a check is 
made to determine if there are any error messages pending output at the 
current error context; if there are not then the status has been set without 
making an error report.
In these cases ERR\_BEGIN will make the error report:

\begin {quote}
\begin {small}
\begin{verbatim}
!! Status set with no error report (improper use of EMS).
\end{verbatim}
\end {small}
\end{quote}

using the given status value before marking a new error context.

Any code which attempts to execute when the inherited status is set to an error
value should be regarded as ``cleaning-up''.

\subsection{\xlabel{intercepting_error_messages}Intercepting error messages \label{intcpt_sect}}
It may sometimes be convenient within an application to obtain access to any 
error messages within the current context via a character variable,
instead of the error output stream.
The Error System provides subroutine ERR\_LOAD to do this; it has the
calling sequence:

\begin {small}
\begin{verbatim}
      CALL ERR_LOAD( PARAM, PARLEN, OPSTR, OPLEN, STATUS )
\end{verbatim}
\end {small}

The behaviour of ERR\_LOAD is the same as ERR\_FLUSH except that, instead of
delivering deferred error messages from the current error context to the
user, the error messages are returned, one by one, through character variables
in a series of calls to ERR\_LOAD. 

On the first call of this routine, the error table for the current 
context is copied into a holding area, the current error context
is annulled and the first message in the holding area is returned.
Thereafter, each time the routine is called, the next message from 
the holding area is returned.
The argument PARAM is the returned message name and PARLEN the length of the
message name in PARAM.
OPSTR is the returned error message text and OPLEN is the length of
the error message in OPSTR.

The status associated with the returned message is returned in STATUS
until there are no more messages to return -- then STATUS is set to
SAI\_\_OK, PARAM and OPSTR are set to blanks and PARLEN and OPLEN to 1.
As for ERR\_FLUSH, a warning message is generated if there are no messages
initially. The status returned with the warning message is EMS\_\_NOMSG.

After STATUS has been returned SAI\_\_OK, the whole process is repeated
for subsequent calls.

The symbolic constants ERR\_\_SZPAR and ERR\_\_SZMSG are provided for 
declaring the lengths of character variables which are to receive message 
names and error messages in this way.
These constants are defined in the include file 
\htmlref
{ERR\_PAR}{compiling_and_linking}\latex{ (see \S\ref{compiling_and_linking})}.

\subsection{\xlabel{protecting_tokens}Protecting tokens}
As a general rule, message tokens should be assigned, using calls to the
MSG\_SET\textit{x} and MSG\_FMT\textit{x} routines, immediately prior to the call 
in which they are to be used.
However, this is not always convenient; \textit{e.g.}\ within an iteration 
or a block IF statement where the same tokens may be used in one of several
potential message reports.
Under these circumstances, it is important to protect the values of assigned
message tokens when subroutines which may fail are called -- when a subroutine
fails it must be assumed that it will make an accompanying error report
using ERR\_REP within the existing error reporting context, thereby annulling
any currently defined message tokens.
The only sure way of protecting against such behaviour is to bracket the 
subroutine call which may fail with calls to ERR\_MARK and ERR\_RLSE.
The same precautions are needed when any subroutine is called which
may in turn call any of MSG\_OUT, MSG\_OUTIF, MSG\_LOAD, ERR\_REP or ERR\_LOAD
(all of which annul tokens).

It is not good practice to assign message tokens which are to be used in
another subroutine.

Here is an example of assigning message tokens outside a block IF statement
to be used by ERR\_REP and MSG\_OUT calls within the IF block. 
The code is a fragment of a routine for re-scaling a single array to a mean 
of unity.
If the call to the subroutine MEAN fails, any assigned message tokens in the 
current error reporting context may be annulled; hence the need to bracket this
call by calls to ERR\_MARK and ERR\_RLSE.

\begin{small}
\begin{verbatim}
*  Get the data arrays.
      CALL GETDAT( X, Y, QUAL, NDATA, STATUS )

*  Check the returned status.
      IF ( STATUS .EQ. SAI__OK ) THEN

*     The data have been obtained successfully, assign the token value 
*     and inform the user of the number of data obtained.
         CALL MSG_SETI( 'NDATA', NDATA )

         IF ( NDATA .LE. 0 ) THEN

*        No data exist, report an error message and abort.
            STATUS = SAI__ERROR
            CALL ERR_REP( 'NDATA_INVAL', 
     :                    'Cannot use this number of data (^NDATA).',
     :                    STATUS )
         ELSE

*        Get the mean of the data.
            CALL ERR_MARK
            CALL MEAN( NDATA, Y, QUAL, MEAN, STATUS )
            CALL ERR_RLSE

*        Check the returned status.
            IF ( STATUS .EQ. SAI__OK ) THEN

*           Deliver the number of data and their mean to the user.
               CALL MSG_SETR( 'MEAN', MEAN )

               IF ( NDATA .EQ. 1 ) THEN
                  CALL MSG_OUT( ' ', 
     :               '^NDATA data value (^MEAN) will be used.', 
     :               STATUS )
               ELSE
                  CALL MSG_OUT( ' ', 
     :               '^NDATA data values with a mean of ^MEAN' //
     :               ' will be used.', STATUS )
               END IF
            ELSE

*           Failed to calculate a mean value for the data (the quality
*           flags were probably all bad). Report an error and abort.
               IF ( NDATA .EQ. 1 ) THEN
                  CALL MSG_SETC( 'VALUE', 'value' )
               ELSE
                  CALL MSG_SETC( 'VALUE', 'values' )
               END IF

               CALL ERR_REP( 'BAD_DATA', 
     :            'No mean available for ^NDATA ^VALUE,', //
     :            ' cannot rescale the data.', STATUS )
            END IF
         END IF
      END IF
\end{verbatim}
\end{small}


\subsection{\xlabel{reporting_status_fortran_io_and_operating_system_error}Reporting Status, Fortran I/O and operating system errors
\label{fio_sect}}
Some of the lower-level Starlink libraries cannot use ERR (or EMS) to make 
error reports; furthermore, some items of software may need to perform Fortran
I/O operations or calls to operating system routines. Any of these may fail 
but will not have made error reports through ERR\_REP.
For this reason it is sometimes useful to convert the given error code into
a message which can be displayed as part of a message at a higher level where
some context information can be added.

Three subroutines exist to enable a message token to be built from
the error code returned under these circumstances. 
These subroutines are:

\begin {quote}
\begin {small}
\begin{verbatim}
ERR_FACER( TOKEN, STATUS )
\end{verbatim}
\end {small}
\end {quote}
where STATUS is a standard Starlink facility status value,
\begin {quote}
\begin {small}
\begin{verbatim}
ERR_FIOER( TOKEN, IOSTAT )
\end{verbatim}
\end {small}
\end {quote}
where IOSTAT is a Fortran I/O status code, and
\begin {quote}
\begin {small}
\begin{verbatim}
ERR_SYSER( TOKEN, SYSTAT )
\end{verbatim}
\end {small}
\end {quote}
where SYSTAT is a status value returned from an operating system routine.

Each of the above routines will assign the message associated with the given 
error code to the specified token, appending the message if the token is already
defined. The error code argument is never altered by these routines.
It is important that the correct routine is called, otherwise the wrong message 
or, at best, only an error number will be obtained.

ERR\_FACER is not likely to be useful for applications programmers because
suitable error reports will probably have been made by higher-level facilities 
called directly by the application -- it is really provided for completeness.
The other two routines will be more useful.

Here is an example of using ERR\_FIOER. 
It is a section of code that writes a character variable to a formatted 
sequential file, given the Fortran logical unit of the file: 

\begin {small}
\begin{verbatim}
*  Write the character variable STR.
      WRITE( UNIT, '(A)', IOSTAT = IOSTAT ) STR

*  Check the Fortran I/O status.
      IF ( IOSTAT .NE. 0 ) THEN

*     Fortran write error, so set STATUS.
         STATUS = SAI__ERROR

*     Define the I/O status and logical unit message tokens and attempt 
*     to obtain the file name.
         CALL ERR_FIOER( 'MESSAGE', IOSTAT )
         CALL MSG_SETI( 'UNIT', UNIT )
         INQUIRE( UNIT, NAME = FNAME, IOSTAT = IOS )

*     Check the returned I/O status from the INQUIRE statement and act.
         IF ( IOS .EQ. 0 ) THEN

*        Define the file name message token.
            CALL MSG_SETC( 'FNAME', FNAME )

*        Report the error.
            CALL ERR_REP( 'PUTSTR_WRERR',
     :                'Error writing to file ^FNAME on ' //
     :                'unit ^UNIT: ^MESSAGE', STATUS )
         ELSE

*        No file name has been found so just report the error.
            CALL ERR_REP( 'PUTSTR_WRERR',
     :                'Error writing to unit ^UNIT: ^MESSAGE', STATUS )
         END IF

         GO TO 999
      END IF

      ...

 999  CONTINUE
      END
\end{verbatim}
\end {small}

Here, the name of the file being read is also obtained in order to construct 
a comprehensive error message, which might be something like:

\begin {quote}
\begin {small}
\begin{verbatim}
!! Error writing to file BLOGGS.DAT on unit 17: Disk quota exceeded.
\end{verbatim}
\end {small}
\end {quote}

Note that the I/O status values used in Fortran do not have universally
defined meanings except for zero (meaning no error), but by using
ERR\_FIOER it is still possible to make high quality error reports about
Fortran I/O errors in a portable manner. 

In a similar way, the subroutine ERR\_SYSER may be used to assign an 
operating system message associated with the 
system status flag SYSTAT to the named message token.
Of course, software that calls operating system routines directly cannot be
portable, but ERR\_SYSER provides a convenient interface for reporting 
errors that occur in such routines in a form that can be easily changed if 
necessary.
For example:

\begin {small}
\begin{verbatim}
      IF ( <system error condition> ) THEN

*     Operating system error, so set STATUS.
         STATUS = SAI__ERROR

*     Report the error and abort.
         CALL ERR_SYSER( 'ERRMSG', SYSTAT )
         CALL ERR_REP( 'ROUTN_SYSER', 'System error: ^ERRMSG', STATUS )
         GO TO 999
      END IF

      ...

 999  CONTINUE
      END
\end{verbatim}
\end {small}

Fortran I/O and operating system error messages, obtained through calls to 
ERR\_FIOER and ERR\_SYSER respectively, will differ depending upon which 
operating system (or even flavour of operating system) an application is run on.

Because of the necessary generality of these messages (and those from
ERR\_FACER), many will appear rather vague and unhelpful without additional 
contextual information.
This is particularly true of UNIX implementations.
It is very important to provide additional contextual information when using
these routines in order to avoid obfuscating rather than clarifying the nature 
of an error.
This can be done either as part of the error message which includes the message
token set by ERR\_FACER, ERR\_FIOER or ERR\_SYSER, or by making a further 
error report.
The examples in this section provide a good illustration of how this can 
be done.


\subsection{\xlabel{incorporating_foreign_routines}Incorporating foreign routines \label{foreign_sect}}

Sometimes ``foreign'' subroutines must be called which do not use the
Starlink error status conventions (\textit{e.g.}\ because they must adhere to some
standard interface definition like GKS).
Unless they are unusually robust, such routines must normally be prevented
from executing under error conditions, either by performing a status check
immediately beforehand, or by enclosing them within an appropriate IF...END 
IF block.
Depending on the form of error indication that such foreign routines use, it 
may also be necessary to check afterwards whether they have succeeded or 
not.
If such a routine fails, then for compatibility with other Starlink
software a status value should be set and an error report made on its
behalf. 

For example, the following code makes a GKS inquiry and checks the 
success of that inquiry:

\begin {small}
\begin{verbatim}
      IF ( STATUS .EQ. SAI__OK ) THEN

*     Inquire the GKS workstation colour facilities available.
         CALL GQCF( WTYPE, ERRIND, NCOLI, COLA, NPCI )

*     Check if a GKS error has occurred.
         IF ( ERRIND .NE. GKS__OK ) THEN

*        An error has occurred, so report it and abort.
            STATUS = SAI__ERROR
            CALL MSG_SETI( 'ERRIND', ERRIND )
            CALL ERR_REP( 'ROUTN_GQCFERR', 
     :              'Error no. ^ERRIND occurred in GKS routine GQCF ' //
                    '(enquire workstation colour facilities).',
     :              STATUS )
            GO TO 999
         END IF
      END IF

      ...

 999  CONTINUE
      END
\end{verbatim}
\end {small}

In some cases, it may be possible to obtain a textual error message from the
error flag, by means of a suitable inquiry routine, which could be used as
the basis of the error report. 

It will be obvious from this example how convenient the inherited error
status strategy is, and how much extra work is involved in obtaining the
same degree of robustness and quality of error reporting from routines
which do not use it. 
It is worth bearing this in mind if you are involved in importing foreign
subroutine libraries for use with Starlink software: the provision of a few
simple routines for automating error reporting, or an extra layer of
subroutine calls where inherited status checking and error reporting can be
performed, can make the final product vastly easier to use. 
Starlink staff will be pleased to offer advice on this matter if consulted.


\subsection{\xlabel{converting_existing_subroutine_libraries}Converting existing subroutine libraries \label{convert_sect}}

When converting existing subroutine
libraries to use the inherited status conventions and the Error System,
it is conceivable that an existing subroutine which does not have a
status argument will acquire the potential to fail and report an error, either 
from within itself or from packages layered beneath it.
Ideally, the argument list of the subroutine should be changed to include a
status argument.
However, 
it may be inconvenient to modify the argument list of a commonly used
subroutine (\textit{i.e.}\ because of the amount of existing code which would have
to be changed), and so an alternative method is needed to determine if status
has  been set during the
call so that the appropriate action can be taken by the caller.
The subroutine ERR\_STAT is provided for recovering the last reported status
value under these conditions.
Here is an example of the use of ERR\_STAT, called from a subroutine which
follows the error reporting conventions:

\begin {small}
\begin{verbatim}
*  Call subroutine NOSTAT.
      CALL NOSTAT( GIVARG, RETARG )
      CALL ERR_STAT( STATUS )
\end{verbatim}
\end {small}

Here, the calls to NOSTAT and ERR\_STAT are equivalent to one subroutine call 
with a status argument.
The use of ERR\_STAT to return the current error status relies upon the use of
ERR\_REP to report errors by the conventions described in this document.
In particular, foreign packages \emph{must} be incorporated in 
\htmlref{the recommended way}{foreign_sect}\latex{as described
in \S\ref{foreign_sect}} for ERR\_STAT to be reliable.

Finally, it is emphasised that ERR\_STAT is \emph{only} for use where there is
no other choice than to use this mechanism to determine the last reported error
status.


\section{\xlabel{tuning}Tuning\label{tuning}}
Some aspects of MSG and ERR systems can be tuned to the user's requirements.
Tuning of the two systems is done separately as potentially each could
be using entirely different devices. (In practice they share the same device
and use EMS at lower levels, so tuning ERR can affect MSG and vice versa.)

Tuning is performed by calling the subroutines MSG\_TUNE and/or ERR\_TUNE,
giving a keyword to specify the parameter to be set, and a value.
See the descriptions of
\htmlref{MSG\_TUNE}{MSG_TUNE} 
and
\htmlref{ERR\_TUNE}{ERR_TUNE}
for details of the parameters and values available and for some further notes.

When the tuning subroutines are called, the given value may be overridden by
setting an environment variable.  The name of the environment variable is
constructed by prefixing the string `MSG\_' or `ERR\_' to the tuning parameter
name. An attempt is made to interpret the translation of the environment
variable as an integer in the required range for the particular parameter.

For example, if the program contains:
\begin{quote} \begin{verbatim}
CALL MSG_TUNE( 'SZOUT', 79, STATUS )
\end{verbatim} \end{quote}
and environment variable MSG\_SZOUT is set to 0, no line wrapping occurs
on output.

If the environment variable is set but the value is invalid, an error message
is reported.

If a tuning subroutine is called with the tuning parameter name set to
'ENVIRONMENT', the given value is ignored but an attempt is made to get values
for all possible tuning parameters of the subsystem (MSG or ERR) from their 
associated environment variable.

\newpage
\section{\xlabel{the_c_interface}\label{the_c_interface}The C Interface}
A preliminary C interface is provided for trial purposes. It may be subject to
change in the light of experience.

The interface obeys the rules defined in (PRO)LUN/10.

Briefly, the function name is generated from the Fortran subroutine name by
forcing the name to lower case apart from the first character following any
underscores, which is forced to upper case. Underscores are then removed.

For example: The C interface function for `MSG\_OUT' is `msgOut'.

Arguments are provided in the same order as for the Fortran routine with the
exception that CHARACTER arrays and returned CHARACTER strings 
have an additional argument of type \texttt{int} (passed by value) immediately 
following them to specify a maximum length for the output string(s) including 
the terminating null for which space must be allowed. 

There is a fixed relationship between the type of the Fortran 
argument and the type of the argument supplied to the C function -- it is 
as follows:
\begin{center}
\begin{tabular}{|l|l|}
\hline
Fortran type & C type \\
\hline
INTEGER & int \\
REAL & float \\
REAL*8 & double \\
DOUBLE PRECISION & double \\
LOGICAL & int \\
CHARACTER & char \\
FUNCTION & \textit{type} (*\textit{name})() \\
SUBROUTINE & void (*\textit{name})() \\
\hline
\end{tabular}
\end{center}

Apart from any argument named `status', given-only scalar arguments (not
including character strings) are passed by value. 
All others are passed by pointer.

Arrays must be passed with the elements stored in the order required by Fortran.

All necessary constants and function prototypes can be defined by:
\begin{quote} \begin{verbatim}
#include "mers.h"
\end{verbatim} \end{quote}
The header file \texttt{mers.h} is contained in directory
\texttt{/star/include}.

\section{\xlabel{compiling_and_linking_with_msg_and_err}Compiling and Linking with MSG and ERR 
\label{compiling_and_linking}\xlabel{compiling_and_linking}}
There are five Fortran include files available for use with the Message and
Error System: 
\htmlref{SAE\_PAR, MSG\_PAR, MSG\_ERR, ERR\_PAR and ERR\_ERR}{incl_sect}.
\latex{(See Appendix \ref{incl_sect} for details of the symbolic constants 
which they define).}

The Starlink convention
is that the name in upper case with no path or extension is specified when 
including these files within Fortran code, 
\textit{e.g.}
\begin{quote}
\begin{small}
\begin{verbatim}
*  Global Constants:
      INCLUDE 'SAE_PAR'
      INCLUDE 'MSG_PAR'
\end{verbatim}
\end{small}
\end{quote}

Equivalent header files are provided for use in C code which is calling MSG
or ERR -- all the required header files, including the function prototypes,
may be included by including the file \textbf{mers.h}.

The syntax
\begin{quote}
\begin{verbatim}
#include "sae_par.h"
#include "mers.h"
\end{verbatim}
\end{quote}
should be used within the C code.

Assuming that the software has been installed in the standard way and
\textbf{/star/bin} has been added to the environment variable
\textbf{PATH}, soft links with these upper-case names pointing to the required 
file are set up in the user's working directory by the the commands:
\begin {quote}
\begin{small}
\begin{verbatim}
% star_dev
% err_dev
\end{verbatim}
\end{small}
\end {quote}

The ADAM version of MSG/ERR is included automatically when
programs are linked using the ADAM application linking commands, \textbf{alink} 
\textit{etc}. These will handle either Fortran or C code.

To compile and link a non-ADAM program with the stand-alone version of
MSG/ERR, the command line would be, \textit{e.g.}
\begin {quote}
\begin{small}
\begin{verbatim}
% f77 -o program program.f -L/star/lib `err_link` 
\end{verbatim}
\end{small}
\end {quote}

On platforms with shareable libraries, \texttt{-L/star/lib} might be replaced 
by \texttt{-L/star/share}.

If it is necessary to link explicitly with the ADAM version of MSG/ERR and any
libraries which it uses (\textit{e.g.}\ to produce a shareable library), the 
script \textbf{err\_link\_adam} is available in \textbf{/star/bin}. 
The link command might be:
\begin {quote}
\begin {small}
\begin{verbatim}
% ld -shared -o libmypkg.so.1.0 -L/star/share -lmypkg `err_link_adam`
\end{verbatim}
\end {small}
\end {quote}

The compilation of C code should be used with the compiler flag
\textbf{-I/star/include}. 
For example the command to compile a C program might be:
\begin {quote}
\begin{verbatim}
% cc -c -I/star/include program.c
\end{verbatim}
\end {quote}

\begin {quote}
\emph{Note that the command used to invoke the C compiler varies from 
one UNIX implementation to another (indeed, there may be more than one C
compiler available on the same machine) -- you should therefore check this
with your Site Manager before proceeding.}
\end {quote}
Because the MERS C interface calls Fortran routines it is also necessary to
explicitly link with the required Fortran libraries, \textit{e.g.}
\begin {quote}
\begin{verbatim}
% cc program.o -L/star/lib `err_link` -lF77 -lm -o program.out
\end{verbatim}
\end {quote}
The naming and number of the Fortran libraries differs between UNIX machines 
and so it is advisable to check in the relevant Fortran documentation for 
further details.

\section{\xlabel{references}References}
\emph{Note}: Only the first author is listed here.

%\begin{latex}
\begin {tabular}{lll}
Lawden, M.D. & \xref{SG/4}{sg4}{} 
--- ADAM -- The Starlink Software Environment.\\
Rees, P.C.T. & \xref{SSN/4}{ssn4}{} 
--- EMS -- Error Message Service.\\
Chipperfield, A.J. & \xref{SUN/185}{sun185}{} 
--- MESSGEN -- Starlink Facility Error Message Generation.\\
Wallace, P.T. & \xref{SGP/16}{sgp16}{}
--- Starlink Application Programming Standard.\\
Charles A.C. & \xref{SUN/40}{sun40}{}
--- CHR -- Character Handling Routines.
\end {tabular}
%\end{latex}

%\begin{htmlonly}
%Lawden, M.D. : \xref{SG/4}{sg4}{} 
%--- ADAM -- The Starlink Software Environment.\\
%Rees, P.C.T. : \xref{SSN/4}{ssn4}{} 
%--- EMS -- Error Message Service.\\
%Chipperfield, A.J. : \xref{SUN/185}{sun185}{} 
%--- MESSGEN -- Starlink Facility Error Message Generation.\\
%Wallace, P.T. : \xref{SGP/16}{sgp16}{}
%--- Starlink Application Programming Standard.\\
%Charles A.C. : \xref{SUN/40}{sun40}{}
%--- CHR -- Character Handling Routines.
%\end{htmlonly}

\newpage
\appendix
\section{\xlabel{include_files}Include Files \label{incl_sect}}
The symbolic constants defined by the five include files used with MSG and ERR
are listed below. The way in which these files are included within Fortran
code is described in the section on 
\htmlref
{compiling and linking}{compiling_and_linking}\latex{ \S\ref{compiling_and_linking}}.
\begin {description}
\item [SAE\_PAR] Defines the non-specific status codes for Starlink.
\begin{description}
\item \textbf{SAI\_\_ERROR} -- Error encountered.
\item \textbf{SAI\_\_OK} -- No error.
\item \textbf{SAI\_\_WARN} -- Warning.
\end {description}
\item [MSG\_PAR] Defines the public Message System constants.
\begin{description}
\item \textbf{MSG\_\_NORM} -- Normal conditional message output level.
\item \textbf{MSG\_\_QUIET} -- Quiet (few messages) conditional message output 
level.
\item \textbf{MSG\_\_SZMSG} -- Maximum length of message text.
\item \textbf{MSG\_\_VERB} -- Verbose (abundant messages) conditional message
output level.
\item \textbf{MSG\_\_DEBUG} -- Debug (many many messages) conditional message output level.
\end {description}
\item [MSG\_ERR] Defines the Message Reporting System errors.
\begin{description}
\item \textbf{MSG\_\_BDPAR} -- Invalid tuning parameter name 
(improper use of MSG\_TUNE).
\item \textbf{MSG\_\_BTUNE} -- Bad tuning value (improper use of MSG\_TUNE).
\item \textbf{MSG\_\_INVIF} -- Invalid conditional message filter value.
\item \textbf{MSG\_\_BDENV} -- Bad MSG environment variable value.
\item \textbf{MSG\_\_OPTER} -- Error encountered during message output.
\item \textbf{MSG\_\_SYNER} -- Error encountered during message synchronisation.
\end {description}
\item [ERR\_PAR] Defines the public Error System constants.
\begin{description}
\item \textbf{ERR\_\_SZMSG} -- Maximum length of error message text.
\item \textbf{ERR\_\_SZPAR} -- Maximum length of error message name.
\end{description}
\item [ERR\_ERR] Defines the Error Reporting System errors.
\begin{description}
\item \textbf{ERR\_\_BADOK} -- Status set to SAI\_\_OK in call to ERR\_REP 
(improper use of ERR\_REP).
\item \textbf{ERR\_\_BDPAR} -- Invalid tuning parameter name 
(improper use of ERR\_TUNE).
\item \textbf{ERR\_\_BTUNE} -- Bad tuning value (improper use of ERR\_TUNE).
\item \textbf{ERR\_\_BDENV} -- Bad ERR environment variable value.
\item \textbf{ERR\_\_OPTER} -- Error encountered during message output. 
\item \textbf{ERR\_\_UNSET} -- Status not set in call to ERR\_REP (improper use 
of ERR\_REP). 
\end {description}
\end {description}

\newpage
\section{\xlabel{subroutine_list}Subroutine List \label{sublis_sect}}
\subsection{\xlabel{message_system_subroutines}Message System subroutines}

\begin {description} 
\item [\htmlref{MSG\_BELL}{MSG_BELL} ( STATUS )] \listline
\textit{ Deliver an ASCII BEL character.}
\item [\htmlref{MSG\_BLANK}{MSG_BLANK} ( STATUS )] \listline
\textit{ Output a blank line.}
\item [\htmlref{MSG\_FMT\textit{x}}{MSG_FMTx} ( TOKEN, FORMAT, VALUE )] \listline
\textit{ Assign a value to a message token (formatted).}
\item [\htmlref{MSG\_IFLEV}{MSG_IFLEV} ( FILTER )] \listline
\textit{ Return the current filter level for conditional message
output.}
\item [\htmlref{MSG\_IFSET}{MSG_IFSET} ( FILTER, STATUS )] \listline 
\textit{ Set the filter level for conditional message output.}
\item [\htmlref{MSG\_LOAD}{MSG_LOAD} ( PARAM, TEXT, OPSTR, OPLEN, STATUS )] \listline 
\textit{ Expand and return a message. }
\item [\htmlref{MSG\_OUT}{MSG_OUT} ( PARAM, TEXT, STATUS )] \listline 
\textit{ Output a message. }
\item [\htmlref{MSG\_OUTIF}{MSG_OUTIF} ( FILTER, PARAM, TEXT, STATUS )] \listline 
\textit{ Conditionally deliver the text of a message to the user.}
\item [\htmlref{MSG\_RENEW}{MSG_RENEW}] \listline
\textit{ Renew any annulled message tokens in the current context.}
\item [\htmlref{MSG\_SET\textit{x}}{MSG_SETx} ( TOKEN, VALUE )] \listline 
\textit{ Assign a value to a message token (concise).}
\item [\htmlref{MSG\_TUNE}{MSG_TUNE} ( PARAM, VALUE, STATUS )] \listline 
\textit{ Set an MSG tuning parameter.}
\end {description}

\subsection{\xlabel{error_system_subroutines}Error System subroutines}
\begin {description}
\item [\htmlref{ERR\_ANNUL}{ERR_ANNUL} ( STATUS )] \listline 
\textit{ Annul the contents of the current error context.}
\item [\htmlref{ERR\_BEGIN}{ERR_BEGIN} ( STATUS )] \listline
\textit{ Begin a new error reporting environment.}
\item [\htmlref{ERR\_END}{ERR_END} ( STATUS )] \listline
\textit{ End the current error reporting environment.}
\item [\htmlref{ERR\_FACER}{ERR_FACER} ( TOKEN, IOSTAT )] \listline 
\textit{ Assign the message associated with a Starlink status to a
token.}
\item [\htmlref{ERR\_FIOER}{ERR_FIOER} ( TOKEN, IOSTAT )] \listline 
\textit{ Assign the message associated with a Fortran I/O error to a
token.}
\item [\htmlref{ERR\_FLBEL}{ERR_FLBEL} ( STATUS )] \listline 
\textit{ Deliver an ASCII BEL and flush the current error context.}
\item [\htmlref{ERR\_FLUSH}{ERR_FLUSH} ( STATUS )] \listline 
\textit{ Flush the current error context.}
\item [\htmlref{ERR\_LEVEL}{ERR_LEVEL} ( LEVEL )] \listline
\textit{ Inquire the current error context level.}
\item [\htmlref{ERR\_LOAD}{ERR_LOAD} ( PARAM, PARLEN, OPSTR, OPLEN, STATUS )] \listline 
\textit{ Return error messages from the current error context.}
\item [\htmlref{ERR\_MARK}{ERR_MARK}] \listline
\textit{ Mark (start) a new error context.}
\item [\htmlref{ERR\_REP}{ERR_REP} ( PARAM, TEXT, STATUS )] \listline 
\textit{ Report an error message.}
\item [\htmlref{ERR\_RLSE}{ERR_RLSE}] \listline
\textit{ Release (end) the current error context.}
\item [\htmlref{ERR\_STAT}{ERR_STAT} ( STATUS )] \listline
\textit{ Inquire the last reported error status.}
\item [\htmlref{ERR\_SYSER}{ERR_SYSER} ( TOKEN, SYSTAT )] \listline 
\textit{ Assign the message associated with an operating system error
to a token.}
\item [\htmlref{ERR\_TUNE}{ERR_TUNE} ( PARAM, VALUE, STATUS )] \listline 
\textit{ Set an ERR tuning parameter.}
\end {description}

\subsection{\xlabel{adam_special_user_subroutines}ADAM-special user subroutines}
These are only for use in ADAM applications.
\begin {description}
\item [\htmlref{MSG\_IFGET}{MSG_IFGET} ( PNAME, STATUS )] \listline 
\textit{ Get the MSG filter level from the ADAM parameter system.}
\item [\htmlref{MSG\_SYNC}{MSG_SYNC} ( STATUS )] \listline 
\textit{ Synchronise message output via the user interface.}
\end{description}

\subsection{\xlabel{adam_special_system_subroutines}ADAM-special system subroutines}
These are only for use by the ADAM system. They are described in
\xref{SSN/4}{ssn4}{routines_specific_to_the_adam_fixed_part}
but listed here for completeness.
\begin {description}
\item [\xref{ERR\_CLEAR}{ssn4}{ERR_CLEAR} ( STATUS )] \listline 
\textit{ Return the error table to the default context and flush its contents.}
\item [\xref{ERR\_START}{ssn4}{ERR_START}] \listline 
\textit{ Initialise the Error Reporting System.}
\item [\xref{ERR\_STOP}{ssn4}{ERR_STOP}] \listline 
\textit{ Close the Error Reporting System.}
\end{description}

\newpage
\section{\xlabel{c_interface_function_prototypes}C Interface Function Prototypes} \label{C_sect}
Where \textit{T} is one of d, i, l, r, and \textit{TYPE} is the corresponding C
type, the function prototypes for the C language interface are:
\begin{description}
\item[void errAnnul( int \texttt{*}status );] \listline
\textit{Annul the contents of the current error context.}
\item[void errBegin( int \texttt{*}status );] \listline
\textit{Begin a new error reporting environment.}
\item[void errClear( int \texttt{*}status );] \listline
\textit{Return the error table to the default context and flush its contents.}
\item[void errEnd( int \texttt{*}status );] \listline
\textit{End the current error reporting environment.}
\item[void errFacer( const char \texttt{*}token, int \texttt{*}status );] \listline
\textit{Assign the message associated with a Starlink status to a token.}
\item[void errFioer( const char \texttt{*}token, int iostat );] \listline
\textit{Assign the message associated with a Fortran I/O error to a token.}
\item[void errFlbel( int \texttt{*}status );] \listline
\textit{Deliver an ASCII BEL and flush the current error context.}
\item[void errFlush( int \texttt{*}status );] \listline
\textit{Flush the current error context.}
\item[void errLevel( int \texttt{*}level );] \listline
\textit{Inquire the current error context level.}
\item[void errLoad] \listline
\textbf{( char \texttt{*}param, int param\_length, int
\texttt{*}parlen,  {\listline}
char \texttt{*}opstr, int opstr\_length, int \texttt{*}oplen,
int \texttt{*}status);} \\
\textit{Return error messages from the current error context.}
\item[void errMark( void );] \listline
\textit{Mark (start) a new error context.}
\item[void errRep( const char \texttt{*}param, const char \texttt{*}text, 
int \texttt{*}status );] \listline
\textit{Report an error message.}
\item[void errRlse( void );] \listline
\textit{Release (end) the current error context.}
\item[void errStart( void );] \listline
\textit{Initialise the Error Reporting System.}
\item[void errStat( int \texttt{*}status );] \listline
\textit{Inquire the last reported error status.}
\item[void errStop( int \texttt{*}status );] \listline
\textit{Close the Error Reporting System.}
\item[void errTune( const char \texttt{*}param, int value, 
int \texttt{*}status );] \listline
\textit{Set an ERR tuning parameter.}
\item[void msgBell( int \texttt{*}status );] \listline
\textit{Deliver an ASCII BEL character.}
\item[void msgBlank( int \texttt{*}status );] \listline
\textit{Output a blank line.}
\item[void msgFmtc( const char \texttt{*}token, const char \texttt{*}format, 
const char \texttt{*}cvalue );] \listline
\textit{Assign a CHARACTER value to a message token (formatted).}
\item[void msgFmt\textit{T}( const char \texttt{*}token,
const char \texttt{*}format, \textit{TYPE} value );] \listline
\textit{Assign a value to a message token (formatted).}
\item[void msgIfget( const char \texttt{*}pname, int \texttt{*}status );] \listline
\textit{Get the MSG filter level from the ADAM parameter system.}
\item[void msgIflev( int \texttt{*}filter );] \listline
\textit{Return the current filter level for conditional message output.}
\item[void msgIfset( int filter, int \texttt{*}status );] \listline
\textit{Set the filter level for conditional message output.}
\item[void msgLoad] \listline
\textbf{( const char \texttt{*}param, const char \texttt{*}text, {\listline}
char \texttt{*}opstr, int opstr\_length, int \texttt{*}oplen,
int \texttt{*}status );}\\
\textit{Expand and return a message.}
\item[void msgOut( const char \texttt{*}param, const char \texttt{*}text, 
int \texttt{*}status );] \listline
\textit{Output a message.}
\item[void msgOutif( int prior, const char \texttt{*}param, 
const char \texttt{*}text, int \texttt{*}status );] \listline
\textit{Conditionally deliver the text of a message to the user.}
\item[void msgRenew( void );] \listline
\textit{Renew any annulled message tokens in the current context.}
\item[void msgSetc( const char \texttt{*}token, 
const char \texttt{*}cvalue );] \listline
\textit{Assign a CHARACTER value to a message token (concise).}
\item[void msgSet\textit{T}( const char \texttt{*}token, 
\textit{TYPE} value );] \listline
\textit{Assign a value to a message token (concise).}
\item[void msgSync( int \texttt{*}status );] \listline
\textit{Synchronise message output via the user interface.}
\item[void errSyser( const char \texttt{*}token, int systat );] \listline
\textit{Assign the message associated with an operating system error to a
token.}
\item[void msgTune( const char \texttt{*}param, int value, 
int \texttt{*}status );] \listline
\textit{Set an MSG tuning parameter.}
\end{description}

\newpage
\section{\xlabel{subroutine_specifications}Subroutine Specifications \label{specs_sect}}
\subsection{\xlabel{message_system_subroutines}Message System subroutines}

\begin {small}
\sstroutine{
   MSG\_BELL
}{
   Deliver an ASCII BEL character
}{
   \sstdescription{
      A bell character and a new line is delivered to the user. If the
      user interface in use supports the ASCII BEL character, this routine
      will ring a bell and print a new line on the terminal.
   }
   \sstinvocation{
      CALL MSG\_BELL( STATUS )
   }
   \sstarguments{
      \sstsubsection{
         STATUS = INTEGER (Given and Returned)
      }{
         The global status.
      }
   }
}
\sstroutine{
   MSG\_BLANK
}{
   Output a blank line
}{
   \sstdescription{
      A blank line is output unless the conditional output filter
      is set to MSG\_\_QUIET. If STATUS is not set
      to SAI\_\_OK on entry, no action is taken. If an output error occurs,
      an error report is made and STATUS returned set to 
      MSG\_\_OPTER.
   }
   \sstinvocation{
      CALL MSG\_BLANK( STATUS )
   }
   \sstarguments{
      \sstsubsection{
         STATUS = INTEGER (Given and Returned)
      }{
         The global status.
      }
   }
}
\sstroutine{
   MSG\_FMTx
}{
   Assign a value to a message token (formatted)
}{
   \sstdescription{
      A given value is encoded using the supplied Fortran
      77 format field and the result assigned to the named message token.
      If the token is already defined, the result is appended to the
      existing token value. The given value may be one of the following 
      Fortran 77 data types and there is one routine provided for each data 
      type:

      \begin {center}
      \begin {tabular}{||l|c||}
      \hline
      \textit{Subroutine} & \textit{Fortran 77 Type}\\
      \hline
      MSG\_FMTD & DOUBLE PRECISION\\
      MSG\_FMTR & REAL\\
      MSG\_FMTI & INTEGER\\
      MSG\_FMTL & LOGICAL\\
      MSG\_FMTC & CHARACTER\\
      \hline
      \end {tabular}
      \end {center}

      If these subroutines fail, it will usually be apparent in any 
      messages which refer to this token.
   }
   \sstinvocation{
      CALL MSG\_FMTx( TOKEN, FORMAT, VALUE )
   }
   \sstarguments{
      \sstsubsection{
         TOKEN = CHARACTER $*$ ( $*$ ) (Given)
      }{
         The message token name.
      }
      \sstsubsection{
         FORMAT = CHARACTER $*$ ( $*$ ) (Given)
      }{
         The Fortran 77 FORMAT field used to encode the supplied value.
      }
      \sstsubsection{
         VALUE = Fortran 77 type (Given)
      }{
         The value to be assigned to the message token.
      }
   }
   \sstdiytopic{
      System-specific
   }{
      The precise effect of failures will depend upon the computer system 
      being used.
   }
}
\sstroutine{
   MSG\_IFLEV
}{
   Return the current filter level for conditional message output
}{
   \sstdescription{
      The value of the current filtering level set for conditional
      message output is returned. The collating sequence:

            MSG\_\_QUIET $<$ MSG\_\_NORM $<$ MSG\_\_VERB

      may be assumed.
   }
   \sstinvocation{
      CALL MSG\_IFLEV( FILTER )
   }
   \sstarguments{
      \sstsubsection{
         FILTER = INTEGER (Returned)
      }{
         The current message filtering level.
      }
   }
}
\sstroutine{
   MSG\_IFSET
}{
   Set the filter level for conditional message output
}{
   \sstdescription{
      The value of the message filtering level is set using the given
      filtering value. If no such level exists, then an error is 
      reported, STATUS returned set to MSG\_\_IFINV and the current 
      filtering level remains unchanged.
   }
   \sstinvocation{
      CALL MSG\_IFSET( FILTER, STATUS )
   }
   \sstarguments{
      \sstsubsection{
         FILTER = INTEGER (Given)
      }{
         The filtering level. This may be one of three levels defined
         in the MSG\_PAR include file:

         \sstitemlist{

            \sstitem
               MSG\_\_QUIET = quiet mode;

            \sstitem
               MSG\_\_NORM = normal mode;

            \sstitem
               MSG\_\_VERB = verbose mode.

	    \sstitem
               MSG\_\_DEBUG = debugging mode
         }
      }
      \sstsubsection{
         STATUS = INTEGER (Given and Returned)
      }{
         The global status.
      }
   }
}
\sstroutine{
   MSG\_LOAD
}{
   Expand and return a message
}{
   \sstdescription{
      Any tokens in the supplied message are expanded and the result is
      returned in the character variable supplied.
      If the expanded message is longer than the length of the supplied 
      character variable, the message is terminated with an ellipsis
      (\textit{i.e.}\ ``...'') but no error results.

      If STATUS is not set to SAI\_\_OK on entry, no action is taken
      except to annul any existing message tokens.
   }
   \sstinvocation{
      CALL MSG\_LOAD( PARAM, TEXT, OPSTR, OPLEN, STATUS )
   }
   \sstarguments{
      \sstsubsection{
         PARAM = CHARACTER $*$ ( $*$ ) (Given)
      }{
         The message name.
      }
      \sstsubsection{
         TEXT = CHARACTER $*$ ( $*$ ) (Given)
      }{
         The raw message text.
      }
      \sstsubsection{
         OPSTR = CHARACTER $*$ ( $*$ ) (Returned)
      }{
         The expanded message text.
      }
      \sstsubsection{
         OPLEN = INTEGER (Returned)
      }{
         The length of the expanded message.
      }
      \sstsubsection{
         STATUS = INTEGER (Given and Returned)
      }{
         The global status.
      }
   }
}
\sstroutine{
   MSG\_OUT
}{
   Output a message
}{
   \sstdescription{
      Any tokens in the supplied message are expanded and the result is output
      to the user unless the conditional output filter
      is set to MSG\_\_QUIET. All existing message tokens are then annulled.
      If the expanded message exceeds the maximum allowed size, it will be
      terminated by an ellipsis (\textit{i.e.}\ ``...'') but no error results.

      If STATUS is not set to SAI\_\_OK on entry,
      no action is taken except to annul existing message tokens.

      If an output error occurs, an error is reported and STATUS is returned 
      set to MSG\_\_OPTER.

      A call to MSG\_OUT is equivalent to a call to MSG\_OUTIF with the 
      message output priority set to MSG\_\_NORM. 
   }
   \sstinvocation{
      CALL MSG\_OUT( PARAM, TEXT, STATUS )
   }
   \sstarguments{
      \sstsubsection{
         PARAM = CHARACTER $*$ ( $*$ ) (Given)
      }{
         The message name.
      }
      \sstsubsection{
         TEXT = CHARACTER $*$ ( $*$ ) (Given)
      }{
         The message text.
      }
      \sstsubsection{
         STATUS = INTEGER (Given and Returned)
      }{
         The global status.
      }
   }
}
\sstroutine{
   MSG\_OUTIF
}{
   Conditionally deliver the text of a message to the user
}{
   \sstdescription{
      If the given message priority, PRIOR, and the current message 
      filtering level (set by MSG\_IFSET or MSG\_IFGET) indicate that 
      the message should be delivered, the message text is expanded
      and output to the user. If not, the message is discarded.
      In either case, any existing message tokens are then annulled.
      If the expanded message exceeds the maximum allowed size, it will be
      terminated by an ellipsis (\textit{i.e.}\ ``...'') but no error results.

      If STATUS is not set to SAI\_\_OK on entry, no action
      is taken except to annul existing message tokens.

      If an output error occurs, an error is reported and STATUS returned 
      set to MSG\_\_OPTER.
   }
   \sstinvocation{
      CALL MSG\_OUTIF( PRIOR, PARAM, TEXT, STATUS )
   }
   \sstarguments{
      \sstsubsection{
         PRIOR = INTEGER (Given)
      }{
         Message output priority. This may be one of three values defined
         in the MSG\_PAR include file:
         \begin{sloppypar}
         \sstitemlist{
            \sstitem
               MSG\_\_QUIET = Always output the message, regardless of the
               output filter setting;

            \sstitem
               MSG\_\_NORM = Output the message unless the current output
               filter is set to \hspace{1cm} MSG\_\_QUIET;

            \sstitem
               MSG\_\_VERB = Do not output the message unless the current
               output filter is set to \hspace{1cm} MSG\_\_VERB.

            \sstitem
               MSG\_\_DEBUG = Do not output the message unless the current
               output filter is set to \hspace{1cm} MSG\_\_DEBUG.
         } 
         \end{sloppypar}
         If any other value is given, STATUS is set to MSG\_\_INVIF
         and an error report made; no further action will be taken.
      }
      \sstsubsection{
         PARAM = CHARACTER $*$ ( $*$ ) (Given)
      }{
         The message name.
      }
      \sstsubsection{
         TEXT = CHARACTER $*$ ( $*$ ) (Given)
      }{
         The message text.
      }
      \sstsubsection{
         STATUS = INTEGER (Given and Returned)
      }{
         The global status.
      }
   }
}
\sstroutine{
   MSG\_RENEW
}{
   Renew any annulled message tokens in the current context
}{
   \sstdescription{
      Any message tokens which have been annulled (by a call to MSG\_OUT,
      ERR\_REP, \textit{etc.})\ are renewed unless any new token value has 
      been defined (using MSG\_SET\textit{x}, MSG\_FMT\textit{x} \textit{etc.})\ since 
      the previous tokens were annulled.

      The intended use of MSG\_RENEW is to allow a set of message token
      values to be used repeatedly in a sequence of messages without having
      to re-define them each time.
   }
   \sstinvocation{
      CALL MSG\_RENEW
   }
}
\sstroutine{
   MSG\_SETx
}{
   Assign a value to a message token (concise)
}{
   \sstdescription{
      A given value is encoded using a concise format and the
      result assigned to the named message token. If the token is
      already defined, the result is appended to the existing token value.
      The given value may be one of the following Fortran 77 data types
      and there is one routine provided for each data type:

      \begin {center}
      \begin {tabular}{||l|c||}
      \hline
      \textit{Subroutine} & \textit{Fortran 77 Type}\\
      \hline
      MSG\_SETD & DOUBLE PRECISION\\
      MSG\_SETR & REAL\\
      MSG\_SETI & INTEGER\\
      MSG\_SETL & LOGICAL\\
      MSG\_SETC & CHARACTER\\
      \hline
      \end {tabular}
      \end {center}

      If these subroutines fail, it will usually be apparent in any 
      messages which refer to this token.
   }
   \sstinvocation{
      CALL MSG\_SETx( TOKEN, CVALUE )
   }
   \sstarguments{
      \sstsubsection{
         TOKEN = CHARACTER $*$ ( $*$ ) (Given)
      }{
         The message token name.
      }
      \sstsubsection{
         VALUE = Fortran 77 type (Given)
      }{
         The value to be assigned to the message token.
      }
   }
   \sstdiytopic{
      System-specific
   }{
      The precise effect of failures will depend upon the computer system 
      being used.
   }
}
\sstroutine{
   MSG\_TUNE
}{
   Set an MSG tuning parameter
}{
   \sstdescription{
      The value of the MSG tuning parameter is set appropriately, according
      to the value given. MSG\_TUNE may be called multiple times for the same
      parameter.

      The given value can be overridden by setting an environment variable,
      MSG\_\textit{PARAM} (where \textit{PARAM} is the tuning parameter name
      in upper case), at run time.
   }
   \sstinvocation{
      CALL MSG\_TUNE( PARAM, VALUE, STATUS )
   }
   \sstarguments{
      \sstsubsection{
         PARAM = CHARACTER$*$($*$) (Given)
      }{
         The tuning parameter to be set (case insensitive).
      }
      \sstsubsection{
         VALUE = INTEGER (Given)
      }{
         The desired value (see Notes).
      }
      \sstsubsection{
         STATUS = INTEGER (Given and Returned)
      }{
         The global status.
      }
   }
   \sstnotes{
      1. The following values of PARAM may be used:
      \sstitemlist{
      \sstitem{'FILTER' Specifies the required MSG conditional message reporting
         level. VALUE may be 1, 2 or 3, corresponding with quiet, normal
         (the default) and verbose levels respectively.
      }
      \sstitem{'SZOUT' Specifies a maximum line length to be used in the line
         wrapping process. By default the message output by MSG is split into
         chunks of no more than the maximum line length, and each chunk is
         written on a new line. The split is made at word boundaries if
         possible. The default maximum line length is 79 characters.

         If VALUE is set to 0, no wrapping will occur. If it is set greater
         than 0, it specifies the maximum output line length.
      }        
      \sstitem{'STREAM' Specifies whether or not MSG should treat its output 
         unintelligently as a stream of characters.
         If VALUE is set to 0 (the default) all non-printing characters are
         replaced by blanks, and line wrapping occurs (subject to SZOUT). 
         If VALUE is set to 1, no cleaning or line wrapping occurs.
      }   
      \sstitem{'ENVIRONMENT' This is not a true tuning parameter name but causes
         the environment variables associated with all the true tuning 
         parameters to be used if set. If the environment variable is 
         not set, the tuning parameter is not altered. The VALUE argument
         is not used.
      }
      }          

      2. The tuning parameters for MSG and ERR operate partially at the EMS
         level and may conflict in their requirements of EMS.

      3. The use of SZOUT and STREAM may be affected by the message delivery
        system in use. For example there may be a limit on the the size of a
        line output by a Fortran WRITE and automatic line wrapping may occur.
        In particular, a NULL character will terminate a message delivered by
        the ADAM message system.
   }
}
\end {small}

\newpage
\subsection{\xlabel{error_system_subroutines}Error System subroutines}

\begin {small}
\sstroutine{
   ERR\_ANNUL
}{
   Annul the contents of the current error context
}{
   \sstdescription{
      Any error messages pending output in the current error context are
      annulled, \textit{i.e.}\ deleted. The values of any existing message tokens
      become undefined and STATUS is reset to SAI\_\_OK.
   }
   \sstinvocation{
      CALL ERR\_ANNUL( STATUS )
   }
   \sstarguments{
      \sstsubsection{
         STATUS = INTEGER (Returned)
      }{
         The global status: it is set to SAI\_\_OK on return.
      }
   }
}
\sstroutine{
   ERR\_BEGIN
}{
   Create a new error reporting environment
}{
   \sstdescription{
      Begin a new error reporting environment by marking a new error
      reporting context and then resetting the status argument to SAI\_\_OK.
      If ERR\_BEGIN is called with STATUS set to an error
      value, a check is made to determine if there are any messages
      pending output in the current context: if there are none, an
      error report to this effect is made on behalf of the calling
      application.
   }
   \sstinvocation{
      CALL ERR\_BEGIN( STATUS )
   }
   \sstarguments{
      \sstsubsection{
         STATUS = INTEGER (Given and Returned)
      }{
         The global status.
      }
   }
}
\sstroutine{
   ERR\_END
}{
   End the current error reporting environment
}{
   \sstdescription{
      Check if any error messages are pending output in the previous
      error reporting context. If there are, the current context is
      annulled and then released; if not, the current context is just
      released. The last reported status value is returned on exit.
   }
   \sstinvocation{
      CALL ERR\_END( STATUS )
   }
   \sstarguments{
      \sstsubsection{
         STATUS = INTEGER (Returned)
      }{
         The global status.
      }
   }
}
\sstroutine{
   ERR\_FACER
}{
   Assign a Starlink facility status message to a token
}{
   \sstdescription{
      The text of the error message associated with a Starlink facility status
      value, STATUS, is assigned to the  named message token. If the token
      is already defined, the message is appended to the existing token value.
   }
   \sstinvocation{
      CALL ERR\_FACER( TOKEN, STATUS )
   }
   \sstarguments{
      \sstsubsection{
         TOKEN = CHARACTER $*$ ( $*$ ) (Given)
      }{
         The message token name.
      }
      \sstsubsection{
         STATUS = INTEGER (Given)
      }{
         The Starlink status value.
      }
   }
   \sstdiytopic{
      System-specific
   }{
      The messages generated using this facility may differ slightly
      depending on the computer system upon which the library is implemented.
   }
}
\sstroutine{
   ERR\_FIOER
}{
   Assign a Fortran I/O error message to a token
}{
   \sstdescription{
      The text of the error message associated with the Fortran I/O status
      value, IOSTAT, is assigned to the  named message token. If the token
      is already defined, the message is appended to the existing token value.
   }
   \sstinvocation{
      CALL ERR\_FIOER( TOKEN, IOSTAT )
   }
   \sstarguments{
      \sstsubsection{
         TOKEN = CHARACTER $*$ ( $*$ ) (Given)
      }{
         The message token name.
      }
      \sstsubsection{
         IOSTAT = INTEGER (Given)
      }{
         The Fortran I/O status value.
      }
   }
   \sstdiytopic{
      System-specific
   }{
      The messages generated using this facility will depend on the 
      computer system upon which the library is implemented.
   }
}
\sstroutine{
   ERR\_FLBEL
}{
   Deliver an ASCII BEL and flush the current error context
}{
   \sstdescription{
      An ASCII BEL character is delivered to the user and then all
      pending error messages in the current error context are delivered
      using a call to ERR\_FLUSH. On successful completion,
      the error context is annulled and STATUS reset to
      SAI\_\_OK.

      If no error messages have been reported in the current error 
      context, a warning message is generated but no error results.

      If an error occurs during output of the error messages, the
      error context is not annulled and STATUS is returned
      set to ERR\_\_OPTER.
   }
   \sstinvocation{
      CALL ERR\_FLBEL( STATUS )
   }
   \sstarguments{
      \sstsubsection{
         STATUS = INTEGER (Returned)
      }{
         The global status: it is set to SAI\_\_OK on return if the
         error message output is successful; if not, it is set to
         ERR\_\_OPTER.
      }
   }
}
\sstroutine{
   ERR\_FLUSH
}{
   Flush the current error context
}{
   \sstdescription{
      Ensure that all pending error messages in the current error
      context have been output to the user. On successful completion, the
      error context is annulled and STATUS reset to SAI\_\_OK

      If no error messages have been reported in the current error 
      context, a warning message is generated but no error results.

      If an error occurs during output of the error messages, the
      error context is not annulled and STATUS is returned
      set to ERR\_\_OPTER.
   }
   \sstinvocation{
      CALL ERR\_FLUSH( STATUS )
   }
   \sstarguments{
      \sstsubsection{
         STATUS = INTEGER (Returned)
      }{
         The global status: it is set to SAI\_\_OK on return if the
         error message output is successful; if not, it is set to 
         ERR\_\_OPTER.
      }
   }
}
\sstroutine{
   ERR\_LEVEL
}{
   Inquire the current error context level
}{
   \sstdescription{
      Return the number of context markers set in the error message table.
   }
   \sstinvocation{
      CALL ERR\_LEVEL( LEVEL )
   }
   \sstarguments{
      \sstsubsection{
         LEVEL = INTEGER (Returned)
      }{
         The error context level: all values greater than one indicate 
         the deferral of reported error messages.
      }
   }
}
\sstroutine{
   ERR\_LOAD
}{
   Return error messages from the current error context
}{
   \sstdescription{
      On the first call of this routine, the error table for the current 
      error context is copied into a holding area, the current context
      is annulled and the first message in the holding area is returned.
      Thereafter, each time the routine is called, the next message from 
      the holding area is returned.
      The argument PARAM is the returned message name and PARLEN the length 
      of the message name.
      OPSTR is the returned error message text and OPLEN is the length of
      the error message.
      If the message text is longer than the declared length of OPSTR, then the
      message is truncated with an ellipsis, \textit{i.e.} ``\ldots'', but no 
      error results.

      The status associated with the returned message is returned in STATUS
      until there are no more messages to return -- then STATUS is set to
      SAI\_\_OK, PARAM and OPSTR are set to blanks and PARLEN and OPLEN to 1.
      If there are no messages pending on the first call, a warning message
      is generated and returned with STATUS set to EMS\_\_NOMSG.

      After STATUS has been returned SAI\_\_OK, the whole process is repeated
      for subsequent calls.
   }
   \sstinvocation{
      ERR\_LOAD( PARAM, PARLEN, OPSTR, OPLEN, STATUS )
   }
   \sstarguments{
      \sstsubsection{
         PARAM = CHARACTER $*$ ( $*$ ) (Returned)
      }{
         The error message name.
      }
      \sstsubsection{
         PARLEN = INTEGER (Returned)
      }{
         The length of the error message name.
      }
      \sstsubsection{
         OPSTR = CHARACTER $*$ ( $*$ ) (Returned)
      }{
         The error message -- or blank if there are no more messages.
      }
      \sstsubsection{
         OPLEN = INTEGER (Returned)
      }{
         The length of the error message.
      }
      \sstsubsection{
         STATUS = INTEGER (Returned)
      }{
         The status associated with the returned error message: 
         it is set to SAI\_\_OK when there are no more messages.
      }
   }
}
\sstroutine{
   ERR\_MARK
}{
   Mark (start) a new error context
}{
   \sstdescription{
      Begin a new error reporting context. Delivery of subsequently
      reported error messages is deferred and the messages held in the
      error table. Calls to ERR\_ANNUL, ERR\_FLUSH and ERR\_LOAD will only
      flush or annul the contents of the error table within this new
      context.
   }
   \sstinvocation{
      CALL ERR\_MARK
   }
}
\sstroutine{
   ERR\_REP
}{
   Report an error message
}{
   \sstdescription{
      Report an error message. According to the error context, the 
      error message is either sent to the user or retained in the 
      error table. The latter case allows the application to take
      further action before deciding if the user should receive the 
      message. On exit the values associated with any existing message 
      tokens are left undefined. On successful completion, STATUS is
      returned unchanged.

      If STATUS is set to SAI\_\_OK on entry, an error report to this
      effect is made on behalf of the application and STATUS is returned 
      set to ERR\_\_BADOK; the given message is still reported and has
      status ERR\_\_UNSET associated with it.

      If an output error occurs, STATUS is returned set to ERR\_\_OPTER. 
   }
   \sstinvocation{
      CALL ERR\_REP( PARAM, TEXT, STATUS )
   }
   \sstarguments{
      \sstsubsection{
         PARAM = CHARACTER $*$ ( $*$ ) (Given)
      }{
         The error message name.
      }
      \sstsubsection{
         TEXT = CHARACTER $*$ ( $*$ ) (Given)
      }{
         The error message text.
      }
      \sstsubsection{
         STATUS = INTEGER (Given and Returned)
      }{
         The global status: it is left unchanged on successful completion,
         or is set an appropriate error value if an internal error has 
         occurred.
      }
   }
}
\sstroutine{
   ERR\_RLSE
}{
   Release (end) the current error context
}{
   \sstdescription{
      Release a ``mark'' in the error message table, returning 
      the Error Reporting System to the previous error context. Any error 
      messages pending output will be passed to this previous context, 
      \emph{not} annulled. When the context level reaches the base level,
      all pending messages will be delivered to the user.
   }
   \sstinvocation{
      CALL ERR\_RLSE
   }
}
\sstroutine{
   ERR\_STAT
}{
   Inquire the last reported error status
}{
   \sstdescription{
      The current error context is checked for any error messages reported
      since the context was created or last annulled.
      If none exist, STATUS is returned set to SAI\_\_OK. 
      If there are any such messages, STATUS is returned set to the status 
      value associated with the last reported error message.
   }
   \sstinvocation{
      CALL ERR\_STAT( STATUS )
   }
   \sstarguments{
      \sstsubsection{
         STATUS = INTEGER (Returned)
      }{
         The last reported error status within the current error context;
         if none exist, STATUS is returned set to SAI\_\_OK.
      }
   }
}
\sstroutine{
   ERR\_SYSER
}{
   Assign an operating system error message to a token
}{
   \sstdescription{
      The text of the error message associated with the operating system
      status value, SYSTAT, is assigned to the named message token. If the 
      token is already defined, the message is appended to the existing 
      token value.
   }
   \sstinvocation{
      CALL ERR\_SYSER( TOKEN, SYSTAT )
   }
   \sstarguments{
      \sstsubsection{
         TOKEN = CHARACTER $*$ ( $*$ ) (Given)
      }{
         The message token name.
      }
      \sstsubsection{
         SYSTAT = INTEGER (Given)
      }{
         The operating system status value.
      }
   }
   \sstdiytopic{
      System-specific
   }{
      The messages generated using this facility will depend on the 
      computer system upon which the library is implemented.
   }
}
\sstroutine{
   ERR\_TUNE
}{
   Set an ERR tuning parameter
}{
   \sstdescription{
      The value of the ERR tuning parameter is set appropriately, according
      to the value given. ERR\_TUNE may be called multiple times for the same
      parameter.

      The given value can be overridden by setting an environment variable,
      ERR\_\textit{PARAM} (where \textit{PARAM} is the tuning parameter name
      in upper case), at run time.

      The routine will attempt to execute regardless of the given value of
      STATUS. If the given value is not SAI\_\_OK, then it is left unchanged,
      even if the routine fails to complete. If the STATUS is SAI\_\_OK on
      entry and the routine fails to complete, STATUS will be set and an
      error report made.
   }
   \sstinvocation{
      CALL ERR\_TUNE( PARAM, VALUE, STATUS )
   }
   \sstarguments{
      \sstsubsection{
         PARAM = CHARACTER$*$($*$) (Given)
      }{
         The tuning parameter to be set (case insensitive).
      }
      \sstsubsection{
         VALUE = INTEGER (Given)
      }{
         The desired value (see Notes).
      }
      \sstsubsection{
         STATUS = INTEGER (Given and Returned)
      }{
         The global status.
      }
   }
   \sstnotes{
      1. The following values of PARAM may be used:
      \sstitemlist{
      \sstitem{'SZOUT' Specifies a maximum line length to be used in the line
           wrapping process. By default the message to be output is split into
           chunks of no more than the maximum line length, and each chunk is
           written on a new line. The split is made at word boundaries if 
           possible.
           The default maximum line length is 79 characters.

           If VALUE is set to 0, no wrapping will occur. If it is set greater
           than 6, it specifies the maximum output line length. Note that the
           minimum VALUE is 7, to allow for exclamation marks and indentation.
      }
      \sstitem{'STREAM' Specifies whether or not ERR should treat its output
           unintelligently as a stream of characters.
           If VALUE is set to 0 (the default) all non-printing characters are
           replaced by blanks, and line wrapping occurs (subject to SZOUT).
           If VALUE is set to 1, no cleaning or line wrapping occurs.
      }
      \sstitem{'REVEAL' Allows the user to display all error messages cancelled
           when ERR\_ANNUL is called. This is a diagnostic tool which enables
           the programmer to see all error reports, even those 'handled' 
           by the program. If VALUE is set to 0 (the default) annulling 
           occurs in the normal way. If VALUE is set to 1, the message 
           will be displayed.
      }
      \sstitem{'ENVIRONMENT' This is not a true tuning parameter name but causes
           the environment variables associated with all the true tuning 
           parameters to be used if set.  If the environment variable is
           not set, the tuning parameter is not altered. The VALUE argument
           is not used.
      }
      }

      2. The tuning parameters for MSG and ERR operate partially at the EMS
        level and may conflict in their requirements of EMS.

      3. The use of SZOUT and STREAM may be affected by the message delivery
        system in use. For example there may be a limit on the the size of a
        line output by a Fortran WRITE and automatic line wrapping may occur.
        In particular, a NULL character will terminate a message delivered by
        the ADAM message system.

      4. With REVEAL, messages are displayed at the time of the ANNUL.
        As REVEAL operates at the EMS level they are displayed with Fortran
        WRITE statements so, depending upon the delivery mechanism for
        normal messages, they may appear out of order.
   }
}
\end {small}
\rule{\textwidth}{0.5mm}

\subsection{\xlabel{deprecated_routine_err_out}Deprecated Routine ERR\_OUT \label{out_sect}}
Purely for compatibility with previous versions of ERR, the routine
ERR\_OUT is provided. It should not be used in any new code -- usually a
call to ERR\_REP is all that is required. If it is essential that the message
be delivered to the user immediately, ERR\_REP should be followed by a call to
ERR\_FLUSH. 

\begin {small}
\sstsubroutine{
   ERR\_OUT
}{
   Report a message then flush and annul the current error context.
}{
   \sstdescription{
      The message is reported using ERR\_REP and then the current
      error context is flushed and annulled using ERR\_FLUSH.
      On successful completion, STATUS is reset to SAI\_\_OK.

      If STATUS is set to SAI\_\_OK on entry, a warning message will be 
      generated but no error results.

      If an error occurs during output of the error messages, the error
      context is not annulled and STATUS is returned set to ERR\_\_OPTER.
   }
   \sstinvocation{
      CALL ERR\_OUT( PARAM, TEXT, STATUS )
   }
   \sstarguments{
      \sstsubsection{
         PARAM = CHARACTER * ( * ) (Given)
      }{
         The error message name.
      }
      \sstsubsection{
         TEXT = CHARACTER * ( * ) (Given)
      }{
         The error message text.
      }
      \sstsubsection{
         STATUS = INTEGER (Given and Returned)
      }{
        The global status. STATUS should not be SAI\_\_OK on entry;
        it is reset to SAI\_\_OK on exit unless an output error occurs 
        in which case it will be set to ERR\_OPTER.
      }
   }
}
\end{small}
\rule{\textwidth}{0.5mm}

\newpage
\section{\xlabel{using_msg_and_err_within_adam}Using MSG and ERR within ADAM \label{adam_sect}}

\subsection{\xlabel{overview}Overview}

If the procedures recommended in this document have been adhered to, 
stand-alone applications which use the Message and Error Reporting Systems can
be converted to run within the ADAM environment without change to the message
and error reporting code.
However, the use of the Message and Error Systems within ADAM applications
provides access to facilities which cannot exist in the stand-alone version.
This section describes these facilities and how they can be used.
In what follows, a prior knowledge of ADAM applications programming to the
level of
\xref{SG/4}{sg4}{} (ADAM -- The Starlink Software Environment) is assumed.


\subsection{\xlabel{using_msg_within_adam_applications}Using MSG within ADAM applications \label{adam_msg}}

\subsubsection{\xlabel{message_parameters}Message parameters \label{mespar_sect}}
In calls to the MSG output subroutines in ADAM applications, the
message name is the name of an ADAM  message parameter which
can be associated with message text specified in the ADAM interface file 
instead of the text given in the subroutine argument.
Message text specified in an interface file may include message tokens in the
normal way.

If the message parameter is not specified in the interface file, 
the text given in the argument list is used.

Here is an example of using MSG\_OUT within an ADAM application:

\begin {small}
\begin{verbatim}
      CALL MSG_OUT( 'RD_TAPE', 'Reading tape.', STATUS )
\end{verbatim}
\end {small}

This statement will result in the message:

\begin {quote}
\begin {small}
\begin{verbatim}
Reading tape.
\end{verbatim}
\end {small}
\end {quote}

If the message parameter, `RD\_TAPE', is associated with a  different text
string in the interface module, \textit{e.g.}

\begin {small}
\begin{verbatim}
message RD_TAPE
   text 'The program is currently reading the tape, please wait.'
endmessage
\end{verbatim}
\end {small}

then the output message would be the one defined in the interface file, {\em
i.e.}

\begin {quote}
\begin {small}
\begin{verbatim}
The program is currently reading the tape, please wait.
\end{verbatim}
\end {small}
\end {quote}

This facility enables ADAM applications to conveniently support foreign
languages. 
However, it is recommended that message parameters used in applications
software are not normally defined in the interface file.
If message parameters are defined in the interface file of an application,
then it is clearly necessary to ensure that the text associated with each
message parameter imparts essentially the same information as the text used
within the program code, even if they are in different languages. 


\subsubsection{\xlabel{parameter_references}Parameter references}

It is often necessary to refer to an ADAM program parameter in a message. 
There are two kinds of reference required: 

\begin {itemize}
\item The keyword associated with a parameter. 
The keyword is the name for a parameter that the user sees in the 
documentation of an application. 
The keyword-parameter association is held in the interface file and is
not directly available to the application. 

\item The name of an object, device or file associated with a parameter. 
This will usually only be relevant for non-primitive parameters 
(\textit{e.g.}\ file names), although primitive parameters (\textit{i.e.}\ any 
numerical, CHARACTER or LOGICAL data type) may have an associated object -- 
often their parameter file entry.
\end {itemize}

Parameter references can be included in the text of a message by prefixing 
their names with the appropriate escape character. 
To include the keyword associated with a parameter, its name is prefixed with
the percent escape character, ``\%'', \textit{e.g.}

\begin {small}
\begin{verbatim}
      CALL MSG_OUT( 'ET_RANGE', '%ET parameter is ignored.', STATUS )
\end{verbatim}
\end {small}

\begin {sloppypar}
Here, the parameter `ET' might be associated in the interface file 
with the keyword ``EXPOSURE\_TIME'', \textit{e.g.}
\end {sloppypar}

\begin {quote}
\begin {small}
\begin{verbatim}
parameter ET
   type '_INTEGER'
   keyword 'EXPOSURE_TIME'
   prompt 'Exposure time required'
endparameter
\end{verbatim}
\end {small}
\end {quote}

In this case, the resultant output would be

\begin {quote}
\begin {small}
\begin{verbatim}
EXPOSURE_TIME parameter is ignored.
\end{verbatim}
\end {small}
\end {quote}

To include the name of an object, device or file associated with a parameter, 
the parameter name is prefixed with the dollar escape character, ``\$'',
\textit{e.g.}

\begin {small}
\begin{verbatim}
      CALL MSG_OUT( 'DS_CREATE', 'Creating $DATASET.', STATUS )
\end{verbatim}
\end {small}

If the parameter `DATASET' is associated with the object called ``SWP1234'' 
then this would produce the output

\begin {quote}
\begin {small}
\begin{verbatim}
Creating SWP1234.
\end{verbatim}
\end {small}
\end {quote}


\subsubsection{\xlabel{using_program_parameters_as_tokens}Using program parameters as tokens}

It is sometimes necessary to refer to a program parameter in a message, where
its name is contained in a variable. 
In order to incorporate its reference into a message, a message token can be
associated with its name. 
Here is an example of how this would be coded:

\begin {small}
\begin{verbatim}
      CALL MSG_SETC( 'PAR', PNAME )
      CALL MSG_OUT( 'PAR_UNEXP', '%^PAR=0.0 unexpected', STATUS )
\end{verbatim}
\end {small}

In this example, the escape sequence ``$^\wedge$'' prefixes the token name,
'PAR'. 
The sequence '$^\wedge$PAR' gets replaced by the contents of the character
string contained in the variable PNAME; this, being prefixed by ``\%'', then
gets replaced in the final message by the keyword associated with the
parameter. 


\subsubsection{\xlabel{reserved_token_status}Reserved token STATUS}

The token name ``STATUS'' is reserved for use by the Error System and should
not be used in Message System calls. It will never produce a useful message
from an MSG subroutine.


\subsubsection{\xlabel{getting_the_conditional_output_level}Getting the conditional output level}

In addition to the subroutine for setting the filter level for conditional
message output, \textit{i.e.}\ MSG\_IFSET, the ADAM version of MSG also provides 
subroutine MSG\_IFGET to get a character string from the parameter system and
use this to set the filter level
The subroutine has the calling sequence:

\begin {small}
\begin{verbatim}
      CALL MSG_IFGET( PNAME, STATUS )
\end{verbatim}
\end {small}

where PNAME is the parameter name.
It is recommended that one parameter name is used universally for this purpose,
namely MSG\_FILTER, in order to clarify the interface file entries.
The three acceptable strings for MSG\_FILTER are:

\begin {quote}
\begin {description}
\item [QUIET] -- representing MSG\_\_QUIET;
\item [NORMAL] -- representing MSG\_\_NORM;
\item [VERBOSE] -- representing MSG\_\_VERB.
\item [DEBUG] -- representing MSG\_\_DEBUG
\end {description}
\end {quote}
Abbreviations are accepted but
any other value will result in an error report and STATUS being
returned set to MSG\_\_INVIF.


\subsubsection{\xlabel{synchronising_message_output}Synchronising message output}

Graphical and textual output from an ADAM application can have problems with
synchronisation.
This is because the message output arrives at the terminal via the command
process and is buffered, whereas the graphical output is sent directly to the
terminal.
The effect of this is that graphical output to the terminal can get corrupted by
message output.
This problem can be avoided by using the subroutine MSG\_SYNC, which flushes
the textual output buffer of the command process:

\begin {small}
\begin{verbatim}
      CALL MSG_SYNC( STATUS )
\end{verbatim}
\end {small}

MSG\_SYNC should be called immediately before any graphical output to avoid any
corruption. 


\subsection{\xlabel{using_err_within_adam_applications}Using ERR within ADAM applications \label{adam_err}}

\subsubsection{\xlabel{error_message_parameters}Error message parameters}

In calls to the subroutine ERR\_REP in ADAM applications, the error name is the
name of an ADAM message parameter which may be associated with message text
specified in the interface file instead of the text supplied in the subroutine
call.
The system operates in the same way as for
\htmlref{MSG}{mespar_sect}\latex{ (see \S\ref{mespar_sect})}.

\subsubsection{\xlabel{initial_error_context_level}Initial error context level}
In the non-ADAM, stand-alone, version of the Error System the initial error
context level is set to one.
Thus, before any calls to ERR\_MARK have been made, calls to ERR\_LEVEL will 
return a value of one and any error reports made at this level to be delivered
immediately to the user.
However, for ADAM applications, error reporting is deferred before
the user's application code is called.
Thus calls to ERR\_LEVEL made before any calls to ERR\_MARK will return a 
value of two to the application,

Any remaining error reports are delivered to the user by the ``fixed-part''
of the ADAM application after execution of
the user's application code.


\subsubsection{\xlabel{reserved_token_status}Reserved token STATUS \label{restok}}
Historically, considerable use in ADAM was made of the VAX/VMS MESSAGE utility, 
both to generate globally unique error codes and to associate each error code 
with the text of an error message.
The utility is also used by the VMS operating system for reporting 
its own messages.
This has resulted in the idea of
a reserved message token, STATUS, which would expand to the message associated
with the given status value.

Every call to ERR\_REP results in the status value given being converted into
its associated VAX/VMS message, and then this message is associated with the 
token STATUS.
Hence the global status may be used to construct an error message as follows:

\begin {small}
\begin{verbatim}
      CALL ERR_REP( 'DSCALE_BAD', '%DSCALE: ^STATUS', STATUS )
\end{verbatim}
\end {small}

This method of constructing error messages makes the assumption that \emph{all}
status values and operating system error numbers form part of a homogeneous 
set which can easily be associated with some appropriate message text.
It is therefore not portable from VAX/VMS to other computer systems.
If the STATUS token is used on other systems, an attempt will be made to
translate it as a Starlink facility status value.

Subroutines following the
\htmlref{recommended error reporting strategy}{when_sect}
\latex{ outlined in \S\ref{when_sect}} will 
always make an accompanying error report when the returned status is set to an
error value.
As a result, error reports based upon the reserved token STATUS are rarely
necessary and limit the portability of ADAM software.

New code should not use the reserved token STATUS.
If it is required to include the message associated with an error code in an
error report, subroutine
\htmlref{ERR\_FACER}{ERR_FACER} should be used to create some other token.

\subsubsection{\xlabel{returning_the_status_to_the_environment}Returning the status to the environment}
ADAM applications are Fortran subroutines which are called by the software
environment. 
An ADAM application has the schematic form 
\begin {small}
\begin{verbatim}
      SUBROUTINE APPLIC( STATUS )

      ...

      <application code>

      ...

      END
\end{verbatim}
\end {small}

where the argument STATUS is given as SAI\_\_OK. 
When the application ends and returns control to the environment, the final
value of STATUS will reflect whether or not the application finished on an
error condition.
This provides the environment with a status to be associated with the
application as a whole. 
The exit status of the application can be used by the environment to decide
what to do next. 
If a status value other than SAI\_\_OK is returned to the ADAM environment,
then any pending error messages are output. 
Applications which fail should therefore return an error status.
This recommendation does not, in fact, differ from that for any other subroutine
obeying the
\htmlref
{inherited status strategy}{inherited_status}\latex{ see \S\ref{inherited_status})}.


\subsection{\xlabel{using_the_error_message_service_for_adam_system_programming}Using the Error Message Service for ADAM system programming}
The Message and Error Systems are intended for use when writing stand-alone and
ADAM applications. 

For ADAM system programming, where no
assumptions may be made about the working order of either the ADAM parameter 
or data systems,
another subroutine library, 
\xref{EMS}{ssn4}{abstract} (the Error Message Service)
is available.
The subroutine calling sequences for EMS are very similar to those of
the Message and Error Systems (indeed many MSG and ERR routines are implemented
as straight-through calls to EMS -- hence the occurrence of EMS error messages 
from MSG or ERR routines). 
\latex{EMS is described in SSN/4.}
The Error Message Service library is intended specifically for use in ADAM 
system or general low level subroutine libraries.
It should not be used in applications software.

\newpage
\subsection{\xlabel{adamspecial_subroutine_specifications}ADAM-special subroutine specifications}
\small
\sstsubroutine{
   MSG\_IFGET
}{
   Get the filter level for conditional message output from the ADAM
   parameter system
}{
   \sstdescription{
      Obtain a value for the named parameter and use it to set the
      value for the filter level for conditional message output. 
      It is recommended that one parameter name is used universally for 
      this purpose, namely MSG\_FILTER, in order to clarify the interface
      file entries.
      The parameter must be of type \_CHAR and
      the three acceptable strings it may take are:
      \sstitemlist{
         \sstitem
            QUIET -- representing MSG\_\_QUIET;

         \sstitem
            NORMAL -- representing MSG\_\_NORM;

         \sstitem
            VERBOSE -- representing MSG\_\_VERB.

         \sstitem
            DEBUG -- representing MSG\_\_DEBUG.
      }
      MSG\_IFGET accepts abbreviations of these strings; any other value 
      will result in an error report and the status value being returned 
      set to MSG\_\_INVIF. If an error occurs getting the parameter value,
      the appropriate status value is returned and an additional error 
      report is made.
   }
   \sstinvocation{
      CALL MSG\_IFGET( PNAME, STATUS )
   }
   \sstarguments{
      \sstsubsection{
         PNAME = CHARACTER $*$ ( $*$ ) (Given)
      }{
         The filtering level parameter name.
      }
      \sstsubsection{
         STATUS = INTEGER (Given and Returned)
      }{
         The global status.
      }
   }
}


\sstsubroutine{
   MSG\_SYNC
}{
   Synchronise message output via the user interface
}{
   \sstdescription{
      This performs a synchronisation handshake with the user interface.
      This is required if the current task has been outputting messages
      via the user interface and now wants to use a graphics cursor on the
      command device. If a synchronisation error occurs, then an error
      report is made and the status value is returned set to MSG\_\_SYNER.
   }
   \sstinvocation{
      CALL MSG\_SYNC( STATUS )
   }
   \sstarguments{
      \sstsubsection{
         STATUS = INTEGER (Given and Returned)
      }{
         The global status: it is returned set to MSG\_\_SYNER on error.
      }
   }
}

\rule{\textwidth}{0.5mm}
\normalsize
\newpage
\section{\xlabel{portability}Portability}

\subsection{\xlabel{overview}Overview}

This section discusses the portability of MSG and ERR, including the coding 
standard adopted for MSG and ERR and a list of those Starlink packages which 
need to be ported to the target machine before a port of MSG or ERR can proceed.


\subsection{\xlabel{coding_and_porting_prerequisites}Coding and porting prerequisites}

The standard of Fortran used for the coding of MSG and ERR is fundamentally
Fortran 77, using the Starlink Fortran coding conventions described in 
\xref{SGP/16}{sgp16}{}.
Several common extensions to the Fortran 77 standard are used in source 
code for MSG and ERR, they are as follows:

\begin {itemize}
\item end-of-line comments using the ``!'' symbol;
\item subprogram names may be longer than six characters (but are 
always shorter than ten characters);
\item subprogram names include the ``\_'' symbol;
\item symbolic constant names may be longer than six characters (but are always
shorter than eleven characters);
\item symbolic constant names may include the ``\_'' symbol;
\item the full ASCII character set is assumed in character constants.
\end {itemize}

To use MSG and ERR on any computer system, the Starlink Error Message Service,
\xref{EMS}{ssn4}{abstract} \latex{(see SSN/4)}
and Character Handling Routines, 
\xref{CHR}{sun40}{abstract} \latex{(see SUN/40)}
must be available.


\subsection{\xlabel{operating_system_specific_routines}Operating system specific routines \label{dep_sect}}
The following routines have system-specific features and may produce differing
results on different platforms. System-specific code (apart from minor format
differences on WRITE statements) is all in the equivalent EMS routines.
\begin{description}
\item [ERR\_FACER] Assign a Starlink facility error message to a token.

Calls its EMS counterpart which is system-specific -- for more details see 
\xref{SSN/4}{ssn4}{operating_system_specific_routines}.
\item [ERR\_FIOER] Assign a Fortran I/O error message to a token.

Calls its EMS counterpart which will generally need to be rewritten for each 
new target platform.
Current versions have the appropriate messages hardwired into the code.
For more details see \xref{SSN/4}{ssn4}{operating_system_specific_routines}.
\item [ERR\_SYSER] Assign an operating system error message to a token.

Calls its EMS counterpart which is platform specific.
For more details see \xref{SSN/4}{ssn4}{operating_system_specific_routines}.
\item [ERR\_FMTx] Assign a value to a message token (formatted).

These routines call their EMS counterpart and behave differently on different 
systems in the event of an error. 
Sometimes the errors are reported by Fortran and can be trapped but 
sometimes they are not. If they are reported, the token remains unset but
if they are not reported, the character string generated can differ.
\end {description}

\section{\xlabel{calculating_globally_unique_error_codes}Calculating Globally Unique Error Codes \label{stat_sect}}
Starlink facility error codes will normally be generated by the Starlink
\xref{MESSGEN}{sun185}{} utility on UNIX\latex{ (see SUN/185)}.

This section presents an alternative method for calculating compatible
error status codes for subroutine libraries.
In order to be used effectively, it requires a Fortran compiler capable of four
byte integer representation.
If this is not the case, then the status values generated will \emph{not} be
globally unique.

The error codes are calculated using the equation:

\begin {equation}
CODE \,= \, 134250498 \,\, + \,\, 65536\times <fac> \,\, + \,\,  8\times <mes> 
\end {equation}

Here, $<mes>$ is the message number (in the range 1 to 4095) assigned to the
error condition by the author of the subroutine library, and $<fac>$ is the
facility number (in the range 1 to 2047) allocated to this subroutine library.
Developers wishing to have facility numbers allocated to subroutine libraries 
should contact the Starlink Software Librarian 
(\textit{i.e.}\ ussc@star.bnsc.rl.ac.uk).

\section{\xlabel{changes_and_new_features_in_version_1_4}Changes and New Features in Version 1.4}
\begin{itemize}
\item New routine ERR\_FACER is released to complete the distinction between
various types of error code and the messages associated with them.
(\S\ref{fio_sect} and Appendix \ref{specs_sect}.)
\item The Starlink MESSGEN utility is now the preferred way of generating 
error codes and include files. (\S\ref{def_sect} and Appendix \ref{stat_sect})
\item The deprecated reserved token, STATUS, is now treated as a Starlink 
facility error message. (\S\ref{restok})
\item This document has been modified to reflect the above changes.
\end{itemize}

\section{\xlabel{changes_and_new_features_in_version_1_41}Changes and New Features in Version 1.4-1}
Version 1.4-1 contains no functional changes. It merely updates the makefiles
and other scripts according to the latest templates.

This document is revised slightly to remove descriptions of the system on VMS
(which is no longer updated). A hypertext version of the document is now
available.

\section{\xlabel{changes_and_new_features_in_version_1_5}Changes and New Features in Version 1.5}
\begin{itemize}
\item A preliminary 
\htmlref{C interface}{the_c_interface}\latex{ (see Appendix
\ref{the_c_interface})} has been added.
\item MSG\_LOAD has been modified to remove an intermediate buffer which
restricted the length of the message which could be handled.
\item In \texttt{msg\_par}, the maximum size of a message 
(\texttt{MSG\_\_SZMSG}) has been increased from 200 to 300 characters.
\item In \texttt{err\_par}, the maximum size of a message parameter name has
been corrected to 15 (from 200).
\item An installation test for the C interface has been added to the test
target in the makefile.
\end{itemize}

\section{\xlabel{changes_and_new_features_in_version_1_6}Changes and New Features in Version 1.6}
Subroutine 
\htmlref{MSG\_TUNE}{MSG_TUNE}
and
\htmlref{ERR\_TUNE}{ERR_TUNE}
have been added to control various aspect of output.

Shared libraries are now produced for Linux as well as Solaris.

This document is revised to add a section on 
\htmlref{tuning}{tuning} and to improve the look of the hypertext form.

\section{\xlabel{changes_and_new_features_in_version_1_7}Changes and New Features in Version 1.7}
This version has been substantially revised internally to use version 2.0 of
\xref{EMS}{ssn4}{},
which is implemented in C.

MERS now uses only the public interface to EMS (formerly a number of internal
EMS routines were called).

Because the lowest level is now in C, output of error message at the base
error reporting level is now done by C printf commands. On some systems this
can lead to incorrect sequencing of messages if other output is done by
Fortran WRITE.

A bug which caused the linker to report subroutine MSG\_IFSET missing if
MSG\_IFGET was called, has been corrected.

This section has been added to this document.

\section{\xlabel{changes_in_version_1_7_1}Changes in Version 1.7-1}

Corrects a problem using the reserved token STATUS if another token precedes
it in the message, and includes a minor update to SUN/104.

\section{\xlabel{changes_in_version_1_8}Changes in Version 1.8}

MSG1_GREF is completely revised to simplify its interface with the parameter
system, using a new routine SUBPAR_GREF. This allows the underlying parameter
system to be replaced more easily. Suboutine MSG1_GLOC is now not requited and
has been removed. This Version of MERS requires PCS Version 4.1 or later.

\section{\xlabel{changes_in_version_1_8_1}Changes in Version 1.8-1}

Builds with the Gnu auto tools and released under the GPL.

\section{\xlabel{changes_in_version_1_8_2}Changes in Version 1.8-2}

Added MSG\_\_DEBUG reporting level.


% ? End of main text
\end{document}

