\documentclass[twoside,11pt,nolof]{starlink}

% ? Specify used packages
% ? End of specify used packages

% -----------------------------------------------------------------------------
% ? Document identification
% Fixed part
\stardoccategory    {Starlink User Note}
\stardocinitials    {SUN}
\stardocsource      {sun\stardocnumber}

% Variable part - replace [xxx] as appropriate.
\stardocnumber      {48.8}
\stardocauthors     {Nicholas Eaton\\Brian McIlwrath}
\stardocdate        {24 April 2013}
\stardoctitle   {AGI --- Applications Graphics Interface Library}
\stardocversion     {Version 2.2}
\stardocmanual      {Programmer's Manual}
\stardoccopyright {Copyright \copyright 2000, Central Laboratory of the Research Councils}

\stardocabstract {
  AGI is a graphics database system that can be used to store
  information about the size and position of a plot on a graphics
  device. This document explains how to interact with this database
  using the AGI library.
}
\startitlepic{\setlength{\unitlength}{1mm}
\begin{center}
\begin{picture}(120,100)

\put(60,50){\oval(100,80)}
\put(50,85){\makebox(20,5){Base}}
\put(20,40){\framebox(50,40)[tl]{Frame}}
\put(25,60){\framebox(40,15)[tl]{Data}}
\put(27,63){\circle*{1}}
\put(29,66){\circle*{1}}
\put(35,66){\circle*{1}}
\put(39,64){\circle*{1}}
\put(40,67){\circle*{1}}
\put(42,66){\circle*{1}}
\put(46,69){\circle*{1}}
\put(50,67){\circle*{1}}
\put(51,70){\circle*{1}}
\put(54,70){\circle*{1}}
\put(56,68){\circle*{1}}
\put(63,73){\circle*{1}}
\put(28,64){\line(5,1){34}}

\put(25,45){\framebox(25,10)[tl]{Data}}
\put(25,48){\line(1,0){2}}
\put(27,48){\line(0,-1){2}}
\put(27,46){\line(1,0){2}}
\put(29,46){\line(0,1){1}}
\put(29,47){\line(1,0){2}}
\put(31,47){\line(0,1){2}}
\put(31,49){\line(1,0){2}}
\put(33,49){\line(0,1){3}}
\put(33,52){\line(1,0){2}}
\put(35,52){\line(0,1){1}}
\put(35,53){\line(1,0){2}}
\put(37,53){\line(0,-1){3}}
\put(37,50){\line(1,0){2}}
\put(39,50){\line(0,1){1}}
\put(39,51){\line(1,0){2}}
\put(41,51){\line(0,-1){3}}
\put(41,48){\line(1,0){2}}
\put(43,48){\line(0,1){1}}
\put(43,49){\line(1,0){2}}
\put(45,49){\line(0,-1){2}}
\put(45,47){\line(1,0){2}}
\put(47,47){\line(0,1){3}}
\put(47,50){\line(1,0){2}}
\put(49,50){\line(0,-1){2}}
\put(49,48){\line(1,0){1}}

\put(40,40){\line(0,-1){20}}
\put(40,20){\line(1,0){50}}
\put(90,20){\line(0,1){50}}
\put(90,70){\line(-1,0){20}}
\put(70,40){\shortstack[l]{t discovery \\ ~all agreed \\obel Prize. \\
ummed up \\ the moon".}}

\put(72,22){\framebox(16,16)[tl]{Data}}
\put(80,30){\circle{10}}
\put(80,30){\vector(-1,4){1}}
\put(80,30){\vector(-1,0){3}}
\put(80,30){\vector(-4,-1){7.5}}
\put(80,30){\vector(-1,-1){4.5}}
\put(80,30){\vector(-1,-3){1.2}}
\put(47,31){\shortstack[l]{Vector display \\ of the results. }}

\put(25,40){\line(0,-1){5}}
\put(25,35){\line(1,0){15}}
\put(70,72){\line(1,0){7}}
\put(77,72){\line(0,-1){2}}

\end{picture}
\end{center}
}
% ? End of document identification
% -----------------------------------------------------------------------------

% -----------------------------------------------------------------------------
% ? Document specific \providecommand or \newenvironment commands.
\providecommand{\noteroutine}[2]{\textbf{#1}\hspace*{\fill}\nopagebreak \\
                             \hspace*{3em}\emph{#2}\hspace*{\fill}\par}

% ? End of document specific commands
% -----------------------------------------------------------------------------

\begin{document}
\scfrontmatter

% ? Main text

\section {Introduction}

AGI is a graphics database system. It can be used to store information
about the size and position of a plot on a graphics device. This can be
used by subsequent applications to find out where the plot has been drawn
and how the coordinate system maps onto the plotting area.

One obvious use of the database is to allow an application to overlay
its output on top of that produced by a different application, for instance
plotting radio contours onto an optical image. The first application draws
its image and informs AGI the size and position of the plot. The second
application uses AGI to set up its graphics coordinate system to match
that of the first application so that the contour map appears in the
correct position on the screen.

Another common use is to allow an application to put up an interactive
cursor to read off positions from a previously drawn image.

Although designed for use within ADAM, AGI can also be used in non-ADAM
applications.

\section {Brief Description \label{descript}}

The AGI database system implementation has changed in the current
version. It now exists as two subroutine libraries -- one, which is
unchanged from earlier versions, to handle all graphics interactions
where GKS is the underlying graphics package (the \textbf{AGI} library)
and a new library which interfaces to native PGPLOT only (the
\textbf{AGP} library). In both these libraries the routines are split into two
functional types. One set of routines deals with the manipulation of,
and navigation around, the database; these routines are prefixed with
\textbf{AGI\_}. The other set of routines provides an interface between
the database and the graphics package used by the application. In the
AGI library these routines are prefixed \textbf{AGS\_} if the
graphics package is SGS (\xref{SUN/85}{sun85}{}), \textbf{AGP\_} if the
package is GKS-based PGPLOT (\xref{SUN/15}{sun15}{}) and \textbf{AGD\_}
if the package is IDI (\xref{SUN/65}{sun65}{}). The new AGP
library only interfaces to native PGPLOT (\xref{SUN/15}{sun15}{}) and
so only contains those  graphics routines prefixed with \textbf{AGP\_}. The two
libraries can create identical database files on disk and thus new
native-PGPLOT applications can be arranged to co-exist with older
applications based on GKS. Current Starlink policy is that GKS is
being phased out and so all new applications should be designed to use
the AGP library only.

The basic structure of the database is an object called a picture. This
represents a rectangular area on the display surface and contains
information about the coordinate system in use when the picture was
created. Pictures are usually created in the database by one of the
interface routines specific to a particular graphics package. For
instance the routine
\htmlref{\textbf{AGS\_SZONE}}{AGS_SZONE} will save the extent and coordinate
system of the current SGS zone as a picture in the database. However
pictures in the database are independent of the graphics package that
created them. This means that, for example, a picture created by one
application using SGS graphics can be used by another application
employing PGPLOT. It is even possible, with care, to use two graphics
packages, such as SGS and IDI, in one application by using AGI to
mediate between them.

Although pictures are independent of a graphics package they are not
independent of the graphics device to which they refer. When a picture
is created it contains an implicit reference to the current device,
even if nothing has been plotted on the screen or even if a graphics
package has not been activated on the device.
The pictures in the database are grouped into a larger unit, here called
a workstation structure, that contains all the pictures on a given device,
stored in the order they were created.

Pictures are given a name which gives an indication of what the plotting
space contains. For effective communication between applications it is
important that the picture names are standardised. It is customary for
a picture containing other pictures to be referred to as a `FRAME' picture,
and a picture containing a plot to be referred to as a `DATA' picture.
A `BASE' picture refers to the whole plotting space and this name should
not be used for other purposes.

As well as a name a picture structure contains a one line comment field
which can be used to further identify a picture in the database. The name,
comment and coordinate system are mandatory components of a picture in
the database, but there are some other components which are optional.
The first is a reference to the data associated with a picture. This
is a complete description of where the data is located, containing both
the file name and the location within the file. This can be used by a
later application to access the same data set without having to prompt
the user again.

The second optional component is a transformation structure.
The default coordinate system of a picture stored in the database is a
simple orthogonal linear set that increases left to right and bottom to top.
If the coordinate system of a plot on the screen does not conform to this
rule then a transformation which defines the relationship can
be saved in the database.
For example, an application may have displayed an image and saved a
transformation from pixel coordinates to right ascension and declination
in the database. A later cursor application can then use the transformation
to report its results in these non-linear coordinates.

The last optional component of a picture is a label. This is basically a
short character string which can be used by an application to uniquely
identify a picture. Unlike the other components of a picture the label
contents can be changed at any time, the only restriction being that
the label is unique on a given device.

If it is necessary to store additional information in a picture
structure, that cannot be accommodated by any of the defined components,
then a MORE component is available for this purpose. It is important to
remember, however, that the structure and contents of the MORE component
can only be interpreted by applications that know how to use it, and
a general purpose application will ignore the MORE structure.

\section {Summary of AGI Calls}

\subsection{Interface to PGPLOT (both libraries)}
\noteroutine{AGP\_ACTIV( status )}
        {Initialise PGPLOT}
\noteroutine{AGP\_ASSOC( param, acmode, pname, border, picid, status)}
        {Associate a device with AGI and PGPLOT through ADAM}
\noteroutine{AGP\_DEACT( status )}
        {Close down PGPLOT}
\noteroutine{AGP\_DEASS( param, parcan, status )}
        {De-associate a device from AGI and PGPLOT}
\noteroutine{AGP\_NVIEW( border, status )}
        {Create a new PGPLOT viewport from the current picture}
\noteroutine{AGP\_SVIEW( pname, coment, picid, status )}
        {Save the current PGPLOT viewport in the database}
\subsection{Interface to SGS (only in AGI library)}
\noteroutine{AGS\_ACTIV( status )}
        {Initialise SGS}
\noteroutine{AGS\_ASSOC( param, acmode, pname, picid, newzon, status )}
        {Associate a device with AGI and SGS through ADAM}
\noteroutine{AGS\_DEACT( status )}
        {Close down SGS}
\noteroutine{AGS\_DEASS( param, parcan, status )}
        {De-associate a device from AGI and SGS}
\noteroutine{AGS\_NZONE( newzon, status )}
        {Create a new SGS zone from the current picture}
\noteroutine{AGS\_SZONE( pname, coment, picid, status )}
        {Save the current SGS zone in the database}

\subsection{Interface to IDI (only in AGI library)}
\noteroutine{AGD\_ACTIV( status )}
        {Initialise IDI}
\noteroutine{AGD\_ASSOC( param, acmode, pname, memid, picid, dispid,
xsize, ysize, xoff, yoff, status )}
        {Associate a device with AGI and IDI through ADAM}
\noteroutine{AGD\_DEACT( status )}
        {Close down IDI}
\noteroutine{AGD\_DEASS( param, parcan, status )}
        {Deassociate a device from AGI and IDI}
\noteroutine{AGD\_NWIND( memid, dispid, xsize, ysize, xoff, yoff, status )}
        {Define an IDI window from the current picture}
\noteroutine{AGD\_SWIND( dispid, memid, xsize, ysize, xoff, yoff,
pname, coment, wx1, wx2, wy1, wy2, picid, status )}
        {Save an IDI window in the database}

\newpage
\subsection{Control (both libraries)}
\noteroutine{AGI\_ANNUL( picid, status )}
        {Annul a picture identifier}
\noteroutine{AGI\_ASSOC( param, acmode, picid, status )}
        {Associate an AGI device with an ADAM parameter}
\noteroutine{AGI\_BEGIN}
        {Mark the beginning of a new AGI scope}
\noteroutine{AGI\_CANCL( param, status )}
        {Cancel the ADAM device parameter}
\noteroutine{AGI\_CLOSE( status )}
        {Close AGI in non-ADAM environments}
\noteroutine{AGI\_END( picid, status )}
        {Mark the end of an AGI scope}
\noteroutine{AGI\_NUPIC( wx1, wx2, wy1, wy2, pname, coment,
newx1, newx2, newy1, newy2, picid, status )}
        {Create a new picture in the database}
\noteroutine{AGI\_OPEN( wkname, acmode, picid, status )}
        {Open an AGI device in a non-ADAM environment}
\noteroutine{AGI\_PDEL( status )}
        {Delete all the pictures on the current device}
\noteroutine{AGI\_SELP( picid, status )}
        {Select the given picture as the current one}
\noteroutine{AGI\_SROOT( status )}
        {Select the root picture for searching}

\subsection{Recall (both libraries)}
\noteroutine{AGI\_RCF( pname, picid, status )}
        {Recall first picture of specified name}
\noteroutine{AGI\_RCFP( pname, x, y, picid, status )}
        {Recall first picture embracing a position}
\noteroutine{AGI\_RCL( pname, picid, status )}
        {Recall last picture of specified name}
\noteroutine{AGI\_RCLP( pname, x, y, picid, status )}
        {Recall last picture embracing a position}
\noteroutine{AGI\_RCP( pname, pstart, picid, status )}
        {Recall preceding picture of specified name}
\noteroutine{AGI\_RCPP( pname, pstart, x, y, picid, status )}
        {Recall preceding picture embracing a position}
\noteroutine{AGI\_RCS( pname, pstart, picid, status )}
        {Recall succeeding picture of specified name}
\noteroutine{AGI\_RCSP( pname, pstart, x, y, picid, status )}
        {Recall succeeding picture embracing a position}

\subsection{Labels (both libraries)}
\noteroutine{AGI\_ILAB( picid, label, status )}
        {Inquire label of a picture}
\noteroutine{AGI\_SLAB( picid, label, status )}
        {Store label in picture}

\subsection{Reference (both libraries)}
\noteroutine{AGI\_GTREF( picid, mode, datref, status )}
        {Get a reference object from a picture}
\noteroutine{AGI\_PTREF( datref, picid, status )}
        {Store a reference object in a picture}

\subsection{Transformations (both libraries)}
\noteroutine{AGI\_TCOPY( trnloc, picid, status )}
        {Copy a transformation structure to the database}
\noteroutine{AGI\_TDDTW( picid, nxy, dx, dy, wx, wy, status )}
        {Transform double precision data to world coordinates}
\noteroutine{AGI\_TDTOW( picid, nxy, dx, dy, wx, wy, status )}
        {Transform data to world coordinates}
\noteroutine{AGI\_TNEW( ncd, ncw, dtow, wtod, picid, status )}
        {Store a transformation in the database}
\noteroutine{AGI\_TRUNC( status )}
        {Truncate the AGI database file by removing unused space}
\noteroutine{AGI\_TWTDD( picid, nxy, wx, wy, dx, dy, status )}
        {Transform double precision world to data coordinates}
\noteroutine{AGI\_TWTOD( picid, nxy, wx, wy, dx, dy, status )}
        {Transform world to data coordinates}

\subsection{Inquiries (both libraries)}
\noteroutine{AGI\_IBASE( picid, status )}
        {Inquire base picture for current device}
\noteroutine{AGI\_ICOM( coment, status )}
        {Inquire comment for the current picture}
\noteroutine{AGI\_ICURP( picid, status )}
        {Inquire the current picture}
\noteroutine{AGI\_INAME( pname, status )}
        {Inquire name of the current picture}
\noteroutine{AGI\_IPOBS( picid, lobs, status )}
        {Is current picture obscured by another?}
\noteroutine{AGI\_ISAMD( picid, lsame, status )}
        {Inquire if pictures are on same device}
\noteroutine{AGI\_ISAMP( picid, lsame, status )}
        {Inquire if two pictures are the same}
\noteroutine{AGI\_ITOBS( nxy, x, y, ltobs, status )}
        {Inquire if test points are obscured}
\noteroutine{AGI\_IWOCO( wx1, wx2, wy1, wy2, status )}
        {Inquire world coordinates of current picture}

\subsection{More (both libraries)}
\noteroutine{AGI\_IMORE( picid, lmore, status )}
        {Inquire if a MORE structure exists}
\noteroutine{AGI\_MORE( picid, acmode, morloc, status )}
        {Return an HDS locator to a MORE structure}

\subsection{Example skeleton applications}\label{exam}

The lengthy subroutine list suggests that any application utilising
AGI could end up looking complicated, but this need not be the case.
By using the wrap-up routines simple applications need only call two
or three routines from the list. As an example consider first an
application that plots something on the screen and makes an entry in
the database. A second application then uses a cursor to read a
position from the graph. The second application is quite general and
will work with any application that creates a DATA picture in the
database, such as KAPPA:DISPLAY. Note that, as this example uses
PGPLOT, it can be linked with either the GKS-based AGI library or the
native-PGPLOT AGP library. However Starlink policy is that GKS is
being phased out and so all new production applications should be
linked with the \textbf{AGP} library as described \slhyperref{here}{in
Section~}{}{linking}.
\begin{terminalv}
      SUBROUTINE PPLOT( STATUS )
      INCLUDE 'SAE_PAR'
      INTEGER I, ID1, ID2, N, STATUS
      PARAMETER ( N = 100 )
      REAL XP( N ), YP( N )

*   Check inherited global status
      IF ( STATUS .NE. SAI__OK ) GOTO 99

*   Define the function to plot
      DO I = 1, N
         XP( I ) = REAL( I )
         YP( I ) = REAL( I )
      ENDDO

*   Open AGI and PGPLOT through the ADAM interface
      CALL AGP_ASSOC( 'DEVICE', 'WRITE', ' ', .TRUE., ID1, STATUS )     ...1
      IF ( STATUS .NE. SAI__OK ) GOTO 99

*   Define the world coordinates of the viewport to match the data
      CALL PGSWIN( 0.0, 100.0, 0.0, 100.0 )

*   Create a box and plot the data
      CALL PGBOX( 'BCNST', 0.0, 0, 'BCNST', 0.0, 0 )
      CALL PGLINE( N, XP, YP )

*   Save the PGPLOT viewport as a picture in the database
      CALL AGP_SVIEW( 'DATA', 'PGLINE output', ID2, STATUS )            ...2

*   Close down AGI and PGPLOT cancelling the parameter
      CALL AGP_DEASS( 'DEVICE', .TRUE., STATUS )                        ...3

  99  CONTINUE
      END
\end{terminalv}
The program requires an interface file to define the DEVICE parameter.
The following example will do the job:-
\begin{terminalv}
interface PPLOT
   parameter DEVICE
      access 'READ'
      vpath 'PROMPT'
      prompt 'Display device '
   endparameter
endinterface
\end{terminalv}

\textbf{Program notes:}
\begin{enumerate}
\item[1.] The routine \htmlref{\textbf{AGP\_ASSOC}}{AGP_ASSOC}
 is used to access the database and
open PGPLOT. It is a wrap-up routine for a number of commonly called AGI
functions. The mode is set to 'WRITE' so that the plotting area is cleared
and the border argument is set true to allow room for annotation of the
axes. Passing an empty string in the name argument results in the current
database picture being used to define the PGPLOT viewport.
\item[2.] The current PGPLOT viewport is saved as an AGI picture in the
database.
\item[3.] The database and PGPLOT are closed using another of the wrap-up
routines. This also tidies up any picture identifiers returned from AGI
routines, and in this example cancels the parameter association.
\end{enumerate}

\begin{terminalv}
      SUBROUTINE PCURS( STATUS )
      INCLUDE 'SAE_PAR'
      INTEGER ID, STATUS
      CHARACTER CH, TEXT*64
      REAL XC, YC

*   Check inherited global status
      IF ( STATUS .NE. SAI__OK ) GOTO 99

*   Open AGI and PGPLOT through the ADAM interface
      CALL AGP_ASSOC( 'DEVICE', 'READ', 'DATA', .FALSE., ID, STATUS )   ...4
      IF ( STATUS .NE. SAI__OK ) GOTO 99

*   Request a PGPLOT cursor
      CALL PGCURSE( XC, YC, CH )

*   Report the result
      WRITE( TEXT, '( ''Cursor position ='', 2F6.1 )' ) XC, YC
      CALL MSG_OUT( 'PCURS', TEXT, STATUS )

*   Close down AGI and PGPLOT cancelling the parameter
      CALL AGP_DEASS( 'DEVICE', .TRUE., STATUS )                        ...5

  99  CONTINUE
      END
\end{terminalv}

\textbf{Program notes:}
\begin{enumerate}
\item[4.] The mode argument in \textbf{AGP\_ASSOC} is set to 'READ' to ensure
that the plot is not cleared. The name argument specifies that the last
picture of type DATA should be recalled.
The border argument is set false to ensure that the PGPLOT viewport is
created to exactly match the DATA picture.
\item[5.] The call to \htmlref{\textbf{AGI\_END}}{AGI_END}
within \htmlref{\textbf{AGP\_DEASS}}{AGP_DEASS} resets the
current picture to be the one current when the application began.
\end{enumerate}

\section {Control}

\subsection{Opening the Database}

AGI can be used independently of a graphics system. When AGI is opened,
no specific graphics system is associated with the database.
When it is required to display something on a device, a graphics package
is selected and the plotting done. The opening and closing of the package
is performed separately from the opening and closing of the database
\footnote{There are routines, such as \textbf{AGP\_ASSOC} and \textbf{AGP\_DEASS},
which package both operations in one unit.}.
The following simplified block diagram illustrates this operation.
\begin{terminalv}
          OPEN AGI ( e.g. AGI_ASSOC )

             OPEN the graphics package ( e.g. AGP_ACTIV )
             OPEN the graphics device ( e.g. AGP_NVIEW )
             plot with the graphics package
             CLOSE the graphics package ( e.g. AGP_DEACT )

          CLOSE AGI ( e.g. AGI_CANCL )
\end{terminalv}

The first step of opening the database is done by calling either
\htmlref{\textbf{AGI\_ASSOC}}{AGI_ASSOC} or
\htmlref{\textbf{AGI\_OPEN}}{AGI_OPEN}.
\textbf{AGI\_ASSOC} is used in ADAM
tasks and obtains the name of the device through the ADAM parameter
system.
\begin{terminalv}
          CALL AGI_ASSOC( PARAM, ACMODE, PICID, STATUS )
\end{terminalv}
PARAM is the name of the parameter in the ADAM interface file through
which the device name is to be obtained. It should be stressed that
\textbf{AGI\_ASSOC} does not open the specified device for plotting, it
opens the database entry for that device. The opening of the device for
plotting is handled by the routines specific to a particular graphics
package, for instance the \textbf{AGS\_} routines.
ACMODE is an access mode which has a particular meaning for AGI. It does
not refer to the access mode of the database, i.e. whether an application
can update the database or not, but instead it refers to the access mode
of the graphics device. It is used to control whether the device should
be cleared when it is opened by a graphics package. The possible modes
are `READ', `UPDATE', and `WRITE'. The modes `READ' and `UPDATE' leave
the display as it is; the `WRITE' mode clears the area of the display
specified by the current picture returned in PICID. PICID returns an
identifier to the current picture in the database; on opening the
database this will be the picture that was current when the database was
last accessed. If there is no previous entry in the database for this
device then a `BASE' picture will be created.
This `BASE' picture will fill the display area and will be created with a
default world coordinate system, in the same way as the SGS base zone.

\htmlref{\textbf{AGI\_OPEN}}{AGI_OPEN} is used in non-ADAM tasks.
\begin{terminalv}
          CALL AGI_OPEN( WKNAME, ACMODE, PICID, STATUS )
\end{terminalv}
WKNAME is the device name. The other arguments are the same as those in
\textbf{AGI\_ASSOC}.

The given device name, whether passed through the parameter system or the
argument list, should obey the syntax rules described
\slhyperref{here}{in the next Section~(}{)}{devnames}.

Devices with windowing capabilities allow the user to change the size of the
display window between applications. This has a detrimental affect on AGI
because the size of the base picture, and the relationship between pictures
changes. Rather than considering this an error status, which could terminate
an application, AGI deletes all the pictures on this device, creates a
new base picture and sends the following message to the user.
\begin{terminalv}
Display has changed size.
Deleting previous database entries.
\end{terminalv}
This allows any application that can continue from such a situation to do so.
For instance it is sensible to allow a display application that uses the
current picture for output to continue in this situation. In this case, after
the existing database entries have been deleted, the current picture will be
the base picture.

\subsection{Device names\label{devnames}}

For applications linked with the GKS based \textbf{AGI} library the user
should supply a valid GNS device name (\xref{SUN/57}{sun57}{}) which
AGI converts into an internal name that locates the workstation
structure in the database. For devices that have more than one
plotting plane, it is desirable that AGI treat pictures plotted on the
different planes as being on the same device, so that plots on the
overlay plane can be lined up with plots below them. Database entries
for pictures on these different planes, which are different devices in
GKS and SGS, will therefore appear in the same AGI workstation
structure.

The \textbf{AGP} library is based on native-PGPLOT style device names of the
form \texttt{<devname>/<devtype>}. The graphics device to be used is
selected by the \texttt{<devtype>} component. The \texttt{<devname>}
component before the \// will be some form of name used by the device
driver (typically a physical device on the system or an output file
name) but is often omitted as most drivers contain a sensible default
value. For example a graphics device of just \texttt{\//gwm} will open a
Starlink GWM X11 window with a default window name of \texttt{xwindows}
while a complete specification of \texttt{mywindow/gwm} will open a GWM
window with the name \texttt{mywindow}.

For native-PGPLOT based applications to co-exist with GKS-based ones
it is important that users can continue to supply ``old\_style'' GKS device
names where these have a sensible translation. Where there \textbf{IS} such
a translation the AGI  database records must also be made to match.
Logic has been incorporated into the AGP library to perform this name
translation process.
Also in AGP typing a single ? for the device name will list all the possible
translations. At the time of writing this document these were (but see also
section \slhyperref{this section}{Section~}{}{aps}):
\begin{terminalv}
gif_l     (/GIF)           GIF-format file, landscape
gif_p     (/VGIF)          GIF-format file, portrait
hpgl_l    (/HPGL)          Hewlett-Packard HP-GL plotters, landscape
hpgl_p    (/VHPGL)         Hewlett-Packard HP-GL plotters, portrait
hpgl2     (/HPGL2)         Hewlett-Packard graphics language
xterm     (/XTERM)         Tektronix terminal emulator
wd_l      (/WD)            X Window dump file, landscape
wd_p      (/VWD)           X Window dump file, portrait
xserve    (/XSERVE)        PGPLOT X window server
tek_4010  (/TEK4010)       Tektronix 4006-4010 storage-tube terminal
tek_4107  (/TEK4100)       Tektronix 4100-series terminals
ps_p      (/VPS)           Postscript printers, monochrome, portrait
ps_l      (/PS)            Postscript printers, monochrome, landscape
epsf_p    (/VPS)           Encapsulated postscript, portrait
epsf_l    (/PS)            Encapsulated postscript, landscape
pscol_p   (/VCPS)          PostScript printers, color, portrait
pscol_l   (/CPS)           PostScript printers, color, landscape
epsfcol_p (/VCPS)          Colour encapsulated postscript, portrait
epsfcol_l (/CPS)           Colour encapsulated postscript, landscape
xwindows  (xwindows/GWM)   Starlink GWM xwindow
x2windows (xwindows2/GWM)  Starlink GWM xwindow
x3windows (xwindows3/GWM)  Starlink GWM xwindow
x4windows (xwindows4/GWM)  Starlink GWM xwindow
\end{terminalv}

As an example of the overall effect native-PGPLOT graphics
specifications of \texttt{xwindows/gwm}, \texttt{\//gwm} or \texttt{xwindows}
are all valid and equivalent in the \textbf{AGP} version of the AGI
library. They will select a GWM window with name \texttt{xwindows} which
will have AGI database records identical to those created by the
\textbf{AGI} library routines for the graphics device with the GNS name of
\texttt{xwindows}.

The user will still see some slight differences from previous GKS behaviour
when using GKS style device names via this scheme. For example:
\begin{itemize}
\item Native-PGPLOT, unlike the GKS-based version, always produces
``encapsulated'' postscript files.
\item The default output file for postscript files will be \textbf{pgplot.ps}
with (unlike GKS) no version numbers.  Thus care must be taken to ensure
that consecutive graphics applications do not over-write this file. An
alternative postscript file name can be specified either by the GKS
syntax ps\_l;fred.ps or the native-PGPLOT syntax fred.ps/PS
\end{itemize}

\subsection{Picture identifiers}

Pictures are flagged by means of an integer identifier (the picture
identifier) which combines knowledge of a device name and a picture
number in the same way that an SGS zone number does. Both
\htmlref{\textbf{AGI\_ASSOC}}{AGI_ASSOC}
and
\htmlref{\textbf{AGI\_OPEN}}{AGI_ASSOC}
return a picture identifier to the current picture on
the given device. Pictures in the database can only be referenced by means
of these identifiers which are returned to the application from the
appropriate AGI routines. These identifiers should only be passed on to
other AGI routines and should not be altered in any way by an application.

Once a picture is no longer required it is good practice to release it using
\htmlref{\textbf{AGI\_ANNUL}}{AGI_ANNUL}
 which liberates internal workspace. Picture identifiers
can be automatically released by using a begin-end block (see
\slhyperref{this section}{Section~}{}{bend}).

\subsection{Closing the Database}

It may seem a bit premature to close the database having only just
described how to open it, but some things are best explained at this stage.
In an ADAM environment the routine
\htmlref{\textbf{AGI\_CANCL}}{AGI_CANCL} cancels the
association of the specified parameter and performs specific tidying
up operations concerning that parameter. If the parameter association
has to be retained outside the program then \textbf{AGI\_CANCL} should not
be called.

If the workstation structure is opened by \textbf{AGI\_OPEN}, a call to
\htmlref{\textbf{AGI\_CLOSE}}{AGI_CLOSE} should be used to close it.

General tidying up operations (e.g. closing the database) are performed
when the last active identifier has been annulled, either with
\textbf{AGI\_ANNUL} or
\htmlref{\textbf{AGI\_END}}{AGI_END}. It is important that tasks clean
up all their identifiers otherwise unusual side effects may occur if
the executable image is permanently resident in memory (such as an
ADAM task).

Once the database has been closed, \htmlref{\textbf{AGI\_TRUNC}}{AGI_TRUNC} can
be called to truncate the database file to remove any unused space, thus
keeping the file size at a minimum.

\subsection{Begin-End blocks\label{bend}}
As the database is not closed until all identifiers have been annulled
it is important that each task keeps its allocation of identifiers
under control. For a complicated application annulling individual
identifiers may be tiresome and so a general tidying up scheme is
available using a begin-end block. The routines
\htmlref{\textbf{AGI\_BEGIN}}{AGI_BEGIN} and
\htmlref{\textbf{AGI\_END}}{AGI_END}
are used to bracket a block of code and all the
identifiers allocated within the block are automatically annulled
by \textbf{AGI\_END}. These routines may be nested, (up to a depth of eight)
but each \textbf{AGI\_BEGIN} must have its corresponding \textbf{AGI\_END}.

As well as annulling identifiers the routine \textbf{AGI\_END} performs another
important function. The routine takes a picture identifier as its input
and this picture is made current. If the identifier is valid then this is
no different to calling \htmlref{\textbf{AGI\_SELP}}{AGI_SELP},
but if the argument passed to
\textbf{AGI\_END} is a negative number then the picture that was current when
the corresponding \textbf{AGI\_BEGIN} was called is made current again. This
is useful for applications that need to restore the current picture to
that which was current when the application began.

The following two examples show different uses of a begin-end block.
In the first example the call to \textbf{AGI\_END} annuls all the
identifiers, which results in the database being closed, since there
are no more active identifiers. As there was no current picture when
the corresponding \textbf{AGI\_BEGIN} was called (since the database had
not been opened) the negative argument in \textbf{AGI\_END} has no effect
in this case, and the picture that was current just before the call to
\textbf{AGI\_END} remains current.
\begin{terminalv}
    *   Begin an AGI scope
          CALL AGI_BEGIN

    *   Open AGI on a device obtained from the parameter system
          CALL AGI_ASSOC( 'DEVICE', 'WRITE', ID, STATUS )

    *   Main body of program
          <main body of program>

    *   Annul identifiers and close the database
          CALL AGI_END( -1, STATUS )
\end{terminalv}
In the second example \textbf{AGI\_BEGIN} is put after \textbf{AGI\_ASSOC}.
When \textbf{AGI\_END} is called the identifier returned from
\htmlref{\textbf{AGI\_ASSOC}}{AGI_ASSOC}
is not annulled, since it is outside the scope of the begin-end block.
The negative argument in \textbf{AGI\_END} makes the picture that was current
when the corresponding \textbf{AGI\_BEGIN} was called current again (in this
example the picture that was current when the application began).
\begin{terminalv}
    *   Open AGI on a device obtained from the parameter system
          CALL AGI_ASSOC( 'DEVICE', 'WRITE', ID, STATUS )

    *   Begin an AGI scope
          CALL AGI_BEGIN

    *   Main body of program
          <main body of program>

    *   Annul identifiers from the main body of the program
    *   and reinstate the current picture
          CALL AGI_END( -1, STATUS )

    *   Annul the initial identifier and close the database
          CALL AGI_ANNUL( ID, STATUS )
\end{terminalv}

\subsection{The Database file}

The database is stored in an HDS container file named agi\_$<$node$>$.sdf
in the directory defined by the environment variable AGI\_USER or in
your home directory if AGI\_USER hasn't been defined
(HDS is the Hierarchical Data System described in \xref{SUN/92}{sun92}{}).
The name of the
node is included in the file name to prevent two or more devices which
have the same name but are on different nodes causing a conflict.
The contents of the file can be examined using the ADAM application TRACE.
In certain circumstances the database file may become corrupted. If this
is suspected it is best to delete the appropriate file and start again.

The name of the container file can be optionally changed by defining the
environment variable AGI\_NODE to be equivalent to some user defined string.
If such a environment variable exists then the container file is constructed
by replacing the default node with the new string. Thus if the environment
variable AGI\_NODE is defined to be `nonode' then the container file will
have the name agi\_nonode.sdf in the directory AGI\_USER. This mechanism
can be used to pick up a database created on a different node by
defining the logical name to be equivalent to the required node name.

\section {Components}

\subsection{Pictures}

Pictures represent rectangular areas on a display.
The pictures have default coordinate systems that are linear and increase
left to right and bottom to top, corresponding to GKS rules.
Each picture can optionally have an associated transformation to allow for
other coordinate systems.
Pictures are stored in the database in the order they were created, so when
recalling the last picture of a given name, for example, the most recently
created picture of that name will be recovered.

The AGI interface to each graphics package will interpret the meaning of a
picture to suit the design of that particular package.
When used in conjunction with SGS a picture is made equivalent to an SGS zone.
For example to save an SGS zone as a picture use
\begin{quote}\texttt{
          CALL \htmlref{AGS\_SZONE}{AGS_SZONE}( PNAME, COMENT, PICID, STATUS )}
\end{quote}
PNAME and COMENT are used to identify the picture in the database as described
in the sections below.
PICID returns the identifier for the new picture in the database.

There is one routine, \htmlref{\textbf{AGI\_NUPIC}}{AGI_NUPIC},
that allows new pictures to be created
in the database without necessarily opening a graphics package first.
\begin{terminalv}
          CALL AGI_NUPIC( WX1, WX2, WY1, WY2, PNAME, COMENT,
         :                NEWX1, NEWX2, NEWY1, NEWY2, PICID, STATUS )
\end{terminalv}
WX1, WX2, WY1, WY2 define the rectangular extent of the new picture in terms of
the world coordinates of the current picture.
PNAME and COMENT describe the picture as discussed in the following
sections.
NEWX1, NEWX2, NEWY1, NEWY2 specify the world coordinate system of the new
picture.
PICID returns the identifier for the new picture in the database.

AGI works with a current picture in an analogous manner to SGS and its
current zone.
A new picture has to be created within the bounds of the current one, and this
then becomes the current picture.
Pictures are recalled from the database only if they lie within the physical
bounds of the current one; the recalled picture then becomes the current one.
A picture can be made current by calling the routine
\htmlref{\textbf{AGI\_SELP}}{AGI_SELP}.

The picture recall routines, \textbf{AGI\_RC*}, provide the means of traversing
the database.
The key to the search is the picture name, and the search ends when a picture
of the given name is found that lies within completely the bounds of the
current picture. This picture then becomes the current picture.
If no suitable picture is found then an error status is returned.
The search can be made more general by giving an empty name string. This
results in the first picture on the search path, that lies within the current
one, being recalled.
\htmlref{\textbf{AGI\_RCL}}{AGI_RCL}
searches backwards from the last (most recent) picture in the database.
\htmlref{\textbf{AGI\_RCF}}{AGI_RCF}
searches forwards from the first (most ancient) picture in the database.
\htmlref{\textbf{AGI\_RCP}}{AGI_RCP}
searches backwards starting at the picture given by the PSTART
identifier, and \htmlref{\textbf{AGI\_RCS}}{AGI_RCS}
does the same searching forwards.

A further restriction on the search can be introduced by using the
\textbf{AGI\_RC*P} routines.
With these, a picture is only recalled if it corresponds to the given name,
lies within the bounds of the current picture, \emph{and} encompasses a given
point in the world coordinate space of the current picture.
This type of recall is useful when a cursor has been put up on the display,
and the user asked to select a picture with the cursor.

A potential problem arises when searching backwards through the database.
The hierarchy of pictures in the database implies that previous pictures
will usually be larger than the current one, but the search strategy
requires the current picture to be larger than the picture being sought.
This problem can be overcome by selecting a picture known to be larger
than any picture being sought as the current one. The base picture is
the most convenient as this is larger than any picture in the database.
The following example shows how to search backwards through the database
for a picture with a particular label (labels are discussed in
\slhyperref{this section}{Section~}{}{lab}).
\begin{terminalv}
    *   Get an identifier for the base picture and select it as current.
          CALL AGI_IBASE( BASEID, STATUS )
          CALL AGI_SELP( BASEID, STATUS )

    *   Recall the last picture in the database of any name and get its label.
          CALL AGI_RCL( ' ', PICID, STATUS )
          CALL AGI_ILAB( PICID, PLABEL, STATUS )

    *   Loop through the database until the a match with LABEL is found.
          DO WHILE ( ( PLABEL .NE. LABEL ) .AND. ( STATUS .EQ. SAI__OK ) )

    *   Reselect the base picture as the current one.
    *   This ensures the search does not fail because a picture is not
    *   within the current one.
             CALL AGI_SELP( BASEID, STATUS )

    *   Search backwards starting at the previously recalled picture
             PICIDS = PICID
             CALL AGI_RCP( ' ', PICIDS, PICID, STATUS )

    *   Inquire the label of this picture
             CALL AGI_ILAB( PICID, PLABEL, STATUS )
          ENDDO
\end{terminalv}

\subsection{The root picture}
For devices that allow the memories to be scrolled independently there is
a potential conflict when searching for a picture. This occurs because
under normal conditions a picture can only be recalled from the database
if it lies within the current picture. If the pictures are located on
different memory planes then this can be negated by scrolling one of the
memories. To overcome this a more general search is allowed by selecting
the root picture which contains all other pictures (including the base
picture). A call to the routine
\htmlref{\textbf{AGI\_SROOT}}{AGI_SROOT} selects the root
picture. A subsequent call to any of the recall routines \textbf{AGI\_RC*}
will obtain the relevant picture without it having to lie within the
current picture. The root picture is deselected in the recall routine, and
so \textbf{AGI\_SROOT} has to be called on every occasion that an unlimited
search is to be made. Recall is the only operation allowed with the root
picture. Any other operation called while the root picture is selected
will result in the previously current picture being used.

\subsection{World coordinates}
When AGI needs to access information in the database by position it uses the
world coordinate system of the current picture.
World coordinates are those set up by the user to represent a useful linear
range of coordinates which fill the picture space.
These world coordinates must be linear and increase from left to right
and from bottom to top. If the required coordinate system does not correspond
to these rules then the user should choose a basic world coordinate system
that does obey them, and then define a transformation to go to and from
the basic system and the required one (see
\slhyperref{this section}{the section on transformations~}{}{tran}).

When a picture is created with one of the package interface routines the
current world coordinates of the graphics zone are stored in the database.
If the user subsequently changes the world coordinate system of the graphics
space on the physical device, errors may result if information is requested
from the database by position.
The routine \htmlref{\textbf{AGI\_IWOCO}}{AGI_IWOCO}
can be used to inquire the world coordinates of the current picture.

\subsection{Name}
The name is a character string indicating the type of picture created.
It should reflect the general intention of the plotting space, and not
describe specifically what the picture contains.
For example, the name `FRAME' is used to indicate that the space will be used
to group together a collection of other plots.
The name `DATA' is used to indicate that the space will contain some sort of
representation of data in graphical form, e.g.\ a grey-scale image, or a
scatter plot.
The name `BASE' is used by AGI to signify the base picture, and this name
should not be used for other purposes.
The length of the name string is limited to the number of characters defined
by the AGI\_\_SZNAM constant from the AGI\_PAR FORTRAN include file
(currently 15 characters).
When stored in the database the character string will have leading blanks
removed and converted into upper case.

There are at present no compulsory names, but clearly consistent usage is
necessary if packages written by different programmers are to integrate
successfully.

\subsection{Comment}
The comment is a character string containing a description of the
picture. The application can use this component to store any text that
will help identify the picture in the database.
It is restricted to one line of text at present, with a maximum length
given by the parameter AGI\_\_CMAX defined in the AGI\_PAR include file.
The comment string is stored with the picture at the time of picture creation.

\subsection{Labels}\label{lab}
There is another character identifier that can be stored in the database
and used to distinguish a picture more precisely. This is the label which
is an optional component of a picture.
It differs from other structures in AGI in that its contents can be
changed at any time (by overwriting the existing contents), whereas
other picture elements can only be defined when they are first created.
Because the label contents can be changed at any time the responsibility
for guarding against misuse is left up to the application/user.

The labels differ in another way from the name component in that each
label within a workstation structure must be unique. If a new label clashes
with an exiting one within a workstation structure then the existing label
will be deleted and replaced by a blank string. Although the labels are
stored with mixed case the comparison is done with leading blanks removed
and in uppercase only.

The routine \htmlref{\textbf{AGI\_SLAB}}{AGI_SLAB}
will associate a label with an existing picture,
indicated by the picture identifier. If the picture identifier is negative
then the current picture is used to store the label. If a label already
exists for a picture then the old one will be overwritten.
The routine \htmlref{\textbf{AGI\_ILAB}}{AGI_ILAB}
will return the label of the requested picture.
If the picture identifier is negative then the current picture is searched.
If no label is associated with this picture then a blank string is returned.

The length of the label string is limited to the number of characters defined
by the AGI\_\_SZLAB constant from the AGI\_PAR include file
(currently 15 characters).

\subsection{Reference to data}
It may often be desirable to access the data used by one application
in a second one. To store the actual data in the database would be wasteful
of disk space since this data would be replicated in the original data file
and in the database. Even more space would be wasted if the same data
was associated with more than one picture in the database. Instead of
saving the data in the database AGI stores a reference to it (a string
describing it's location) that enables other applications to recover the
data.

A reference to a data object is not a compulsory component of a database
picture, and as such will not be created when a new picture is inserted
in the database.
An explicit call to one of the appropriate routines is used
to create or access a reference object in the database.
The subroutine argument that defines the reference object can be either
an HDS locator or any character string reference. If the string is a
valid HDS locator then a reference is constructed to point to the
relevant object, otherwise the string is assumed to be a reference
itself and is stored as supplied. This dual definition of what defines
a reference is intended to smooth the transition from applications
using raw HDS calls to using NDF routines to access the data. When the
NDF reference mechanism is well established the HDS interface will be
removed.

A reference is saved in the database by calling the routine
\htmlref{\textbf{AGI\_PTREF}}{AGI_PTREF}. The reference
structure is stored within the picture indicated by the picture
identifier. If the picture identifier is negative then the current
picture is used to store the reference. If a reference already exists
in the picture then an error status will be returned.

A reference is obtained from the database by calling the routine
\htmlref{\textbf{AGI\_GTREF}}{AGI_GTREF}.
The reference is obtained from the picture indicated
by the picture identifier. If the picture identifier is negative then
the current picture is used to obtain the reference.
If the reference was created from an HDS locator then an HDS locator
is returned and this should be annulled in the application using
\xref{\textbf{REF\_ANNUL}}{sun31}{REF_ANNUL}
rather than \xref{\textbf{DAT\_ANNUL}}{sun92}{DAT_ANNUL}. This will ensure that
the file containing the reference is properly shut down.

One important point to note is that AGI does not check if the data being
referenced is valid or not.

\subsection{Coordinate transformations}\label{tran}
The default coordinate system used by AGI is an orthogonal linear
system of world coordinates that increase from left to right and
bottom to top. If the data coordinate system used by an application
complies with these rules then a basic picture saving operation
will store sufficient information to recreate these coordinates at a
later stage.

If the transformation between data coordinates and the world coordinates
of a rectangular piece of the display screen is more complicated than
this simple system then a separate transformation has to be stored in
the database to define the relationships.
AGI uses the TRANSFORM facility described in \xref{SUN/61}{sun61}{}
to define the
transformations. AGI does not offer the full flexibility of the TRANSFORM
package as it is only interested in the limited case of transforming
from the data coordinates into the flat, world coordinates of the display
device.

There are two types of transformation to be defined, the first is the
transformation from data to world coordinates, and the second is the
transformation from world to data coordinates. Usually both will be
defined, although this is not a necessity.
The present implementation only allows for 2-dimensional data
coordinates, and an error will be returned if the number of data
variables differs from this.

The transformations are defined as character strings which describe the
mathematical formulas as if they appeared in a piece of FORTRAN code.
Therefore the description of the formulas should follow all the FORTRAN
rules for operators and functions. As an example the case of data
coordinates in polar coordinates will be used. The transformation
from polar coordinates $(r,\theta)$ to Cartesian world coordinates (x,y)
is defined by the equations
$x = r \cos( \theta )$ and $y = r \sin( \theta )$.
The inverse transformation is defined by the equations
$r = \sqrt{x^2+y^2}$ and $\theta = \tan^{-1}(y/x)$.
These are formulated in a program in the following way.
\begin{terminalv}
    *   Define the number of input ( data ) and output ( world ) variables
    *   Note: This should be 2 for each transformation.
          INTEGER NCD, NCW
          PARAMETER ( NCD = 2, NCW = 2 )

    *   Declare arrays for the two sets of transformations
          CHARACTER * 32 DTOW( NCW ), WTOD( NCD )

    *   Assign the data to world transformation functions
          DTOW( 1 ) = 'X = R * COS( THETA )'
          DTOW( 2 ) = 'Y = R * SIN( THETA )'

    *   Assign the world to data transformation functions
          WTOD( 1 ) = 'R = SQRT( X * X + Y * Y )'
          WTOD( 2 ) = 'THETA = ATAN2( Y, X )'
\end{terminalv}

The transformation is then stored in the database using the routine
\htmlref{\textbf{AGI\_TNEW}}{AGI_TNEW}.

If a transformation already exists in an HDS structure which converts
the data coordinates into the two dimensional world coordinates of the
display screen then the transformation can be copied into the database
using the routine \htmlref{\textbf{AGI\_TCOPY}}{AGI_TCOPY}.

The AGI interface has four routines which will perform transformations
that have been stored in the database. The routine
\htmlref{\textbf{AGI\_TDTOW}}{AGI_TDTOW}
will transform data coordinates into world coordinates, and the routine
\htmlref{\textbf{AGI\_TWTOD}}{AGI_TWTOD} will perform the inverse.
The argument list for these routines includes a picture identifier which
indicates the picture containing the transformation. If the identifier is
negative then the current picture is searched.
The routines take two arrays containing the x and y coordinates of the
points to be transformed and returns two arrays containing the new
x and y coordinates. In this context x and y need not mean a
Cartesian system when used for the data coordinates, they are simply
names for the first and second coordinate of the data system. In the above
example r and $\theta$ correspond to the x and y coordinates of the
data. If no transformation is found in the indicated picture then the
identity transformation (output = input) is used. The transformations
can be executed with double precision by calling the equivalent routines
\htmlref{\textbf{AGI\_TDDTW}}{AGI_TDDTW} and \htmlref{\textbf{AGI\_TWTDD}}{AGI_TWTDD}.

\subsection{Inquiries}
There are a number of inquiry routines available to investigate the properties
of the current picture.
\htmlref{\textbf{AGI\_INAME}}{AGI_INAME} and
\htmlref{\textbf{AGI\_ICOM}}{AGI_ICOM} inquire the name and description of the
current picture.
\htmlref{\textbf{AGI\_ISAMD}}{AGI_ISAMD} inquires if the given picture is on
the same physical device as the current picture.
\htmlref{\textbf{AGI\_ISAMP}}{AGI_ISAMP} inquires if the given picture identifier
points to the same picture in the database as the current picture identifier.
\htmlref{\textbf{AGI\_IPOBS}}{AGI_IPOBS} tests if the current picture is
obscured either partially or totally by a specified picture in the database.
Obscuration means here that the given picture intersects the current picture
and that it overlays the current picture (was created more recently).
The obscuration of the current picture can be tested against all subsequent
pictures in the database by specifying a negative number in place of the
picture identifier in the argument list.
\htmlref{\textbf{AGI\_ITOBS}}{AGI_ITOBS} tests the obscuration of a number
of test points.
The test points are specified in the coordinate system of the current picture,
and each point is tested in turn for obscuration by any picture overlying the
current one.

\subsection{More}
It may be desirable for an application to store information in the database
that is outside the scope of the routines provided. For this reason an
application can access a MORE structure using the routine
\htmlref{\textbf{AGI\_MORE}}{AGI_MORE}
and save arbitrary HDS objects within it. A MORE structure can be created
for each of the pictures in the database. It must be stressed that if an
application uses this facility then it is responsible for the contents
and whatever any subsequent application may make of them. Also an
application may need to cope with the situation of a corrupted structure.

The MORE structure is accessed for reading and writing using the routine
\textbf{AGI\_MORE}
\begin{terminalv}
          CALL AGI_MORE( PICID, ACMODE, MORLOC, STATUS )
\end{terminalv}
The access mode (ACMODE) controls the action of this routine. If the mode
is 'WRITE' then an HDS locator to an empty MORE structure is returned. If
there was no previous MORE structure then one is created, and if there was
a structure present then the previous contents are erased. At present
there is no lock to prevent one application erasing a MORE structure being
accessed by another. If the access mode is 'READ' or 'UPDATE' then the
routine returns a locator to the top of the MORE structure, unless there
was no structure present in which case an error is returned. The HDS
locator returned by this routine has to be annulled by the calling
application.

The presence of a MORE structure can be tested using the routine
\textbf{AGI\_IMORE}. The return argument is true if a MORE structure exists
for the given picture, otherwise the argument is false.

%\newpage

\section {Interface to Graphics Systems}

The AGI database is useful in two basic contexts. The first is to pass
information between separate applications that may or may not be
plotting on the device with the same graphics package. An example of two
independent PGPLOT based applications has been given
\slhyperref{earlier}{in Section~}{}{exam} while the use of the database file
to allow native-PGPLOT based applications to interact with GKS-based
applications is described \slhyperref{here}{in Section~}{}{devnames}.
For applications using the \textbf{AGI} GKS-based library a second use is
to allow different graphics packages to interact within a single
application, for example by using SGS to draw on top of an image
displayed with IDI. These two packages have completely different
models of the display device and AGI is used to mediate between them.
The major restriction with this second mode of use is that the two
graphics packages cannot both be open at the same time. There will be
other restrictions depending on how the physical device
characteristics are utilized by each graphics package; for example the
size of the display area may be different in the different packages
and this will cause problems if a picture defined by one package is
beyond the limits of another.

The following simplified block diagram illustrates the recommended operation
of AGI and two graphics packages within one application.
\begin{terminalv}
          OPEN AGI ( e.g. AGI_ASSOC )

             OPEN the first graphics package ( e.g. AGD_ACTIV )
             OPEN the device ( e.g. AGD_NWIND )
             plot with the first graphics package
             CLOSE the first graphics package ( e.g. AGD_DEACT )

             OPEN the second graphics package ( e.g. AGS_ACTIV )
             OPEN the device ( e.g. AGS_NZONE )
             plot with the second graphics package
             CLOSE the second graphics package ( e.g. AGS_DEACT )

          CLOSE AGI ( e.g. AGI_CANCL )
\end{terminalv}

\subsection{Interface to PGPLOT}

As described \slhyperref{previously}{in Section~}{}{descript} versions
of the PGPLOT interface routines with prefix \textbf{AGP\_} exist in two
libraries. The \textbf{AGI} library includes these and SGS routines based
on GKS graphics together with IDI routines. The \textbf{AGP} library is
designed to only interface to native-PGPLOT and has its own link
commands described \slhyperref{here}{in Section~}{}{linking}. The
\textbf{AGP\_} routines have indentical names and argument lists for both
libraries. Note, however, that the \textbf{AGP} library does \textbf{not}
support the concept of PGPLOT plotting into an SGS zone.

A set of routines are supplied to interface the database to the PGPLOT
graphics package, these are prefixed by AGP\_ instead of AGI\_. A
picture in the database is equated with a PGPLOT viewport. Previous
versions of PGPLOT only allowed one device to be open at one time and
this restriction still exists in the \textbf{AGP\_} routines. The PGPLOT
interface to AGI signifies this by returning an error status if
another device is requested while a current PGPLOT device is open.

PGPLOT is activated with a call to
\htmlref{\textbf{AGP\_ACTIV}}{AGP_ACTIV}, but it does not
actually open a device for plotting.
A device is opened when the first call to
\htmlref{\textbf{AGP\_NVIEW}}{AGP_NVIEW} is made.
The actual device that is opened will be the one associated with the
current picture in the database.
\textbf{AGP\_NVIEW} creates a new PGPLOT viewport from the current AGI
picture.
If the border argument in \textbf{AGP\_NVIEW} is false then the
new viewport fills the area specified by the picture and the world
coordinate system of the database picture is recreated in the viewport.
If however the border argument in \textbf{AGP\_NVIEW} is true then
the viewport is created smaller than the area defined by the picture so
that a standard width border surrounds the viewport. The world coordinates
of this viewport are given the default values of 0 to 1 in each axis.

The current viewport can be saved as a picture in the database using a
call to \htmlref{\textbf{AGP\_SVIEW}}{AGP_SVIEW}.
The size of the new picture must be less than
or equal to the size of the current picture in the database, otherwise
an error status is returned.
Although PGPLOT uses the concept of a border around a plot in which to
put annotations, it does not let an application inquire the size of this
border and so it is not possible to save the border area as a 'FRAME'
picture in the database from within the associated viewport. If such a
'FRAME' picture is required in the database then it must be saved before
the viewport is created, or must be recreated from an existing picture.

PGPLOT is shut down using \htmlref{\textbf{AGP\_DEACT}}{AGP_DEACT}.

The following code segment gives an example of an application that
creates a viewport that matches the size of the current picture if the
overlay flag is true, or creates a viewport with a border for annotation
if the overlay flag is false.
\begin{terminalv}
    *   Open AGI on a device obtained from the parameter system.
    *   Do not clear the viewport if overlay mode is requested.
          IF ( OVER ) THEN
             CALL AGI_ASSOC( 'DEVICE', 'UPDATE', ID1, STATUS )
          ELSE
             CALL AGI_ASSOC( 'DEVICE', 'WRITE', ID1, STATUS )
          ENDIF

    *   Activate the PGPLOT interface to AGI
          CALL AGP_ACTIV( STATUS )

    *   If plotting over previous plot find the last data picture and
    *   match the PGPLOT viewport to it
          IF ( OVER ) THEN

    *   Use the current picture if it is a data picture otherwise
    *   find the last data picture that fits within the current one
             CALL AGI_INAME( CNAME, STATUS )
             IF ( CNAME .NE. 'DATA' ) THEN
                CALL AGI_RCL( 'DATA', ID2, STATUS )
             ENDIF

    *   If a suitable data picture was not found then abort
             IF ( STATUS .NE. SAI__OK ) <goto closedown>

    *   Create the viewport to match the data picture
             CALL AGP_NVIEW( .FALSE., STATUS )

    *   If not using an overlay then create a viewport with a border
          ELSE
             CALL AGP_NVIEW( .TRUE., STATUS )
          ENDIF
\end{terminalv}

The next piece of code shows how the viewport and frame from the previous
example can be stored in the database as pictures. If the overlay mode has
been used then the viewport and associated frame will be the same size.
\begin{terminalv}
    *   Select the picture within which the viewport was created
    *   and inquire its world coordinates
          CALL AGI_SELP( ID1, STATUS )
          CALL AGI_IWOCO( WX1, WX2, WY1, WY2, STATUS )

    *   Save this in the database as the frame picture using the
    *   full extent of the picture.
    *   Define a default world coordinate system for the new picture.
          CALL AGI_NUPIC( WX1, WX2, WY1, WY2, 'FRAME', 'Frame picture',
         :                0.0, 1.0, 0.0, 1.0, FID, STATUS )

    *   Save the current viewport as the data picture
          CALL AGP_SVIEW( 'DATA', 'Data picture', VID, STATUS )
\end{terminalv}

There are wrap-up routines to open and close AGI and PGPLOT within the
ADAM environment using single calls. The sequence of calls \textbf{AGI\_ASSOC},
\textbf{AGI\_BEGIN},  \textbf{AGP\_ACTIV}, \textbf{AGP\_NVIEW} can be replaced by a
single call to
\htmlref{\textbf{AGP\_ASSOC}}{AGP_ASSOC}. As with \textbf{AGS\_ASSOC} an optional call
to \htmlref{\textbf{AGI\_RCL}}{AGI_RCL} can be made by passing a non-blank
name argument.
The sequence of calls \textbf{AGP\_DEACT}, \textbf{AGI\_END}, \textbf{AGI\_CANCL}
or \textbf{AGI\_ANNUL} can be replaced by a single call to
\htmlref{\textbf{AGP\_DEASS}}{AGP_DEASS}.

PGPLOT can be used as a stand-alone package. When GKS-based PGPLOT is
used from the \textbf{AGI} library PGPLOT can also be started from
within an SGS zone. In this case PGPLOT
is initialised while SGS is still open. This is the only situation where
AGI allows more than one graphics package to be open at one time.
The following simplified program section illustrates the opening of
PGPLOT from within an SGS zone.
\begin{terminalv}
    *   Open AGI and SGS on a device obtained from the parameter system
          CALL AGS_ASSOC( 'DEVICE', 'WRITE', ID, IZONE, STATUS )

    *   Create a new SGS zone of the required size and shape
          <Create a new zone with SGS calls>

    *   Open up PGPLOT and create a new viewport matching the current zone
          CALL AGP_ACTIV( STATUS )
          CALL AGP_NVIEW( .FALSE., STATUS )

    *   Create a new PGPLOT environment and draw the picture
          <Use PGPLOT to draw the picture>

    *   Close down PGPLOT, SGS and AGI
          CALL AGP_DEACT( STATUS )
          CALL AGS_DEASS( 'DEVICE', .TRUE., STATUS )
\end{terminalv}

Note that when using the AGP routines to open PGPLOT in the current SGS
zone it is essential that SGS has been opened using the AGS routines.
Normally \textbf{AGP\_NVIEW} creates a viewport that matches the size of
the current picture in the database, but in this instance the size of
the viewport matches the current SGS zone.
The new viewport inherits the world coordinate system of the current picture
in the usual way.

The job of opening and closing the device is handled by the AGP\_ routines
and therefore PGBEGIN and PGEND should not be called in an application
using AGI. Similarly the routines PGENV, PGPAPER and PGVSTAND should not be
called as these subvert the underlying coordinate system.

\subsection{Interface to SGS -- (\textbf{AGI} library only)}

A set of routines are supplied to interface the database to the SGS graphics
package, these are prefixed by AGS\_ instead of AGI\_.
A picture in the database is equated with an SGS zone.
SGS is activated with a call to \htmlref{\textbf{AGS\_ACTIV}}{AGS_ACTIV}.
This does not open a device for plotting.
A device is opened when the first call to \htmlref{\textbf{AGS\_NZONE}}{AGS_NZONE}
is made.
The actual device that is opened will be the one associated with the
current picture in the database.
This means that AGI has to be open before \textbf{AGS\_NZONE} can be called.
\textbf{AGS\_NZONE} creates an SGS zone from the current AGI picture, and returns
the zone identifier.
An SGS zone is saved as a picture in the database using a call to
\htmlref{\textbf{AGS\_SZONE}}{AGS_SZONE}.
The size of the new picture must be less than or equal
to the size of the current picture in the database otherwise and error
status is returned.
SGS is shut down using \htmlref{\textbf{AGS\_DEACT}}{AGS_DEACT}.

The job of opening and closing the device is handled by the AGS\_ routines
and therefore the routines \xref{\textbf{SGS\_ASSOC}}{sun113}{SGS_ASSOC}
and \xref{\textbf{SGS\_CANCL}}{sun113}{SGS_CANCL} or
\xref{\textbf{SGS\_OPEN}}{sun85}{SGS_OPEN} and
\xref{\textbf{SGS\_CLOSE}}{sun85}{SGS_CLOSE} should not be called when using the
AGS\_ routines.

The first example shows how an application might save an SGS zone in the
database.
\begin{terminalv}
    *   Obtain the device name from the ADAM parameter system
          CALL AGI_ASSOC( 'DEVICE', 'WRITE', ID1, STATUS )
          CALL AGI_BEGIN

    *   Open the given device for plotting
          CALL AGS_ACTIV( STATUS )
          CALL AGS_NZONE( IZONE, STATUS )

    *   Plot the data in an SGS zone
          <plot the data with SGS>

    *   Save the current zone as a picture in the database
          CALL AGS_SZONE( 'DATA', 'Description', ID2, STATUS )

    *   Close down
          CALL AGS_DEACT( STATUS )
          CALL AGI_END( -1, STATUS )
          CALL AGI_CANCL( 'DEVICE', STATUS )
\end{terminalv}

The second example shows how another application could recreate a zone,
perhaps to overlay one plot with another, or to obtain coordinates with
a cursor. It assumes that the current picture in the database is the one
that is wanted. If this is not the case the recall routines can be used
to search the database for the correct picture.
\begin{terminalv}
    *   Obtain the device name from the ADAM parameter system
    *   Use update mode to ensure the original plot is not cleared
          CALL AGI_ASSOC( 'DEVICE', 'UPDATE', ID1, STATUS )
          CALL AGI_BEGIN

    *   Open the given device for plotting recreating the current
    *   picture as an SGS zone
          CALL AGS_ACTIV( STATUS )
          CALL AGS_NZONE( IZONE, STATUS )

    *   Plot with SGS
          <plot with SGS>

    *   Close down
          CALL AGS_DEACT( STATUS )
          CALL AGI_END( -1, STATUS )
          CALL AGI_CANCL( 'DEVICE', STATUS )
\end{terminalv}

These examples show that the AGI operations of accessing the database and
opening the graphics package are common to both applications. These
common opening and closing operations have been packaged up into single
routines. Thus the sequence of calls \textbf{AGI\_ASSOC}, \textbf{AGI\_BEGIN},
\textbf{AGS\_ACTIV}, \textbf{AGS\_NZONE} can be replaced by a single call to
\htmlref{\textbf{AGS\_ASSOC}}{AGS_ASSOC},
and the sequence of calls \textbf{AGS\_DEACT},
\textbf{AGI\_END}, \textbf{AGI\_CANCL} can be replaced by a single call to
\htmlref{\textbf{AGS\_DEASS}}{AGS_DEASS}.
Note that these wrap-up routines only exist for the
ADAM interface. If the application is using the non-ADAM routines
\textbf{AGI\_OPEN} and \textbf{AGI\_CLOSE} then the sequence of calls has to
be made explicitly. Using these wrap-up routines then second application
above would become
\begin{terminalv}
    *   Obtain the device name from the ADAM parameter system and open
    *   SGS on the given device
          CALL AGS_ASSOC( 'DEVICE', 'UPDATE', ' ', ID1, IZONE, STATUS )

    *   Plot with SGS
          <plot with SGS>

    *   Close down
          CALL AGS_DEASS( 'DEVICE', .TRUE., STATUS )
\end{terminalv}
The third (picture name) argument in \textbf{AGS\_ASSOC} can optionally be
used to recall the last picture of the given name using \htmlref{\textbf{AGI\_RCL}}{AGI_RCL}.
If the name string is blank (as in the example above) then this recall
routine is not called and the current database picture is returned.
If the second argument in \textbf{AGS\_DEASS} is false then the ADAM parameter
is not cancelled and instead the picture identifier returned from
\textbf{AGS\_ASSOC} is annulled.

If GKS calls are mixed in with the SGS calls to produce the plot
then some GKS operations may result in the plot being cleared when the
device is shut down. This can happen if the device is opened with 'WRITE'
mode, which results in the zone being initially cleared, and then some
aspect of the device, such as a pen colour, is changed using GKS calls.
GKS believes that the original plot is now wrong because the colours are
wrong and requests that the plot be redrawn in the new colours.
By default this request does not happen immediately but is deferred until
closedown. An application can force the update to happen immediately by
setting the deferral mode:
\begin{terminalv}
    *   Set the GKS deferral mode to 'ASAP'
          CALL SGS_ICURW( IWKID )
          CALL GSDS( IWKID, 0, 1 )

\end{terminalv}

\subsection{Interface to IDI -- (\textbf{AGI} library only)}

A set of routines are supplied to interface the database to the IDI graphics
package, these are prefixed by AGD\_ instead of AGI\_. IDI is a low-level
graphics package that does not have the kind of abstraction that packages
like SGS and PGPLOT have. The coordinate system is not device independent
and there is no natural structure in the specification that corresponds to
a picture in the way that an SGS zone does. The AGI interface to IDI is
therefore more contrived than for the other packages.

IDI instructions are directed to a specific device by means of a display
identifier that has to be supplied as an argument to the IDI subroutines.
The routine \htmlref{\textbf{AGD\_NWIND}}{AGD_NWIND}
returns a display identifier that corresponds
to the device associated with the current picture. This identifier can
then be used as input to the IDI routines. If the device is not already
open then \textbf{AGD\_NWIND} will open the relevant device. Subsequent calls
to \textbf{AGD\_NWIND} will return the same identifier if the current picture
is associated with the same device.

The routines \htmlref{\textbf{AGD\_ACTIV}}{AGD_ACTIV} and
\htmlref{\textbf{AGD\_DEACT}}{AGD_DEACT} should be called before
and after any other AGD\_ routines.

IDI has no structure that matches the concept of a picture in the database,
and so an abstract space has been defined, here called simply a window, which
represents a rectangular space in the memory of the device. The definition
of such a window is similar to a the definition of a transfer window in IDI,
which is used to load images into the display memory.
IDI works in pixel coordinates and a window is defined by its size in
pixels along each axis (XSIZE and YSIZE), and by an offset, in pixels,
of its bottom left hand corner from the memory origin (XOFF and YOFF).
The routine \htmlref{\textbf{AGD\_SWIND}}{AGD_SWIND}
will save such a window as a picture in the
database. The following code segment shows how a transfer window which has
been set up to load an image into a memory can be saved as a picture in the
database.
\begin{terminalv}
    *   Set up the transfer window defined by XSIZE, YSIZE, XOFF, YOFF
    *   to match the size of the image NX * NY
          XSIZE = NX
          YSIZE = NY
          CALL IIMSTW( DISPID, MEMID, 0, XSIZE, YSIZE, 8, XOFF, YOFF, STATUS )

    *   Load the image held in the array ADATA
          NPIX = NX * NY
          CALL IIMWMY( DISPID, MEMID, ADATA, NPIX, 8, 1, 0, 0, STATUS )

    *   Store the transfer window as a picture in the database
          CALL AGD_SWIND( DISPID, MEMID, XSIZE, YSIZE, XOFF, YOFF,
         :                'DATA', 'IDI transfer window', 0.0, REAL( NX ),
         :                 0.0, REAL( NY ), ID, STATUS )
\end{terminalv}
As usual when creating a picture in the database if this new picture lies
outside the current picture an error status is returned.

The routine \htmlref{\textbf{AGD\_NWIND}}{AGD_NWIND}
will return the size and offsets of a window that
matches the current picture in the database. The returned arguments should
not be used to define a transfer window directly unless it is known that
the picture exactly matches the size and shape of the image to be loaded
into the transfer window. This is because IDI uses the transfer window to
define where the end of one row occurs in the input stream of pixel values.
If the transfer window is the wrong shape for the image then the line
breaks will not occur in the correct place and the image will appear
garbled. When using \textbf{AGD\_NWIND} to set up a transfer window it is best
to compare the returned arguments to the size of the image and set up the
transfer window accordingly. The following code segment shows how a
transfer window is set up so that the image loads from the bottom left of
the picture. The transfer window is set up so that the bottom left corner
of the image is aligned with the bottom left corner of the picture, but the
size of the transfer window is defined by the size of the image and not the
size of the picture.
\begin{terminalv}
    *   Get the size of the current picture from the database
          CALL AGD_NWIND( DISPID, MEMID, XSIZE, YSIZE, XOFF, YOFF, STATUS )

    *   Check that the image does not overflow the picture
          IF ( ( NX .LE. XSIZE ) .AND. ( NY .LE. YSIZE ) ) THEN

    *   Set up the transfer window to match the size of the image NX * NY
    *   and to begin at the bottom left of the picture
             CALL IIMSTW( DISPID, MEMID, 0, NX, NY, 8, XOFF, YOFF, STATUS )

    *   Load the image held in the array ADATA
             NPIX = NX * NY
             CALL IIMWMY( DISPID, MEMID, ADATA, NPIX, 8, 1, 0, 0, STATUS )
          ENDIF
\end{terminalv}
There is a slight difference between the definition of an IDI window and
a picture in the database due to the former being stored as integers and
the latter being stored as floating point numbers. This does not matter
when saving a window in the database because the integer coordinates are
simply expressed as reals in the database, but a picture already defined in
the database may have limits that correspond to partial pixels on the screen.
In the latter case the pixel coordinates returned by the routine
\textbf{AGD\_NWIND} are rounded up from the picture coordinates so that the
window contains the whole picture. This means that the resulting window
may be slightly larger (by fractions of a pixel) than the original picture.

There are wrap-up routines to open and close AGI and IDI within the
ADAM environment using single calls. The sequence of calls \textbf{AGI\_ASSOC},
\textbf{AGI\_BEGIN},  \textbf{AGD\_ACTIV}, \textbf{AGD\_NWIND} can be replaced by a
single call to \htmlref{\textbf{AGD\_ASSOC}}{AGD_ASSOC}.
As with \textbf{AGS\_ASSOC} an optional call
to \htmlref{\textbf{AGI\_RCL}}{AGI_RCL}
can be made by passing a non-blank name argument.
The sequence of calls \textbf{AGD\_DEACT}, \textbf{AGI\_END}, \textbf{AGI\_CANCL}
or \textbf{AGI\_ANNUL} can be replaced by a single call to
\htmlref{\textbf{AGD\_DEASS}}{AGD_DEASS}.

The job of opening and closing the device is handled by the AGD\_ routines
and therefore the routines \textbf{IDI\_ASSOC} and \textbf{IDI\_CANCL} or
\textbf{IIDOPN} and \textbf{IIDCLO} should not be called when using the
AGD\_ routines.

\section {Additional Postscript Features for Native PGPLOT}\label{aps}
When AGI is used to open the native PGPLOT graphics system for output to
a Postscript file, several extra ``pseudo-devices'' are recognised that
provide additional features over and above the corresponding standard
PGPLOT devices. These device names are:

\begin{center}
\begin{tabular}{ l l l }
GNS name  & PGPLOT name  & Description \\
\hline
aps\_p    & /AVPS        & Accumulating EPS, monochrome, portrait \\
aps\_l    & /APS         & Accumulating EPS, monochrome, landscape \\
apscol\_p & /AVCPS       & Accumulating EPS, color, portrait \\
apscol\_l & /ACPS        & Accumulating EPS, color, landscape \\
\end{tabular}
\end{center}

AGI strips the leading ``A'' from the above PGPLOT device names to create
the device names that are actually used by PGPLOT. The name of the file
to receive the Postscript output can be included in the device name as
normal (\emph{e.g.} ``\texttt{fred.ps/ACPS}'' or ``\texttt{apscol\_l;fred.ps}'').

If one of these devices is used, then the AGI library will automatically
concatenate the new Postscript output created by PGPLOT, with any old
Postscript in the same file after PGPLOT is closed down. It does this by
temporarily changing the name of the output file by adding the string
``AGIPS\_'' to the start of the file name before opening PGPLOT. PGPLOT
then writes the new Postscript to this temporary file. When PGPLOT is
closed down, AGI appends the contents of this temporary file to the end
of any pre-existing file with the specified name\footnote{If no file with
the given name already exists, then the temporary file is just renamed by
the removal of the ``AGIPS\_'' prefix.}.

Thus, using these devices allows complex composite Postscript pictures to
be created without the need to use external tools such as psmerge (see
\xref{SUN/164}{sun164}{}) to merge individual Postscript files.

Some care is taken to keep the size of the merged Postscript file to a
minimum. For instance, if a new picture obscures an old picture (and the
new picture has an opaque background) then the old picture is not
included in the merged Postscript file. Additionally, Postscript files
that do not generate any visible output are excluded from the merged
file. Such empty files can be created for instance (when using the
standard Postscript devices), by the KAPPA:PICSEL command that simply
selects a different AGI picture but does not draw anything.

An additional minor feature of this merging process - the \texttt{BoundingBox}
comment that PGPLOT places at the end of the Postscript output is moved
to the start by this process. This allows a wider range of applications
to read the resulting Postscript file.

Some Postscript viewing tools such as Okular (see \url{http://okular.kde.org}) will
automatically re-draw the display if the contents of the displayed file
changes on disk. Combined with these ``accumulating'' Postscript devices,
this provides a scheme that is similar to the use of a traditional
persistent X-window device. A typical scenario could be:

\begin{terminalv}
   % rm pgplot.ps
   % kappa
   % gdset /acps
   % gdclear
   % okular pgplot.ps &
   % picdef mode=a xpic=2 ypic=2 prefix=a outline=no
   % display $KAPPA_DIR/m31 accept
   % picsel a2
   % linplot $KAPPA_DIR/m31'(,150)' style=def
   % linplot clear=no $KAPPA_DIR/m31'(,140)' style='+colour=red'
   % picsel a3
   % display $KAPPA_DIR/m57 accept
\end{terminalv}

The \texttt{okular} display will update as each KAPPA command adds additional
pictures into the \texttt{pgplot.ps} file. Some notes:

\begin{enumerate}
\item The \texttt{okular} command was put into the background by the trailing
``\&'' character in order to allow subsequent commands to execute.
\item The LINPLOT style setting begins with a ``+'' to indicate that it
should be used only for one invocation of LINPLOT.
\end{enumerate}

\section {Linking \label{linking}}

To include either of the AGI include files, create links in the directory
containing the program source code with the command:
\begin{quote}\texttt{
agi\_dev}
\end{quote}
and use and include statement such as:
\begin{terminalv}
      INCLUDE 'AGI_ERR'
      INCLUDE 'AGI_PAR'
\end{terminalv}


For programs in the ADAM environment one of the the shell scripts
\texttt{agi\_link\_adam} or \texttt{agp\_link\_adam}
should be used. For example to compile and link an ADAM task called `task.f'
which is designed to use PGPLOT only the following is used:
\begin{terminalv}
          % alink task.f `agp_link_adam`
\end{terminalv}
(note the use of the backward quotes).

A standalone program is linked by specifying one of \texttt{`agi\_link`}
or \texttt{`agp\_link`} on the compiler
command line. Thus to compile and link a standalone application called
`prog.f' which, this time, uses GKS based graphics the following is used:
\begin{terminalv}
          % f77 prog.f -o prog `agi_link`
\end{terminalv}

\newpage

\section {Routine Specifications}
\begin{small}
\sstroutine{
   AGD\_ACTIV
}{
   Initialise IDI
}{
   \sstdescription{
      Initialise IDI. This has to be called before any other AGD or
      IDI routines. An error is returned if this or any other graphics
      interface is already active.
   }
   \sstinvocation{
      CALL AGD\_ACTIV( STATUS )
   }
   \sstarguments{
      \sstsubsection{
         STATUS = INTEGER (Given and Returned)
      }{
         The global status
      }
   }
}
\sstroutine{
   AGD\_ASSOC
}{
   Associate a device with AGI and IDI
}{
   \sstdescription{
      This is a wrap-up routine to associate a device with the AGI
      database via the ADAM parameter system and open IDI on it.
      The size and position of an IDI window corresponding to the
      current picture in the database is returned. This routine calls
      \htmlref{AGI\_ASSOC}{AGI_ASSOC}, \htmlref{AGI\_BEGIN}{AGI_BEGIN} ,
      \htmlref{AGD\_ACTIV}{AGD_ACTIV} and \htmlref{AGD\_NWIND}{AGD_NWIND}.
      Also if the name string is not blank then
      \htmlref{AGI\_RCL}{AGI_RCL} is called to recall the last
      picture of that name. This routine should be matched by a closing
      call to \htmlref{AGD\_DEASS}{AGD_DEASS}.
   }
   \sstinvocation{
       \parbox[t]{135mm}{CALL AGD\_ASSOC ( PARAM, ACMODE, PNAME, MEMID, PICID,
                      DISPID, \\ XSIZE, YSIZE, XOFF, YOFF, STATUS )}
   }
   \sstarguments{
      \sstsubsection{
         PARAM = CHARACTER $*$ ( $*$ ) (Given)
      }{
         The name of the ADAM parameter for accessing device names
      }
      \sstsubsection{
         ACMODE = CHARACTER $*$ ( $*$ ) (Given)
      }{
         Access mode for pictures. \texttt{'}READ\texttt{'}, \texttt{'}WRITE\texttt{'} or \texttt{'}UPDATE\texttt{'}.
      }
      \sstsubsection{
         PNAME = CHARACTER $*$ ( $*$ ) (Given)
      }{
         Recall last picture of this name if not blank.
      }
      \sstsubsection{
         MEMID = INTEGER (Given)
      }{
         IDI Memory identifier.
      }
      \sstsubsection{
         PICID = INTEGER (Returned)
      }{
         Picture identifier for current picture on given device.
      }
      \sstsubsection{
         DISPID = INTEGER (Returned)
      }{
         IDI Display identifier.
      }
      \sstsubsection{
         XSIZE = INTEGER (Returned)
      }{
         X size of window
      }
      \sstsubsection{
         YSIZE = INTEGER (Returned)
      }{
         Y size of window
      }
      \sstsubsection{
         XOFF = INTEGER (Returned)
      }{
         X offset of window from memory origin
      }
      \sstsubsection{
         YOFF = INTEGER (Returned)
      }{
         Y offset of window from memory origin
      }
      \sstsubsection{
         STATUS = INTEGER (Given and Returned)
      }{
         The global status.
      }
   }
}
\sstroutine{
   AGD\_DEACT
}{
   Close down IDI
}{
   \sstdescription{
      Close down IDI whatever the value of status. This should be called
      after all AGD and IDI routines.
   }
   \sstinvocation{
      CALL AGD\_DEACT( STATUS )
   }
   \sstarguments{
      \sstsubsection{
         STATUS = INTEGER (Given and Returned)
      }{
         The global status
      }
   }
}
\sstroutine{
   AGD\_DEASS
}{
   Deassociate a device from AGI and IDI
}{
   \sstdescription{
      This is a wrap-up routine to deassociate a device from the AGI
      database and to close down IDI. The picture current when
      AGD\_ASSOC was called is reinstated. This routine calls
      \htmlref{AGD\_DEACT}{AGD_DEACT}, \htmlref{AGI\_END}{AGI_END}
      and either \htmlref{AGI\_CANCL}{AGI_CANCL} or
      \htmlref{AGI\_ANNUL}{AGI_ANNUL}. This routine is
      executed regardless of the given value of status.
   }
   \sstinvocation{
      CALL AGD\_DEASS( PARAM, PARCAN, STATUS )
   }
   \sstarguments{
      \sstsubsection{
         PARAM = CHARACTER $*$ ( $*$ ) (Given)
      }{
         The name of the ADAM parameter associated with the device.
      }
      \sstsubsection{
         PARCAN = LOGICAL (Given)
      }{
         If true the parameter given by PARAM is cancelled, otherwise
         it is annulled.
      }
      \sstsubsection{
         STATUS = INTEGER (Given and Returned)
      }{
         The global status.
      }
   }
}
\sstroutine{
   AGD\_NWIND
}{
   Define an IDI window from the current picture
}{
   \sstdescription{
      Define an IDI window in the given memory from the current picture.
      The window coordinates define the size of the rectangular area in
      pixels and the offset of its bottom left corner from the memory
      origin. The window is defined as the smallest possible area, made
      up of whole pixels, that completely contains the picture. If the
      device associated with the current picture is not already open
      then this routine will open the device and return the display
      identifier. Furthermore if the device was opened with \texttt{'}WRITE\texttt{'}
      access (in \htmlref{AGI\_ASSOC}{AGI_ASSOC} or
      \htmlref{AGI\_OPEN}{AGI_OPEN}) then the window will be cleared.
      If the device is already open then the display identifier will be
      the same as previously and the device will not be cleared.
   }
   \sstinvocation{
      CALL AGD\_NWIND( MEMID, DISPID, XSIZE, YSIZE, XOFF, YOFF, STATUS )
   }
   \sstarguments{
      \sstsubsection{
         MEMID = INTEGER (Given)
      }{
         Memory identifier
      }
      \sstsubsection{
         DISPID = INTEGER (Returned)
      }{
         Display identifier
      }
      \sstsubsection{
         XSIZE = INTEGER (Returned)
      }{
         X size of window
      }
      \sstsubsection{
         YSIZE = INTEGER (Returned)
      }{
         Y size of window
      }
      \sstsubsection{
         XOFF = INTEGER (Returned)
      }{
         X offset of window from origin
      }
      \sstsubsection{
         YOFF = INTEGER (Returned)
      }{
         Y offset of window from origin
      }
      \sstsubsection{
         STATUS = INTEGER (Given and Returned)
      }{
         The global status
      }
   }
}
\sstroutine{
   AGD\_SWIND
}{
   Save an IDI window in the database
}{
   \sstdescription{
      Save an IDI window as a picture in the database. The new picture
      must be equal in size or smaller than the current picture in the
      database. The window coordinates define the size of the
      rectangular area in pixels and the offset of its bottom left
      corner from the memory origin. The name of the picture and a
      comment are used to identify the picture in the database. The name
      string has leading blanks removed and is converted to upper case.
      The world coordinates define the user coordinate system and are
      saved in the database as given. They should be linear and increasing
      left to right and bottom to top. If the picture was successfully
      created then a valid picture identifier is returned and the new
      picture becomes the current picture.
   }
   \sstinvocation{
       \parbox[t]{135mm}{CALL AGD\_SWIND( DISPID, MEMID, XSIZE, YSIZE, XOFF, YOFF,
                      PNAME, COMENT, \\  WX1, WX2, WY1, WY2, PICID, STATUS )}
   }
   \sstarguments{
      \sstsubsection{
         DISPID = INTEGER (Given)
      }{
         Display identifier
      }
      \sstsubsection{
         MEMID = INTEGER (Given)
      }{
         Memory identifier
      }
      \sstsubsection{
         XSIZE = INTEGER (Given)
      }{
         X size of window (pixels)
      }
      \sstsubsection{
         YSIZE = INTEGER (Given)
      }{
         Y size of window (pixels)
      }
      \sstsubsection{
         XOFF = INTEGER (Given)
      }{
         X offset of window from origin (pixels)
      }
      \sstsubsection{
         YOFF = INTEGER (Given)
      }{
         Y offset of window from origin (pixels)
      }
      \sstsubsection{
         PNAME = CHARACTER $*$ ( $*$ ) (Given)
      }{
         Name of picture
      }
      \sstsubsection{
         COMENT = CHARACTER $*$ ( $*$ ) (Given)
      }{
         Description of picture
      }
      \sstsubsection{
         WX1 = REAL (Given)
      }{
         World coordinate of left edge of new picture
      }
      \sstsubsection{
         WX2 = REAL (Given)
      }{
         World coordinate of right edge of new picture
      }
      \sstsubsection{
         WY1 = REAL (Given)
      }{
         World coordinate of bottom edge of new picture
      }
      \sstsubsection{
         WY2 = REAL (Given)
      }{
         World coordinate of top edge of new picture
      }
      \sstsubsection{
         PICID = INTEGER (Returned)
      }{
         Picture identifier
      }
      \sstsubsection{
         STATUS = INTEGER (Given and Returned)
      }{
         The global status
      }
   }
}
\sstroutine{
   AGI\_ANNUL
}{
   Annul a picture identifier
}{
   \sstdescription{
      Annul the picture identifier. If this is the last active
      identifier then the database is closed. This routine is
      executed regardless of the given value of status.
   }
   \sstinvocation{
      CALL AGI\_ANNUL( PICID, STATUS )
   }
   \sstarguments{
      \sstsubsection{
         PICID = INTEGER (Given)
      }{
         Picture identifier
      }
      \sstsubsection{
         STATUS = INTEGER (Given and Returned)
      }{
         The global status
      }
   }
}
\sstroutine{
   AGI\_ASSOC
}{
   Associate an AGI device with an ADAM parameter
}{
   \sstdescription{
      Associate an AGI device with a parameter in the ADAM environment
      and return an identifier to the current picture. If there are no
      pictures on the device then a base picture is created and made
      current. If the size of the display window has changed since a
      previous database operation the database is cleared and a message
      sent to the user. The access mode does not affect the database
      operation, but it is used by the graphics system to determine if
      the display should be cleared the first time the device is opened;
      \texttt{'}READ\texttt{'} and \texttt{'}UPDATE\texttt{'} access do not clear the display, but \texttt{'}WRITE\texttt{'}
      access does.
   }
   \sstinvocation{
      CALL AGI\_ASSOC( PARAM, ACMODE, PICID, STATUS )
   }
   \sstarguments{
      \sstsubsection{
         PARAM = CHARACTER $*$ ( $*$ ) (Given)
      }{
         Name of the parameter used for accessing the device
      }
      \sstsubsection{
         ACMODE = CHARACTER $*$ ( $*$ ) (Given)
      }{
         Access mode: \texttt{'}READ\texttt{'}, \texttt{'}WRITE\texttt{'} or \texttt{'}UPDATE\texttt{'}
      }
      \sstsubsection{
         PICID = INTEGER (Returned)
      }{
         Identifier for current picture on the given device
      }
      \sstsubsection{
         STATUS = INTEGER (Given and Returned)
      }{
         The global status
      }
   }
}
\sstroutine{
   AGI\_BEGIN
}{
   Mark the beginning of a new AGI scope
}{
   \sstdescription{
      Mark the beginning of a new AGI scope. This should be matched with
      a call to \htmlref{AGI\_END}{AGI_END}. Up to eight levels of nested
      begin-end blocks are allowed.
   }
   \sstinvocation{
      CALL AGI\_BEGIN
   }
}
\sstroutine{
   AGI\_CANCL
}{
   Cancel the ADAM device parameter
}{
   \sstdescription{
      Cancel the association of the ADAM device parameter with AGI. Any
      picture identifiers associated with this parameter are annulled.
      This routine is executed regardless of the given value of status.
   }
   \sstinvocation{
      CALL AGI\_CANCL( PARAM, STATUS )
   }
   \sstarguments{
      \sstsubsection{
         PARAM = CHARACTER $*$ ( $*$ ) (Given)
      }{
         Name of the parameter used for accessing the device
      }
      \sstsubsection{
         STATUS = INTEGER (Given and Returned)
      }{
         The global status
      }
   }
}
\sstroutine{
   AGI\_CLOSE
}{
   Close AGI in non-ADAM environments
}{
   \sstdescription{
      Close AGI in non-ADAM environments. The database file is closed.
   }
   \sstinvocation{
      CALL AGI\_CLOSE( STATUS )
   }
   \sstarguments{
      \sstsubsection{
         STATUS = INTEGER (Given and Returned)
      }{
         The global status
      }
   }
}
\sstroutine{
   AGI\_END
}{
   Mark the end of an AGI scope
}{
   \sstdescription{
      Mark the end of an AGI scope. The given picture is made the
      current one. If the argument is a negative number then the picture
      current when the matching \htmlref{AGI\_BEGIN}{AGI_BEGIN}
      was called is made current.
      All identifiers allocated within this begin-end block are annulled.
      If the last active identifier is annulled the database is closed.
   }
   \sstinvocation{
      CALL AGI\_END( PICID, STATUS )
   }
   \sstarguments{
      \sstsubsection{
         PICID = INTEGER (Given)
      }{
         Picture identifier
      }
      \sstsubsection{
         STATUS = INTEGER (Given and Returned)
      }{
         The global status
      }
   }
}
\sstroutine{
   AGI\_GTREF
}{
   Get a reference object from a picture
}{
   \sstdescription{
      This returns a reference to a data object which has been stored
      in the database. The picture identifier signifies which picture
      to obtain the reference from. If this identifier is negative then
      the current picture is used. If no reference object is found for
      the given picture an error is returned. If the reference was
      created from an HDS locator then an HDS locator is returned, and
      this should be annulled in the application using
      \xref{REF\_ANNUL}{sun31}{REF_ANNUL} to ensure the file is properly closed.
   }
   \sstinvocation{
      CALL AGI\_GTREF ( PICID, MODE, DATREF, STATUS )
   }
   \sstarguments{
      \sstsubsection{
         PICID = INTEGER (Given)
      }{
         Picture identifier
      }
      \sstsubsection{
         MODE = CHARACTER $*$ ( $*$ ) (Given)
      }{
         Access mode for object, \texttt{'}READ\texttt{'}, \texttt{'}WRITE\texttt{'}, or \texttt{'}UPDATE\texttt{'}
      }
      \sstsubsection{
         DATREF = CHARACTER $*$ ( $*$ ) (Returned)
      }{
         String containing reference object
      }
      \sstsubsection{
         STATUS = INTEGER (Given and Returned)
      }{
         The global status
      }
   }
}
\sstroutine{
   AGI\_IBASE
}{
   Inquire base picture for current device
}{
   \sstdescription{
      A picture identifier for the base picture on the current device
      is returned. The picture is not selected as the current picture.
   }
   \sstinvocation{
      CALL AGI\_IBASE( PICID, STATUS )
   }
   \sstarguments{
      \sstsubsection{
         PICID = INTEGER (Returned)
      }{
         Identifier for base picture
      }
      \sstsubsection{
         STATUS = INTEGER (Given and Returned)
      }{
         The global status
      }
   }
}
\sstroutine{
   AGI\_ICOM
}{
   Inquire comment for the current picture
}{
   \sstdescription{
      The comment string for the current picture is returned.
   }
   \sstinvocation{
      CALL AGI\_ICOM( COMENT, STATUS )
   }
   \sstarguments{
      \sstsubsection{
         COMENT = CHARACTER $*$ ( $*$ ) (Returned)
      }{
         Comment for current picture
      }
      \sstsubsection{
         STATUS = INTEGER (Given and Returned)
      }{
         The global status
      }
   }
}
\sstroutine{
   AGI\_ICURP
}{
   Inquire the current picture
}{
   \sstdescription{
      The picture identifier of the current picture is returned.
   }
   \sstinvocation{
      CALL AGI\_ICURP( PICID, STATUS )
   }
   \sstarguments{
      \sstsubsection{
         PICID = INTEGER (Returned)
      }{
         Picture identifier
      }
      \sstsubsection{
         STATUS = INTEGER (Given and Returned)
      }{
         The global status
      }
   }
}
\sstroutine{
   AGI\_ILAB
}{
   Inquire label of a picture
}{
   \sstdescription{
      Inquire the label of a picture referenced by the identifier. If
      the picture identifier is negative then the current picture is
      searched. If no label is associated with this picture then a
      blank string is returned.
   }
   \sstinvocation{
      CALL AGI\_ILAB ( PICID, LABEL, STATUS )
   }
   \sstarguments{
      \sstsubsection{
         PICID = INTEGER (Given)
      }{
         Picture identifier
      }
      \sstsubsection{
         LABEL = CHARACTER $*$ (AGI\_\_SZLAB) (Returned)
      }{
         Label string
      }
      \sstsubsection{
         STATUS = INTEGER (Given and Returned)
      }{
         The global status
      }
   }
}
\sstroutine{
   AGI\_IMORE
}{
   Inquire if a MORE structure exists
}{
   \sstdescription{
      Inquire if a MORE structure exists for the given picture. If the
      given value for PICID is negative then the current picture is used.
      The return argument is true if a MORE structure exists for the
      picture otherwise it is false.
   }
   \sstinvocation{
      CALL AGI\_IMORE( PICID, LMORE, STATUS )
   }
   \sstarguments{
      \sstsubsection{
         PICID = INTEGER (Given)
      }{
         Picture identifier
      }
      \sstsubsection{
         LMORE = LOGICAL (Returned)
      }{
         Locator to the transformation structure
      }
      \sstsubsection{
         STATUS = INTEGER (Given and Returned)
      }{
         The global status
      }
   }
}
\sstroutine{
   AGI\_INAME
}{
   Inquire name of the current picture
}{
   \sstdescription{
      The name of the current picture is returned.
   }
   \sstinvocation{
      CALL AGI\_INAME ( PNAME, STATUS )
   }
   \sstarguments{
      \sstsubsection{
         PNAME = CHARACTER $*$ ( $*$ ) (Returned)
      }{
         Name of current picture
      }
      \sstsubsection{
         STATUS = INTEGER (Given and Returned)
      }{
         The global status
      }
   }
}
\sstroutine{
   AGI\_IPOBS
}{
   Is current picture obscured by another?
}{
   \sstdescription{
      Inquire if the current picture is obscured, either totally or
      partially by another picture. Obscured means that a picture
      intersects the current picture and was created more recently.
      If the input value for the picture identifier is negative the
      current picture is tested against all other overlying pictures;
      i.e. those created more recently than the current picture. If the
      picture identifier corresponds to a valid picture then the current
      one is only tested against the given one.
   }
   \sstinvocation{
      CALL AGI\_IPOBS( PICID, LOBS, STATUS )
   }
   \sstarguments{
      \sstsubsection{
         PICID = INTEGER (Given)
      }{
         Picture identifier.
      }
      \sstsubsection{
         LOBS = LOGICAL (Returned)
      }{
         True if picture is obscured, otherwise false.
      }
      \sstsubsection{
         STATUS = INTEGER (Given and Returned)
      }{
         The global status
      }
   }
}
\sstroutine{
   AGI\_ISAMD
}{
   Inquire if pictures are on same device
}{
   \sstdescription{
      Inquire if the given picture is on the same device as the current
      picture.
   }
   \sstinvocation{
      CALL AGI\_ISAMD( PICID, LSAME, STATUS )
   }
   \sstarguments{
      \sstsubsection{
         PICID = INTEGER (Given)
      }{
         Picture identifier
      }
      \sstsubsection{
         LSAME = LOGICAL (Returned)
      }{
         True if pictures on same device, otherwise false.
      }
      \sstsubsection{
         STATUS = INTEGER (Given and Returned)
      }{
         The global status
      }
   }
}
\sstroutine{
   AGI\_ISAMP
}{
   Inquire if two pictures are the same
}{
   \sstdescription{
      Inquire if a picture identifier references the same picture as
      the current picture. The picture referenced by the given picture
      identifier is compared with the current picture to see if they
      point to the same picture in the database.
   }
   \sstinvocation{
      CALL AGI\_ISAMP( PICID, LSAME, STATUS )
   }
   \sstarguments{
      \sstsubsection{
         PICID = INTEGER (Given)
      }{
         Picture identifier.
      }
      \sstsubsection{
         LSAME = LOGICAL (Returned)
      }{
         True if the pictures are the same, otherwise false.
      }
      \sstsubsection{
         STATUS = INTEGER (Given and Returned)
      }{
         The global status.
      }
   }
}
\sstroutine{
   AGI\_ITOBS
}{
   Inquire if test points are obscured
}{
   \sstdescription{
      Inquire if the members of the array of test points are obscured
      by any picture overlying (create more recently than) the current
      picture. The points are defined in the world coordinate system of
      the current picture. An array of logical values is returned
      containing true if the corresponding point is obscured, otherwise
      false.
   }
   \sstinvocation{
      CALL AGI\_ITOBS( NXY, X, Y, LTOBS, STATUS )
   }
   \sstarguments{
      \sstsubsection{
         NXY = INTEGER (Given)
      }{
         Number of test points
      }
      \sstsubsection{
         X = REAL(NXY) (Given)
      }{
         Array of x coordinates of test points
      }
      \sstsubsection{
         Y = REAL(NXY) (Given)
      }{
         Array of y coordinates of test points
      }
      \sstsubsection{
         LTOBS = LOGICAL(NXY) (Returned)
      }{
         Array of results. True if point is obscured, otherwise false.
      }
      \sstsubsection{
         STATUS = INTEGER (Given and Returned)
      }{
         The global status
      }
   }
}
\sstroutine{
   AGI\_IWOCO
}{
   Inquire world coordinates of current picture
}{
   \sstdescription{
      Return the world coordinate limits of the current picture.
   }
   \sstinvocation{
      CALL AGI\_IWOCO( WX1, WX2, WY1, WY2, STATUS )
   }
   \sstarguments{
      \sstsubsection{
         WX1 = REAL (Returned)
      }{
         World coordinate of left edge of picture
      }
      \sstsubsection{
         WX2 = REAL (Returned)
      }{
         World coordinate of right edge of picture
      }
      \sstsubsection{
         WY1 = REAL (Returned)
      }{
         World coordinate of bottom edge of picture
      }
      \sstsubsection{
         WY2 = REAL (Returned)
      }{
         World coordinate of top edge of picture
      }
      \sstsubsection{
         STATUS = INTEGER (Given and Returned)
      }{
         The global status
      }
   }
}
\sstroutine{
   AGI\_MORE
}{
   Return an HDS locator to a MORE structure
}{
   \sstdescription{
      An HDS locator that points to a MORE structure in the database
      is returned. A MORE structure can be associated with any picture
      and it can be used to store any application specific information.
      If the given value for PICID is negative then the current picture
      is used. If the access mode is \texttt{'}WRITE\texttt{'} then an empty MORE structure
      is created if none existed. If there is an existing structure then
      \texttt{'}WRITE\texttt{'} mode will erase the existing contents and return a locator
      to an empty structure. If the access mode is \texttt{'}READ\texttt{'} or \texttt{'}UPDATE\texttt{'} then
      a locator to an existing structure is returned. In this case an
      error is returned if there is not a MORE structure for the given
      picture. The database is not responsible for what goes in the MORE
      structure or how the information is used. The application is also
      responsible for annulling the returned locator.
   }
   \sstinvocation{
      CALL AGI\_MORE( PICID, ACMODE, MORLOC, STATUS )
   }
   \sstarguments{
      \sstsubsection{
         PICID = INTEGER (Given)
      }{
         Picture identifier
      }
      \sstsubsection{
         ACMODE = CHARACTER $*$ ( $*$ ) (Given)
      }{
         Access mode for MORE structure. \texttt{'}READ\texttt{'}, \texttt{'}WRITE\texttt{'} or \texttt{'}UPDATE\texttt{'}.
      }
      \sstsubsection{
         MORLOC = CHARACTER $*$ (DAT\_\_SZLOC) (Returned)
      }{
         Locator to the transformation structure
      }
      \sstsubsection{
         STATUS = INTEGER (Given and Returned)
      }{
         The global status
      }
   }
}
\sstroutine{
   AGI\_NUPIC
}{
   Create a new picture in the database
}{
   \sstdescription{
      Create a new picture in the database. The extent of the new
      picture is defined in the world coordinate system of the current
      picture. The world coordinates of this new picture are set to the
      values passed. The name string has leading blanks removed and is
      converted to upper case. The new picture is selected as the
      current one and a picture identifier returned.
   }
   \sstinvocation{
       \parbox[t]{135mm}{CALL AGI\_NUPIC( WX1, WX2, WY1, WY2, PNAME, COMENT, \\
                      NEWX1, NEWX2, NEWY1, NEWY2, PICID, STATUS )}
   }
   \sstarguments{
      \sstsubsection{
         WX1 = REAL (Given)
      }{
         Current world coordinate of left edge of picture
      }
      \sstsubsection{
         WX2 = REAL (Given)
      }{
         Current world coordinate of right edge of picture
      }
      \sstsubsection{
         WY1 = REAL (Given)
      }{
         Current world coordinate of bottom edge of picture
      }
      \sstsubsection{
         WY2 = REAL (Given)
      }{
         Current world coordinate of top edge of picture
      }
      \sstsubsection{
         PNAME = CHARACTER $*$ ( $*$ ) (Given)
      }{
         Name of new picture
      }
      \sstsubsection{
         COMENT = CHARACTER $*$ ( $*$ ) (Given)
      }{
         Comment for new picture
      }
      \sstsubsection{
         NEWX1 = REAL (Given)
      }{
         World coordinate of left edge of new picture
      }
      \sstsubsection{
         NEWX2 = REAL (Given)
      }{
         World coordinate of right edge of new picture
      }
      \sstsubsection{
         NEWY1 = REAL (Given)
      }{
         World coordinate of bottom edge of new picture
      }
      \sstsubsection{
         NEWY2 = REAL (Given)
      }{
         World coordinate of top edge of new picture
      }
      \sstsubsection{
         PICID = INTEGER (Returned)
      }{
         Picture identifier
      }
      \sstsubsection{
         STATUS = INTEGER (Given and Returned)
      }{
         The global status
      }
   }
}
\sstroutine{
   AGI\_OPEN
}{
   Open an AGI device in a non-ADAM environment
}{
   \sstdescription{
      Open an AGI device and return an identifier to the current picture.
      If there are no pictures on the device then a base picture is
      created and made current. If the size of the display window has
      changed since a previous database operation the database is cleared
      and a message sent to the user. The access mode does not affect the
      database operation, but it is used by the graphics system to
      determine if the display should be cleared the first time a zone is
      created; \texttt{'}READ\texttt{'} and \texttt{'}UPDATE\texttt{'} access do not clear the display, but
      \texttt{'}WRITE\texttt{'} access does.
   }
   \sstinvocation{
      CALL AGI\_OPEN( WKNAME, ACMODE, PICID, STATUS )
   }
   \sstarguments{
      \sstsubsection{
         WKNAME = CHARACTER $*$ ( $*$ ) (Given)
      }{
         Name of the device to open
      }
      \sstsubsection{
         ACMODE = CHARACTER $*$ ( $*$ ) (Given)
      }{
         Access mode: \texttt{'}READ\texttt{'}, \texttt{'}WRITE\texttt{'} or \texttt{'}UPDATE\texttt{'}
      }
      \sstsubsection{
         PICID = INTEGER (Returned)
      }{
         Identifier for current picture on the given device
      }
      \sstsubsection{
         STATUS = INTEGER (Given and Returned)
      }{
         The global status
      }
   }
}
\sstroutine{
   AGI\_PDEL
}{
   Delete all the pictures on the current device
}{
   \sstdescription{
      Delete all the pictures (except the base picture) on the current
      device. This routine will only execute if the current picture is
      the base picture, otherwise no action is taken. All picture
      identifiers associated with this device are released except for
      the current one.
   }
   \sstinvocation{
      CALL AGI\_PDEL( STATUS )
   }
   \sstarguments{
      \sstsubsection{
         STATUS = INTEGER (Given and returned)
      }{
         The global status.
      }
   }
}
\sstroutine{
   AGI\_PTREF
}{
   Store a reference object in a picture
}{
   \sstdescription{
      This creates a reference to a data object in the database. The
      argument can be either an HDS locator or any character string
      reference. If the string is a valid HDS locator then a reference
      is constructed to point to the relevant object, otherwise the
      string is assumed to be a reference itself and is stored as
      supplied. The picture identifier signifies which picture to put
      the reference into. If this identifier is negative then the
      current picture is used. If a reference already exists for the
      given picture then an error is returned.
   }
   \sstinvocation{
      CALL AGI\_PTREF( DATREF, PICID, STATUS )
   }
   \sstarguments{
      \sstsubsection{
         DATREF = CHARACTER $*$ ( $*$ ) (Given)
      }{
         String containing reference object
      }
      \sstsubsection{
         PICID = INTEGER (Given)
      }{
         Picture identifier
      }
      \sstsubsection{
         STATUS = INTEGER (Given and Returned)
      }{
         The global status
      }
   }
}
\sstroutine{
   AGI\_RCF
}{
   Recall first picture of specified name
}{
   \sstdescription{
      Recall the first picture on the current device that has the
      specified name and lies within the bounds of the current picture.
      The name string has leading blanks removed and is converted to
      upper case before being compared. An empty name string (just
      spaces) results in a search for a picture of any name. This
      picture becomes the current picture. If no picture fulfills the
      conditions an error is returned.
   }
   \sstinvocation{
      CALL AGI\_RCF( PNAME, PICID, STATUS )
   }
   \sstarguments{
      \sstsubsection{
         PNAME = CHARACTER $*$ ( $*$ ) (Given)
      }{
         Name of picture
      }
      \sstsubsection{
         PICID = INTEGER (Returned)
      }{
         Picture identifier
      }
      \sstsubsection{
         STATUS = INTEGER (Given and Returned)
      }{
         The global status
      }
   }
}
\sstroutine{
   AGI\_RCFP
}{
   Recall first picture embracing a position
}{
   \sstdescription{
      Recall the first picture on the current device that has the
      specified name, embraces the given position and lies within the
      bounds of the current picture. The name string has leading blanks
      removed and is converted to upper case before being compared.
      An empty name string (just spaces) results in a search for a
      picture of any name. The position has to be given in the world
      coordinates of the current picture. This picture becomes the
      current picture. If no picture fulfills the conditions an error is
      returned.
   }
   \sstinvocation{
      CALL AGI\_RCFP( PNAME, X, Y, PICID, STATUS )
   }
   \sstarguments{
      \sstsubsection{
         PNAME = CHARACTER $*$ ( $*$ ) (Given)
      }{
         Name of picture
      }
      \sstsubsection{
         X = REAL (Given)
      }{
         X position of test point
      }
      \sstsubsection{
         Y = REAL (Given)
      }{
         Y position of test point
      }
      \sstsubsection{
         PICID = INTEGER (Returned)
      }{
         Picture identifier
      }
      \sstsubsection{
         STATUS = INTEGER (Given and Returned)
      }{
         The global status
      }
   }
}
\sstroutine{
   AGI\_RCL
}{
   Recall last picture of specified name
}{
   \sstdescription{
      Recall the last picture on the current device that has the
      specified name and lies within the bounds of the current picture.
      The name string has leading blanks removed and is converted to
      upper case before being compared. An empty name string (just
      spaces) results in a search for a picture of any name. This
      picture becomes the current picture. If no picture fulfills the
      conditions an error is returned.
   }
   \sstinvocation{
      CALL AGI\_RCL( PNAME, PICID, STATUS )
   }
   \sstarguments{
      \sstsubsection{
         PNAME = CHARACTER $*$ ( $*$ ) (Given)
      }{
         Name of picture
      }
      \sstsubsection{
         PICID = INTEGER (Returned)
      }{
         Picture identifier
      }
      \sstsubsection{
         STATUS = INTEGER (Given and Returned)
      }{
         The global status
      }
   }
}
\sstroutine{
   AGI\_RCLP
}{
   Recall last picture embracing a position
}{
   \sstdescription{
      Recall the last picture on the current device that has the
      specified name, embraces the given position and lies within the
      bounds of the current picture. The name string has leading blanks
      removed and is converted to upper case before being compared.
      An empty name string (just spaces) results in a search for a
      picture of any name. The position has to be given in the world
      coordinates of the current picture. This picture becomes the
      current picture. If no picture fulfills the conditions an error is
      returned.
   }
   \sstinvocation{
      CALL AGI\_RCLP( PNAME, X, Y, PICID, STATUS )
   }
   \sstarguments{
      \sstsubsection{
         PNAME = CHARACTER $*$ ( $*$ ) (Given)
      }{
         Name of picture
      }
      \sstsubsection{
         X = REAL (Given)
      }{
         X position of test point
      }
      \sstsubsection{
         Y = REAL (Given)
      }{
         Y position of test point
      }
      \sstsubsection{
         PICID = INTEGER (Returned)
      }{
         Picture identifier
      }
      \sstsubsection{
         STATUS = INTEGER (Given and Returned)
      }{
         The global status
      }
   }
}
\sstroutine{
   AGI\_RCP
}{
   Recall preceding picture of specified name
}{
   \sstdescription{
      Recall the picture preceding the given one on the current device
      that has the specified name and lies within the bounds of the
      current picture. The search is started at the picture identified
      by the PSTART argument. The name string has leading blanks removed
      and is converted to upper case before being compared. An empty
      name string (just spaces) results in a search for a picture of any
      name. This picture becomes the current picture. If no picture
      fulfills the conditions an error is returned.
   }
   \sstinvocation{
      CALL AGI\_RCP( PNAME, PSTART, PICID, STATUS )
   }
   \sstarguments{
      \sstsubsection{
         PNAME = CHARACTER $*$ ( $*$ ) (Given)
      }{
         Name of picture
      }
      \sstsubsection{
         PSTART = INTEGER (Given)
      }{
         Identifier of picture starting the search
      }
      \sstsubsection{
         PICID = INTEGER (Returned)
      }{
         Picture identifier
      }
      \sstsubsection{
         STATUS = INTEGER (Given and Returned)
      }{
         The global status
      }
   }
}
\sstroutine{
   AGI\_RCPP
}{
   Recall preceding picture embracing a position
}{
   \sstdescription{
      Recall the picture preceding the given one on the current device
      that has the specified name, embraces the given position and lies
      within the bounds of the current picture. The search is started at
      the picture identified by the PSTART argument. The name string has
      leading blanks removed and is converted to upper case before being
      compared. An empty name string (just spaces) results in a search
      for a picture of any name. This picture becomes the current
      picture. If no picture in the workstation structure fulfills the
      conditions an error is returned.
   }
   \sstinvocation{
      CALL AGI\_RCPP( PNAME, PSTART, X, Y, PICID, STATUS )
   }
   \sstarguments{
      \sstsubsection{
         PNAME = CHARACTER $*$ ( $*$ ) (Given)
      }{
         Name of picture
      }
      \sstsubsection{
         PSTART = INTEGER (Given)
      }{
         Identifier of picture starting the search
      }
      \sstsubsection{
         X = REAL (Given)
      }{
         X position of test point
      }
      \sstsubsection{
         Y = REAL (Given)
      }{
         Y position of test point
      }
      \sstsubsection{
         PICID = INTEGER (Returned)
      }{
         Picture identifier
      }
      \sstsubsection{
         STATUS = INTEGER (Given and Returned)
      }{
         The global status
      }
   }
}
\sstroutine{
   AGI\_RCS
}{
   Recall succeeding picture of specified name
}{
   \sstdescription{
      Recall the picture succeeding the given one on the current device
      that has the specified name and lies within the bounds of the
      current picture. The search is started at the picture identified
      by the PSTART argument. The name string has leading blanks removed
      and is converted to upper case before being compared. An empty
      name string (just spaces) results in a search for a picture of any
      name. This picture becomes the current picture. If no picture
      fulfills the conditions an error is returned.
   }
   \sstinvocation{
      CALL AGI\_RCS( PNAME, PSTART, PICID, STATUS )
   }
   \sstarguments{
      \sstsubsection{
         PNAME = CHARACTER $*$ ( $*$ ) (Given)
      }{
         Name of picture
      }
      \sstsubsection{
         PSTART = INTEGER (Given)
      }{
         Identifier of picture starting the search
      }
      \sstsubsection{
         PICID = INTEGER (Returned)
      }{
         Picture identifier
      }
      \sstsubsection{
         STATUS = INTEGER (Given and Returned)
      }{
         The global status
      }
   }
}
\sstroutine{
   AGI\_RCSP
}{
   Recall succeeding picture embracing a position
}{
   \sstdescription{
      Recall the picture succeeding the given one on the current device
      that has the specified name, embraces the given position and lies
      within the bounds of the current picture. The search is started at
      the picture identified by the PSTART argument. The name string has
      leading blanks removed and is converted to upper case before being
      compared. An empty name string (just spaces) results in a search
      for a picture of any name. This picture becomes the current
      picture. If no picture in the workstation structure fulfills the
      conditions an error is returned.
   }
   \sstinvocation{
      CALL AGI\_RCSP( PNAME, PSTART, X, Y, PICID, STATUS )
   }
   \sstarguments{
      \sstsubsection{
         PNAME = CHARACTER $*$ ( $*$ ) (Given)
      }{
         Name of picture
      }
      \sstsubsection{
         PSTART = INTEGER (Given)
      }{
         Identifier of picture starting the search
      }
      \sstsubsection{
         X = REAL (Given)
      }{
         X position of test point
      }
      \sstsubsection{
         Y = REAL (Given)
      }{
         Y position of test point
      }
      \sstsubsection{
         PICID = INTEGER (Returned)
      }{
         Picture identifier
      }
      \sstsubsection{
         STATUS = INTEGER (Given and Returned)
      }{
         The global status
      }
   }
}
\sstroutine{
   AGI\_SELP
}{
   Select the given picture as the current one
}{
   \sstdescription{
      Select the given picture as the current one.
   }
   \sstinvocation{
      CALL AGI\_SELP( PICID, STATUS )
   }
   \sstarguments{
      \sstsubsection{
         PICID = INTEGER (Given)
      }{
         Picture identifier
      }
      \sstsubsection{
         STATUS = INTEGER (Given and Returned)
      }{
         The global status
      }
   }
}
\sstroutine{
   AGI\_SLAB
}{
   Store label in picture
}{
   \sstdescription{
      Store a label in the picture referenced by the identifier. If the
      picture identifier is negative then the current picture is used
      to store the label. If a label already exists for the picture
      then the old one is overwritten. If this label clashes with
      another on the same device then the existing label will be
      replaced with a blank string. For comparison purposes the label
      string has leading blanks removed and is converted to upper case
      before being processed, although it is stored as supplied. An
      empty label string will delete any label stored for that picture.
   }
   \sstinvocation{
      CALL AGI\_SLAB ( PICID, LABEL, STATUS )
   }
   \sstarguments{
      \sstsubsection{
         PICID = INTEGER (Given)
      }{
         Picture identifier
      }
      \sstsubsection{
         LABEL = CHARACTER $*$ (AGI\_\_SZLAB) (Given)
      }{
         Label string
      }
      \sstsubsection{
         STATUS = INTEGER (Given and Returned)
      }{
         The global status
      }
   }
}
\sstroutine{
   AGI\_SROOT
}{
   Select the root picture for searching
}{
   \sstdescription{
      The root picture is selected for searching operations. The root
      picture contains all other pictures (including the base picture)
      and can be used to recall any picture whether it lies within the
      current picture or not. This is used to override the usual
      restriction that a recalled picture must lie within the bounds of
      the current picture. The root picture is automatically deselected
      after a call to any of the recall routines AGI\_RC $*$ . Recall is the
      only operation allowed with the root picture, any other operation
      called while the root picture is selected will use the current
      picture.
   }
   \sstinvocation{
      CALL AGI\_SROOT( STATUS )
   }
   \sstarguments{
      \sstsubsection{
         STATUS = INTEGER (Given and Returned)
      }{
         The global status
      }
   }
}
\sstroutine{
   AGI\_TCOPY
}{
   Copy a transformation structure to the database
}{
   \sstdescription{
      The transformation pointed to by the HDS locator is stored in
      the database. The picture identifier signifies which picture is
      to receive the transformation structure. If this identifier is
      negative then the current picture will be used. If a transformation
      already exists for this picture then an error will be returned.
      The supplied transformation should convert data coordinates into
      the world coordinates of the database picture, as if it had been
      created with a call to \htmlref{AGI\_TNEW}{AGI_TNEW}.
   }
   \sstinvocation{
      CALL AGI\_TCOPY( TRNLOC, PICID, STATUS )
   }
   \sstarguments{
      \sstsubsection{
         TRNLOC = CHARACTER $*$ (DAT\_\_SZLOC) (Given)
      }{
         Locator to the transformation structure
      }
      \sstsubsection{
         PICID = INTEGER (Given)
      }{
         Picture identifier
      }
      \sstsubsection{
         STATUS = INTEGER (Given and Returned)
      }{
         The global status
      }
   }
}
\sstroutine{
   AGI\_TDDTW
}{
   Transform double precision data to world coordinates
}{
   \sstdescription{
      Transform a set of double precision data coordinates into world
      coordinates using a transformation stored in the database. The
      picture identifier signifies which picture contains the required
      transformation structure. If this identifier is negative then
      the current picture is used. If no transformation structure
      is found then the identity transformation is used, i.e. the
      world coordinates equal the data coordinates.
   }
   \sstinvocation{
      CALL AGI\_TDDTW( PICID, NXY, DX, DY, WX, WY, STATUS )
   }
   \sstarguments{
      \sstsubsection{
         PICID = INTEGER (Given)
      }{
         Picture identifier
      }
      \sstsubsection{
         NXY = INTEGER (Given)
      }{
         Number of data points to transform
      }
      \sstsubsection{
         DX = DBLE(NXY) (Given)
      }{
         Array of x data coordinates
      }
      \sstsubsection{
         DY = DBLE(NXY) (Given)
      }{
         Array of y data coordinates
      }
      \sstsubsection{
         WX = DBLE(NXY) (Returned)
      }{
         Array of x world coordinates
      }
      \sstsubsection{
         WY = DBLE(NXY) (Returned)
      }{
         Array of y world coordinates
      }
      \sstsubsection{
         STATUS = INTEGER (Given and Returned)
      }{
         The global status
      }
   }
}
\sstroutine{
   AGI\_TDTOW
}{
   Transform data to world coordinates
}{
   \sstdescription{
      Transform a set of data coordinates into world coordinates
      using a transformation stored in the database. The picture
      identifier signifies which picture contains the required
      transformation structure. If this identifier is negative then
      the current picture is used. If no transformation structure
      is found then the identity transformation is used, i.e. the
      world coordinates equal the data coordinates.
   }
   \sstinvocation{
      CALL AGI\_TDTOW( PICID, NXY, DX, DY, WX, WY, STATUS )
   }
   \sstarguments{
      \sstsubsection{
         PICID = INTEGER (Given)
      }{
         Picture identifier
      }
      \sstsubsection{
         NXY = INTEGER (Given)
      }{
         Number of data points to transform
      }
      \sstsubsection{
         DX = REAL(NXY) (Given)
      }{
         Array of x data coordinates
      }
      \sstsubsection{
         DY = REAL(NXY) (Given)
      }{
         Array of y data coordinates
      }
      \sstsubsection{
         WX = REAL(NXY) (Returned)
      }{
         Array of x world coordinates
      }
      \sstsubsection{
         WY = REAL(NXY) (Returned)
      }{
         Array of y world coordinates
      }
      \sstsubsection{
         STATUS = INTEGER (Given and Returned)
      }{
         The global status
      }
   }
}
\sstroutine{
   AGI\_TNEW
}{
   Store a transformation in the database
}{
   \sstdescription{
      The transformation defined by the pseudo-code FORTRAN statements
      DTOW and WTOD is stored in the database. The picture identifier
      signifies which picture is to receive the transformation
      structure. If this identifier is negative then the current picture
      will be used. If a transformation already exists for this picture
      then an error will be returned. The number of world variables NCW
      has to be equal to 2 otherwise an error will be returned. The
      number of data variables NCD also has to equal 2 in the present
      implementation.
   }
   \sstinvocation{
      CALL AGI\_TNEW ( NCD, NCW, DTOW, WTOD, PICID, STATUS )
   }
   \sstarguments{
      \sstsubsection{
         NCD = INTEGER (Given)
      }{
         Number of data variables
      }
      \sstsubsection{
         NCW = INTEGER (Given)
      }{
         Number of world variables
      }
      \sstsubsection{
         DTOW = CHARACTER $*$ ( $*$ )(NCW) (Given)
      }{
         Array of forward transformation functions
      }
      \sstsubsection{
         WTOD = CHARACTER $*$ ( $*$ )(NCD) (Given)
      }{
         Array of inverse transformation functions
      }
      \sstsubsection{
         PICID = INTEGER (Given)
      }{
         Picture identifier
      }
      \sstsubsection{
         STATUS = INTEGER (Given and Returned)
      }{
         The global status
      }
   }
}

\sstroutine{
   AGI\_TRUNC
}{
   Truncate the AGI database file by removing unused space
}{
   \sstdescription{
      This routine attempts to reduce the size of the AGI database
      file by removing any unused space from the end. The database
      must be closed before calling this routine (an error is reported
      otherwise).
   }
   \sstinvocation{
      CALL AGI\_TRUNC( STATUS )
   }
   \sstarguments{
      \sstsubsection{
         STATUS = INTEGER (Given and Returned)
      }{
         The global status
      }
   }
}

\sstroutine{
   AGI\_TWTDD
}{
   Transform double precision world to data coordinates
}{
   \sstdescription{
      Transform a set of double precision world coordinates into data
      coordinates using a transformation stored in the database. The
      picture identifier signifies which picture contains the required
      transformation structure. If this identifier is negative then
      the current picture is used. If no transformation structure
      is found then the identity transformation is used, i.e. the
      data coordinates equal the world coordinates.
   }
   \sstinvocation{
      CALL AGI\_TWTDD( PICID, NXY, WX, WY, DX, DY, STATUS )
   }
   \sstarguments{
      \sstsubsection{
         PICID = INTEGER (Given)
      }{
         Picture identifier
      }
      \sstsubsection{
         NXY = INTEGER (Given)
      }{
         Number of data points to transform
      }
      \sstsubsection{
         WX = DBLE(NXY) (Given)
      }{
         Array of x world coordinates
      }
      \sstsubsection{
         WY = DBLE(NXY) (Given)
      }{
         Array of y world coordinates
      }
      \sstsubsection{
         DX = DBLE(NXY) (Returned)
      }{
         Array of x data coordinates
      }
      \sstsubsection{
         DY = DBLE(NXY) (Returned)
      }{
         Array of y data coordinates
      }
      \sstsubsection{
         STATUS = INTEGER (Given and Returned)
      }{
         The global status
      }
   }
}
\sstroutine{
   AGI\_TWTOD
}{
   Transform world to data coordinates
}{
   \sstdescription{
      Transform a set of world coordinates into data coordinates
      using a transformation stored in the database. The picture
      identifier signifies which picture contains the required
      transformation structure. If this identifier is negative then
      the current picture is used. If no transformation structure
      is found then the identity transformation is used, i.e. the
      data coordinates equal the world coordinates.
   }
   \sstinvocation{
      CALL AGI\_TWTOD( PICID, NXY, WX, WY, DX, DY, STATUS )
   }
   \sstarguments{
      \sstsubsection{
         PICID = INTEGER (Given)
      }{
         Picture identifier
      }
      \sstsubsection{
         NXY = INTEGER (Given)
      }{
         Number of data points to transform
      }
      \sstsubsection{
         WX = REAL(NXY) (Given)
      }{
         Array of x world coordinates
      }
      \sstsubsection{
         WY = REAL(NXY) (Given)
      }{
         Array of y world coordinates
      }
      \sstsubsection{
         DX = REAL(NXY) (Returned)
      }{
         Array of x data coordinates
      }
      \sstsubsection{
         DY = REAL(NXY) (Returned)
      }{
         Array of y data coordinates
      }
      \sstsubsection{
         STATUS = INTEGER (Given and Returned)
      }{
         The global status
      }
   }
}
\sstroutine{
   AGP\_ACTIV
}{
   Initialise PGPLOT
}{
   \sstdescription{
      Initialise PGPLOT. This has to be called before any other AGP or
      PGPLOT routines. An error is returned if this or any other
      graphics interface, other than AGS\_, is active.
   }
   \sstinvocation{
      CALL AGP\_ACTIV( STATUS )
   }
   \sstarguments{
      \sstsubsection{
         STATUS = INTEGER (Given and Returned)
      }{
         The global status
      }
   }
}
\sstroutine{
   AGP\_ASSOC
}{
   Associate a device with AGI and PGPLOT
}{
   \sstdescription{
      This is a wrap-up routine to associate a device with the AGI
      database via the ADAM parameter system and open PGPLOT on it.
      A PGPLOT viewport corresponding to the current picture in the
      database is created. This routine calls
      \htmlref{AGI\_ASSOC}{AGI_ASSOC}, \htmlref{AGI\_BEGIN}{AGI_BEGIN}
      \htmlref{AGP\_ACTIV}{AGP_ACTIV} and \htmlref{AGP\_NVIEW}{AGP_NVIEW}.
      Also if the name string is not blank then \htmlref{AGI\_RCL}{AGI_RCL}
      is called to recall the last picture of that name.  This routine
      should be matched by a closing call to \htmlref{AGP\_DEASS}{AGP_DEASS}.
   }
   \sstinvocation{
      CALL AGP\_ASSOC( PARAM, ACMODE, PNAME, BORDER, PICID, STATUS )
   }
   \sstarguments{
      \sstsubsection{
         PARAM = CHARACTER $*$ ( $*$ ) (Given)
      }{
         The name of the ADAM parameter for accessing device names
      }
      \sstsubsection{
         ACMODE = CHARACTER $*$ ( $*$ ) (Given)
      }{
         Access mode for pictures. \texttt{'}READ\texttt{'}, \texttt{'}WRITE\texttt{'} or \texttt{'}UPDATE\texttt{'}.
      }
      \sstsubsection{
         PNAME = CHARACTER $*$ ( $*$ ) (Given)
      }{
         Recall last picture of this name if not blank.
      }
      \sstsubsection{
         BORDER = LOGICAL (Given)
      }{
         Flag to indicate if a border is to be left around the viewport.
      }
      \sstsubsection{
         PICID = INTEGER (Returned)
      }{
         Picture identifier for current picture on given device.
      }
      \sstsubsection{
         STATUS = INTEGER (Given and Returned)
      }{
         The global status.
      }
   }
}
\sstroutine{
   AGP\_DEACT
}{
   Close down PGPLOT
}{
   \sstdescription{
      Close down PGPLOT whatever the value of status. This should be
      called after all AGP and PGPLOT routines.
   }
   \sstinvocation{
      CALL AGP\_DEACT( STATUS )
   }
   \sstarguments{
      \sstsubsection{
         STATUS = INTEGER (Given and Returned)
      }{
         The global status
      }
   }
}
\sstroutine{
   AGP\_DEASS
}{
   Deassociate a device from AGI and PGPLOT
}{
   \sstdescription{
      This is a wrap-up routine to deassociate a device from the AGI
      database and to close down PGPLOT. The picture current when
      \htmlref{AGP\_ASSOC}{AGP_ASSOC} was called is reinstated. This
      routine calls \htmlref{AGP\_DEACT}{AGP_DEACT},
      \htmlref{AGI\_END}{AGI_END} and either \htmlref{AGI\_CANCL}{AGI_CANCL}
      or \htmlref{AGI\_ANNUL}{AGI_ANNUL}. This routine is
      executed regardless of the given value of status.
   }
   \sstinvocation{
      CALL AGP\_DEASS( PARAM, PARCAN, STATUS )
   }
   \sstarguments{
      \sstsubsection{
         PARAM = CHARACTER $*$ ( $*$ ) (Given)
      }{
         The name of the ADAM parameter associated with the device.
      }
      \sstsubsection{
         PARCAN = LOGICAL (Given)
      }{
         If true the parameter given by PARAM is cancelled, otherwise
         it is annulled.
      }
      \sstsubsection{
         STATUS = INTEGER (Given and Returned)
      }{
         The global status.
      }
   }
}
\sstroutine{
   AGP\_NVIEW
}{
   Create a new PGPLOT viewport from the current picture
}{
   \sstdescription{
      Create a new PGPLOT viewport from the current picture. The
      viewport will be created with the coordinate system of the
      current picture. The border flag allocates space around the plot
      for annotation if required. If true the viewport is made
      approximately 10\% smaller than the picture to allow space for
      annotation. If false the viewport matches the picture exactly.
      If the device associated with the current picture is not open then
      this routine will open it, and additionally the viewport will be
      cleared if the access mode is \texttt{'}WRITE\texttt{'} (in
      \htmlref{AGI\_ASSOC}{AGI_ASSOC} or \htmlref{AGI\_OPEN}{AGI_OPEN}).
      If the SGS interface is already active then the first call to
      this routine will open PGPLOT in the current SGS zone. In this
      case the viewport normalised device coordinates will not match the
      coordinate system of the current picture, but will have the
      default range of 0 to 1.
   }
   \sstinvocation{
      CALL AGP\_NVIEW ( BORDER, STATUS )
   }
   \sstarguments{
      \sstsubsection{
         BORDER = LOGICAL (Given)
      }{
         Flag to indicate if a border is to be left around the viewport
      }
      \sstsubsection{
         STATUS = INTEGER (Given and Returned)
      }{
         The global status
      }
   }
}
\sstroutine{
   AGP\_SVIEW
}{
   Save the current PGPLOT viewport in the database
}{
   \sstdescription{
      Save the current PGPLOT viewport as a picture in the database.
      The new picture must be equal in size or smaller than the current
      picture in the database. The name of the picture and a comment are
      used to identify the picture in the database. The name string has
      leading blanks removed and is converted to upper case. If the
      picture was successfully created then a valid picture identifier
      is returned and the new picture becomes the current picture.
   }
   \sstinvocation{
      CALL AGP\_SVIEW( PICNAM, COMENT, PICID, STATUS )
   }
   \sstarguments{
      \sstsubsection{
         PICNAM = CHARACTER $*$ ( $*$ ) (Given)
      }{
         Name of picture
      }
      \sstsubsection{
         COMENT = CHARACTER $*$ ( $*$ ) (Given)
      }{
         Description of picture
      }
      \sstsubsection{
         PICID = INTEGER (Returned)
      }{
         Picture identifier
      }
      \sstsubsection{
         STATUS = INTEGER (Given and Returned)
      }{
         The global status
      }
   }
}
\sstroutine{
   AGS\_ACTIV
}{
   Initialise SGS
}{
   \sstdescription{
      Initialise SGS. This has to be called before any other AGS or
      SGS routines. An error is returned if this or any other graphics
      interface is already active.
   }
   \sstinvocation{
      CALL AGS\_ACTIV( STATUS )
   }
   \sstarguments{
      \sstsubsection{
         STATUS = INTEGER (Given and Returned)
      }{
         The global status
      }
   }
}
\sstroutine{
   AGS\_ASSOC
}{
   Associate a device with AGI and SGS
}{
   \sstdescription{
      This is a wrap-up routine to associate a device with the AGI
      database via the ADAM parameter system and open SGS on it. An
      SGS zone corresponding to the current picture in the database
      is created. This routine calls \htmlref{AGI\_ASSOC}{AGI_ASSOC} ,
      \htmlref{AGI\_BEGIN}{AGI_BEGIN}, \htmlref{AGS\_ACTIV}{AGS_ACTIV}
      and \htmlref{AGS\_NZONE}{AGS_NZONE}. Also if the name string is
      not blank then \htmlref{AGI\_RCL}{AGI_RCL}
      is called to recall the last picture of that name. This routine
      should be matched by a closing call to \htmlref{AGS\_DEASS}{AGS_DEASS}.
   }
   \sstinvocation{
      CALL AGS\_ASSOC( PARAM, ACMODE, PNAME, PICID, NEWZON, STATUS )
   }
   \sstarguments{
      \sstsubsection{
         PARAM = CHARACTER $*$ ( $*$ ) (Given)
      }{
         The name of the ADAM parameter for accessing device names
      }
      \sstsubsection{
         ACMODE = CHARACTER $*$ ( $*$ ) (Given)
      }{
         Access mode for pictures. \texttt{'}READ\texttt{'}, \texttt{'}WRITE\texttt{'} or \texttt{'}UPDATE\texttt{'}.
      }
      \sstsubsection{
         PNAME = CHARACTER $*$ ( $*$ ) (Given)
      }{
         Recall last picture of this name if not blank.
      }
      \sstsubsection{
         PICID = INTEGER (Returned)
      }{
         Picture identifier for current picture on given device.
      }
      \sstsubsection{
         NEWZON = INTEGER (Returned)
      }{
         The new SGS zone that matches the current picture.
      }
      \sstsubsection{
         STATUS = INTEGER (Given and Returned)
      }{
         The global status.
      }
   }
}
\sstroutine{
   AGS\_DEACT
}{
   Close down SGS
}{
   \sstdescription{
      Close down SGS whatever the value of status. This should be
      called after all AGS and SGS routines.
   }
   \sstinvocation{
      CALL AGS\_DEACT( STATUS )
   }
   \sstarguments{
      \sstsubsection{
         STATUS = INTEGER (Given and Returned)
      }{
         The global status
      }
   }
}
\sstroutine{
   AGS\_DEASS
}{
   Deassociate a device from AGI and SGS
}{
   \sstdescription{
      This is a wrap-up routine to deassociate a device from the AGI
      database and to close down SGS. The picture current when
      \htmlref{AGS\_ASSOC}{AGS_ASSOC} was called is reinstated. This
      routine calls \htmlref{AGS\_DEACT}{AGS_DEACT},
      \htmlref{AGI\_END}{AGI_END} and either \htmlref{AGI\_CANCL}{AGI_CANCL}
      or \htmlref{AGI\_ANNUL}{AGI_ANNUL}. This routine is executed regardless
      of the given value of status.
   }
   \sstinvocation{
      CALL AGS\_DEASS( PARAM, PARCAN, STATUS )
   }
   \sstarguments{
      \sstsubsection{
         PARAM = CHARACTER $*$ ( $*$ ) (Given)
      }{
         The name of the ADAM parameter associated with the device.
      }
      \sstsubsection{
         PARCAN = LOGICAL (Given)
      }{
         If true the parameter given by PARAM is cancelled, otherwise
         it is annulled.
      }
      \sstsubsection{
         STATUS = INTEGER (Given and Returned)
      }{
         The global status.
      }
   }
}
\sstroutine{
   AGS\_NZONE
}{
   Create a new SGS zone from the current picture
}{
   \sstdescription{
      Create a new SGS zone from the current picture. The zone will be
      created with the coordinate system of the current picture.
      If the device associated with the current picture is not already
      open then this routine will open it, and additionally the zone
      will be cleared if the access mode is \texttt{'}WRITE\texttt{'} (in
      \htmlref{AGI\_ASSOC}{AGI_ASSOC} or \htmlref{AGI\_OPEN}{AGI_OPEN}).
   }
   \sstinvocation{
      CALL AGS\_NZONE ( NEWZON, STATUS )
   }
   \sstarguments{
      \sstsubsection{
         NEWZON = INTEGER (Returned)
      }{
         SGS zone identifier of new zone
      }
      \sstsubsection{
         STATUS = INTEGER (Given and Returned)
      }{
         The global status
      }
   }
}
\sstroutine{
   AGS\_SZONE
}{
   Save the current SGS zone in the database
}{
   \sstdescription{
      Save the current SGS zone as a picture in the database. The new
      picture must be equal in size or smaller than the current picture
      in the database. The name of the picture and a comment are used to
      identify the picture in the database. The name string has leading
      blanks removed and is converted to upper case. If the picture was
      successfully created then a valid picture identifier is returned
      and the new picture becomes the current picture.
   }
   \sstinvocation{
      CALL AGS\_SZONE( PNAME, COMENT, PICID, STATUS )
   }
   \sstarguments{
      \sstsubsection{
         PNAME = CHARACTER $*$ ( $*$ ) (Given)
      }{
         Name of picture
      }
      \sstsubsection{
         COMENT = CHARACTER $*$ ( $*$ ) (Given)
      }{
         Description of picture
      }
      \sstsubsection{
         PICID = INTEGER (Returned)
      }{
         Picture identifier
      }
      \sstsubsection{
         STATUS = INTEGER (Given and Returned)
      }{
         The global status
      }
   }
}
\end{small}

% ? End of main text
\end{document}
