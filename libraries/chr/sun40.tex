\documentclass[twoside,11pt,nolof]{starlink}

% ? Specify used packages
% ? End of specify used packages



% -----------------------------------------------------------------------------
% ? Document identification
% Fixed part
\stardoccategory    {Starlink User Note}
\stardocinitials    {SUN}
\stardocsource      {sun\stardocnumber}
\stardoccopyright   {Copyright \copyright\ 2018 East Asian Observatory}

% Variable part - replace [xxx] as appropriate.
\stardocnumber      {40.6}
\stardocauthors     {A C Charles\\P C T Rees\\A J Chipperfield\\ T Jenness\\ D Berry }
\stardocdate        {16 May 2018}
\stardoctitle       {CHR \\ [\latex{1ex}]
                                Character Handling Routines}
\stardocversion     {3.0}
\stardocmanual      {Programmer's Manual}
\stardocabstract  {
This document describes the Character Handling Routine library, CHR,
and its use.
The CHR library augments the limited character handling facilities
provided by the Fortran 77 standard.
It offers a range of character handling facilities: from formatting
Fortran data types into text strings and the reverse, to higher level
functions such as wild card matching, string sorting, paragraph
reformatting and justification.
The library may be used simply for building text strings for interactive
applications or as a basis for more complex text processing applications.
}
% ? End of document identification
% -----------------------------------------------------------------------------
% -----------------------------------------------------------------------------
% ? Document specific \providecommand or \newenvironment commands.

\providecommand{\qt}[1]{'#1'}

% ? End of document specific commands
% -----------------------------------------------------------------------------
%  Title Page.
%  ===========
\begin{document}
\scfrontmatter

\section {Introduction}

The Fortran 77 standard provides a data type for the storage of
character strings (the type \texttt{CHARACTER}), an operator specific to
character data (the \texttt{//} operator for string concatenation), and
eight intrinsic functions specifically for handling \texttt{CHARACTER} typed
variables (\texttt{CHAR}, \texttt{ICHAR}, \texttt{INDEX}, \texttt{LEN}, \texttt{LGE},
\texttt{LGT}, \texttt{LLE}, \texttt{LLT}).
Facilities to write and read Fortran data types to and from
character strings (using internal files and the \texttt{WRITE} and \texttt{READ}
statements respectively) are also provided by the Fortran 77 standard.
Although these features of the Fortran language are of considerable utility
when handling character variables in Fortran programs, they constitute only the
basic tools for the more extensive processing of textual data sometimes
required in applications.

The CHR library augments the limited character handling facilities
provided by the Fortran 77 standard.
It offers a range of character handling facilities: from formatting
Fortran data types into text strings and the reverse, to higher level
functions such as wild card matching, string sorting, paragraph
reformatting and justification.
The library may be used simply for building text strings for interactive
applications or as a basis for more complex text processing applications.

The functions performed by the CHR library may be categorised as follows:

\begin{itemize}
\item change case
\item compare strings
\item decode Fortran data types
\item edit strings
\item encode Fortran data types
\item enquire
\item facilitate portability
\item search strings
\end{itemize}

A classified list of of these routines is given in Appendix \ref{class_sect}.

Equivalent functions written entirely in C are available for a subset of
the CHR routines (see Appendix \ref{c_funcs}). These are provided mainly
for use when porting Fortran application code to C.

\section {Error Handling \label{err_sect} }

None of the CHR routines report error messages; \emph{i.e.}\ they do not use
\xref{the Starlink Error Reporting System}{sun104}{}
\latex{, ERR (see SUN/104)}.
However, some routines do use the
\xref{inherited status conventions}{sun104}{inherited_status}
\latex{(described in SUN/104) }to indicate success or failure.
Those CHR routines which do not have a status argument handle any errors
internally and have a specified behaviour on error.
Those CHR routines which do have a status argument fall into two categories:
those which obey the full inherited status conventions and return without
action if given a status value other than SAI\_\_OK, and those which ignore
the given status and just return a status value on exit.
The following routines obey the full inherited status conventions:

\begin {itemize}
\item CHR\_BTOI
\item CHR\_CTOD
\item CHR\_CTOI
\item CHR\_CTOL
\item CHR\_CTOR
\item CHR\_FIWE
\item CHR\_FIWS
\item CHR\_HTOI
\item CHR\_OTOI
\end {itemize}

The routines CHR\_COPY, CHR\_DCWRD and CHR\_TRCHR all just return a status
set to a value not equal to SAI\_\_OK on error.

The symbolic names of the error values used by CHR are given in Appendix
\ref{incl_sect}.


\section {Compiling and Linking \label{compile_sect}}
The two include files available for use with the Character Handling Routines
are named \textbf{sae\_par} and \textbf{chr\_err} on UNIX
machines, and reside in the directory \textbf{/star/include}.

When including these files within Fortran code, the Starlink convention
is that the name in upper case with no path or extension is specified when
including these files within Fortran code,
\emph{e.g.}
\begin{quote}
\begin{small}
\begin{terminalv}
*  Global Constants:
      INCLUDE 'SAE_PAR'
      INCLUDE 'CHR_ERR'
\end{terminalv}
\end{small}
\end{quote}
Assuming that the software has been installed in the standard way and
\textbf{/star/bin} has been added to the environment variable
\textbf{PATH}, soft links with these upper-case names pointing to the required
file are set up in the user's working directory by the the commands:
\begin {quote}
\begin{small}
\begin{terminalv}
% star_dev
% chr_dev
\end{terminalv}
\end{small}
\end {quote}
Then to compile and link a non-ADAM program, the command line would be,
\emph{e.g.}
\begin {quote}
\begin{terminalv}
% f77 -o program program.f -L/star/lib `chr_link`
\end{terminalv}
\end {quote}
The CHR library is included automatically when programs are linked using the
ADAM application linking commands, \textbf{alink} \emph{etc}.

If it is necessary to link explicitly with the ADAM version of CHR
(\emph{e.g.}\ to produce a shareable library), the
script \textbf{chr\_link\_adam} is available in \textbf{/star/bin}.
The link command might be:
\begin {quote}
\begin {small}
\begin{terminalv}
% ld -shared -o libmypkg.so.1.0 -lmypkg `chr_link_adam`
\end{terminalv}
\end {small}
\end {quote}

\section{Efficiency Considerations}
Several routines provided by CHR have implications for the efficiency of
applications which make heavy use of them.
These routines when used judiciously present no efficiency problem, but
when they are used indiscriminantly they can have a marked effect on
execution times.

\begin {description}
\item [CHR\_CTOx] These routines make use of the Fortran internal \texttt{READ}
statements which can have an impact upon execution times when used heavily.
Within an application it is only absolutely necessary to perform this type
conversion once, when it is needed.
If both representations are needed within an application, store both.
\item [CHR\_LEN] Checking the used length of a given \texttt{CHARACTER}
argument within every subroutine can have a significant impact upon execution
times and should be avoided.
For character strings which are not modified, CHR\_LEN need only be called
once.
The used length of the string may then be passed as an additional subroutine
argument, \emph{e.g.}
\begin {small}
\begin{terminalv}
      STRLEN = CHR_LEN( STRING )
      CALL SUBN( STRING, STRLEN, STATUS )
\end{terminalv}
\end {small}
or the substring that represents the filled part of the string may be passed
to the subroutine, \emph{e.g.}
\begin {small}
\begin{terminalv}
      STRLEN = CHR_LEN( STRING )
      CALL SUBN( STRING( 1 : STRLEN ), STATUS )
\end{terminalv}
\end {small}
\item [CHR\_xTOC] These routines make use of the Fortran internal \texttt{WRITE}
statements which can have an impact upon execution times when used heavily.
Within an application it is only absolutely necessary to perform this type
conversion once, when it is needed.
If both representations are needed within an application, store both.
\end {description}


\appendix
\section {Include Files \label{incl_sect}}
There are two include files used by the Character Handling Routines
to define global constants during compilation.
These files have the logical names SAE\_PAR and CHR\_ERR.
The contents of each of these include files are given below.

\begin {description}
\item [CHR\_ERR] Defines the Character Handling Routine errors.
\begin {description}
\item \textbf{CHR\_\_EOSNT} -- End of sentence.
\item \textbf{CHR\_\_WNOTF} -- Word not found.
\end {description}
\item [SAE\_PAR] Defines the global constants SAI\_\_OK and SAI\_\_ERROR.
\begin{description}
\item \textbf{SAI\_\_ERROR} -- Error encountered.
\item \textbf{SAI\_\_OK} -- No error.
\item \textbf{SAI\_\_WARN} -- Warning.
\end {description}
\end {description}

\newpage
\section {Classified List of Routines \label{class_sect}}

The classifications are in alphabetical order as follows:

\begin{description}
\item[Change case] --
    Change the case of a character string.
\item[Compare strings] --
    Compare two character strings.
\item[Decode Fortran data types] --
    Convert a character string into a Fortran data type and return its
    value.
\item[Editing strings] --
    Edit character strings by replacing, adding or removing defined
    substrings.
\item[Encode Fortran data types] --
    Convert a Fortran data type into a character string representation of
    its value.
\item[Enquire] --
    Return information about a character string.
\item[Facilitate portability] --
    Tools for assisting portability, especially to or from non-ASCII
    environments.
\item[Search strings] --
    Search a character string.
\end{description}


\subsection {Change case}

\begin {description}
\item [CHR\_LCASE] -- Convert a string to lower case.
\item [CHR\_LOWER] -- Return the lower case equivalent of a character.
\item [CHR\_UCASE] -- Convert a string to upper case.
\item [CHR\_UPPER] -- Return the upper-case equivalent of a character.
\end {description}


\subsection {Compare strings}

\begin {description}
\item [CHR\_ABBRV] -- Return whether two strings are equal apart from case,
permitting  abbreviation.
\item [CHR\_SCOMP] -- Compare two character strings using the ASCII character
set.
\item [CHR\_SIMLR] -- Return whether two strings are equal, apart from case.
\item [CHR\_WILD] -- Return whether a string matches a wild-card pattern.
\end {description}


\subsection {Decode Fortran data types}

\begin {description}
\item [CHR\_BTOI] -- Read an INTEGER value from a binary string.
\item [CHR\_CTOC] -- Write a CHARACTER string into another string.
\item [CHR\_CTOD] -- Read a DOUBLE PRECISION value from a string.
\item [CHR\_CTOI] -- Read an INTEGER value from a string.
\item [CHR\_CTOL] -- Read a LOGICAL value from a string.
\item [CHR\_CTOR] -- Read a REAL value from a string.
\item [CHR\_HTOI] -- Read an INTEGER value from a hexadecimal string.
\item [CHR\_OTOI] -- Read an integer from an octal string.
\end {description}


\subsection {Edit strings}

\begin {description}
\item [CHR\_APPND] -- Copy one string into another, ignoring trailing blanks.
\item [CHR\_CLEAN] -- Remove all unprintable characters from a string.
\item [CHR\_COPY] -- Copy one string into another, checking for truncation.
\item [CHR\_DCWRD] -- Split a string into its component words.
\item [CHR\_FILL] -- Fill a string with a given character.
\item [CHR\_LDBLK] -- Remove any leading blanks from a string.
\item [CHR\_LINBR] -- Break a line of text into a sequence of shorter lines.
\item [CHR\_PFORM] -- Reformat a paragraph to a new width.
\item [CHR\_PREFX] -- Prefix a string with a substring.
\item [CHR\_RJUST] -- Right-justify a string.
\item [CHR\_RMBLK] -- Remove all blanks from a string.
\item [CHR\_RMCHR] -- Remove all specified characters from a string.
\item [CHR\_SORT] -- Sort an array of character strings into alphabetical order.
\item [CHR\_SWAP] -- Swap two single-character variables.
\item [CHR\_TERM] -- Terminate a string by padding out with blanks.
\item [CHR\_TRCHR] -- Translate the specified characters in a string.
\item [CHR\_TRUNC] -- Truncate a string at a given delimiter.
\end {description}


\subsection {Encode Fortran data types}

\begin{description}
\item [CHR\_DTOAN] -- Write a DOUBLE PRECISION value into a string as
hr/deg:min:sec.
\item [CHR\_DTOC] -- Write a DOUBLE PRECISION value into a string.
\item [CHR\_ITOB] -- Write an INTEGER value as a binary string.
\item [CHR\_ITOC] -- Write an INTEGER value as a decimal string.
\item [CHR\_ITOH] -- Write an INTEGER value as a hexadecimal string.
\item [CHR\_ITOO] -- Write an INTEGER value as an octal string.
\item [CHR\_LTOC] -- Write a LOGICAL value into a string.
\item [CHR\_PUTC] -- Put a CHARACTER string into another at a given position.
\item [CHR\_PUTD] -- Put a DOUBLE PRECISION value into a string at a given position.
\item [CHR\_PUTI] -- Put an INTEGER value into a string at a given position.
\item [CHR\_PUTL] -- Put a LOGICAL value into a string at a given position.
\item [CHR\_PUTR] -- Put a REAL value into a string at a given position.
\item [CHR\_RTOAN] -- Write a REAL value into a string as hr/deg:min:sec.
\item [CHR\_RTOC] -- Write a REAL value into a string.
\end {description}


\subsection {Enquire}

\begin{description}
\item [CHR\_NTH] -- Return the two-character abbreviation for a specified
integer.
\item [CHR\_INSET] -- Return whether a string is a member of a given set.
\item [CHR\_ISALF] -- Return whether a character is alphabetic.
\item [CHR\_ISALM] -- Return whether a character is alphanumeric.
\item [CHR\_ISDIG] -- Return whether a character is a digit.
\item [CHR\_ISNAM] -- Return whether a string is a valid name.
\item [CHR\_LEN] -- Return the length of a string, ignoring trailing blanks.
\end{description}


\subsection{Facilitate Portability}

\begin{description}
\item [CHR\_ACHR] -- Return the character for a given ASCII value.
\item [CHR\_ATOK] -- Return the character for a given ASCII character token.
\item [CHR\_ATOM] -- Translate a string from ASCII to the machine's character
set.
\item [CHR\_ETOM] -- Translate a string from EBCDIC to machine's character set.
\item [CHR\_IACHR] -- Return the ASCII value for a given character.
\item [CHR\_MTOA] -- Translate a string from machine's character set to ASCII.
\item [CHR\_MTOE] -- Translate a string from machine's character set to EBCDIC.
\end{description}


\subsection{Search strings}

\begin {description}
\item [CHR\_DELIM] -- Locate a substring using a given delimiter character.
\item [CHR\_FANDL] -- Find the first and last non-blank characters in a string.
\item [CHR\_FIND] -- Find the next occurrence of a given substring within a string.
\item [CHR\_FIWE] -- Find the next end-of-word within a string.
\item [CHR\_FIWS] -- Find the start of the next word within a string.
\item [CHR\_FPARX] -- Find a parenthesised expression in a character string.
\item [CHR\_LASTO] -- Find the last occurrence of character in a string.
\item [CHR\_SKCHR] -- Skip over all specified characters in a string.
\item [CHR\_TOCHR] -- Skip to the next specified character in a string.
\end {description}


\newpage
\section{Routine Descriptions}

\begin{small}
\sstroutine{
   CHR\_ABBRV
}{
   Return whether two strings are equal apart from case,
   permitting abbreviations
}{
   \sstdescription{
      Returns a logical result indicating whether two strings
      are the same, apart from case. In assessing this, the first
      string is allowed to be an abbreviation of the second string,
      as long as it contains a specified minimum number of characters.
   }
   \sstinvocation{
      RESULT = CHR\_ABBRV( STR1, STR2, NCHAR )
   }
   \sstarguments{
      \sstsubsection{
         STR1 = CHARACTER $*$ ( $*$ ) (Given)
      }{
         The first string, which may be an abbreviation.
      }
      \sstsubsection{
         STR2 = CHARACTER $*$ ( $*$ ) (Given)
      }{
         The second string.
      }
      \sstsubsection{
         NCHAR = INTEGER (Given)
      }{
         The minimum number of characters to which the first string may
         be abbreviated (a smaller number will be accepted if there are
         actually fewer than NCHAR characters in STR2).
      }
   }
   \sstreturnedvalue{
      \sstsubsection{
         CHR\_ABBRV = LOGICAL
      }{
         Whether the two strings match after allowing for case and
         abbreviation of the first string to no less than NCHAR
         characters.
      }
   }
}
\sstroutine{
   CHR\_ACHR
}{
   Return the character for a given ASCII value
}{
   \sstdescription{
      The given ASCII value is converted to a single returned character
      in the machine's character set. If no such character
      exists within the machine's character set, the character code 0
      (the ASCII NUL character) is returned.
   }
   \sstinvocation{
      RESULT = CHR\_ACHR( ASCII )
   }
   \sstarguments{
      \sstsubsection{
         ASCII = INTEGER (Given)
      }{
         The position of the character within the ASCII character set.
      }
   }
   \sstreturnedvalue{
      \sstsubsection{
         CHR\_ACHR = CHARACTER $*$ 1
      }{
         A character value within the machine's character set.
      }
   }
}
\sstroutine{
   CHR\_APPND
}{
   Copy one string into another, ignoring trailing blanks
}{
   \sstdescription{
      The string STR1 (or as much of it as there is room for) is
      copied into the part of STR2 beginning at position IPOSN$+$1.
      IPOSN is updated to indicate the final length of STR2 after
      this operation. Trailing blanks in STR1 are ignored.
   }
   \sstinvocation{
      CALL CHR\_APPND( STR1, STR2, IPOSN )
   }
   \sstarguments{
      \sstsubsection{
         STR1 = CHARACTER $*$ ( $*$ ) (Given)
      }{
         The string to be copied.
      }
      \sstsubsection{
         STR2 = CHARACTER $*$ ( $*$ ) (Given and Returned)
      }{
         The string to be updated.
      }
      \sstsubsection{
         IPOSN = INTEGER (Given and Returned)
      }{
         The position in STR2 at which STR1 is to be appended. This
         value is returned updated to be the position of the last
         non-blank character in STR2 after the copy.
      }
   }
}
\sstroutine{
   CHR\_ATOK
}{
   Return the character for a given ASCII character token
}{
   \sstdescription{
      The given ASCII character token is converted to a single returned
      character in the machine's character set.  All
      non-printable ASCII characters are represented by their equivalent
      token strings. If no such ASCII character exists, the character
      code 0 (the ASCII NUL character) is returned. The routine is intended
      for the portable initialisation of unprintable characters.
   }
   \sstinvocation{
      RESULT = CHR\_ATOK( TOKEN )
   }
   \sstarguments{
      \sstsubsection{
         TOKEN = CHARACTER $*$ ( $*$ ) (Given)
      }{
         A printable character string representing the character to
         be returned, e.g., \qt{BEL}, \qt{BS}, etc.
      }
   }
   \sstreturnedvalue{
      \sstsubsection{
         CHR\_ATOK = CHARACTER $*$ 1
      }{
         The character code within the ASCII character set.
      }
   }
}
\sstroutine{
   CHR\_ATOM
}{
   Translate a string from ASCII to the machine's character set
}{
   \sstdescription{
      The string STR1, which has been written on a machine which uses
      the ASCII character set and subsequently read on another
      machine is returned in STR2 translated into the correct
      character set for that machine.
   }
   \sstinvocation{
      CALL CHR\_ATOM( STR1, STR2 )
   }
   \sstarguments{
      \sstsubsection{
         STR1 = CHARACTER $*$ ( $*$ ) (Given)
      }{
         The character string written on a machine with an ASCII
         character set and read on a machine which may not use
         ASCII.
      }
      \sstsubsection{
         STR2 = CHARACTER $*$ ( $*$ ) (Returned)
      }{
         The character string translated into the machine's character
         set. If STR2 is shorter than STR1, the translated string will
         be truncated; if STR2 is longer than STR1, STR2 will be padded
         with blanks beyond the translated string.
      }
   }
}
\sstroutine{
   CHR\_BTOI
}{
   Read an INTEGER value from a binary string
}{
   \sstdescription{
      The given binary string is decoded into an INTEGER value.
   }
   \sstinvocation{
      CALL CHR\_BTOI( STRING, IVALUE, STATUS )
   }
   \sstarguments{
      \sstsubsection{
         STRING = CHARACTER $*$ ( $*$ ) (Given)
      }{
         String to be decoded, e.g. \qt{10101100}.
      }
      \sstsubsection{
         IVALUE = INTEGER (Returned)
      }{
         Value decoded from the given string.
      }
      \sstsubsection{
         STATUS = INTEGER (Given and Returned)
      }{
         The status value. If this value is not SAI\_\_OK on input, the
         routine returns without action. If the routine fails to
         complete successfully, STATUS is returned set to SAI\_\_ERROR.
      }
   }
   \sstnotes{
      This subroutine assumes a 32-bit, twos-complement representation
      of an INTEGER.
   }
}
\sstroutine{
   CHR\_CLEAN
}{
   Remove all unprintable characters from a string
}{
   \sstdescription{
      Replace all unprintable characters in the given string with blanks.
   }
   \sstinvocation{
      CALL CHR\_CLEAN( STRING )
   }
   \sstarguments{
      \sstsubsection{
         STRING = CHARACTER $*$ ( $*$ ) (Given and Returned)
      }{
         String to be cleaned.
      }
   }
}
\sstroutine{
   CHR\_COPY
}{
   Copy one string to another, checking for truncation
}{
   \sstdescription{
      This routine copies one character string to another, checking
      for truncation caused by the returned string being too short to
      accommodate the entire given string. As much of the given string
      as possible is copied to the returned string, ignoring any
      trailing blanks. If truncation is found, it is indicated by the
      returned status.
      Optionally, the last character of the returned string may also
      be set to \qt{\#} if truncation occurs.
   }
   \sstinvocation{
      CALL CHR\_COPY( STR1, TRUNC, STR2, LSTAT )
   }
   \sstarguments{
      \sstsubsection{
         STR1 = CHARACTER $*$ ( $*$ ) (Given)
      }{
         The given string.
      }
      \sstsubsection{
         TRUNC = LOGICAL (Given)
      }{
         A logical flag indicating the action to be taken if
         truncation occurs: if TRUNC is .TRUE., a \qt{\#} will be
         written into the last element of the returned string on
         truncation; if TRUNC is .FALSE., no \qt{\#} is written to the
         returned string.
      }
      \sstsubsection{
         STR2 = CHARACTER $*$ ( $*$ ) (Returned)
      }{
         The returned string. This will contain the given string,
         possibly truncated.
      }
      \sstsubsection{
         LSTAT = INTEGER (Returned)
      }{
         The status: 0 for success, 1 if truncation occurs.
      }
   }
}
\sstroutine{
   CHR\_CTOC
}{
   Write a CHARACTER string into another string
}{
   \sstdescription{
      Write the given character string into the returned character string.
      If the given string is longer than the returned string, the given
      string is truncated. If the returned string is longer than the
      given character variable, the remainder of the returned string
      is padded with blanks.
   }
   \sstinvocation{
      CALL CHR\_CTOC( STR1, STR2, NCHAR )
   }
   \sstarguments{
      \sstsubsection{
         STR1 = CHARACTER $*$ ( $*$ ) (Given)
      }{
         The value to be written.
      }
      \sstsubsection{
         STR2 = CHARACTER $*$ ( $*$ ) (Returned)
      }{
         The character string into which the value is to be written.
      }
      \sstsubsection{
         NCHAR = INTEGER (Returned)
      }{
         The resulting length of the character string, ignoring
         trailing blanks.
      }
   }
}
\sstroutine{
   CHR\_CTOD
}{
   Read a DOUBLE PRECISION value from a string
}{
   \sstdescription{
      Read a DOUBLE PRECISION value from the given character string.
   }
   \sstinvocation{
      CALL CHR\_CTOD( STRING, DVALUE, STATUS )
   }
   \sstarguments{
      \sstsubsection{
         STRING = CHARACTER $*$ ( $*$ ) (Given)
      }{
         The string from which a DOUBLE PRECISION value is to be read.
      }
      \sstsubsection{
         DVALUE = DOUBLE PRECISION (Returned)
      }{
         The resulting DOUBLE PRECISION value.
      }
      \sstsubsection{
         STATUS = INTEGER (Given and Returned)
      }{
         The status value: if this value is not SAI\_\_OK on input,
         the routine returns without action; if the routine does
         not complete successfully, STATUS is returned set to
         SAI\_\_ERROR.
      }
   }
}
\sstroutine{
   CHR\_CTOI
}{
   Read an INTEGER value from a string
}{
   \sstdescription{
      Read an INTEGER value from the given character string.
   }
   \sstinvocation{
      CALL CHR\_CTOI( STRING, IVALUE, STATUS )
   }
   \sstarguments{
      \sstsubsection{
         STRING = CHARACTER $*$ ( $*$ ) (Given)
      }{
         The string from which an INTEGER value is to be read.
      }
      \sstsubsection{
         IVALUE = INTEGER (Returned)
      }{
         The resulting INTEGER value.
      }
      \sstsubsection{
         STATUS = INTEGER (Given and Returned)
      }{
         The status value. If this value is not SAI\_\_OK on input,
         the routine returns without action; if the routine does
         not complete successfully, STATUS is returned set to
         SAI\_\_ERROR.
      }
   }
}
\sstroutine{
   CHR\_CTOL
}{
   Read a LOGICAL value from a string
}{
   \sstdescription{
      The given string is decoded as a logical value. TRUE, T, YES,
      Y and FALSE, F, NO, N are recognised, regardless of case. Other
      strings result in STATUS being set to SAI\_\_ERROR.
   }
   \sstinvocation{
      CALL CHR\_CTOL( STRING, LVALUE, STATUS )
   }
   \sstarguments{
      \sstsubsection{
         STRING = CHARACTER $*$ ( $*$ ) (Given)
      }{
         The string from which a LOGICAL value is to be read.
      }
      \sstsubsection{
         LVALUE = LOGICAL (Returned)
      }{
         The resulting LOGICAL value.
      }
      \sstsubsection{
         STATUS = INTEGER (Given and Returned)
      }{
         The status value: if this value is not SAI\_\_OK on input,
         the routine returns without action; if the routine does
         not complete successfully, STATUS is returned set to
         SAI\_\_ERROR.
      }
   }
}
\sstroutine{
   CHR\_CTOR
}{
   Read a REAL value from a string
}{
   \sstdescription{
      Read a REAL value from the given character string.
   }
   \sstinvocation{
      CALL CHR\_CTOR( STRING, RVALUE, STATUS )
   }
   \sstarguments{
      \sstsubsection{
         STRING = CHARACTER $*$ ( $*$ ) (Given)
      }{
         The string from which a REAL value is to be read.
      }
      \sstsubsection{
         RVALUE = REAL (Returned)
      }{
         The resulting REAL value.
      }
      \sstsubsection{
         STATUS = INTEGER (Given and Returned)
      }{
         The status value: if this value is not SAI\_\_OK on input,
         the routine returns without action; if the routine does
         not complete successfully, STATUS is returned set to
         SAI\_\_ERROR.
      }
   }
}
\sstroutine{
   CHR\_DCWRD
}{
   Split a string into its component words
}{
   \sstdescription{
      All the words in the given character string are detected and
      returned as individual elements of a character array. In this
      context, a word is defined as a continuous string of non-blank
      characters. Hence words must be separated from each other by
      one or more blanks.
   }
   \sstinvocation{
      CALL CHR\_DCWRD( STRING, MXWRD, NWRD, START, STOP, WORDS, LSTAT )
   }
   \sstarguments{
      \sstsubsection{
         STRING = CHARACTER $*$ ( $*$ ) (Given)
      }{
         The string to be split into its constituent words.
      }
      \sstsubsection{
         MXWRD = INTEGER (Given)
      }{
         The maximum number of words that can be extracted from the
         given string: if there are more than MXWRD words in the
         string, only the first MXWRD will be returned.
      }
      \sstsubsection{
         NWRD = INTEGER (Returned)
      }{
         The number of words located in the string.
      }
      \sstsubsection{
         START( MXWRD ) = INTEGER (Returned)
      }{
         The Ith element contains the position of the first element
         of the Ith word in the given string.
      }
      \sstsubsection{
         STOP( MXWRD ) = INTEGER (Returned)
      }{
         The Ith element contains the position of the last element of
         the Ith word in the given string.
      }
      \sstsubsection{
         WORDS( MXWRD ) = CHARACTER $*$ ( $*$ ) (Returned)
      }{
         The Ith element contains the Ith word located in the given
         string.
      }
      \sstsubsection{
         LSTAT = INTEGER (Returned)
      }{
         The local status. This is a return status only: the routine
         is not affected by the value on input. It has the following
         values: SAI\_\_OK for successful completion, SAI\_\_ERROR if the
         number of words exceeds MXWRD.
      }
   }
}
\sstroutine{
   CHR\_DELIM
}{
   Locate a substring using a given delimiter character
}{
   \sstdescription{
      The given character string is examined to see if it contains a
      substring delimited by the character, DELIM. The indices of the
      first and last characters of the substring are returned as
      INDEX1 and INDEX2 respectively. If no occurrence of the specified
      delimiter is found, or if the only occurrence is the last
      character of the string, then the indices are returned pointing
      to the whole of the input string. If only one occurrence of the
      delimiter is found and it is not the last character in the string,
      INDEX1 will point to this position and INDEX2 will point to the
      last character in the string. If there are more than two
      of the occurrences of the delimiter character, INDEX1 will point
      to the first occurrence and INDEX2 to the last occurrence.
   }
   \sstinvocation{
      CALL CHR\_DELIM( STRING, DELIM, INDEX1, INDEX2 )
   }
   \sstarguments{
      \sstsubsection{
         STRING = CHARACTER $*$ ( $*$ ) (Given)
      }{
         The character string to be searched.
      }
      \sstsubsection{
         DELIM = CHARACTER $*$ 1 (Given)
      }{
         The substring delimiting character.
      }
      \sstsubsection{
         INDEX1 = INTEGER (Returned)
      }{
         The position of the first occurrence of the delimiter, or the
         first character in the string.
      }
      \sstsubsection{
         INDEX2 = INTEGER (Returned)
      }{
         The position of the last occurrence of the delimiter, or the
         last character in the string.
      }
   }
}
\sstroutine{
   CHR\_DTOAN
}{
   Write a DOUBLE PRECISION value into a string as hr/deg:min:sec
}{
   \sstdescription{
      Format a DOUBLE PRECISION value as hours/degrees:minutes:seconds
      and write it into a character string. This routine is for writing
      angular measures into a character string in a format suitable
      for presentation to an astronomer.

      If the absolute value of the number to be written exceeds
      a predefined maximum a conversion is not attempted, but the
      number is written as a real number in Fortran `exponential'
      format and a couple of question marks are appended to its
      end. This prevents silly results when very large numbers
      are input. The variable UNITS controls the maximum permitted
      value for the conversion to be carried out.

      The value is written into the part of the string beginning
      at position IPOSN$+$1 and IPOSN is returned updated to the
      position of the end of the encoded angle in STRING.
   }
   \sstinvocation{
      CALL CHR\_DTOAN( DVALUE, UNITS, STRING, IPOSN )
   }
   \sstarguments{
      \sstsubsection{
         DVALUE = DOUBLE PRECISION (Given)
      }{
         The value to be encoded into the string. This value should
         represent an angular measure.
      }
      \sstsubsection{
         UNITS = CHARACTER $*$ ( $*$ ) (Given)
      }{
         This string controls the maximum value which will be formatted
         as hr/deg:min:sec: if UNITS = \qt{HOURS}, the maximum permitted
         value is 24.0; if UNITS = \qt{DEGREES}, the  maximum permitted is
         360.0. In all other cases the maximum is 1000.0.
      }
      \sstsubsection{
         STRING = CHARACTER $*$ ( $*$ ) (Given and Returned)
      }{
         The string into which DVALUE is written.
      }
      \sstsubsection{
         IPOSN = INTEGER (Given and Returned)
      }{
         Given as the last element in STRING before the beginning of the
         encoded angle. Returned as the element in STRING corresponding
         to the end of the encoded angle.
      }
   }
}
\sstroutine{
   CHR\_DTOC
}{
   Encode a DOUBLE PRECISION value as a string
}{
   \sstdescription{
      Encode a DOUBLE PRECISION value as a character string, using as
      concise a format as possible, and return the number of characters
      used. In the event of an error, \qt{$*$}s are written to the string.
   }
   \sstinvocation{
      CALL CHR\_DTOC( DVALUE, STRING, NCHAR )
   }
   \sstarguments{
      \sstsubsection{
         DVALUE = DOUBLE PRECISION (Given)
      }{
         The value to be encoded.
      }
      \sstsubsection{
         STRING = CHARACTER $*$ ( $*$ ) (Returned)
      }{
         The string into which the value is to be encoded.
      }
      \sstsubsection{
         NCHAR = INTEGER (Returned)
      }{
         The field width used in encoding the value.
      }
   }
}
\sstroutine{
   CHR\_EQUAL
}{
   Return whether two strings are equal
}{
   \sstdescription{
      Determine whether the two given strings are the same, with case
      distinction. Their lengths must be identical after removing
      trailing blanks.
   }
   \sstinvocation{
      RESULT = CHR\_EQUAL( STR1, STR2 )
   }
   \sstarguments{
      \sstsubsection{
         STR1 = CHARACTER $*$ ( $*$ ) (Given)
      }{
         The first string.
      }
      \sstsubsection{
         STR2 = CHARACTER $*$ ( $*$ ) (Given)
      }{
         The second string.
      }
   }
   \sstreturnedvalue{
      \sstsubsection{
         CHR\_EQUAL = LOGICAL
      }{
         Returned as .TRUE. if the two given strings are the same,
         otherwise .FALSE.
      }
   }
   \sstnotes{
      This routine is OBSOLETE.  It exists for historical reasons.
      Its function is better performed by a Fortran relational expression.
   }
}
\sstroutine{
   CHR\_ETOM
}{
   Translate a string from EBCDIC to the machine's character set
}{
   \sstdescription{
      The string STR1, which has been written on a machine which uses
      the EBCDIC character set and subsequently read on a machine which
      may not use the EBCDIC character set to represent characters in
      Fortran, is returned in STR2 translated into the correct
      character set for the host machine.
   }
   \sstinvocation{
      CALL CHR\_ETOM( STR1, STR2 )
   }
   \sstarguments{
      \sstsubsection{
         STR1 = CHARACTER $*$ ( $*$ ) (Given)
      }{
         The character string written on a machine with an EBCDIC
         character set and read on a machine which may not use
         EBCDIC to represent characters in Fortran.
      }
      \sstsubsection{
         STR2 = CHARACTER $*$ ( $*$ ) (Returned)
      }{
         The translated EBCDIC character string. If STR2 is shorter
         than STR1, the translated string will be truncated; if
         STR2 is longer than STR1, STR2 will be padded with blanks
         beyond the translated string.
      }
   }
   \sstnotes{
      This subroutine has been implemenred for machines which use the
      ASCII character set.
   }
}
\sstroutine{
   CHR\_FANDL
}{
   Find the first and last non-blank characters in a string
}{
   \sstdescription{
      Find the indices of the first and last non-blank characters in
      the given string. If the string is all blank, the first index
      is returned set to the end of the string and the last index is
      returned set to zero, i.e. INDEX1 is greater than INDEX2.
      If the string has no length, i.e. it is a substring with the first
      index greater than the second, both indices are returned set to
      zero.
   }
   \sstinvocation{
      CALL CHR\_FANDL( STRING, INDEX1, INDEX2 )
   }
   \sstarguments{
      \sstsubsection{
         STRING = CHARACTER $*$ ( $*$ ) (Given)
      }{
         The character string.
      }
      \sstsubsection{
         INDEX1 = INTEGER (Returned)
      }{
         The position of first non-blank character.
      }
      \sstsubsection{
         INDEX2 = INTEGER (Returned)
      }{
         The position of last non-blank character.
      }
   }
}
\sstroutine{
   CHR\_FILL
}{
   Fill a string with a given character
}{
   \sstdescription{
      The given character string is filled with the specified character.
   }
   \sstinvocation{
      CALL CHR\_FILL( CVALUE, STRING )
   }
   \sstarguments{
      \sstsubsection{
         CVALUE = CHARACTER (Given)
      }{
         The character specified to fill the string.
      }
      \sstsubsection{
         STRING = CHARACTER $*$ ( $*$ ) (Returned)
      }{
         The string to be filled.
      }
   }
}
\sstroutine{
   CHR\_FIND
}{
   Find the next occurrence of given substring within a string
}{
   \sstdescription{
      Increments a pointer to a character position within the given
      string and checks if the following sequence of characters matches
      the specified substring, ignoring differences in case. The search
      may be performed either forwards or backwards. If a match is found,
      the position of the substring is returned. If no match exists,
      the pointer is set to one more than the length of the string
      if the search is forwards, zero if the search is backwards.
   }
   \sstinvocation{
      CALL CHR\_FIND( STRING, SUBSTR, FORWD, IPOSN )
   }
   \sstarguments{
      \sstsubsection{
         STRING = CHARACTER $*$ ( $*$ ) (Given)
      }{
         The string to be searched.
      }
      \sstsubsection{
         SUBSTR = CHARACTER $*$ ( $*$ ) (Given)
      }{
         The substring to be searched for, ignoring case.
      }
      \sstsubsection{
         FORWD = LOGICAL (Given)
      }{
         The search direction: if .TRUE., proceed through the string
         in a forward direction, otherwise work backwards.
      }
      \sstsubsection{
         IPOSN = INTEGER (Given and Returned)
      }{
         The starting position for the search. If the initial value of
         IPOSN does not point at a character within the string, the
         routine returns without action.
      }
   }
}
\sstroutine{
   CHR\_FIWE
}{
   Find the next end-of-word within a string
}{
   \sstdescription{
      Find the next end-of-word, signified by the following character
      being a word delimiter (SPACE, TAB or COMMA). Note that the
      start of the next word is not found before looking for the
      next word delimiter so it is possible for IPOSN to remain
      unchanged and indeed to point to a word delimiter rather than a
      true end of a word. This routine is expected to be used in
      conjunction with CHR\_FIWS.
   }
   \sstinvocation{
      CALL CHR\_FIWE (STRING, IPOSN, STATUS)
   }
   \sstarguments{
      \sstsubsection{
         STRING = CHARACTER $*$ ( $*$ ) (Given)
      }{
         The string to be searched.
      }
      \sstsubsection{
         IPOSN = INTEGER (Given and Returned)
      }{
         The given value is the character position within the string
         at which searching is to start. If IPOSN is less than 1, the
         search starts at position 1. The returned value is the
         character position preceding the next word delimiter. If
         IPOSN already points to a character preceding a delimiter,
         it is returned unchanged. If no delimiter is found, IPOSN is
         returned pointing to the end of the string, and STATUS is
         returned set.
      }
      \sstsubsection{
         STATUS = INTEGER (Given and Returned)
      }{
         The status value: if this value is not SAI\_\_OK on entry,
         the routine returns without action; if the next word delimiter
         is not found before the end of the string, STATUS is returned
         set to CHR\_\_EOSNT.
         Note: The CHR\_\_EOSNT symbolic constant is defined in the CHR\_ERR
         include file.
      }
   }
}
\sstroutine{
   CHR\_FIWS
}{
   Find the start of the next word within a string
}{
   \sstdescription{
      Find the start of the next word, signified by the character not
      being a word delimiter, i.e. SPACE, TAB, or COMMA. Note that the
      end of the current word is not found before looking for the start
      of the next. This routine is expected to be used in conjunction
      with CHR\_FIWE.
   }
   \sstinvocation{
      CALL CHR\_FIWS( STRING, IPOSN, STATUS )
   }
   \sstarguments{
      \sstsubsection{
         STRING = CHARACTER $*$ ( $*$ ) (Given)
      }{
         The string to be searched.
      }
      \sstsubsection{
         IPOSN = INTEGER (Given and Returned)
      }{
         The given value is the character position within the string
         at which searching is to start. If IPOSN is less than 1, the
         search starts at position 1. The returned value is the
         character position at which the next word starts. If IPOSN
         already points to a character within a word, it is returned
         unchanged. If no word is found, IPOSN is returned pointing to
         the end of the string, and STATUS is returned set.
      }
      \sstsubsection{
         STATUS = INTEGER (Given and Returned)
      }{
         The status value: if this value is not SAI\_\_OK on entry,
         the routine returns without action; if no word is found, STATUS
         is returned set to CHR\_\_WNOTF.
         Note: The CHR\_\_WNOTF symbolic constant is defined in the CHR\_ERR
         include file.
      }
   }
}
\sstroutine{
   CHR\_FPARX
}{
   Find a parenthesised expression in a character string
}{
   \sstdescription{
      The routine searches the string STR to identify a sub-string
      containing a parenthesised expression and returns the character
      positions of the opening and closing parentheses in the F and L
      arguments. Allowance is made for nested parentheses. If a
      parenthesised expression was not found, then the returned value
      of F will be greater than the returned value of L.
   }
   \sstinvocation{
      CALL CHR\_FPARX( STR, OPPAR, CLPAR, F, L )
   }
   \sstarguments{
      \sstsubsection{
         STR = CHARACTER $*$ ( $*$ ) (Given)
      }{
         String to be searched.
      }
      \sstsubsection{
         OPPAR = CHARACTER $*$ ( 1 ) (Given)
      }{
         The opening parenthesis character.
      }
      \sstsubsection{
         CLPAR = CHARACTER $*$ ( 1 ) (Given)
      }{
         The closing parenthesis character.
      }
      \sstsubsection{
         F = INTEGER (Returned)
      }{
         Character position of the opening parenthesis.
      }
      \sstsubsection{
         L = INTEGER (Returned)
      }{
         Character position of the closing parenthesis.
      }
   }
}
\sstroutine{
   CHR\_HTOI
}{
   Read an INTEGER value from a hexadecimal string
}{
   \sstdescription{
      The given hexadecimal string is decoded into an INTEGER value.
   }
   \sstinvocation{
      CALL CHR\_HTOI( STRING, IVALUE, STATUS )
   }
   \sstarguments{
      \sstsubsection{
         STRING = CHARACTER $*$ ( $*$ ) (Given)
      }{
         String to be decoded.
      }
      \sstsubsection{
         IVALUE = INTEGER (Returned)
      }{
         Value decoded from the given string.
      }
      \sstsubsection{
         STATUS = INTEGER (Given and Returned)
      }{
         The status value. If this value is not SAI\_\_OK on input,
         the routine returns without action. If the routine fails
         to complete successfully, STATUS is returned set to
         SAI\_\_ERROR.
      }
   }
   \sstnotes{
      This subroutine assumes a 32-bit, twos-complement representation
      of an INTEGER.
   }
}
\sstroutine{
   CHR\_IACHR
}{
   Return the ASCII value for the given character
}{
   \sstdescription{
      The given character, encoded using the machine's character set,
      is converted to an integer indicating its position in the ASCII
      character set. If no such character exists, zero is returned.
   }
   \sstinvocation{
      RESULT = CHR\_IACHR( CVALUE )
   }
   \sstarguments{
      \sstsubsection{
         CVALUE = CHARACTER $*$ 1 (Given)
      }{
         The character to be converted to its position within the
         ASCII character set.
      }
   }
   \sstreturnedvalue{
      \sstsubsection{
         CHR\_IACHR = INTEGER
      }{
         An integer position within the ASCII character set.
      }
   }
}
\sstroutine{
   CHR\_INDEX
}{
   Return the index of a substring in a string
}{
   \sstdescription{
      Find the position of a substring within a given string. If no
      substring is found, the value zero is returned.
   }
   \sstinvocation{
      RESULT = CHR\_INDEX( STRING, SUBSTR )
   }
   \sstarguments{
      \sstsubsection{
         STRING = CHARACTER $*$ ( $*$ ) (Given)
      }{
         The string to be searched.
      }
      \sstsubsection{
         SUBSTR = CHARACTER $*$ ( $*$ ) (Given)
      }{
         The substring to be used in the search.
      }
   }
   \sstreturnedvalue{
      \sstsubsection{
         CHR\_INDEX = INTEGER
      }{
         The position of SUBSTR within STRING.
      }
   }
   \sstnotes{
      This routine is OBSOLETE.  It exists for historical reasons.
      Its function is identical to the Fortran intrinsic function INDEX.
      It is recommended that the INDEX intrinsic function be called
      directly.
   }
}
\sstroutine{
   CHR\_INSET
}{
   Return whether a string is a member of a given set
}{
   \sstdescription{
      The character string is compared with each of the values
      given in the given set.
      The strings in the set can be any length and can differ
      in length throughout the set. Each value is separated by a
      comma. Upper and lowercase are treated as being equivalent and
      trailing blanks are ignored.
   }
   \sstinvocation{
      RESULT = CHR\_INSET( SET, STRING )
   }
   \sstarguments{
      \sstsubsection{
         SET = CHARACTER $*$ ( $*$ ) (Given)
      }{
         The set of character values. It takes the form
         \qt{string1,string2,.......,stringN} where each of the
         substring values from string1 to stringN can be of different
         lengths.
      }
      \sstsubsection{
         STRING = CHARACTER $*$ ( $*$ ) (Given)
      }{
         The character string to be checked for membership of the set.
      }
   }
   \sstreturnedvalue{
      \sstsubsection{
         CHR\_INSET = LOGICAL
      }{
         Returns .TRUE. if the character string is a member of the
         given set, returns .FALSE. otherwise.
      }
   }
}
\sstroutine{
   CHR\_ISALF
}{
   Return whether a character is alphabetic
}{
   \sstdescription{
      The given character is tested for being alphabetic, i.e. A - Z
      or a - z.
   }
   \sstinvocation{
      RESULT = CHR\_ISALF( CVALUE )
   }
   \sstarguments{
      \sstsubsection{
         CVALUE = CHARACTER (Given)
      }{
         The character to be tested.
      }
   }
   \sstreturnedvalue{
      \sstsubsection{
         CHR\_ISALF = LOGICAL
      }{
         Returns .TRUE. if the given character is alphabetic,
         returns .FALSE. otherwise.
      }
   }
}
\sstroutine{
   CHR\_ISALM
}{
   Return whether a character is alphanumeric
}{
   \sstdescription{
      Determine whether a character is alphanumeric, i.e. A - Z,
      a - z, 0 - 9 or \_. Note that this routine treats the underscore
      character as an alphanumeric character.
   }
   \sstinvocation{
      RESULT = CHR\_ISALM( CVALUE )
   }
   \sstarguments{
      \sstsubsection{
         CVALUE = CHARACTER (Given)
      }{
         The character to be tested.
      }
   }
   \sstreturnedvalue{
      \sstsubsection{
         CHR\_ISALM = LOGICAL
      }{
         Returns .TRUE. if the given character is alphanumeric,
         returns .FALSE. otherwise.
      }
   }
}
\sstroutine{
   CHR\_ISDIG
}{
   Return whether a character is a digit
}{
   \sstdescription{
      Determine whether the given character is a digit, i.e. 0 - 9.
   }
   \sstinvocation{
      RESULT = CHR\_ISDIG( CVALUE )
   }
   \sstarguments{
      \sstsubsection{
         CVALUE = CHARACTER (Given)
      }{
         The character to be tested.
      }
   }
   \sstreturnedvalue{
      \sstsubsection{
         CHR\_ISDIG = LOGICAL
      }{
         Returns .TRUE. if the given character is a digit, returns
         .FALSE. otherwise.
      }
   }
}
\sstroutine{
   CHR\_ISNAM
}{
   Return whether a string is a valid name
}{
   \sstdescription{
      Determine whether the given string is a valid name: i.e.
      whether it starts with an alphabetic character and continues
      with alphanumeric or underscore characters.
   }
   \sstinvocation{
      RESULT = CHR\_ISNAM( STRING )
   }
   \sstarguments{
      \sstsubsection{
         STRING = CHARACTER $*$ ( $*$ ) (Given)
      }{
         The string to be tested.
      }
   }
   \sstreturnedvalue{
      \sstsubsection{
         CHR\_ISNAM = LOGICAL
      }{
         Returns .TRUE. if the given string is a valid name, returns
         .FALSE. otherwise.
      }
   }
}
\sstroutine{
   CHR\_ITOB
}{
   Write an INTEGER value into a binary string
}{
   \sstdescription{
      Encode an INTEGER value into a binary string. The result is
      right-justified in the returned string. In the event of an error,
      \qt{$*$}s are written to the string.
   }
   \sstinvocation{
      CALL CHR\_ITOB( IVALUE, STRING, STATUS )
   }
   \sstarguments{
      \sstsubsection{
         IVALUE = INTEGER (Given)
      }{
         Value to be encoded.
      }
      \sstsubsection{
         STRING = CHARACTER $*$ ( $*$ ) (Returned)
      }{
         Binary string encoded from the given value.
      }
   }
   \sstnotes{
      This subroutine assumes a 32-bit, twos-complement representation
      of an INTEGER.
   }
}
\sstroutine{
   CHR\_ITOC
}{
   Encode an INTEGER value as a string
}{
   \sstdescription{
      Encode an integer value as a (decimal) character string, using as
      concise a format as possible, and return the number of characters
      used. In the event of an error, \qt{$*$}s will be written into to the
      string.
   }
   \sstinvocation{
      CALL CHR\_ITOC( IVALUE, STRING, NCHAR )
   }
   \sstarguments{
      \sstsubsection{
         IVALUE = INTEGER (Given)
      }{
         The value to be encoded.
      }
      \sstsubsection{
         STRING = CHARACTER $*$ ( $*$ ) (Returned)
      }{
         The string into which the integer value is encoded.
      }
      \sstsubsection{
         NCHAR = INTEGER (Returned)
      }{
         The field width used in encoding the value.
      }
   }
}
\sstroutine{
   CHR\_ITOH
}{
   Write a hexadecimal string from an INTEGER value
}{
   \sstdescription{
      Encode an INTEGER value into a hexadecimal string using the
      machine's character set. The result is right-justified in the
      returned string. In the event of an error, \qt{$*$}s are written to
      the string.
   }
   \sstinvocation{
      CALL CHR\_ITOH( IVALUE, STRING, STATUS )
   }
   \sstarguments{
      \sstsubsection{
         IVALUE = INTEGER (Given)
      }{
         Value to be encoded.
      }
      \sstsubsection{
         STRING = CHARACTER $*$ ( $*$ ) (Returned)
      }{
         Hexadecimal string encoded from the given value.
      }
      \sstsubsection{
         STATUS = INTEGER (Given and Returned)
      }{
         The status value. If this value is not SAI\_\_OK on input, the
         routine returns without action. If the routine fails to
         complete successfully, STATUS is returned set to SAI\_\_ERROR.
      }
   }
   \sstnotes{
      This subroutine assumes a 32-bit, twos-complement representation
      of an INTEGER.
   }
}
\sstroutine{
   CHR\_ITOO
}{
   Write an octal string from an INTEGER value
}{
   \sstdescription{
      Encode an INTEGER value into an octal string using the host
      machine's character set. The result is right-justified in the
      returned string. In the event of an error, \qt{$*$}s are written to
      the string.
   }
   \sstinvocation{
      CALL CHR\_ITOO( IVALUE, STRING, STATUS )
   }
   \sstarguments{
      \sstsubsection{
         IVALUE = INTEGER (Given)
      }{
         Value to be encoded.
      }
      \sstsubsection{
         STRING = CHARACTER $*$ ( $*$ ) (Returned)
      }{
         Octal string encoded from the given value.
      }
   }
   \sstnotes{
      This subroutine assumes a 32-bit, twos complement representation
      of an INTEGER.
   }
}
\sstroutine{
   CHR\_LASTO
}{
   Locates the last occurence of CVAL in STRING
}{
   \sstdescription{
      The routine locates the last occurence of the single character
      CVAL in STRING. If an occurence is not located then IAT is
      returned as 0.
   }
   \sstinvocation{
      CALL CHR\_LASTO( STRING, CVAL, IAT )
   }
   \sstarguments{
      \sstsubsection{
         STRING = CHARACTER $*$ ( $*$ ) (Given)
      }{
         String to be searched for occurences of CVAL.
      }
      \sstsubsection{
         CVAL = CHARACTER $*$ ( 1 ) (Given)
      }{
         Character whose last occurence is to be located.
      }
      \sstsubsection{
         IAT = INTEGER (Returned)
      }{
         Position  within STRING at which last occurence of CVAL is
         located. Set to 0 if the character is not found.
      }
   }
}
\sstroutine{
   CHR\_LCASE
}{
   Convert a string to lowercase
}{
   \sstdescription{
      The characters in the string are all converted to lowercase
      in situ.
   }
   \sstinvocation{
      CALL CHR\_LCASE( STRING )
   }
   \sstarguments{
      \sstsubsection{
         STRING = CHARACTER $*$ ( $*$ ) (Given and Returned)
      }{
         The string to be converted to lowercase.
      }
   }
}
\sstroutine{
   CHR\_LDBLK
}{
   Remove any leading blanks from a string
}{
   \sstdescription{
      Remove any leading blanks from the character string.
      The remaining characters are moved to the left to eliminate the
      resulting empty space, and the end of the string is filled with
      blanks.
   }
   \sstinvocation{
      CALL CHR\_LDBLK( STRING )
   }
   \sstarguments{
      \sstsubsection{
         STRING = CHARACTER $*$ ( $*$ ) (Given and Returned)
      }{
         The string from which the leading blanks are to be removed.
      }
   }
}
\sstroutine{
   CHR\_LEN
}{
   Return the length of a string, ignoring trailing blanks
}{
   \sstdescription{
      Find length of string, ignoring trailing blanks.
   }
   \sstinvocation{
      RESULT = CHR\_LEN( STRING )
   }
   \sstarguments{
      \sstsubsection{
         STRING = CHARACTER $*$ ( $*$ ) (Given)
      }{
         The string whose length is to be determined.
      }
   }
   \sstreturnedvalue{
      \sstsubsection{
         CHR\_LEN = INTEGER
      }{
         Returns the used length of the string.
      }
   }
}
\sstroutine{
   CHR\_LINBR
}{
   Break a line of text into a sequence of shorter lines
}{
   \sstdescription{
      Break a long line of text into a sequence of shorter lines,
      making the breaks between words at spaces if possible.
      The maximum length of an output line is determined by
      the size of the character variable supplied to contain it.
      This routine should be called repeatedly to generate successive
      output lines from a single long input line.  Initially, the
      context argument IPOSN should be set to zero; it will be updated
      after each call, ready to generate the next output line. A value
      of zero is returned for IPOSN when there are no more output
      lines. Any unprintable characters (e.g. tabs) are treated as if
      they were blanks for the purpose of identifying line-breaks.
   }
   \sstinvocation{
      CALL CHR\_LINBR( STR1, IPOSN, STR2 )
   }
   \sstarguments{
      \sstsubsection{
         STR1 = CHARACTER $*$ ( $*$ ) (Given)
      }{
         The line of text to be broken into shorter lines. Leading
         blanks are ignored.
      }
      \sstsubsection{
         IPOSN = INTEGER (Given and Returned)
      }{
         On entry, this argument specifies the character position in
         STR1 from which to start generating the next returned line.
         If a value less than 1 is given, then 1 will be used.

         On exit, this argument is set to one more than the position
         in STR1 of the last non-blank character which appears in the
         returned line STR2 (i.e. the position at which generation of the
         next returned line should begin). If STR2 is blank because there
         are no more characters to process, then IPOSN is returned set
         to zero.
      }
      \sstsubsection{
         STR2 = CHARACTER $*$ ( $*$ ) (Returned)
      }{
         The returned line, left justified. The length of this argument
         determines the maximum length of the returned line.
      }
   }
}
\sstroutine{
   CHR\_LOWER
}{
   Return the lowercase equivalent of a character
}{
   \sstdescription{
      If the given character is uppercase, the lowercase equivalent
      is returned, otherwise the character will be returned unchanged.
   }
   \sstinvocation{
      RESULT = CHR\_LOWER( CVALUE )
   }
   \sstarguments{
      \sstsubsection{
         CVALUE = CHARACTER $*$ 1 (Given)
      }{
         The character to be converted.
      }
   }
   \sstreturnedvalue{
      \sstsubsection{
         CHR\_LOWER = CHARACTER $*$ 1 (Returned)
      }{
         Lowercase equivalent of the given character, if the given
         character is an uppercase letter; otherwise the character
         is returned unchanged.
      }
   }
}
\sstroutine{
   CHR\_LTOC
}{
   Encode a LOGICAL value as a string
}{
   \sstdescription{
      Encode the given LOGICAL value as one of the character strings
      \qt{TRUE} or \qt{FALSE}.
   }
   \sstinvocation{
      CALL CHR\_LTOC( LVALUE, STRING, NCHAR )
   }
   \sstarguments{
      \sstsubsection{
         LVALUE = LOGICAL (Given)
      }{
         The value to be encoded.
      }
      \sstsubsection{
         STRING = CHARACTER $*$ ( $*$ ) (Returned)
      }{
         The string into which the value is to be encoded.
      }
      \sstsubsection{
         NCHAR = INTEGER (Returned)
      }{
         The field width used in encoding the value.
      }
   }
}
\sstroutine{
   CHR\_MOVE
}{
   Move one string into another
}{
   \sstdescription{
      The string STR1, or as much of it as there is room for, is
      copied into STR2 beginning at position 1.
   }
   \sstinvocation{
      CALL CHR\_MOVE( STR1, STR2 )
   }
   \sstarguments{
      \sstsubsection{
         STR1 = CHARACTER $*$ ( $*$ ) (Given)
      }{
         The given string.
      }
      \sstsubsection{
         STR2 = CHARACTER $*$ ( $*$ ) (Returned)
      }{
         The returned string.
      }
   }
   \sstnotes{
      This routine is OBSOLETE.  It exists for historical reasons.
      Its function is identical to a Fortran assignment statement.
      It is recommended that an assignment statement be used instead
      of CHR\_MOVE.
   }
}
\sstroutine{
   CHR\_MTOA
}{
   Translate a string from the machine's characters set to ASCII
}{
   \sstdescription{
      The string STR1, encoded in the host machine's character set, is
      returned in STR2 translated into a form which can be written and
      subsequently read correctly by a machine which uses the ASCII
      character set.
   }
   \sstinvocation{
      CALL CHR\_MTOA( STR1, STR2 )
   }
   \sstarguments{
      \sstsubsection{
         STR1 = CHARACTER $*$ ( $*$ ) (Given)
      }{
         A string represented by the host machine's character set.
      }
      \sstsubsection{
         STR2 = CHARACTER $*$ ( $*$ ) (Returned)
      }{
         A string represented by the ASCII character set.
         If STR2 is shorter than STR1, the translated string will be
         truncated; if STR2 is longer than STR1, STR2 will be padded
         with blanks beyond the translated string.
      }
   }
}
\sstroutine{
   CHR\_MTOE
}{
   Translate a string from the machine's character set to EBCDIC
}{
   \sstdescription{
      The string STR1, which is a Fortran 77 CHARACTER string, is
      returned in STR2 translated into a form which can be written and
      subsequently read correctly by a machine which uses the EBCDIC
      character set.

      Any characters which are not represented in the
      EBCDIC character set are translated to EBCDIC SPACE. Non-printable
      characters are translated where possible.
   }
   \sstinvocation{
      CALL CHR\_MTOE( STR1, STR2 )
   }
   \sstarguments{
      \sstsubsection{
         STR1 = CHARACTER $*$ ( $*$ ) (Given)
      }{
         The Fortran 77 character string.
      }
      \sstsubsection{
         STR2 = CHARACTER $*$ ( $*$ ) (Returned)
      }{
         A character string which may be written and subsequently read
         correctly by a machine which uses the EBCDIC character set
         to represent characters in Fortran. If STR2 is shorter than
         STR1, the translated string will be truncated; if STR2 is
         longer than STR1, STR2 will be padded with blanks beyond the
         translated string.
      }
   }
   \sstdiytopic{
      System-specific
   }{
      This subroutine has been implemented for machines which use
      the ASCII character set.
   }
}
\sstroutine{
   CHR\_NTH
}{
   Return the two-character ordinal abbreviation for a specified integer
}{
   \sstdescription{
      Return the two character ordinal abbreviation (i.e. st, nd, rd, th)
      appropriate for the given integer value.
   }
   \sstinvocation{
      RESULT = CHR\_NTH( IVALUE )
   }
   \sstarguments{
      \sstsubsection{
         IVALUE = INTEGER (Given)
      }{
         The integer for which the abbreviation is required.
      }
   }
   \sstreturnedvalue{
      \sstsubsection{
         CHR\_NTH = CHARACTER $*$ 2
      }{
         The appropriate two character abbreviation for the given
         integer value.
      }
   }
}
\sstroutine{
   CHR\_OTOI
}{
   Read an INTEGER value from an octal string
}{
   \sstdescription{
      The given octal string is decoded into an INTEGER value.
   }
   \sstinvocation{
      CALL CHR\_OTOI( STRING, IVALUE, STATUS )
   }
   \sstarguments{
      \sstsubsection{
         STRING = CHARACTER $*$ ( $*$ ) (Given)
      }{
         String to be decoded.
      }
      \sstsubsection{
         IVALUE = INTEGER (Returned)
      }{
         Value decoded from the given string.
      }
      \sstsubsection{
         STATUS = INTEGER (Given and Returned)
      }{
         The status value. If this value is not SAI\_\_OK on input, the
         routine returns without action. If the routine fails to
         complete successfully, STATUS is returned set to SAI\_\_ERROR.
      }
   }
   \sstnotes{
   }{
      This subroutine assumes a 32-bit, twos-complement representation
      of an INTEGER.
   }
}
\sstroutine{
   CHR\_PFORM
}{
   Reformat a paragraph to a new width
}{
   \sstdescription{
      This subroutine is called repeatedly to reformat the given
      paragraph to a new width (given by the declared length of the
      returned character variable). The output may be optionally
      justified to the right margin (i.e. the end of the returned
      character variable). This routine should be called repeatedly
      to generate successive returned lines from the given paragraph
      array. Initially, the context argument IPOSN should be set to zero;
      it will be updated after each call, ready to generate the next
      output line. A value of zero is returned for IPOSN when there are
      no more lines to return. Any unprintable characters (e.g. tabs)
      are treated as if they were spaces for the purpose of generating
      line-breaks.
   }
   \sstinvocation{
      CALL CHR\_PFORM( MXPAR, PARRAY, JUSTFY, IPOSN, STRING )
   }
   \sstarguments{
      \sstsubsection{
         MXPAR = INTEGER (Given)
      }{
         The maximum length of the given paragraph array, PARRAY.
      }
      \sstsubsection{
         PARRAY( MXPAR ) = CHARACTER $*$ ( $*$ ) (Given)
      }{
         The character array which contains the paragraph text to be
         reformatted, one line per array element. Leading blanks are
         ignored. A line-break is interpreted as the start of a new
         word.
      }
      \sstsubsection{
         JUSTFY = LOGICAL (Given)
      }{
         The right justification flag: if this is given as .TRUE.,
         the text is returned right justified; otherwise the text is
         returned with a ragged right margin.
      }
      \sstsubsection{
         IPOSN = INTEGER (Given and Returned)
      }{
         On entry, this argument specifies the character position in
         PARRAY from which to start generating the next returned line.
         It is given as the number of characters from the start of the
         first character in the first element in PARRAY. If a value less
         than 1 is used, then 1 will be used.

         On exit, this argument is set to one more than the character
         offset of the start of PARRAY of the last non-blank character
         which appears in the returned line STRING (i.e. the position
         at which the generation of the next output line should start).
         If STRING is blank because there are no more characters to
         process, then IPOSN is returned set to zero.
      }
      \sstsubsection{
         STRING = CHARACTER $*$ ( $*$ ) (Returned)
      }{
         The returned line of text in the paragraph, left justified. The
         length of this argument defines the maximum length of the
         returned paragraph line.
      }
   }
}
\sstroutine{
   CHR\_PREFX
}{
   Prefix a string with a substring
}{
   \sstdescription{
      The substring STR1 is prefixed to the string STR2, moving the
      string STR2 along to make room. The given string in STR2 may be
      truncated by adding the prefix. The final length of the string
      STR2, ignoring trailing blanks, is returned in LEN2.
   }
   \sstinvocation{
      CALL CHR\_PREFX( STR1, STR2, LEN2 )
   }
   \sstarguments{
      \sstsubsection{
         STR1 = CHARACTER $*$ ( $*$ ) (Given)
      }{
         The prefix string.
      }
      \sstsubsection{
         STR2 = CHARACTER $*$ ( $*$ ) (Given and Returned)
      }{
         The string to be prefixed.
      }
      \sstsubsection{
         LEN2 = INTEGER (Returned)
      }{
         The resultant length of the string STR2, ignoring trailing
         blanks.
      }
   }
}
\sstroutine{
   CHR\_PUTC
}{
   Put a CHARACTER string into another at a given position
}{
   \sstdescription{
      The string STR1 (or as much of it as there is room for) is
      copied into the part of STR2 beginning at position IPOSN$+$1.
      IPOSN is updated to indicate the end position of the copy of
      STR1 within STR2 after this operation. If no copying is
      done, IPOSN is returned unchanged. The sizes of STR1 and
      STR2 are based on the declared Fortran 77 size given by the
      intrinsic function LEN.
   }
   \sstinvocation{
      CALL CHR\_PUTC( STR1, STR2, IPOSN )
   }
   \sstarguments{
      \sstsubsection{
         STR1 = CHARACTER $*$ ( $*$ ) (Given)
      }{
         The string to be copied.
      }
      \sstsubsection{
         STR2 = CHARACTER $*$ ( $*$ ) (Given and Returned)
      }{
         The string into which STR1 is to be copied.
      }
      \sstsubsection{
         IPOSN = INTEGER (Given and Returned)
      }{
         The position pointer within STR2.
      }
   }
}
\sstroutine{
   CHR\_PUTD
}{
   Put a DOUBLE PRECISION value into a string at a given position
}{
   \sstdescription{
      The DOUBLE PRECISION value is encoded into a concise string
      which is then copied into the given string beginning at
      position IPOSN$+$1. IPOSN is returned updated to indicate the
      end position of the encoded number within STRING. This is a
      combination of CHR\_DTOC and CHR\_PUTC.
   }
   \sstinvocation{
      CALL CHR\_PUTD( DVALUE, STRING, IPOSN )
   }
   \sstarguments{
      \sstsubsection{
         DVALUE = DOUBLE PRECISION (Given)
      }{
         The value to be encoded into the string.
      }
      \sstsubsection{
         STRING = CHARACTER $*$ ( $*$ ) (Given and Returned)
      }{
         The string into which DVALUE is to be copied.
      }
      \sstsubsection{
         IPOSN = INTEGER (Given and Returned)
      }{
         The position pointer within STRING.
      }
   }
}
\sstroutine{
   CHR\_PUTI
}{
   Put an INTEGER value into a string at a given position
}{
   \sstdescription{
      The INTEGER value is encoded into a concise string which is
      then copied into the given string beginning at position IPOSN$+$1.
      IPOSN is returned updated to indicate the end position of the
      encoded number within STRING. This is a combination of CHR\_ITOC
      and CHR\_PUTC.
   }
   \sstinvocation{
      CALL CHR\_PUTI( IVALUE, STRING, IPOSN )
   }
   \sstarguments{
      \sstsubsection{
         IVALUE = INTEGER (Given)
      }{
         The value to be encoded into the string.
      }
      \sstsubsection{
         STRING = CHARACTER $*$ ( $*$ ) (Given and Returned)
      }{
         The string into which IVALUE is to be copied.
      }
      \sstsubsection{
         IPOSN = INTEGER (Given and Returned)
      }{
         The position pointer within STRING.
      }
   }
}
\sstroutine{
   CHR\_PUTL
}{
   Put a LOGICAL value into a string at a given position
}{
   \sstdescription{
      The LOGICAL value is encoded into \qt{T} or \qt{F} which is then
      copied into the given string beginning at position IPOSN$+$1.
      IPOSN is is returned updated to indicate the end position
      of the encoded logical value within STRING.
   }
   \sstinvocation{
      CALL CHR\_PUTL( LVALUE, STRING, IPOSN )
   }
   \sstarguments{
      \sstsubsection{
         LVALUE = LOGICAL (Given)
      }{
         The LOGICAL value to be encoded into the string.
      }
      \sstsubsection{
         STRING = CHARACTER $*$ ( $*$ ) (Given and Returned)
      }{
         The string into which LVALUE is to be copied.
      }
      \sstsubsection{
         IPOSN = INTEGER (Given and Returned)
      }{
         The position pointer within STRING.
      }
   }
}
\sstroutine{
   CHR\_PUTR
}{
   Put a REAL value into a string at a given position
}{
   \sstdescription{
      The REAL value is encoded into a concise string which is then
      copied into the given string beginning at position IPOSN$+$1.
      IPOSN is returned updated to indicate the end position of the
      encoded number within STRING. This is a combination of CHR\_RTOC
      and CHR\_PUTC.
   }
   \sstinvocation{
      CALL CHR\_PUTR( RVALUE, STRING, IPOSN )
   }
   \sstarguments{
      \sstsubsection{
         RVALUE = REAL (Given)
      }{
         The value to be encoded into the string.
      }
      \sstsubsection{
         STRING = CHARACTER $*$ ( $*$ ) (Given and Returned)
      }{
         The string into which DVALUE is to be copied.
      }
      \sstsubsection{
         IPOSN = INTEGER (Given and Returned)
      }{
         The position pointer within STRING.
      }
   }
}
\sstroutine{
   CHR\_RJUST
}{
   Right-justify a string
}{
   \sstdescription{
      The given string is right-justified by filling out the spaces
      between words with additional blank space. The right margin is
      taken as the declared length of the given string. Unprintable
      characters are interpreted as blanks.
   }
   \sstinvocation{
      CALL CHR\_RJUST( STRING )
   }
   \sstarguments{
      \sstsubsection{
         STRING = CHARACTER $*$ ( $*$ ) (Given and Returned)
      }{
         The string to be right-justified and returned.
      }
   }
}
\sstroutine{
   CHR\_RMBLK
}{
   Remove all blanks from a string
}{
   \sstdescription{
      All leading and embedded blanks in the string are removed.
      The remaining characters are moved to the left to eliminate the
      resulting empty space, and the end of the string is filled with
      blanks.
   }
   \sstinvocation{
      CALL CHR\_RMBLK( STRING )
   }
   \sstarguments{
      \sstsubsection{
         STRING = CHARACTER $*$ ( $*$ ) (Given and Returned)
      }{
         The string from which all leading and embedded blanks are
         removed.
      }
   }
}
\sstroutine{
   CHR\_RMCHR
}{
   Remove all specified characters from a string
}{
   \sstdescription{
      Remove a specified set of characters from a string in situ.
      The remaining characters are moved to the left to eliminate
      the resulting empty space, and the end of the string is filled
      with blanks.
   }
   \sstinvocation{
      CALL CHR\_RMCHR( CHARS, STRING )
   }
   \sstarguments{
      \sstsubsection{
         CHARS = CHARACTER $*$ ( $*$ ) (Given)
      }{
         A string specifying all the characters which are to be removed.
      }
      \sstsubsection{
         STRING = CHARACTER $*$ ( $*$ ) (Given and Returned)
      }{
         The string from which the characters are removed.
      }
   }
}
\sstroutine{
   CHR\_RTOAN
}{
   Write a REAL value into a string as hr/deg:min:sec
}{
   \sstdescription{
      Format a REAL value as hours/degrees:minutes:seconds and write
      it into a character string. This routine is for writing
      angular measures into a character string in a format suitable
      for presentation to an astronomer.

      If the absolute value of the number to be written exceeds
      a predefined maximum a conversion is not attempted, but the
      number is written as a real number in Fortran \qt{exponential}
      format and a couple of question marks are appended to its
      end. This prevents silly results when very large numbers
      are input. The variable UNITS controls the maximum permitted
      value for the conversion to be carried out.

      The value is written into the part of the string beginning
      at position IPOSN$+$1 and IPOSN is returned updated to the
      position of the end of the encoded angle in STRING.
   }
   \sstinvocation{
      CALL CHR\_RTOAN( RVALUE, UNITS, STRING, IPOSN )
   }
   \sstarguments{
      \sstsubsection{
         RVALUE = REAL (Given)
      }{
         The value to be encoded into the string. This value should
         represent an angular measure.
      }
      \sstsubsection{
         UNITS = CHARACTER $*$ ( $*$ ) (Given)
      }{
         This string controls the maximum value which will be formatted
         as hr/deg:min:sec: if UNITS = \qt{HOURS}, the maximum permitted
         value is 24.0; if UNITS = \qt{DEGREES}, the  maximum permitted is
         360.0. In all other cases the maximum is 1000.0.
      }
      \sstsubsection{
         STRING = CHARACTER $*$ ( $*$ ) (Given and Returned)
      }{
         The string into which RVALUE is written.
      }
      \sstsubsection{
         IPOSN = INTEGER (Given and Returned)
      }{
         Given as the last element in STRING before the beginning of the
         encoded angle. Returned as the element in STRING corresponding
         to the end of the encoded angle.
      }
   }
}
\sstroutine{
   CHR\_RTOC
}{
   Encode a REAL value as a string
}{
   \sstdescription{
      Encode a REAL value as a character string, using as concise a
      format as possible, and return the number of characters used.
      In the event of an error, \qt{$*$}s are written to the string.
   }
   \sstinvocation{
      CALL CHR\_RTOC( RVALUE, STRING, NCHAR )
   }
   \sstarguments{
      \sstsubsection{
         RVALUE = REAL (Given)
      }{
         The value to be encoded.
      }
      \sstsubsection{
         STRING = CHARACTER $*$ ( $*$ ) (Returned)
      }{
         The string into which the value is to be encoded.
      }
      \sstsubsection{
         NCHAR = INTEGER (Returned)
      }{
         The field width used in encoding the value.
      }
   }
}
\sstroutine{
   CHR\_SCOMP
}{
   Compare two character strings using the ASCII character set
}{
   \sstdescription{
      The first string is compared with the second using the ASCII
      character set, giving precedence to the left hand side of the
      string. If the first string is less than or equal to the second,
      the value .TRUE. is returned; otherwise the value .FALSE. is
      returned.
   }
   \sstinvocation{
      RESULT = CHR\_SCOMP( STR1, STR2 )
   }
   \sstarguments{
      \sstsubsection{
         STR1 = CHARACTER $*$ ( $*$ ) (Given)
      }{
         The first character string.
      }
      \sstsubsection{
         STR2 = CHARACTER $*$ ( $*$ ) (Given)
      }{
         The second character string.
      }
   }
   \sstreturnedvalue{
      \sstsubsection{
         CHR\_SCOMP = LOGICAL
      }{
         Whether the first character string is less than or equal to
         the second, using the ASCII character set.
      }
   }
}
\sstroutine{
   CHR\_SIMLR
}{
   Return whether two strings are equal, apart from case
}{
   \sstdescription{
      Determine whether two strings are the same, ignoring
      distinctions between upper and lowercase letters.
      Their lengths must be identical after removing trailing blanks.
   }
   \sstinvocation{
      RESULT = CHR\_SIMLR( STR1, STR2 )
   }
   \sstarguments{
      \sstsubsection{
         STR1 = CHARACTER $*$ ( $*$ ) (Given)
      }{
         The first string.
      }
      \sstsubsection{
         STR2 = CHARACTER $*$ ( $*$ ) (Given)
      }{
         The second string.
      }
   }
   \sstreturnedvalue{
      \sstsubsection{
         CHR\_SIMLR = LOGICAL
      }{
         Returned as .TRUE. if the two strings are the same
         ignoring case distinctions; otherwise .FALSE.
      }
   }
}
\sstroutine{
   CHR\_SIZE
}{
   Return the declared size of a string
}{
   \sstdescription{
      Give the declared size of a Fortran 77 character string variable,
      including trailing blanks.
   }
   \sstinvocation{
      RESULT = CHR\_SIZE( STRING )
   }
   \sstarguments{
      \sstsubsection{
         STRING = CHARACTER $*$ ( $*$ ) (Given)
      }{
         The character string of whose length is determined.
      }
   }
   \sstreturnedvalue{
      \sstsubsection{
         CHR\_SIZE = INTEGER
      }{
         Returns the declared size of the string.
      }
   }
   \sstnotes{
      This routine is OBSOLETE.  It exists for historical reasons.
      Its function is identical to the Fortran 77 intrinsic function
      LEN. It is recommended that the intrinsic function LEN be called
      directly.
   }
}
\sstroutine{
   CHR\_SKCHR
}{
   Skip over all specified characters in a string
}{
   \sstdescription{
      Increment a character pointer, IPOSN, either forward or
      backward through a string, until the character pointed to
      is not one of a specified set of characters. The direction of
      the search is given by the argument FORWD. If no such
      character position exists (i.e all remaining characters in the
      string are members of the specified set), the pointer is returned
      set to one more than the length of the string if the search is
      in the forward direction, or zero if the search is in the reverse
      direction. If the initial value of IPOSN does not point at one of
      the characters in the string, then the routine will return without
      action.
   }
   \sstinvocation{
      CALL CHR\_SKCHR( CHARS, STRING, FORWD, IPOSN )
   }
   \sstarguments{
      \sstsubsection{
         CHARS = CHARACTER $*$ ( $*$ ) (Given)
      }{
         A string consisting of the set of characters to be skipped.
      }
      \sstsubsection{
         STRING = CHARACTER $*$ ( $*$ ) (Given)
      }{
         The string to be searched.
      }
      \sstsubsection{
         FORWD = LOGICAL (Given)
      }{
         The search direction: if .TRUE. then proceed through the string
         in a forward direction, otherwise work backwards through the
         string.
      }
      \sstsubsection{
         IPOSN = INTEGER (Given and Returned)
      }{
         The character pointer.
      }
   }
}
\sstroutine{
   CHR\_SORT
}{
   Sort an array of character variables into alphabetical order
}{
   \sstdescription{
      Sort an array of character variables into alphabetical order
      using the collating sequence provided by the routine CHR\_SCOMP.
      After the sort, a search is made to remove any values which
      occur more than once. The total number of unique values is
      returned.
   }
   \sstinvocation{
      CALL CHR\_SORT( CHR\_SCOMP, MXARY, ARRAY, NSORT )
   }
   \sstarguments{
      \sstsubsection{
         CHR\_SCOMP = LOGICAL FUNCTION (Given)
      }{
         An external function which compares two character strings
         and returns whether the first string is less than the second.
      }
      \sstsubsection{
         MXARY = INTEGER (Given)
      }{
         The number of character values to sort.
      }
      \sstsubsection{
         ARRAY( MXARY ) = CHARACTER $*$ ( $*$ ) (Given and Returned)
      }{
         The array of character values to be sorted.
      }
      \sstsubsection{
         NSORT = INTEGER (Returned)
      }{
         The number of unique character values returned.
      }
   }
   \sstnotes{
      To use this subroutine it is necessary to declare the function
      CHR\_SCOMP, or its equivalent, to be EXTERNAL in the calling
      routine.
   }
}
\sstroutine{
   CHR\_SWAP
}{
   Swap two single-character variables
}{
   \sstdescription{
      Exchange the values of two single-character variables.
   }
   \sstinvocation{
      CALL CHR\_SWAP( CHAR1, CHAR2 )
   }
   \sstarguments{
      \sstsubsection{
         CHAR1 = CHARACTER $*$ 1 (Given and Returned)
      }{
         The first character.
      }
      \sstsubsection{
         CHAR2 = CHARACTER $*$ 1 (Given and Returned)
      }{
         The second character.
      }
   }
}
\sstroutine{
   CHR\_TERM
}{
   Terminate a string by padding out with blanks
}{
   \sstdescription{
      The given string, STRING, is terminated to a length of LENGTH
      characters by filling the remainder of its declared length
      with blanks.
   }
   \sstinvocation{
      CALL CHR\_TERM( LENGTH, STRING )
   }
   \sstarguments{
      \sstsubsection{
         LENGTH = INTEGER (Given)
      }{
         The required length for the string: it must be positive
         and not greater than the declared length of the string.
      }
      \sstsubsection{
         STRING = CHARACTER $*$ ( $*$ ) (Given and Returned)
      }{
         The string to be terminated.
      }
   }
}
\sstroutine{
   CHR\_TOCHR
}{
   Skip to the next specified character in a string
}{
   \sstdescription{
      Increment a character pointer, IPOSN, either forward or
      backward through a string, until the character pointed to
      is one of a specified set of characters. The direction of
      the search is given by the argument FORWD. If no such
      character position exists (i.e. none of the remaining
      characters in the string are members of the specified set),
      the pointer is returned set to one more than the length of
      the string if the search is in the forward direction, or
      zero if the search is in the reverse direction. If the initial
      value of IPOSN does not point at one of the characters in the
      string, then the routine will return without action.
   }
   \sstinvocation{
      CALL CHR\_TOCHR( CHARS, STRING, FORWD, IPOSN )
   }
   \sstarguments{
      \sstsubsection{
         CHARS = CHARACTER $*$ ( $*$ ) (Given)
      }{
         A string consisting of the set of characters to be searched
         for.
      }
      \sstsubsection{
         STRING = CHARACTER $*$ ( $*$ ) (Given)
      }{
         The string to be searched.
      }
      \sstsubsection{
         FORWD = LOGICAL (Given)
      }{
         The search direction: if .TRUE. then proceed through the string
         in a forward direction, otherwise work backwards through the
         string.
      }
      \sstsubsection{
         IPOSN = INTEGER (Given and Returned)
      }{
         The character pointer.
      }
   }
}
\sstroutine{
   CHR\_TRCHR
}{
   Translate the specified characters in a string
}{
   \sstdescription{
      Translate a specified set of characters within a string. The
      character translation is controlled by the translation table
      given by the character strings FROM and TO. Any characters
      not appearing in the translation table are left unchanged.
      If the status is set on entry, no action is taken. If the
      strings FROM and TO are unequal in length, STATUS is returned
      set to SAI\_\_ERROR.
   }
   \sstinvocation{
      CALL CHR\_TRCHR( FROM, TO, STRING, STATUS )
   }
   \sstarguments{
      \sstsubsection{
         FROM = CHARACTER $*$ ( $*$ ) (Given)
      }{
         A string specifying the characters to be translated.
      }
      \sstsubsection{
         TO = CHARACTER $*$ ( $*$ ) (Given)
      }{
         A string specifying the translation values for each of the
         characters in the FROM argument. The lengths of the FROM and
         TO arguments must be the same.
      }
      \sstsubsection{
         STRING = CHARACTER $*$ ( $*$ ) (Given and Returned)
      }{
         The string to be translated. Any character matching one of the
         characters specified in the FROM argument is converted to the
         corresponding character specified in the TO argument. All
         other characters are left unchanged.
      }
      \sstsubsection{
         STATUS = INTEGER (Given and Returned)
      }{
         The global status: returned set to SAI\_\_ERROR if FROM and TO
         have unequal lengths.
      }
   }
}
\sstroutine{
   CHR\_TRUNC
}{
   Truncate a string at a given delimiter
}{
   \sstdescription{
      The given string is truncated at the first occurrence of the
      given delimiter character. The delimiter character and all
      subsequent characters are replaced by blanks. If no delimiter
      character is found in the string, no truncation takes place.
      This routine is effectively a combination of INDEX and CHR\_TERM.
   }
   \sstinvocation{
      CALL CHR\_TRUNC( DELIM, STRING )
   }
   \sstarguments{
      \sstsubsection{
         DELIM = CHARACTER $*$ 1 (Given)
      }{
         The truncation delimiter character.
      }
      \sstsubsection{
         STRING = CHARACTER $*$ ( $*$ ) (Given and Returned)
      }{
         The string to be truncated. All characters from, and
         including, the first occurrence of DELIM will be
         replaced with blanks.
      }
   }
}
\sstroutine{
   CHR\_UCASE
}{
   Convert a string to uppercase
}{
   \sstdescription{
      The characters in the string are all converted to uppercase
      in situ.
   }
   \sstinvocation{
      CALL CHR\_UCASE( STRING )
   }
   \sstarguments{
      \sstsubsection{
         STRING = CHARACTER $*$ ( $*$ ) (Given and Returned)
      }{
         The string to be converted to uppercase.
      }
   }
}
\sstroutine{
   CHR\_UPPER
}{
   Return uppercase equivalent of a character
}{
   \sstdescription{
      If the given character is lowercase, the uppercase equivalent
      is returned, otherwise the character will be returned unchanged.
   }
   \sstinvocation{
      RESULT = CHR\_UPPER( CVALUE )
   }
   \sstarguments{
      \sstsubsection{
         CVALUE = CHARACTER $*$ 1 (Given)
      }{
         The character to be converted.
      }
   }
   \sstreturnedvalue{
      \sstsubsection{
         CHR\_UPPER = CHARACTER $*$ 1
      }{
         Uppercase equivalent of the given character, if the given
         character is a lowercase letter; otherwise the character is
         returned unchanged.
      }
   }
}
\sstroutine{
   CHR\_WILD
}{
   Return whether a string matches a wild-card pattern
}{
   \sstdescription{
      A candidate string is matched with a another character
      string containing a pattern of characters and wild-card
      characters.  The wild-cards used are:

         \% a single character wild-card;
         $*$ an arbitrary length string wild-card, including zero length.

      There is also a literal escape character \qt{$\backslash$} for use when the
      characters \qt{$*$} and \qt{\%} are to be interpreted literally within
      the wild-card pattern.
   }
   \sstinvocation{
      RESULT = CHR\_WILD( STRING, WILDS, MATCH )
   }
   \sstarguments{
      \sstsubsection{
         STRING = CHARACTER $*$ ( $*$ ) (Given)
      }{
         The candidate string to be matched.
      }
      \sstsubsection{
         WILDS = CHARACTER $*$ ( $*$ ) (Given)
      }{
         The wild-card pattern to be used in the match.
      }
      \sstsubsection{
         MATCH = CHARACTER $*$ ( $*$ ) (Returned)
      }{
         The wild-card match: this string must be the same length as
         STRING. All characters matched individually are returned as
         blanks in MATCH, and all characters matched by wild-cards are
         returned assigned to the particular wild-cards they matched.
         If the length of MATCH is less than that of STRING, then
         CHR\_WILD returns the value .FALSE.
      }
   }
   \sstreturnedvalue{
      \sstsubsection{
         CHR\_WILD = LOGICAL
      }{
         Whether the two strings match after expanding the wild-card
         pattern.
      }
   }
}
\end{small}




\section {C Function Descriptions \label{c_funcs}}

\subsection{Overview}
Equivalent functions written entirely in C are available for the
following subset of the CHR routines. These are provided mainly for use
when porting Fortran application code to C.

The C and Fortran implementations are completely independent of each
other - no ``wrapper'' layer is involved. This is done to avoid
efficiency problems caused by the continual copying of strings between
Fortran and C that would be required if the C interface was provided by a
wrapper layer on top of the Fortran coide (or vice versa). The behaviour
of C functions is slightly different to that of the Fortran routines
because the two languages index strings in different ways (C is
zero-based, Fortran is one-based). Also C has no equivalent to the
Fortran intrinsic function LEN, and so each C function that returns a
string has an extra argument specifying the length of the buffer reserved
for the output string.

\newpage
\sstroutine{
   chrAppnd
}{
   Copy one string into another, ignoring trailing blanks
}{
   \sstdescription{
      The string \texttt{"}str1\texttt{"} (or as much of it as there is room for) is copied
      into the part of \texttt{"}str2\texttt{"} beginning at position \texttt{"}iposn\texttt{"}. \texttt{"}iposn\texttt{"} is
      updated to indicate the final length of \texttt{"}str2\texttt{"} after this operation.
      Trailing blanks in \texttt{"}str1\texttt{"} are ignored.
   }
   \sstinvocation{
      void chrAppnd( const char $*$str1, char $*$str2, size\_t str2\_length,
                     size\_t $*$iposn )
   }
   \sstnotes{
      If the output string is too small to append the entire input
      string, truncation will occur and \texttt{"}iposn\texttt{"} will be set to the total
      length of \texttt{"}str2\texttt{"} (excluding the trailing null).
   }
   \sstdiylist{
      Parameters
   }{
      \sstsubsection{
         str1
      }{
         Pointer to a null terminated string holding the string to be
         copied.
      }
      \sstsubsection{
         str2
      }{
         Pointer to a null terminated string holding the string to be
         updated.
      }
      \sstsubsection{
         str2\_length
      }{
         The maximum length of the \texttt{'} str2\texttt{'}  string. This should include
         room for the terminating null.
      }
      \sstsubsection{
         $*$iposn
      }{
         The used length of the \texttt{"}str2\texttt{"}. If \texttt{"}str2\texttt{"} is empty on entry, then
         this should be supplied as zero. On exit, the supplied value
         is incremented by the length of \texttt{"}str1\texttt{"} (ignoring any trailing
         spaces).
      }
   }
}
\sstroutine{
   chrClean
}{
   Remove all unprintable characters from a string
}{
   \sstdescription{
      Replace all unprintable characters in the given string with blanks.
   }
   \sstinvocation{
      void chrClean( char $*$string )
   }
   \sstdiylist{
      Parameters
   }{
      \sstsubsection{
         string
      }{
         Pointer to a null terminated string holding the string to be
         cleaned.
      }
   }
}
\sstroutine{
   chrCtod
}{
   Read a double value from a string
}{
   \sstdescription{
      Read a double value from the given character string.
   }
   \sstinvocation{
      void chrCtod( const char $*$string, double $*$dvalue, int $*$status )
   }
   \sstdiylist{
      Parameters
   }{
      \sstsubsection{
         string
      }{
         Pointer to a null terminated string holding the string from which
         a double value is to be read.
      }
      \sstsubsection{
         $*$dvalue
      }{
         Returned holding the resulting double value.
      }
      \sstsubsection{
         $*$status
      }{
         The status value: if this value is not SAI\_\_OK on input, the
         function returns without action; if the function does not complete
         successfully, \texttt{"}status\texttt{"} is returned set to SAI\_\_ERROR.
      }
   }
}
\sstroutine{
   chrCtoi
}{
   Read an integer value from a string
}{
   \sstdescription{
      Read an integer value from the given character string.
   }
   \sstinvocation{
      void chrCtoi( const char $*$string, int $*$ivalue, int $*$status )
   }
   \sstdiylist{
      Parameters
   }{
      \sstsubsection{
         string
      }{
         Pointer to a null terminated string holding the string from which
         an integer value is to be read.
      }
      \sstsubsection{
         $*$ivalue
      }{
         Returned holding the resulting integer value.
      }
      \sstsubsection{
         $*$status
      }{
         The status value. If this value is not SAI\_\_OK on input, the
         function returns without action; if the function does not complete
         successfully, \texttt{"}status\texttt{"} is returned set to SAI\_\_ERROR.
      }
   }
}
\sstroutine{
   chrCtor
}{
   Read a float value from a string
}{
   \sstdescription{
      Read a float value from the given character string.
   }
   \sstinvocation{
      void chrCtor( const char $*$string, float $*$rvalue, int $*$status )
   }
   \sstdiylist{
      Parameters
   }{
      \sstsubsection{
         string
      }{
         Pointer to a null terminated string holding the string from which
         a float value is to be read.
      }
      \sstsubsection{
         $*$rvalue
      }{
         Returned holding the resulting float value.
      }
      \sstsubsection{
         $*$status
      }{
         The status value: if this value is not SAI\_\_OK on input, the
         function returns without action; if the function does not complete
         successfully, \texttt{"}status\texttt{"} is returned set to SAI\_\_ERROR.
      }
   }
}
\sstroutine{
   chrFandl
}{
   Find the first and last non-blank characters in a string
}{
   \sstdescription{
      Find the indices of the first and last non-blank characters in the
      given string. If the string contains no non-blank characters, the first
      index is returned set to 1 and last index is returned set to 0, i.e.
      \texttt{"}index1\texttt{"} is greater than \texttt{"}index2\texttt{"}.
   }
   \sstinvocation{
      void chrFandl( const char $*$string, size\_t $*$index1, size\_t $*$index2 )
   }
   \sstnotes{
      \sstitemlist{

         \sstitem
          For consistency with the Fortran routine CHR\_FANDL, this function
         only checks for spaces. Other forms of whitespace characters such
         as tabs, line-feeds, etc are considered to be non-blank.
      }
   }
   \sstdiylist{
      Parameters
   }{
      \sstsubsection{
         string
      }{
         Pointer to a null terminated string holding the character string.
      }
      \sstsubsection{
         $*$index1
      }{
         Returned holding the zero-based position of first non-blank character.
      }
      \sstsubsection{
         $*$index2
      }{
         Returned holding the zero-based position of last non-blank character.
      }
   }
}
\sstroutine{
   chrFill
}{
   Fill a string with a given character
}{
   \sstdescription{
      The given character string is filled with the specified character.
   }
   \sstinvocation{
      void chrFill( char cvalue, char $*$string, size\_t string\_length )
   }
   \sstdiylist{
      Parameters
   }{
      \sstsubsection{
         cvalue
      }{
         The character specified to fill the string.
      }
      \sstsubsection{
         string
      }{
         Pointer to an array in which to return a null terminated string
         holding the string to be filled.
      }
      \sstsubsection{
         string\_length
      }{
         The declared length of the supplied \texttt{'} string\texttt{'}  array. This should
         include room for the terminating null.
      }
   }
}
\sstroutine{
   chrFparx
}{
   Find a parenthesised expression in a character string
}{
   \sstdescription{
      This function searches the string \texttt{"}str\texttt{"} to identify a sub-string
      containing a parenthesised expression and returns the character
      positions of the opening and closing parentheses in the \texttt{"}f\texttt{"} and \texttt{"}l\texttt{"}
      arguments. Allowance is made for nested parentheses. If a
      parenthesised expression was not found, then the returned value of
      \texttt{"}f\texttt{"} will be greater than the returned value of \texttt{"}l\texttt{"}.
   }
   \sstinvocation{
      void chrFparx( const char $*$str, char oppar, char clpar, size\_t $*$f,
                     size\_t $*$l )
   }
   \sstdiylist{
      Parameters
   }{
      \sstsubsection{
         str
      }{
         Pointer to a null terminated string holding the string to be
         searched.
      }
      \sstsubsection{
         oppar
      }{
         The opening parenthesis character.
      }
      \sstsubsection{
         clpar
      }{
         The closing parenthesis character.
      }
      \sstsubsection{
         $*$f
      }{
         Returned holding the zero-based character position of the opening
         parenthesis.
      }
      \sstsubsection{
         $*$l
      }{
         Returned holding the zero-based character position of the closing
         parenthesis.
      }
   }
}
\sstroutine{
   chrIsalm
}{
   Return whether a character is alphanumeric
}{
   \sstdescription{
      Determine whether a character is alphanumeric, i.e. A - Z, a - z, 0 -
      9 or \_. Note that this function treats the underscore character as an
      alphanumeric character.
   }
   \sstinvocation{
      int chrIsalm( char cvalue )
   }
   \sstreturnedvalue{
      \sstsubsection{
         Returns non-zero if the given character is alphanumeric,
      }{
      }
   }
   \sstdiylist{
      Parameters
   }{
      \sstsubsection{
         cvalue
      }{
         The character to be tested.
      }
   }
}
\sstroutine{
   chrIsnam
}{
   Return whether a string is a valid name
}{
   \sstdescription{
      Determine whether the given string is a valid name: i.e. whether it
      starts with an alphabetic character and continues with alphanumeric
      or underscore characters.
   }
   \sstinvocation{
      int chrIsnam( const char $*$string )
   }
   \sstreturnedvalue{
      \sstsubsection{
         Returns non-zero if the given string is a valid name, returns
      }{
      }
   }
   \sstdiylist{
      Parameters
   }{
      \sstsubsection{
         string
      }{
         Pointer to a null terminated string holding the string to be
         tested.
      }
   }
}
\sstroutine{
   chrItoc
}{
   Encode an integer value as a string
}{
   \sstdescription{
      Encode an integer value as a (decimal) character string, using as
      concise a format as possible, and return the number of characters
      used. In the event of an error, \texttt{"}$*$\texttt{'} s will be written into to the
      string.
   }
   \sstinvocation{
      void chrItoc( int ivalue, char $*$string, size\_t string\_length,
                    size\_t $*$nchar )
   }
   \sstdiylist{
      Parameters
   }{
      \sstsubsection{
         ivalue
      }{
         The value to be encoded.
      }
      \sstsubsection{
         string
      }{
         Pointer to an array in which to return a null terminated string
         holding the string into which the integer value is encoded.
      }
      \sstsubsection{
         string\_length
      }{
         The maximum length of the supplied \texttt{'} string\texttt{'}  array. This should
         include room for the terminating null.
      }
      \sstsubsection{
         $*$nchar
      }{
         Returned holding the field width used in encoding the value.
      }
   }
}
\sstroutine{
   chrLdblk
}{
   Remove any leading blanks from a string
}{
   \sstdescription{
      Remove any leading blanks from the character string. The remaining
      characters are moved to the left to eliminate the resulting empty
      space, and a terminating null is appended to the end.
   }
   \sstinvocation{
      void chrLdblk( char $*$string )
   }
   \sstdiylist{
      Parameters
   }{
      \sstsubsection{
         string
      }{
         Pointer to a null terminated string holding the string from which
         the leading blanks are to be removed.
      }
   }
}
\sstroutine{
   chrLen
}{
   Return the length of a string, ignoring trailing spaces
}{
   \sstdescription{
      Find length of string, ignoring trailing spaces.
   }
   \sstinvocation{
      size\_t chrLen( const char $*$string )
   }
   \sstreturnedvalue{
      \sstsubsection{
         Returns the length of the string, not including the
      }{
      }
      \sstsubsection{
         terminating null or any trailing spaces.
      }{
      }
   }
   \sstdiylist{
      Parameters
   }{
      \sstsubsection{
         string
      }{
         Pointer to a null terminated string holding the string whose
         length is to be determined.
      }
   }
}
\sstroutine{
   chrPutc
}{
   Put a character string into another at a given position
}{
   \sstdescription{
      The string \texttt{"}str1\texttt{"} (or as much of it as there is room for) is copied
      into the part of \texttt{"}str2\texttt{"} beginning at position \texttt{"}iposn$+$1\texttt{"}. \texttt{"}iposn\texttt{"} is
      updated to indicate the end position of the copy of \texttt{"}str1\texttt{"} within
      \texttt{"}str2\texttt{"} after this operation. If no copying is done, \texttt{"}iposn\texttt{"} is
      returned unchanged.
   }
   \sstinvocation{
      void chrPutc( const char $*$str1, char $*$str2, size\_t str2\_length,
                    size\_t $*$iposn )
   }
   \sstdiylist{
      Parameters
   }{
      \sstsubsection{
         str1
      }{
         Pointer to a null terminated string holding the string to be
         copied.
      }
      \sstsubsection{
         str2
      }{
         Pointer to a null terminated string holding the string into which
         \texttt{"}str1\texttt{"} is to be copied.
      }
      \sstsubsection{
         str2\_length
      }{
         The declared length of the supplied \texttt{'} str2\texttt{'}  array. This should include
         room for the terminating null.
      }
      \sstsubsection{
         $*$iposn
      }{
         The zero-based position pointer within \texttt{"}str2\texttt{"}.
      }
   }
}
\sstroutine{
   chrPuti
}{
   Put an integer value into a string at a given position
}{
   \sstdescription{
      The integer value is encoded into a concise string which is then
      copied into the given string beginning at position \texttt{"}iposn$+$1\texttt{"}. \texttt{"}iposn\texttt{"}
      is returned updated to indicate the end position of the encoded
      number within \texttt{"}string\texttt{"}. This is a combination of chrItoc and chrPutc.
   }
   \sstinvocation{
      void chrPuti( int ivalue, char $*$string, size\_t string\_length,
                    size\_t $*$iposn )
   }
   \sstdiylist{
      Parameters
   }{
      \sstsubsection{
         ivalue
      }{
         The value to be encoded into the string.
      }
      \sstsubsection{
         string
      }{
         Pointer to a null terminated string holding the string into which
         \texttt{"}ivalue\texttt{"} is to be copied.
      }
      \sstsubsection{
         string\_length
      }{
         The declared length of the supplied \texttt{'} string\texttt{'}  array. This should
         include room for the terminating null.
      }
      \sstsubsection{
         $*$iposn
      }{
         The zero-based position pointer within \texttt{"}string\texttt{"}.
      }
   }
}
\sstroutine{
   chrRmblk
}{
   Remove all blanks from a string
}{
   \sstdescription{
      All leading and embedded blanks in the string are removed. The
      remaining characters are moved to the left to eliminate the resulting
      empty space, and a terminating null is appended to the end of the
      string.
   }
   \sstinvocation{
      void chrRmblk( char $*$string )
   }
   \sstdiylist{
      Parameters
   }{
      \sstsubsection{
         string
      }{
         Pointer to a null terminated string holding the string from which
         all leading and embedded blanks are removed.
      }
   }
}
\sstroutine{
   chrSimlr
}{
   Return whether two strings are equal, apart from case
}{
   \sstdescription{
      Determine whether two strings are the same, ignoring distinctions
      between upper and lowercase letters. Their lengths must be identical
      after removing trailing blanks.
   }
   \sstinvocation{
      int chrSimlr( const char $*$str1, const char $*$str2 )
   }
   \sstreturnedvalue{
      \sstsubsection{
         Returned as non-zero if the two strings are the same
      }{
      }
   }
   \sstdiylist{
      Parameters
   }{
      \sstsubsection{
         str1
      }{
         Pointer to a null terminated string holding the first string.
      }
      \sstsubsection{
         str2
      }{
         Pointer to a null terminated string holding the second string.
      }
   }
}
\sstroutine{
   chrSimlrN
}{
   Return whether the starts of two strings are equal, apart from case
}{
   \sstdescription{
      Determine whether the first \texttt{"} n\texttt{"}  characters of two strings are the
      same, ignoring distinctions between upper and lowercase letters.
   }
   \sstinvocation{
      int chrSimlrN( const char $*$str1, const char $*$str2, size\_t n )
   }
   \sstreturnedvalue{
      \sstsubsection{
         Returned as non-zero if the two strings are the same
      }{
      }
   }
   \sstdiylist{
      Parameters
   }{
      \sstsubsection{
         str1
      }{
         Pointer to a null terminated string holding the first string.
      }
      \sstsubsection{
         str2
      }{
         Pointer to a null terminated string holding the second string.
      }
      \sstsubsection{
         n
      }{
         The number of characters that must match at the start of each
         string. If the length of either string is less than \texttt{"} n\texttt{"} , zero
         will be returned.
      }
   }
}
\sstroutine{
   chrSizetoc
}{
   Encode a size\_t value as a string
}{
   \sstdescription{
      Encode a size\_t value as a (decimal) character string, using as
      concise a format as possible, and return the number of characters
      used. In the event of an error, \texttt{"} $*$\texttt{'} s will be written into to the
      string.
   }
   \sstinvocation{
      void chrSizetoc( size\_t value, char $*$string, size\_t string\_length,
                       size\_t $*$nchar )
   }
   \sstdiylist{
      Parameters
   }{
      \sstsubsection{
         value
      }{
         The value to be encoded.
      }
      \sstsubsection{
         string
      }{
         Pointer to an array in which to return a null terminated string
         holding the string into which the integer value is encoded.
      }
      \sstsubsection{
         string\_length
      }{
         The maximum length of the supplied \texttt{'} string\texttt{'}  array. This should
         include room for the terminating null.
      }
      \sstsubsection{
         $*$nchar
      }{
         Returned holding the field width used in encoding the value.
      }
   }
}
\sstroutine{
   chrUcase
}{
   Convert a string to uppercase
}{
   \sstdescription{
      The characters in the string are all converted to uppercase in situ.
   }
   \sstinvocation{
      void chrUcase( char $*$string )
   }
   \sstdiylist{
      Parameters
   }{
      \sstsubsection{
         string
      }{
         Pointer to a null terminated string holding the string to be
         converted to uppercase.
      }
   }
}

\newpage

\section {Portability}

\subsection {Overview}

This section discusses the portability of CHR, including the coding
standard adopted for CHR and a list of those routines which
may need to be modified when porting CHR to a new target machine.


\subsection {Coding and porting prerequisites}

The standard of Fortran used for the coding of CHR is fundamentally
Fortran 77, using the Starlink Fortran coding conventions described in SGP/16.
Several common extensions to the Fortran 77 standard are used in source
code for CHR, they are as follows:

\begin {itemize}
\item End-of-line comments using the ``!'' symbol;
\item Symbolic subprogram names may be longer than six characters (but are
always shorter than ten characters);
\item Symbolic subprogram names include the ``\_'' symbol;
\item Symbolic constant names may be longer than six characters (but are always
shorter than eleven characters);
\item Symbolic constant names may include the ``\_'' symbol;
\item The full ASCII character set is assumed in character constants.
\end {itemize}

The CHR library currently has no dependence upon any other package.
To use CHR on any computer system porting effort is required only for those
CHR routines which have operating system dependencies: these routines are
listed in the following section.


\subsection {Operating system specific routines}

Several CHR routines make use of features specific to the operating system or
language implementation.
The names of these routines and their purpose are as follows:

\begin {quote}
\begin{description}
\item [CHR\_ACHR] -- Return the character for a given ASCII value.
\item [CHR\_ATOK] -- Return the character for a given ASCII character token.
\item [CHR\_ATOM] -- Translate a string from ASCII to machine's character set.
\item [CHR\_BTOI] -- Read an INTEGER value from a binary string.
\item [CHR\_HTOI] -- Read an INTEGER value from a hexadecimal string.
\item [CHR\_IACHR] -- Return the ASCII value for a given character.
\item [CHR\_ITOB] -- Encode an INTEGER value as a binary string.
\item [CHR\_ITOH] -- Encode an INTEGER value as a hexadecimal string.
\item [CHR\_ITOO] -- Encode an INTEGER value as an octal string.
\item [CHR\_LOWER] -- Return the lower case equivalent of a character.
\item [CHR\_MTOA] -- Translate a string from machine's character set to ASCII.
\item [CHR\_OTOI] -- Read an integer from an octal string.
\item [CHR\_UPPER] -- Return the upper-case equivalent of a character.
\end {description}
\end {quote}

These routines may have to be rewritten specifically for each new
operating system and Fortran implementation.


\section {Changes and New Features in Version 2.0}

\subsection {Obsolete routines}

Several routines exist in CHR purely for historical reasons.
These routines are obsolete and are only provided by the CHR library because
existing Starlink code may depend upon them.
They should not be used in any new software.

\begin{description}
\item [CHR\_EQUAL] -- Return whether two strings are equal.
\item [CHR\_INDEX] -- Return the index of a substring within a string.
\item [CHR\_MOVE] -- Move one string into another.
\item [CHR\_SIZE] -- Return the declared length of a string.
\end {description}


\subsection {Changes in behaviour of existing routines}

\begin{description}
\item [CHR\_DTOC] -- The former algorithm has been replaced -- this may result
in the returned string being different from that returned from the previous
version.
\item [CHR\_RTOC] -- The former algorithm has been replaced -- this may result
in the returned string being different from that returned from the previous
version.
\end {description}


\subsection {New routines}

\begin{description}
\item [CHR\_ABBRV] -- Return whether two strings are equal apart from case,
permitting abbreviation.
\item [CHR\_ACHR] -- Return the character for a given ASCII value.
\item [CHR\_ATOK] -- Return the character for a given ASCII character token.
\item [CHR\_ATOM] -- Translate a string from ASCII to machine's characters set.
\item [CHR\_BTOI] -- Read an INTEGER value from a binary string.
\item [CHR\_DTOAN] -- Write a DOUBLE PRECISION value into a string as
hr/deg:min:sec.
\item [CHR\_FIND] -- Find the next occurrence of a given substring within a
string.
\item [CHR\_IACHR] -- Return the ASCII value for a given character.
\item [CHR\_ITOB] -- Encode an INTEGER value as a binary string.
\item [CHR\_ITOH] -- Encode an INTEGER value as a hexadecimal string.
\item [CHR\_ITOO] -- Encode an INTEGER value as an octal string.
\item [CHR\_LINBR] -- Break a line of text into a sequence of shorter lines.
\item [CHR\_MTOA] -- Translate a string from machine's characters set to ASCII.
\item [CHR\_NTH] -- Return the two-character abbreviation for a specified
integer.
\item [CHR\_PFORM] -- Reformat a paragraph to a new width.
\item [CHR\_PREFX] -- Prefix a string with a substring.
\item [CHR\_RJUST] -- Right-justify a string.
\item [CHR\_RMCHR] -- Remove all specified characters from a string.
\item [CHR\_SCOMP] -- Compare two character strings using the ASCII character
set.
\item [CHR\_SKCHR] -- Skip over all specified characters in a string.
\item [CHR\_SORT] -- Sort an array of character strings into alphabetical order.
\item [CHR\_TOCHR] -- Skip to the next specified character in a string.
\item [CHR\_TRCHR] -- Translate the specified characters in a string.
\item [CHR\_WILD] -- Return whether a string matches a wild-card pattern.
\end {description}


\subsection {Other changes}

\begin{description}
\item [CHR\_DIR:LOGICAL.COM] This file has been renamed CHR\_DIR:CHR\_DEV.COM.
\item [CHR\_DIR:CHRLINK.OPT] This file has been renamed CHR\_DIR:CHR\_LINK.OPT.
\end {description}

\section {Changes and New Features in Version 2.2}
\subsection {Changes in behaviour of existing routines}
The algorithm used in CHR\_RTOC and CHR\_DTOC has been changed --
this may result in the returned string being different from that returned
from the previous versions.

\subsection{Documentation Changes}
This document has been modified to remove references to the use of CHR on VMS.

\section {Changes and New Features in Version 3.0}
C equivalents for a subset of the Fortran routines have been added to the
CHR library to assist in porting Fortran code to C.

\end{document}

