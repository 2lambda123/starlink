\documentclass[oneside,11pt]{starlink}

\usepackage{rotating}

% -----------------------------------------------------------------------------
% Starlink set up

% ? Document identification
% Fixed part
%\title{SUN/229}
\stardoccategory  {Starlink User Note}
\stardocsource     {sun\stardocnumber}
\stardocinitials    {SUN}

% Variable part - replace [xxx] as appropriate.
\stardocnumber   {229.2}
\stardocauthors   {Tim Jenness, Remo Tilanus, \\ Horst Meyerdierks, Jon Fairclough}
\stardocdate       {30 November 2014}
\stardoctitle        {The Global Section Datafile (GSD)  access library}
\stardocversion  {1.0}
\stardocmanual    {Programmer's Manual}
\stardocabstract  {%
This document describes the Global Section Datafile (GSD) access library.
This library provides read-only access to GSD files created at the
James Clerk Maxwell Telescope. A description of GSD itself is presented
in addition to descriptions of the library routines.
}
\starfunders{Particle Physics and Astronomy Research Council}

%%\stardocname{\stardocinitials /\stardocnumber}

% -----------------------------------------------------------------------------

\begin{document}

\scfrontmatter

% ? Main text

\section{History}

The Global Section Datafile (GSD) subroutine library package was written in
1987 by Jon Fairclough \cite{F89} to permit the fast reading and writing of
data at the James Clerk Maxwell Telescope. Development was stimulated by the
need to provide fast filing of data in the so-called ``General Single Dish
Data'' (GSDD) format developed by MRAO, IRAM and NRAO. The JCMT used the GSD
format for data storage from all instrumentation until the arrival of SCUBA
\cite{scuba} in 1996 (which uses NDF \cite{ndf}) and data files in this format
will continue to be written by the heterodyne system until the delivery of
ACSIS in 2001.

The original GSD I/O library was written in VAX Fortran and has never been
ported to a Unix environment. With the move from VMS to Unix in 1994/1995 it
was clear that a version of the GSD library was required that would be able to
read GSD files (those in the archive as well as new files) without having to
change the existing telescope acquisition system or require the use of a VMS
application to convert the format on demand. A read-only version of the
library was written in C by Remo Tilanus and Horst Meyerdierks in 1994 and was
incorporated into the Starlink releases of JCMTDR \cite{jcmtdr} and SPECX
\cite{specx}.

\section{Introduction}

The JCMT uses the GSD file format for its current heterodyne acquisition
system (via the Dutch Autocorrelation Spectrometer (DAS)) and for its archive
of pre-SCUBA data.  This document describes the C version of the GSD library
(a FORTRAN interface is layered on top), now distributed as a standalone
package and not part of an application. The current library cannot be used for
creating or modifying GSD files.


\section{C interface}

The fundamental calling interface is from C and this is documented in appendix
\ref{app:desc}. Routine prototypes can be found in the \texttt{gsd.h} include
file. An example C program that lists the contents of a GSD file
(\texttt{gsdprint}) is provided in the distribution.

\section{Fortran interface}

A Fortran interface is provided that uses the original names of the
subroutines rather than C function names. For example, the C routine
\texttt{gsdOpenRead} should be called from Fortran as \texttt{gsd\_open\_read}.
The Fortran binding is incomplete (only covering the supplied C routines) and
existing Fortran code may have to be changed before this library can be
used. An example routine (\texttt{gsd\_print.f}) is provided to demonstrate
the interface. Additional changes:

\begin{itemize}
\item The Fortran include file is now called \texttt{GSD\_PAR} to fit in with
the Starlink naming convention (as opposed to \texttt{GSDPARS} in the original
VAX library).

\item \texttt{GSD\_PAR} is incomplete. Prior inclusion of \texttt{PRM\_PAR}
is required before \texttt{GSD\_PAR} can be included.

\item Error status values have changed. Zero is good status, non-zero is
   bad status, but no particular status value can be expected.

\item \texttt{gsd\_inquire\_array} is not implemented. Instead
\texttt{gsd\_inq\_size} must be used, although it was previously labelled
``obsolete''.

\end{itemize}

\section{Perl interface}

A Perl interface to the GSD library is available, but is not part of this
distribution. Please contact Tim Jenness (\texttt{t.jenness@jach.hawaii.edu})
for more information.

\section{Programming Notes}

This section describes some of the basic features of the library in
comparison with the VAX version:

\begin{itemize}
\item The library provides only read access.

\item Only VAX binary GSD files can be read.

\item Bad values in the file are converted to PRIMDAT bad
values\cite{primdat}, which differ from the traditional VAX/GSD bad
values. (This conversion is in memory only, the file itself is unchanged.)

\item Type conversion is possible only between numeric types (including
logical). Numeric to character or character to numeric conversion is not
provided by the library.

\end{itemize}

\section{Programming Tools}

The distribution comes with the following programming tools:

\begin{description}
\item[gsd\_link] \mbox{}

Link script used during the link phase to make sure that the correct
libraries are used:
\begin{quote}
\begin{verbatim}
f77 gsd_print.f -L/star/lib `gsd_link`
\end{verbatim}
\end{quote}

The math library (\texttt{-lm}) is required when using the C interface.

\end{description}

\section{Release Notes}

This section provides the release notes for the GSD package.

\begin{description}
\item[VAX implementation] \mbox{}

Implemented in VAX Fortran for the JCMT by Jon Fairclough (1987-1989).

\item[SpecxV6.7] \mbox{}

C read-only version released as part of Specx V6.7 in 1995.

\item[V1.0] \mbox{}

First version released to Starlink standalone. Unbundled from the
SPECX and JCMTDR distributions. First release for Linux.

\end{description}

\begin{thebibliography}{}
\addcontentsline{toc}{section}{References}

\bibitem{F89}
Fairclough~J.~H., 1989, {\it GSD -- Global Section Datafile System},
JCMT Note MT/IN/33

\bibitem{scuba}
Holland~W.~S., Robson~E.I., Gear~W.K., Lightfoot~J.~F., Jenness~T.,
Ivison~R.~J., Stevens~J.~A., Cunningham~C.~R., Ade~P.~A.~R.,
Griffin~M.~J., Duncan~W.~D., Murphy~J.~A., Naylor~D.~A., 1999,
\textit{MNRAS}, \textbf{303}, 659

\bibitem{ndf}
Warren-Smith~R.~F., 1998, {\it NDF -- Routines for Accessing the Extensible
N-Dimensional Data Format}, \xref{Starlink User Note 33}{sun33}{}

\bibitem{jcmtdr}
Lightfoot~J.~F., Harrison~P.~A., Meyerdierks~H., 1995,
{\it JCMTDR -- Applications for reducing JCMT data},
\xref{Starlink User Note 132}{sun132}{}

\bibitem{specx}
Prestage~R.~M., Meyerdierks~H., Lightfoot~J.~F., 1995, {\it SPECX -- A
millimetre wave spectral reduction package},
\xref{Starlink User Note 17}{sun17}{}

\bibitem{primdat}
Warren-Smith~R.~F., 1995, {\it PRIMDAT -- processing of primitive numerical
data}, \xref{Starlink User Note 39}{sun39}{}

\end{thebibliography}


\appendix

\section{Technical Overview}

A GSD file has a fairly simple layout. It consists of a `prolog' followed by
`data'. The prolog describes the data and can be used for retrieving it.  The
prolog consists of a single ``file descriptor'' and ``item descriptors'', one
for each data item. The item descriptor locates the required item in the byte
stream. More information on the file structure can be found in \cite{F89}.

The outermost layer of routines is the Fortran binding. This is in
\texttt{gsd\_f77.c}. The routines can only be called from Fortran. They share
static external variables amongst themselves (but with no other routines) to
record references to up to 100 open GSD files. The calling code only needs to
keep the old file identifier returned by \texttt{gsd\_open\_read}. The routine
\texttt{gsd\_inquire\_array} is not implemented, \texttt{gsd\_inq\_size} must
be used instead.

The next inner layer is the external C binding. The C binding is similar to
the Fortran binding in that there is a one-to-one relationship between
routines. The C binding does not use inherited status, but returns a status as
the function value. Also, given scalar arguments are passed by value, not by
reference. An open GSD file is identfied by no less than four pointers, all of
which must be kept by the calling code.  The C binding consists of
\texttt{gsdOpenRead.c}, \texttt{gsdClose.c}, \texttt{gsdFind.c},
\texttt{gsdItem.c}, \texttt{gsdInqSize.c}, \texttt{gsdGet0x.c},
\texttt{gsdGet1x.c}.

The next inner layer contains the \texttt{gsd1\_} routines. There are three
routines used by \texttt{gsdOpenRead} to open the file and read its contents
into memory.  The fourth routine \texttt{gsd1\_getval} returns information
about and values of items to the caller. It retrieves this information from
the memory copy of the file as created by \texttt{gsdOpenRead}.

The innermost layer are the \texttt{gsd2\_} routines. There are the
\texttt{gsd2\_nativx} routines, which are used by the \texttt{gsd1\_}
routines. They convert VAX binary file contents to equivalent numbers in the
format of the local machine.  They also convert VAX/GSD bad values to
local/PRIMDAT bad values\cite{primdat}. Then there is the \texttt{gsd2\_copya}
routine, which is used by \texttt{gsd1\_getval} to convert from the data type
as copied from the file to the data type as required by the calling routine.

\section{Subroutine List}

\begin{description}
\item[\htmlref{gsdClose}{gsdClose}] \mbox{}

Close a GSD file

\item[\htmlref{gsdFind}{gsdFind}] \mbox{}

Find GSD item by name

% The funny htmlref is required because the sstroutine name
% includes < and > symbols
\item[\htmlref{gsdGet0x}{gsdGet0_SPMlt_t_SPMgt_}] \mbox{}

Get a scalar value from a GSD file

% similarly for gsdGet1x
\item[\htmlref{gsdGet1x}{gsdGet1_SPMlt_t_SPMgt_}] \mbox{}

Get an array from a GSD file

\item[\htmlref{gsdInqSize}{gsdInqSize}] \mbox{}

Inquire array size

\item[\htmlref{gsdItem}{gsdItem}] \mbox{}

Get GSD item by number

\item[\htmlref{gsdOpenRead}{gsdOpenRead}] \mbox{}

Open a GSD file for reading and map it

\end{description}


\section{Routine Descriptions\label{app:desc}}

This section describes the library interface available to C programmers.
The Fortran interface is similar, except the routine names are of the
form \texttt{GSD\_XXX} rather than \texttt{gsdXxx}.

\sstroutine{
   gsdClose
}{
   Close a GSD file
}{
   \sstdescription{
      This routine closes a GSD file opened previously with
      \texttt{gsdOpenRead}. It also releases the memory that
      \texttt{gsdOpenRead} allocated in connection to that file. For this
      purpose this routine must be given the standard C file pointer, the
      pointer to the GSD file descriptor, the pointer to the GSD item
      descriptors, and the pointer to the data buffer.
   }
   \sstinvocation{
      int gsdClose( FILE $*$fptr, void $*$file\_dsc, void $*$item\_dsc,
         char $*$data\_ptr);
   }
   \sstarguments{
      \sstsubsection{
         FILE $*$fptr (Given)
      }{
         The file descriptor for the GSD file to be closed.
      }
      \sstsubsection{
         void $*$file\_dsc (Given)
      }{
         The GSD file descriptor related to the file opened on fptr.
      }
      \sstsubsection{
         void $*$item\_dsc (Given)
      }{
         The array of GSD item descriptors related to the file opened on fptr.
      }
      \sstsubsection{
         char $*$data\_ptr (Given)
      }{
         The buffer with all the data from the GSD file opened on fptr.
      }
   }
   \sstreturnedvalue{
      \sstsubsection{
         int gsdClose();
      }{
         Status from fclose.
      }
   }
   \sstdiytopic{
      Prototype
   }{
      available via \#include {\tt "}gsd.h{\tt "}
   }
   \sstdiytopic{
      Copyright
   }{
      Copyright (C) 1986-1999 Particle Physics and Astronomy Research Council.
      All Rights Reserved.
   }
}
\sstroutine{
   gsdFind
}{
   Find GSD item by name
}{
   \sstdescription{
      This routine looks up the GSD item specified by its name and returns the
      number of the item. This routine also returns the unit string, the type
      specification and the array flag.
   }
   \sstinvocation{
      int gsdFind( void $*$file\_dsc, void $*$item\_dsc, char $*$name, int $*$itemno,
         char $*$unit, char $*$type, char $*$array );
   }
   \sstarguments{
      \sstsubsection{
         void $*$file\_dsc (Given)
      }{
         The GSD file descriptor related to the file opened on fptr.
      }
      \sstsubsection{
         void $*$item\_dsc (Given)
      }{
         The array of GSD item descriptors related to the file opened on fptr.
      }
      \sstsubsection{
         char $*$data\_ptr (Given)
      }{
         The buffer with all the data from the GSD file opened on fptr.
      }
      \sstsubsection{
         char $*$name (Given)
      }{
         The name of the item. This should be an array of 16 characters (char
         name[16]) and a null-terminated string.
      }
      \sstsubsection{
         int $*$itemno (Returned)
      }{
         The number of the item in the GSD file.
      }
      \sstsubsection{
         char $*$unit (Returned)
      }{
         The unit of the item. This should be an array of 11 characters (char
         name[11]) and will be a null-terminated string.
      }
      \sstsubsection{
         char $*$type (Returned)
      }{
         The data type of the item. This is a single character and one of
         B, L, W, I, R, D, C.
      }
      \sstsubsection{
         char $*$array (Returned)
      }{
         The array flag. This is a single character and true (false) if the
         item is (is not) and array.
      }
   }
   \sstreturnedvalue{
      \sstsubsection{
         int gsdFind();
      }{
         Status.
         \sstitemlist{

            \sstitem
             [1:] If the named item cannot be found.

            \sstitem
             [0:] Otherwise.
         }
      }
   }
   \sstdiytopic{
      Prototype
   }{
      available via \#include {\tt "}gsd.h{\tt "}
   }
   \sstdiytopic{
      Copyright
   }{
      Copyright (C) 1986-1999 Particle Physics and Astronomy Research Council.
      All Rights Reserved.
   }
}
\sstroutine{
   gsdGet0$<$t$>$
}{
   Get a scalar value from a GSD file
}{
   \sstdescription{
      This routine returns the value of a scalar GSD item. The item must be
      specified by the file desciptor, item descriptor array, data array and
      item number.

\begin{center}
\begin{tabular}{clll}
\hline $<$t$>$& $<$type$>$ &    Fortran &      GSD\\ \hline
       b &  char     & byte          &  byte\\
       l &  char     & logical$*$1   &  logical\\
       w &  short    & integer$*$2   &  word\\
       i &  int      & integer$*$4   &  integer\\
       r &  float    & real$*$4      &  real\\
       d &  double   & real$*$8      &  double \\
       c &  char[17] & character$*$16&  char\\ \hline\hline
\end{tabular}
\end{center}

      This routine will convert between numeric types (all but GSD type char).
      That is to say, the calling routine can request, say, an integer value
      by calling \texttt{gsdGet0i}, even if the item in the GSD file has a
      different numeric type, say real. C casting rules are applied, which may
      differ from Fortran truncation rules. No test for conversion errors is
      performed.
   }
   \sstinvocation{
      int gsdGet0\{blwirdc\}( void $*$file\_dsc, void $*$item\_dsc, char $*$data\_ptr,
         int itemno, $<$type$>$ $*$value );
   }
   \sstarguments{
      \sstsubsection{
         void $*$file\_dsc (Given)
      }{
         The GSD file descriptor.
      }
      \sstsubsection{
         void $*$item\_dsc (Given)
      }{
         The array of GSD item descriptors related to the GSD file.
      }
      \sstsubsection{
         char $*$data\_ptr (Given)
      }{
         The buffer with all the data from the GSD file.
      }
      \sstsubsection{
         int itemno (Given)
      }{
         The number of the item in the GSD file.
      }
      \sstsubsection{
         $<$type$>$ $*$value (Returned)
      }{
         The data value. For \texttt{gsdGet0c} value should be declared with length 17
         at least. The returned string is null-terminated in value[16].
      }
   }
   \sstreturnedvalue{
      \sstsubsection{
         int gsdGet0$<$t$>$();
      }{
         Status.
         \sstitemlist{

            \sstitem
             [1:] Failure to read the item value.

            \sstitem
             [2:] Numbered item cannot be found.

            \sstitem
             [3:] Item is not scalar.

            \sstitem
             [0:] Otherwise.
         }
      }
   }
   \sstdiytopic{
      Prototype
   }{
      available via \#include {\tt "}gsd.h{\tt "}
   }
   \sstdiytopic{
      Copyright
   }{
      Copyright (C) 1986-1999 Particle Physics and Astronomy Research Council.
      All Rights Reserved.
   }
}
\sstroutine{
   gsdGet1$<$t$>$
}{
   Get an array from a GSD file
}{
   \sstdescription{
      This routine returns the value of a scalar GSD item. The item must be
      specified by the file desciptor, item descriptor array, data array and
      item number.

\begin{center}
\begin{tabular}{clll}
\hline      $<$t$>$& $<$type$>$  &   Fortran &      GSD\\\hline
       b &  char     & byte           & byte\\
       l &  char     & logical$*$1    & logical\\
       w &  short    & integer$*$2    & word\\
       i &  int      & integer$*$4    & integer\\
       r &  float    & real$*$4       & real\\
       d &  double   & real$*$8       & double\\
       c &  char[16] & character$*$16 & char\\ \hline\hline
\end{tabular}
\end{center}

      This routine does not convert between types. If the type of the GSD item
      does not match the type of the routine, then it returns with an error.

      It is possible to get only part of the array. Although the part can be
      specified in terms of an N-dimensional array, this routine does not take
      a proper N-D section of the array. The caller can specify the start
      pixel in N dimensions and the end pixel in N dimensions. These two pixels
      will be converted to memory locations and all memory between the two is
      returned. This emulates the old GSD library. It is useful really only for
      parts of 1-D arrays, parts of rows, or single pixels.
   }
   \sstinvocation{
      int gsdGet1\{blwird\}( void $*$file\_dsc, void $*$item\_dsc, char $*$data\_ptr,
         int itemno, int ndims, int $*$dimvals, int $*$start, int $*$end,
         $<$type$>$ $*$values, int $*$actvals );
   }
   \sstarguments{
      \sstsubsection{
         void $*$file\_dsc (Given)
      }{
         The GSD file descriptor.
      }
      \sstsubsection{
         void $*$item\_dsc (Given)
      }{
         The array of GSD item descriptors related to the GSD file.
      }
      \sstsubsection{
         char $*$data\_ptr (Given)
      }{
         The buffer with all the data from the GSD file.
      }
      \sstsubsection{
         int itemno (Given)
      }{
         The number of the item in the GSD file.
      }
      \sstsubsection{
         int ndims (Given)
      }{
         The dimensionality the calling routine uses to specify the start and
         end elements.
      }
      \sstsubsection{
         int $*$dimvals (Given)
      }{
         The array of ndims dimensions (array sizes along each axis).
      }
      \sstsubsection{
         int $*$start (Given)
      }{
         The array indices for the first element.
      }
      \sstsubsection{
         int $*$end
      }{
         The array indices for the last element.
      }
      \sstsubsection{
         $<$type$>$ $*$value (Returned)
      }{
         The data values. The calling routine must make sure that sufficient
         memory is provided. Thus it must find out the data type and array size
         before calling this routine.
         If the data type is character, then the routine returns a byte
         buffer with all strings concatenated. There are no string
         terminators in the buffer and there is none at the end. Each
         string is 16 byte long and immediately followed by the next string.
      }
      \sstsubsection{
         int $*$actvals (Returned)
      }{
         The number of array values returned. This saves the caller to work out
         how many array elements correspond to start and end given the dimvals.
      }
   }
   \sstreturnedvalue{
      \sstsubsection{
         int gsdGet1$<$t$>$();
      }{
         Status.
         \sstitemlist{

            \sstitem
             [1:] Failure to read the item values.

            \sstitem
             [2:] Numbered item cannot be found.

            \sstitem
             [4:] Given start and end are inconsistent.

            \sstitem
             [0:] Otherwise.
         }
      }
   }
   \sstdiytopic{
      Prototype
   }{
      available via \#include {\tt "}gsd.h{\tt "}
   }
   \sstdiytopic{
      Copyright
   }{
      Copyright (C) 1986-1999 Particle Physics and Astronomy Research Council.
      All Rights Reserved.
   }
}
\sstroutine{
   gsdInqSize
}{
   Inquire array size
}{
   \sstdescription{
      This routine returns information about the specified array. Returned are
      the names and units of each dimension, the size along each dimension, and
      the overall size.
   }
   \sstinvocation{
      int gsdInqSize( void $*$file\_dsc, void $*$item\_dsc, char $*$data\_ptr,
         int itemno, int maxdims,
         char $*$$*$dimnames, char $*$$*$dimunits, int $*$dimvals,
         int $*$actdims, int $*$size );
   }
   \sstarguments{
      \sstsubsection{
         void $*$file\_dsc (Given)
      }{
         The GSD file descriptor.
      }
      \sstsubsection{
         void $*$item\_dsc (Given)
      }{
         The array of GSD item descriptors related to the GSD file.
      }
      \sstsubsection{
         char $*$data\_ptr (Given)
      }{
         The buffer with all the data from the GSD file.
      }
      \sstsubsection{
         int itemno (Given)
      }{
         The number of the item in the GSD file.
      }
      \sstsubsection{
         int maxdims (Given)
      }{
         The number of dimensions required and accommodated by the calling
         routine.
      }
      \sstsubsection{
         char $*$$*$dimnames (Returned)
      }{
         The names for each dimension. The calling routine must provide maxdims
         pointers to strings. It must also provide the space for the strings,
         16 bytes. See Notes for how to declare and pass dimnames.
      }
      \sstsubsection{
         char $*$$*$dimunits (Returned)
      }{
         The units for each dimension. The calling routine must provide maxdims
         pointers to strings. It must also provide the space for the strings,
         11 bytes. See Notes for how to declare and pass dimunits.
      }
      \sstsubsection{
         int $*$dimvals (Returned)
      }{
         The values for each dimension. The calling routine must provide an
         array of maxdims integers. This would probably be declared as
         int dimvals[MAXDIMS];
      }
      \sstsubsection{
         int $*$actdims (Returned)
      }{
         The actual number of dimensions. If actdims is less than maxdims, then
         only actims elements are returned in dimnames, dimunits, dimvals.
         Further elements declared by the caller are unchanged by this routine.
      }
      \sstsubsection{
         int $*$size (Returned)
      }{
         The total number of elements in the array.
      }
   }
   \sstreturnedvalue{
      \sstsubsection{
         int gsdInqSize();
      }{
         Status.
         \sstitemlist{

            \sstitem
             [1:] Failed to get a dimension value and name.

            \sstitem
             [2:] Numbered item does not exist.

            \sstitem
             [3:] Array has more dimensions than accommodated by calling routine.

            \sstitem
             [0:] Otherwise.
         }
      }
   }
   \sstdiytopic{
      Prototype
   }{
      available via \#include {\tt "}gsd.h{\tt "}
   }
   \sstdiytopic{
      Note
   }{
      The calling routine will probably allocate storage for dimension names by
      declaring a two-dimensional array. That is not suitable for passing to
      this routine though. The pointers to each string must be copied into an
      array of pointers. For example:
\begin{quote}
\texttt{%
         char  actual\_space[MAXDIMS][16]; \\
         char $*$pointr\_array[MAXDIMS];\\
         for ( i = 0; i $<$ MAXDIMS; i$++$ ) pointr\_array[i] = actual\_space[i];\\
         status = gsdInqSize( ..., pointr\_array, ... );
}
\end{quote}

      The reason why this call works but passing actual\_space does not work,
      is that \texttt{gsdInqSize} uses the given value as a char $*$$*$. So in this
      routine given[1] goes forward in memory by the size of a char $*$ or the
      number of bytes needed to store a pointer. actual\_space would need a
      step in memory by 16 bytes, i.e. the distance from one string to the
      next. The main routine knows about this, because it declared
      actual\_space \_and\_ pointr\_array.
} \sstdiytopic{
    Copyright
}{
      Copyright (C) 1986-1999 Particle Physics and Astronomy Research Council.
      All Rights Reserved.
}
}
\sstroutine{
   gsdItem
}{
   Get GSD item by number
}{
   \sstdescription{
      This routine looks up the GSD item specified by its number and returns
      the name of the item. This routine also returns the unit string, the type
      specification and the array flag.
   }
   \sstinvocation{
      int gsdItem( void $*$file\_dsc, void $*$item\_dsc, int itemno, char $*$name,
         char $*$unit, char $*$type, char $*$array );
   }
   \sstarguments{
      \sstsubsection{
         void $*$file\_dsc (Given)
      }{
         The GSD file descriptor related to the file opened on fptr.
      }
      \sstsubsection{
         void $*$item\_dsc (Given)
      }{
         The array of GSD item descriptors related to the file opened on fptr.
      }
      \sstsubsection{
         char $*$data\_ptr (Given)
      }{
         The buffer with all the data from the GSD file opened on fptr.
      }
      \sstsubsection{
         int itemno (Given)
      }{
         The number of the item in the GSD file.
      }
      \sstsubsection{
         char $*$name (Returned)
      }{
         The name of the item. This should be an array of 16 characters (char
         name[16]) and will be a null-terminated string.
      }
      \sstsubsection{
         char $*$unit (Returned)
      }{
         The unit of the item. This should be an array of 11 characters (char
         name[11]) and will be a null-terminated string.
      }
      \sstsubsection{
         char $*$type (Returned)
      }{
         The data type of the item. This is a single character and one of
         B, L, W, I, R, D, C.
      }
      \sstsubsection{
         char $*$array (Returned)
      }{
         The array flag. This is a single character and true (false) if the
         item is (is not) and array.
      }
   }
   \sstreturnedvalue{
      \sstsubsection{
         int gsdFind();
      }{
         Status.
         \sstitemlist{

            \sstitem
             [1:] If the named item cannot be found.

            \sstitem
             [0:] Otherwise.
         }
      }
   }
   \sstdiytopic{
      Prototype
   }{
      available via \#include {\tt "}gsd.h{\tt "}
   }
   \sstdiytopic{
      Copyright
   }{
      Copyright (C) 1986-1999 Particle Physics and Astronomy Research Council.
      All Rights Reserved.
   }
}
\sstroutine{
   gsdOpenRead
}{
   Open a GSD file for reading and map it
}{
   \sstdescription{
      This routine opens the named GSD file and reads its contents into memory.
      It returns a standard C file descriptor, a GSD file descriptor, a pointer
      to the array of GSD item descriptors, and a pointer to the collective
      data.

      This routine allocates memory to accommodate the GSD file descriptor, the
      GSD item descriptors, and the data from the GSD file. It also leaves the
      GSD file open. Any call to this routine must be matched with a call to
      \texttt{gsdClose} with the information returned by this routine. \texttt{gsdClose} will
      close the file and release the memory allocated by this routine.
   }
   \sstinvocation{
      int gsdOpenRead( char $*$file, float $*$version, char $*$label, int $*$no\_items,
         FILE $*$$*$fptr, void $*$$*$file\_dsc, void $*$$*$item\_dsc, char $*$$*$data\_ptr );
   }
   \sstarguments{
      \sstsubsection{
         char $*$file (Given)
      }{
         The name of the GSD file to be opened.
      }
      \sstsubsection{
         float $*$version (Returned)
      }{
         The GSD file version number.
      }
      \sstsubsection{
         char $*$label (Returned)
      }{
         The GSD file label. This is a null-terminated string. It should be
         declared by the calling routine with length 41.
      }
      \sstsubsection{
         int $*$no\_items (Returned)
      }{
         The number of items in the GSD file.
      }
      \sstsubsection{
         FILE $*$$*$fptr (Returned)
      }{
         The file descriptor for the GSD file opened.
      }
      \sstsubsection{
         void $*$$*$file\_dsc (Returned)
      }{
         The GSD file descriptor. This routine allocates the memory necessary
         and fills it with the relevant information from the GSD file. A call
         to \texttt{gsdClose} will release this memory (given the pointer).
      }
      \sstsubsection{
         void $*$$*$item\_dsc (Returned)
      }{
         The array of GSD item descriptors. This routine allocates the memory
         necessary and fills it with the relevant information from the GSD
         file. A call to \texttt{gsdClose} will release this memory (given the pointer).
         The number of array elements is returned in no\_items.
      }
      \sstsubsection{
         char $*$$*$data\_ptr (Returned)
      }{
         The buffer with all the data from the GSD file. This routine allocates
         the memory necessary and reads the data into it. A call to \texttt{gsdClose}
         will release this memory (given the pointer). The size of this buffer
         does not matter, but it can be calculated in bytes as
            file\_dsc-$>$end\_data - file\_dsc-$>$str\_data $+$ 1
         if you know what a struct file\_descriptor looks like.
      }
   }
   \sstreturnedvalue{
      \sstsubsection{
         int gsdOpenRead();
      }{
         Status. Status is set to
         \sstitemlist{

            \sstitem
             [1:] Failure to open named file,

            \sstitem
             [2:] Failure to read file\_dsc from file,

            \sstitem
             [3:] Failure to allocate memory for item\_dsc,

            \sstitem
             [4:] Failure to read item\_dsc from file,

            \sstitem
             [6:] Failure to read data\_ptr from file,

            \sstitem
             [7:] Failure to allocate memory for data\_ptr,

            \sstitem
             [0:] Otherwise.
         }
      }
   }
   \sstdiytopic{
      Prototype
   }{
      available via \#include {\tt "}gsd.h{\tt "}
   }
   \sstdiytopic{
      Copyright
   }{
      Copyright (C) 1986-1999 Particle Physics and Astronomy Research Council.
      All Rights Reserved.
   }
}

\clearpage
\section{Mapping}

The following tables define the mapping (in version 5.3 of the JCMT
storage task) from item names that will be found in the data files
(the ``NRAO'' name) to the JCMT name. Also included is the nominal
FITS header equivalent, emphasis indicating a JCMT-specific variant,
and a text description of the field.


\begin{small}
\begin{landscape}
\begin{sllongtable}{lllp{12cm}}{Mapping of GSD names to FITS equivalents.}
\hline
NRAO & JCMT & FITS & Description\\
\hline
\endhead
\hline
\endfoot

CELL\_V2Y & CELL\_V2Y & \emph{YPOSANG} & Position angle of cell y axis (CCW)\\
UAZ & SDIS(36) & UXPNT & User az correction\\
UEL & SDIS(37) & UYPNT & User el correction\\
C1BKE & BACKEND & BACKEND & Name of the backend\\
C1BTYP & BE\_TYPE & \emph{BACKTYPE} & Type of backend\\
C1DP & DATA\_PRECISION & PRECIS & Precision of the data from the backend\\
C1FTYP & FE\_TYPE & \emph{FRONTYPE} & Type of frontend\\
C1HGT & TEL\_HEIGHT & \emph{HEIGHT} & Height of telescope above sea level\\
C1IFS & IF\_DEVICE & IFDEVICE & Name of the IF device\\
C1LAT & TEL\_LATITUDE & \emph{LATITUDE} & Geodetic latitude of telescope (North +ve)\\
C1LONG & TEL\_LONGITUDE & \emph{LONGITUD} & Geographical longitude of telescope (West +ve)\\
C1OBS & PROJECT\_OBS\_1 & OBSID & Name of the primary observer\\
C1ONA & PROJECT\_OBS\_1 & OBSERVER & Name of the primary observer\\
C1ONA1 & PROJECT\_OBS\_2 & OBSERVER & Name of the support scientist\\
C1ONA2 & PROJECT\_OBS\_3 & \emph{OPERATOR} & Name of the telescope operator\\
C1PID & PROJECT & PROJID & Identifies the observing program\\
C1RCV & FRONTEND & FRONTEND & Name of the frontend\\
C1SNA1 & CENTRE\_NAME\_1 & OBJECT & Source name part 1\\
C1SNA2 & CENTRE\_NAME\_2 & \emph{OBJECT2} & Source name part 2 or altern. name\\
C1SNO & NOBS & SCAN & Observation number\\
C1STC & OBS\_TYPE & OBSTYPE & Type of observation\\
C1TEL & TEL\_NAME & TELESCOP & Telescope name\\
C2FL & DY & FOCUSL & Secondary mirror y displacement from nominal at observation start\\
C2FR & DZ & FOCUSR & Secondary mirror z displacement from nominal at observation start\\
C2FV & DX & FOCUSV & Secondary mirror x displacement from nominal at observation start\\
C2ORI & SECONDARY\_ORI & ORIENT & Rotation or polarization angle orientation of the frontend/reflector\\
C2PC1 & TEL\_PC\_LAN & PTCON1 & Angle by which lower axis is north of ideal\\
C2PC2 & TEL\_PC\_LAE & PTCON2 & Angle by which lower axis is east of ideal\\
C2PC3 & TEL\_PC\_UANP & PTCON3 & Angle by which upper axis is not perpendicular to lower\\
C2PC4 & TEL\_PC\_BNP & PTCON4 & Angle by which beam is not perpendicular to upper axis\\
C2XPC & TEL\_PC\_LAZ & AXPOINT & Azimuth/RA enc.zero; enc.reading = az + pc\_laz\\
C2YPC & TEL\_PC\_UAZ & AYPOINT & Altitude/DEC enc.zero; enc.reading = az + pc\_uaz\\
C3BEFENULO & BES\_FE\_NULO & \emph{BEFENULO} & Copy of frontend LO frequency per backend section\\
C3BEFESB & BES\_FE\_SB\_SIGN & \emph{BEFESB} & Copy of frontend sideband sign per backend section\\
C3BEINCON & BE\_IN\_CONN & \emph{BEINCON} & IF output channels connected to BE input channels\\
C3BESCONN & BES\_CONN & \emph{BESCON} & BE input channels connected to this section\\
C3BESSPEC & BES\_SPECTRUM & \emph{BESSPEC} & Subsystem nr to which each backend section belongs.\\
C3BETOTIF & BES\_TOT\_IF & \emph{BETOTIF} & Total IF per backend section\\
C3CAL & OBS\_CALIBRATION & \emph{OBSCAL} & Calibration observation?\\
C3CEN & OBS\_CENTRE & \emph{OBSCEN} & Centre moves between scans?\\
C3CL & CYCLE\_TIME & CYCLLEN & Duration of each cycle\\
C3CONFIGNR & DAS\_CONF\_NR & \emph{CONFIG} & Backend configuration\\
C3DASCALSRC & DAS\_CAL\_SOURCE & \emph{CALSRC} & DAS calibration source for backend calibration (POWER or DATA)\\
C3DASOUTPUT & DAS\_OUTPUT & \emph{OUTPUT} & Description of output in DAS DATA (SPECTRUM, T\_REC, T\_SYS, etc.)\\
C3DASSHFTFRAC & DAS\_SHIFT\_FRAC & \emph{SHFTFRAC} & DAS calibration source for backend calibration (POWER or DATA)\\
C3DAT & OBS\_UT1D & UTDATE & UT1 date of observation\\
C3FLY & OBS\_CONTINUOUS & \emph{OBSFLY} & Data taken on the fly or in discrete mode?\\
C3FOCUS & OBS\_FOCUS & \emph{OBSFOCUS} & Focus observation?\\
C3INTT & INTGRN\_TIME & \emph{INTGR} & Scan integration time\\
C3LSPC & NO\_BES\_O\_CH & \emph{NOBESDCH} & Number of channels per backend section\\
C3LST & OBS\_LST & LST & Local sidereal time at the start of the observation\\
C3MAP & OBS\_MAP & \emph{OBSMAP} & Map observation?\\
C3MXP & NO\_SCAN\_PNTS & \emph{NOSCNPTS} & Maximum number of map points done in a phase\\
C3NCH & NO\_BE\_O\_CH & \emph{NOBEOCH} & No.backend output channels\\
C3NCI & NO\_CYCLES & \emph{NOCYCLES} & Maximum number of cycles in the scan\\
C3NCP & NO\_CYCLE\_PNTS & \emph{NOCYCPTS} & Total number of xy positions observed during a cycle\\
C3NCYCLE & NCYCLE & \emph{NCYCLE} & Number of cycles done in the scan\\
C3NFOC & NO\_FE\_O\_CH & \emph{NOFCHAN} & NO\_FE\_O\_CH:No. of frontend output channels\\
C3NIS & NO\_SCANS & \emph{NOSCANS} & Number of scans\\
C3NLOOPS & NO\_SCANS & NSCANS & Number of scans per observation commanded at observation start\\
C3NMAP & NO\_MAP\_PNTS & NOPTS & Number of map points\\
C3NOIFPBES & NO\_IF\_PER\_BES & \emph{NOIFPBES} & Number of IF inputs to each section (2 for correlator, 1 for AOS)\\
C3NO\_SCAN\_VARS1 & NO\_SCAN\_VARS1 & \emph{NOSCNVR1} & Number of scan table 1 variables\\
C3NO\_SCAN\_VARS2 & NO\_SCAN\_VARS2 & \emph{NOSCNVR2} & Number of scan table 2 variables\\
C3NPP & NO\_MAP\_DIMS & \emph{NOMAPDIM} & Number of dimension in the map table\\
C3NRC & NRC & NORCHAN & NO\_BE\_I\_CH:No.backend input channels\\
C3NRS & NO\_BES & \emph{NORSECT} & Number of backend sections\\
C3NSAMPLE & NSCAN & \emph{NSCAN} & Number of scans done\\
C3NSV & NO\_PHASE\_VARS & NOSWVAR & Number of phase table variables\\
C3OVERLAP & BES\_OVERLAP & \emph{OVERLAP} & Subband overlap\\
C3PPC & NO\_PHASES & NOPHASE & Number of phases per cycle\\
C3SRT & SCAN\_TIME & SAMPRAT & Total time of scan (=total integration time if OBS\_CONTINUOUS = .FALSE.)\\
C3UT & OBS\_UT1H & UT & UT1 hour of observation\\
C3UT1C & OBS\_UT1C & \emph{UT1C} & UT1-UTC correction interpolated from time service telex (in days)\\
C4AMPL\_EW & AMPL\_EW & CHOPAZ & Secondary mirror chopping amplitude parallel to lower axis\\
C4AMPL\_NS & AMPL\_NS & CHOPEL & Secondary mirror chopping amplitude parallel to upper axis\\
C4AXY & CELL\_X2Y & \emph{XYANGLE} & Angle between cell y axis and x-axis (CCW)\\
C4AZ & CENTRE\_AZ & AZ & Azimuth at observation date\\
C4AZERR & SDIS(7) & XPOINT & DAZ:Net Az offset at start (inc.tracker ball setting and user correction)\\
C4CECO & CENTRE\_CODE & \emph{COORDCD2} & centre coords. AZ= 1;EQ=3;RD=4;RB=6;RJ=7;GA=8\\
C4CSC & CENTRE\_COORDS & COORDCD & Character code of commanded centre or source coordinate system\\
C4DAZ & DAZ\_SM & XPOINT & Telescope lower axis correction for secondary mirror XYZ\\
C4DECDATE & CENTRE\_DEC & \emph{DECDATE} & Declination of date\\
C4DEL & DEL\_SM & YPOINT & Telescope upper axis correction for secondary mirror XYZ\\
C4DO1 & CELL\_X & DESORG1 & Cell x dimension; descriptive origin item 1\\
C4DO2 & CELL\_Y & DESORG2 & Cell y dimension; descriptive origin item 2\\
C4DO3 & CELL\_V2X & DESORG3 & Angle by which the cell x axis is oriented with respect to local vertical\\
C4EDC & CENTRE\_DEC & EPOCDEC & Declination of date\\
C4EDEC & CENTRE\_DEC1950 & EPOCDEC & Declination of source for EPOCH\\
C4EDEC2000 & CENTRE\_DEC2000 & \emph{DECJ2000} & Declination J2000\\
C4EL & CENTRE\_EL & EL & Elevation at observation date\\
C4ELERR & SDIS(8) & YPOINT & DEL:Net El offset at start (inc.tracker ball setting and user correction)\\
C4EPH & CENTRE\_EPOCH & EPOCH & Date of the RA/DEC coordinates (1950)\\
C4EPT & EPOCH\_TYPE & \emph{EPOCHTYP} & Type of epoch, JULIAN, BESSELIAN or APPARENT\\
C4ERA & CENTRE\_RA1950 & EPOCRA & Right ascension of source for EPOCH\\
C4EW\_ENCODER & AMPL\_E\_SET & \emph{EW\_ENCOD} & Secondary mirror ew encoder value\\
C4EW\_SCALE & EW\_AMPL\_SCALE & \emph{EW\_SCALE} & Secondary mirror ew chop scale\\
C4FRQ & FREQUENCY & \emph{CHOPFREQ} & Secondary mirror chopping period\\
C4FUN & WAVEFORM & \emph{WAVEFORM} & Secondary mirror chopping waveform\\
C4GB & CENTRE\_GB & GALLAT & Galactic latitude\\
C4GL & CENTRE\_GL & GALLONG & Galactic longitude\\
C4LSC & CELL\_COORDS & FRAME & Char. code for local x-y coord.system\\
C4MCF & CENTRE\_MOVING & \emph{CENMOVE} & Centre moving flag (solar system object)\\
C4MOCO & TEL\_COORDS & \emph{MOUNTING} & Mounting of telescope; defined as LOWER/UPPER axes, e.g; AZ/ALT\\
C4NS\_ENCODER & AMPL\_N\_SET & \emph{NS\_ENCOD} & Secondary mirror ns encoder value\\
C4NS\_SCALE & NS\_AMPL\_SCALE & \emph{NS\_SCALE} & Secondary mirror ns chop scale\\
C4ODCO & CELL\_UNIT & \emph{MAPUNIT} & Units of cell and mapping coordinates;offset definition code\\
C4OFFS\_EW & OFFS\_EW & \emph{EWTILT} & Secondary mirror offset parallel to lower axis (East-West Tilt)\\
C4OFFS\_NS & OFFS\_NS & \emph{NSTILT} & Secondary mirror offset parallel to upper axis (North-South Tilt)\\
C4PER & PERIOD & CHOPTIME & Secondary mirror chopping period\\
C4POSANG & POSANG & \emph{CHOPDIRN} & Secondary mirror chop position angle\\
C4RA2000 & CENTRE\_RA2000 & \emph{RAJ2000} & Right ascension J2000\\
C4RADATE & CENTRE\_RA & \emph{RADATE} & Right Ascension of date\\
C4RX & REFERENCE\_X & \emph{XREF} & Reference x position (JCMT cells wrt to centre; NRAO abs. degrees)\\
C4RY & REFERENCE\_Y & \emph{YREF} & Reference y position (JCMT cells wrt to centre; NRAO abs. degrees)\\
C4SM & CHOPPING & \emph{CHOPPING} & Secondary mirror is chopping\\
C4SMCO & COORDS & \emph{CHOPCOOR} & Secondary mirror chopping coordinate system\\
C4SX & CENTRE\_OFFSET\_X & \emph{XSOURCE} & Commanded x centre position (JCMT cells wrt to centre; NRAO abs. degrees)\\
C4SY & CENTRE\_OFFSET\_Y & \emph{YSOURCE} & Commanded y centre position (JCMT cells wrt to centre; NRAO abs. degrees)\\
C4THROW & THROW & \emph{CHOPTHRW} & Secondary mirror chop throw\\
C4X & X & \emph{AFOCUSV} & Secondary mirror absolute X position at observation start\\
C4Y & Y & \emph{AFOCUSH} & Secondary mirror absolute Y position at observation start\\
C4Z & Z & \emph{AFOCUSR} & Secondary mirror absolute Z position at observation start\\
C5AT & TAMB & TAMB & Ambient temperature\\
C5DP & DEW\_POINT & DEW\_PT & Mean atmospheric dew point\\
C5IR & REF\_INDEX & REFRAC & Mean atmospheric refractive index (alternative)\\
C5IR1 & TEL\_REFR\_A & REFRAC1 & Refraction constant A (see MTIN026)\\
C5IR2 & TEL\_REFR\_B & REFRAC2 & Refraction constant B (see MTIN026)\\
C5IR3 & TEL\_REFR\_C & REFRAC3 & Refraction constant C (see MTIN026)\\
C5MM & VAP\_PRESSURE & MMH2O & Mean atmospheric vapour pressure\\
C5PRS & PAMB & PRESSURE & Mean atmospheric pressure\\
C5RH & HAMB & HUMIDITY & Mean atmospheric relative humidity\\
C6CYCLREV & CYCLE\_REVERSAL & \emph{CYCLREV} & Cycle reversal flag\\
C6DX & CELL\_X & DELTAX & Cell x dim,; descriptive origin item 1\\
C6DY & CELL\_Y & DELTAY & Cell y dimension; descriptive origin item 2\\
C6FC & CELL\_CODE & \emph{FRAME2} & Local x-y AZ= 1;EQ=3;RD=4;RB=6;RJ=7;GA=8\\
C6MODE & SWITCH\_MODE & \emph{SWMODE} & Observation mode\\
C6MSA & CELL\_V2X & SCANANG & Scanning angle - angle from local vertical to x axis measured CW\\
C6NP & NCYCLE\_PNTS & \emph{NCYCPTS} & Number of sky points completed in the observation\\
C6REV & SCAN\_REVERSAL & \emph{REVERSAL} & Map rows scanned in alternate directions?\\
C6SD & OBS\_DIRECTION & DIRECTN & Map rows are in X (horizontal) or Y(vertical) direction\\
C6ST & OBS\_TYPE & OBSMODE & Type of observation\\
C6XGC & X\_MAP\_START & XCELL0 & X coordinate of the first map point\\
C6XNP & NO\_X\_MAP\_PNTS & NOXPTS & X map dimension; number of points in the x-direction\\
C6XPOS & X\_MAP\_POSITIVE & \emph{XSIGN} & In first row x increases (TRUE) or decreases (FALSE)\\
C6YGC & Y\_MAP\_START & YCELL0 & Y coordinate of the first map point\\
C6YNP & NO\_Y\_MAP\_PNTS & NOYPTS & Y map dimension; number of points in the y-direction\\
C6YPOS & Y\_MAP\_POSITIVE & \emph{YSIGN} & In first row y increases (TRUE) or decreases (FALSE)\\
C7AP & APERTURE & \emph{APERTURE} & Aperture\\
C7BCV & BAD\_CHANNEL & BADCHV & Bad channel value\\
C7CAL & CAL\_TYPE & TYPECAL & Calibration type (standard or direct, chopperwheel)\\
C7FIL & FILTER & \emph{FILTER} & Filter\\
C7HP & FWHM &  & FWHM of the beam profile (mean)\\
C7NIF & NO\_IF\_CH & NOIFCHAN & Number of IF channels\\
C7OSN & OBS\_REF\_SCAN & OFFSCAN & Calibration observation number (in which a standard was observed)\\
C7PHASE & PHASE & \emph{PHASE} & Lockin phase\\
C7SEEING & SAO\_SEEING & \emph{SEEING} & Seeing at JCMT\\
C7SEETIME & SAO\_YYMMDDHHMM & \emph{SEETIME} & SAO seeing time (YYMMDDHHMM)\\
C7SNSTVTY & SENSITIVITY & \emph{SNSTVTY} & Lockin sensitivity in scale range units\\
C7SNTVTYRG & RANGE & \emph{SNTVTYRG} & Sensitivity range of lockin\\
C7SZVRAD & SZVRAD & SZVRAD & Number of elements of vradial array\\
C7TAU225 & CSO\_TAU & \emph{TAU225} & CSO tau at 225GHz\\
C7TAURMS & CSO\_TAU\_RMS & \emph{TAURMS} & CSO tau rms\\
C7TAUTIME & CSO\_YYMMDDHHMM & \emph{TAUTIME} & CSO tau time (YYMMDDHHMM)\\
C7TIMECNST & TIME\_CONSTANT & \emph{TIMECNST} & Lockin time constant\\
C7VC & VEL\_COR & RVSYS & Velocity correction\\
C7VR & VELOCITY & VELOCITY & Radial velocity of the source\\
C7VRD & VEL\_DEF & VELDEF & Velocity definition code; method of computing the velocity\\
C7VREF & VEL\_REF & VELREF & Velocity reference code; reference point for telescope \& source velocity\\
C8AAE & APERTURE\_EFF & APPEFF & Ratio total power observed/incident on the telescope\\
C8ABE & BEAM\_EFF & BEAMEFF & Fraction of beam in diffraction limited main beam\\
C8EF & ETAFSS & ETAFSS & Forward spillover and scattering efficiency\\
C8EL & ETAL & ETAL & Rear spillover and scattering efficiency\\
C8GN & ANTENNA\_GAIN & ANTGAIN & Antenna gain\\
C9OT & TEL\_TOLERANCE & OBSTOL & Observing tolerance\\
C11PHA & PHASE\_TABLE & \emph{PHASTB} & Phase table: switching scheme dependent\\
C11VD & PHASE\_VARS & \emph{VARDES} & Names of the cols. of phase table\\
C12ALPHA & BES\_ALPHA & \emph{ALPHA} & Ratio of signal sideband to image sideband sky transmission\\
C12BM & BES\_BITMODE & BM & Correlation bit mode\\
C12BW & BANDWIDTH & BW & Bandwidth\\
C12CAL & DATA\_UNITS & \emph{DATAUNIT} & Units of spectrum data\\
C12CALTASK & BE\_CAL\_TASK & \emph{CALTASK} & Calibration instrument used (FE, BE, or USER)\\
C12CALTYPE & BE\_CAL\_TYPE & \emph{CALTYPE} & Type of calibration (THREELOADS or TWOLOADS)\\
C12CF & FE\_NUOBS & OBSFREQ & Centre frequency\\
C12CM & BES\_CORR\_MODE & CM & Correlation function mode\\
C12CT & TCAL & TCAL & Calibration temperature\\
C12ETASKY & BES\_ETA\_SKY & \emph{ETA\_SKY} & Sky transmission from last calibration\\
C12ETASKYIM & BES\_ETA\_SKY\_IM & \emph{ETA\_SKY\_} & Frontend-derived sky transmission\\
C12ETATEL & BES\_ETA\_TEL & \emph{ETA\_TEL} & Telescope transmission\\
C12FR & DELTANU & FREQRES & Frequency resolution [MHz]\\
C12GAINS & GAIN & \emph{GAINS} & Gains\\
C12GNORM & G\_NORM & GNORM & Data normalisation factor\\
C12GREC & GREC &  & Raw data units per Kelvin\\
C12GS & BES\_G\_S & \emph{G\_S} & Normalizes signal sideband gain\\
C12INFREQ & BE\_NUIN & \emph{INFREQ} & BE input frequencies [GHz]\\
C12NOI & NOISE & \emph{NOISE} & Noise value\\
C12REDMODE & BE\_RED\_MODE & \emph{REDMODE} & Way of calibrating the data (RATIO or DIFFERENCE)\\
C12RF & FE\_NUREST & RESTFREQ & Rest frequency\\
C12RST & TSYS\_OFF & RTSYS & Reference system temperature\\
C12RT & TREC & TRX & Receiver temperature\\
C12SBRAT & BE\_SB\_RATIO & \emph{SBRATIO} & Sideband ratio\\
C12SCAN\_TABLE\_1 & SCAN\_TABLE1 & \emph{SCANTAB1} & Begin scan table\\
C12SCAN\_TABLE\_2 & SCAN\_TABLE2 & \emph{SCANTAB2} & End scan table\\
C12SCAN\_VARS1 & SCAN\_VARS1 & \emph{SCANVRS1} & Names of the cols. of scan table1\\
C12SCAN\_VARS2 & SCAN\_VARS2 & \emph{SCANVRS2} & Names of the cols. of scan table2\\
C12SST & TSYS\_ON & STSYS & Source system temperature\\
C12TAMB & T\_HOT & \emph{THOT} & Ambient load temperature\\
C12TASKY & BES\_TA\_SKY & \emph{TA\_SKY} & Ratio of signal sideband to image sideband sky transmission\\
C12TCOLD & T\_COLD & \emph{TCOLD} & Cold load temperature\\
C12TSKY & TSKY & \emph{TSKY} & Sky temperature at last calibration\\
C12TSKYIM & BES\_T\_SKY\_IM & \emph{T\_SKY\_IM} & Frontend-derived Tsky, image sideband\\
C12TSYSIM & BES\_T\_SYS\_IM & \emph{T\_SYS\_IM} & Frontend-derived Tsys, image sideband\\
C12TTEL & TTEL & \emph{TTEL} & Telescope temp. from last skydip\\
C12VCOLD & IF\_V\_COLD & \emph{VCOLD} & \\
C12VDEF & VEL\_DEFN &  & Velocity definition code - radio, optical, or relativistic\\
C12VHOT & IF\_V\_HOT & \emph{VHOT} & \\
C12VREF & VEL\_REF &  & Velocity frame of reference - LSR, Bary-, Helio-, or Geo- centric\\
C12VSKY & IF\_V\_SKY & \emph{VSKY} & \\
C12WO & H2O\_OPACITY & TAUH2O & Water opacity\\
C13DAT & DATA & SPECT & Reduced data\\
C13ERR & ERROR & TRMS & Standard error\\
C13RAW\_ERROR & DATA\_ERROR & \emph{RAWERROR} & \\
C13RAW\_ERROR\_OP & RAW\_ERROR\_OP & \emph{RAWERROR} & Raw (out of phase) error also to be stored at end scan\\
C13RESP & RESP & \emph{RESP} & array of responsivities\\
C13SPV & SAMPLES & \emph{RAWSPECT} & \\
C13SPV\_OP & SAMPLES\_OP & \emph{RAWSPECT} & Raw out of phase data samples in each phase\\
C13STD & STDEV & \emph{STDSPECT} & Phase data standard deviation\\
C14PHIST & MAP\_TABLE & \emph{MAPTABLE} & List of xy offsets for each scan\\
C90T & TEL\_TOLERANCE & OBSTOL & Observing tolerance\\

\hline
\end{sllongtable}
\end{landscape}
\end{small}



\end{document}
