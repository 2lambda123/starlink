\documentclass[twoside,11pt]{article}

% ? Specify used packages
% \usepackage{graphicx}        %  Use this one for final production.
% \usepackage[draft]{graphicx} %  Use this one for drafting.
% ? End of specify used packages

\pagestyle{myheadings}

% -----------------------------------------------------------------------------
% ? Document identification
% Fixed part
\newcommand{\stardoccategory}  {Starlink User Note}
\newcommand{\stardocinitials}  {SUN}
\newcommand{\stardocsource}    {sun\stardocnumber}
\newcommand{\stardoccopyright} 
{Copyright \copyright\ 2009 Science and Technology Facilities Council.}

% Variable part - replace [xxx] as appropriate.
\newcommand{\stardocnumber}    {261.1}
\newcommand{\stardocauthors}   {D.S. Berry \& Malcolm J. Currie}
\newcommand{\stardocdate}      {2009 August 9}
\newcommand{\stardoctitle}     {IRQ --- Handling of QUALITY in NDFs}
\newcommand{\stardocversion}   {Version 3.0}
\newcommand{\stardocmanual}    {User's Guide}
\newcommand{\stardocabstract}  {
This library is a set of Fortran routines for manipulation of quality
information within NDFs.  In particular it uses names that will be more
memorable than bits to assign and set quality attributes of data values
within an NDF.
}
% ? End of document identification
% -----------------------------------------------------------------------------

% +
%  Name:
%     sun261.tex
%
%  Purpose:
%     This is a description of the IRQ library
%
%  Authors:
%     MJC: Malcolm J. Currie (STARLINK, RAL)
%
%  History:
%      2009 August 8 (MJC):
%        Original version, based on the IRAS ID6, updated to modern SUN style.
%
% -

\newcommand{\stardocname}{\stardocinitials /\stardocnumber}
\markboth{\stardocname}{\stardocname}
\setlength{\textwidth}{160mm}
\setlength{\textheight}{230mm}
\setlength{\topmargin}{-2mm}
\setlength{\oddsidemargin}{0mm}
\setlength{\evensidemargin}{0mm}
\setlength{\parindent}{0mm}
\setlength{\parskip}{\medskipamount}
\setlength{\unitlength}{1mm}
\newcommand{\noteroutine}[2]{\textbf{#1}\hspace*{\fill}\nopagebreak \\
                             \hspace*{3em}\emph{#2}\hspace*{\fill}\par}
%\newcommand{\noteroutine}[2]{{\small \bf #1} \\
%                              \hspace*{3em} {\em #2} \\[1.5ex]}


% -----------------------------------------------------------------------------
%  Hypertext definitions.
%  ======================
%  These are used by the LaTeX2HTML translator in conjunction with star2html.

%  Comment.sty: version 2.0, 19 June 1992
%  Selectively in/exclude pieces of text.
%
%  Author
%    Victor Eijkhout                                      <eijkhout@cs.utk.edu>
%    Department of Computer Science
%    University Tennessee at Knoxville
%    104 Ayres Hall
%    Knoxville, TN 37996
%    USA

%  Do not remove the %begin{latexonly} and %end{latexonly} lines (used by 
%  LaTeX2HTML to signify text it shouldn't process).
%begin{latexonly}
\makeatletter
\def\makeinnocent#1{\catcode`#1=12 }
\def\csarg#1#2{\expandafter#1\csname#2\endcsname}

\def\ThrowAwayComment#1{\begingroup
    \def\CurrentComment{#1}%
    \let\do\makeinnocent \dospecials
    \makeinnocent\^^L% and whatever other special cases
    \endlinechar`\^^M \catcode`\^^M=12 \xComment}
{\catcode`\^^M=12 \endlinechar=-1 %
 \gdef\xComment#1^^M{\def\test{#1}
      \csarg\ifx{PlainEnd\CurrentComment Test}\test
          \let\html@next\endgroup
      \else \csarg\ifx{LaLaEnd\CurrentComment Test}\test
            \edef\html@next{\endgroup\noexpand\end{\CurrentComment}}
      \else \let\html@next\xComment
      \fi \fi \html@next}
}
\makeatother

\def\includecomment
 #1{\expandafter\def\csname#1\endcsname{}%
    \expandafter\def\csname end#1\endcsname{}}
\def\excludecomment
 #1{\expandafter\def\csname#1\endcsname{\ThrowAwayComment{#1}}%
    {\escapechar=-1\relax
     \csarg\xdef{PlainEnd#1Test}{\string\\end#1}%
     \csarg\xdef{LaLaEnd#1Test}{\string\\end\string\{#1\string\}}%
    }}

%  Define environments that ignore their contents.
\excludecomment{comment}
\excludecomment{rawhtml}
\excludecomment{htmlonly}

%  Hypertext commands etc. This is a condensed version of the html.sty
%  file supplied with LaTeX2HTML by: Nikos Drakos <nikos@cbl.leeds.ac.uk> &
%  Jelle van Zeijl <jvzeijl@isou17.estec.esa.nl>. The LaTeX2HTML documentation
%  should be consulted about all commands (and the environments defined above)
%  except \xref and \xlabel which are Starlink specific.

\newcommand{\htmladdnormallinkfoot}[2]{#1\footnote{#2}}
\newcommand{\htmladdnormallink}[2]{#1}
\newcommand{\htmladdimg}[1]{}
\newcommand{\hyperref}[4]{#2\ref{#4}#3}
\newcommand{\htmlref}[2]{#1}
\newcommand{\htmlimage}[1]{}
\newcommand{\htmladdtonavigation}[1]{}

\newenvironment{latexonly}{}{}
\newcommand{\latex}[1]{#1}
\newcommand{\html}[1]{}
\newcommand{\latexhtml}[2]{#1}
\newcommand{\HTMLcode}[2][]{}

%  Starlink cross-references and labels.
\newcommand{\xref}[3]{#1}
\newcommand{\xlabel}[1]{}

%  LaTeX2HTML symbol.
\newcommand{\latextohtml}{\LaTeX2\texttt{HTML}}

%  Define command to re-centre underscore for Latex and leave as normal
%  for HTML (severe problems with \_ in tabbing environments and \_\_
%  generally otherwise).
\renewcommand{\_}{\texttt{\symbol{95}}}

% -----------------------------------------------------------------------------
%  Debugging.
%  =========
%  Remove % on the following to debug links in the HTML version using Latex.

% \newcommand{\hotlink}[2]{\fbox{\begin{tabular}[t]{@{}c@{}}#1\\\hline{\footnotesize #2}\end{tabular}}}
% \renewcommand{\htmladdnormallinkfoot}[2]{\hotlink{#1}{#2}}
% \renewcommand{\htmladdnormallink}[2]{\hotlink{#1}{#2}}
% \renewcommand{\hyperref}[4]{\hotlink{#1}{\S\ref{#4}}}
% \renewcommand{\htmlref}[2]{\hotlink{#1}{\S\ref{#2}}}
% \renewcommand{\xref}[3]{\hotlink{#1}{#2 -- #3}}
%end{latexonly}
% -----------------------------------------------------------------------------
% ? Document specific \newcommand or \newenvironment commands.
\newcommand{\KAPPA}{{\footnotesize KAPPA}\normalsize}
\newcommand{\KAPPAref}{\xref{\KAPPA}{sun95}{}}

\newcommand{\numcir}[1]{\latexhtml{\mbox{\hspace{3ex}$\bigcirc$\hspace{-1.7ex}{\small #1}}}{\htmlref{Note #1}{NOTE#1}}}

% -----------------------------------------------------------------------------

%+
%  Name:
%     SST.TEX

%  Purpose:
%     Define LaTeX commands for laying out Starlink routine descriptions.

%  Language:
%     LaTeX

%  Type of Module:
%     LaTeX data file.

%  Description:
%     This file defines LaTeX commands which allow routine documentation
%     produced by the SST application PROLAT to be processed by LaTeX and
%     by LaTeX2html. The contents of this file should be included in the
%     source prior to any statements that make of the sst commnds.

%  Notes:
%     The style file html.sty provided with LaTeX2html needs to be used.
%     This must be before this file.

%  Authors:
%     RFWS: R.F. Warren-Smith (STARLINK)
%     PDRAPER: P.W. Draper (Starlink - Durham University)

%  History:
%     10-SEP-1990 (RFWS):
%        Original version.
%     10-SEP-1990 (RFWS):
%        Added the implementation status section.
%     12-SEP-1990 (RFWS):
%        Added support for the usage section and adjusted various spacings.
%     8-DEC-1994 (PDRAPER):
%        Added support for simplified formatting using LaTeX2html.
%     {enter_further_changes_here}

%  Bugs:
%     {note_any_bugs_here}

%-

%  Define length variables.
\newlength{\sstbannerlength}
\newlength{\sstcaptionlength}
\newlength{\sstexampleslength}
\newlength{\sstexampleswidth}

%  Define a \tt font of the required size.
\latex{\newfont{\ssttt}{cmtt10 scaled 1095}}
\html{\newcommand{\ssttt}{\tt}}

%  Define a command to produce a routine header, including its name,
%  a purpose description and the rest of the routine's documentation.
\newcommand{\sstroutine}[3]{
   \goodbreak
   \rule{\textwidth}{0.5mm}
   \vspace{-7ex}
   \newline
   \settowidth{\sstbannerlength}{{\Large {\bf #1}}}
   \setlength{\sstcaptionlength}{\textwidth}
   \setlength{\sstexampleslength}{\textwidth}
   \addtolength{\sstbannerlength}{0.5em}
   \addtolength{\sstcaptionlength}{-2.0\sstbannerlength}
   \addtolength{\sstcaptionlength}{-5.0pt}
   \settowidth{\sstexampleswidth}{{\bf Examples:}}
   \addtolength{\sstexampleslength}{-\sstexampleswidth}
   \parbox[t]{\sstbannerlength}{\flushleft{\Large {\bf #1}}}
   \parbox[t]{\sstcaptionlength}{\center{\Large #2}}
   \parbox[t]{\sstbannerlength}{\flushright{\Large {\bf #1}}}
   \begin{description}
      #3
   \end{description}
}

%  Format the description section.
\newcommand{\sstdescription}[1]{\item[Description:] #1}

%  Format the usage section.
\newcommand{\sstusage}[1]{\item[Usage:] \mbox{}
\\[1.3ex]{\raggedright \ssttt #1}}

%  Format the invocation section.
\newcommand{\sstinvocation}[1]{\item[Invocation:]\hspace{0.4em}{\tt #1}}

%  Format the arguments section.
\newcommand{\sstarguments}[1]{
   \goodbreak
   \item[Arguments:] \mbox{} \\
   \vspace{-3.5ex}
   \begin{description}
      #1
   \end{description}
}

%  Format the returned value section (for a function).
\newcommand{\sstreturnedvalue}[1]{
   \item[Returned Value:] \mbox{} \\
   \vspace{-3.5ex}
   \begin{description}
      #1
   \end{description}
}

%  Format the parameters section (for an application).
\newcommand{\sstparameters}[1]{
   \item[Parameters:] \mbox{} \\
   \vspace{-3.5ex}
   \begin{description}
      #1
   \end{description}
}

%  Format the examples section.
\newcommand{\sstexamples}[1]{
   \item[Examples:] \mbox{} \\
   \vspace{-3.5ex}
   \begin{description}
      #1
   \end{description}
}

%  Define the format of a subsection in a normal section.
\newcommand{\sstsubsection}[1]{ \item[{#1}] \mbox{} \\}

%  Define the format of a subsection in the examples section.
\newcommand{\sstexamplesubsection}[2]{\sloppy
\item[\parbox{\sstexampleslength}{\ssttt #1}] \mbox{} \vspace{1.0ex}
\\ #2 }

%  Format the notes section.
\newcommand{\sstnotes}[1]{\goodbreak\item[Notes:] \mbox{} \\[1.3ex] #1}

%  Provide a general-purpose format for additional (DIY) sections.
\newcommand{\sstdiytopic}[2]{\goodbreak\item[{\hspace{-0.35em}#1\hspace{-0.35em}:}]
\mbox{} \\[1.3ex] #2}

%  Format the implementation status section.
\newcommand{\sstimplementationstatus}[1]{
   \item[{Implementation Status:}] \mbox{} \\[1.3ex] #1}

%  Format the bugs section.
\newcommand{\sstbugs}[1]{\item[Bugs:] #1}

%  Format a list of items while in paragraph mode.
\newcommand{\sstitemlist}[1]{
  \mbox{} \\
  \vspace{-3.5ex}
  \begin{itemize}
     #1
  \end{itemize}
}

%  Define the format of an item.
\newcommand{\sstitem}{\item}

%% Now define html equivalents of those already set. These are used by
%  latex2html and are defined in the html.sty files.
\begin{htmlonly}

%  sstroutine.
   \newcommand{\sstroutine}[3]{
      \subsection{#1\xlabel{#1}-\label{#1}#2}
      \begin{description}
         #3
      \end{description}
      \\
   }

%  sstdescription
   \newcommand{\sstdescription}[1]{\item[Description:]
      \begin{description}
         #1
      \end{description}
      \\
   }

%  sstusage
   \newcommand{\sstusage}[1]{\item[Usage:]
      \begin{description}
         {\ssttt #1}
      \end{description}
      \\
   }

%  sstinvocation
   \newcommand{\sstinvocation}[1]{\item[Invocation:]
      \begin{description}
         {\ssttt #1}
      \end{description}
      \\
   }

%  sstarguments
   \newcommand{\sstarguments}[1]{
      \item[Arguments:] \\
      \begin{description}
         #1
      \end{description}
      \\
   }

%  sstreturnedvalue
   \newcommand{\sstreturnedvalue}[1]{
      \item[Returned Value:] \\
      \begin{description}
         #1
      \end{description}
      \\
   }

%  sstparameters
   \newcommand{\sstparameters}[1]{
      \item[Parameters:] \\
      \begin{description}
         #1
      \end{description}
      \\
   }

%  sstexamples
   \newcommand{\sstexamples}[1]{
      \item[Examples:] \\
      \begin{description}
         #1
      \end{description}
      \\
   }

%  sstsubsection
   \newcommand{\sstsubsection}[1]{\item[{#1}]}

%  sstexamplesubsection
   \newcommand{\sstexamplesubsection}[2]{\item[{\ssttt #1}] #2}

%  sstnotes
   \newcommand{\sstnotes}[1]{\item[Notes:] #1 }

%  sstdiytopic
   \newcommand{\sstdiytopic}[2]{\item[{#1}] #2 }

%  sstimplementationstatus
   \newcommand{\sstimplementationstatus}[1]{
      \item[Implementation Status:] #1
   }

%  sstitemlist
   \newcommand{\sstitemlist}[1]{
      \begin{itemize}
         #1
      \end{itemize}
      \\
   }
%  sstitem
   \newcommand{\sstitem}{\item}

\end{htmlonly}

%  End of "sst.tex" layout definitions.
%.


% ? End of document specific commands
% -----------------------------------------------------------------------------
%  Title Page.
%  ===========
\renewcommand{\thepage}{\roman{page}}
\begin{document}
\thispagestyle{empty}

%  Latex document header.
%  ======================
\begin{latexonly}
   \textsc{Rutherford Appleton Laboratory} \hfill \textbf{\stardocname}\\
   {\large Science \& Technology Facilities Council}\\
   {\large Starlink Software Collection\\}
   {\large \stardoccategory\ \stardocnumber}
   \begin{flushright}
   \stardocauthors\\
   \stardocdate
   \end{flushright}
   \vspace{-4mm}
   \rule{\textwidth}{0.5mm}
   \vspace{5mm}
   \begin{center}
   {\Huge\textbf{\stardoctitle \\ [2.5ex]}}
   {\LARGE\textbf{\stardocversion \\ [4ex]}}
   {\Huge\textbf{\stardocmanual}}
   \end{center}
   \vspace{5mm}

% ? Add picture here if required for the LaTeX version.
%   e.g. \includegraphics[scale=0.3]{filename.ps}
% ? End of picture

% ? Heading for abstract if used.
   \vspace{10mm}
   \begin{center}
      {\Large\textbf{Abstract}}
   \end{center}
% ? End of heading for abstract.
\end{latexonly}

%  HTML documentation header.
%  ==========================
\begin{htmlonly}
   \xlabel{}
   \begin{rawhtml} <H1> \end{rawhtml}
      \stardoctitle\\
      \stardocversion\\
      \stardocmanual
   \begin{rawhtml} </H1> <HR> \end{rawhtml}

% ? Add picture here if required for the hypertext version.
%   e.g. \includegraphics[scale=0.7]{filename.ps}
% ? End of picture

   \begin{rawhtml} <P> <I> \end{rawhtml}
   \stardoccategory\ \stardocnumber \\
   \stardocauthors \\
   \stardocdate
   \begin{rawhtml} </I> </P> <H3> \end{rawhtml}
      \htmladdnormallink{Rutherford Appleton Laboratory}
                        {http://www.scitech.ac.uk} \\
      \htmladdnormallink{Science \& Technology Facilities Council}
                        {http://www.scitech.ac.uk} \\
   \begin{rawhtml} </H3> <H2> \end{rawhtml}
      \htmladdnormallink{Starlink Project}{http://www.starlink.ac.uk/}
   \begin{rawhtml} </H2> \end{rawhtml}
   \htmladdnormallink{\htmladdimg{source.gif} Retrieve hardcopy}
      {http://www.starlink.ac.uk/cgi-bin/hcserver?\stardocsource}\\

%  HTML document table of contents. 
%  ================================
%  Add table of contents header and a navigation button to return to this 
%  point in the document (this should always go before the abstract \section). 
  \label{stardoccontents}
  \begin{rawhtml} 
    <HR>
    <H2>Contents</H2>
  \end{rawhtml}
  \htmladdtonavigation{\htmlref{\htmladdimg{contents_motif.gif}}
        {stardoccontents}}

% ? New section for abstract if used.
  \section{\xlabel{abstract}Abstract}
% ? End of new section for abstract
\end{htmlonly}

% -----------------------------------------------------------------------------

% -----------------------------------------------------------------------------
% ? Document Abstract. (if used)
%  ==================
\stardocabstract
% ? End of document abstract

% -----------------------------------------------------------------------------
% ? Latex Copyright Statement
%  =========================
\begin{latexonly}
\newpage
\vspace*{\fill}
\stardoccopyright
\end{latexonly}
% ? End of Latex copyright statement

% -----------------------------------------------------------------------------
% ? Latex document Table of Contents (if used).
%  ===========================================
  \newpage
  \begin{latexonly}
    \setlength{\parskip}{0mm}
    \tableofcontents
    \setlength{\parskip}{\medskipamount}
    \markboth{\stardocname}{\stardocname}
  \end{latexonly}
% ? End of Latex document table of contents
% -----------------------------------------------------------------------------

\cleardoublepage
\renewcommand{\thepage}{\arabic{page}}
\setcounter{page}{1}

%  End of "sst.tex" layout definitions.
% .
% %Z%%M%   %I%   %E% %U%   %D% %T%

% ? Main text

\section {\xlabel{SEC:intro}Introduction to QUALITY}

A QUALITY structure is one of the standard components of an NDF
structure, and is described fully in
\xref{SUN/33}{sun33}{the_quality_component_in_more_detail}. Briefly,
if an NDF has a QUALITY component which is in a {\em defined} state,
then each pixel within the NDF DATA component has a corresponding
value in the QUALITY component. Currently, each QUALITY value consists
of an unsigned byte (\emph{i.e.} 8 bits). In Fortran the
least-significant bit is usually called Bit 0 and the most significant
bit is usually called Bit 7. Within the IRQ package the
least-significant bit is called Bit 1 and the most-significant bit is
called Bit 8. Each bit within the QUALITY value can be used to
indicate if the corresponding pixel in the DATA component holds some
specific {\em quality}. For instance, Bit 3 of the QUALITY component
may be used to indicate if any DATA pixels are saturated. A particular
pixel in the QUALITY component would have Bit 3 set (\emph{i.e.} equal
to 1) if the corresponding DATA pixel is saturated, or cleared
(\emph{i.e.} equal to 0) if the corresponding DATA pixel is not
saturated. 

Another option for flagging saturated data is to replace saturated DATA
pixel values with a `bad' (or `magic') value. This has the disadvantage
that the datum is permanently destroyed by being flagged, and also there
is no distinction between data values thst are set bad because of the
fact they were saturated, and pixels set bad for any other reason. 

SUN/33 doesn't specify how the facilities of the QUALITY component are to
be used, and many possibilities exist. Obviously some co-ordination
between applications is needed so that different applications interpret
the QUALITY values in a consistent manner (\{emph{i.e.} using the above
example, later applications must know that Bit 3 is a saturation flag).
The IRQ package provides a set of routines for doing this.

You may be are wondering about the name IRQ.  The library was originally
developed for the IRAS90 package, and its subroutine libraries had IR
prefix.

\section {\xlabel{SEC:IRQ}Introduction to the facilities provided by the
IRQ library}
\label {SEC:IRQ}

IRQ provides a system for handling Boolean qualities (\emph{i.e.}
qualities that are either held or not held by each DATA pixel).  From the
point of view of an application, each defined quality is identified by a
{\em quality name} rather than by a bit number. Information about these
quality names is stored in an NDF extension, so that later applications
can determine which pixels within the NDF hold a given quality (or
combination of qualities). 

Within IRQ, each defined quality name is usually associated with a bit in
the QUALITY array, and this bit is set if the corresponding DATA pixels
are assigned the specified quality. A typical application need know
nothing about which QUALITY bit is associated with which quality name.
For instance, if {\em every} pixel in an NDF holds a certain quality (or
alternatively, if {\em no} pixels hold the quality), then it is not
necessary to reserve a bit in the QUALITY array to represent the quality.
Instead, a single Boolean scalar value can be stored with the quality
name in the NDF extension. This scalar is set to {\tt .TRUE.} if {\em
all} pixels hold the quality, and {\tt .FALSE.} if {\em no} pixels hold
the quality. In this way, the number of defined quality names can
sometimes exceed the number of bits in the QUALITY component. The
handling of such situations is done within IRQ and is completely
invisible to the calling application. 

IRQ provides the following facilities.
\begin{itemize}
\item Add quality name definitions (and associated descriptive comments)
to an NDF. There is a limit to the number of quality names which may be
defined within an NDF. The exact number depends on how many quality names
require a QUALITY bit, but it will always be at least eight. 
\item Remove quality name definitions from an NDF.
\item List all quality names defined within an NDF.
\item Assign a given quality to selected pixels.
\item Remove a given quality from selected pixels.
\item Set `bad' those pixels of a supplied array which hold a certain 
combination of qualities.
\end{itemize}

\section {A Set of Four Typical IRQ Applications}

The \KAPPA\ package contains a set of four typical IRQ applications,
which give an indication of the benefits which IRQ can provide for the
user. The A-TASK documentation for these four applications is included
in \xref{SUN/95}{sun95}{se_qualitymask}, and they are briefly described
in this section. 

\subsection{SETQUAL}

\xref{SETQUAL}{sun95}{SETQUAL} creates quality name definitions and
stores them within a specified extension of an NDF. It also assigns a
specified quality to a sub-set of the pixels with the NDF. In its
simplest mode, the user provides an NDF, and a string to use as a quality
name (such as {\tt SATURATED}). If this quality name is not already
defined within the NDF, then it is added to the list of defined quality
names, together with a user-supplied comment describing the quality. The
user also provides another NDF to be used as a `mask'. The `bad' pixels
within the mask NDF define the pixels which are to be assigned the given
quality. For instance, a user may have an image containing saturated
pixels. If he wants these pixels flagged without being permanently
destroyed, he could proceed as follows.

\begin{enumerate}

\item Produce a mask NDF from the original NDF by setting all DATA pixels 
above the saturation value to the `bad' value. 

\item Run SETQUAL on the original NDF, giving some quality name such as
{\tt SATURATED}, and specifying the mask created in the previous step. This
leaves the DATA component of the original NDF unchanged, but assigns the
quality {\tt SATURATED} to all the pixels which correspond to `bad'
pixels in the mask.  

\end{enumerate}

Alternatively, instead of using a mask, the pixels to which the quality is
assigned may be specified by an explicit list of pixel indices stored in a
text file.

\subsection{REMQUAL}

The \xref{REMQUAL}{sun95}{REMQUAL} application removes quality name
definitions from an NDF. This may be necessary if there is no room for
any more quality name definitions within an NDF. In this case the user
may choose to remove some unimportant quality name definitions to make
room for new, more-important quality names. REMQUAL can also remove {\em
all} quality names information from an NDF. This can be useful if for any
reason the quality-name information becomes corrupted. 

\subsection{SHOWQUAL}
The \xref{SHOWQUAL}{sun95}{SHOWQUAL} application displays all currently
defined quality names within an NDF, together with the associated
descriptive comments. Optionally, the number of pixels that hold each
quality can be displayed.

\subsection{QUALTOBAD}

The \xref{QUALTOBAD}{sun95}{QUALITYTOBAD} application sets selected
pixels within an NDF to the `bad' value on the basis of the pixel's
quality. In effect, QUALTOBAD performs the reverse operation of 
\xref{SETQUAL}{SETQUAL}. For instance, using the example of saturated
data described above, the user may remove some varying background surface
from his original data, and then want to set pixels which were saturated
in the original data to the `bad' value . In this case he would run
QUALTOBAD specifying the background removed NDF as input, and specifying
a {\em quality expression} of {\tt SATURATED}. This would cause all pixels
which hold the quality {\tt SATURATED} to be set bad in the output NDF. 

The `quality expression' can be more complex than a single quality
name. In fact, a quality expression can consist of several quality
names combined together using the usual operators of Boolean algebra.
For instance, if the quality expression {\tt
SATURATED.AND..NOT.(SOURCE\_A .OR. SOURCE\_B)} was given to 
QUALTOBAD, then pixels would be set `bad' only if they had the quality
{\tt SATURATED}, but did not have either of the qualities {\tt
SOURCE\_A} or {\tt SOURCE\_B}.

\section{Quality Names}

A quality name must contain 15 or fewer characters. Any leading blanks
are removed from supplied quality names, and they are converted to
upper case before being stored in the NDF. Quality names may contain
embedded blanks, but may not contain full-stop ({\tt "."}) characters.
Certain names are reserved and may not be used. These are {\tt ANY},
{\tt IRQ\_BAD\_SLOT} and {\tt IRQ\_FREE\_SLOT}.

\section{\xlabel{SEC:QEXP}Quality Expressions}
\label{SEC:QEXP}

A `Quality Expression' consists of a set of quality names combined
together using Boolean operators into a legal Boolean expression. See
\latexhtml{Section~\ref{SEC:QFIND}}{\htmlref{this section}{SEC:QFIND}} 
for a description of the use of quality expressions within IRQ. In the
following, the symbols {\tt A} and {\tt B} are used to represent two
qualities. These can be considered as Boolean values; {\tt true} if a
pixel holds the quality, and {\tt false} otherwise. The supported
Boolean operators are listed below.  

\begin {description}
\item [.AND.] - The expression {\tt (A.AND.B)} is true if and only if
both {\tt A} and {\tt B} are true.  
\item [.OR.] - The expression {\tt (A.OR.B)} is true if and only if 
either {\tt A} or {\tt B} is true.
\item [.XOR.] - The
expression {\tt (A.XOR.B)} is true if and only if either {\tt A} is true and {\tt B}
is false, or {\tt A} is false and {\tt B} is true. 
\item [.EQV.] - The expression {\tt (A.EQV.B)} is true if and only if either {\tt A} 
is true and {\tt B} is true, or A is false and {\tt B} is false.
\item [.NOT.] - The expression {\tt (.NOT.A)} is true if and only if {\tt A} is 
false. 
\end{description} 
In addition to the above operators, the Boolean constants {\tt
.FALSE.} and {\tt .TRUE.} can be included within a quality expression.
Expressions may contain several levels of nested parentheses. 

The precedence of these operators decreases in the following order;
.NOT., .AND., .OR., .XOR., .EQV. (the final two have equal
precedence). In an expression such as {\tt (A.XOR.B.EQV.C.XOR.D)} in
which all operators have equal precedence, the evaluation proceeds
from left to right., \emph{i.e.} the expression is evaluated as
(((A.XOR.B).EQV.C).XOR.D). If there is any doubt about the order in
which an expression will be evaluated, parentheses should be used to
ensure the required order of evaluation.

Some attempts are made to simplify a quality expression to reduce the
run time needed to evaluate the expression for every pixel. 

Quality expressions can be up to 254 characters long, and must not
contain more than forty symbols (Boolean operators, constants, or
quality names).

\section {Using IRQ routines}

This section gives a brief outline of the IRQ routines which are
available to perform some common tasks. The specific details required
to use these routines are not included here but can be found in the
subroutine specifications contained in \latexhtml{Appendix~\ref
{SEC:FULLSPEC}}{\htmlref{the full specifications}{SEC:FULLSPEC}}.

\subsection {\xlabel{SEC:CON}Constants and Error Values}
\label {SEC:CON}

The IRQ package has associated with it various symbolic constants.
These values consist of a name of up to five characters prefixed by
``IRQ\_\_'' (note the {\em double} underscore), and can be made
available to an application by including the following line at the
start of the routines which uses them: 

\begin{verbatim}
      INCLUDE 'IRQ_PAR'
\end{verbatim}

This assumes that the IRQ library has been installed as part of the
UNIX Starlink Software Collection. 
% If this is not the case, then the alternative instructions given in
%Appendix~\ref {APP:NOSTAR} should be followed.

The values thus defined are described in the following sections, and
also in the subroutine specifications. Another set of symbolic
constants is made available by the statements 

\begin{verbatim}
      INCLUDE 'IRQ_ERR'
\end{verbatim}

These values have the same format of those contained in IRQ\_PAR, put define
various error conditions which can be generated within the IRQ package.
Applications can compare the $STATUS$ argument with these values to check for
specific error conditions. These values are described in 
\latexhtml{Appendix~\ref {APP:ERRORS}}{\htmlref{IRQ Error
Codes}{APP:ERRORS}}. 

\subsection{Initialising an NDF for use with IRQ}

Certain HDS structures must be created within an NDF before the NDF can
be used by IRQ. These structures hold information describing the
currently defined quality names within the NDF. Routine
\htmlref{IRQ\_FIND}{IRQ_FIND} can be used to see if such structures exist
within an NDF (see below). If no such structure yet exists within an NDF,
routine \htmlref{IRQ\_NEW}{IRQ_NEW} must be called to create the
structure. This structure is held in a specified NDF extension. IRQ\_NEW
returns an array of HDS locators which must be passed to subsequent IRQ
routines. One of these locators points to a cloned copy of the NDF
identifier. All access to the NDF by subsequent IRQ routines is achieved
through this cloned identifier. Once access to the quality names
information is no longer required, the resources used by these HDS
locators (including the cloned NDF identifier) should be annulled by
calling IRQ\_RLSE. Note, routines \htmlref{IRQ\_ANNUL}{IRQ_ANNUL} and 
\htmlref{IRQ\_CLOSE}{IRQ_CLOSE} play no part in releasing resources used
for accessing quality name information. These two routines are only used
for releasing resources used to store {\em compiled quality expressions}
(see \latexhtml{Section~\ref {SEC:QFIND}}{\htmlref{ a description 
here}{SEC:QFIND}}). IRQ\_RLSE is the only routine which needs
to be called to release resources used for accessing {\em quality names}.


\subsection{Using previously initialised NDFs within IRQ}

If the NDF has been initialised for use by IRQ, then the structure
holding the quality names information must be found before other IRQ
routines can be called. Routine \htmlref{IRQ\_FIND}{IRQ_FIND} does this.
This routine looks through all the NDF extensions until a suitable
structure is found. If no such structure is found the status value
{\tt IRQ\_\_NOQNI} is returned. If more than one extension contains such a
structure, then the status value {\tt IRQ\_\_MULT} is returned. If the routine
runs successfully, then an array of HDS locators is returned similar to
the array returned by IRQ\_NEW. Once access to the quality names
information is no longer required, the resources used by these HDS
locators (including the cloned NDF identifier) should be annulled by
calling \htmlref{IRQ\_RLSE}{IRQ_RLSE}. 

\subsection{Accessing the quality names information stored in an NDF}

To access the quality names information stored within an NDF, a set of
HDS locators to the information must first be obtained by calling either
\htmlref{IRQ\_NEW}{IRQ_NEW} or \htmlref{IRQ\_FIND}{IRQ_FIND}\latex{ (see
above)}. Once this has been done the information can be read, modified,
or added to. \htmlref{IRQ\_ADDQN}{IRQ_ADDQN} adds a new quality name
definition to an NDF (so long as there is room for it within the
structures and the QUALITY component). \htmlref{IRQ\_CHKQN}{IRQ_CHKQN}
checks to see if a given name is defined. \htmlref{IRQ\_GETQN}{IRQ_GETQN}
searches for a specified quality name, and returns various items of
information about it (such as which bit of the QUALITY component it is
assigned to). \htmlref{IRQ\_NUMQN}{IRQ_NUMQN} returns the number of
defined quality names. \htmlref{IRQ\_NXTQN}{IRQ_NXTQN} returns the next
defined quality name. Repeated calls to IRQ\_NXTQN can be made to get a
list of all the quality names defined within an NDF.
\htmlref{IRQ\_REMQN}{IRQ_REMQN} removes the definition of a quality name
from an NDF, so long as the quality name has not been flagged as
`read-only' (see \htmlref{IRQ\_RWQN}{IRQ_RWQN}).

\subsection{Assigning and removing qualities to and from NDF pixels}

Once quality names have been defined, they can be assigned to selected
pixels within the NDF. There are two ways of specifying which pixels are
`selected'. Routine \htmlref{IRQ\_SETQM}{IRQ_SETQM} requires a mask image
to be provided in which the `bad' pixels define the pixels in the NDF
DATA component to which the quality is to be assigned. Routine 
\htmlref{IRQ\_SETQL}{IRQ_SETQL} requires a list of pixel indices to be
provided which defines the pixels in the NDF DATA component to which the
quality is to be assigned. The quality of the NDF pixels which are {\em
not} selected is left unchanged. Thus, if an unselected pixel already has
the specified quality, these routines will not {\em remove} the quality
from those pixels. 

The routines \htmlref{IRQ\_RESQM}{IRQ_RESQM} and
\htmlref{IRQ\_RESQL}{IRQ_RESQL}, are complementary to
\htmlref{IRQ\_SETQM}{IRQ_SETQM} and \htmlref{IRQ\_SETQL}{IRQ_SETQL}.
Instead of ensuring that the selected pixels hold the specified quality,
these routines ensure that the selected pixels {\em do not} hold the
specified quality. Again, the quality of unselected pixels is left
unchanged.

Routine \htmlref{IRQ\_SETQ}{IRQ_SETQ} ensures that {\em all} pixels in an
NDF have a specified quality, and routine IRQ\_RESQ ensures that {\em no}
pixels in an NDF have a specified quality. 

The routine \htmlref{IRQ\_CNTQ}{IRQ_CNTQ} will count the number of bits
set in each bit plane in the QUALITY component. 

\subsection{\xlabel{SEC:QFIND}Finding NDF pixels which satisfy a quality expression}
\label {SEC:QFIND}

If a subset of pixels are to be operated on by an application, the
application will usually obtain a quality expression from the user which
defines the pixels to be operated on. That is to say, pixels which have
qualities which do not satisfy the quality expression would not usually
be used by the application. The quality expression is usually obtained
from the user using the \xref{ADAM parameter system}{sun114}{}. It is
then passed to \htmlref{IRQ\_COMP}{IRQ_COMP} to be compiled. If any
problems are detected with the expression (such as syntax errors, or
undefined quality names) IRQ\_COMP will report an error, and it is then
up to the applications to decide what to do. Typically, it will flush the
error, and reprompt the user for a new quality expression. The routine
\htmlref{IRQ\_SYNTX}{IRQ_SYNTX} will check a quality expression for
syntax errors without actually compiling it or checking that the
referenced quality names are defined.

If IRQ\_COMP successfully compiles the quality expression, it returns an
identifier for the compiled quality expression. This identifier is
actually a pointer to a temporary HDS structure holding information about
the quality expression. The identifier is passed to
\htmlref{IRQ\_SBAD}{IRQ_SBAD} which locates all NDF pixels that have
qualities that satisfy the quality expression. The corresponding pixels
in an array supplied to IRQ\_SBAD are set `bad'. The other pixels are
left unchanged. Alternatively, pixels with qualities which {\em do not}
satisfy the quality expression can be set bad in the supplied array.
Note, the array supplied to IRQ\_SBAD must be the same shape and size as
the NDF supplied to \htmlref{IRQ\_FIND}{IRQ_FIND} or
\htmlref{IRQ\_NEW}{IRQ_NEW}. If an NDF section was used, then the array
supplied to IRQ\_SBAD must be the same shape and size as the NDF section,
{\em not} the base NDF. 

Up to ten compiled quality expressions can be active at once. The
identifiers for each compiled quality expression should be annulled
when it is no longer needed by calling
\htmlref{IRQ\_ANNUL}{IRQ_ANNUL}. \htmlref{IRQ\_CLOSE}{IRQ_CLOSE}
should be called when all compiled quality expressions have been
finished with. 

\section {Compiling and Linking with IRQ}
\label{SEC:LINK}
This section describes how to compile and link applications which use IRQ
subroutines, on UNIX systems. It is assumed that the IRQ library
is installed as part of the Starlink Software Collection.


\subsection{Standalone Applications}
Standalone applications which use IRQ\_ routines may be linked by
including execution of the command ``irq\_link'' on the compiler
command line. Thus, to compile and link a Fortran application called
``prog'', the following might be used:

\small
\begin{verbatim}
   % f77 -I$STARLINK_DIR/include prog.f -L$STARLINK_DIR/lib `irq_link` -o prog
\end{verbatim}
\normalsize

Note the use of backward quote characters, which cause the
``irq\_link'' command to be executed and its result substituted into
the compiler command.

\subsection{\label{ss:buildingadamapplications}ADAM Applications}
Users of the \xref{ADAM}{sg4}{} programming environment
\latex{(SG/4)} should use
the \xref{{\bf alink}}{sun144}{ADAM_link_scripts} command
(\xref{SUN/144}{sun144}{}) to compile and link applications, and can
access the IRQ\_ library by including execution of the command
{\tt irq\_link\_adam} on the command line, as follows:

\small
\begin{verbatim}
   % alink adamprog.f `irq_link_adam`
\end{verbatim}
\normalsize

where {\tt adamprog.f} is the Fortran source file for the A-TASK. Again
note the use of opening apostrophies (`) instead of the more usual
closing apostrophy (') in the above {\bf alink} command.

To build a program written in C (instead of Fortran), simply name the
source file {\tt adamprog.c}, instead of {\tt adamprog.f}.

\newpage
\appendix
\section {Routine Descriptions}
% Command for displaying routines in routine lists:
% =================================================

\noteroutine{IRQ\_ADDQN( LOCS, QNAME, DEFLT, COMMNT, STATUS )}
   {Define a new quality name.}
\noteroutine{IRQ\_ANNUL( IDQ, STATUS )}
   {Annul an identifier for a compiled quality expression.}
\noteroutine{IRQ\_CHKQN( LOCS, QNAME, THERE, STATUS )}
   {Check that a specified quality name is defined.}
\noteroutine{IRQ\_CLOSE( STATUS )}
   {Close down the compiled quality expression identifier system.}
\noteroutine{IRQ\_CNTQ( LOCS, SIZE, SET, STATUS )}
   {Count the number of pixels with each bit set in the QUALITY component.}
\noteroutine{IRQ\_COMP( LOCS, SIZE, INFO, QEXP, UNDEF, NUNDEF, ERRPNT, IDQ, STATUS )}
   {Compile a quality expression.}
\noteroutine{CALL IRQ\_DELET( INDF, STATUS )}
   {Delete all quality-name information from an NDF.}
\noteroutine{IRQ\_FIND( INDF, LOCS, XNAME, STATUS )}
   {Find a structure containing quality names information.}
\noteroutine{IRQ\_FXBIT( LOCS, QNAME, BIT, SET, FIXBIT, STATUS )}
   {Assign a fixed bit number to a quality name.}
\noteroutine{IRQ\_GETQN( LOCS, QNAME, FIXED, VALUE, BIT, COMMNT, STATUS )}
   {Get information about a specified quality name.}
\noteroutine{CALL IRQ\_GETQX( PARAM, QEXP, STATUS )}
   {Get a quality expression from the user and check for syntax errors.}
\noteroutine{IRQ\_NEW( INDF, XNAME, LOCS, STATUS )}
   {Create a structure to hold quality names information.}
\noteroutine{IRQ\_NUMQN( LOCS, NAMES, STATUS )}
   {Return the number of defined quality names.}
\noteroutine{IRQ\_NXTQN( LOCS, CONTXT, QNAME, FIXED, VALUE, BIT, COMMNT, DONE, STATUS )}
   {Return information about the next defined quality name.}
\noteroutine{IRQ\_RBIT( LOCS, QNAME, BIT, STATUS )}
   {Reserve a bit number for a given quality name.}
\noteroutine{IRQ\_REMQN( LOCS, QNAME, STATUS )}
   {Remove the definition of a quality name.}
\noteroutine{IRQ\_RESQ( LOCS, QNAME, STATUS )}
   {Ensure no pixels hold a specified quality.}
\noteroutine{IRQ\_RESQL( LOCS, LISTED, QNAME, NDIM, NCOORD, LIST, SET, STATUS~)}
   {Ensure pixels selected by a list do not hold a specified quality.}
\noteroutine{IRQ\_RESQM( LOCS, BAD, QNAME, SIZE, MASK, SET, STATUS )}
   {Ensure pixels selected by a mask do not hold a specified quality.}
\noteroutine{IRQ\_RLSE( LOCS, STATUS )}
   {Release the resource used to locate quality name information.}
\noteroutine{IRQ\_RWQN( LOCS, QNAME, SET, NEWVAL, OLDVAL, STATUS )}
   {Get and/or set the read-only flag for a quality name.}
\noteroutine{IRQ\_SBADx( IDQ, HELD, SIZE, VEC, ALLBAD, NOBAD, STATUS )}
   {Set pixels bad which satisfy a given quality expression.}
\noteroutine{IRQ\_SETQ( LOCS, QNAME, STATUS )}
   {Ensure all pixels hold a specified quality.}
\noteroutine{IRQ\_SETQL( LOCS, LISTED, QNAME, NDIM, NCOORD, LIST, SET,
STATUS~)}
   {Ensure pixels selected by a list hold a specified quality.}
\noteroutine{IRQ\_SETQM( LOCS, BAD, QNAME, SIZE, MASK, SET, STATUS )}
   {Ensure pixels selected by a mask hold a specified quality.}
\noteroutine{IRQ\_SYNTX( QEXP, ERRPNT, STATUS )}
   {Check a quality expression for syntax errors.}

\newpage
\section {Classified List}

\subsection{Gaining Access to Quality Name Information Within an NDF}
\noteroutine{IRQ\_FIND( INDF, LOCS, XNAME, STATUS )}
   {Find a structure containing quality names information.}
\noteroutine{IRQ\_NEW( INDF, XNAME, LOCS, STATUS )}
   {Create a structure to hold quality names information.}
\noteroutine{IRQ\_RLSE( LOCS, STATUS )}
   {Release the resource used to locate quality name information.}

\subsection{Storing, Retrieving and Deleting Quality Names}
\noteroutine{IRQ\_ADDQN( LOCS, QNAME, DEFLT, COMMNT, STATUS )}
   {Define a new quality name.}
\noteroutine{IRQ\_CHKQN( LOCS, QNAME, THERE, STATUS )}
   {Check that a specified quality name is defined.}
\noteroutine{CALL IRQ\_DELET( INDF, STATUS )}
   {Delete all quality-name information from an NDF}
\noteroutine{IRQ\_FXBIT( LOCS, QNAME, BIT, SET, FIXBIT, STATUS )}
   {Assign a fixed bit number to a quality name.}
\noteroutine{IRQ\_GETQN( LOCS, QNAME, FIXED, VALUE, BIT, COMMNT, STATUS )}
   {Get information about a specified quality name.}
\noteroutine{IRQ\_NUMQN( LOCS, NAMES, STATUS )}
   {Return the number of defined quality names.}
\noteroutine{IRQ\_NXTQN( LOCS, CONTXT, QNAME, FIXED, VALUE, BIT, COMMNT, DONE, STATUS )}
   {Return information about the next defined quality name.}
\noteroutine{IRQ\_RBIT( LOCS, QNAME, BIT, STATUS )}
   {Reserve a bit number for a given quality name.}
\noteroutine{IRQ\_REMQN( LOCS, QNAME, STATUS )}
   {Remove the definition of a quality name.}
\noteroutine{IRQ\_RWQN( LOCS, QNAME, SET, NEWVAL, OLDVAL, STATUS )}
   {Get and/or set the read-only flag for a quality name.}

\subsection{Handling Quality Expressions}
\noteroutine{IRQ\_ANNUL( IDQ, STATUS )}
   {Annul an identifier for a compiled quality expression.}
\noteroutine{IRQ\_CLOSE( STATUS )}
   {Close down the compiled quality expression identifier system.}
\noteroutine{IRQ\_COMP( LOCS, SIZE, INFO, QEXP, UNDEF, NUNDEF, ERRPNT, IDQ, STATUS )}
   {Compile a quality expression.}
\noteroutine{IRQ\_GETQN( LOCS, QNAME, FIXED, VALUE, BIT, COMMNT, STATUS )}
   {Get information about a specified quality name.}
\noteroutine{IRQ\_SBADx( IDQ, HELD, SIZE, VEC, ALLBAD, NOBAD, STATUS )}
   {Set pixels bad which satisfy a given quality expression.}
\noteroutine{IRQ\_SYNTX( QEXP, ERRPNT, STATUS )}
   {Check a quality expression for syntax errors.}

\subsection{Assigning Qualities to Selected Pixels}
\noteroutine{IRQ\_RESQ( LOCS, QNAME, STATUS )}
   {Ensure no pixels hold a specified quality.}
\noteroutine{IRQ\_RESQL( LOCS, LISTED, QNAME, NDIM, NCOORD, LIST, SET,
STATUS~)}
   {Ensure pixels selected by a list do not hold a specified quality.}
\noteroutine{IRQ\_RESQM( LOCS, BAD, QNAME, SIZE, MASK, SET, STATUS )}
   {Ensure pixels selected by a mask do not hold a specified quality.}
\noteroutine{IRQ\_SETQ( LOCS, QNAME, STATUS )}
   {Ensure all pixels hold a specified quality.}
\noteroutine{IRQ\_SETQL( LOCS, LISTED, QNAME, NDIM, NCOORD, LIST, SET,
STATUS~)}
   {Ensure pixels selected by a list hold a specified quality.}
\noteroutine{IRQ\_SETQM( LOCS, BAD, QNAME, SIZE, MASK, SET, STATUS )}
   {Ensure pixels selected by a mask hold a specified quality.}

\subsection{Enquiring Pixel Quality}
\noteroutine{IRQ\_CNTQ( LOCS, SIZE, SET, STATUS )}
   {Count the number of pixels with each bit set in the QUALITY component.}

\newpage
\section {\xlabel{SEC:FULLSPEC}Full Routine Specifications}
\label {SEC:FULLSPEC}

\sstroutine{
   IRQ\_ADDQN
}{
   Define a new quality name
}{
   \sstdescription{
      This routine adds the quality name specified by QNAME to the NDF
      specified by LOCS. LOCS must previously have been assigned values
      by one of the routines \htmlref{IRQ\_FIND}{IRQ_FIND} or 
      \htmlref{IRQ\_NEW}{IRQ_NEW}. If the quality name
      is already defined, an error is reported. Note, this routine does
      not reserve a bit in the QUALITY component for the new quality
      name, it mearly established a default value for the quality which
      will be used for all pixels in the NDF if no subsequent call to
      \htmlref{IRQ\_SETQC}{IRQ_SETQC} is made. Note, the string {\tt ANY} 
      cannot be used as a quality name. Also, quality names may not contain 
      any full stops.

      An error is reported if only READ access is available to the NDF.
   }
   \sstinvocation{
      CALL IRQ\_ADDQN( LOCS, QNAME, DEFLT, COMMNT, STATUS )
   }
   \sstarguments{
      \sstsubsection{
         LOCS(5) = CHARACTER $*$ ( $*$ ) (Given)
      }{
         An array of five HDS locators. These locators identify the NDF
         and the associated quality name information.  They should have
         been obtained using routine IRQ\_FIND or routine IRQ\_NEW.
      }
      \sstsubsection{
         QNAME = CHARACTER $*$ ( $*$ ) (Given)
      }{
         The new quality name to store. The maximum length of this
         string is given by symbolic constant IRQ\_\_SZQNM which currently
         has the value 15. Leading spaces are ignored, and the stored
         name is converted to upper case.
      }
      \sstsubsection{
         DEFLT = LOGICAL (Given)
      }{
         If {\tt true}, then by default all pixels are assumed to hold the
         quality specified by QNAME. If {\tt false}, then it is assumed that
         no pixels hold the quality.
      }
      \sstsubsection{
         COMMNT = CHARACTER $*$ ( $*$ ) (Given)
      }{
         A descriptive comment to store with the quality name. The
         maximum length of this string is given by symbolic constant
         IRQ\_\_SZCOM, which currently has the value 50. Any characters
         beyond this length are ignored.
      }
      \sstsubsection{
         STATUS = INTEGER (Given and Returned)
      }{
         The global status.
      }
   }
}
\sstroutine{
   IRQ\_ANNUL
}{
   Release an IRQ identifier
}{
   \sstdescription{
      All internal resources used by the specified compiled quality
      expression identifier (created by \htmlref{IRQ\_COMP}{IRQ_COMP}) are released.

      This routine attempts to execute even if STATUS is bad on entry,
      although no further error report will be made if it subsequently
      fails under these circumstances.
   }
   \sstinvocation{
      CALL IRQ\_ANNUL( IDQ, STATUS )
   }
   \sstarguments{
      \sstsubsection{
         IDQ = INTEGER (Given)
      }{
         An IRQ identifier for a compiled quality expression.
      }
      \sstsubsection{
         STATUS = INTEGER (Given and Returned)
      }{
         The global status.
      }
   }
}
\sstroutine{
   IRQ\_CHKQN
}{
   Check a specified quality name to see if it is defined
}{
   \sstdescription{
      This routine searches for a specified quality name in the quality
      name specified by LOCS. If it is found, THERE is returned {\tt true}.
      Otherwise THERE is returned {\tt false}.
   }
   \sstinvocation{
      CALL IRQ\_CHKQN( LOCS, QNAME, THERE, STATUS )
   }
   \sstarguments{
      \sstsubsection{
         LOCS(5) = CHARACTER $*$ ( $*$ ) (Given)
      }{
         An array of five HDS locators. These locators identify the NDF
         and the associated quality name information.  They should have
         been obtained using routine \htmlref{IRQ\_FIND}{IRQ_FIND} or 
         routine \htmlref{IRQ\_NEW}{IRQ_NEW}.
      }
      \sstsubsection{
         QNAME = CHARACTER $*$ ( $*$ ) (Given)
      }{
         The quality name to search for. Leading blanks are ignored and
         the search is case-insensitive. The maximum allowed length for
         quality names is given by symbolic constant IRQ\_\_SZQNM which
         currently has the value of 15.
      }
      \sstsubsection{
         THERE = LOGICAL (Returned)
      }{
         If {\tt true}, then the quality name is defined within the NDF
         specified by LOCS. If {\tt false}, then the quality name is
         undefined.
      }
      \sstsubsection{
         STATUS = INTEGER (Given and Returned)
      }{
         The global status.
      }
   }
}
\sstroutine{
   IRQ\_CLOSE
}{
   Close down the IRQ identifier system
}{
   \sstdescription{
      This routine must be called once all use of compiled quality
      expression identifiers (as generated by \htmlref{IRQ\_COMP}{IRQ_COMP}) 
      has been completed.  All internal resources used by any such identifiers
      currently in use are released. Note, this routine does not
      release the locators created by \htmlref{IRQ\_NEW}{IRQ_NEW} or 
      \htmlref{IRQ\_FIND}{IRQ_FIND}. \htmlref{IRQ\_RLSE}{IRQ_RLSE}
      must be called to release these locators.

      This routine attempts to execute even if STATUS is bad on entry,
      although no further error report will be made if it subsequently
      fails under these circumstances.
   }
   \sstinvocation{
      CALL IRQ\_CLOSE( STATUS )
   }
   \sstarguments{
      \sstsubsection{
         STATUS = INTEGER (Given and Returned)
      }{
         The global status.
      }
   }
}
\sstroutine{
   IRQ\_CNTQ
}{
   Count the number of pixels which are set in each bit-plane of the
   QUALITY component
}{
   \sstdescription{
      Each bit plane of the NDF QUALITY component corresponds to a
      different quality, described by a name stored in the quality
      names information structure in an NDF extension.  A pixel is set
      in a bit plane of the QUALITY component if the pixel has the
      quality associated with the bit plane. This routine counts the
      number of such pixels in each bit plane.

      Note, write or update access must be available for the NDF (as
      set up by routine \xref{NDF\_ASSOC}{sun33}{NDF_ASSOC} for instance), 
      and the QUALITY component of the NDF must not be mapped on entry 
      to this routine.
   }
   \sstinvocation{
      CALL IRQ\_CNTQ( LOCS, SIZE, SET, STATUS )
   }
   \sstarguments{
      \sstsubsection{
         LOCS(5) = CHARACTER $*$ ( $*$ ) (Given)
      }{
         An array of five HDS locators. These locators identify the NDF
         and the associated quality name information.  They should have
         been obtained using routine \htmlref{IRQ\_FIND}{IRQ_FIND} or 
         routine \htmlref{IRQ\_NEW}{IRQ_NEW}.
      }
      \sstsubsection{
         SIZE = INTEGER (Given)
      }{
         The number of bit planes for which a count of set pixels is
         required.
      }
      \sstsubsection{
         SET( SIZE ) = INTEGER (Returned)
      }{
         The number of pixels holding the corresponding quality in each
         of bit planes 1 to SIZE. The least-significant bit is Bit 1.
         If SIZE is larger than the number of bit planes in the QUALITY
         component, the unused elements are set to zero.
      }
      \sstsubsection{
         STATUS = INTEGER (Given and Returned)
      }{
         The global status.
      }
   }
}
\sstroutine{
   IRQ\_COMP
}{
   Compile a quality expression
}{
   \sstdescription{
      All the quality names referenced in the given quality expression
      (QEXP) are identified. If all quality names referenced in QEXP are
      defined within the NDF specified in LOCS, then the quality
      expression is \emph{compiled}, \emph{i.e.} converted into a form
      that can be used by \\htmlref{IRQ\_SBADx}{IRQ\_SBADx}. The compiled
      quality expression is identified by the returned IRQ identifier
      which should be released using \htmlref{IRQ\_ANNUL}{IRQ_ANNUL} when
      no longer needed. If any error is reported, then IRQ is returned
      set to the value IRQ\_\_NOID.

      If any quality names referenced in the quality expression are not
      defined in the NDF specified by LOCS, they are returned in UNDEF,
      the number of such undefined quality names is returned in NUNDEF,
      an error is reported and STATUS is returned with value
      IRQ\_\_NOQNM. Additionally, if INFO is true, then a message is
      generated identifying each undefined quality name.

      If any of the STATUS values IRQ\_\_BADSY, IRQ\_\_MSOPT or IRQ\_\_MSOPD
      are returned (all of which correspond to various forms of syntax
      error in the quality expression, 
      \latexhtml{see Appendix~\ref{APP:ERRORS}}{\htmlref{listed here}{APP:ERRORS}}), 
      a pointer to the approximate position of the error within the quality
      expression is returned in ERRPNT.
   }
   \sstinvocation{
      CALL IRQ\_COMP( LOCS, SIZE, INFO, QEXP, UNDEF, NUNDEF, ERRPNT,
                     IDQ, STATUS )
   }
   \sstarguments{
      \sstsubsection{
         LOCS(5) = CHARACTER $*$ ( $*$ ) (Given)
      }{
         An array of five HDS locators. These locators identify the NDF
         and the associated quality name information.  They should have
         been obtained using routine \htmlref{IRQ\_FIND}{IRQ_FIND} or 
         routine \htmlref{IRQ\_NEW}{IRQ_NEW}.
      }
      \sstsubsection{
         SIZE = INTEGER (Given)
      }{
         The size of the UNDEF array. This should be at least equal to
         the value of the symbolic constant IRQ\_\_QNREF
      }
      \sstsubsection{
         INFO = LOGICAL (Given)
      }{
         If set to .TRUE., then messages are produced identifying any
         undefined quality names.
      }
      \sstsubsection{
         QEXP = CHARACTER$*$($*$) (Given and Returned)
      }{
         A quality expression. See 
         \latexhtml{Section~\ref{SEC:QEXP}}{\htmlref{Quality expressions}{SEC:QEXP}}
         for details of the allowed formats for quality expressions.
	 On exit, the string is converted to upper case and any
	 leading blanks are removed.
      }
      \sstsubsection{
         UNDEF( SIZE ) = CHARACTER $*$ ( $*$ ) (Returned)
      }{
         An array holding any undefined quality names referenced in the
         quality expression. The array should have at least IRQ\_\_QNREF
         elements, each element being a string of length IRQ\_\_SZQNM.
      }
      \sstsubsection{
         NUNDEF = INTEGER (Returned)
      }{
         The number of undefined quality names referenced in the quality
         expression.
      }
      \sstsubsection{
         ERRPNT = INTEGER (Returned)
      }{
         If any of the STATUS values IRQ\_\_BADSY( {\tt "}Unrecognised logical
         operator or constant{\tt "}), IRQ\_\_MSOPT ( {\tt "}Missing operator{\tt "}) or
         IRQ\_\_MSOPD ( {\tt "}Missing operand{\tt "}) are returned, then ERRPNT
         returns the offset within the quality expression at which the
         error was detected. Note, the offset refers to the returned
         form of QEXP, not the given form. These will be different if
         the given form of QEXP has any leading blanks. An offset of
         zero is returned if none of the errors associated with the
         above STATUS values occur.
      }
      \sstsubsection{
         IDQ = INTEGER (Returned)
      }{
         An IRQ identifier for the compiled quality expression. This
         identifier can be passed to IRQ\_SBADx. This identifier should
         be annulled using routine \htmlref{IRQ\_ANNUL}{IRQ_ANNUL} or 
         \htmlref{IRQ\_CLOSE}{IRQ_CLOSE} when it is no
         longer needed.  If an error is reported, then an invalid
         identifier (equal to IRQ\_\_NOID) is returned.
      }
      \sstsubsection{
         STATUS = INTEGER (Given and Returned)
      }{
         The global status.
      }
   }
}
\newpage
\sstroutine{
   IRQ\_DELET
}{
   Delete all quality-name information from an NDF
}{
   \sstdescription{
      A search is made through the extensions contained within the
      supplied NDF for an HDS structure containing quality name
      information. If found, the QUALITY\_NAMES structure containing the
      quality names is deleted.
   }
   \sstinvocation{
      CALL IRQ\_DELET( INDF, STATUS )
   }
   \sstarguments{
      \sstsubsection{
         INDF = INTEGER (Given)
      }{
         The input NDF.
      }
      \sstsubsection{
         STATUS = INTEGER (Given and Returned)
      }{
         The global status.
      }
   }
}
\sstroutine{
   IRQ\_FIND
}{
   Find quality name information within an NDF
}{
   \sstdescription{
      A search is made through the extensions contained within the
      supplied NDF for an HDS structure containing quality name
      information. Such information is held in an HDS object named
      QUALITY\_NAMES (these objects can be created using
      \htmlref{IRQ\_NEW}{IRQ_NEW}).  If no such object is found, an
      error is reported and the status IRQ\_\_NOQNI is returned (if
      more than one such object is found, an error is reported and the
      status IRQ\_\_MULT is returned).  The name of the NDF extension
      in which the object was found is returned in XNAME. An array of
      five HDS locators is returned which is needed when calling other
      IRQ routines. The first locator points to a temporary object
      which holds a cloned identifier for the NDF, the other four point
      to components of the QUALITY\_NAMES structure contained in the
      NDF.  \htmlref{IRQ\_RLSE}{IRQ_RLSE} should be called to annul
      these locators (and the NDF identifier) when no further access
      to the NDFs quality names information is required.

      The LOCS argument returned by this routine specifies the NDF
      which will be operated on by subsequent IRQ routines.
      Specifically, LOCS determines the bounds of the NDF. Care should
      therefore be taken that subsequent calls to IRQ routines refer to
      the NDF specified by the INDF argument to this routine, and not
      for instance to a section of the NDF which will in general have
      different bounds.
   }
   \sstinvocation{
      CALL IRQ\_FIND( INDF, LOCS, XNAME, STATUS )
   }
   \sstarguments{
      \sstsubsection{
         INDF = INTEGER (Given)
      }{
         The input NDF.
      }
      \sstsubsection{
         LOCS(5) = CHARACTER $*$ ( $*$ ) (Returned)
      }{
         A set of HDS locators as described above.  The character
         variables supplied for this argument should have a declared
         length equal to symbolic constant DAT\_\_SZLOC. These locator
         are annulled by calling \htmlref{IRQ\_RLSE}{IRQ_RLSE}.
      }
      \sstsubsection{
         XNAME = CHARACTER $*$ ( $*$ ) (Returned)
      }{
         The name of the NDF extension in which the quality name
         information was found. The character variable supplied for
         this argument should have a declared length equal to symbolic
         constant DAT\_\_SZNAM.
      }
      \sstsubsection{
         STATUS = INTEGER (Given and Returned)
      }{
         The global status.
      }
   }
}
\sstroutine{
   IRQ\_FXBIT
}{
   Assign a fixed bit number to a quality name
}{
   \sstdescription{
      This routine associates a fixed bit number with a specified quality
      name. Normally, IRQ manages the allocation of bit numbers to named
      qualities, but this routine allows the calling application to specify
      which bit is to be used for a given quality.

      By default, a QUALITY-array bit is associated with a quality name
      only if some pixels hold the quality and some do not hold the
      quality (\emph{i.e.} there is a mix of values). Otherwise, a flag is stored
      in the QUALITY\_NAMES structure indicating this, and any quality bit
      previously associated with the quality name is released for re-use.

      This default behaviour is changed by calling this routine. The
      specified bit number will continue to be associated with the quality
      name even if all pixels do, or do not, hold the quality.

      An error will be returned if the named quality is already associated
      with a different bit number when this routine is called. An error
      will also be reported if the specified bit number is already associated
      with a different quality name.
   }
   \sstinvocation{
      CALL IRQ\_FXBIT( LOCS, QNAME, BIT, SET, FIXBIT, STATUS )
   }
   \sstarguments{
      \sstsubsection{
         LOCS(5) = CHARACTER $*$ ( $*$ ) (Given)
      }{
         An array of five HDS locators. These locators identify the NDF
         and the associated quality name information.  They should have
         been obtained using routine \htmlref{IRQ\_FIND}{IRQ_FIND} or 
         routine \htmlref{IRQ\_NEW}{IRQ_NEW}.
      }
      \sstsubsection{
         QNAME = CHARACTER $*$ ( $*$ ) (Given)
      }{
         The quality name to use. Leading blanks are ignored and
         the search is case-insensitive. The maximum allowed length for
         quality names is given by symbolic constant IRQ\_\_SZQNM which
         currently has the value of 15.
      }
      \sstsubsection{
         BIT = INTEGER (Given)
      }{
         The bit number to use. The least-significant bit is Bit 1, not
         Bit 0. If value below 0 or above 8 is supplied, the properties
         of the quality name are left unchanged, but the FIXBIT value is
         still returned.
      }
      \sstsubsection{
         FIXBIT = LOGICAL (Returned)
      }{
         Returned {\tt .TRUE.} if the specified quality name had a fixed bit
         number on entry to this routine, and {\tt .FALSE.} otherwise.
      }
      \sstsubsection{
         STATUS = INTEGER (Given and Returned)
      }{
         The global status.
      }
   }
}
\sstroutine{
   IRQ\_GETQN
}{
   Search for a specified quality name
}{
   \sstdescription{
      This routine searches for a specified quality name in the quality
      name specified by LOCS, and returns information related to the
      quality name. If the quality name is not defined then an error is
      reported and STATUS returned equal to IRQ\_\_NOQNM.
   }
   \sstinvocation{
      CALL IRQ\_GETQN( LOCS, QNAME, FIXED, VALUE, BIT, COMMNT, STATUS )
   }
   \sstarguments{
      \sstsubsection{
         LOCS(5) = CHARACTER $*$ ( $*$ ) (Given)
      }{
         An array of five HDS locators. These locators identify the NDF
         and the associated quality name information.  They should have
         been obtained using routine \htmlref{IRQ\_FIND}{IRQ_FIND} or 
         routine \htmlref{IRQ\_NEW}{IRQ_NEW}.
      }
      \sstsubsection{
         QNAME = CHARACTER $*$ ( $*$ ) (Given)
      }{
         The quality name to search for. Leading blanks are ignored and
         the search is case-insensitive. The maximum allowed length for
         quality names is given by symbolic constant IRQ\_\_SZQNM which
         currently has the value of 15.
      }
      \sstsubsection{
         FIXED = LOGICAL (Returned)
      }{
         If {\tt true}, then the quality is either held by all pixels, or by
         no pixels. In this case the quality may not have a
         corresponding bit in the QUALITY component. If {\tt false}, then
         some pixels have the quality and some do not, as indicated by
         the corresponding bit in the QUALITY component.
      }
      \sstsubsection{
         VALUE = LOGICAL (Returned)
      }{
         If FIXED is {\tt true}, then VALUE specifies whether all pixels hold
         the quality (VALUE = .TRUE.), or whether no pixels hold the
         quality (VALUE = .FALSE.). If FIXED is {\tt false}, then VALUE is
         indeterminate.
      }
      \sstsubsection{
         BIT = INTEGER (Returned)
      }{
         BIT holds the corresponding bit number in the QUALITY component.
         The least-significant bit is called Bit 1 (not Bit 0). If there
         is no corresponding bit, a value of zero is returned, and FIXED
         is returned {\tt .TRUE.}.
      }
      \sstsubsection{
         COMMNT = CHARACTER $*$ ( $*$ ) (Returned)
      }{
         The descriptive comment which was stored with the quality name.
         The supplied character variable should have a declared length
         given by symbolic constant IRQ\_\_SZCOM.
      }
      \sstsubsection{
         STATUS = INTEGER (Given and Returned)
      }{
         The global status.
      }
   }
}
\sstroutine{
   IRQ\_GETQX
}{
   Get a quality expression from the user and check for syntax errors
}{
   \sstdescription{
      A string is obtained from the environment using the supplied ADAM
      parameter. This string is check to see if it has correct syntax
      for a quality expression (no checks are made to ensure that the
      quality names referenced within the expression are defined within
      any specific NDF). If a syntax error is detected, the quality
      expression is displayed with an exclamation mark under the
      position at which the syntax error was detected. If any problem is
      found with the supplied expression, a new value is obtained from
      the environment. The returned string is converted to upper case
      and leading blanks are removed.
   }
   \sstinvocation{
      CALL IRQ\_GETQX( PARAM, QEXP, STATUS )
   }
   \sstarguments{
      \sstsubsection{
         PARAM = CHARACTER $*$ ( $*$ ) (Given)
      }{
         The name of an ADAM parameter of type LITERAL.
      }
      \sstsubsection{
         QEXP = CHARACTER $*$ ( $*$ ) (Returned)
      }{
         The returned quality expression.
      }
      \sstsubsection{
         STATUS = INTEGER (Given and Returned)
      }{
         The global status.
      }
   }
}
\sstroutine{
   IRQ\_NEW
}{
   Create a new structure to hold quality name information within an
   NDF extension
}{
   \sstdescription{
      An HDS object (named QUALITY\_NAMES) is created to hold quality
      name information within the specified NDF extension. An error is
      reported if the NDF extension does not exist. If the extension does
      exist, an array of five HDS locators is returned which is needed
      when calling other IRQ routines. The first locator points to a
      temporary object which holds a cloned identifier for the NDF,
      the other four point to components of the QUALITY\_NAMES structure
      contained in the NDF. \htmlref{IRQ\_RLSE}{IRQ_RLSE} should be 
      called to annul these locators (and the NDF identifier) when no 
      further access to the NDFs quality names information is required.

      The QUALITY component of the NDF is reset to an undefined state by
      this routine. Therefore, the QUALITY component should not be mapped
      for access prior to calling this routine.

      The LOCS argument returned by this routine specifies the NDF
      which will be operated on by subsequent IRQ routines.
      Specifically, LOCS determines the bounds of the NDF. Care should
      therefore be taken that subsequent calls to IRQ routines refer to
      the NDF specified by the INDF argument to this routine, and not
      for instance to a section of the NDF which will in general have
      different bounds.

      Note, an error is reported if only READ access is available to the
      NDF.
   }
   \sstinvocation{
      CALL IRQ\_NEW( INDF, XNAME, LOCS, STATUS )
   }
   \sstarguments{
      \sstsubsection{
         INDF = INTEGER (Given)
      }{
         The NDF identifier.
      }
      \sstsubsection{
         XNAME = CHARACTER $*$ ( $*$ ) (Given)
      }{
         The name of the NDF extension in which the quality name
         information is to be stored. If this extension does not exist
         then an error is reported.
      }
      \sstsubsection{
         LOCS(5) = CHARACTER $*$ ( $*$ ) (Returned)
      }{
         A set of HDS locators as described above.  The character
         variables supplied for this argument should have a declared
         length equal to symbolic constant DAT\_\_SZLOC. These locator
         are annulled by calling \htmlref{IRQ\_RLSE}{IRQ_RLSE}.
      }
      \sstsubsection{
         STATUS = INTEGER (Given and Returned)
      }{
         The global status.
      }
   }
}
\sstroutine{
   IRQ\_NUMQN
}{
   Return number of defined quality names
}{
   \sstdescription{
      The number of quality names defined in the NDF specified by
      LOCS is returned.
   }
   \sstinvocation{
      CALL IRQ\_NUMQN( LOCS, NAMES, STATUS )
   }
   \sstarguments{
      \sstsubsection{
         LOCS(5) = CHARACTER $*$ ( $*$ ) (Given)
      }{
         An array of five HDS locators. These locators identify the NDF
         and the associated quality name information.  They should have
         been obtained using routine \htmlref{IRQ\_FIND}{IRQ_FIND} or
         routine \htmlref{IRQ\_NEW}{IRQ_NEW}.
      }
      \sstsubsection{
         NAMES = INTEGER (Returned)
      }{
         The number of quality names defined in the structure located by
         LOCS.
      }
      \sstsubsection{
         STATUS = INTEGER (Given and Returned)
      }{
         The global status.
      }
   }
}
\sstroutine{
   IRQ\_NXTQN
}{
   Return the next quality name
}{
   \sstdescription{
      This routine returns the next quality name defined in the NDF
      specified by LOCS, together with supplementary information.  The
      next quality name is determined by the value of CONTXT. If CONTXT
      is zero on entry then the first quality name is returned.  On
      exit, CONTXT is set to a value which indicates where the next
      quality name is stored within the NDF. This value can be passed
      to a subsequent call to this routine to retrieve information
      about the next quality name.
   }
   \sstinvocation{
      CALL IRQ\_NXTQN( LOCS, CONTXT, QNAME, FIXED, VALUE, BIT,
                      COMMNT, DONE, STATUS )
   }
   \sstarguments{
      \sstsubsection{
         LOCS(5) = CHARACTER $*$ ( $*$ ) (Given)
      }{
         An array of five HDS locators. These locators identify the NDF
         and the associated quality name information.  They should have
         been obtained using routine \htmlref{IRQ\_FIND}{IRQ_FIND} or 
         routine \htmlref{IRQ\_NEW}{IRQ_NEW}.
      }
      \sstsubsection{
         CONTXT = INTEGER (Given and Returned)
      }{
         The context of the current call. This should be set to zero
         before the first call to this routine, and then left unchanged
         between subsequent calls.
      }
      \sstsubsection{
         QNAME = CHARACTER $*$ ( $*$ ) (Returned)
      }{
         The next quality name. The character variable supplied for this
         argument should have a declared length equal to the symbolic
         constant IRQ\_\_SZQNM.
      }
      \sstsubsection{
         FIXED = LOGICAL (Returned)
      }{
         If {\tt true}, then the quality is either held by all pixels, or by
         no pixels. In this case the quality may not have a
         corresponding bit in the QUALITY component. If {\tt false}, then
         some pixels have the quality and some do not, as indicated by
         the corresponding bit in the QUALITY component.
      }
      \sstsubsection{
         VALUE = LOGICAL (Returned)
      }{
         If FIXED is {\tt true}, then VALUE specifies whether all pixels hold
         the quality (VALUE = .TRUE.), or whether no pixels hold the
         quality (VALUE = .FALSE.). If FIXED is {\tt false}, then VALUE is
         indeterminate.
      }
      \sstsubsection{
         BIT = INTEGER (Returned)
      }{
         BIT holds the corresponding bit number in the QUALITY component.
         The least-significant bit is called Bit 1 (not Bit 0).  A value
         of zero is returned if the quality has no associated bit in the
         quality array. In this case, the FIXED argument will indicate if
         all pixels do, or do not, hold the quality.
      }
      \sstsubsection{
         COMMNT = CHARACTER $*$ ( $*$ ) (Returned)
      }{
         The descriptive comment which was stored with the quality name.
         The supplied character variable should have a declared length
         given by symbolic constant IRQ\_\_SZCOM.
      }
      \sstsubsection{
         DONE = LOGICAL (Returned)
      }{
         Returned true if this routine is called when no more names
         remain to be returned.
      }
      \sstsubsection{
         STATUS = INTEGER (Given and Returned)
      }{
         The global status.
      }
   }
}
\sstroutine{
   IRQ\_RBIT
}{
   Reserve a bit number for a given quality name
}{
   \sstdescription{
      If the supplied quality name already has a bit number associated
      with it, the bit number is returned. Otherwise, the next available
      plane in the quality array is asigned to the quality name, and its
      bit number is returned.

      Note, write or update access must be available for the NDF (as
      set up by routine \xref{LPG\_ASSOC}{sun263}{LPG_ASSOC} for instance).
   }
   \sstinvocation{
      CALL IRQ\_RBIT( LOCS, QNAME, BIT, STATUS )
   }
   \sstarguments{
      \sstsubsection{
         LOCS(5) = CHARACTER $*$ ( $*$ ) (Given)
      }{
         An array of five HDS locators. These locators identify the NDF
         and the associated quality name information.  They should have
         been obtained using routine \htmlref{IRQ\_FIND}{IRQ_FIND} or
         routine \htmlref{IRQ\_NEW}{IRQ_NEW}.
      }
      \sstsubsection{
         QNAME = CHARACTER $*$ ( $*$ ) (Given)
      }{
         The quality name. This quality name must be defined in the NDF
         specified by LOCS. Name definitions can be added to the NDF
         using routine \htmlref{IRQ\_ADDQN}{IRQ_ADDQN}.
      }
      \sstsubsection{
         BIT = INTEGER (Returned)
      }{
         The bit number used by the quality name within the quality
         array. Note, the least-significant bit is Bit 1, not Bit 0.
      }
      \sstsubsection{
         STATUS = INTEGER (Given and Returned)
      }{
         The global status.
      }
   }
}
\sstroutine{
   IRQ\_REMQN
}{
   Remove the definition of a specified quality name
}{
   \sstdescription{
      The specified quality name is removed from the NDF specified
      by LOCS. Any associated bit in the QUALITY array is freed for
      future use. If the name is not defined an error is reported.
      A value of ANY for the quality names causes all defined quality
      names to be removed.

      Note, an error is reported if only read access is available to the
      NDF, or if the quality name has been flagged as read-only using
      routine IRQ\_RWQN.
   }
   \sstinvocation{
      CALL IRQ\_REMQN( LOCS, QNAME, STATUS )
   }
   \sstarguments{
      \sstsubsection{
         LOCS(5) = CHARACTER $*$ ( $*$ ) (Given)
      }{
         An array of five HDS locators. These locators identify the NDF
         and the associated quality name information.  They should have
         been obtained using routine \htmlref{IRQ\_FIND}{IRQ_FIND} or 
         routine \htmlref{IRQ\_NEW}{IRQ_NEW}.
      }
      \sstsubsection{
         QNAME = CHARACTER $*$ ( $*$ ) (Given)
      }{
         The quality name to remove, or {\tt 'ANY'} if all quality names are
         to be removed.
      }
      \sstsubsection{
         STATUS = INTEGER (Given and Returned)
      }{
         The global status.
      }
   }
}
\sstroutine{
   IRQ\_RESQ
}{
   Remove a given quality from all pixels in the NDF
}{
   \sstdescription{
      The quality specified by QNAME is removed from all pixels in the
      NDF specified by LOCS (LOCS should be obtained either by calling
      \htmlref{IRQ\_FIND}{IRQ_FIND} or \htmlref{IRQ\_NEW}{IRQ_NEW}). 
      An error is reported if the quality name is undefined within the NDF.

      Note, write or update access must be available for the NDF (as
      set up by routine \xref{LPG\_ASSOC}{sun263}{LPG_ASSOC} for instance).
   }
   \sstinvocation{
      CALL IRQ\_RESQ( LOCS, QNAME, STATUS )
   }
   \sstarguments{
      \sstsubsection{
         LOCS(5) = CHARACTER $*$ ( $*$ ) (Given)
      }{
         An array of five HDS locators. These locators identify the NDF
         and the associated quality name information.  They should have
         been obtained using routine IRQ\_FIND or routine IRQ\_NEW.
      }
      \sstsubsection{
         QNAME = CHARACTER $*$ ( $*$ ) (Given)
      }{
         The quality name to be removed from all pixels in the NDF. This
         quality name must be defined in the NDF specified by LOC. Name
         definitions can be added to the NDF using routine 
         \htmlref{IRQ\_ADDQN}{IRQ_ADDQN}.
      }
      \sstsubsection{
         STATUS = INTEGER (Given and Returned)
      }{
         The global status.
      }
   }
}
\sstroutine{
   IRQ\_RESQL
}{
   Remove a quality from a list of pixels, leaving unlisted pixels
   unchanged
}{
   \sstdescription{
      The quality specified by QNAME is removed from all pixels
      included in (or, if LISTED is false, not included in) the
      supplied list of pixel indices. The quality of other pixels
      is left unaltered.  The quality name must be defined in the NDF
      specified by LOCS (LOCS should be obtained either by calling
      \htmlref{IRQ\_FIND}{IRQ_FIND} or \htmlref{IRQ\_NEW}{IRQ_NEW}). 
      An error is reported if the quality name is undefined.

      Note, write or update access must be available for the NDF (as
      set up by routine \xref{NDF\_ASSOC}{sun33}{NDF_ASSOC} for instance), 
      and the QUALITY component must not be mapped on entry to this routine.
   }
   \sstinvocation{
      CALL IRQ\_RESQL( LOCS, LISTED, QNAME, NDIM, NCOORD, LIST, SET,
                      STATUS~)
   }
   \sstarguments{
      \sstsubsection{
         LOCS(5) = CHARACTER $*$ ( $*$ ) (Given)
      }{
         An array of five HDS locators. These locators identify the NDF
         and the associated quality name information.  They should have
         been obtained using routine IRQ\_FIND or routine IRQ\_NEW.
      }
      \sstsubsection{
         LISTED = LOGICAL (Given)
      }{
         If {\tt true}, then the quality is removed from all pixels included
         in the list given by LIST.  If {\tt false}, then the quality is
         removed from all pixels not included in the list given by LIST.
      }
      \sstsubsection{
         QNAME = CHARACTER $*$ ( $*$ ) (Given)
      }{
         The quality name to be removed from the selected pixels. This
         quality name must be defined in the NDF specified by LOC. Name
         definitions can be added to the NDF using routine 
         \htmlref{IRQ\_ADDQN}{IRQ_ADDQN}.
      }
      \sstsubsection{
         NDIM = INTEGER (Given)
      }{
         The number of values required to specify a pixel position
         (\emph{i.e.} the number of dimensions in the NDF).
      }
      \sstsubsection{
         NCOORD = INTEGER (Given)
      }{
         The number of pixels included in the input list.
      }
      \sstsubsection{
         LIST( NDIM, NCOORD ) = INTEGER (Given)
      }{
         The list of pixel indices. Any indices which lie outside the
         bounds of the NDF are ignored.
      }
      \sstsubsection{
         SET = INTEGER (Returned)
      }{
         The number of pixels which hold the quality.
      }
      \sstsubsection{
         STATUS = INTEGER (Given and Returned)
      }{
         The global status.
      }
   }
}
\newpage
\sstroutine{
   IRQ\_RESQM
}{
   Remove a quality from pixels selected using a mask image, leaving
   unselected pixels unchanged
}{
   \sstdescription{
      The quality specified by QNAME is removed from all NDF pixels
      which either do (or, if BAD is {\tt false}, do not) correspond to \emph{bad}
      pixels in the input mask array.  The quality of all other pixels
      is left unchanged.  The quality name must be defined in the NDF
      specified by LOCS (LOCS should be obtained either by calling
      \htmlref{IRQ\_FIND}{IRQ_FIND} or \htmlref{IRQ\_NEW}{IRQ_NEW}). An 
      error is reported if the quality name is undefined.

      Note, write or update access must be available for the NDF (as
      set up by routine \xref{LPG\_ASSOC}{sun263}{LPG_ASSOC} for instance), 
      and the QUALITY component of the NDF must not be mapped on entry to 
      this routine.
   }
   \sstinvocation{
      CALL IRQ\_RESQM( LOCS, BAD, QNAME, SIZE, MASK, SET, STATUS )
   }
   \sstarguments{
      \sstsubsection{
         LOCS(5) = CHARACTER $*$ ( $*$ ) (Given)
      }{
         An array of five HDS locators. These locators identify the NDF
         and the associated quality name information.  They should have
         been obtained using routine IRQ\_FIND or routine IRQ\_NEW.
      }
      \sstsubsection{
         BAD = LOGICAL (Given)
      }{
         If {\tt true}, then the quality is removed from all NDF pixels
         corresponding to `bad' pixels in the mask.  If {\tt false}, then the
         quality is removed from all NDF pixels corresponding to pixels
         which are not `bad' in the mask.
      }
      \sstsubsection{
         QNAME = CHARACTER $*$ ( $*$ ) (Given)
      }{
         The quality name to be removed from the selected pixels. This
         quality name must be defined in the NDF specified by LOC. Name
         definitions can be added to the NDF using routine 
         \htmlref{IRQ\_ADDQN}{IRQ_ADDQN}.
      }
      \sstsubsection{
         SIZE = INTEGER (Given)
      }{
         The total number of pixels in the MASK array.
      }
      \sstsubsection{
         MASK( SIZE ) = REAL (Given)
      }{
         A vector which defines the pixels from which the quality
         specified by QNAME is to be removed.  It is assumed that this
         vector corresponds pixel-for-pixel with the vectorised NDF as
         supplied to routine IRQ\_FIND or IRQ\_NEW.
      }
      \sstsubsection{
         SET = INTEGER (Returned)
      }{
         The number of pixels in the NDF which hold the quality.
      }
      \sstsubsection{
         STATUS = INTEGER (Given and Returned)
      }{
         The global status.
      }
   }
}
\sstroutine{
   IRQ\_RLSE
}{
   Release a temporary structure created by IRQ\_NEW or IRQ\_FIND
}{
   \sstdescription{
      This routine releases the resources reserved by a call to
      \htmlref{IRQ\_NEW}{IRQ_NEW} or \htmlref{IRQ\_FIND}{IRQ_FIND}.
      The cloned NDF identifier held in LOCS(1) is annulled, and then
      all the five HDS locators in LOCS are annulled. If no defined
      quality names exist within the NDF, then the structure used to
      hold such names is deleted and the QUALITY component of the NDF
      is reset to an undefined state.

      Note, this routine attempts to execute even if STATUS is set on
      entry, although no further error report will be made if it
      subsequently fails under these circumstances.
   }
   \sstinvocation{
      CALL IRQ\_RLSE( LOCS, STATUS )
   }
   \sstarguments{
      \sstsubsection{
         LOCS(5) = CHARACTER $*$ ( $*$ ) (Given)
      }{
         An array of five HDS locators. These locators identify the NDF
         and the associated quality name information.  They should have
         been obtained using routine IRQ\_FIND or routine IRQ\_NEW.
      }
      \sstsubsection{
         STATUS = INTEGER (Given and Returned)
      }{
         The global status.
      }
   }
}
\sstroutine{
   IRQ\_RWQN
}{
   Get and/or set the read-only flag for a quality name
}{
   \sstdescription{
      This routine returns the current value of the read-only flag
      associated with a quality name, and optionally assigns a new
      value to the flag.

      If the read-only flag is set for a quality name, any attempt to
      remove the quality name using \htmlref{IRQ\_REMQN}{IRQ_REMQN} 
      will result in an error being reported.
   }
   \sstinvocation{
      CALL IRQ\_RWQN( LOCS, QNAME, SET, NEWVAL, OLDVAL, STATUS )
   }
   \sstarguments{
      \sstsubsection{
         LOCS(5) = CHARACTER $*$ ( $*$ ) (Given)
      }{
         An array of five HDS locators. These locators identify the NDF
         and the associated quality name information.  They should have
         been obtained using routine \htmlref{IRQ\_FIND}{IRQ_FIND} or
         routine \htmlref{IRQ\_NEW}{IRQ_NEW}.
      }
      \sstsubsection{
         QNAME = CHARACTER $*$ ( $*$ ) (Given)
      }{
         The quality name to use. Leading blanks are ignored and
         the search is case-insensitive. The maximum allowed length for
         quality names is given by symbolic constant IRQ\_\_SZQNM which
         currently has the value of 15.
      }
      \sstsubsection{
         SET = LOGICAL (Given)
      }{
         If {\tt true}, then the read-only flag for the quality name will be
         set to the value supplied in NEWVAL. Otherwise, the current
         value of the flag will be left unchanged.
      }
      \sstsubsection{
         NEWVAL = LOGICAL (Given)
      }{
         The new value for the read-only flag. Only accessed if SET is
         {\tt true}.
      }
      \sstsubsection{
         OLDVAL = LOGICAL (Returned)
      }{
         The value of the read-only flag on entry to this routine. If the
         old value is of no interest, it is safe to supply the same
         variable for OLDVAL as for NEWVAL since OLDVAL is updated after
         NEWVAL is used.
      }
      \sstsubsection{
         STATUS = INTEGER (Given and Returned)
      }{
         The global status.
      }
   }
}
\sstroutine{
   IRQ\_SBADx
}{
   Set pixels `bad' which satisfy a given quality expression
}{
   \sstdescription{
      \htmlref{IRQ\_COMP}{IRQ_COMP} should be called before this
      routine to produce the compiled quality expression identified by
      IDQ. The QUALITY component of the NDF to which the quality
      expression refers (see IRQ\_COMP argument LOCS) is mapped as a
      one dimensional vector. The supplied array VEC must correspond
      pixel-for-pixel with the mapped QUALITY vector.  All pixels
      which hold a QUALITY satisfying the quality expression are
      found. If HELD is {\tt true}, then the corresponding pixels in VEC are
      set to the `bad' value (other pixels are left unaltered).
      If HELD is {\tt false}, the corresponding pixels in VEC are left as
      they are, but all the other pixels in VEC are set to the
      `bad' value. ALLBAD and NOBAD indicate if the output VEC
      values are either all bad or all good.

      Note, if the QUALITY component of the NDF is mapped for WRITE or
      UPDATE access on entry to this routine, an error is reported.
   }
   \sstinvocation{
      CALL IRQ\_SBADx( IDQ, HELD, SIZE, VEC, ALLBAD, NOBAD, STATUS )
   }
   \sstarguments{
      \sstsubsection{
         IDQ = INTEGER (Given)
      }{
         An identifier for a compiled quality expression, produced by
         routine IRQ\_COMP. This identifier determines the NDF to which
         the expression refers.
      }
      \sstsubsection{
         HELD = LOGICAL (Given)
      }{
         If {\tt true} then those VEC pixels that hold a quality satisfying
         the supplied quality expression are set `bad'. Otherwise, those
         pixels which do not hold such a quality are set `bad'.
      }
      \sstsubsection{
         SIZE = INTEGER (Given)
      }{
         The total number of pixels in VEC. An error is reported if
         this is not the same as the total number of pixels in the NDF
         determined by IDQ.
      }
      \sstsubsection{
         VEC( SIZE ) = ? (Given and Returned)
      }{
         The data to be set `bad', depending on the corresponding
         quality values stored in the NDF. It must be the same size as
         the NDF, and must correspond pixel-for-pixel with the
         vectorised NDF. Pixels which are not explicitly set `bad' by
         this routine retain the values they had on entry.
      }
      \sstsubsection{
         ALLBAD = LOGICAL (Returned)
      }{
         Returned {\tt true} if all pixels in VEC are returned with `bad'
         values, and {\tt false} if any returned pixel values are not `bad'.
      }
      \sstsubsection{
         NOBAD = LOGICAL (Returned)
      }{
         Returned {\tt true} if no pixels in VEC are returned with `bad'
         values. It is {\tt false} if any `bad' pixel values are returned.
      }
      \sstsubsection{
         STATUS = INTEGER (Given and Returned)
      }{
         The global status.
      }
   }
   \sstnotes{
      \sstitemlist{
         \sstitem
         There is a routine for each numeric data type: replace {\tt "x"} in
         the routine name by D, R, I, W, UW, B or UB as appropriate.  The
         VEC array supplied to the routine must have the data type
         specified.
      }
   }
}
\sstroutine{
   IRQ\_SETQ
}{
   Assign a given quality to all pixels in the NDF
}{
   \sstdescription{
      The quality specified by QNAME is assigned to all pixels in the
      NDF specified by LOCS (LOCS should be obtained either by calling
      \htmlref{IRQ\_FIND}{IRQ_FIND} or \htmlref{IRQ\_NEW}{IRQ_NEW}). 
      An error is reported if the quality name is undefined within the NDF.

      Note, write or update access must be available for the NDF (as
      set up by routine \xref{NDF\_ASSOC}{sun33}{NDF_ASSOC} for instance).
   }
   \sstinvocation{
      CALL IRQ\_SETQ( LOCS, QNAME, STATUS )
   }
   \sstarguments{
      \sstsubsection{
         LOCS(5) = CHARACTER $*$ ( $*$ ) (Given)
      }{
         An array of five HDS locators. These locators identify the NDF
         and the associated quality name information.  They should have
         been obtained using routine IRQ\_FIND or routine IRQ\_NEW.
      }
      \sstsubsection{
         QNAME = CHARACTER $*$ ( $*$ ) (Given)
      }{
         The quality name to assign to all pixels in the NDF. This
         quality name must be defined in the NDF specified by LOC. Name
         definitions can be added to the NDF using routine
         \htmlref{IRQ\_ADDQN}{IRQ_ADDQN}.
      }
      \sstsubsection{
         STATUS = INTEGER (Given and Returned)
      }{
         The global status.
      }
   }
}
\sstroutine{
   IRQ\_SETQL
}{
   Assign a given quality to a list of pixels, leaving unlisted
   pixels unchanged
}{
   \sstdescription{
      The quality specified by QNAME is assigned to all pixels included
      (or, if LISTED is {\tt false}, not included) in the supplied list of
      pixel indices. The quality of other pixels is left unaltered.
      The quality name must be defined in the NDF specified by LOCS
      (LOCS should be obtained either by calling \htmlref{IRQ\_FIND}{IRQ_FIND} 
      or \htmlref{IRQ\_NEW}{IRQ_NEW}).
      An error is reported if the quality name is undefined.

      Note, write or update access must be available for the NDF (as
      set up by routine \xref{NDF\_ASSOC}{sun33}{NDF_ASSOC} for instance), 
      and the QUALITY component must not be mapped on entry to this routine.
   }
   \sstinvocation{
      CALL IRQ\_SETQL( LOCS, LISTED, QNAME, NDIM, NCOORD, LIST, SET,
                      STATUS~)
   }
   \sstarguments{
      \sstsubsection{
         LOCS(5) = CHARACTER $*$ ( $*$ ) (Given)
      }{
         An array of five HDS locators. These locators identify the NDF
         and the associated quality name information.  They should have
         been obtained using routine IRQ\_FIND or routine IRQ\_NEW.
      }
      \sstsubsection{
         LISTED = LOGICAL (Given)
      }{
         If true, then the quality is assigned to all pixels included
         in the list given by LIST.  If false, then the quality is
         assigned to all pixels not included in the list given by LIST.
      }
      \sstsubsection{
         QNAME = CHARACTER $*$ ( $*$ ) (Given)
      }{
         The quality name to assign to the selected pixels. This quality
         name must be defined in the NDF specified by LOC. Name
         definitions can be added to the NDF using routine IRQ\_ADDQN.
      }
      \sstsubsection{
         NDIM = INTEGER (Given)
      }{
         The number of values required to specify a pixel position
         (\emph{i.e.} the number of dimensions in the NDF).
      }
      \sstsubsection{
         NCOORD = INTEGER (Given)
      }{
         The number of pixels included in the input list.
      }
      \sstsubsection{
         LIST( NDIM, NCOORD ) = INTEGER (Given)
      }{
         The list of pixel indices. Any indices which lie outside the
         bounds of the NDF are ignored.
      }
      \sstsubsection{
         SET = INTEGER (Returned)
      }{
         The number of pixels in the NDF which hold the quality.
      }
      \sstsubsection{
         STATUS = INTEGER (Given and Returned)
      }{
         The global status.
      }
   }
}
\sstroutine{
   IRQ\_SETQM
}{
   Assign a quality to pixels selected using a mask image, leaving
   unselected pixels unchanged
}{
   \sstdescription{
      The quality specified by QNAME is assigned to all NDF pixels
      which either do (or, if BAD is false,  do not) correspond to `bad'
      pixels in the input mask array.  The quality of all other pixels
      is left unchanged.  The quality name must be defined in the NDF
      specified by LOCS (LOCS should be obtained either by calling
      \htmlref{IRQ\_FIND}{IRQ_FIND} or \htmlref{IRQ\_NEW}{IRQ_NEW}). 
      An error is reported if the quality name is undefined.

      Note, write or update access must be available for the NDF (as
      set up by routine NDF\_ASSOC for instance), and the QUALITY
      component of the NDF must not be mapped on entry to this routine.
   }
   \sstinvocation{
      CALL IRQ\_SETQM( LOCS, BAD, QNAME, SIZE, MASK, SET, STATUS )
   }
   \sstarguments{
      \sstsubsection{
         LOCS(5) = CHARACTER $*$ ( $*$ ) (Given)
      }{
         An array of five HDS locators. These locators identify the NDF
         and the associated quality name information.  They should have
         been obtained using routine IRQ\_FIND or routine IRQ\_NEW.
      }
      \sstsubsection{
         BAD = LOGICAL (Given)
      }{
         If {\tt true}, then the quality is assigned to all NDF pixels
         corresponding to `bad' pixels in the mask.  If {\tt false}, then the
         quality is assigned to all NDF pixels corresponding to pixels
         which are not `bad' in the mask.
      }
      \sstsubsection{
         QNAME = CHARACTER $*$ ( $*$ ) (Given)
      }{
         The quality name to assign to the selected pixels. This quality
         name must be defined in the NDF specified by LOC. Name
         definitions can be added to the NDF using routine 
         \htmlref{IRQ\_ADDQN}{IRQ_ADDQN}.
      }
      \sstsubsection{
         SIZE = INTEGER (Given)
      }{
         The total number of pixels in the MASK array.
      }
      \sstsubsection{
         MASK( SIZE ) = REAL (Given)
      }{
         A vector which defines the pixels to which the quality
         specified by QNAME is to be assigned.  It is assumed that this
         vector corresponds pixel-for-pixel with the vectorised NDF
         supplied to \htmlref{IRQ\_NEW}{IRQ_NEW} or \htmlref{IRQ\_FIND}{IRQ_FIND}.
      }
      \sstsubsection{
         SET = INTEGER (Returned)
      }{
         The number of pixels in the NDF which hold the quality.
      }
      \sstsubsection{
         STATUS = INTEGER (Given and Returned)
      }{
         The global status.
      }
   }
}
\sstroutine{
   IRQ\_SYNTX
}{
   Check the syntax of a quality expression
}{
   \sstdescription{
      The syntax of the supplied quality expression is checked, and an
      error is reported if a syntax error is detected.  If any of the
      STATUS values IRQ\_\_BADSY, IRQ\_\_MSOPT or IRQ\_\_MSOPD are returned
      (all of which correspond to various forms of syntax error in the
      quality expression, \latexhtml{see Appendix~\ref{APP:ERRORS}}{\htmlref{listed 
      here}{APP:ERRORS}}), a pointer to the
      approximate position of the error within the quality expression
      is returned in ERRPNT.  Note, in order for a quality expression
      to compile successfully (using \htmlref{IRQ\_COMP}{IRQ_COMP}), it must not 
      only contain no syntax errors, but must also contain no undefined quality
      names. IRQ\_SYNTX cannot check for undefined quality names.
   }
   \sstinvocation{
      CALL IRQ\_SYNTX( QEXP, ERRPNT, STATUS )
   }
   \sstarguments{
      \sstsubsection{
         QEXP = CHARACTER$*$($*$) (Given and Returned)
      }{
         A quality expression. See
         \latexhtml{Section~\ref{SEC:QEXP}}{\htmlref{Quality expressions}{SEC:QEXP}}
         for details of the
         allowed formats for quality expressions. On exit, the string is
         converted to upper case and any leading blanks are removed.
      }
      \sstsubsection{
         ERRPNT = INTEGER (Returned)
      }{
         If any of the STATUS values IRQ\_\_BADSY( {\tt "Unrecognised logical
         operator or constant"}), IRQ\_\_MSOPT ( {\tt "Missing operator"}) or
         IRQ\_\_MSOPD ( {\tt "Missing operand"}) are returned, then ERRPNT
         returns the offset within the quality expression at which the
         error was detected. Note, the offset refers to the returned
         form of QEXP, not the given form. These will be different if
         the given form of QEXP has any leading blanks. An offset of
         zero is returned if none of the errors associated with the
         above STATUS values occur.
      }
      \sstsubsection{
         STATUS = INTEGER (Given and Returned)
      }{
         The global status.
      }
   }
}

\newpage
\section {HDS Data Structures}

The IRQ package uses several different HDS data structures. These are
described in this appendix. HDS {\em names} are indicated by being
placed within square brackets ({[} {]}), and HDS {\em types} are
indicated by being placed within angled brackets ($<$ $>$).

\subsection{Quality names information stored in an NDF}

Information describing the quality names which are defined within an
NDF is stored in a structure called {[}{\bf QUALITY\_NAMES}{]}, with
HDS {\em type} $<${\bf QUALITY\_NAMES}$>$. This structure can be
stored in any extension within the NDF. The components of this
structure are shown in \latexhtml{Table~\ref {TAB:QN}}{the table
below}.

\begin{table}[htb]
\centering
\latex{\caption{Components of a $<${\bf QUALITY\_NAMES}$>$ structure}}
\html{\caption{{\bf 1:} Components of a $<${\bf QUALITY\_NAMES}$>$ structure}}
\label{TAB:QN}
\begin{tabular}{|l|l|l|}
\hline
Component Name & TYPE & Brief Description \\ \hline
{[}{\bf QUAL}{]} & $<${\bf IRQ\_QUAL}$>$ & Vector of quality name definitions\\
{[}{\bf LAST\_USED}{]} & $<${\bf \_INTEGER}$>$ & Highest used index within 
{[}{\bf QUAL}{]}\\
{[}{\bf NFREE}{]} & $<${\bf \_INTEGER}$>$ & No. of un-used cells within
{[}{\bf QUAL}{]}\\
{[}{\bf FREE}(NFREE){]} & $<${\bf \_INTEGER}$>$ & Vector holding indices of 
un-used cells in {[}{\bf QUAL}{]}\\
\hline
\end{tabular}
\end{table}

The {[}{\bf QUAL}{]} component is a vector in which each cell is a
structure holding various items of information needed to define a
single quality name. The array has an initial size of 8 but is
increased if necessary. There will usually be some unused cells within
{[}{\bf QUAL}{]}, and the other components of the {[}{\bf
QUALITY\_NAMES}{]} structure listed in \latexhtml{Table~\ref
{TAB:QN}}{Table~1} are used to locate these unused cells. In
particular, {[}{\bf FREE}{]} is a vector which holds the indices of
all unused cells within {[}{\bf QUAL}{]}. These indices may appear in
any order within in {[}{\bf FREE}{]}. 

The set of five HDS locators returned by routines
\htmlref{IRQ\_NEW}{IRQ_NEW} and \htmlref{IRQ\_FIND}{IRQ_FIND} include
locators to each of the four components of the {[}{\bf
QUALITY\_NAMES}{]} structure, together with a locator for a temporary
HDS structure holding a single $<${\bf \_INTEGER}$>$ scalar used to
store the cloned NDF identifier. 

The information describing each individual quality name is stored in a single
cell of the {[}{\bf QUAL}{]} vector. The components within each cell of this 
array are listed in 
\latexhtml{Table~\ref {TAB:QUAL}}{Table~2}.

\begin{table}[htb]
\centering
\latex{\caption{Components of a $<${\bf IRQ\_QUAL}$>$ structure}}
\html{\caption{{\bf 2:} Components of a $<${\bf IRQ\_QUAL}$>$ structure}}
\label{TAB:QUAL}
\begin{tabular}{|l|l|l|}
\hline
Component Name & TYPE & Brief Description \\ \hline
{[}{\bf NAME}{]} & $<${\bf \_CHAR$\ast$15}$>$ & A quality name (maximum of
\\
{[}{\bf FIXED}{]} & $<${\bf \_LOGICAL}$>$ & True if all pixels are in the same 
state\\
{[}{\bf VALUE}{]} & $<${\bf \_LOGICAL}$>$ & The state of all pixels, if fixed\\
{[}{\bf BIT}{]} & $<${\bf \_INTEGER}$>$ & QUALITY bit used to store this 
quality\\
{[}{\bf COMMENT}{]} & $<${\bf \_CHAR$\ast$}$>$ & A descriptive comment for the 
quality\\
\hline
\end{tabular}
\end{table}

If {\em all} pixels hold a given quality, or if {\em no} pixels hold
the quality, then a true value is stored for {[}{\bf FIXED}{]}. A true
value is stored for {[}{\bf VALUE}{]} if {\em all} pixels hold the
quality, and a false value if {\em no} pixels hold the quality. In
either case, {[}{\bf BIT}{]} is ignored since no QUALITY bit needs to
be reserved for the quality name, thus allowing more than eight
quality names to be defined simultaneously. 

If some pixels {\em do} hold the quality but some {\em do not}, then a
false value is stored for {[}{\bf FIXED}{]} and {[}{\bf VALUE}{]} is
ignored. In this case, a QUALITY bit is reserved to represent the
quality and its bit number (in the range 1 to 8) is stored in {[}{\bf
BIT}{]}.

\subsection{Temporary structures used to hold compiled quality
expressions} When routine \htmlref{IRQ\_COMP}{IRQ_COMP} is called to
compile a quality expression, the resulting information (known as a
`compiled quality expression') is stored in a temporary HDS structure.
Up to ten compiled quality expressions can exist simultaneously, each
being stored in one cell of an array of temporary structures. The
identifier returned by IRQ\_COMP is just an index within this array.
Each cell of the array has an HDS {\em name} of {[}{\bf QEXP}{]} and
an HDS {\em type} of $<${\bf QEXP}$>$, and contains the components
listed in \latexhtml{Table~\ref {TAB:QEXP}}{Table 3}. 

\begin{table}[htb]
\centering
\latex{\caption{Components of a $<${\bf QEXP}$>$ structure}}
\html{\caption{{\bf 3:} Components of a $<${\bf QEXP}$>$ structure}}
\label{TAB:QEXP}
\begin{tabular}{|l|l|l|}
\hline
Component Name & TYPE & Brief Description \\ \hline
{[}{\bf MASKS}{]} & $<${\bf \_INTEGER}$>$ & A vector of bit masks\\
{[}{\bf OPCODE}{]} & $<${\bf \_INTEGER}$>$ & A vector of instruction codes\\
\hline
\end{tabular}
\end{table}

Each bit mask held in {[}{\bf MASKS}{]} specifies a set of QUALITY bits which
are to be tested as part of the evaluation of a quality expression performed by
routine IRQ\_SBAD. The instruction codes held in {[}{\bf OPCODE}{]} represent
the operations which must be performed on a ``First In - Last Out'' stack in
order to evaluate a quality expression. The sizes of theses vectors are held in
common. 

\newpage
\section {Examples of Using IRQ}
\label{APP:EXAMS}
\subsection {Adding a new quality name}

This example shows a code fragment which adds the quality name {\tt
SATURATED} to an NDF and then assigns the quality to all pixels with
value greater than 10.0. An error is reported if the quality name is
already in use.

\begin{quote}
\begin{tabbing} % Not clear why this is needed, since tabs are not used,
                % but the leading spaces get lost otherwise.

\verb#*  Include ADAM, IRQ and NDF symbolic constants.       #\\
\verb#      INCLUDE 'SAE_PAR'                                #\\
\verb#      INCLUDE 'IRQ_PAR'                                #\\
\verb#      INCLUDE 'IRQ_ERR'                                #\\
\verb#      INCLUDE 'NDF_PAR'                                #\\
\verb#                                                       #\\
\verb#*  Declare local INTEGER variables.                    #\\
\verb#      INTEGER NDFIN, NDF2, NDIM, LBND(NDF__MXDIM), NEL,#\\
\verb#     :        UBND(NDF__MXDIM), PNT, PNT2, PLACE,      #\\
\verb#     :        STATUS                                   #\\
\verb#                                                       #\\
\verb#*  Declare local LOGICAL variables.                    #\\
\verb#      INTEGER FOUND                                    #\\
\verb#                                                       #\\
\verb#*  Declare local CHARACTER variables.                  #\\
\verb#      CHARACTER*(DAT__SZLOC) LOCS(5),XLOC              #\\
\verb#      CHARACTER*(DAT__SZNAM) XNAME                     #\\
\verb#                                                       #\\
\verb#*  Start an NDF context.                               #\\
\verb#      CALL NDF_BEGIN                                   #\\
\verb#                                                       #\\
\verb#*  Obtain an identifier for the input NDF.             #\\
\verb#      CALL NDF_ASSOC( 'IN', 'READ', NDFIN, STATUS )    #\\
\verb#                                                       #\\
\verb#*  Attempt to locate any existing quality name         #\\
\verb#*  information in the input NDF. If such information is#\\
\verb#*  found, LOCS is returned holding a set of five HDS   #\\
\verb#*  locators which identify the NDF and various items of#\\
\verb#*  quality information. XNAME is returned holding the  #\\
\verb#*  name of the NDF extension in which the information  #\\
\verb#*  was found. If no quality name information is found, #\\
\verb#*  then an error is reported.                          #\\
\verb#      CALL IRQ_FIND( NDFIN, LOCS, XNAME, STATUS )      #\\
\verb#                                                       #\\
\verb#*  If no quality name information was found, annul the #\\
\verb#*  error. New quality names information will be set up #\\
\verb#*  in the "IRAS" NDF extension.                        #\\
\verb#      IF( STATUS .EQ. IRQ__NOQNI ) THEN                #\\
\verb#         CALL ERR_ANNUL( STATUS )                      #\\
\verb#                                                       #\\
\verb#*  If the "IRAS" extension does not exist, create it.  #\\
\verb#         CALL NDF_XSTAT( NDFIN, 'IRAS', FOUND, STATUS )#\\
\verb#         IF( .NOT. FOUND ) THEN                        #\\
\verb#            CALL NDF_XNEW( NDFIN, 'IRAS', 'IRAS', 0, 0,#\\
\verb#     :                     XLOC, STATUS )              #\\
\verb#            CALL DAT_ANNUL( XLOC, STATUS )             #\\
\verb#         END IF                                        #\\
\verb#                                                       #\\
\verb#*  Create a new structure to hold quality information  #\\
\verb#*  in the "IRAS" NDF extension.                        #\\
\verb#         CALL IRQ_NEW( NDFIN, 'IRAS', LOCS, STATUS )   #\\
\verb#      END IF                                           #\\
\verb#                                                       #\\
\verb#*  Attempt to add the quality name "SATURATED" to the  #\\
\verb#*  NDF. If the name already exists an error will be    #\\
\verb#*  reported.                                           #\\
\verb#      CALL IRQ_ADDQN( LOCS, 'SATURATED', .FALSE.,      #\\
\verb#     :  'Pixels with value greater than 10.0', STATUS )#\\
\verb#                                                       #\\
\verb#*  Get a temporary NDF which is the same shape as the  #\\
\verb#*  input NDF.                                          #\\
\verb#      CALL NDF_BOUND( NDFIN, NDF__MXDIM, LBND, UBND,   #\\
\verb#     :                NDIM, STATUS )                   #\\
\verb#      CALL NDF_TEMP( PLACE, STATUS )                   #\\
\verb#      CALL NDF_NEW( '_REAL', NDIM, LBND, UBND, PLACE,  #\\
\verb#     :              NDF2, STATUS )                     #\\
\verb#                                                       #\\
\verb#*  Map the DATA array, initialising its contents to    #\\
\verb#*  zero.                                               #\\
\verb#      CALL NDF_MAP( NDF2, 'DATA', '_REAL',             #\\
\verb#     :              'WRITE/ZERO', PNT2, NEL, STATUS )  #\\
\verb#                                                       #\\
\verb#*  Map the DATA array of the input NDF.                #\\
\verb#      CALL NDF_MAP( NDFIN, 'DATA', '_REAL', 'READ',    #\\
\verb#     :              PNT, NEL, STATUS )                 #\\
\verb#                                                       #\\
\verb#*  Set pixels bad in the temporary array which         #\\
\verb#*  correspond to pixels greater than 10.0 in the input #\\
\verb#*  NDF.                                                #\\
\verb#      CALL MASKIT( %VAL(PNT), %VAL(PNT2), NEL, 10.0,   #\\
\verb#     :             STATUS )                            #\\
\verb#                                                       #\\
\verb#*  Assign the quality SATURATED to all the pixels which#\\
\verb#*  are bad in the temporary NDF.                       #\\
\verb#      CALL IRQ_SETQM( LOCS, .TRUE., 'SATURATED', NEL,  #\\
\verb#     :               %VAL( PNT2 ), STATUS )            #\\
\verb#                                                       #\\
\verb#*  Release the resources used by IRQ.                  #\\
\verb#      CALL IRQ_RLSE( LOCS, STATUS )                    #\\
\verb#                                                       #\\
\verb#*  End the NDF context.                                #\\
\verb#      CALL NDF_END( STATUS )                           #\\

\end{tabbing}
\end{quote}

\subsection {Finding pixels which satisfy a quality expression}
This example produces a copy of an input NDF in which all pixels which do not 
satisfy the quality expression ``.NOT.( BACKGROUND .OR. SATURATED ) '' are
set bad.


\begin{quote}
\begin{tabbing} 

\verb#*  Attempt to locate any quality name information in   #\\
\verb#*  the input NDF. If no quality name information is    #\\
\verb#*  found, then an error is reported.                   #\\
\verb#      CALL IRQ_FIND( NDFIN, LOCS, XNAME, STATUS )      #\numcir{1}\\
\verb#                                                       #\\
\verb#*  Attempt to compile the quality expression.          #\\
\verb#      QEXP = '.NOT. ( SATURATED .OR. BACKGROUND )'     #\\
\verb#      CALL IRQ_COMP( LOCS, IRQ__QNREF, .TRUE., QEXP,   #\numcir{2}\\
\verb#     :             UNDEF, NUNDEF, ERRPNT, IDQ, STATUS )#\\
\verb#                                                       #\\
\verb#*  Produce a temporary copy of the input NDF.          #\\
\verb#      CALL NDF_TEMP( PLACE, STATUS )                   #\\
\verb#      CALL NDF_COPY( NDFIN, PLACE, NDF2, STATUS )      #\\
\verb#                                                       #\\
\verb#*  Map the DATA array for UPDATE access.               #\\
\verb#      CALL NDF_MAP( NDF2, 'DATA', '_REAL', 'UPDATE',   #\\
\verb#     :              PNT2, NEL, STATUS )                #\\
\verb#                                                       #\\
\verb#*  Find all pixels which do not satisfy the quality    #\\
\verb#*  expression and set them bad in the NDF copy.        #\\
\verb#      CALL IRQ_SBAD( IDQ, .FALSE., NEL, %VAL( PNT2 ),  #\\
\verb#     :               ALLBAD, NOBAD, STATUS )           #\\
\verb#                                                       #\\
\verb#*  Unmap the NDF copy.                                 #\\
\verb#      CALL NDF_UNMAP( NDFIN, 'DATA', STATUS )          #\\
\verb#                                                       #\\
\verb#*  Close down the IRQ identifier system.               #\\
\verb#      CALL IRQ_CLOSE( STATUS )                         #\\
\verb#                                                       #\\
\verb#*  Release the resources used by IRQ.                  #\\
\verb#      CALL IRQ_RLSE( LOCS, STATUS )                    #\\

\end{tabbing}
\end{quote}

Programming notes:

\begin{enumerate}

\label{NOTE1}
\item NDFIN is the NDF identifier for the input NDF and should have been
obtained previously. If the NDF contains no quality name information, then
an error will be reported by \htmlref{IRQ\_FIND}{IRQ_FIND}.

\label{NOTE2}
\item The quality expression must be assigned to a character variable,
since the call to \htmlref{IRQ\_COMP}{IRQ_COMP} updates the expression by
removing leading blanks, and converting the expression to upper case. An
access violation would result if a literal string were supplied as an
argument, instead of a character variable. If either of the quality names
{\tt SATURATED} or {\tt BACKGROUND} is not defined in the input NDF, then
an error will be reported by IRQ\_COMP, and the undefined names will be
returned in the UNDEF array. 

\end{enumerate}

\section {Packages Called by IRQ}

IRQ\_ makes calls to the following packages:
\begin {description}
\item [CHR\_] - The CHR character handling package; see \xref{SUN/40}{sun40}{}.
\item [CMP\_] - HDS; see  \xref{SUN/92}{sun92}{}.
\item [DAT\_] - HDS; see SUN/92.
\item [ERR\_] - The Starlink error reporting package; see \xref{SUN/104}{sun104}{}.
\item [MSG\_] - The Starlink message reporting package; see SUN/104.
\item [NDF\_] - The NDF access package; see  \xref{SUN/33}{sun33}{}.
\item [VEC\_] - The PRIMDAT package; see  \xref{SUN/39}{sun39}{}.
\end{description}

Access to these packages, together with packages called from within these 
packages, is necessary to use IRQ. 

\newpage

\section {\xlabel{APP:ERRORS}IRQ Error Codes}
\label {APP:ERRORS}

IRQ routines can return any STATUS value generated by the subroutine 
packages which it calls. In addition it can return the following
IRQ-specific values.

\begin{description}
\item {\bf IRQ\_\_BADBT}\\ 
Bit value outside range [1,8] supplied.
\item {\bf IRQ\_\_BADDM}\\
Incorrect NDF dimensions supplied.
\item {\bf IRQ\_\_BADNM}\\ 
An illegal quality name has been given.
\item {\bf IRQ\_\_BADQN}\\ 
Incomplete QUALITY\_NAMES structure found.
\item {\bf IRQ\_\_BADSL}\\ 
Invalid slot number supplied.
\item {\bf IRQ\_\_BADST}\\ 
Incomplete slot structure found.
\item {\bf IRQ\_\_BADSY}\\ 
Unrecognised logical operator or constant in quality expression.
\item {\bf IRQ\_\_CMPLX}\\ 
Too many symbols in quality expression.
\item {\bf IRQ\_\_INCOM}\\ 
Supplied vector has different size to the NDF
\item {\bf IRQ\_\_INTER}\\ 
Internal IRQ error (report to maintainer of IRQ).
\item {\bf IRQ\_\_INVID}\\ 
Invalid IRQ identifier supplied.
\item {\bf IRQ\_\_IVNDF}\\ 
Invalid NDF identifier found.
\item {\bf IRQ\_\_LSHRT}\\ 
Character variable too short.
\item {\bf IRQ\_\_MSDOT}\\ 
Missing delimiter "." in quality expression.
\item {\bf IRQ\_\_MSOPD}\\ 
Missing operand in quality expression.
\item {\bf IRQ\_\_MSOPT}\\ 
Missing or invalid operator in quality expression.
\item {\bf IRQ\_\_MSPAR}\\ 
Unpaired parentheses in quality expression.
\item {\bf IRQ\_\_MULT }\\ 
More than one structure found holding quality names information.
\item {\bf IRQ\_\_NOMOR}\\
No identifiers left for compiled quality expressions.
\item {\bf IRQ\_\_NOOPS}\\ 
No operands can be found.
\item {\bf IRQ\_\_NOQNM}\\ 
A quality name could not be found.
\item {\bf IRQ\_\_NOQNI}\\ 
No quality names structure found in the NDF.
\item {\bf IRQ\_\_NOSPA}\\ 
Can't reclaim space in an instruction array.
\item {\bf IRQ\_\_NOWRT}\\ 
Write access to the NDF is unavailable.
\item {\bf IRQ\_\_QBAD }\\ 
BAD values exists in the QUALITY component.
\item {\bf IRQ\_\_QEXPL}\\ 
Too many characters in quality expression.
\item {\bf IRQ\_\_QIEXS}\\ 
Extension already contains a quality names structure.
\item {\bf IRQ\_\_QNEXS}\\ 
Quality name is already defined.
\item {\bf IRQ\_\_QLONG}\\ 
Quality name too long.
\item {\bf IRQ\_\_QREFS}\\ 
Too many quality names in quality expression.
\item {\bf IRQ\_\_QUNDF}\\ 
QUALITY component is undefined.
\item {\bf IRQ\_\_RDONL}\\ 
An attempt has been made to remove a read-only quality name.
\item {\bf IRQ\_\_STKOV}\\ 
Evaluation stack overflow.
\item {\bf IRQ\_\_STKUN}\\ 
Evaluation stack underflow.
\item {\bf IRQ\_\_XBITS}\\ 
No bits left in QUALITY component.
\end{description}

\section {Changes Introduced in the Current Version of this Document}
\label {SEC:CHANGES}

\begin{itemize}
   \item It has a separate identity.  It was previously bundled into
   KAPLIBS after being recovered from IRAS90.
   \item There is documentation.
\end{itemize}

\end{document}
