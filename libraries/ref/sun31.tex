\documentclass[twoside,11pt]{article}

% ? Specify used packages
% \usepackage{graphicx}        %  Use this one for final production.
% \usepackage[draft]{graphicx} %  Use this one for drafting.
% ? End of specify used packages

\pagestyle{myheadings}

% -----------------------------------------------------------------------------
% ? Document identification
% Fixed part
\newcommand{\stardoccategory}  {Starlink User Note}
\newcommand{\stardocinitials}  {SUN}
\newcommand{\stardocsource}    {sun\stardocnumber}

% Variable part - replace [xxx] as appropriate.
\newcommand{\stardocnumber}    {31.5}
\newcommand{\stardocauthors}   {R.F.~Warren-Smith \& \\
                                A.J.~Chipperfield}
\newcommand{\stardocdate}      {27 April 1998}
\newcommand{\stardoctitle}     {REF \\ [\latex{1ex}]
                                Routines for Handling References
                                to HDS Objects}
\newcommand{\stardocversion}   {Version 1.1}
\newcommand{\stardocmanual}    {Programmer's Manual}
\newcommand{\stardocabstract}  {%
It is sometimes useful to use the Hierarchical Data System HDS
(\xref{SUN/92}{sun92}{}) to store \emph{references } or \emph{pointers}
to other HDS objects. For instance, this allows the same data object to
be used in several places without the need to have more than one copy.
The REF library is provided to facilitate this data object
\emph{referencing} process and the subsequent accessing of objects
which have been referenced in this way.  }

% ? End of document identification
% -----------------------------------------------------------------------------

% +
%  Name:
%     sun.tex
%
%  Purpose:
%     Template for Starlink User Note (SUN) documents.
%     Refer to SUN/199
%
%  Authors:
%     AJC: A.J.Chipperfield (Starlink, RAL)
%     BLY: M.J.Bly (Starlink, RAL)
%
%  History:
%     17-JAN-1996 (AJC):
%        Original with hypertext macros, based on MDL plain originals.
%     16-JUN-1997 (BLY):
%        Adapted for LaTeX2e.
%        Added picture commands.
%     {Add further history here}
%
% -

\newcommand{\stardocname}{\stardocinitials /\stardocnumber}
\markboth{\stardocname}{\stardocname}
\setlength{\textwidth}{160mm}
\setlength{\textheight}{230mm}
\setlength{\topmargin}{-2mm}
\setlength{\oddsidemargin}{0mm}
\setlength{\evensidemargin}{0mm}
\setlength{\parindent}{0mm}
\setlength{\parskip}{\medskipamount}
\setlength{\unitlength}{1mm}

% -----------------------------------------------------------------------------
%  Hypertext definitions.
%  ======================
%  These are used by the LaTeX2HTML translator in conjunction with star2html.

%  Comment.sty: version 2.0, 19 June 1992
%  Selectively in/exclude pieces of text.
%
%  Author
%    Victor Eijkhout                                      <eijkhout@cs.utk.edu>
%    Department of Computer Science
%    University Tennessee at Knoxville
%    104 Ayres Hall
%    Knoxville, TN 37996
%    USA

%  Do not remove the %begin{latexonly} and %end{latexonly} lines (used by 
%  star2html to signify raw TeX that latex2html cannot process).
%begin{latexonly}
\makeatletter
\def\makeinnocent#1{\catcode`#1=12 }
\def\csarg#1#2{\expandafter#1\csname#2\endcsname}

\def\ThrowAwayComment#1{\begingroup
    \def\CurrentComment{#1}%
    \let\do\makeinnocent \dospecials
    \makeinnocent\^^L% and whatever other special cases
    \endlinechar`\^^M \catcode`\^^M=12 \xComment}
{\catcode`\^^M=12 \endlinechar=-1 %
 \gdef\xComment#1^^M{\def\test{#1}
      \csarg\ifx{PlainEnd\CurrentComment Test}\test
          \let\html@next\endgroup
      \else \csarg\ifx{LaLaEnd\CurrentComment Test}\test
            \edef\html@next{\endgroup\noexpand\end{\CurrentComment}}
      \else \let\html@next\xComment
      \fi \fi \html@next}
}
\makeatother

\def\includecomment
 #1{\expandafter\def\csname#1\endcsname{}%
    \expandafter\def\csname end#1\endcsname{}}
\def\excludecomment
 #1{\expandafter\def\csname#1\endcsname{\ThrowAwayComment{#1}}%
    {\escapechar=-1\relax
     \csarg\xdef{PlainEnd#1Test}{\string\\end#1}%
     \csarg\xdef{LaLaEnd#1Test}{\string\\end\string\{#1\string\}}%
    }}

%  Define environments that ignore their contents.
\excludecomment{comment}
\excludecomment{rawhtml}
\excludecomment{htmlonly}

%  Hypertext commands etc. This is a condensed version of the html.sty
%  file supplied with LaTeX2HTML by: Nikos Drakos <nikos@cbl.leeds.ac.uk> &
%  Jelle van Zeijl <jvzeijl@isou17.estec.esa.nl>. The LaTeX2HTML documentation
%  should be consulted about all commands (and the environments defined above)
%  except \xref and \xlabel which are Starlink specific.

\newcommand{\htmladdnormallinkfoot}[2]{#1\footnote{#2}}
\newcommand{\htmladdnormallink}[2]{#1}
\newcommand{\htmladdimg}[1]{}
\newenvironment{latexonly}{}{}
\newcommand{\hyperref}[4]{#2\ref{#4}#3}
\newcommand{\htmlref}[2]{#1}
\newcommand{\htmlimage}[1]{}
\newcommand{\htmladdtonavigation}[1]{}
\newcommand{\latexhtml}[2]{#1}
\newcommand{\html}[1]{}

%  Starlink cross-references and labels.
\newcommand{\xref}[3]{#1}
\newcommand{\xlabel}[1]{}

%  LaTeX2HTML symbol.
\newcommand{\latextohtml}{{\bf LaTeX}{2}{\tt{HTML}}}

%  Define command to re-centre underscore for Latex and leave as normal
%  for HTML (severe problems with \_ in tabbing environments and \_\_
%  generally otherwise).
\newcommand{\latex}[1]{#1}
\newcommand{\setunderscore}{\renewcommand{\_}{{\tt\symbol{95}}}}
\latex{\setunderscore}

% -----------------------------------------------------------------------------
%  Debugging.
%  =========
%  Remove % on the following to debug links in the HTML version using Latex.

% \newcommand{\hotlink}[2]{\fbox{\begin{tabular}[t]{@{}c@{}}#1\\\hline{\footnotesize #2}\end{tabular}}}
% \renewcommand{\htmladdnormallinkfoot}[2]{\hotlink{#1}{#2}}
% \renewcommand{\htmladdnormallink}[2]{\hotlink{#1}{#2}}
% \renewcommand{\hyperref}[4]{\hotlink{#1}{\S\ref{#4}}}
% \renewcommand{\htmlref}[2]{\hotlink{#1}{\S\ref{#2}}}
% \renewcommand{\xref}[3]{\hotlink{#1}{#2 -- #3}}
%end{latexonly}
% -----------------------------------------------------------------------------
% ? Document specific \newcommand or \newenvironment commands.

%+
%  Name:
%     SST.TEX

%  Purpose:
%     Define LaTeX commands for laying out Starlink routine descriptions.

%  Language:
%     LaTeX

%  Type of Module:
%     LaTeX data file.

%  Description:
%     This file defines LaTeX commands which allow routine documentation
%     produced by the SST application PROLAT to be processed by LaTeX and
%     by LaTeX2html. The contents of this file should be included in the
%     source prior to any statements that make of the sst commnds.

%  Notes:
%     The style file html.sty provided with LaTeX2html needs to be used.
%     This must be before this file.

%  Authors:
%     RFWS: R.F. Warren-Smith (STARLINK)
%     PDRAPER: P.W. Draper (Starlink - Durham University)

%  History:
%     10-SEP-1990 (RFWS):
%        Original version.
%     10-SEP-1990 (RFWS):
%        Added the implementation status section.
%     12-SEP-1990 (RFWS):
%        Added support for the usage section and adjusted various spacings.
%     8-DEC-1994 (PDRAPER):
%        Added support for simplified formatting using LaTeX2html.
%     {enter_further_changes_here}

%  Bugs:
%     {note_any_bugs_here}

%-

%  Define length variables.
\newlength{\sstbannerlength}
\newlength{\sstcaptionlength}
\newlength{\sstexampleslength}
\newlength{\sstexampleswidth}

%  Define a \tt font of the required size.
\latex{\newfont{\ssttt}{cmtt10 scaled 1095}}
\html{\newcommand{\ssttt}{\tt}}

%  Define a command to produce a routine header, including its name,
%  a purpose description and the rest of the routine's documentation.
\newcommand{\sstroutine}[3]{
   \goodbreak
   \rule{\textwidth}{0.5mm}
   \vspace{-7ex}
   \newline
   \settowidth{\sstbannerlength}{{\Large {\bf #1}}}
   \setlength{\sstcaptionlength}{\textwidth}
   \setlength{\sstexampleslength}{\textwidth}
   \addtolength{\sstbannerlength}{0.5em}
   \addtolength{\sstcaptionlength}{-2.0\sstbannerlength}
   \addtolength{\sstcaptionlength}{-5.0pt}
   \settowidth{\sstexampleswidth}{{\bf Examples:}}
   \addtolength{\sstexampleslength}{-\sstexampleswidth}
   \parbox[t]{\sstbannerlength}{\flushleft{\Large {\bf #1}}}
   \parbox[t]{\sstcaptionlength}{\center{\Large #2}}
   \parbox[t]{\sstbannerlength}{\flushright{\Large {\bf #1}}}
   \begin{description}
      #3
   \end{description}
}

%  Format the description section.
\newcommand{\sstdescription}[1]{\item[Description:] #1}

%  Format the usage section.
\newcommand{\sstusage}[1]{\item[Usage:] \mbox{}
\\[1.3ex]{\raggedright \ssttt #1}}

%  Format the invocation section.
\newcommand{\sstinvocation}[1]{\item[Invocation:]\hspace{0.4em}{\tt #1}}

%  Format the arguments section.
\newcommand{\sstarguments}[1]{
   \item[Arguments:] \mbox{} \\
   \vspace{-3.5ex}
   \begin{description}
      #1
   \end{description}
}

%  Format the returned value section (for a function).
\newcommand{\sstreturnedvalue}[1]{
   \item[Returned Value:] \mbox{} \\
   \vspace{-3.5ex}
   \begin{description}
      #1
   \end{description}
}

%  Format the parameters section (for an application).
\newcommand{\sstparameters}[1]{
   \item[Parameters:] \mbox{} \\
   \vspace{-3.5ex}
   \begin{description}
      #1
   \end{description}
}

%  Format the examples section.
\newcommand{\sstexamples}[1]{
   \item[Examples:] \mbox{} \\
   \vspace{-3.5ex}
   \begin{description}
      #1
   \end{description}
}

%  Define the format of a subsection in a normal section.
\newcommand{\sstsubsection}[1]{ \item[{#1}] \mbox{} \\}

%  Define the format of a subsection in the examples section.
\newcommand{\sstexamplesubsection}[2]{\sloppy
\item[\parbox{\sstexampleslength}{\ssttt #1}] \mbox{} \vspace{1.0ex}
\\ #2 }

%  Format the notes section.
\newcommand{\sstnotes}[1]{\item[Notes:] \mbox{} \\[1.3ex] #1}

%  Provide a general-purpose format for additional (DIY) sections.
\newcommand{\sstdiytopic}[2]{\item[{\hspace{-0.35em}#1\hspace{-0.35em}:}]
\mbox{} \\[1.3ex] #2}

%  Format the implementation status section.
\newcommand{\sstimplementationstatus}[1]{
   \item[{Implementation Status:}] \mbox{} \\[1.3ex] #1}

%  Format the bugs section.
\newcommand{\sstbugs}[1]{\item[Bugs:] #1}

%  Format a list of items while in paragraph mode.
\newcommand{\sstitemlist}[1]{
  \mbox{} \\
  \vspace{-3.5ex}
  \begin{itemize}
     #1
  \end{itemize}
}

%  Define the format of an item.
\newcommand{\sstitem}{\item}

%% Now define html equivalents of those already set. These are used by
%  latex2html and are defined in the html.sty files.
\begin{htmlonly}

%  sstroutine.
   \newcommand{\sstroutine}[3]{
      \subsection{#1\xlabel{#1}-\label{#1}#2}
      \begin{description}
         #3
      \end{description}
   }

%  sstdescription
   \newcommand{\sstdescription}[1]{\item[Description:]
      \begin{description}
         #1
      \end{description}
      \\
   }

%  sstusage
   \newcommand{\sstusage}[1]{\item[Usage:]
      \begin{description}
         {\ssttt #1}
      \end{description}
      \\
   }

%  sstinvocation
   \newcommand{\sstinvocation}[1]{\item[Invocation:]
      \begin{description}
         {\ssttt #1}
      \end{description}
      \\
   }

%  sstarguments
   \newcommand{\sstarguments}[1]{
      \item[Arguments:] \\
      \begin{description}
         #1
      \end{description}
      \\
   }

%  sstreturnedvalue
   \newcommand{\sstreturnedvalue}[1]{
      \item[Returned Value:] \\
      \begin{description}
         #1
      \end{description}
      \\
   }

%  sstparameters
   \newcommand{\sstparameters}[1]{
      \item[Parameters:] \\
      \begin{description}
         #1
      \end{description}
      \\
   }

%  sstexamples
   \newcommand{\sstexamples}[1]{
      \item[Examples:] \\
      \begin{description}
         #1
      \end{description}
      \\
   }

%  sstsubsection
   \newcommand{\sstsubsection}[1]{\item[{#1}]}

%  sstexamplesubsection
   \newcommand{\sstexamplesubsection}[2]{\item[{\ssttt #1}] #2}

%  sstnotes
   \newcommand{\sstnotes}[1]{\item[Notes:] #1 }

%  sstdiytopic
   \newcommand{\sstdiytopic}[2]{\item[{#1}] #2 }

%  sstimplementationstatus
   \newcommand{\sstimplementationstatus}[1]{
      \item[Implementation Status:] #1
   }

%  sstitemlist
   \newcommand{\sstitemlist}[1]{
      \begin{itemize}
         #1
      \end{itemize}
      \\
   }
%  sstitem
   \newcommand{\sstitem}{\item}

\end{htmlonly}

%  End of "sst.tex" layout definitions.
%.



% ? End of document specific commands
% -----------------------------------------------------------------------------
%  Title Page.
%  ===========
\renewcommand{\thepage}{\roman{page}}
\begin{document}
\thispagestyle{empty}

%  Latex document header.
%  ======================
\begin{latexonly}
   CCLRC / {\sc Rutherford Appleton Laboratory} \hfill {\bf \stardocname}\\
   {\large Particle Physics \& Astronomy Research Council}\\
   {\large Starlink Project\\}
   {\large \stardoccategory\ \stardocnumber}
   \begin{flushright}
   \stardocauthors\\
   \stardocdate
   \end{flushright}
   \vspace{-4mm}
   \rule{\textwidth}{0.5mm}
   \vspace{5mm}
   \begin{center}
   {\Huge\bf  \stardoctitle \\ [2.5ex]}
   {\LARGE\bf \stardocversion \\ [4ex]}
   {\Huge\bf  \stardocmanual}
   \end{center}
   \vspace{5mm}

% ? Add picture here if required for the LaTeX version.
%   e.g. \includegraphics[scale=0.3]{filename.ps}
% ? End of picture

% ? Heading for abstract if used.
   \vspace{10mm}
   \begin{center}
      {\Large\bf Abstract}
   \end{center}
% ? End of heading for abstract.
\end{latexonly}

%  HTML documentation header.
%  ==========================
\begin{htmlonly}
   \xlabel{}
   \begin{rawhtml} <H1> \end{rawhtml}
      \stardoctitle\\
      \stardocversion\\
      \stardocmanual
   \begin{rawhtml} </H1> \end{rawhtml}

% ? Add picture here if required for the hypertext version.
%   e.g. \includegraphics[scale=0.7]{filename.ps}
% ? End of picture

   \begin{rawhtml} <P> <I> \end{rawhtml}
   \stardoccategory\ \stardocnumber \\
   \stardocauthors \\
   \stardocdate
   \begin{rawhtml} </I> </P> <H3> \end{rawhtml}
      \htmladdnormallink{CCLRC}{http://www.cclrc.ac.uk} /
      \htmladdnormallink{Rutherford Appleton Laboratory}
                        {http://www.cclrc.ac.uk/ral} \\
      \htmladdnormallink{Particle Physics \& Astronomy Research Council}
                        {http://www.pparc.ac.uk} \\
   \begin{rawhtml} </H3> <H2> \end{rawhtml}
      \htmladdnormallink{Starlink Project}{http://star-www.rl.ac.uk/}
   \begin{rawhtml} </H2> \end{rawhtml}
   \htmladdnormallink{\htmladdimg{source.gif} Retrieve hardcopy}
      {http://star-www.rl.ac.uk/cgi-bin/hcserver?\stardocsource}\\

%  HTML document table of contents. 
%  ================================
%  Add table of contents header and a navigation button to return to this 
%  point in the document (this should always go before the abstract \section). 
  \label{stardoccontents}
  \begin{rawhtml} 
    <HR>
    <H2>Contents</H2>
  \end{rawhtml}
  \htmladdtonavigation{\htmlref{\htmladdimg{contents_motif.gif}}
        {stardoccontents}}

% ? New section for abstract if used.
  \section{\xlabel{abstract}Abstract}
% ? End of new section for abstract
\end{htmlonly}

% -----------------------------------------------------------------------------
% ? Document Abstract. (if used)
%  ==================
\stardocabstract
% ? End of document abstract
% -----------------------------------------------------------------------------
% ? Latex document Table of Contents (if used).
%  ===========================================
  \newpage
  \begin{latexonly}
\null\vspace {5mm}
\begin {center}
\rule{80mm}{0.5mm} \\ [1ex]
{\Large\bf \stardoctitle \\ [2.5ex]
           \stardocversion} \\ [2ex]
\rule{80mm}{0.5mm}
\end{center}

    \setlength{\parskip}{0mm}
    \tableofcontents
    \setlength{\parskip}{\medskipamount}
    \markboth{\stardocname}{\stardocname}
  \end{latexonly}
% ? End of Latex document table of contents
% -----------------------------------------------------------------------------
\cleardoublepage
\renewcommand{\thepage}{\arabic{page}}
\setcounter{page}{1}

% ? Main text

\begin{latexonly}
\null\vspace {5mm}
\begin {center}
\rule{80mm}{0.5mm} \\ [1ex]
{\Large\bf \stardoctitle \\ [2.5ex]
           \stardocversion} \\ [2ex]
\rule{80mm}{0.5mm}
\end{center}
\vspace{30mm}
\end{latexonly}

\section{\xlabel{introduction}Introduction}
\label{introduction}

This package enables the user to store references to HDS objects in
special HDS reference objects.  Although it would be possible for users
to concoct their own scheme, the use of this package will assist in
portability and will in any case avoid re-inventing the wheel.

\section{\xlabel{facilities}Facilities}
\label{facilities}

The package allows reference objects to be created and written and it
allows locators to referenced objects to be obtained.

The referenced object may be defined as \emph{internal\/} in which case
it is assumed to be within the same container file as the reference
object itself, even if the reference object is copied to another
container file.  In that case the reference must point to an object
which has the same pathname within the new file as it had in the old
one.  References which are not \emph{internal\/} will point to a named
container file.

Reference objects may be copied and erased using DAT\_COPY and
DAT\_ERASE\@.  Care must be taken when copying reference objects or
referenced objects; otherwise the reference may no longer point to the
referenced object.

Referenced objects must exist at the time the reference is made or used.

The following subroutines are available:
\begin{description}
\item [REF\_CRPUT] --- Create a reference object and put a reference in it.
\item [REF\_FIND] --- Obtain locator to an object (possibly \emph{via} a
reference).
\item [REF\_GET] --- Obtain a locator to a referenced object.
\item [REF\_NEW] --- Create an empty reference object.
\item [REF\_PUT] --- Put a reference into a reference object.
\item [REF\_ANNUL] --- Annul a locator which may have been obtained 
\emph{via} a reference.
\end{description}

\section{\xlabel{using_the_package}Using the package}
\label{using_the_package}

Two main uses for this package are foreseen:

\begin{enumerate}
\item To maintain a catalogue of HDS objects.
\item To avoid duplicating a large dataset.
\end{enumerate}

As an example of the second case, suppose that a large dataset is
logically required to form part of a number of other datasets.  To
avoid duplicating the common dataset, the others may contain a
reference to it.

For example:
\begin{quote}
\begin{verbatim}
   Name                type                  Comments

DATA                DATA_SETS
  .SET1             SPECTRUM
     .AXIS1         _REAL(1024)          Actual axis data
     .DATA_ARRAY    _REAL(1024)
  .SET2             SPECTRUM
     .AXIS1         REFERENCE_OBJ        Reference to DATA.SET1.AXIS1
     .DATA_ARRAY    _REAL(1024)
  .SET3             SPECTRUM
     .AXIS1         REFERENCE_OBJ        Reference to DATA.SET1.AXIS1
     .DATA_ARRAY    _REAL(1024)  
   -
  etc.
\end{verbatim}
\end{quote}

Then a piece of code which handles structures of type SPECTRUM, which
would normally contain the axis data in .AXIS1 (as SET1 does), could be
modified as follows to handle an object .AXIS1 containing either the
actual axis data or a reference to the object which does contain the
actual axis data.

\begin{quote}
\begin{verbatim}
*    LOC1 is a locator associated with a SPECTRUM object
*    Obtain locator to AXIS data
      CALL DAT_FIND(LOC1, 'AXIS1', LOC2, STATUS)
*    Modification to allow AXIS1 to be a reference object
*    Check type of object
      CALL DAT_TYPE(LOC2, TYPE, STATUS)
      IF (TYPE .EQ. 'REFERENCE_OBJ') THEN
          CALL REF_GET(LOC2, 'READ', LOC3, STATUS)
          CALL DAT_ANNUL(LOC2, STATUS)
          CALL DAT_CLONE(LOC3, LOC2, STATUS)
          CALL DAT_ANNUL(LOC3, STATUS)
      ENDIF
*    End of modification
*    LOC2 now locates the axis data wherever it is.
\end{verbatim}
\end{quote}

This code has been packaged into the subroutine \textbf{REF\_FIND} which
can be used instead of DAT\_FIND in cases where the component requested
may be a reference object.

When a locator which has been obtained in this way is finished with, it
should be annulled using REF\_ANNUL rather than DAT\_ANNUL.  This is so
that, if the locator was obtained \emph{via} a reference, the HDS\_OPEN
for the container file may be matched by an HDS\_CLOSE\@.  \emph{Note
that this should only be done when any other locators derived from the
locator to the referenced object are also finished with.}

\section{\xlabel{implementation}Implementation}
\label{implementation}

The way in which the package is implemented is described here for interest.
Programmers should not make use of this information; otherwise portability
is compromised.

A reference object is an HDS structure of type \verb%REFERENCE_OBJ% with
two components, \texttt{FILE} and \texttt{PATH}, of type 
\verb%_CHAR*(REF__SZREF)%\@.  \texttt{REF\_\_SZREF} is defined in the 
\texttt{REF\_PAR} include file.

\begin{description}

\item[FILE] contains the name of the container file for the referenced
object.  This is set to spaces if the reference is \emph{internal}\/.

\item[PATH] contains the pathname of the referenced object (as supplied
by HDS\_TRACE)\@.  The name of the top level component of the pathname
will not be used in finding the locator for the referenced object.
This fact allows structures containing internal references to be copied
but the path below the top level must still lead to an appropriate
object.

\end{description}

Locators obtained {\em via} a reference are flagged as such by being
linked to the  group \$\$REFERENCED\$ using the subroutine HDS\_LINK.
This fact is used by REF\_ANNUL in determining whether or not
HDS\_CLOSE should be called for the container file of the object
specified by the locator argument.  Note that the effect of calling
HDS\_CLOSE is to counter the HDS\_OPEN done in obtaining a locator to
the referenced object.  The container file will only be physically
closed if the container file reference count goes to zero.

\section{\xlabel{error_handling}Error handling}
\label{error_handling}

The REF routines adhere throughout to the Starlink error-handling
strategy described in the MERS document, \xref{SUN/104}{sun104}{}.  Most of the routines
therefore carry an integer inherited status argument called STATUS and
will return without action unless this is set to the value
SAI\_\_OK\footnote{The symbolic constant SAI\_\_OK is defined in the
include file SAE\_PAR.} when they are invoked. When necessary, error
reports are made through the EMS\_ routines in the manner described in
\xref{SSN/4}{ssn4}{}.  This gives complete compatibility with the use
of ERR\_ and MSG\_ routines in applications
(\xref{SUN/104}{sun104}{}).

\section{\xlabel{compiling_and_linking}Compiling and linking}
\label{compiling_and_linking}

Before compiling applications which use the REF library on UNIX
systems, you should normally ``log in'' for REF software development
with the following shell command:

\begin{quote}
\begin{verbatim}
% ref_dev
\end{verbatim}
\end{quote}

This will create links in your current working directory which refer to
the REF include files. You may then refer to these files using their
standard (upper case) names without having to know where they actually
reside. These links will persist, but may be removed at any time,
either explicitly or with the command:

\begin{quote}
\begin{verbatim}
% ref_dev remove
\end{verbatim}
\end{quote}

If you do not ``log in'' in this way, then references to REF include files
should be in lower case and must contain an absolute pathname identifying the
Starlink include file directory, thus:

\begin{quote}
\begin{verbatim}
INCLUDE '/star/include/ref_par'
\end{verbatim}
\end{quote}

The former method is recommended.

Applications which use the ADAM programming environment
(\xref{SG/4}{sg4}{}) may be linked with the REF library by specifying
\texttt{`ref\_link\_adam`} on the appropriate command line. Thus, for
instance, an ADAM A-task which calls REF routines might be linked as
follows:

\begin{quote}
\begin{verbatim}
% alink adamprog.f `ref_link_adam`
\end{verbatim}
\end{quote}

(note the use of backward quote characters, which are required).

``Stand-alone'' (\emph{i.e.}\ non-ADAM) applications which use the REF library
may be linked by specifying \texttt{`ref\_link`} on the compiler
command line. Thus, to compile and link a stand-alone application
called \texttt{`prog'}, the following might be used:

\begin{quote}
\begin{verbatim}
% f77 prog.f `ref_link` -o prog
\end{verbatim}
\end{quote}

%  End of main part.

\appendix
\cleardoublepage
\section{\xlabel{routine_descriptions}Routine Descriptions}
\label{routine_descriptions}

\sstroutine{
   REF\_ANNUL
}{
   Annul a locator to a referenced object
}{
   \sstdescription{
      This routine annuls the locator and, if the locator was linked to
      group \$\$REFERENCED\$, issues HDS\_CLOSE for the container file of
      the object.
   }
   \sstinvocation{
      CALL REF\_ANNUL( LOC, STATUS )
   }
   \sstarguments{
      \sstsubsection{
         LOC = CHARACTER $*$ ( DAT\_\_SZLOC ) (Given and Returned)
      }{
         Locator to be annulled.
      }
      \sstsubsection{
         STATUS = INTEGER (Given and Returned)
      }{
         Inherited global status.
      }
   }
   \sstnotes{
      This routine attempts to execute even if STATUS is set on entry,
      although no further error report will be made if it subsequently
      fails under these circumstances. In particular, it will fail if
      the locator supplied is not initially valid, but this will only
      be reported if STATUS is set to SAI\_\_OK on entry.
   }
}
\sstroutine{
   REF\_CRPUT
}{
   Create and write a reference object
}{
   \sstdescription{
      This routine creates a reference object as a component of a
      specified structure and writes a reference to an HDS object in
      it.  If the specified component already exists and is a reference
      object, it will be used. If it is not a reference object, an
      error is reported.  The reference may be described as {\tt "}internal{\tt "}
      which means that the referenced object is in the same container
      file as the reference object.
   }
   \sstinvocation{
      CALL REF\_CRPUT( ELOC, CNAME, LOC, INTERN, STATUS )
   }
   \sstarguments{
      \sstsubsection{
         ELOC = CHARACTER $*$ ( DAT\_\_SZLOC ) (Given)
      }{
         A locator associated with the structure which is to contain
         the reference object.
      }
      \sstsubsection{
         CNAME = CHARACTER $*$ ( DAT\_\_SZNAM ) (Given)
      }{
         The component name of the reference object to be created.
      }
      \sstsubsection{
         LOC = CHARACTER $*$ ( $*$ ) (Given)
      }{
         A locator associated with the object to be referenced.
      }
      \sstsubsection{
         INTERN = LOGICAL (Given)
      }{
         Whether or not the referenced object is {\tt "}internal{\tt "}.  Set this
         to .TRUE. if the reference is {\tt "}internal{\tt "} and to .FALSE. if it
         is not.
      }
      \sstsubsection{
         STATUS = INTEGER (Given and Returned)
      }{
         Inherited global status.
      }
   }
}
\sstroutine{
   REF\_FIND
}{
   Get locator to data object (via reference if necessary)
}{
   \sstdescription{
      This routine gets a locator to a component of a specified
      structure or, if the component is a reference object, it gets a
      locator to the object referenced. Any locator obtained in this
      way should be annulled, when finished with, by REF\_ANNUL so that
      the top-level object will also be closed if the locator was
      obtained via a reference.
   }
   \sstinvocation{
      CALL REF\_FIND( ELOC, CNAME, MODE, LOC, STATUS )
   }
   \sstarguments{
      \sstsubsection{
         ELOC = CHARACTER $*$ ( $*$ ) (Given)
      }{
         The locator of a structure.
      }
      \sstsubsection{
         CNAME = CHARACTER $*$ ( $*$ ) (Given)
      }{
         The name of the component of the specified structure.
      }
      \sstsubsection{
         MODE = CHARACTER $*$ ( $*$ ) (Given)
      }{
         Mode of access required to the object. ({\tt '}READ{\tt '}, {\tt '}WRITE{\tt '} or
         {\tt '}UPDATE{\tt '}). This is specified so that the container file of any
         referenced object can be opened in the correct mode.
      }
      \sstsubsection{
         LOC = CHARACTER $*$ ( DAT\_\_SZLOC ) (Returned)
      }{
         A locator associated with the object found.
      }
      \sstsubsection{
         STATUS = INTEGER (given and Returned)
      }{
         Inherited global status.
      }
   }
}
\sstroutine{
   REF\_GET
}{
   Get locator to referenced data object
}{
   \sstdescription{
      This routine gets a locator to an HDS object referenced in a
      reference object and links it to the group \$\$REFERENCED\$.  Any
      locator obtained in this way should be annulled, when finished
      with, by REF\_ANNUL so that the top-level object will also be
      closed.
   }
   \sstinvocation{
      CALL REF\_GET( ELOC, MODE, LOC, STATUS )
   }
   \sstarguments{
      \sstsubsection{
         ELOC = CHARACTER $*$ ( DAT\_\_SZLOC ) (Given)
      }{
         A locator associated with the reference object
      }
      \sstsubsection{
         MODE = CHARACTER $*$ ( $*$ ) (Given)
      }{
         Mode of access required to the object. ({\tt '}READ{\tt '}, {\tt '}WRITE{\tt '} or
         {\tt '}UPDATE{\tt '}). This is specified so that the container file of any
         referenced object can be opened in the correct mode.
      }
      \sstsubsection{
         LOC = CHARACTER $*$ ( DAT\_\_SZLOC ) (Returned)
      }{
         A locator pointing to the object referenced.
      }
      \sstsubsection{
         STATUS = INTEGER (Given and Returned)
      }{
         Inherited global status.
      }
   }
}
\newpage
\sstroutine{
   REF\_NEW
}{
   Create a new reference object
}{
   \sstdescription{
      This routine creates a reference object as a component of a
      specified structure.  If the component already exists, an error
      is reported.
   }
   \sstinvocation{
      CALL REF\_NEW( ELOC, CNAME, STATUS )
   }
   \sstarguments{
      \sstsubsection{
         ELOC = CHARACTER $*$ ( DAT\_\_SZLOC ) (Given)
      }{
         A locator associated with the structure which is to contain
         the reference object.
      }
      \sstsubsection{
         CNAME = CHARACTER $*$ ( DAT\_\_SZNAM ) (Given)
      }{
         The name of the component to be created in the structure
         located by ELOC.
      }
      \sstsubsection{
         STATUS = INTEGER (Given and Returned)
      }{
         Inherited global status.
      }
   }
}
\sstroutine{
   REF\_PUT
}{
   Write a reference into a reference object
}{
   \sstdescription{
      This routine writes a reference to an HDS object into an existing
      reference structure.  An error is reported if an attempt is made
      to write a reference into an object which is not a reference
      object.  The reference may be described as {\tt "}internal{\tt "} which means
      that the referenced object is in the same container file as the
      reference object.
   }
   \sstinvocation{
      CALL REF\_PUT( ELOC, LOC, INTERN, STATUS )
   }
   \sstarguments{
      \sstsubsection{
         ELOC = CHARACTER $*$ ( $*$ ) (Given)
      }{
         A locator associated with the reference object.
      }
      \sstsubsection{
         LOC = CHARACTER $*$ ( $*$ ) (Given)
      }{
         A locator associated with the object to be referenced.
      }
      \sstsubsection{
         INTERN = LOGICAL (Given)
      }{
         Whether or not the referenced object is {\tt "}internal{\tt "}.  Set this
         to .TRUE. if the reference is {\tt "}internal{\tt "} and to .FALSE. if it
         is not.
      }
      \sstsubsection{
         STATUS = INTEGER (Given and Returned)
      }{
         Inherited global status.
      }
   }
}
\normalsize

\newpage
\section{\xlabel{adamstandalone_differences}ADAM/Stand-alone Differences}
\label{adamstandalone_differences}

Note that when using the stand-alone version of the REF library, it is
currently necessary to ensure that HDS\_START is called to activate HDS prior
to making calls to any REF routines. This requirement will be removed in
future, and is currently not required with the ADAM version.

\section{\xlabel{machinedependent_features}Machine-dependent Features}
\label{machinedependent_features}

The REF library contains no explicit use of machine-dependent features,
so its behaviour should be the same on all platforms on which it is
implemented.

However, external references to HDS objects (those not identified as
``internal'' to the REF library) will contain explicit file names, so
true portability of data (in the manner provided by HDS) cannot be
expected with REF when working with operating systems which have
different file naming conventions. If complete data portability is
required, then use of the REF library should be restricted to internal
references only.

\section{\xlabel{software_dependencies}Software Dependencies}
\label{software_dependencies}

The REF library explicitly depends on the following other Starlink packages:

\begin{description}
\item [HDS] --- Hierarchical data system (\xref{SUN/92}{sun92}{})
\item [EMS] --- Error message service (\xref{SSN/4}{ssn4}{})
\end{description}

Note that these packages may also depend on other sub-packages. Please
consult the relevant documentation for details.

% ? End of main text
\end{document}
