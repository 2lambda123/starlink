%------------------------------------------------------------------------------
%+				N B S . T E X
%
%  Function:
%    Starlink User Note describing the Noticeboard System
%
%  History:
%    11-Apr-88 - WFL - Original version (version 2.1)
%    20-May-89 - WFL - Version 2.2
%    08-Feb-90 - WFL - Version 2.3   (all files explicitly included)
%    15-Feb-90 - WFL - Version 2.3.1 (all files explicitly included)
%    20-Apr-93 - DJA - Version 2.4
%     5-Nov-93 - DJA - Version 2.4.1
%     2-Jun-94 - DJA - Version 2.4.2
%    16-Aug-94 - DJA - Version 2.4.3
%    14-Dec-94 - DJA - Version 2.5
%    16-Feb-95 - DJA - Version 2.5.1
%    17-May-95 - DJA - Version 2.5.2
%    26-Jul-95 - DJA - Version 2.5.3
%    11-Sep-95 - DJA - Version 2.5.4
%    17-Oct-95 - MJB - version 2.5.5 (2.5.4 never released)
%    30-Apr-98 - TIMJ- convert Tex to latex2e so that WWW version
%                      can be generated.
%
%  Notes:
%
%    To generate the
%
%-
%------------------------------------------------------------------------------


\documentclass[twoside,11pt]{article}
\pagestyle{headings}

%------------------------------------------------------------------------------
\newcommand{\stardoccategory}  {Starlink User Note}
\newcommand{\stardocinitials}  {SUN}
\newcommand{\stardocnumber}    {77.8}
\newcommand{\stardocauthors}   {D.J. Allan}
\newcommand{\stardocdate}      {11 September 1995}
\newcommand{\stardoctitle}     {NBS \\ [1ex] The Noticeboard System}
\newcommand{\stardocversion}   {Version 2.5.5}
\newcommand{\stardocsource}    {sun\stardocnumber}
\newcommand{\stardocmanual}    {Programmer's Manual}
\newcommand{\stardocabstract}  {This document describes Version 2.5.5 of the Noticeboard System. It
is aimed at programmers interested in or intending to use the system and
contains all information necessary to use it.
}
%------------------------------------------------------------------------------

% ? End of document identification
% -----------------------------------------------------------------------------

% +
%  Name:
%     sun.tex
%
%  Purpose:
%     Template for Starlink User Note (SUN) documents.
%     Refer to SUN/199
%
%  Authors:
%     AJC: A.J.Chipperfield (Starlink, RAL)
%     BLY: M.J.Bly (Starlink, RAL)
%     PWD: Peter W. Draper (Starlink, Durham University)
%
%  History:
%     17-JAN-1996 (AJC):
%        Original with hypertext macros, based on MDL plain originals.
%     16-JUN-1997 (BLY):
%        Adapted for LaTeX2e.
%        Added picture commands.
%     13-AUG-1998 (PWD):
%        Converted for use with LaTeX2HTML version 98.2 and
%        Star2HTML version 1.3.
%     {Add further history here}
%
% -

\newcommand{\stardocname}{\stardocinitials /\stardocnumber}
\markboth{\stardocname}{\stardocname}
\setlength{\textwidth}{160mm}
\setlength{\textheight}{230mm}
\setlength{\topmargin}{-2mm}
\setlength{\oddsidemargin}{0mm}
\setlength{\evensidemargin}{0mm}
\setlength{\parindent}{0mm}
\setlength{\parskip}{\medskipamount}
\setlength{\unitlength}{1mm}


% -----------------------------------------------------------------------------
%  Hypertext definitions.
%  ======================
%  These are used by the LaTeX2HTML translator in conjunction with star2html.

%  Comment.sty: version 2.0, 19 June 1992
%  Selectively in/exclude pieces of text.
%
%  Author
%    Victor Eijkhout                                      <eijkhout@cs.utk.edu>
%    Department of Computer Science
%    University Tennessee at Knoxville
%    104 Ayres Hall
%    Knoxville, TN 37996
%    USA

%  Do not remove the %begin{latexonly} and %end{latexonly} lines (used by
%  LaTeX2HTML to signify text it shouldn't process).
%begin{latexonly}
\makeatletter
\def\makeinnocent#1{\catcode`#1=12 }
\def\csarg#1#2{\expandafter#1\csname#2\endcsname}

\def\ThrowAwayComment#1{\begingroup
    \def\CurrentComment{#1}%
    \let\do\makeinnocent \dospecials
    \makeinnocent\^^L% and whatever other special cases
    \endlinechar`\^^M \catcode`\^^M=12 \xComment}
{\catcode`\^^M=12 \endlinechar=-1 %
 \gdef\xComment#1^^M{\def\test{#1}
      \csarg\ifx{PlainEnd\CurrentComment Test}\test
          \let\html@next\endgroup
      \else \csarg\ifx{LaLaEnd\CurrentComment Test}\test
            \edef\html@next{\endgroup\noexpand\end{\CurrentComment}}
      \else \let\html@next\xComment
      \fi \fi \html@next}
}
\makeatother

\def\includecomment
 #1{\expandafter\def\csname#1\endcsname{}%
    \expandafter\def\csname end#1\endcsname{}}
\def\excludecomment
 #1{\expandafter\def\csname#1\endcsname{\ThrowAwayComment{#1}}%
    {\escapechar=-1\relax
     \csarg\xdef{PlainEnd#1Test}{\string\\end#1}%
     \csarg\xdef{LaLaEnd#1Test}{\string\\end\string\{#1\string\}}%
    }}

%  Define environments that ignore their contents.
\excludecomment{comment}
\excludecomment{rawhtml}
\excludecomment{htmlonly}

%  Hypertext commands etc. This is a condensed version of the html.sty
%  file supplied with LaTeX2HTML by: Nikos Drakos <nikos@cbl.leeds.ac.uk> &
%  Jelle van Zeijl <jvzeijl@isou17.estec.esa.nl>. The LaTeX2HTML documentation
%  should be consulted about all commands (and the environments defined above)
%  except \xref and \xlabel which are Starlink specific.


\newcommand{\htmladdnormallinkfoot}[2]{#1\footnote{#2}}
\newcommand{\htmladdnormallink}[2]{#1}
\newcommand{\htmladdimg}[1]{}
\newcommand{\hyperref}[4]{#2\ref{#4}#3}
\newcommand{\htmlref}[2]{#1}
\newcommand{\htmlimage}[1]{}
\newcommand{\htmladdtonavigation}[1]{}

\newenvironment{latexonly}{}{}
\newcommand{\latex}[1]{#1}
\newcommand{\html}[1]{}
\newcommand{\latexhtml}[2]{#1}
\newcommand{\HTMLcode}[2][]{}

%  Starlink cross-references and labels.
\newcommand{\xref}[3]{#1}
\newcommand{\xlabel}[1]{}

%  LaTeX2HTML symbol.
\newcommand{\latextohtml}{\LaTeX2\texttt{HTML}}

%  Define command to re-centre underscore for Latex and leave as normal
%  for HTML (severe problems with \_ in tabbing environments and \_\_
%  generally otherwise).
\renewcommand{\_}{\texttt{\symbol{95}}}

% -----------------------------------------------------------------------------
%  Debugging.
%  =========
%  Remove % on the following to debug links in the HTML version using Latex.

% \newcommand{\hotlink}[2]{\fbox{\begin{tabular}[t]{@{}c@{}}#1\\\hline{\footnotesize #2}\end{tabular}}}
% \renewcommand{\htmladdnormallinkfoot}[2]{\hotlink{#1}{#2}}
% \renewcommand{\htmladdnormallink}[2]{\hotlink{#1}{#2}}
% \renewcommand{\hyperref}[4]{\hotlink{#1}{\S\ref{#4}}}
% \renewcommand{\htmlref}[2]{\hotlink{#1}{\S\ref{#2}}}
% \renewcommand{\xref}[3]{\hotlink{#1}{#2 -- #3}}
%end{latexonly}
% -----------------------------------------------------------------------------
% ? Document specific \newcommand or \newenvironment commands.


%+
%  Name:
%     SST.TEX

%  Purpose:
%     Define LaTeX commands for laying out Starlink routine descriptions.

%  Language:
%     LaTeX

%  Type of Module:
%     LaTeX data file.

%  Description:
%     This file defines LaTeX commands which allow routine documentation
%     produced by the SST application PROLAT to be processed by LaTeX and
%     by LaTeX2html. The contents of this file should be included in the
%     source prior to any statements that make of the sst commnds.

%  Notes:
%     The style file html.sty provided with LaTeX2html needs to be used.
%     This must be before this file.

%  Authors:
%     RFWS: R.F. Warren-Smith (STARLINK)
%     PDRAPER: P.W. Draper (Starlink - Durham University)

%  History:
%     10-SEP-1990 (RFWS):
%        Original version.
%     10-SEP-1990 (RFWS):
%        Added the implementation status section.
%     12-SEP-1990 (RFWS):
%        Added support for the usage section and adjusted various spacings.
%     8-DEC-1994 (PDRAPER):
%        Added support for simplified formatting using LaTeX2html.
%     {enter_further_changes_here}

%  Bugs:
%     {note_any_bugs_here}

%-

%  Define length variables.
\newlength{\sstbannerlength}
\newlength{\sstcaptionlength}
\newlength{\sstexampleslength}
\newlength{\sstexampleswidth}

%  Define a \tt font of the required size.
\latex{\newfont{\ssttt}{cmtt10 scaled 1095}}
\html{\newcommand{\ssttt}{\tt}}

%  Define a command to produce a routine header, including its name,
%  a purpose description and the rest of the routine's documentation.
\newcommand{\sstroutine}[3]{
   \goodbreak
   \rule{\textwidth}{0.5mm}
   \vspace{-7ex}
   \newline
   \settowidth{\sstbannerlength}{{\Large {\bf #1}}}
   \setlength{\sstcaptionlength}{\textwidth}
   \setlength{\sstexampleslength}{\textwidth}
   \addtolength{\sstbannerlength}{0.5em}
   \addtolength{\sstcaptionlength}{-1.0\sstbannerlength}
   \addtolength{\sstcaptionlength}{-5.0pt}
   \settowidth{\sstexampleswidth}{{\bf Examples:}}
   \addtolength{\sstexampleslength}{-\sstexampleswidth}
   \parbox[t]{\sstbannerlength}{\flushleft{\Large {\bf #1}}}
   \parbox[t]{\sstcaptionlength}{\center{\Large #2}}
%   \parbox[t]{\sstbannerlength}{\flushright{\Large {\bf #1}}}
   \begin{description}
      #3
   \end{description}
}

%  Format the description section.
\newcommand{\sstdescription}[1]{\item[Description:] #1}

%  Format the usage section.
\newcommand{\sstusage}[1]{\item[Usage:] \mbox{}
\\[1.3ex]{\raggedright \ssttt #1}}

%  Format the invocation section.
\newcommand{\sstinvocation}[1]{\item[Invocation:]\hspace{0.4em}{\tt #1}}

%  Format the arguments section.
\newcommand{\sstarguments}[1]{
   \item[Arguments:] \mbox{} \\
   \vspace{-3.5ex}
   \begin{description}
      #1
   \end{description}
}

%  Format the returned value section (for a function).
\newcommand{\sstreturnedvalue}[1]{
   \item[Returned Value:] \mbox{} \\
   \vspace{-3.5ex}
   \begin{description}
      #1
   \end{description}
}

%  Format the parameters section (for an application).
\newcommand{\sstparameters}[1]{
   \item[Parameters:] \mbox{} \\
   \vspace{-3.5ex}
   \begin{description}
      #1
   \end{description}
}

%  Format the examples section.
\newcommand{\sstexamples}[1]{
   \item[Examples:] \mbox{} \\
   \vspace{-3.5ex}
   \begin{description}
      #1
   \end{description}
}

%  Define the format of a subsection in a normal section.
\newcommand{\sstsubsection}[1]{ \item[{#1}] \mbox{} \\}

%  Define the format of a subsection in the examples section.
\newcommand{\sstexamplesubsection}[2]{\sloppy
\item[\parbox{\sstexampleslength}{\ssttt #1}] \mbox{} \vspace{1.0ex}
\\ #2 }

%  Format the notes section.
\newcommand{\sstnotes}[1]{\item[Notes:] \mbox{} \\[1.3ex] #1}

%  Provide a general-purpose format for additional (DIY) sections.
\newcommand{\sstdiytopic}[2]{\item[{\hspace{-0.35em}#1\hspace{-0.35em}:}]
\mbox{} \\[1.3ex] #2}

%  Format the implementation status section.
\newcommand{\sstimplementationstatus}[1]{
   \item[{Implementation Status:}] \mbox{} \\[1.3ex] #1}

%  Format the bugs section.
\newcommand{\sstbugs}[1]{\item[Bugs:] #1}

%  Format a list of items while in paragraph mode.
\newcommand{\sstitemlist}[1]{
  \mbox{} \\
  \vspace{-3.5ex}
  \begin{itemize}
     #1
  \end{itemize}
}

%  Define the format of an item.
\newcommand{\sstitem}{\item}

%% Now define html equivalents of those already set. These are used by
%  latex2html and are defined in the html.sty files.
\begin{htmlonly}

%  sstroutine.
   \newcommand{\sstroutine}[3]{
      \subsection{#1\xlabel{#1}-\label{#1}#2}
      \begin{description}
         #3
      \end{description}
   }

%  sstdescription
   \newcommand{\sstdescription}[1]{\item[Description:]
      \begin{description}
         #1
      \end{description}
      \\
   }

%  sstusage
   \newcommand{\sstusage}[1]{\item[Usage:]
      \begin{description}
         {\ssttt #1}
      \end{description}
      \\
   }

%  sstinvocation
   \newcommand{\sstinvocation}[1]{\item[Invocation:]
      \begin{description}
         {\ssttt #1}
      \end{description}
      \\
   }

%  sstarguments
   \newcommand{\sstarguments}[1]{
      \item[Arguments:] \\
      \begin{description}
         #1
      \end{description}
      \\
   }

%  sstreturnedvalue
   \newcommand{\sstreturnedvalue}[1]{
      \item[Returned Value:] \\
      \begin{description}
         #1
      \end{description}
      \\
   }

%  sstparameters
   \newcommand{\sstparameters}[1]{
      \item[Parameters:] \\
      \begin{description}
         #1
      \end{description}
      \\
   }

%  sstexamples
   \newcommand{\sstexamples}[1]{
      \item[Examples:] \\
      \begin{description}
         #1
      \end{description}
      \\
   }

%  sstsubsection
   \newcommand{\sstsubsection}[1]{\item[{#1}]}

%  sstexamplesubsection
   \newcommand{\sstexamplesubsection}[2]{\item[{\ssttt #1}] #2}

%  sstnotes
   \newcommand{\sstnotes}[1]{\item[Notes:] #1 }

%  sstdiytopic
   \newcommand{\sstdiytopic}[2]{\item[{#1}] #2 }

%  sstimplementationstatus
   \newcommand{\sstimplementationstatus}[1]{
      \item[Implementation Status:] #1
   }

%  sstitemlist
   \newcommand{\sstitemlist}[1]{
      \begin{itemize}
         #1
      \end{itemize}
      \\
   }
%  sstitem
   \newcommand{\sstitem}{\item}

\end{htmlonly}

%  End of "sst.tex" layout definitions.
%.

%+                              M A N . S T Y
%
%  Module name:
%    MAN.STY
%
%  Function:
%    Default definitions for \LaTeX\ macros used in MAN output
%
%  Description:
%    As much as possible of the output from the MAN automatic manual generator
%    uses calls to user-alterable macros rather than direct calls to built-in
%    \LaTeX\ macros. This file contains the default definitions for these
%    macros.
%
%  Language:
%    \LaTeX
%
%  Support:
%    William Lupton, {AAO}
%-
%  History:
%    16-Nov-88 - WFL - Add definitions to permit hyphenation to work on
%		 words containing special characters and in teletype fonts.
%    21-Mar-90 - WFL - Use \makeatletter and \makeatother; tidy
%    14-Nov-91 - WFL - Add \manroutinebreakitem that inserts a line break
%	       after any text that appears on the initial line of the paragraph
%    15-Nov-91 - WFL - Always put \mbox{} after \item to protect against missing
%	       item text; remove spaces before { and [
%    20-Aug-92 - WFL - Extra parameter to \manroutine and shunt up existing ones
%	       (to facilitate use of \label and \ref)
%    10-Aug-94 - WFL - Cut out "\_" definition (screws up latex2html!); added
%	       unprocessed item name argument to \mansectionitem
%    18-Oct-94 - WFL - Decided that changing args to macros is too naughty so
%	       instead added new \mansectionitemA macro with the extra arg
%    19-Jul-96 - WFL - replaced hyphenchar with DeclreFontFamily for LaTeX2e

\typeout{Default MAN macros. Released 19th July 1996}

% permit use of @ characters in names

\makeatletter

% permit hyphenation when in teletype font (support 9,10,11,12 point only -
% could extend), define lccodes for special characters so that the hyphen-
% ation algorithm is not switched off. Define underscore character to be
% explicit underscore rather than lots of kerns etc.

%\hyphenchar\nintt=`-\hyphenchar\tentt=`-\hyphenchar\elvtt=`-\hyphenchar\twltt=`-
% Latex2e - tell cmtt to hyphenate
\DeclareFontFamily{OT1}{cmtt}{\hyphenchar\font=45}

% \lccode`_=`_\lccode`$=`$

%\renewcommand{\_}{{\tt\char'137}}

%+                      M A N _ I N T R O
%
%  Section name:
%    MAN_INTRO
%
%  Function:
%    Macros used in the .TEX_INTRO file
%
%  Description:
%    There are no such special macros.
%-

%+                      M A N _ S U M M A R Y
%
%  Section name:
%    MAN_SUMMARY
%
%  Function:
%    Macros used in the .TEX_SUMMARY file
%
%  Description:
%    There is a command to introduce a new section (mansection) and a list-like
%    environment (mansectionroutines) that handles the list of routines in the
%    current section. In addition a mansectionitem command can be used instead
%    of the item command to introduce a new routine in the current section.
%-

\newcommand{\mansection}[2]{\subsection{#1 --- #2}}

\newenvironment{mansectionroutines}{\begin{quote}\begin{description}}%
{\end{description}\end{quote}}

\newcommand{\mansectionitem}[1]{\item[#1:]\mbox{}}

% This form (which wflman V2.0 generates) passes an un-processed version of the
% section item name, useful for cross-references
\newcommand{\mansectionitemA}[2]{\item[#1:]\mbox{}}

%+                      M A N _ D E S C R
%
%  Section name:
%    MAN_DESCR
%
%  Function:
%    Macros used in the .TEX_DESCR file
%
%  Description:
%    There is a command to introduce a new routine (manroutine) and a list-like
%    environment (manroutinedescription) that handles the list of paragraphs
%    describing the current routine. In addition a manroutineitem or
%    manroutinebreakitem command can be used instead of the item command to
%    introduce a new paragraph for the current routine.
%
%    Two-column tables (the ones that can occur anywhere and which are
%    triggered by "=>" as the second token on a line) are bracketed by a
%    new environment (mantwocolumntable). Other sorts of table are introduced
%    by relevant  environments (manparametertable, manfunctiontable and
%    manvaluetable). The definitions of these environments call various other
%    user-alterable commands, thus allowing considerable user control over such
%    tables... (to be filled in when the commands have been written)
%-

\newcommand{\manrule}{\rule{\textwidth}{0.5mm}}

\newcommand{\manroutine}[3]{\subsection{#1 --- #2}}

\newenvironment{manroutinedescription}{\begin{description}}{\end{description}%
\manrule}

\newenvironment{mansubparameterdescription}{\begin{description}}%
{\end{description}}

\newcommand{\manroutineitem}[2]{\item[#1:] #2\mbox{}}

\newcommand{\manroutinebreakitem}[2]{\item[#1:] #2\hfill\\}

% two column tables

\newcommand{\mantwocolumncols}{||l|p{80mm}||}

\newcommand{\mantwocolumntop}{\hline}

\newcommand{\mantwocolumnblank}{\mantwocolumn@ss\mantwocolumn@hl%
\gdef\mantwocolumn@hl{}\gdef\mantwocolumn@ss{}}

\newcommand{\mantwocolumnbottom}{\mantwocolumn@ss\mantwocolumn@hl}

\newenvironment{mantwocolumntable}{%
\hspace*{\fill}\vspace*{-\partopsep}%
\begin{center}%
\begin{tabular}{\mantwocolumncols}%
\mantwocolumntop%
\gdef\mantwocolumn@ss{}\gdef\mantwocolumn@hl{}}%
{\mantwocolumnbottom%
\end{tabular}%
\end{center}}

\newcommand{\mantwocolumnentry}[1]{\mantwocolumn@ss\gdef\mantwocolumn@ss{\\}%
\gdef\mantwocolumn@hl{\hline}#1 & }

% parameter tables

\newcommand{\manparametercols}{lllp{80mm}}

\newcommand{\manparameterorder}[3]{#1 & #2 & #3 & }

\newcommand{\manparametertop}{}

\newcommand{\manparameterblank}{\gdef\manparameter@hl{}\gdef\manparameter@ss{}}

\newcommand{\manparameterbottom}{}

\newenvironment{manparametertable}{\gdef\manparameter@ss{}%
\gdef\manparameter@hl{}\hspace*{\fill}\vspace*{-\partopsep}\begin{trivlist}%
\item[]\begin{tabular}{\manparametercols}\manparametertop}{\manparameterbottom%
\end{tabular}\end{trivlist}}

\newcommand{\manparameterentry}[3]{\manparameter@ss\gdef\manparameter@ss{\\}%
\gdef\manparameter@hl{\hline}\manparameterorder{#1}{#2}{#3}}

% return tables

\newcommand{\manreturncols}{lllp{80mm}}

\newcommand{\manreturnorder}[3]{#1 & #2 & #3 & }

\newcommand{\manreturntop}{}

\newcommand{\manreturnblank}{\gdef\manreturn@hl{}\gdef\manreturn@ss{}}

\newcommand{\manreturnbottom}{}

\newenvironment{manreturntable}{\gdef\manreturn@ss{}%
\gdef\manreturn@hl{}\hspace*{\fill}\vspace*{-\partopsep}\begin{trivlist}%
\item[]\begin{tabular}{\manreturncols}\manreturntop}{\manreturnbottom%
\end{tabular}\end{trivlist}}

\newcommand{\manreturnentry}[3]{\manreturn@ss\gdef\manreturn@ss{\\}%
\gdef\manreturn@hl{\hline}\manreturnorder{#1}{#2}{#3}}

% function tables

\newcommand{\manfunctioncols}{||l|l|p{80mm}||}

\newcommand{\manfunctionorder}[2]{#1 & #2 & }

\newcommand{\manfunctiontop}{\hline}

\newcommand{\manfunctionblank}{\manfunction@ss\manfunction@hl%
\gdef\manfunction@ss{}\gdef\manfunction@hl{}}

\newcommand{\manfunctionbottom}{\manfunction@ss\manfunction@hl}

\newenvironment{manfunctiontable}{\gdef\manfunction@ss{}\gdef\manfunction@hl{}%
\hspace*{\fill}\vspace*{-\partopsep}\begin{center}\begin{tabular}%
{\manfunctioncols}\manfunctiontop}{\manfunctionbottom\end{tabular}\end{center}}

\newcommand{\manfunctionentry}[2]{\manfunction@ss\gdef\manfunction@ss{\\}%
\gdef\manfunction@hl{\hline}\manfunctionorder{#1}{#2}}

% value tables

\newcommand{\manvaluecols}{||l|l|l|p{80mm}||}

\newcommand{\manvalueorder}[3]{#1 & #2 & #3 & }

\newcommand{\manvaluetop}{\hline}

\newcommand{\manvalueblank}{\manvalue@ss\manvalue@hl\gdef\manvalue@ss{}%
\gdef\manvalue@hl{}}

\newcommand{\manvaluebottom}{\manvalue@ss\manvalue@hl}

\newenvironment{manvaluetable}{\gdef\manvalue@ss{}\gdef\manvalue@hl{}%
\hspace*{\fill}\vspace*{-\partopsep}\begin{center}\begin{tabular}%
{\manvaluecols}\manvaluetop}{\manvaluebottom\end{tabular}\end{center}}

\newcommand{\manvalueentry}[3]{\manvalue@ss\gdef\manvalue@ss{\\}%
\gdef\manvalue@hl{\hline}\manvalueorder{#1}{#2}{#3}}

% list environments

\newenvironment{manenumerate}{\begin{enumerate}}{\end{enumerate}}

\newcommand{\manenumerateitem}[1]{\item[#1]\mbox{}}

\newenvironment{manitemize}{\begin{itemize}}{\end{itemize}}

\newcommand{\manitemizeitem}{\item\mbox{}}

\newenvironment{mandescription}{\begin{description}\begin{description}}%
{\end{description}\end{description}}

\newcommand{\mandescriptionitem}[1]{\item[#1]\mbox{}}

% teletype conversion

\newcommand{\mantt}{\tt}

% "semi-verbatim" environment (modified from LaTeX source)

\def\@mansemiverbatim{\trivlist\item[]\if@minipage\else\vskip\parskip\fi%
\leftskip\@totalleftmargin\rightskip\z@%
\parindent\z@\parfillskip\@flushglue\parskip\z@%
\@tempswafalse\def\par{\if@tempswa\hbox{}\fi\@tempswatrue\@@par}%
\obeylines}

\def\mansemiverbatim{\@mansemiverbatim\frenchspacing\@vobeyspaces}

\let\endmansemiverbatim=\endtrivlist

% forbid use of @ characters in names

\makeatother





% ? End of document specific commands
% -----------------------------------------------------------------------------
%  Title Page.
%  ===========
\renewcommand{\thepage}{\roman{page}}
\begin{document}
\thispagestyle{empty}

%  Latex document header.
%  ======================
\begin{latexonly}
   CCLRC / {\textsc Rutherford Appleton Laboratory} \hfill {\textbf \stardocname}\\
   {\large Particle Physics \& Astronomy Research Council}\\
   {\large Starlink Project\\}
   {\large \stardoccategory\ \stardocnumber}
   \begin{flushright}
   \stardocauthors\\
   \stardocdate
   \end{flushright}
   \vspace{-4mm}
   \rule{\textwidth}{0.5mm}
   \vspace{5mm}
   \begin{center}
   {\Huge\textbf  \stardoctitle \\ [2.5ex]}
   {\LARGE\textbf \stardocversion \\ [4ex]}
   {\Huge\textbf  \stardocmanual}
   \end{center}
   \vspace{5mm}

% ? Add picture here if required for the LaTeX version.
%   e.g. \includegraphics[scale=0.3]{filename.ps}
% ? End of picture

% ? Heading for abstract if used.
   \vspace{10mm}
   \begin{center}
      {\Large\textbf Abstract}
   \end{center}
% ? End of heading for abstract.
\end{latexonly}

%  HTML documentation header.
%  ==========================
\begin{htmlonly}
   \xlabel{}
   \begin{rawhtml} <H1 ALIGN=CENTER> \end{rawhtml}
      \stardoctitle\\
      \stardocversion\\
      \stardocmanual
   \begin{rawhtml} </H1> <HR> \end{rawhtml}

% ? Add picture here if required for the hypertext version.
%   e.g. \includegraphics[scale=0.7]{filename.ps}
% ? End of picture

   \begin{rawhtml} <P> <I> \end{rawhtml}
   \stardoccategory\ \stardocnumber \\
   \stardocauthors \\
   \stardocdate
   \begin{rawhtml} </I> </P> <H3> \end{rawhtml}
      \htmladdnormallink{CCLRC}{http://www.cclrc.ac.uk} /
      \htmladdnormallink{Rutherford Appleton Laboratory}
                        {http://www.cclrc.ac.uk/ral} \\
      \htmladdnormallink{Particle Physics \& Astronomy Research Council}
                        {http://www.pparc.ac.uk} \\
   \begin{rawhtml} </H3> <H2> \end{rawhtml}
      \htmladdnormallink{Starlink Project}{http://www.starlink.ac.uk/}
   \begin{rawhtml} </H2> \end{rawhtml}
   \htmladdnormallink{\htmladdimg{source.gif} Retrieve hardcopy}
      {http://www.starlink.ac.uk/cgi-bin/hcserver?\stardocsource}\\

%  HTML document table of contents.
%  ================================
%  Add table of contents header and a navigation button to return to this
%  point in the document (this should always go before the abstract \section).
  \label{stardoccontents}
  \begin{rawhtml}
    <HR>
    <H2>Contents</H2>
  \end{rawhtml}
  \htmladdtonavigation{\htmlref{\htmladdimg{contents_motif.gif}}
        {stardoccontents}}

% ? New section for abstract if used.
  \section{\xlabel{abstract}Abstract}
% ? End of new section for abstract
\end{htmlonly}

% -----------------------------------------------------------------------------
% ? Document Abstract. (if used)
%  ==================
\stardocabstract
% ? End of document abstract
% -----------------------------------------------------------------------------
% ? Latex document Table of Contents (if used).
%  ===========================================
  \newpage
  \begin{latexonly}
    \setlength{\parskip}{0mm}
    \tableofcontents
    \setlength{\parskip}{\medskipamount}
    \markboth{\stardocname}{\stardocname}
  \end{latexonly}
% ? End of Latex document table of contents
% -----------------------------------------------------------------------------
\cleardoublepage
\renewcommand{\thepage}{\arabic{page}}
\setcounter{page}{1}


\section {Introduction}

%  To regenerate the NBS.TEX_* files, do MAN NBS.C.

%------------------------------------------------------------------------------
%\input {nbs.tex_intro}

      The noticeboard system routines provide a fast means for processes to
      share data in global memory. A given process may own as many noticeboards
      as it wishes and may access noticeboards owned by other processes.
      Normally the only process that writes to a noticeboard is its owner but
      other processes that know what they are doing can subvert this rule
      either by calling a special routine or else by accessing noticeboard data
      by using a pointer.

      The original interfaces were defined at the {AAO} workshop in October
      1985. {V1.0} was implemented in C by William Lupton at {RGO} in January
      1986. The changes for {V2.3} onwards were made by David Allan at the University
      of Birmingham.
      Refer to Section~\ref{V2.5 New Features} for details.

      Noticeboards are identified by name and each can contain a hierarchy of
      items. Each item has a name, a type, a structure / primitive attribute,
      and, if primitive, a maximum number of dimensions, a maximum number of
      bytes, a current shape and a current value. The type and shape are not
      used by the routines but their values can be put and got and they can be
      used when implementing higher-level routines on top of the noticeboard
      routines. Noticeboards are self-defining --- a process can find and access
      data from a noticeboard without knowing anything about what it contains.

      \section {General Description}
      \subsection {Names and Types} All names and types are converted to
      upper-case and all white space and non-printing characters in them are
      ignored. Thus `` {\tt William Lupton} '' is regarded as ``{\tt%
WILLIAMLUPTON}''. An
      upper limit of 16 characters is imposed on both names and types (the
      limit applies to the length after the removal of white space and
      non-printing characters).

      \subsection {Data Consistency} Provided there is only one writer to a
      noticeboard and the standard get and put routines are used, the routines
      will guarantee that consistent data is read from the noticeboard. A
      writer will never have to wait but a reader will retry potentially until
      a timeout occurs.

      \subsection {Static Definition} A (deliberate) restriction of the
      noticeboard system is that a noticeboard is static in structure. Its
      structure must be defined before any values are put into it and once the
      definition is complete no more items can be created. There are a set of
      routines called {\tt NBS\_DEFINE\_* } and {\tt NBS\_*\_DEFINITION}
      which allow definition
      of noticeboard contents, saving definitions to and restoring them from
      file (the initial state of a noticeboard is that all items have zero
      length values). Only once the definition phase is complete can the
      {\tt NBS\_PUT\_*} and {\tt NBS\_GET\_*} routines be used.

      The initial reason for this was efficiency. However it is not in fact
      particularly difficult to allow new items to created on the fly without
      compromising efficiency and this restriction should be seen more as a way
      of preventing the noticeboard routines from being used for purposes for
      which they were not designed and for which better tools (such as
      {\tt HDS}) exist. Note that this ``static'' restriction refers only to
      the creation of
      items in the noticeboard. It is always possible to change item shapes and
      values.

      \subsection {Saving to Disc} When saving a noticeboard to disc, the
      programmer has a choice as to whether to save only the definition
      or else the definition plus the data. If only the definition is saved
      then each time that the definition is restored
      all items will revert to having zero lengths. However if the data
      is saved as well the {\tt NBS\_RESTORE\_NOTICEBOARD} routine can be used
      to restore both the definition and the data, and a subsequent call
      to {\tt NBS\_SAVE\_NOTICEBOARD} will update the disc file so that
      the next
      time that the noticeboard is restored it will be in the same state
      as when it was saved.

      \subsection {C-callable Version} The {\tt NBS} routines have been %
      implemented to make them easy to use from Fortran, and for this
      reason all character strings are passed as standard Fortran character
      arguments (address of {\tt VMS} descriptor on {\tt VMS}, character
      pointer with trailing hidden length argument on most other architectures)
      and all other parameters are passed by reference.

      The routines are actually written in C and it
      would be unreasonable and wasteful to force C programmers to build
      descriptors just so that they could be decoded back to the same C strings
      that they started off as. For this reason, every {\tt NBS} routine %
(except a few which handle character strings) has an
      associated {\tt NBC} routine which passes all character parameters %
as C
      zero-terminated strings, all input scalar parameters by value and all
      other parameters by reference. See the source of the demonstration
      programs described in Appendix~\ref {Demonstration Programs} for examples
      of the use of this C-callable version.

      \subsection {Portability to UNIX}
      {\tt NBS} V2.4 was the first version of {\tt NBS} to be fully portable
      to UNIX -- V2.2 onwards was written portably but had several
      deficiencies and bugs. The shared memory services of UNIX System V
      are sufficiently close to the {\tt VMS} global section system
      services to require only a small amount of conditionally compiled code
      in the low-level ``create section'' and ``map section'' routines.
      There are however two limitations which may affect users planning
      to use the software on {\tt UNIX} platforms.

\begin{itemize}
\item The first and least serious is that {\tt UNIX} does not permit
a process to have a shared memory segment mapped more than once
the process's address space. {\tt NBS} has been written to take
account of this and will always return the same address in a request
to map a noticeboard once it is mapped the first time. It does however
keep a reference count, and will not unmap a noticeboard until this
drops to zero.
\item The system identifier (or key) for shared memory sections (and hence
noticeboards) on {\tt UNIX} is a four byte integer. NBS therefore hashes
the user supplied name into this integer. The algorithm for doing this
is much improved in V2.5.5 but it is still not proof against different
names mapping on to the same key value.
\end{itemize}

      \subsection {Status Conventions} All the {\tt NBS} routines use {%
\tt ADAM} modified
      status conventions in that they will do nothing if status is not {%
\tt NBS\_\_OK}
      (0) on entry. They can all optionally be called as functions, in which
      case the function value is the same as the returned status value. In
      all cases where an {\tt NBS} routine is the source of the error, an
      error report will be made using the error and message reporting service.

      \subsection {Terminology} These routines use the terminology ``item'' to
      refer to either a structured noticeboard object (ie, one with no values
      but possibly having lower-level objects) or to a primitive noticeboard
      object (ie, one with a shape and with values). The terms
      ``item identifier'', ``identifier'' or ``{\tt ID}'' refer to ``%
handles'' (cf {\tt HDS}
      locators) which allow access to all information pertaining to an item. As
      far as callers are concerned these are just integers. As far as the
      noticeboard system is concerned they are pointers to data structures
      called ``item\_descriptor''s. In the routine specifications they are
      regarded as integers. A zero {\mantt{ID}} is always an invalid {\mantt{%
ID}} and all routines
      will detect an attempt to use a zero {\mantt{ID}}. Similarly all %
routines that
      return {\mantt{ID}}'s will return zero {\mantt{ID}}'s on failure. Note %
that there is a danger
      of an access violation if true garbage {\mantt{ID}}'s are passed. This %
is unlikely
      to happen, because of the use of the modified status convention.

%------------------------------------------------------------------------------

\section {Examples of Using the {\tt NBS} Routines}

This section takes the form of a tutorial. Starting with a listing of a
noticeboard contents, we present the calls necessary to define that
noticeboard, to put the required values into it and for another process to
access those values.

Suppose that we wish to create and access a simple noticeboard. We will
use as an example a subset of the noticeboard that is maintained by the
{\tt AUTOFIB} control software. The notation used is the same as that used by
the {\tt NBTRACE} (Trace Noticeboard) program (see Section~\ref
{NBTRACE --- Trace Noticeboard Contents}) --- each structure entry is of the form:

\begin {verbatim}
    type name (children)
\end{verbatim}

and each primitive entry is of the form:

\begin {verbatim}
    type name[(maxd) dim1,dim2...] (actb/maxb/mod) val1,val2,...
\end{verbatim}

with appropriately indented entries. The shape information is shown only if
the maximum number of dimensions ({\tt maxd}) is greater than zero. The three
slash-separated numbers are actual number of bytes, maximum number of bytes and
modified count.

The result of running {\tt NBTRACE} on this noticeboard might be:

\begin {verbatim}
    Software version   = 5 (5)
    Size of section    = 13284 (33e4)
    Size of definition = 8952 (22f8)
    Noticeboard owner  = 405 (195)
    Modified count     = 2 (2)

    NOTICEBOARD AUTOFIB (3)
        _CHAR            CURRENT_CONFIG         (18/132/2) "current.fib"
        CURRENT_STATUS   CURRENT_STATUS[(1) 72] (0/4608/0)
        FIBRE_PARAMETERS FIBRE_PARAMETERS (1)
            _INTEGER     TRANS_MATRIX[(2) 2,2]  (16/16/2)  1,0,0,1
\end{verbatim}

\begin {itemize}
\item Type and shape have no significance to the {\tt NBS} routines --- the type
is just a character string and the shape is just a list of numbers. In the
above, {\tt HDS} primitive type conventions have been used and this allows a
sensible representation of values. Remember though that it is the {\tt NBTRACE}
program that is making use of the type information, not the {\tt NBS} routines.
\item The initial state of an item is that the actual number of bytes and the
modified count are both zero. {\tt CURRENT\_STATUS} above is in its initial
state.
\item It is possible to give an item a shape during the definition phase so an
item can initially have a non-scalar shape.
\item The noticeboard modified count is incremented just after updating any
item in the noticeboard. An ``update'' may be an update of an item's value, its
shape or its size.
\item An item's modified count is incremented just before updating an item
and just afterwards. Thus if it is odd then an item is in the process of
being updated, and the number of completed updates is {\tt MODIFIED~/~2}. An
``update'' may be an update of either an item's value or its shape.
\item You can use the {\tt NBS\_GET\_UPDATED} routine to determine whether
a noticeboard or a specific item has been updated since the last call to {\tt
NBS\_GET\_UPDATED}.
\end {itemize}

\subsection {Defining the Noticeboard Contents}

The following Fortran-like calls will define the contents of the above
noticeboard and save the definition in a disc file.

\begin {verbatim}
      INCLUDE   `NBS_ERR'       ! Error code definitions
      INCLUDE   `SAE_PAR'       ! Error code definitions

      INTEGER   STATUS          ! Modified STATUS variable
      INTEGER   TOPSID          ! Top-level static ID
      INTEGER   SID             ! General purpose static ID
      INTEGER   FIBSID          ! FIBRE_PARAMETERS static ID
      INTEGER   DIMS(2)         ! TRANS_MATRIX dimensions

      STATUS = SAI__OK          ! Initially set status to be OK
\end{verbatim}

Begin the noticeboard definition.

\begin {verbatim}
      NBS_BEGIN_DEFINITION (TOPSID,STATUS)      ! Top-level static ID
\end{verbatim}

Define the primitive item {\tt CURRENT\_CONFIG}. This is a character string
of maximum length 132 bytes. We choose to regard it as a scalar.

\begin {verbatim}
      NBS_DEFINE_PRIMITIVE (TOPSID,             ! Parent static ID
                            `CURRENT_CONFIG',   ! Name of item
                            `_CHAR',            ! Type of item
                            0,132,              ! Max # dims and bytes
                            SID,STATUS)         ! Returned static ID
\end{verbatim}

Define the primitive item {\tt CURRENT\_STATUS}. This is a 1D array of 72 64
byte records.

\begin {verbatim}
      NBS_DEFINE_PRIMITIVE (TOPSID,             ! Parent static ID
                            `CURRENT_STATUS',   ! Name of item
                            `CURRENT_STATUS',   ! Type of item
                            1,72*64,            ! Max # dims and bytes
                            SID,STATUS)         ! Returned static ID

      NBS_DEFINE_SHAPE     (SID,                ! Static ID
                            1,72,STATUS)        ! Actual # dims and dims
\end{verbatim}

Define the structured item {\tt FIBRE\_PARAMETERS}. Structured items cannot
have shapes or values.

\begin {verbatim}
      NBS_DEFINE_STRUCTURE (TOPSID,             ! Parent static ID
                            `FIBRE_PARAMETERS', ! Name of item
                            `FIBRE_PARAMETERS', ! Type of item
                            FIBSID,STATUS)      ! Returned static ID
\end{verbatim}

Define the lower-level primitive item {\tt TRANS\_MATRIX}. This is a 2D array
of 2 x 2 integers.

\begin {verbatim}
      NBS_DEFINE_PRIMITIVE (FIBSID,             ! Parent static ID
                            `TRANS_MATRIX',     ! Name of item
                            `_INTEGER',         ! Type of item
                            2,4*4,              ! Max # dims and bytes
                            SID,STATUS)         ! Returned static ID
      DIMS(1) = 2
      DIMS(2) = 2
      NBS_DEFINE_SHAPE     (SID,                ! Static ID
                            2,DIMS,STATUS)      ! Actual # dims and dims
\end{verbatim}

End the definition, writing to file {\tt AUTOFIB.NBD}.

\begin {verbatim}
      NBS_END_DEFINITION   (`AUTOFIB',          ! Name of noticeboard file
                            `DEFINITION_SAVE',  ! Write definition to disc
                            STATUS)
\end{verbatim}

\begin {itemize}
\item All of the routines used during the definition phase are called
{\tt NBS\_*\_DEFINITION} \\
or {\tt NBS\_DEFINE\_*}.
\end {itemize}

\subsection {Creating the Noticeboard}

Having defined the noticeboard contents, the actual noticeboard must be
created. There are three ways of doing this, depending on whether the
definition is being read from a file, the definition plus data is being read
from a file (this is illustrated in Section~\ref {Saving and Restoring of
Values}), or the noticeboard is being created in memory.

Restore the noticeboard definition from a file and create the noticeboard
\ldots

\begin {verbatim}
      NBS_RESTORE_DEFINITION (`AUTOFIB',            ! Noticeboard name
                              `AUTOFIB',            ! Name of noticeboard file
                              STATUS)
\end{verbatim}

\ldots or, instead of issuing the above {\tt NBS\_END\_DEFINITION} call, issue
this one.

\begin {verbatim}
      NBS_END_DEFINITION     (`AUTOFIB',            ! Noticeboard name
                              `CREATE_NOTICEBOARD', ! Don't write a file
                              STATUS)
\end{verbatim}

\begin {itemize}
\item Regardless of which method is used, the situation is the same after either
of these calls. The former would normally be used if the noticeboard definition
had been compiled earlier by another program and the latter would be used if
the noticeboard had just been defined and if no noticeboard definition file was
required.
\item The noticeboard now exists and is owned by this process. All items in it
have an actual size of zero and a modified count of zero.
\end {itemize}

\subsection {Finding the Noticeboard}

Any process that wants to use a noticeboard must map it. This is done in
precisely the same way by both the noticeboard owner and any other processes
wishing to use it.

\begin {verbatim}
      INTEGER   TOPID                           ! Top-level noticeboard ID
\end{verbatim}

Map (find) the noticeboard.

\begin {verbatim}
      NBS_FIND_NOTICEBOARD (`AUTOFIB',          ! Noticeboard name
                            TOPID,STATUS)       ! Top-level ID
\end{verbatim}

\begin {itemize}
\item This is the first {\tt NBS} routine that a non-owner need call.
\item It does no harm to make multiple calls to this routine, but each call will
map the noticeboard into a new part of virtual memory (on {\tt VMS} only ---
the {\tt UNIX} version will return the same virtual memory address) and will
return a different {\tt ID}.
\item This {\tt TOPID} is {\em not} the same as the {\tt TOPSID} returned by
the {\tt NBS\_BEGIN\_DEFINITION} routine. Static identifiers are used only
during the definition phase and we are no longer in this phase.
\end {itemize}

\subsection {Putting Values into the Noticeboard}

We can now put values into primitive items.

\begin {verbatim}
      INTEGER   CONID                   ! CURRENT_CONFIG ID
      INTEGER   STAID                   ! CURRENT_STATUS ID
      INTEGER   FIBID                   ! FIBRE_PARAMETERS ID
      INTEGER   MATID                   ! COORD_MATRIX ID

      CHARACTER STRING*(*)              ! String to write to CURRENT_CONFIG
      PARAMETER (STRING=`CURRENT.FIB')
      INTEGER   MATRIX(2,2)             ! Matrix to write to COORD_MATRIX
      DATA      MATRIX /1,0,0,1/
\end{verbatim}

Get the {\tt ID}s for all the items we want to write to.

\begin {verbatim}
      NBS_FIND_ITEM (TOPID,`CURRENT_CONFIG  ',CONID,STATUS)
      NBS_FIND_ITEM (TOPID,`CURRENT_STATUS  ',STAID,STATUS)
      NBS_FIND_ITEM (TOPID,`FIBRE_PARAMETERS',FIBID,STATUS)
      NBS_FIND_ITEM (FIBID,`COORD_MATRIX    ',MATID,STATUS)
\end{verbatim}

Put values to some of them.

\begin {verbatim}
      NBS_PUT_CVALUE (CONID,0,STRING,STATUS)
      NBS_PUT_VALUE  (MATID,0,16,MATRIX,STATUS)
      NBS_PUT_VALUE  (CONID,0,LEN(STRING),%REF(STRING),STATUS)
\end{verbatim}

\begin {itemize}
\item Normally one would get {\tt ID}s for the items of interest just after
finding the noticeboard. Finding an item involves a search and is relatively
slow, whereas putting and getting a value is very fast.
\item The second parameter to {\tt NBS\_PUT\_VALUE} and {\tt NBS\_PUT\_CVALUE}
is a byte offset. You
can use non-zero offsets to put a slice of the data associated with an item.
The current size of an item's data is the high-water mark of all {\tt PUT}s.
You can use {\tt NBS\_PUT\_SIZE} explicitly to set the size of an item's data.
\item Because the {\tt NBS} routines don't use the type information, it is
necessary to pass all values by reference. In Fortran this is the default
passing mechanism for all variables except for character strings, hence the
routine {\tt NBS\_PUT\_CVALUE} to handle this case.
\item However, on {\tt VMS} {\em only} the {\tt \%REF} function enables the
user to pass the actual character pointer to {\tt NBS\_PUT\_VALUE}, and supply
the length of the character string (or indeed, any other number) as the
maximum number of bytes to be read.
\item The noticeboard is now in the same state as in the listing at the start
of this section. If another process ran {\tt NBTRACE} at this stage, it would get
precisely the same output.
\end {itemize}

\subsection {Getting values from the Noticeboard}

Values can be accessed either by copying to and from the user's buffer or else
directly to and from the noticeboard. Both methods are illustrated here.

\begin {verbatim}
      INTEGER   ACTBYTES                ! Actual number of bytes stored
      INTEGER   POINTER                 ! Pointer to MATRIX data
      CHARACTER*20  CONF                ! Configuration to be read
\end{verbatim}

Get current value of {\tt MATRIX}. We expect the actual number of bytes to
be at least 16.

\begin {verbatim}
      NBS_GET_VALUE (MATID,0,16,MATRIX,ACTBYTES,STATUS)
      IF (ACTBYTES .LT. 16) THEN
         panic
      ENDIF
\end{verbatim}

Again, reading character strings is slightly different,

\begin {verbatim}
      NBS_GET_CVALUE (MATID,0,CONF,     ! Use the portable routine
                     ACTBYTES,STATUS)
      NBS_GET_VALUE (MATID,0,LEN(CONF), ! or the VMS specific call
          %REF(CONFIG),ACTBYTES,STATUS) ! using %REF
\end{verbatim}

Alternatively get a pointer to the actual data in the noticeboard.

\begin {verbatim}
      NBS_GET_POINTER (MATID,POINTER,STATUS)
      MATRIX_OP       (%VAL(POINTER),STATUS)
\end{verbatim}

\begin {itemize}
\item The second parameter to {\tt NBS\_GET\_VALUE} and {\tt NBS\_GET\_CVALUE}
is a byte offset. You can use non-zero offsets to get a slice of the
data associated with an item.
\item The length of the string passed into {\tt NBS\_GET\_CVALUE} serves
the same purpose as the third argument to {\tt NBS\_GET\_VALUE} --- ie. it
limits the amount of data that can be read.
\item It is not an {\tt NBS} error if an item contains fewer bytes than are
being asked for.
\item When accessing data directly via pointer, no checks are possible and the
modified count is not (and could not be) accessed.
\end {itemize}

\subsection {Finding out about Items}

There is a complete set of enquiry routines that allow programs that know
nothing about a noticeboard to navigate through it. {\tt NBTRACE} is just such a
program. A few of these routines are illustrated here.

\begin {verbatim}
      INCLUDE   `NBS_PAR'               ! Parameter definitions

      CHARACTER NAME*(NBS_K_MAXNAME)    ! Item name
      CHARACTER TYPE*(NBS_K_MAXTYPE)    ! Item type
      INTEGER   MAXBYTES                ! Maximum number of bytes
      INTEGER   ACTBYTES                ! Actual number of bytes
      INTEGER   MAXDIMS                 ! Maximum number of dimensions
      INTEGER   DIMS(7)                 ! Actual dimensions
      INTEGER   ACTDIMS                 ! Actual number of dimensions

      NBS_GET_NAME  (MATID,NAME,STATUS)
      NBS_GET_TYPE  (MATID,TYPE,STATUS)
      NBS_GET_SIZE  (MATID,MAXBYTES,ACTBYTES,STATUS)
      MAXDIMS = 7                       ! MAXDIMS is a MODIFIED parameter
      NBS_GET_SHAPE (MATID,MAXDIMS,DIMS,ACTDIMS,STATUS)
\end{verbatim}

\begin {itemize}
\item Higher-level routines may wish to check consistency between shape, size
and type. For example, in the above we expect {\tt ACTBYTES} to be 4 (the number
of bytes per pixel) times the product of the dimensions.
\end {itemize}

\subsection {Saving and Restoring of Values}
\label {Saving and Restoring of Values}

It is possible to save and restore the noticeboard data as well as just the
definition by using a different option when ending the noticeboard definition.

End the definition, writing it and the noticeboard data to file {\tt
AUTOFIB.NBD}.

\begin {verbatim}
      NBS_END_DEFINITION   (`AUTOFIB',          ! Name of noticeboard file
                            `NOTICEBOARD_SAVE', ! Write noticeboard to disc
                            STATUS)
\end{verbatim}

Restore the noticeboard definition plus data from a file and create the
noticeboard.

\begin {verbatim}
      NBS_RESTORE_NOTICEBOARD (`AUTOFIB',       ! Noticeboard name
                               `AUTOFIB',       ! Name of noticeboard file
                               STATUS)
\end{verbatim}

Map (find) the noticeboard.

\begin {verbatim}
      NBS_FIND_NOTICEBOARD (`AUTOFIB',          ! Noticeboard name
                            TOPID,STATUS)       ! Top-level ID
\end{verbatim}

Get the {\tt ID} for an item we want to write to and put its value.

\begin {verbatim}
      NBS_FIND_ITEM (TOPID,`CURRENT_CONFIG',CONID,STATUS)
      NBS_PUT_CVALUE (CONID,0,STRING,STATUS)
\end{verbatim}

Save the noticeboard to disc.

\begin {verbatim}
      NBS_SAVE_NOTICEBOARD (TOPID,              ! ID of any item on noticeboard
                            STATUS)
\end{verbatim}

\begin {itemize}
\item {\tt NBS\_RESTORE\_NOTICEBOARD} can be used to restore from a file
that contains only a definition as well as from a file that contains a
definition plus data. However if a file does not contain data the {\tt
NBS\_\_DATANOTSAVED} warning status is returned --- this indicates that the
noticeboard has been created but that it is in its initial state.
\item {\tt NBS\_RESTORE\_DEFINITION} cannot be used to restore from a file
that contains data in addition to a definition. In cases where both sorts
of file are to be accessed, use {\tt NBS\_RESTORE\_NOTICEBOARD}.
\item The implementation of {\tt NBS\_SAVE\_NOTICEBOARD} is inefficient
in that the entire definition plus data is copied to disc. Under operating
systems that support the mapping of files into virtual memory it could be
implemented using an ``update section'' call.
\item {\tt NBS\_SAVE\_NOTICEBOARD} does not create a new version of the
noticeboard definition file; it always overwrites the data in the file from
which it was restored.
\end {itemize}

\subsection {Losing Noticeboards}

Noticeboards can be explicitly lost (unmapped). Often you will not need to
worry about this and will be happy to allow them to be unmapped on image exit.
Sometimes, however, you may require explicit control. Note though that a
noticeboard is not actually deleted until the last process unmaps it.

Find a noticeboard and an item within it.

\begin{verbatim}
      NBS_FIND_NOTICEBOARD (`AUTOFIB',          ! Noticeboard name
                            TOPID,STATUS)       ! Top-level ID
      NBS_FIND_ITEM (TOPID,`CURRENT_CONFIG',CONID,STATUS)
\end{verbatim}

Lose the item and the noticeboard.

\begin{verbatim}
      NBS_LOSE_ITEM (CONID,`CHECK',STATUS)
      NBS_LOSE_NOTICEBOARD (TOPID,`CHECK',STATUS)
\end{verbatim}

Alternatively, if you like living dangerously, try this.

\begin{verbatim}
      NBS_LOSE_NOTICEBOARD (TOPID,`FORCE',STATUS)
\end{verbatim}

\begin{itemize}
\item {\tt FIND} and {\tt LOSE} calls are normally paired. If the {\tt CHECK}
flag is given, you can only lose items and noticeboards for which all derived
{\tt ID}s have already been lost.
\item If you know that there is no chance that any of a noticeboard's or item's
derived {\tt ID}s will be used, you can specify the {\tt FORCE} flag. This
allows the item or noticeboard to be lost regardless of whether any derived
{\tt ID}s are still in use.
\end{itemize}

\section {Compiling and Linking Applications}

\subsection{Compiling}
On both {\tt VMS} and {\tt UNIX} systems, FORTRAN source code may
include {\tt NBS} error codes and {\tt NBS} public constants using
the symbolic names {\tt NBS\_ERR} and {\tt NBS\_PAR} respectively.
On {\tt VMS} these are logical names and {\tt UNIX} the appropriate
symbolic links can be made in your current directory by typing,

\begin {verbatim}
    nbs_dev
\end{verbatim}

These links can be deleted using {\tt nbs\_dev remove} if required.

The error codes are standard {\tt VMS} ones and use facility number 1802. This
number is one of the ones allocated to the AAO. See Appendix C for a list of
the symbolic error codes.

\subsection{Linking on {\tt VMS}}

{\tt NBS} can be linked either in a stand-alone or ADAM fashion. The
approriate link commands are,

\begin {verbatim}
    LINK program,NBS_LINK/OPT        ! Stand-alone

    ALINK program,NBS_LINK/OPT       ! ADAM version
\end{verbatim}

The following {\tt LINK} options file is used stand-alone,

\begin {verbatim}
    NBS_IMAGE/SHARE
\end{verbatim}

and this one for linking with {\tt ADAM},

\begin {verbatim}
    NBS_IMAGE_ADAM/SHARE
\end{verbatim}

If you are linking in lots of other sub-systems as well, you may need to
incorporate these lines into your own options file (unfortunately, {\tt LINK}
options files cannot be nested).

\subsection{Linking on {\tt UNIX}}

The linking process on UNIX is even simpler than that of {\tt VMS}. To link
against the stand-alone version simply append a \verb+`nbs_link`+ to the
compiler or loader command line.

\begin {verbatim}
    f77 program.f `nbs_link`
\end{verbatim}

Similarly for the {\tt ADAM} version,

\begin {verbatim}
    alink program.f `nbs_link_adam`
\end{verbatim}

\newpage
\section {{\tt NBS} Routines Listed by Category}

In this section, routines are listed by category. For each routine there
is a one-line description of what it does. There is a full alphabetical
description of the routines in Appendix~\ref {Description of Individual
NBS Routines}.

%------------------------------------------------------------------------------
%\input {nbs.tex_summary}
\mansection {{\mantt{NBS\_TUNE}}}{      Routines used to alter global %
noticeboard system parameters}

      Two types of parameters can be altered. The first type is global to a
      process and not to a noticeboard (they are essentially Fortran {\mantt{%
COMMON}}
      block or C static variables). The second type is global to a noticeboard
      and are thus shared by all processes accessing that noticeboard. The
      original {\mantt{NBS\_TUNE}} routine is used to alter parameters of the %
first type
      and the newer {\mantt{NBS\_TUNE\_NOTICEBOARD}} is used to tune %
parameters of the
      second type. When a noticeboard is created, its initial parameter values
      are copied from the current values of the first type.

      When a parameter is altered, its previous value is returned and this
      permits a routine to alter a parameter, use the new value and then
      restore the parameter to its previous value.

      The {\mantt{NBS\_FIND}}, {\mantt{NBS\_GET}} and {\mantt{NBS\_PUT}} %
routines make rather complicated use of
      these values. For those parameters which are logical flags, they use
      the {\mantt{OR}} of the value of the first type and the value of the %
second type.
      For those parameters which are numeric values, they use the value of the
      first type.

\begin{mansectionroutines}
\mansectionitem {{\mantt{NBS\_TUNE}}}
      Alter the value of a global parameter
\mansectionitem {{\mantt{NBS\_TUNE\_NOTICEBOARD}}}
      Alter the value of a noticeboard-specific global parameter
\end{mansectionroutines}
\mansection {{\mantt{NBS\_DEFINITION}}}{      Routines used to define the %
noticeboard structure}

      A private memory area is obtained and item definitions are created in it.
      Storage is not allocated for data but nevertheless data pointers are
      defined as though the data began at address zero. These pointers are
      relocated at the time that the definition is either written to disc or
      else copied to a noticeboard.

\begin{mansectionroutines}
\mansectionitem {{\mantt{NBS\_BEGIN\_DEFINITION}}}
      Begin definition of the contents of a noticeboard
      and return a static identifier to the top level of the noticeboard
\mansectionitem {{\mantt{NBS\_DEFINE\_STRUCTURE}}}
      Define a new entry for a structured item within another structured item
      and return a static identifier to the new item
\mansectionitem {{\mantt{NBS\_DEFINE\_PRIMITIVE}}}
      Define a new entry for a primitive item within another structured item
      and return a static identifier to the new item
\mansectionitem {{\mantt{NBS\_DEFINE\_SHAPE}}}
      Define an initial shape for a primitive item
\mansectionitem {{\mantt{NBS\_END\_DEFINITION}}}
      End the definition of a noticeboard
      and then create the noticeboard, save the definition in a file, or save
      the definition plus data in a file
\mansectionitem {{\mantt{NBS\_RESTORE\_DEFINITION}}}
      Restore a noticeboard definition from file
      and create the noticeboard
\mansectionitem {{\mantt{NBS\_RESTORE\_NOTICEBOARD}}}
      Restore a noticeboard definition and data from file
      and create the noticeboard
\mansectionitem {{\mantt{NBS\_SAVE\_NOTICEBOARD}}}
      Save a noticeboard to its noticeboard definition file
\end{mansectionroutines}
\mansection {{\mantt{NBS\_FIND}}}{      Routines that find noticeboards and %
items within noticeboards}

      Items can be found either by name or by position. When a noticeboard
      is found a private copy of its item descriptors is made in which all
      non {\mantt{NIL}} pointers are relocated by the virtual address of the %
start of
      the relevant noticeboard. Thus, once a noticeboard has been found, all
      information pertaining to it can in fact be accessed using standard C
      structures.

\begin{mansectionroutines}
\mansectionitem {{\mantt{NBS\_FIND\_NOTICEBOARD}}}
      Find a named noticeboard and return an identifier to it
\mansectionitem {{\mantt{NBS\_FIND\_ITEM}}}
      Find an item with a specified name contained in a structure
      associated with a specified identifier and return the located item's
      {\mantt{ID}}
\mansectionitem {{\mantt{NBS\_FIND\_NTH\_ITEM}}}
      Find the Nth item contained in a structure
      associated with a specified identifier and return the located item's {%
\mantt{ID}}
\end{mansectionroutines}
\mansection {{\mantt{NBS\_LOSE}}}{      Routines that lose noticeboards and %
items within noticeboards}

      The noticeboard system maintains counts of how many items have been
      found using {\mantt{NBS\_FIND\_ITEM}} and {\mantt{NBS\_FIND\_NTH\_ITEM}}%
. The routines in this
      section can be used to indicate that noticeboards and items are no
      longer required. When a noticeboard is no longer required, it is marked
      for deletion. When an item is no longer required, its parent's item
      count is decremented. Normally, noticeboards and items can only be lost
      if the count of items found from them is zero (this is to minimise the
      chance of using an item {\mantt{ID}} to access a non-existent %
noticeboard), but
      this can be overridden.

\begin{mansectionroutines}
\mansectionitem {{\mantt{NBS\_LOSE\_NOTICEBOARD}}}
      Unmap a specified noticeboard
\mansectionitem {{\mantt{NBS\_LOSE\_ITEM}}}
      Declare an intention never again to use a specified item
\end{mansectionroutines}
\mansection {{\mantt{NBS\_PUT}}}{      Routines that write information to a %
noticeboard}

      This can normally only be done by the owner of the noticeboard. The only
      things which can be written are shape information and data. The item's
      modified count is incremented both before and after each write using the
      {\mantt{ADD\_INTERLOCKED}} instruction. The noticeboard's modified %
count is
      incremented after each write. These modified counts allow other processes
      to monitor the noticeboard for changes.

      When writing data to an item, it is possible to specify the offset
      at which the data is to start. The actual size of the item's data
      is the high-water mark of all the data that has been written to the
      item. For example, if 10 bytes are written at offset 0, the size is
      10 bytes, but if then 100 bytes are written at offset 4000, the size
      is 4100 bytes, even though bytes 11 to 4000 have not been written.
      It is guaranteed that all such unwritten bytes are zero. It is
      occasionally necessary to adjust the actual size of an item's data
      and there is a routine explicitly to do this. For example, the size
      of the above item could be adjusted back to 10 bytes. Note that if
      1 byte were now written at an offset of 10000, the item size would
      now be 10001 bytes and the 100 bytes at offset 4000 would retain the
      values that they were given earlier (they would not be reset to zero).

      There are two global parameters which can be altered using the
      {\mantt{NBS\_TUNE\_NOTICEBOARD}} routine and which affect the behaviour %
of these
      routines:

      1. {\mantt{WORLD\_WRITE}} is {\mantt{FALSE}} by default and this %
prevents processes other
         than the noticeboard's owner from writing to it. If set {\mantt{TRUE}%
}, this
         process will be allowed to write to the noticeboard even if it does
         not own it. If set {\mantt{TRUE}} then it is possible for a reader %
to read data
         that is currently varying, so only do this if you know what you are
         doing. All processes can always access all noticeboard's directly if
         they do so via pointer --- such access is not affected by the
         {\mantt{WORLD\_WRITE}} flag.

      2. {\mantt{INCREMENT\_MODIFY}} is {\mantt{TRUE}} by default and this %
causes the item's
         modified count to be incremented before and after the update and the
         noticeboard's modified count to be incremented after the update. If
         set {\mantt{FALSE}}, neither count is altered. Clearly, if this is %
done, other
         processes monitoring the modified counts will not realise that
         noticeboard data is changing.

      There is also a lower-level routine that must only be used by people who
      really know what they are doing. It increments either the noticeboard
      modified count or a primitive item's modified count. This {\em must} be
      called according to the same conventions as are applied in {\mantt{NBS\_%
PUT\_VALUE}}.
      If this is not done, the whole system is subverted.

      Finally, there is a routine which specifies a user-supplied routine to be
      called whenever an item's shape, data, size or modified count is altered.
      This routine is called in the context of the process which alters the
      noticeboard --- there are no facilities for notifying directly any other
      processes (although the user-supplied routine may of course do this
      itself).

\begin{mansectionroutines}
\mansectionitem {{\mantt{NBS\_PUT\_VALUE}}}
      Put a byte array into a slice of a primitive item
      associated with a specified identifier
\mansectionitem {{\mantt{NBS\_PUT\_CVALUE}}}
      Put a character string into a slice of a primitive item
      associated with a specified identifier
\mansectionitem {{\mantt{NBS\_PUT\_SHAPE}}}
      Put a new shape to a primitive item
      associated with a specified identifier
\mansectionitem {{\mantt{NBS\_PUT\_SIZE}}}
      Put a new size to a primitive item
      associated with a specified identifier
\mansectionitem {{\mantt{NBS\_INC\_MODIFIED}}}
      Increment the noticeboard modified count or an item modified count
      depending on whether this is a structured or primitive item
\mansectionitem {{\mantt{NBS\_PUT\_TRIGGER}}}
      Specify a routine to be called whenever a primitive item is updated
\end{mansectionroutines}
\mansection {{\mantt{NBS\_GET}}}{      Routines that read information from a %
noticeboard}

      Normally this is a very straightforward operation, but when reading
      shape information and data it is necessary to check that the owner of
      the information has not altered it whilst it is being read.

      There are three global parameters which can be altered using the
      {\mantt{NBS\_TUNE\_NOTICEBOARD}} routine and which affect the behaviour %
of the
      shape and data reading routines:

      1. {\mantt{CHECK\_MODIFY}} is {\mantt{TRUE}} by default and this means %
that the item's
         modified count is checked both before and after reading the data. The
         data is read repeatedly until the value of this count is even and
         unchanging or until a timeout occurs. If set {\mantt{FALSE}}, no %
such checks are
         made and no timeout can occur.

      2. {\mantt{TIMEOUT\_COUNT}} is 100 by default and is the maximum number %
of times
         that the data reading will be tried.

      3. {\mantt{TIMEOUT\_INTERVAL}} is 100 by default and is the delay in %
milliseconds
         between tries.

\begin{mansectionroutines}
\mansectionitem {{\mantt{NBS\_GET\_VALUE}}}
      Get a byte array from a slice of a primitive item
      associated with the specified identifier
\mansectionitem {{\mantt{NBS\_GET\_CVALUE}}}
      Get a byte array from a slice of a primitive item
      associated with the specified identifier and
      store in a character string
\mansectionitem {{\mantt{NBS\_GET\_SHAPE}}}
      Get the shape of a primitive item
      associated with the specified identifier
\mansectionitem {{\mantt{NBS\_GET\_MODIFIED}}}
      Get the noticeboard modified count or an item modified count
      depending on whether this is a structured or primitive item
\mansectionitem {{\mantt{NBS\_GET\_MODIFIED\_POINTER}}}
      Get a pointer to the noticeboard modified count or an item modified count
      depending on whether this is a structured or primitive item
\mansectionitem {{\mantt{NBS\_GET\_UPDATED}}}
      Determine whether a primitive item or the noticeboard has been updated
      since the noticeboard was found or this routine was last called.
\mansectionitem {{\mantt{NBS\_GET\_POINTER}}}
      Return a pointer to the first byte of the data of a primitive item
      associated with the specified identifier
\mansectionitem {{\mantt{NBS\_GET\_NAME}}}
      Get the name of an item
      associated with the specified identifier
\mansectionitem {{\mantt{NBS\_GET\_TYPE}}}
      Get the type of an item
      associated with the specified identifier
\mansectionitem {{\mantt{NBS\_GET\_SIZE}}}
      Get the maximum and actual sizes of a primitive item
      associated with the specified identifier
\mansectionitem {{\mantt{NBS\_GET\_PRIMITIVE}}}
      Determine whether or not an item is primitive
\mansectionitem {{\mantt{NBS\_GET\_PARENT}}}
      Get the identifier of an item's parent structure
\mansectionitem {{\mantt{NBS\_GET\_CHILDREN}}}
      Get the number of children of a structured item
\mansectionitem {{\mantt{NBS\_GET\_INFO}}}
      Get general non-character information on a given noticeboard
\mansectionitem {{\mantt{NBS\_GET\_CINFO}}}
      Get general character information on a given noticeboard
\end{mansectionroutines}
%------------------------------------------------------------------------------

\section {Routine Timings}
\label {Routine Timings}

Several operations are tested below on all architectures on which NBS has
been extensively tested. The
figures given are for a noticeboard with 100 items all at the same level and
are the {\em lowest} measured values of {\tt CPU} microseconds:

\begin{small}
\begin {center}
\begin {tabular}{||l|l|r|r|r||}
\hline
{\scriptsize Test} \# & {\scriptsize Test name} & {\scriptsize $\mu$VAX $ 3500$}  &
{\scriptsize DEC Alpha AXP 3000/400 }& {\scriptsize SPARCstation 10 }\\ \hline
0       & Define, save and find noticeboard & 820000    &  16666 & 16666 \\
1       & Scalar assignment             & 1         &    0.1     &   -   \\
2       & Put scalar                    & 36        &    2.0     &   1.6 \\
3       & Get scalar                    & 34        &    1.9     &   1.0 \\
4       & Move 10 longwords directly    & 17        &    0.3     &   0.6 \\
5       & Put 10 longword array         & 44        &    2.2     &   3.3 \\
6       & Get 10 longword array         & 43        &    2.2     &   3.2 \\
7       & Move 1024 longwords directly  & 591       &     16     &   87 \\
8       & Put 1024 longword array       & 670       &     33     &   89 \\
9       & Get 1024 longword array       & 646       &     32     &   89 \\
10      & Find 100th item by position   & 277       &     35     &   23 \\
11      & Find 100th item by name       & 544       &     58     &   31 \\
12      & Enquire name                  & 42        &    1.2     &   2.5 \\
13      & Enquire size                  & 15        &    0.7     &   1.0 \\
\hline
\end {tabular}
\end {center}
\end{small}

These are fairly consistent with the times reported for the {\tt MON} system.
They are also pretty close to those for V2.2 of the noticeboard system and are
better in some cases, presumably partly because of the greater use of inline
code in V2.3.1 onwards. The speed of the {\tt UNIX} machine is particularly
noticeable on file related operations where the speed up is an order of
magnitude greater those operations simply involved with moving data around.

\section {Omissions and Future Plans}

The following are possible improvements and developments:

\begin {itemize}
\item There are no provided exit handlers and if a process crashed in the
middle of an update then a modified count could get left as an odd number which
would cause {\tt NBS\_GET\_VALUE} and {\tt NBS\_GET\_SHAPE} to time out.
\item There are inefficiencies associated with the definition and creation
(not the use) of noticeboards with large numbers of sibling items. Definition
involves linear searches with character comparison through linked lists
and creation involves very deep recursion.
\item It has been proposed that tables (rather like {\tt HDS} structure arrays)
should be supported.
\item It has been proposed that it should be possible to associate ``extra''
information with noticeboard items. This could be used for holding things like
units.
\item The restriction on the names of {\tt UNIX} noticeboards is a
nuisance. Perhaps a global list of names could be maintained, but this
system may be succeptible to access conflicts on a system with intensive
use of {\tt NBS}.
\end {itemize}

\subsection{New Features}
\label{V2.5 New Features}
The only major new feature in V2.5 is that the noticeboard listing
program previously named {\tt ln} has been renamed to {\tt nbtrace} and
is now installed during the NBS installation process.

\subsection {Bug Fixes}

Two problems are fixed in V2.5.5. NBS has hitherto allocated the data for
primitive noticeboard items using exactly the data size supplied in the
item definition. In V2.5.5 this size is rounded up to a multiple of the
size of a double precision number. This ensures that items composed of
types likely to mapped using {\tt NBS\_GET\_POINTER} or
{\tt nbc\_get\_pointer} will be correctly aligned.

The second problem concerns the algorithm used to generate UNIX shared
memory identifiers (4 byte integers) from memory section names. The method
used in previous releases simply copied the values of the first three
characters into an integer. The new algorithm adds the value of each
character code in the name shifted leftwards by a number of bits equal
to its position in the string. This means that all the characters in the
name are significant (only 16 are allowed, which means all the bits in the
last character are significant if used), and that anagrams of names map
on to different key values.

\appendix
\newpage
\section {Description of Individual {\tt NBS} Routines}
\label {Description of Individual NBS Routines}

These descriptions are taken directly from the code. All of the routines are
written in C and some C-specific terms are used.

%------------------------------------------------------------------------------
%------------------------------------------------------------------------------
\begin{small}
% include PROLAT output for NBS.C at this point
\sstroutine{NBS\_TUNE}{Alter the value of a global parameter}{
   \sstdescription{
      Check that the parameter name is legal. \\
      Copy the previous value of the parameter to the supplied variable. \\
      Alter the specified global parameter.

      There are currently six global parameters which can be altered in
      this way:

      [MAX\_DEFN\_SIZE:] an integer that indicates how much memory should
         be allocated for building the noticeboard definition during the
         definition phase. The default is 32768 bytes.

      [TIMEOUT\_COUNT:] an integer that indicates how many times to loop
         before timing out when finding noticeboards or when getting values
         or shapes. The default is 100.

      [TIMEOUT\_INTERVAL:] an integer that indicates how many milliseconds
         to wait between attempts at finding noticeboards or when getting
         values or shapes. The default is 100.

      [WORLD\_WRITE:] a logical (only the lsb is used) that indicates whether
         the world (ie non-owners) can write to noticeboards. The default is
         FALSE.

      [INCREMENT\_MODIFY:] a logical (only the lsb is used) that indicates
         whether the modified count should be incremented when putting the
         values of items. The default is TRUE.

      [CHECK\_MODIFY:] a logical (only the lsb is used) that indicates whether
         the modified count should be checked when getting the values of items.
         The default is TRUE.

      Note that the parameters which can be altered are global to a process and
      not to a noticeboard (they are essentially Fortran COMMON block or C
      static variables). When a parameter is altered its previous value is
      returned and this permits a routine to alter a parameter, use the new
      value and then restore the parameter to its previous value.

      The NBS\_FIND, NBS\_GET and NBS\_PUT routines make rather complicated use of
      these values. For those parameters which are logical flags, they use
      the OR of the default value (or the value set using NBS\_TUNE) and the
      value set using NBS\_TUNE\_NOTICEBOARD. For those parameters which are
      numeric values, they use the value set using NBS\_TUNE.

      When a noticeboard is created, it inherits the default values or the
      values set using NBS\_TUNE and these values may subsequently be altered
      using the \\
      NBS\_TUNE\_NOTICEBOARD routine.
   }
   \sstinvocation{
      (Int) = NBS\_TUNE (NAME,VALUE,OLDVALUE,STATUS)
   }
   \sstarguments{
      \sstsubsection{
         NAME = CHARACTER$*$($*$) (Given)
      }{
         The name of the parameter to alter. See the above list. Can be
         abbreviated so long as it remains unambiguous but this is not
         recommended because new parameters may be supported in the future.
         Case is not significant.
      }
      \sstsubsection{
         VALUE = INTEGER (Given)
      }{
         The value that the parameter is to take.
      }
      \sstsubsection{
         OLDVALUE = INTEGER (Returned)
      }{
         The old value of the parameter.
      }
      \sstsubsection{
         STATUS = INTEGER (Given and returned)
      }{
         The global status. Possible return values are,
\begin{tabbing}
\mantt{NBS\_\_XXXXXXXXXXXXXXXXXXX} \= \kill
\mantt{NBS\_\_BADOPTION} \= Illegal parameter name \\
\end{tabbing}
      }
   }
   \sstdiytopic{
      Prior Requirements
   }{
      None.
   }

}
\sstroutine{
   NBS\_TUNE\_NOTICEBOARD
}{
   Alter the value of a noticeboard-specific global parameter
}{
   \sstdescription{
      Check that the parameter name is legal. \\
      Copy the previous value of the parameter to the supplied variable. \\
      Alter the specified value in the noticeboard.

      There are currently three global parameters which can be altered in
      this way:

      [WORLD\_WRITE:] a logical (only the lsb is used) that indicates whether
         the world (ie non-owners) can write to noticeboards. The default is
         FALSE.

      [INCREMENT\_MODIFY:] a logical (only the lsb is used) that indicates
         whether the modified count should be incremented when putting the
         values of items. The default is TRUE.

      [CHECK\_MODIFY:] a logical (only the lsb is used) that indicates whether
         the modified count should be checked when getting the values of items.
         The default is TRUE.

      Note that the parameters which can be altered are global to a specific
      noticeboard. When a parameter is altered its previous value is returned
      and this permits a routine to alter a parameter, use the new value and
      then restore the parameter to its previous value.

      The NBS\_FIND, NBS\_GET and NBS\_PUT routines make rather complicated use of
      these values. They use the OR of the default value (or the value set
      using NBS\_TUNE) and the value set using NBS\_TUNE\_NOTICEBOARD.

      When a noticeboard is created, it inherits the default values or the
      values set using NBS\_TUNE and these values may subsequently be altered
      using the \\
      NBS\_TUNE\_NOTICEBOARD routine.
   }
   \sstinvocation{
      (Int) = NBS\_TUNE\_NOTICEBOARD (ID,NAME,VALUE,OLDVALUE,STATUS)
   }
   \sstarguments{
      \sstsubsection{
         ID = INTEGER (Given)
      }{
         Identifier of noticeboard or of any item in it,
         whose parameter value is to be altered.
      }
      \sstsubsection{
         NAME = CHARACTER$*$($*$) (Given)
      }{
         The name of the parameter to alter. See the above list. Can be
         abbreviated so long as it remains unambiguous but this is not
         recommended because new parameters may be supported in the future.
         Case is not significant.
      }
      \sstsubsection{
         VALUE = INTEGER (Given)
      }{
         The value that the parameter is to take.
      }
      \sstsubsection{
         OLDVALUE = INTEGER (Returned)
      }{
         The old value of the parameter.
      }
      \sstsubsection{
         STATUS = INTEGER (Given and returned)
      }{
         The global status. Possible return values are,
\begin{tabbing}
\mantt{NBS\_\_XXXXXXXXXXXXXXXXXXX} \= \kill
\mantt{NBS\_\_NILID} \> NIL ID \\
\mantt{NBS\_\_BADOPTION} \> Illegal parameter name \\
\end{tabbing}
      }
   }
}
\sstroutine{
   NBS\_BEGIN\_DEFINITION
}{
   Begin definition of the contents of a noticeboard
   and return a static identifier to the top level of the noticeboard
}{
   \sstdescription{
      Check that we are not currently in the middle of defining a noticeboard.\\
      Allocate the memory area in which the noticeboard definition is built. \\
      From this area, allocate space to describe the new noticeboard and fill
      in fields appropriate to the top level of a noticeboard. \\
      Remember the address of the memory area and note that we are now in the
      middle of defining a noticeboard. \\
      Return the address of the item descriptor to the caller for use in
      subsequent calls.
   }
   \sstinvocation{
      (Int) = NBS\_BEGIN\_DEFINITION (SID,STATUS)
   }
   \sstarguments{
      \sstsubsection{
         SID = INTEGER (Returned)
      }{
         Static identifier of the top-level of the noticeboard. This should
         be used in subsequent calls to the NBS\_DEFINE\_$*$ routines.
      }
      \sstsubsection{
         STATUS = INTEGER (Given and returned)
      }{
         The global status. Possible return values are,
\begin{tabbing}
\mantt{NBS\_\_XXXXXXXXXXXXXXXXXXX} \= \kill
\mantt{NBS\_\_NILID} \> NIL ID \\
\mantt{NBS\_\_BADOPTION} \> Illegal parameter name \\
\mantt{NBS\_\_DEFINING  } \> Already defining a noticeboard \\
\mantt{NBS\_\_INITALLOCFAILED} \> Storage allocation failed \\
\mantt{NBS\_\_NOMOREROOM} \> Storage area is full up \\
\end{tabbing}
      }
   }
}
\sstroutine{
   NBS\_DEFINE\_STRUCTURE
}{
   Define a new entry for a structured item within another structured item
   and return a static identifier to the new item
}{
   \sstdescription{
      Check that we are currently in the middle of defining a noticeboard. \\
      Check that the environment static ID is not NIL and does not pertain to
      a primitive item. \\
      Allocate space to describe the new item and fill in fields appropriate to
      a structured item such that items at this level are in alphabetical order.
      (If an item of this name already exists, create a new item but position it
      before the existing item). \\
      Return the address of the item descriptor to the caller for use in
      subsequent calls.
   }
   \sstinvocation{
      (Int) = NBS\_DEFINE\_STRUCTURE (ENVSID,NAME,TYPE,SID,STATUS)
   }
   \sstarguments{
      \sstsubsection{
         ENVSID = INTEGER (Given)
      }{
         Static identifier of the item in the noticeboard which is the
         parent of the item to be created.
      }
      \sstsubsection{
         NAME = CHARACTER$*$($*$) (Given)
      }{
         Name of the new item.
      }
      \sstsubsection{
         TYPE = CHARACTER$*$($*$) (Given)
      }{
         Type of the new item.
      }
      \sstsubsection{
         SID = INTEGER (Returned)
      }{
         Static identifier of the new structured item. This should be used in
         subsequent calls to the NBS\_DEFINE\_$*$ routines.
      }
      \sstsubsection{
         STATUS = INTEGER (Given and returned)
      }{
         The global status. Possible return values are,
\begin{tabbing}
\mantt{NBS\_\_XXXXXXXXXXXXXXXXXXX} \= \kill
\mantt{NBS\_\_NOTDEFINING} \> Not currently defining a noticeboard \\
\mantt{NBS\_\_NILSID} \> NIL static ID \\
\mantt{NBS\_\_PRIMITIVE} \> Prospective parent is primitive \\
\mantt{NBS\_\_NOMOREROOM} \> Storage area is full up \\
\end{tabbing}
      }
   }
}
\sstroutine{
   NBS\_DEFINE\_PRIMITIVE
}{
   Define a new entry for a primitive item within another structured item
   and return a static identifier to the new item
}{
   \sstdescription{
      Check that we are currently in the middle of defining a noticeboard. \\
      Check that the environment static ID is not NIL and does not pertain to
      a primitive item. \\
      Allocate space to describe the new item and fill in fields appropriate to
      a primitive item such that items at this level are in alphabetical order.
      (If an item of this name already exists, create a new item but position it
      before the existing item). \\
      Return the address of the item descriptor to the caller for use in
      subsequent calls.
   }
   \sstinvocation{
      (Int) = NBS\_DEFINE\_PRIMITIVE (ENVSID,NAME,TYPE,MAXDIMS,MAXBYTES,SID,\\
           STATUS)
   }
   \sstarguments{
      \sstsubsection{
         ENVSID = INTEGER (Given)
      }{
         Static identifier of the item in the noticeboard which is the
         parent of the item to be created.
      }
      \sstsubsection{
         NAME = CHARACTER$*$($*$) (Given)
      }{
         Name of the new item.
      }
      \sstsubsection{
         TYPE = CHARACTER$*$($*$) (Given)
      }{
         Type of the new item.
      }
      \sstsubsection{
         MAXDIMS = INTEGER (Given)
      }{
         Maximum number of dimensions possessed by this item.
      }
      \sstsubsection{
         MAXBYTES = INTEGER (Given)
      }{
         Maximum number of bytes in this item{\tt '}s value
      }
      \sstsubsection{
         SID = INTEGER (Returned)
      }{
         Static identifier of the new structured item. This should be used in
         subsequent calls to the NBS\_DEFINE\_$*$ routines (only NBS\_DEFINE\_SHAPE
         is permitted though).
      }
      \sstsubsection{
         STATUS = INTEGER (Given and returned)
      }{
         The global status. Possible return values are,
\begin{tabbing}
\mantt{NBS\_\_XXXXXXXXXXXXXXXXXXX} \= \kill
\mantt{NBS\_\_NOTDEFINING} \> Not currently defining a noticeboard \\
\mantt{NBS\_\_NILSID} \> NIL static ID \\
\mantt{NBS\_\_PRIMITIVE } \> Prospective parent is primitive \\
\mantt{NBS\_\_NOMOREROOM} \> Storage area is full up \\
\end{tabbing}
      }
   }
}
\sstroutine{
   NBS\_DEFINE\_SHAPE
}{
   Define an initial shape for a primitive item
}{
   \sstdescription{
      Check that we are currently in the middle of defining a noticeboard. \\
      Check that the environment static ID is not NIL and pertains to a
      primitive item. \\
      Check that the requested number of dimensions is not too large. \\
      Copy the shape information to the relevant parts of the item information.
   }
   \sstinvocation{
      (Int) = NBS\_DEFINE\_SHAPE (SID,NDIMS,DIMS,STATUS)
   }
   \sstarguments{
      \sstsubsection{
         SID = INTEGER (Given)
      }{
         Static identifier of the item in the noticeboard
         which is to be given an initial shape.
      }
      \sstsubsection{
         NDIM = INTEGER (Given)
      }{
         Number of dimensions.
      }
      \sstsubsection{
         DIMS = INTEGER($*$) (Given)
      }{
         Dimensions.
      }
      \sstsubsection{
         STATUS = INTEGER (Given and returned)
      }{
         The global status. Possible return values are,
\begin{tabbing}
\mantt{NBS\_\_XXXXXXXXXXXXXXXXXXX} \= \kill
\mantt{NBS\_\_NOTDEFINING} \> Not currently defining a noticeboard \\
\mantt{NBS\_\_NILSID	   } \> NIL static ID \\
\mantt{NBS\_\_PRIMITIVE } \> Prospective parent is primitive \\
\mantt{NBS\_\_TOOMANYDIMS} \> Too many dimensions \\
\end{tabbing}
      }
   }
}
\sstroutine{
   NBS\_END\_DEFINITION
}{
   End the definition of a noticeboard
   and then create the noticeboard, save the definition in a file, or save
   the definition plus data in a file
}{
   \sstdescription{
      Check that we are currently in the middle of defining a noticeboard. \\
      Calculate how large the definition and data parts of the noticeboard
      are. \\
      Relocate all pointers in the definition so that they are relative to zero
      rather than being actual program virtual addresses (actually they are
      made relative to a small positive integer to avoid problems with zero
      pointers). \\
      If the option parameter indicates, write the definition and optionally
      the data to a file (a default file extension of .NBD is applied). \\
      Otherwise, create the noticeboard, copy the definition to it, write the
      calling process{\tt '} id to the global section to denote ownership and mark
      the noticeboard as being valid. \\
      De-allocate the memory area used to amass the noticeboard definition and
      note that we are no longer defining a noticeboard. \\

      If the noticeboard already existed, NBS\_\_SECTIONEXISTED status is
      returned and the calling process becomes its owner.
   }
   \sstinvocation{
      (int) = NBS\_END\_DEFINITION (NAME,OPTION,STATUS)
   }
   \sstarguments{
      \sstsubsection{
         NAME = CHARACTER$*$($*$) (Given)
      }{
         If OPTION is DEFINITION\_SAVE or NOTICEBOARD\_SAVE, the name of the
         file to write the definition or definition plus data to (with a
         default file type of .NBD). Otherwise (OPTION is CREATE\_NOTICEBOARD)
         the name of the noticeboard to create.
      }
      \sstsubsection{
         OPTION = CHARACTER$*$($*$) (Given)
      }{
         Option that governs whether the noticeboard definition or definition
         plus data is saved to a file or whether the noticeboard is simply
         created on the spot without being associated with a file. Can be
         abbreviated so long as it remains unambiguous but this is not
         recommended because new options may be supported in the future. Case
         is not significant. Possible values are:

         [DEFINITION\_SAVE:] Save the definition to a file that
         does not contain space allocated for the data.

         [NOTICEBOARD\_SAVE:] Save the definition to a file that
         does contains space allocated for the data.

         [CREATE\_NOTICEBOARD:] Create the noticeboard immediately
         without associating it with a file. This is assumed
         if an illegal value of OPTION is given.
      }
      \sstsubsection{
         STATUS = INTEGER (Given and returned)
      }{
            The global status. Possible return values are,
\begin{tabbing}
\mantt{NBS\_\_XXXXXXXXXXXXXXXXXXX} \= \kill
\mantt{NBS\_\_NOTDEFINING} \> Not currently defining a noticeboard \\
\mantt{NBS\_\_CANTOPEN  } \> Can't create the definition file \\
\mantt{NBS\_\_CANTWRITE } \> Can't write the definition file \\
\mantt{NBS\_\_SECTIONEXISTED} \> Noticeboard of this name already \\
                          \> existed. \\
 \> \\
\mantt{SS\$\_*} \> System service codes from \mantt{SYS\$CRMPSC} \\
\end{tabbing}
      }
   }
}
\sstroutine{
   NBS\_RESTORE\_DEFINITION
}{
   Restore a noticeboard definition from file
   and create the noticeboard
}{
   \sstdescription{
      Open the file and determine the noticeboard size. \\
      Check that the data part was not saved to the file. \\
      Create the noticeboard. \\
      Read the definition part of the file into the noticeboard. \\
      Write the calling process{\tt '} id to the noticeboard to denote ownership and
      mark the noticeboard as being valid. \\
      Close the file.

      If the noticeboard already existed, NBS\_\_SECTIONEXISTED status is
      returned and the calling process becomes its owner.
   }
   \sstinvocation{
      (Int) = NBS\_RESTORE\_DEFINITION (NAME,SAVE\_NAME,STATUS)
   }
   \sstarguments{
      \sstsubsection{
         NAME = CHARACTER$*$($*$) (Given)
      }{
         The name to give the noticeboard (and thus the name of the
         noticeboard).
      }
      \sstsubsection{
         SAVE\_NAME = CHARACTER$*$($*$) (Given)
      }{
         The name of the file from which to read the definition (with a
         default file type of .NBD)
      }
      \sstsubsection{
         STATUS = INTEGER (Given and returned)
      }{
         The global status. Possible return values are,
\begin{tabbing}
\mantt{NBS\_\_XXXXXXXXXXXXXXXXXXX} \= \kill
\mantt{NBS\_\_CANTOPEN  } \> Can't create the definition file \\
\mantt{NBS\_\_DATASAVED } \> Noticeboard data was saved to the \\
      \> definition file --- cannot restore only defn \\
\mantt{NBS\_\_CANTREAD  } \> Can't read the definition file \\
\mantt{NBS\_\_BADVERSION} \> Wrong version in definition file \\
\mantt{NBS\_\_SECTIONEXISTED} \> Noticeboard of this name already \\
                          \> existed. \\
 \> \\
\mantt{SS\$\_*} \> System service codes from \mantt{SYS\$CRMPSC} \\
\end{tabbing}
      }
   }
   \sstdiytopic{
      Prior Requirements
   }{
      None.
   }

}
\sstroutine{
   NBS\_RESTORE\_NOTICEBOARD
}{
   Restore a noticeboard definition and data from file
   and create the noticeboard
}{
   \sstdescription{
      Open the file and determine the noticeboard size. \\
      Create the noticeboard. \\
      Read the file into the noticeboard. \\
      Write the calling process{\tt '} id to the noticeboard to denote ownership and
      mark the noticeboard as being valid. \\
      Close the file. \\
      If the file only contained the definition and not the noticeboard,
      return a warning status.

      If the noticeboard already existed, NBS\_\_SECTIONEXISTED status is
      returned and the calling process becomes its owner.
   }
   \sstinvocation{
      (Int) = NBS\_RESTORE\_NOTICEBOARD (NAME,SAVE\_NAME,STATUS)
   }
   \sstarguments{
      \sstsubsection{
         NAME = CHARACTER$*$($*$) (Given)
      }{
         The name to give the noticeboard.
      }
      \sstsubsection{
         SAVE\_NAME = CHARACTER$*$($*$) (Given)
      }{
         The name of the file from which to read the definition (with a
         default file type of .NBD)
      }
      \sstsubsection{
         STATUS = INTEGER (Given and returned)
      }{
         The global status. Possible return values are,
\begin{tabbing}
\mantt{NBS\_\_XXXXXXXXXXXXXXXXXXX} \= \kill
\mantt{NBS\_\_CANTOPEN  } \> Can't create the definition file \\
\mantt{NBS\_\_CANTREAD  } \> Can't read the definition file \\
\mantt{NBS\_\_BADVERSION} \> Wrong version in definition file \\
\mantt{NBS\_\_SECTIONEXISTED} \> Noticeboard of this name already \\
                          \> existed. \\
\mantt{NBS\_\_DATANOTSAVED} \> Noticeboard data not saved to the \\
			    \> definition file, so not restored \\
 \> \\
\mantt{SS\$\_*} \> System service codes from \mantt{SYS\$CRMPSC} or \\
 \> \mantt{SYS\$DELTVA} (VMS only).
\end{tabbing}
      }
   }
   \sstdiytopic{
      Prior Requirements
   }{
      None.
   }

}
\sstroutine{
   NBS\_SAVE\_NOTICEBOARD
}{
   Save a noticeboard to its noticeboard definition file
}{
   \sstdescription{
      Check that the ID is not NIL. \\
      Check that the caller owns the noticeboard. \\
      Check that the noticeboard was restored from a file which has room for
      the noticeboard data. \\
      Check that the file is open for write access. \\
      Write the noticeboard data to it.
   }
   \sstinvocation{
      (Int) = NBS\_SAVE\_NOTICEBOARD (ID,STATUS)
   }
   \sstarguments{
      \sstsubsection{
         ID = INTEGER (Given)
      }{
         Identifier of noticeboard or of any item in
      the noticeboard whose data is to be saved.
      }
      \sstsubsection{
         STATUS = INTEGER (Given and returned)
      }{
         The global status. Possible return values are,
\begin{tabbing}
\mantt{NBS\_\_XXXXXXXXXXXXXXXXXXX} \= \kill
\mantt{NBS\_\_NILID	   } \> NIL ID \\
\mantt{NBS\_\_CANTOPEN  } \> Can't open the definition file \\
\mantt{NBS\_\_NOTOWNER  } \> Caller does not own the noticeboard \\
\mantt{NBS\_\_DATANOTRESTORED} \> Noticeboard data was not restored \\
			     \> from the definition file (so can't save it) \\
\end{tabbing}
      }
   }
   \sstdiytopic{
      Prior Requirements
   }{
      None.
   }

}
\sstroutine{
   NBS\_FIND\_NOTICEBOARD
}{
   Find a named noticeboard and return an identifier to it
}{
   \sstdescription{
      Map the noticeboard of the given name. \\
      Allocate an item descriptor and relocate it so that the version number
      can be checked and the definition size determined. \\
      Free that item descriptor and allocate a block of memory big enough for
      the entire definition. \\
      Copy the definition part to this memory and relocate it so that all
      pointers are once again virtual memory addresses. \\
      Get the process{\tt '} id for future checking against the owner{\tt '}s process id. \\
      If this is the owner process and it hasn{\tt '}t already been done, unmap the
      copy mapped earlier.

      Note that it is somewhat wasteful to build a complete copy of the
      noticeboard definition in private memory when in fact only the pointers
      must have private versions. A later version of the software should take
      copies only of the pointers but this requires the pointer space to be
      allocated separately during the definition phase --- in the current
      implementation the pointer space is not contiguous; pointers are mixed up
      with fixed information, shape information and board information.
   }
   \sstinvocation{
      (Int) = NBS\_FIND\_NOTICEBOARD (NAME,ID,STATUS)
   }
   \sstarguments{
      \sstsubsection{
         NAME = CHARACTER$*$($*$) (Given)
      }{
         The name of the noticeboard which is to be found.
      }
      \sstsubsection{
         ID = INTEGER (Returned)
      }{
         Identifier of the top-level of the noticeboard.
      }
      \sstsubsection{
         STATUS = INTEGER (Given and returned)
      }{
         The global status. Possible return values are,
\begin{tabbing}
\mantt{NBS\_\_XXXXXXXXXXXXXXXXXXX} \= \kill
\mantt{NBS\_\_SECTIONNOTFOUND} \> No section called NAME existed \\
\mantt{NBS\_\_TIMEOUT   } \> Timeout awaiting valid noticeboard \\
\mantt{NBS\_\_NOMOREROOM} \> Failed to allocate private memory area \\
\mantt{NBS\_\_BADVERSION} \> Wrong version in noticeboard \\
 \> \\
\mantt{SS\$\_*} \> System service codes from \mantt{SYS\$CRMPSC} or \\
 \> \mantt{SYS\$DELTVA} (VMS only).
\end{tabbing}
      }
   }
}
\sstroutine{
   NBS\_FIND\_ITEM
}{
   Find an item with a specified name contained in a structure
   associated with a specified identifier and return the located item{\tt '}s
   ID
}{
   \sstdescription{
      Check that the environment ID is not NIL and does not pertain to a
      primitive item. \\
      Search for an item of the required name (don{\tt '}t assume that they are in
      any particular order).

      The searching is performed using a binary search. This means that
      it takes roughly the same time to find all the items --- best case
      performance is degraded but worst case performance is substantially
      improved.
   }
   \sstinvocation{
      (Int) = NBS\_FIND\_ITEM (ENVID,NAME,ID,STATUS)
   }
   \sstarguments{
      \sstsubsection{
         ENVID = INTEGER (Given)
      }{
         Identifier of the parent of the item which is to be found.
      }
      \sstsubsection{
         NAME = CHARACTER$*$($*$) (Given)
      }{
         The name of the item to be found.
      }
      \sstsubsection{
         ID = INTEGER (Returned)
      }{
         Identifier of the found item (zero if not found).
      }
      \sstsubsection{
         STATUS = INTEGER (Given and returned)
      }{
         The global status. Possible return values are,
\begin{tabbing}
\mantt{NBS\_\_XXXXXXXXXXXXXXXXXXX} \= \kill
\mantt{NBS\_\_NILID	 } \> NIL ID \\
\mantt{NBS\_\_PRIMITIVE	 } \> Parent is primitive \\
\mantt{NBS\_\_ITEMNOTFOUND} \> No item of this name exists \\
\end{tabbing}
      }
   }
}
\sstroutine{
   NBS\_FIND\_NTH\_ITEM
}{
   Find the Nth item contained in a structure
   associated with a specified identifier and return the located item{\tt '}s ID
}{
   \sstdescription{
      Check that the environment ID is not NIL and does not pertain to a
      primitive item. \\
      Extract the Nth item.
   }
   \sstinvocation{
      (Int) = NBS\_FIND\_NTH\_ITEM (ENVID,POSN,ID,STATUS)
   }
   \sstarguments{
      \sstsubsection{
         ENVID = INTEGER (Given)
      }{
         Identifier of the parent of the item which is to be found.
      }
      \sstsubsection{
         POSN = INTEGER (Given)
      }{
         Number of item to find (first item is item \#1).
      }
      \sstsubsection{
         ID = INTEGER (Returned)
      }{
         Identifier of the found item (zero if not found).
      }
      \sstsubsection{
         STATUS = INTEGER (Given and returned)
      }{
         The global status. Possible return values are,
\begin{tabbing}
\mantt{NBS\_\_XXXXXXXXXXXXXXXXXXX} \= \kill
\mantt{NBS\_\_NILID	 } \> NIL ID \\
\mantt{NBS\_\_PRIMITIVE	 } \> Parent is primitive \\
\mantt{NBS\_\_ITEMNOTFOUND} \> No item of this name exists \\
\end{tabbing}
      }
   }
}
\sstroutine{
   NBS\_LOSE\_NOTICEBOARD
}{
   Unmap a specified noticeboard
}{
   \sstdescription{
      Check that the ID is indeed a top-level one. \\
      Check that no items are currently derived from the top level of the
      noticeboard (not if the {\tt "}FORCE{\tt "} option is specified). \\
      Unmap the noticeboard. \\
      Free the local copy of the noticeboard definition.

      Calls to this routine should match calls to NBS\_FIND\_NOTICEBOARD.
      After calling this routine, all identifiers associated with this
      noticeboard will be invalid. The noticeboard will only be deleted if no
      other process has it mapped.
   }
   \sstinvocation{
      (Int) = NBS\_LOSE\_NOTICEBOARD (ID,OPTION,STATUS)
   }
   \sstarguments{
      \sstsubsection{
         ID = INTEGER (Given)
      }{
         Identifier of the top-level of the noticeboard.
      }
      \sstsubsection{
         OPTION = CHARACTER$*$($*$) (Given)
      }{
         Option that governs whether to check that there are no identifiers
         currently derived from this one. Can be abbreviated so long as it
         remains unambiguous but this is not recommended because new options
         may be supported in the future. Case is not significant. Possible
         values are:

      [FORCE:] Unmap the noticeboard regardless of whether there or not
         there are identifiers derived from this one.

      [CHECK:] Check that no identifiers are derived from this
               one (assumed if invalid option is given).
      }
      \sstsubsection{
         STATUS = INTEGER (Given and returned)
      }{
         The global status. Possible return values are,
\begin{tabbing}
\mantt{NBS\_\_XXXXXXXXXXXXXXXXXXX} \= \kill
\mantt{NBS\_\_NILID	 } \> NIL ID \\
\mantt{NBS\_\_NOTTOPLEVEL } \> ID is not a top-level identifier \\
\mantt{NBS\_\_HASIDS	 } \> Noticeboard has identifiers derived \\
				  \> from it \\
 \> \\
\mantt{SS\$\_*} \> System service codes from \mantt{SYS\$DELTVA} (VMS only). \\
\end{tabbing}
      }
   }
}
\sstroutine{
   NBS\_LOSE\_ITEM
}{
   Declare an intention never again to use a specified item
}{
   \sstdescription{
      Check that the ID is not a top-level one. \\
      Check that at least one item is derived from this item{\tt '}s parent. \\
      Check that no items are currently derived from this item (not if the
      {\tt "}FORCE{\tt "} option is specified). \\
      Decrement the parent{\tt '}s count of derived items.

      Calls to this routine should match calls to NBS\_FIND\_ITEM /
      NBS\_FIND\_NTH\_ITEM. After calling it, this identifier should not
      be used again (even though it is in fact still valid until
      NBS\_LOSE\_NOTICEBOARD is called).
   }
   \sstinvocation{
      (Int) = NBS\_LOSE\_ITEM (ID,OPTION,STATUS)
   }
   \sstarguments{
      \sstsubsection{
         ID = INTEGER (Given)
      }{
         Identifier of the item which is to be lost.
      }
      \sstsubsection{
         OPTION = CHARACTER$*$($*$) (Given)
      }{
         Option that governs whether to check that there are no identifiers
         currently derived from this one. Can be abbreviated so long as it
         remains unambiguous but this is not recommended because new options
         may be supported in the future. Case is not significant. Possible
         values are:

      [FORCE:] Unmap the item regardless of whether there
         are identifiers derived from this one.

      [CHECK:] Check that no identifiers are derived from this
               one (assumed if invalid option is given).
      }
      \sstsubsection{
         STATUS = INTEGER (Given and returned)
      }{
         The global status. Possible return values are,
\begin{tabbing}
\mantt{NBS\_\_XXXXXXXXXXXXXXXXXXX} \= \kill
\mantt{NBS\_\_NILID	 } \> NIL ID \\
\mantt{NBS\_\_NOTTOPLEVEL } \> ID is not a top-level identifier \\
\mantt{NBS\_\_NEVERFOUND  } \> Item was never found (or more items \\
 \> have been lost than were found) \\
\mantt{NBS\_\_HASIDS	 } \> Noticeboard has identifiers derived \\
				     \> from it \\
 \> \\
\mantt{SS\$\_*} \> System service codes from \mantt{SYS\$DELTVA} (VMS only).
\end{tabbing}
      }
   }
}
\sstroutine{
   NBS\_PUT\_VALUE
}{
   Put a byte array into a slice of a primitive item
   associated with a specified identifier
}{
   \sstdescription{
      Check that the ID is not NIL and that it pertains to a primitive item. \\
      Check that the caller owns the noticeboard (or WORLD\_WRITE is TRUE). \\
      Check that the offset into the data is not negative. \\
      Check that the item is large enough to accept all of the supplied
      values. \\
      Increment the item{\tt '}s modified count. \\
      Update the item size (maintain a high-water mark). \\
      Copy the values to the noticeboard. \\
      Increment the item{\tt '}s modified count again. \\
      Increment the noticeboard modified count.

      The item and noticeboard modified counts will not be incremented if
      INCREMENT\_MODIFY is FALSE.

      Note that this routine only alters the specified number of bytes starting
      at the specified offset --- all bytes are initially zero and bytes above
      and below those that are being altered will not be affected. The actual
      size of the item will be adjusted upwards if the new data extends past the
      previous end of the data but the size of the item cannot be decreased by
      this routine. To alter the size of an item, use the NBS\_PUT\_SIZE routine.
   }
   \sstinvocation{
      (Int) = NBS\_PUT\_VALUE (ID,OFFSET,NBYTES,BYTE\_ARRAY,STATUS)
   }
   \sstarguments{
      \sstsubsection{
         ID = INTEGER (Given)
      }{
         Identifier of the item which the value is to be put.
      }
      \sstsubsection{
         OFFSET = INTEGER (Given)
      }{
         Byte offset into item data.
      }
      \sstsubsection{
         NBYTES = INTEGER (Given)
      }{
         Number of bytes to put.
      }
      \sstsubsection{
         BYTES = BYTE($*$) (Given)
      }{
         Bytes to be put.
      }
      \sstsubsection{
         STATUS = INTEGER (Given and returned)
      }{
         The global status. Possible return values are,
\begin{tabbing}
\mantt{NBS\_\_XXXXXXXXXXXXXXXXXXX} \= \kill
\mantt{NBS\_\_NILID	 } \> NIL ID \\
\mantt{NBS\_\_NOTPRIMITIVE} \> Item is not primitive \\
\mantt{NBS\_\_NOTOWNER 	 } \> Caller does not own the noticeboard \\
\mantt{NBS\_\_BADOFFSET	 } \> Negative offset specified \\
\mantt{NBS\_\_TOOMANYBYTES} \> Not room to put all the data \\
\end{tabbing}
      }
   }
   \sstnotes{
      In versions prior to \{V2.4.0\} this routine could be used to write
      character strings on the VAX using the \%REF() mechanism to pass
      the character data. A new routine, NBS\_PUT\_CVALUE, has been provided
      to remove the need for \%REF and make the method portable.
   }
}
\sstroutine{
   NBS\_PUT\_CVALUE
}{
   Put a character string into a slice of a primitive item
   associated with a specified identifier
}{
   \sstdescription{
      Simply extract the string data pointer and length and pass this
      information to NBS\_PUT\_VALUE.
   }
   \sstinvocation{
      (Int) = NBS\_PUT\_CVALUE (ID,OFFSET,STRING,STATUS)
   }
   \sstarguments{
      \sstsubsection{
         ID = INTEGER (Given)
      }{
         Identifier of the item which the value is to be put.
      }
      \sstsubsection{
         OFFSET = INTEGER (Given)
      }{
         Byte offset into item data.
      }
      \sstsubsection{
         STRING = CHARACTER$*$($*$) (Given)
      }{
         The string to be put.
      }
      \sstsubsection{
         STATUS = INTEGER (Given and returned)
      }{
         The global status. Possible return values are,
\begin{tabbing}
\mantt{NBS\_\_XXXXXXXXXXXXXXXXXXX} \= \kill
\mantt{NBS\_\_NILID	 } \> NIL ID \\
\mantt{NBS\_\_NOTPRIMITIVE} \> Item is not primitive \\
\mantt{NBS\_\_NOTOWNER 	 } \> Caller does not own the noticeboard \\
\mantt{NBS\_\_BADOFFSET	 } \> Negative offset specified \\
\mantt{NBS\_\_TOOMANYBYTES} \> Not room to put all the data \\
\end{tabbing}
      }
   }
   \sstnotes{
      Replaces NBS\_PUT\_VALUE when writing character strings to noticeboards.
   }
}
\sstroutine{
   NBS\_PUT\_SHAPE
}{
   Put a new shape to a primitive item
   associated with a specified identifier
}{
   \sstdescription{
      Check that the ID is not NIL and that it pertains to a primitive item. \\
      Check that the caller owns the noticeboard (or WORLD\_WRITE is TRUE). \\
      Check that the item has enough potential dimensions to accept all of the
      supplied dimensions. \\
      Increment the modified count for this item. \\
      Copy the dimensions to the noticeboard. \\
      Increment the modified count once more. \\
      Increment the noticeboard modified count.

      The item and noticeboard modified counts will not be incremented if
      INCREMENT\_MODIFY is FALSE.
   }
   \sstinvocation{
      (Int) = NBS\_PUT\_SHAPE (ID,NDIMS,DIMS,STATUS)
   }
   \sstarguments{
      \sstsubsection{
         ID = INTEGER (Given)
      }{
         Identifier of the item which the shape is to be put.
      }
      \sstsubsection{
         NDIM = INTEGER (Given)
      }{
         Number of dimensions to be put.
      }
      \sstsubsection{
         DIMS = INTEGER($*$) (Given)
      }{
         Dimensions to be put.
      }
      \sstsubsection{
         STATUS = INTEGER (Given and returned)
      }{
         The global status. Possible return values are,
\begin{tabbing}
\mantt{NBS\_\_XXXXXXXXXXXXXXXXXXX} \= \kill
\mantt{NBS\_\_NILID	 } \> NIL ID \\
\mantt{NBS\_\_NOTPRIMITIVE} \> Item is not primitive \\
\mantt{NBS\_\_NOTOWNER 	 } \> Caller does not own the noticeboard \\
\mantt{NBS\_\_TOOMANYDIMS } \> NDIMS is greater than item max dims \\
\end{tabbing}
      }
   }
}
\sstroutine{
   NBS\_PUT\_SIZE
}{
   Put a new size to a primitive item
   associated with a specified identifier
}{
   \sstdescription{
      Check that the ID is not NIL and that it pertains to a primitive item. \\
      Check that the caller owns the noticeboard (or WORLD\_WRITE is TRUE). \\
      Check that the item is large enough to be the proposed size. \\
      Increment the item{\tt '}s modified count. \\
      Update the internal record of the item{\tt '}s size. \\
      Increment the item{\tt '}s modified count again. \\
      Increment the noticeboard modified count.

      The item and noticeboard modified counts will not be incremented if
      INCREMENT\_MODIFY is FALSE.
   }
   \sstinvocation{
      (Int) = NBS\_PUT\_SIZE (ID,NBYTES,STATUS)
   }
   \sstarguments{
      \sstsubsection{
         ID = INTEGER (Given)
      }{
         Identifier of the item which the shape is to be put.
      }
      \sstsubsection{
         NBYTES = INTEGER (Given)
      }{
         new item size in bytes.
      }
      \sstsubsection{
         STATUS = INTEGER (Given and returned)
      }{
         The global status. Possible return values are,
\begin{tabbing}
\mantt{NBS\_\_XXXXXXXXXXXXXXXXXXX} \= \kill
\mantt{NBS\_\_NILID	 } \> NIL ID \\
\mantt{NBS\_\_NOTPRIMITIVE} \> Item is not primitive \\
\mantt{NBS\_\_NOTOWNER 	 } \> Caller does not own the noticeboard \\
\mantt{NBS\_\_TOOMANYBYTES} \> NBYTES is greater than item size \\
\end{tabbing}
      }
   }
}
\sstroutine{
   NBS\_INC\_MODIFIED
}{
   Increment the noticeboard modified count or an item modified count
   depending on whether this is a structured or primitive item
}{
   \sstdescription{
      Check that the ID is not NIL. \\
      Check that the caller owns the noticeboard (or WORLD\_WRITE is TRUE). \\
      If the item is structured increment the noticeboard modified count. \\
      If the item is primitive increment the item{\tt '}s modified count.

      Note that this is a very dangerous routine when called on behalf of
      primitive items. Calls to it {\em must} be paired. Any reader of an item
      will time out if the modified count for the item being read is an odd
      number.
   }
   \sstinvocation{
      (Int) = NBS\_INC\_MODIFIED (ID,STATUS)
   }
   \sstarguments{
      \sstsubsection{
         ID = INTEGER (Given)
      }{
         Identifier of the item whose modified count is be incremented.
      }
      \sstsubsection{
         STATUS = INTEGER (Given and returned)
      }{
         The global status. Possible return values are,
\begin{tabbing}
\mantt{NBS\_\_XXXXXXXXXXXXXXXXXXX} \= \kill
\mantt{NBS\_\_NILID	 } \> NIL ID \\
\mantt{NBS\_\_NOTOWNER 	 } \> Caller does not own the noticeboard \\
\end{tabbing}
      }
   }
}
\sstroutine{
   NBS\_PUT\_TRIGGER
}{
   Specify a routine to be called whenever a primitive item is updated
}{
   \sstdescription{
      Check that the ID is not NIL and that it pertains to a primitive item. \\
      Check that the caller owns the noticeboard (or WORLD\_WRITE is TRUE). \\
      Copy the address of the routine to be called on item update.

      In this context {\tt "}update{\tt "} means any change to the item{\tt '}s shape, data, size
      or modified count. The supplied routine is called with the following
      calling sequence:

      TRIGGER (ID,STATUS)

      where ID is the identifier of the item which has been altered and STATUS
      is as usual. Any bad status returned by the trigger routine will be
      passed back to the caller.
   }
   \sstinvocation{
      (Int) = NBS\_PUT\_TRIGGER (ID,TRIGGER,STATUS)
   }
   \sstarguments{
      \sstsubsection{
         ID = INTEGER (Given)
      }{
         Identifier of the item for which a trigger routine is to be specified.
      }
      \sstsubsection{
         TRIGGER = EXTERNAL (Given)
      }{
         The address of the routine to call whenever the item is updated.
         From FORTRAN, declare it as EXTERNAL. Pass zero (requires \%VAL(0)
         from FORTRAN) to disable the facility).
      }
      \sstsubsection{
         STATUS = INTEGER (Given and returned)
      }{
         The global status. Possible return values are,
\begin{tabbing}
\mantt{NBS\_\_XXXXXXXXXXXXXXXXXXX} \= \kill
\mantt{NBS\_\_NILID	 } \> NIL ID \\
\mantt{NBS\_\_NOTOWNER 	 } \> Caller does not own the noticeboard \\
\end{tabbing}
      }
   }
}
\sstroutine{
   NBS\_GET\_VALUE
}{
   Get a byte array from a slice of a primitive item
   associated with the specified identifier
}{
   \sstdescription{
      \begin {tabbing}
      Check that the ID is not NIL and that it pertains to a primitive item. \\
      Check that the offset into the data is not negative. \\
      Repeat \\
      XXX \= \kill
        \> \{ \\
        \> Read the modified count for this item. \\
        \> Copy as many bytes as there is room for in the user{\tt '}s buffer
           from the noticeboard starting \\
        \> at the specified offset and return the actual number of bytes
           in the item. \\
        \> Read the modified count for this item once more. \\
        \> \} \\
      Until time out or the two modified counts are equal and even \\
        \> (which means that the values were not updated whilst they
                      were being read).
      \end {tabbing}

      If CHECK\_MODIFY is FALSE, the item{\tt '}s modified count is not checked
      at all and a timeout cannot occur.

      If the specified offset is greater than the current size of the item{\tt '}s
      data, no error status will be returned and no data will be copied, but
      the returned number of bytes (ACTBYTES) will be less than the offset
      (OFFSET) and this case should always be checked for.
   }
   \sstinvocation{
      (Int) = NBS\_GET\_VALUE (ID,OFFSET,MAXBYTES,BYTE\_ARRAY,ACTBYTES,STATUS)
   }
   \sstarguments{
      \sstsubsection{
         ID = INTEGER (Given)
      }{
         Identifier of the item from which thew value is to be got.
      }
      \sstsubsection{
         OFFSET = INTEGER (Given)
      }{
         Byte offset into item data.
      }
      \sstsubsection{
         MAXBYTES = INTEGER (Given)
      }{
         Size in bytes of the user{\tt '}s buffer.
      }
      \sstsubsection{
         BYTE\_ARRAY = BYTE($*$) (Returned)
      }{
         User{\tt '}s buffer into which bytes will be got.
      }
      \sstsubsection{
         ACTBYTES = INTEGER (Returned)
      }{
         Actual number of values associated with the item. This may be
         greater than OFFSET $+$ MAXBYTES but no more than MAXBYTES bytes
         will be copied into the user{\tt '}s buffer.
      }
      \sstsubsection{
         STATUS = INTEGER (Given and returned)
      }{
         The global status. Possible return values are,
\begin{tabbing}
\mantt{NBS\_\_XXXXXXXXXXXXXXXXXXX} \= \kill
\mantt{NBS\_\_NILID	 } \> NIL ID \\
\mantt{NBS\_\_NOTPRIMITIVE} \> Item is not primitive \\
\mantt{NBS\_\_BADOFFSET	 } \> Negative offset specified \\
\mantt{NBS\_\_TIMEOUT	 } \> Timeout awaiting valid data \\
\end{tabbing}
      }
   }
}
\sstroutine{
   NBS\_GET\_CVALUE
}{
   Get a character string from a slice of a primitive item
   associated with the specified identifier
}{
   \sstdescription{
      Extracts the string data pointer and string length from the
      FORTRAN argument list, and passes this information to NBS\_GET\_VALUE.
      Thus, the number of bytes read is at most the length the string
      supplied.
   }
   \sstinvocation{
      (Int) = NBS\_GET\_CVALUE (ID,OFFSET,STRING,ACTBYTES,STATUS)
   }
   \sstarguments{
      \sstsubsection{
         ID = INTEGER (Given)
      }{
         Identifier of the item from which thew value is to be got.
      }
      \sstsubsection{
         OFFSET = INTEGER (Given)
      }{
         Byte offset into item data.
      }
      \sstsubsection{
         STRING = CHARACTER$*$($*$) (Returned)
      }{
         Users string buffer into which item bytes will be got.
      }
      \sstsubsection{
         ACTBYTES = INTEGER (Returned)
      }{
         Actual number of values associated with the item. This may be
         greater than OFFSET $+$ LEN(STRING) but no more than LEN(STRING) bytes
         will be copied into the user{\tt '}s buffer.
      }
      \sstsubsection{
         STATUS = INTEGER (Given and returned)
      }{
         The global status. Possible return values are,
\begin{tabbing}
\mantt{NBS\_\_XXXXXXXXXXXXXXXXXXX} \= \kill
\mantt{NBS\_\_NILID	 } \> NIL ID \\
\mantt{NBS\_\_NOTPRIMITIVE} \> Item is not primitive \\
\mantt{NBS\_\_BADOFFSET	 } \> Negative offset specified \\
\mantt{NBS\_\_TIMEOUT	 } \> Timeout awaiting valid data \\
\end{tabbing}
      }
   }
   \sstnotes{
      No C version of this routine is supplied because C strings are
      by convention null terminated. As NBS\_GET\_CVALUE writes new
      string data, its correct functioning in the C case would rely on
      the unwarranted assumption that sufficient space existed in the
      destination string for the data to be written. By forcing the use
      of NBS\_GET\_VALUE in this case, the user must at least state the
      destination length explicitly.
   }
}
\sstroutine{
   NBS\_GET\_SHAPE
}{
   Get the shape of a primitive item
   associated with the specified identifier
}{
   \sstdescription{
      \begin {tabbing}
      Check that the ID is not NIL and that it pertains to a primitive item. \\
      Repeat \\
      XXX \= \kill
        \> \{ \\
             \> Read the modified count for this item. \\
             \> Copy as many dimensions as there is room for in the user{\tt '}s
                buffer from the noticeboard and \\
             \> return the actual number of dimensions in the item. \\
             \> Read the modified count for this item once more. \\
        \> \} \\
             \> Until time out or the two modified counts are equal and even \\
             \> (which means that the values were not updated whilst they
                      were being read).
      \end {tabbing}

      If CHECK\_MODIFY is FALSE, the item{\tt '}s modified count is not checked
      at all and a timeout cannot occur.

      Note also that the MAXDIMS parameter to this routine is a MODIFIED
      parameter.
   }
   \sstinvocation{
      (Int) = NBS\_GET\_SHAPE (ID,MAXDIMS,DIMS,ACTDIMS,STATUS)
   }
   \sstarguments{
      \sstsubsection{
         ID = INTEGER (Given)
      }{
         Identifier of the item from which the shape is to be got.
      }
      \sstsubsection{
         MAXDIMS = INTEGER (Given and returned)
      }{
         On entry, size of the DIMS array. On exit the maximum number
         of dimensions that this item can have.
      }
      \sstsubsection{
         DIMS = INTEGER($*$) (Returned)
      }{
         Returned dimensions.
      }
      \sstsubsection{
         ACTDIMS = INTEGER (Returned)
      }{
         Actual number of dimensions associated with the item. This may
         be greater than MAXDIMS but no more than MAXDIMS values will be
         copied into the DIMS array.
      }
      \sstsubsection{
         MAXBYTES = INTEGER (Given)
      }{
         Size in bytes of the user{\tt '}s buffer.
      }
      \sstsubsection{
         BYTE\_ARRAY = BYTE($*$) (Returned)
      }{
         User{\tt '}s buffer into which bytes will be got.
      }
      \sstsubsection{
         ACTBYTES = INTEGER (Returned)
      }{
         Actual number of values associated with the item. This may be
         greater than OFFSET $+$ MAXBYTES but no more than MAXBYTES bytes
         will be copied into the user{\tt '}s buffer.
      }
      \sstsubsection{
         STATUS = INTEGER (Given and returned)
      }{
         The global status. Possible return values are,
\begin{tabbing}
\mantt{NBS\_\_XXXXXXXXXXXXXXXXXXX} \= \kill
\mantt{NBS\_\_NILID	 } \> NIL ID \\
\mantt{NBS\_\_NOTPRIMITIVE} \> Item is not primitive \\
\mantt{NBS\_\_TIMEOUT	 } \> Timeout awaiting valid data \\
\end{tabbing}
      }
   }
}
\sstroutine{
   NBS\_GET\_MODIFIED
}{
   Get the noticeboard modified count or an item modified count
   depending on whether this is a structured or primitive item
}{
   \sstdescription{
      Check that the ID is not NIL. \\
      If the item is structured get the noticeboard modified count. \\
      If the item is primitive get the item{\tt '}s modified count.

      For structured items, this value is incremented each time an item in the
      noticeboard is updated.

      For primitive items, if this value is even then the associated values
      are not currently being updated. If it is odd then they are currently
      being updated. The total number of updates to this item is half the
      value of the modified count.

      An {\tt "}update{\tt "} is an update of an item{\tt '}s value, shape or size.

      Note that when item data is accessed directly via pointer then the
      modified count is not updated unless this is done explicitly using the
      NBS\_INC\_MODIFIED routine.
   }
   \sstinvocation{
      (Int) = NBS\_GET\_MODIFIED (ID,MODIFIED,STATUS)
   }
   \sstarguments{
      \sstsubsection{
         ID = INTEGER (Given)
      }{
         Identifier of the item from which the modified count is to be got.
      }
      \sstsubsection{
         MODIFIED = INTEGER (Returned)
      }{
         The current value of the noticeboard or item{\tt '}s modified count.
      }
      \sstsubsection{
         STATUS = INTEGER (Given and returned)
      }{
         The global status. Possible return values are,
\begin{tabbing}
\mantt{NBS\_\_XXXXXXXXXXXXXXXXXXX} \= \kill
\mantt{NBS\_\_NILID	 } \> NIL ID \\
\end{tabbing}
      }
   }
}
\sstroutine{
   NBS\_GET\_MODIFIED\_POINTER
}{
   Get a pointer to the noticeboard modified count or an item modified count
   depending on whether this is a structured or primitive item
}{
   \sstdescription{
      Check that the ID is not NIL. \\
      If the item is structured get the noticeboard modified count. \\
      If the item is primitive get the item{\tt '}s modified count. \\
      Return a pointer to the appropriate modified count.

      For structured items, this value is incremented each time an item in the
      noticeboard is updated.

      For primitive items, if this value is even then the associated values
      are not currently being updated. If it is odd then they are currently
      being updated. The total number of updates to this item is half the
      value of the modified count.

      An {\tt "}update{\tt "} is an update of an item{\tt '}s value, shape or size.

      Note that when item data is accessed directly via pointer then the
      modified count is not updated unless this is done explicitly using the
      NBS\_INC\_MODIFIED routine.
   }
   \sstinvocation{
      (Int) = NBS\_GET\_MODIFIED\_POINTER (ID,POINTER,STATUS)
   }
   \sstarguments{
      \sstsubsection{
         ID = INTEGER (Given)
      }{
         Identifier of the item for which the pointer to its modified count
         is to be got.
      }
      \sstsubsection{
         POINTER = INTEGER (Returned)
      }{
         The address of the item{\tt '}s modified count.
      }
      \sstsubsection{
         STATUS = INTEGER (Given and returned)
      }{
         The global status. Possible return values are,
\begin{tabbing}
\mantt{NBS\_\_XXXXXXXXXXXXXXXXXXX} \= \kill
\mantt{NBS\_\_NILID	 } \> NIL ID \\
\end{tabbing}
      }
   }
}
\sstroutine{
   NBS\_GET\_UPDATED
}{
   Determine whether a primitive item or the noticeboard has been updated
   since the noticeboard was found or this routine was last called
}{
   \sstdescription{
      Check that the ID is not NIL. \\
      If the item is structured get the noticeboard modified count. \\
      If the item is primitive get the item{\tt '}s modified count. \\
      Return TRUE (1) if the modified count is greater than the count the last
      time that this routine was called. \\
      Return FALSE (0) otherwise. \\
      Remember the modified count for next time.

      For structured items, always use the same ID, since the remembered count
      is associated with the ID and not with the noticeboard.
   }
   \sstinvocation{
      (Int) = NBS\_GET\_UPDATED (ID,UPDATED,STATUS)
   }
   \sstarguments{
      \sstsubsection{
         ID = INTEGER (Given)
      }{
         Identifier of the item for which to determine whether it has been
         updated since the last call on its behalf.
      }
      \sstsubsection{
         UPDATED = INTEGER (Returned)
      }{
         Whether updated (1) or not (0).
      }
      \sstsubsection{
         STATUS = INTEGER (Given and returned)
      }{
         The global status. Possible return values are,
\begin{tabbing}
\mantt{NBS\_\_XXXXXXXXXXXXXXXXXXX} \= \kill
\mantt{NBS\_\_NILID	 } \> NIL ID \\
\end{tabbing}
      }
   }
}
\sstroutine{
   NBS\_GET\_POINTER
}{
   Return a pointer to the first byte of the data of a primitive item
   associated with the specified identifier
}{
   \sstdescription{
      Check that the ID is not NIL and that it pertains to a primitive item. \\
      Return the address of the start of the item{\tt '}s noticeboard data.
   }
   \sstinvocation{
      (Int) = NBS\_GET\_POINTER (ID,POINTER,STATUS)
   }
   \sstarguments{
      \sstsubsection{
         ID = INTEGER (Given)
      }{
         Identifier of the item for which the pointer to its noticeboard
         data is to be got.
      }
      \sstsubsection{
         POINTER = INTEGER (Returned)
      }{
         The address of the first byte of this item{\tt '}s noticeboard data.
      }
      \sstsubsection{
         STATUS = INTEGER (Given and returned)
      }{
         The global status. Possible return values are,
\begin{tabbing}
\mantt{NBS\_\_XXXXXXXXXXXXXXXXXXX} \= \kill
\mantt{NBS\_\_NILID	 } \> NIL ID \\
\mantt{NBS\_\_NOTPRIMITIVE} \> Item is not primitive \\
\end{tabbing}
      }
   }
}
\sstroutine{
   NBS\_GET\_NAME
}{
   Get the name of an item associated with the specified identifier
}{
   \sstdescription{
      Check that the ID is not NIL. \\
      Return the item{\tt '}s name.
   }
   \sstinvocation{
      (Int) = NBS\_GET\_NAME (ID,NAME,STATUS)
   }
   \sstarguments{
      \sstsubsection{
         ID = INTEGER (Given)
      }{
         Identifier of the item for whose name is to be got.
      }
      \sstsubsection{
         NAME = CHARACTER$*$($*$) (Returned)
      }{
         The item{\tt '}s name.
      }
      \sstsubsection{
         STATUS = INTEGER (Given and returned)
      }{
         The global status. Possible return values are,
\begin{tabbing}
\mantt{NBS\_\_XXXXXXXXXXXXXXXXXXX} \= \kill
\mantt{NBS\_\_NILID	 } \> NIL ID \\
\end{tabbing}
      }
   }
}
\sstroutine{
   NBS\_GET\_TYPE
}{
   Get the type of an item associated with the specified identifier
}{
   \sstdescription{
      Check that the ID is not NIL. \\
      Return the item{\tt '}s type.
   }
   \sstinvocation{
      (Int) = NBS\_GET\_TYPE (ID,TYPE,STATUS)
   }
   \sstarguments{
      \sstsubsection{
         ID = INTEGER (Given)
      }{
         Identifier of the item for whose type is to be got.
      }
      \sstsubsection{
         TYPE = CHARACTER$*$($*$) (Returned)
      }{
         The item{\tt '}s type.
      }
      \sstsubsection{
         STATUS = INTEGER (Given and returned)
      }{
         The global status. Possible return values are,
\begin{tabbing}
\mantt{NBS\_\_XXXXXXXXXXXXXXXXXXX} \= \kill
\mantt{NBS\_\_NILID	 } \> NIL ID \\
\end{tabbing}
      }
   }
}
\sstroutine{
   NBS\_GET\_SIZE
}{
   Get the maximum and actual sizes of a primitive item
   associated with the specified identifier
}{
   \sstdescription{
      Check that the ID is not NIL and that it pertains to a primitive item. \\
      Return the maximum and actual sizes of the item{\tt '}s noticeboard data.
   }
   \sstinvocation{
      (Int) = NBS\_GET\_SIZE (ID,MAXBYTES,ACTBYTES,STATUS)
   }
   \sstarguments{
      \sstsubsection{
         ID = INTEGER (Given)
      }{
         Identifier of the item for whose size is to be got.
      }
      \sstsubsection{
         MAXBYTES = INTEGER (Returned)
      }{
         Maximum size in bytes.
      }
      \sstsubsection{
         ACTBYTES = INTEGER (Returned)
      }{
         Actual size in bytes.
      }
      \sstsubsection{
         STATUS = INTEGER (Given and returned)
      }{
         The global status. Possible return values are,
\begin{tabbing}
\mantt{NBS\_\_XXXXXXXXXXXXXXXXXXX} \= \kill
\mantt{NBS\_\_NILID	 } \> NIL ID \\
\mantt{NBS\_\_NOPRIMITIVE } \> Item is not primitive \\
\end{tabbing}
      }
   }
   \sstdiytopic{
      Prior Requirements
   }{
      NBS\_FIND\_NOTICEBOARD must have been called.
   }

}
\sstroutine{
   NBS\_GET\_PRIMITIVE
}{
   Determine whether or not an item is primitive
}{
   \sstdescription{
      Check that the ID is not NIL. \\
      Return FALSE (0) if the item is a structure and TRUE (1) if it is
      primitive.
   }
   \sstinvocation{
      (Int) = NBS\_GET\_PRIMITIVE (ID,PRIMITIVE,STATUS)
   }
   \sstarguments{
      \sstsubsection{
         ID = INTEGER (Given)
      }{
         Identifier of the item concerned.
      }
      \sstsubsection{
         PRIMITIVE = INTEGER (Returned)
      }{
         Whether primitive (1) or structured (0).
      }
      \sstsubsection{
         STATUS = INTEGER (Given and returned)
      }{
         The global status. Possible return values are,
\begin{tabbing}
\mantt{NBS\_\_XXXXXXXXXXXXXXXXXXX} \= \kill
\mantt{NBS\_\_NILID	 } \> NIL ID \\
\end{tabbing}
      }
   }
}
\sstroutine{
   NBS\_GET\_PARENT
}{
   Get the identifier of an item{\tt '}s parent structure
}{
   \sstdescription{
      Check that the ID is not NIL. \\
      Return the identifier of the item{\tt '}s parent.

      If the item has no parent (ie, if it pertains to a noticeboard), then a
      zero ID will be returned.
   }
   \sstinvocation{
      (Int) = NBS\_GET\_PARENT (ID,PARENT,STATUS)
   }
   \sstarguments{
      \sstsubsection{
         ID = INTEGER (Given)
      }{
         Identifier of the item whose parent is to be got.
      }
      \sstsubsection{
         PARENT = INTEGER (Returned)
      }{
         Identifier of item{\tt '}s parent. If the item has no parent then
         a NIL ID will be returned.
      }
      \sstsubsection{
         STATUS = INTEGER (Given and returned)
      }{
         The global status. Possible return values are,
\begin{tabbing}
\mantt{NBS\_\_XXXXXXXXXXXXXXXXXXX} \= \kill
\mantt{NBS\_\_NILID	 } \> NIL ID \\
\end{tabbing}
      }
   }
}
\sstroutine{
   NBS\_GET\_CHILDREN
}{
   Get the number of children of a structured item
}{
   \sstdescription{
      Check that the ID is not NIL and does not pertain to a primitive item. \\
      Return the number of children that it has.
   }
   \sstinvocation{
      (Int) = NBS\_GET\_CHILDREN (ID,CHILDREN,STATUS)
   }
   \sstarguments{
      \sstsubsection{
         ID = INTEGER (Given)
      }{
         Identifier of the item whose number of children is to be got.
      }
      \sstsubsection{
         CHILDREN = INTEGER (Returned)
      }{
         Number of children.
      }
      \sstsubsection{
         STATUS = INTEGER (Given and returned)
      }{
         The global status. Possible return values are,
\begin{tabbing}
\mantt{NBS\_\_XXXXXXXXXXXXXXXXXXX} \= \kill
\mantt{NBS\_\_NILID	 } \> NIL ID \\
\mantt{NBS\_\_PRIMITIVE	 } \> Parent is primitive \\
\end{tabbing}
      }
   }
}
\sstroutine{
   NBS\_GET\_INFO
}{
   Get general non-character information on a given noticeboard
}{
   \sstdescription{
      Check that the item name is legal. \\
      Copy the current value of the item from the relevant noticeboard.

      There are currently eight items which can be returned. Most are from
      a common noticeboard area but GLOBAL\_BASE is an address within the
      address space of the caller. Unless otherwise stated, all are integers.

      CHAN  =$>$ Channel to open noticeboard file (zero if not open)
      DEFN\_SIZE  =$>$ Size of definition part of noticeboard
      FILE\_SIZE  =$>$ Size of noticeboard definition file
      GLOBAL\_BASE =$>$ Address of noticeboard start
      MODIFIED  =$>$ Total number of times values have been modified
      PID  =$>$ PID of owner of this noticeboard
      SECTION\_SIZE =$>$ Total size of noticeboard including data
      VERSION  =$>$ Software version creating file / noticeboard
   }
   \sstinvocation{
      (Int) = NBS\_GET\_INFO (ID,NAME,VALUE,STATUS)
   }
   \sstarguments{
      \sstsubsection{
         ID = INTEGER (Given)
      }{
         Identifier of noticeboard or of any item in the relevant noticeboard.
      }
      \sstsubsection{
         NAME = CHARACTER$*$($*$) (Given)
      }{
         The name of the item to obtain. See the above list. Can be abbreviated
         so long as it remains unambiguous but this is not recommended because
         new items may be supported in the future. Case is not significant.
      }
      \sstsubsection{
         VALUE = Depends on NAME (Returned)
      }{
         The item{\tt '}s value. Declared as pointer to integer, but may be
         coerced to pointer to real.
      }
      \sstsubsection{
         STATUS = INTEGER (Given and returned)
      }{
         The global status. Possible return values are,
\begin{tabbing}
\mantt{NBS\_\_XXXXXXXXXXXXXXXXXXX} \= \kill
\mantt{NBS\_\_NILID	 } \> NIL ID \\
\mantt{NBS\_\_BADOPTION	 } \> Illegal item name \\
\end{tabbing}
      }
   }
   \sstdiytopic{
      Prior Requirements
   }{
      NBS\_FIND\_NOTICEBOARD must have been called.
   }

}
\sstroutine{
   NBS\_GET\_CINFO
}{
   Get general character information on a given noticeboard
}{
   \sstdescription{
      Check that the item name is legal. \\
      Copy the current value of the item from the relevant noticeboard.

      There is currently only one character item which can be returned.

      SAVE\_NAME  =$>$ Name of noticeboard file (character)
   }
   \sstinvocation{
      (Int) = NBS\_GET\_CINFO (ID,NAME,VALUE,STATUS)
   }
   \sstarguments{
      \sstsubsection{
         ID = INTEGER (Given)
      }{
         Identifier of noticeboard or of any item in the relevant noticeboard.
      }
      \sstsubsection{
         NAME = CHARACTER$*$($*$) (Given)
      }{
         The name of the item to obtain. See the above list. Can be abbreviated
         so long as it remains unambiguous but this is not recommended because
         new items may be supported in the future. Case is not significant.
      }
      \sstsubsection{
         VALUE = CHARACTER$*$($*$) (Returned)
      }{
         The item{\tt '}s value.
      }
      \sstsubsection{
         STATUS = INTEGER (Given and returned)
      }{
         The global status. Possible return values are,
\begin{tabbing}
\mantt{NBS\_\_XXXXXXXXXXXXXXXXXXX} \= \kill
\mantt{NBS\_\_NILID	 } \> NIL ID \\
\mantt{NBS\_\_BADOPTION	 } \> Illegal item name \\
\end{tabbing}
      }
   }
   \sstdiytopic{
      Prior Requirements
   }{
      NBS\_FIND\_NOTICEBOARD must have been called.
   }

}
\end{small}


%------------------------------------------------------------------------------

\newpage
\section {{\tt NBS} Error Codes}

This is edited from the source file that is processed by the {\tt MESSAGE}
utility and the (slightly modified) {\tt ADAM} {\tt ERRGEN} utility to generate
the C ({\tt NBS\_DIR:\-NBS\_ERR.H}) and Fortran ({\tt
NBS\_DIR:\-NBS\_ERR.INC}) versions of the error code {\tt INCLUDE} files.

\begin {center}
\begin {tabular}{||l|l|p{80mm}||}
\hline
Severity        & Name                  & Description                   \\
\hline
Information     & \tt SECTIONEXISTED    & Noticeboard already existed \\
\hline
Warning         & \tt TOOMANYDIMS       & More dimensions than maximum allowed \\
                & \tt TOOMANYBYTES      & More bytes than maximum allowed \\
                & \tt BADOFFSET         & Offset is less than zero \\
                & \tt BADOPTION         & Illegal parameter / item name \\
                & \tt DATANOTSAVED      & Data part of noticeboard not saved --- cannot restore it \\
\hline
Error           & \tt DEFINING          & Currently defining noticeboard contents \\
                & \tt NOTDEFINING       & Not currently defining noticeboard contents \\
                & \tt NILSID            & {\tt NIL} static {\tt ID} \\
                & \tt NILID             & {\tt NIL} item {\tt ID} \\
                & \tt PRIMITIVE         & Item is primitive \\
                & \tt NOTPRIMITIVE      & Item is not primitive \\
                & \tt ITEMNOTFOUND      & Item does not exist \\
                & \tt SECTIONNOTFOUND   & Noticeboard does not exist \\
                & \tt CANTOPEN          & Can't open noticeboard definition file \\
                & \tt CANTWRITE         & Can't write noticeboard definition file \\
                & \tt CANTREAD          & Can't read noticeboard definition file \\
                & \tt NOTOWNER          & Non-owner attempted to alter noticeboard \\
                & \tt TIMEOUT           & Time out finding noticeboard or getting item value or shape\\
                & \tt DATASAVED         & Data part of noticeboard saved --- cannot restore definition \\
                & \tt DATANOTRESTORED   & Data was not restored from noticeboard file --- cannot save it \\
		& \tt HASIDS		& Item / noticeboard has items derived from it --- cannot lose it \\
		& \tt NOTTOPLEVEL	& Item is not top-level (ie not noticeboard) --- cannot lose it \\
		& \tt TOPLEVEL		& Item is top-level (ie noticeboard) --- cannot lose it \\
		& \tt NEVERFOUND	& Parent has no items derived from it --- cannot lose it \\
\hline
Fatal           & \tt INITALLOCFAILED   & Couldn't initialise storage allocator \\
                & \tt NOMOREROOM        & Couldn't get memory --- increase {\tt MAX\_\-DEFN\_\-SIZE} if when defining \\
                & \tt BADVERSION        & Noticeboard or definition file had wrong version \\
                & \tt IMPOSSIBLE        & Something impossible happened --- system error \\
\hline
\end {tabular}
\end {center}

\newpage
\section {Demonstration Programs}
\label {Demonstration Programs}

Several demonstration programs are shipped with the system and are described
here (it is assumed that the symbol ``{\tt program}'' has been set up to run the
program {\tt program}). All of these programs are written in C.

\subsection {{\tt NBTRACE} --- Trace Noticeboard Contents}
\label {NBTRACE --- Trace Noticeboard Contents}

{\tt NBTRACE} lists the contents of a noticeboard definition file or of an active
noticeboard.

The name of the definition file (assumed extension {\tt .NBD}) or of a
noticeboard item can be given as a command line parameter and will be prompted
for if it is not given. The program first attempts to find the noticeboard and
if that fails (because the noticeboard doesn't exist) it attempts to restore
the noticeboard from the definition file. Then it locates the specified item
and lists it and all items below it. VMS-style wild cards can be used. Thus

\begin{verbatim}
    $ nbtrace gct.ifl.enq*
\end{verbatim}

might result in the following

\begin{verbatim}
    Noticeboard data not restored because it was not saved
    STRUC    IFL (140)
       _CHAR    ENQ_DEV_DESCR   (0/132/0)
       _CHAR    ENQ_DEV_TYPE    (0/132/0)
       _CHAR    ENQ_VER_DATE    (0/132/0)
       _CHAR    ENQ_VER_NUM     (0/132/0)
\end{verbatim}

and

\begin{verbatim}
    $ nbtrace gct.fits.*.x
\end{verbatim}

is a useful trick to suppress listing of lower level items (on the assumption
that none of them are called {\tt X}).

\begin{verbatim}
    Noticeboard data not restored because it was not saved
    STRUC   FITS (15)
       DRT_STRUCT   ALT_OBS (3)
       DRT_STRUCT   COMMENT (2)
       DRT_STRUCT   HAEND (3)
       DRT_STRUCT   HASTART (3)
       DRT_STRUCT   INSTRUME (3)
       DRT_STRUCT   LAT_OBS (3)
       DRT_STRUCT   LONG_OBS (3)
       DRT_STRUCT   OBJECT (3)
       DRT_STRUCT   ORIGIN (3)
       DRT_STRUCT   RUN (3)
       DRT_STRUCT   SPEED (3)
       DRT_STRUCT   UTDATE (3)
       DRT_STRUCT   WINDOW (3)
       DRT_STRUCT   ZDEND (3)
       DRT_STRUCT   ZDSTART (3)
\end{verbatim}

When no item name is given, general information about the noticeboard's size
and owner is given, as in

\begin {verbatim}
    $ nbtrace words
    Software version   = 5 (5)
    Size of section    = 1080 (438)
    Size of definition = 1080 (438)
    Noticeboard owner  = 2959 (b8f)
    Modified count     = 0 (0)

    NOTICEBOARD WORDS (9)
       WORD     BROWN   (0/0/0)
       WORD     DOG     (0/0/0)
       WORD     FOX     (0/0/0)
       WORD     JUMPS   (0/0/0)
       WORD     LAZY    (0/0/0)
       WORD     OVER    (0/0/0)
       WORD     QUICK   (0/0/0)
       WORD     THE     (0/0/0)
       WORD     THE     (0/0/0)
\end{verbatim}

\subsection {{\tt TIME} --- Time Noticeboard Operations}

{\tt TIME} times various noticeboard operations. It produced the timings listed
in Section~\ref {Routine Timings}.

The user chooses how many items the noticeboard should contain, which item
should be used for timing {\tt NBS\_FIND\_ITEM}, how many iterations to perform
and the values of the {\tt INC\-RE\-MENT\_\-MOD\-IFY} and {\tt
CHECK\_\-MOD\-IFY} flags. The program produces a report in {\tt TIME.LIS}. For
example:

\begin {verbatim}
    Noticeboard system timing program at Thu Mar 31 17:49:00 1988
    -------------------------------------------------------------

    Number of iterations   = 10000
    Number of items        = 100
    Item used for searches = 100
    Increment modify flag  = 1
    Check modify flag      = 1

    Test 0, define, save and find noticeboard
            cpu microseconds = 2330000

    Test 1, scalar assignment
            cpu microseconds per iteration = 3

    Test 2, put scalar
            cpu microseconds per iteration = 124

    Test 3, get scalar
            cpu microseconds per iteration = 90
    .
    .
\end{verbatim}

\subsection {{\tt EXERCISE} --- Exercise Noticeboard Routines}

{\tt EXERCISE} calls all {\tt NBS} routines and triggers all reasonable
errors. It produces a report in {\tt EXERCISE.LIS} in which any unexpected
results have an asterisk in the first column. The total error count is reported
at the bottom and this should always be zero. If it is non-zero there could
be a resource-related problem (or even a bug in the {\tt NBS} routines).
For example:

\begin {verbatim}
    Noticeboard system exercise program at Thu Mar 31 10:28:55 1988
    ---------------------------------------------------------------

    NBS_TUNE
    --------
     nbs_tune max_defn_size: ok
     nbs_tune of doesnt_exist: illegal parameter / item name

    NBS_DEFINE_* errors
    -------------------
     nbs_define_structure: not currently defining noticeboard contents
     .
     .

    NBS_PUT
    -------
     nbs_get_primitive: NIL item ID
     nbs_get_parent: NIL item ID
     nbs_get_children: NIL item ID
     nbs_get_children: item is primitive
     nbs_get_info: NIL item ID
     nbs_get_info: illegal parameter / item name

    Error count = 0
    ----------------
\end{verbatim}

\subsection {{\tt WORDS} --- Generate Tree-structured Noticeboard}

{\tt WORDS} generates a tree-structured noticeboard with one item for each
word found in a file provided by the user. The user also specifies a ``cluster
size'', which is a measure of how deep the tree is (the tree has about
$\log_{2} ($cluster size$)$ levels). Once the noticeboard has been created
the user can type in words and is told whether they appeared in the file.
Exit with \verb|^Z|. The example given for {\tt NBTRACE} is of a noticeboard
created using {\tt WORDS}.

\end {document}
