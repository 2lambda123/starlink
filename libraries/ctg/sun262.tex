\documentclass[twoside,11pt]{article}

% ? Specify used packages
% \usepackage{graphicx}        %  Use this one for final production.
% \usepackage[draft]{graphicx} %  Use this one for drafting.
% ? End of specify used packages

\pagestyle{myheadings}

% -----------------------------------------------------------------------------
% ? Document identification
% Fixed part
\newcommand{\stardoccategory}  {Starlink User Note}
\newcommand{\stardocinitials}  {SUN}
\newcommand{\stardocsource}    {sun\stardocnumber}
\newcommand{\stardoccopyright} 
{Copyright \copyright\ 2009 Science and Technology Facilities Council.}

% Variable part - replace [xxx] as appropriate.
\newcommand{\stardocnumber}    {262.1}
\newcommand{\stardocauthors}   {D.S.\ Berry \\ Malcolm J. Currie}
\newcommand{\stardocdate}      {2009 August 11}
\newcommand{\stardoctitle}     {CTG --- Accessing Groups of catalogues}
\newcommand{\stardocversion}   {Version 3.0}
\newcommand{\stardocmanual}    {Programmer's Manual}
\newcommand{\stardocabstract}  {
This document describes the routines provided within the CTG subroutine 
library for accessing groups of catalogues.
}
% ? End of document identification
% -----------------------------------------------------------------------------

% +
%  Name:
%     sun.tex
%
%  Purpose:
%     Template for Starlink User Note (SUN) documents.
%     Refer to SUN/199
%
%  Authors:
%     AJC: A.J.Chipperfield (Starlink, RAL)
%     BLY: M.J.Bly (Starlink, RAL)
%     PWD: Peter W. Draper (Starlink, Durham University)
%
%  History:
%     17-JAN-1996 (AJC):
%        Original with hypertext macros, based on MDL plain originals.
%     16-JUN-1997 (BLY):
%        Adapted for LaTeX2e.
%        Added picture commands.
%     13-AUG-1998 (PWD):
%        Converted for use with LaTeX2HTML version 98.2 and
%        Star2HTML version 1.3.
%      1-FEB-2000 (AJC):
%        Add Copyright statement in LaTeX
%     {Add further history here}
%
% -

\newcommand{\stardocname}{\stardocinitials /\stardocnumber}
\markboth{\stardocname}{\stardocname}
\setlength{\textwidth}{160mm}
\setlength{\textheight}{230mm}
\setlength{\topmargin}{-2mm}
\setlength{\oddsidemargin}{0mm}
\setlength{\evensidemargin}{0mm}
\setlength{\parindent}{0mm}
\setlength{\parskip}{\medskipamount}
\setlength{\unitlength}{1mm}

% -----------------------------------------------------------------------------
%  Hypertext definitions.
%  ======================
%  These are used by the LaTeX2HTML translator in conjunction with star2html.

%  Comment.sty: version 2.0, 19 June 1992
%  Selectively in/exclude pieces of text.
%
%  Author
%    Victor Eijkhout                                      <eijkhout@cs.utk.edu>
%    Department of Computer Science
%    University Tennessee at Knoxville
%    104 Ayres Hall
%    Knoxville, TN 37996
%    USA

%  Do not remove the %begin{latexonly} and %end{latexonly} lines (used by 
%  LaTeX2HTML to signify text it shouldn't process).
%begin{latexonly}
\makeatletter
\def\makeinnocent#1{\catcode`#1=12 }
\def\csarg#1#2{\expandafter#1\csname#2\endcsname}

\def\ThrowAwayComment#1{\begingroup
    \def\CurrentComment{#1}%
    \let\do\makeinnocent \dospecials
    \makeinnocent\^^L% and whatever other special cases
    \endlinechar`\^^M \catcode`\^^M=12 \xComment}
{\catcode`\^^M=12 \endlinechar=-1 %
 \gdef\xComment#1^^M{\def\test{#1}
      \csarg\ifx{PlainEnd\CurrentComment Test}\test
          \let\html@next\endgroup
      \else \csarg\ifx{LaLaEnd\CurrentComment Test}\test
            \edef\html@next{\endgroup\noexpand\end{\CurrentComment}}
      \else \let\html@next\xComment
      \fi \fi \html@next}
}
\makeatother

\def\includecomment
 #1{\expandafter\def\csname#1\endcsname{}%
    \expandafter\def\csname end#1\endcsname{}}
\def\excludecomment
 #1{\expandafter\def\csname#1\endcsname{\ThrowAwayComment{#1}}%
    {\escapechar=-1\relax
     \csarg\xdef{PlainEnd#1Test}{\string\\end#1}%
     \csarg\xdef{LaLaEnd#1Test}{\string\\end\string\{#1\string\}}%
    }}

%  Define environments that ignore their contents.
\excludecomment{comment}
\excludecomment{rawhtml}
\excludecomment{htmlonly}

%  Hypertext commands etc. This is a condensed version of the html.sty
%  file supplied with LaTeX2HTML by: Nikos Drakos <nikos@cbl.leeds.ac.uk> &
%  Jelle van Zeijl <jvzeijl@isou17.estec.esa.nl>. The LaTeX2HTML documentation
%  should be consulted about all commands (and the environments defined above)
%  except \xref and \xlabel which are Starlink specific.

\newcommand{\htmladdnormallinkfoot}[2]{#1\footnote{#2}}
\newcommand{\htmladdnormallink}[2]{#1}
\newcommand{\htmladdimg}[1]{}
\newcommand{\hyperref}[4]{#2\ref{#4}#3}
\newcommand{\htmlref}[2]{#1}
\newcommand{\htmlimage}[1]{}
\newcommand{\htmladdtonavigation}[1]{}

\newenvironment{latexonly}{}{}
\newcommand{\latex}[1]{#1}
\newcommand{\html}[1]{}
\newcommand{\latexhtml}[2]{#1}
\newcommand{\HTMLcode}[2][]{}

%  Starlink cross-references and labels.
\newcommand{\xref}[3]{#1}
\newcommand{\xlabel}[1]{}

%  LaTeX2HTML symbol.
\newcommand{\latextohtml}{\LaTeX2\texttt{HTML}}

%  Define command to re-centre underscore for Latex and leave as normal
%  for HTML (severe problems with \_ in tabbing environments and \_\_
%  generally otherwise).
\renewcommand{\_}{\texttt{\symbol{95}}}

% -----------------------------------------------------------------------------
%  Debugging.
%  =========
%  Remove % on the following to debug links in the HTML version using Latex.

% \newcommand{\hotlink}[2]{\fbox{\begin{tabular}[t]{@{}c@{}}#1\\\hline{\footnotesize #2}\end{tabular}}}
% \renewcommand{\htmladdnormallinkfoot}[2]{\hotlink{#1}{#2}}
% \renewcommand{\htmladdnormallink}[2]{\hotlink{#1}{#2}}
% \renewcommand{\hyperref}[4]{\hotlink{#1}{\S\ref{#4}}}
% \renewcommand{\htmlref}[2]{\hotlink{#1}{\S\ref{#2}}}
% \renewcommand{\xref}[3]{\hotlink{#1}{#2 -- #3}}
%end{latexonly}
% -----------------------------------------------------------------------------
% ? Document specific \newcommand or \newenvironment commands.

% Command for displaying routines in routine lists:
% =================================================

\newcommand{\noteroutine}[2]{{\small \bf #1} \\
                              \hspace*{3em} {\em #2} \\[1.5ex]}
\begin{htmlonly}
   \newcommand{\noteroutine}[2]{
      \begin{description}
         \item [{\bf {#1}}] {\em #2}
      \end{description}
   }
\end{htmlonly}

% Latex only section. Surround this with a latexonly environment.
\newcommand{\latexonlysection}[1]{\section{#1}}
\begin{htmlonly}
   \newcommand{\latexonlysection}[1]{#1}
\end{htmlonly}

%+
%  Name:
%     SST.TEX

%  Purpose:
%     Define LaTeX commands for laying out Starlink routine descriptions.

%  Language:
%     LaTeX

%  Type of Module:
%     LaTeX data file.

%  Description:
%     This file defines LaTeX commands which allow routine documentation
%     produced by the SST application PROLAT to be processed by LaTeX and
%     by LaTeX2html. The contents of this file should be included in the
%     source prior to any statements that make of the sst commnds.

%  Notes:
%     The style file html.sty provided with LaTeX2html needs to be used.
%     This must be before this file.

%  Authors:
%     RFWS: R.F. Warren-Smith (STARLINK)
%     PDRAPER: P.W. Draper (Starlink - Durham University)

%  History:
%     10-SEP-1990 (RFWS):
%        Original version.
%     10-SEP-1990 (RFWS):
%        Added the implementation status section.
%     12-SEP-1990 (RFWS):
%        Added support for the usage section and adjusted various spacings.
%     8-DEC-1994 (PDRAPER):
%        Added support for simplified formatting using LaTeX2html.
%     {enter_further_changes_here}

%  Bugs:
%     {note_any_bugs_here}

%-

%  Define length variables.
\newlength{\sstbannerlength}
\newlength{\sstcaptionlength}
\newlength{\sstexampleslength}
\newlength{\sstexampleswidth}

%  Define a \tt font of the required size.
\latex{\newfont{\ssttt}{cmtt10 scaled 1095}}
\html{\newcommand{\ssttt}{\tt}}

%  Define a command to produce a routine header, including its name,
%  a purpose description and the rest of the routine's documentation.
\newcommand{\sstroutine}[3]{
   \goodbreak
   \rule{\textwidth}{0.5mm}
   \vspace{-7ex}
   \newline
   \settowidth{\sstbannerlength}{{\Large {\bf #1}}}
   \setlength{\sstcaptionlength}{\textwidth}
   \setlength{\sstexampleslength}{\textwidth}
   \addtolength{\sstbannerlength}{0.5em}
   \addtolength{\sstcaptionlength}{-2.0\sstbannerlength}
   \addtolength{\sstcaptionlength}{-5.0pt}
   \settowidth{\sstexampleswidth}{{\bf Examples:}}
   \addtolength{\sstexampleslength}{-\sstexampleswidth}
   \parbox[t]{\sstbannerlength}{\flushleft{\Large {\bf #1}}}
   \parbox[t]{\sstcaptionlength}{\center{\Large #2}}
   \parbox[t]{\sstbannerlength}{\flushright{\Large {\bf #1}}}
   \begin{description}
      #3
   \end{description}
}

%  Format the description section.
\newcommand{\sstdescription}[1]{\item[Description:] #1}

%  Format the usage section.
\newcommand{\sstusage}[1]{\item[Usage:] \mbox{}
\\[1.3ex]{\raggedright \ssttt #1}}

%  Format the invocation section.
\newcommand{\sstinvocation}[1]{\item[Invocation:]\hspace{0.4em}{\tt #1}}

%  Format the arguments section.
\newcommand{\sstarguments}[1]{
   \item[Arguments:] \mbox{} \\
   \vspace{-3.5ex}
   \begin{description}
      #1
   \end{description}
}

%  Format the returned value section (for a function).
\newcommand{\sstreturnedvalue}[1]{
   \item[Returned Value:] \mbox{} \\
   \vspace{-3.5ex}
   \begin{description}
      #1
   \end{description}
}

%  Format the parameters section (for an application).
\newcommand{\sstparameters}[1]{
   \item[Parameters:] \mbox{} \\
   \vspace{-3.5ex}
   \begin{description}
      #1
   \end{description}
}

%  Format the examples section.
\newcommand{\sstexamples}[1]{
   \item[Examples:] \mbox{} \\
   \vspace{-3.5ex}
   \begin{description}
      #1
   \end{description}
}

%  Define the format of a subsection in a normal section.
\newcommand{\sstsubsection}[1]{ \item[{#1}] \mbox{} \\}

%  Define the format of a subsection in the examples section.
\newcommand{\sstexamplesubsection}[2]{\sloppy
\item[\parbox{\sstexampleslength}{\ssttt #1}] \mbox{} \vspace{1.0ex}
\\ #2 }

%  Format the notes section.
\newcommand{\sstnotes}[1]{\item[Notes:] \mbox{} \\[1.3ex] #1}

%  Provide a general-purpose format for additional (DIY) sections.
\newcommand{\sstdiytopic}[2]{\item[{\hspace{-0.35em}#1\hspace{-0.35em}:}]
\mbox{} \\[1.3ex] #2}

%  Format the implementation status section.
\newcommand{\sstimplementationstatus}[1]{
   \item[{Implementation Status:}] \mbox{} \\[1.3ex] #1}

%  Format the bugs section.
\newcommand{\sstbugs}[1]{\item[Bugs:] #1}

%  Format a list of items while in paragraph mode.
\newcommand{\sstitemlist}[1]{
  \mbox{} \\
  \vspace{-3.5ex}
  \begin{itemize}
     #1
  \end{itemize}
}

%  Define the format of an item.
\newcommand{\sstitem}{\item}

%% Now define html equivalents of those already set. These are used by
%  latex2html and are defined in the html.sty files.
\begin{htmlonly}

%  sstroutine.
   \newcommand{\sstroutine}[3]{
      \subsection{#1\xlabel{#1}-\label{#1}#2}
      \begin{description}
         #3
      \end{description}
   }

%  sstdescription
   \newcommand{\sstdescription}[1]{\item[Description:]
      \begin{description}
         #1
      \end{description}
      \\
   }

%  sstusage
   \newcommand{\sstusage}[1]{\item[Usage:]
      \begin{description}
         {\ssttt #1}
      \end{description}
      \\
   }

%  sstinvocation
   \newcommand{\sstinvocation}[1]{\item[Invocation:]
      \begin{description}
         {\ssttt #1}
      \end{description}
      \\
   }

%  sstarguments
   \newcommand{\sstarguments}[1]{
      \item[Arguments:] \\
      \begin{description}
         #1
      \end{description}
      \\
   }

%  sstreturnedvalue
   \newcommand{\sstreturnedvalue}[1]{
      \item[Returned Value:] \\
      \begin{description}
         #1
      \end{description}
      \\
   }

%  sstparameters
   \newcommand{\sstparameters}[1]{
      \item[Parameters:] \\
      \begin{description}
         #1
      \end{description}
      \\
   }

%  sstexamples
   \newcommand{\sstexamples}[1]{
      \item[Examples:] \\
      \begin{description}
         #1
      \end{description}
      \\
   }

%  sstsubsection
   \newcommand{\sstsubsection}[1]{\item[{#1}]}

%  sstexamplesubsection
   \newcommand{\sstexamplesubsection}[2]{\item[{\ssttt #1}] #2}

%  sstnotes
   \newcommand{\sstnotes}[1]{\item[Notes:] #1 }

%  sstdiytopic
   \newcommand{\sstdiytopic}[2]{\item[{#1}] #2 }

%  sstimplementationstatus
   \newcommand{\sstimplementationstatus}[1]{
      \item[Implementation Status:] #1
   }

%  sstitemlist
   \newcommand{\sstitemlist}[1]{
      \begin{itemize}
         #1
      \end{itemize}
      \\
   }
%  sstitem
   \newcommand{\sstitem}{\item}

\end{htmlonly}

%  End of "sst.tex" layout definitions.
%.



% ? End of document specific commands
% -----------------------------------------------------------------------------
%  Title Page.
%  ===========
\renewcommand{\thepage}{\roman{page}}
\begin{document}
\thispagestyle{empty}

%  Latex document header.
%  ======================
\begin{latexonly}
   \textsc{Rutherford Appleton Laboratory} \hfill \textbf{\stardocname}\\
   {\large Science \& Technology Facilities Council}\\
   {\large Starlink Software Collection\\}
   {\large \stardoccategory\ \stardocnumber}
   \begin{flushright}
   \stardocauthors\\
   \stardocdate
   \end{flushright}
   \vspace{-4mm}
   \rule{\textwidth}{0.5mm}
   \vspace{5mm}
   \begin{center}
   {\Huge\textbf{\stardoctitle \\ [2.5ex]}}
   {\LARGE\textbf{\stardocversion \\ [4ex]}}
   {\Huge\textbf{\stardocmanual}}
   \end{center}
   \vspace{5mm}

% ? Add picture here if required for the LaTeX version.
%   e.g. \includegraphics[scale=0.3]{filename.ps}
% ? End of picture

% ? Heading for abstract if used.
   \vspace{10mm}
   \begin{center}
      {\Large\textbf{Abstract}}
   \end{center}
% ? End of heading for abstract.
\end{latexonly}

%  HTML documentation header.
%  ==========================
\begin{htmlonly}
   \xlabel{}
   \begin{rawhtml} <H1> \end{rawhtml}
      \stardoctitle\\
      \stardocversion\\
      \stardocmanual
   \begin{rawhtml} </H1> <HR> \end{rawhtml}

% ? Add picture here if required for the hypertext version.
%   e.g. \includegraphics[scale=0.7]{filename.ps}
% ? End of picture

   \begin{rawhtml} <P> <I> \end{rawhtml}
   \stardoccategory\ \stardocnumber \\
   \stardocauthors \\
   \stardocdate
   \begin{rawhtml} </I> </P> <H3> \end{rawhtml}
      \htmladdnormallink{Rutherford Appleton Laboratory}
                        {http://www.scitech.ac.uk} \\
      \htmladdnormallink{Science \& Technology Facilities Council}
                        {http://www.scitech.ac.uk} \\
   \begin{rawhtml} </H3> <H2> \end{rawhtml}
      \htmladdnormallink{Starlink Project}{http://www.starlink.ac.uk/}
   \begin{rawhtml} </H2> \end{rawhtml}
   \htmladdnormallink{\htmladdimg{source.gif} Retrieve hardcopy}
      {http://www.starlink.ac.uk/cgi-bin/hcserver?\stardocsource}\\

%  HTML document table of contents. 
%  ================================
%  Add table of contents header and a navigation button to return to this 
%  point in the document (this should always go before the abstract \section). 
  \label{stardoccontents}
  \begin{rawhtml} 
    <HR>
    <H2>Contents</H2>
  \end{rawhtml}
  \htmladdtonavigation{\htmlref{\htmladdimg{contents_motif.gif}}
        {stardoccontents}}

% ? New section for abstract if used.
  \section{\xlabel{abstract}Abstract}
% ? End of new section for abstract
\end{htmlonly}

% -----------------------------------------------------------------------------
% ? Document Abstract. (if used)
%  ==================
\stardocabstract
% ? End of document abstract

% -----------------------------------------------------------------------------
% ? Latex Copyright Statement
%  =========================
\begin{latexonly}
\newpage
\vspace*{\fill}
\stardoccopyright
\end{latexonly}
% ? End of Latex copyright statement

% -----------------------------------------------------------------------------
% ? Latex document Table of Contents (if used).
%  ===========================================
  \newpage
  \begin{latexonly}
    \setlength{\parskip}{0mm}
    \tableofcontents
    \setlength{\parskip}{\medskipamount}
    \markboth{\stardocname}{\stardocname}
  \end{latexonly}
% ? End of Latex document table of contents
% -----------------------------------------------------------------------------

\cleardoublepage
\renewcommand{\thepage}{\arabic{page}}
\setcounter{page}{1}

\section {Introduction}

If an application prompts the user for a catalogue (or table) using the
facilities of the CAT\_ system (see \xref{SUN/181}{sun181}{}), the
user may only reply with the name of a single catalogue. Some
applications allow many input catalogues to be specified and the need to
type in every catalogue name explicitly each time the program is run can
become time consuming. The CTG package provides a means of giving the
user the ability to specify a list (or \emph{Group}) of catalogues as a reply
to a single prompt for a parameter. 

\section {Interaction Between CTG and GRP}

CTG uses the facilities of the GRP package and users of CTG should be
familiar with the content of \xref{SUN/150}{sun150}{} which describes
the GRP package. Groups created by CTG routines should be deleted when
no longer needed using \xref{GRP\_DELET}{sun150}{GRP\_DELET}.

\section {General overview of the CTG\_ system}
As a broad outline, applications use the CTG\_ package as follows:

\begin{enumerate}

\item A call is made to \htmlref{CTG\_ASSOC}{CTG_ASSOC} which causes the user 
to be prompted for a single parameter. This parameter can be of any type.
The user replies with a \emph{group expression} (see
\xref{SUN/150}{sun150}{}), which contains the names of a group of {\em
existing} catalogues to be used as inputs by the application\
For instance, the group expression may be

\begin{verbatim}
     coma_b[1-3].fits,coma_r[45]s.txt,coma_b[23]?{2}.fit,^files.lis
\end{verbatim}

This is a complicated example, probably more complicated than would be
used in practice, but it highlights the facilities of the GRP and CTG
packages, \emph{e.g.} wild cards ({\tt "?"}, {\tt "$*$}" or {\tt "[..]"} ),
lists of files, or indirection through a text file ({\tt "\verb+^+}").
The braces indicate the second FITS extension.

The \htmlref{CTG\_ASSOC}{CTG_ASSOC} routine produces a list of
explicit catalogue names, which are stored internally within the GRP system.

\item What happens next depends on the application, but a common example may be
the initiation of a DO loop to loop through the input catalogues (CTG\_ASSOC returns
the total number of catalogue names in the group). 

\item To access a particular catalogue, the application calls routine 
\htmlref{CTG\_CATAS}{CTG_CATAS} supplying an index, $n$, within the
group (i.e $n$ is an integer in the range 1 to the group size returned
by CTG\_ASSOC). CTG\_CATAS returns a CAT identifier to the $n$th
catalogue in the group. This identifier can then be used to access the
catalogue in the normal manner using the CAT\_ routines
(\xref{SUN/181}{sun181}{}). The identifier should be released when it
is no longer needed, and freeing resoures using
\xref{CAT\_TRLSE}{sun181}{CAT_TRLSE} in the normal way.

\item Once the application has finished processing the group of
catalogues, it calls \xref{GRP\_DELET}{sun150}{GRP\_DELET} which deletes 
the group, releasing all resources reserved by the group.

\item Routine CTG\_ASSOC can also be used to append a list of catalogue names 
obtained from the environment, to a previously defined group.

The routine \htmlref{CTG\_CREAT}{CTG_CREAT} produces a group
containing the names of cataloguess that are to be created by the
application. The routine \htmlref{CTG\_CATCR}{CTG_CATCR} will create
a new catalogue with a name given by a group member, and returns a CAT
identifier to it.

The names of output catalogues given by users usually relate to the
input catalogue names.  When CTG\_CREAT is called, it creates a group
of catalogue names either by modifying all the names in a specified
input group using a \emph{modification element} (see
\xref{SUN/150}{sun150}{}), or by getting a list of new names from the
user. 

\item Applications which produce a group of output catalogues could
also produce a text file holding the names of the output catalogues. Such a
file can be used as input to the next application, using the
indirection facility. A text file listing of all the catalogues in a group
can be produced by routine \xref{GRP\_LIST}{sun150}{GRP_LIST} (or 
\xref{GRP\_LISTF}{sun150}{GRP_LISTF}).

\end{enumerate}

See the detailed descriptions of \htmlref{CTG\_ASSOC}{CTG_ASSOC} and
\htmlref{CTG\_CREAT}{CTG_CREAT} below for details of the processing
of existing and new catalogue names.

\section{An example CTG application}

The following gives a short example of how CTG routines might be 
used within an ADAM task.
\begin{quote}
\latexonly{\small}
\begin{verbatim}
      SUBROUTINE COPY( STATUS )

*  Global Constants:
      INCLUDE 'GRP_PAR'          ! Standard GRP constants

*  Local Variables:
      INTEGER CIN                ! GRP identifier for an input catalogue
      INTEGER COUT               ! GRP identifer for an output catalogue
      LOGICAL FLAG               ! Has group ended with flag character?
      INTEGER GIDIN              ! GRP identifier for group of input cat's
      INTEGER GIDOUT             ! GRP identifer for group of output cat's
      INTEGER I                  ! Loop counter
      INTEGER NUMIN              ! Number of input catalogues
      INTEGER NUMOUT             ! Number of output catalogues
      INTEGER STATUS             ! The global status
*.

*  Inialise the group identifiers to indicate that groups do not have
*  any initial members.
      GIDIN = GRP__NOID
      GIDOUT = GRP__NOID

*  Create group of input catalogues using the parameter IN.
      CALL CTG_ASSOC( 'IN', .TRUE., GIDIN, NUMIN, FLAG, STATUS )

*  Create group of output catalogues using the parameter OUT, that 
*  possibly works by modifying the values in the input group.
      CALL CTG_CREAT( 'OUT', GIDIN, GIDOUT, NUMOUT, FLAG, STATUS )

*  Loop over group members.
      DO I = 1, NUMIN

*  Get the identifier for an existing catalogue from the input group.
         CALL CTG_NDFAS( GIDIN, I, 'READ', CIN, STATUS )

*  Get the identifier for the output catalogue from the output group.
         CALL CTG_NDFAS( GIDOUT, I, 'WRITE', COUT, STATUS )

*  Proceed to create output catalogue by editing the input catalogue
*  using the CAT library.
               :             :             :             :

*  Release CAT resources.
         CALL CAT_TRLSE( CIN, STATUS )
         CALL CAT_TRLSE( COUT, STATUS )
      END DO

*  Release GRP resources.
      CALL GRP_DELET( GIDIN, STATUS )
      CALL GRP_DELET( GIDOUT, STATUS )

      END
\end{verbatim}
\end{quote}
When this program is compiled and run, the user is asked for 
two ADAM parameters, IN and OUT.  Each catalogue in the list specified
by the IN parameter is simply copied to the corresponding name in the 
list specified by the OUT parameter.  For instance, running
\begin{quote}
\latexonly{\small}
\begin{verbatim}
copy in=data[12].fits out=*-new
\end{verbatim}
\end{quote}
would write new catalogues {\tt data1-new.fits} and {\tt data2-new.fits}
which were edited copies of the existing files {\tt data1.fits} and
{\tt data2.fits}.
If {\tt data1} and {\tt data2} do not represent FITS catalogues
an error will be signalled and the user will be prompted to enter
a different value for IN.

Note that a few corners have been cut in the above code, in particular
checking that the input and output groups have the same size and
STATUS testing.  Additionally, no action is taken when the FLAG
character is given at the end of a group 
specification---conventionally this would indicate that the user
should be allowed to add further members.

\section {Compiling and Linking with CTG}

This section describes how to compile and link applications which use
CTG subroutines, on UNIX systems. It is assumed that the CTG library
is installed as part of the Starlink Software Collection.

The library only has an ADAM interface to obtain the groups of
catalogues.

\subsection{\label{ss:buildingadamapplications}ADAM Applications}
Users of the \xref{ADAM}{sg4}{} programming environment
\latex{(SG/4)} should use
the \xref{{\bf alink}}{sun144}{ADAM_link_scripts} command
(\xref{SUN/144}{sun144}{}) to compile and link applications, and can
access the CTG\_ library by including execution of the command
{\tt ctg\_link} on the command line, as follows:

\small
\begin{verbatim}
   % alink prog.f `ctg_link`
\end{verbatim}
\normalsize

where {\tt prog.f} is the Fortran source file for the A-TASK. Again
note the use of opening apostrophies (`) instead of the more usual
closing apostrophy (') in the above {\bf alink} command.

To build a program written in C (instead of Fortran), simply name the
source file {\tt prog.c}, instead of {\tt prog.f}.

\appendix

\begin{latexonly}
\latexonlysection{List of Routines}

\noteroutine{
      CALL CTG\_ASSO1( PARAM, VERB, MODE, CI, FIELDS, STATUS )
}{
   Obtain an identifier for a single existing catalogue using a specified
   parameter
}
\noteroutine{
      CALL CTG\_ASSOC( PARAM, VERB, IGRP, SIZE, FLAG, STATUS )
}{
   Store names of existing cataloguess specified through the environment
}
\noteroutine{
      CALL CTG\_CATAS( IGRP, INDEX, MODE, CI, STATUS )
}{
   Obtain a CAT identifier for an existing catalogue
}
\noteroutine{
      CALL CTG\_CATCR( IGRP, INDEX, CI, STATUS )
}{
   Obtain a CAT identifier for a new catalogue
}
\noteroutine{
      CALL CTG\_CREA1( PARAM, FTYPE, NDIM, LBND, UBND, CI, NAME,
                      STATUS )
}{
   Create a single new catalogue using a specified parameter
}
\noteroutine{
      CALL CTG\_CREAT( PARAM, IGRP0, IGRP, SIZE, FLAG, STATUS )
}{
   Obtain the names of a group of catalogues to be created from the
   environment
}
\noteroutine{
      CALL CTG\_GTSUP( IGRP, I, FIELDS, STATUS )
}{
   Get supplemental information for a catalogue
}
\noteroutine{
      CALL CTG\_PTSUP( IGRP, I, FIELDS, STATUS )
}{
   Store suplemental information for a catalogue
}
\noteroutine{
      CALL CTG\_SETSZ( IGRP, SIZE, STATUS )
}{
   Reduces the size of an CTG group
}

\end{latexonly}


\section{Full Fortran Routine Specifications}
\label {SEC:FULLSPEC}

% Routine descriptions:
% =====================
\small

\sstroutine{
   CTG\_ASSO1
}{
   Obtain an identifier for a single existing catalogue using a
   specified parameter
}{
   \sstdescription{
      This routine is equivalent to \xref{CAT\_ASSOC}{sun181}{CAT_ASSOC} except
      that it allows the catalogue to be specified using a GRP group
      expression (for instance, its name may be given within a text
      file, \emph{etc.}). The first catalogue in the group expression
      is returned. Any other names in the group expression are
      ignored. Supplemental information describing the separate fields
      in the catalogue specification are also returned.
   }
   \sstinvocation{
      CALL CTG\_ASSO1( PARAM, VERB, MODE, CI, FIELDS, STATUS )
   }
   \sstarguments{
      \sstsubsection{
         PARAM = CHARACTER $*$ ( $*$ ) (Given)
      }{
         Name of the ADAM parameter.
      }
      \sstsubsection{
         VERB = LOGICAL (Given)
      }{
         If TRUE then errors which occur whilst accessing supplied catalogues
         are flushed so that the user can see them before re-prompting for
         a new catalogue (\emph{verbose} mode). Otherwise, they are annulled and
         a general {\tt "Cannot access file xyz"} message is displayed before
         re-prompting.
      }
      \sstsubsection{
         MODE = CHARACTER $*$ ( $*$ ) (Given)
      }{
         Type of catalogue access required: {\tt 'READ'}, {\tt 'UPDATE'} or {\tt 'WRITE'}.
      }
      \sstsubsection{
         CI = INTEGER (Returned)
      }{
         catalogue identifier.
      }
      \sstsubsection{
         FIELDS( 5 ) = CHARACTER $*$ ( $*$ ) (Given)
      }{
         Each element contains the following on exit.

         \sstitemlist{
            \sstitem -- FITS extension specification (\emph{e.g.}{\tt "\{3\}"}) if any
            \sstitem -- File type
            \sstitem -- Base file name
            \sstitem -- Directory path
            \sstitem -- Full catalogue specification
         }
      }
      \sstsubsection{
         STATUS = INTEGER (Given and Returned)
      }{
         The global status.
      }
   }
}
\sstroutine{
   CTG\_ASSOC
}{
   Store names of existing catalogues specified through the environment
}{
   \sstdescription{
      A group expression is obtained from the environment using the
      supplied parameter. The expression is parsed (using the facilities
      of the GRP routine  \xref{GRP\_GROUP}{sun150}{GRP_GROUP}\latex{,see SUN/150}) 
      to produce a list of explicit names for existing catalogues
      which are appended to the end of the supplied group (a new group
      is created if none is supplied). If an error occurs while
      parsing the group expression, the user is re-prompted for a new
      group expression. CAT identifiers for particular members of the
      group can be obtained using \htmlref{CTG\_CATAS}{CTG_CATAS}.
   }
   \sstinvocation{
      CALL CTG\_ASSOC( PARAM, VERB, IGRP, SIZE, FLAG, STATUS )
   }
   \sstarguments{
      \sstsubsection{
         PARAM = CHARACTER$*$($*$) (Given)
      }{
         The parameter with which to associate the group expression.
      }
      \sstsubsection{
         VERB = LOGICAL (Given)
      }{
         If TRUE then errors which occur whilst accessing supplied catalogues
         are flushed so that the user can see them before re-prompting for
         a new catalogue \emph{verbose} mode). Otherwise, they are annulled and
         a general {\tt "Cannot access file xyz"} message is displayed before
         re-prompting.
      }
      \sstsubsection{
         IGRP = INTEGER (Given and Returned)
      }{
         The identifier of the group in which the catalogue names are to be
         stored. A new group is created if the supplied value is GRP\_\_NOID.
         It should be deleted when no longer needed using
         \xref{GRP\_DELET}{sun150}{GRP_DELET}.
      }
      \sstsubsection{
         SIZE = INTEGER (Returned)
      }{
         The total number of catalogue names in the returned group.
      }
      \sstsubsection{
         FLAG = LOGICAL (Returned)
      }{
         If the group expression was terminated by the GRP \emph{flag
         character}, then FLAG is returned {\tt .TRUE.}. Otherwise it is
         returned {\tt .FALSE.}. Returned {\tt .FALSE.} if an error occurs.
      }
      \sstsubsection{
         STATUS = INTEGER (Given and Returned)
      }{
         The global status.
      }
   }
   \sstnotes{
      \sstitemlist{

         \sstitem
         Any file names containing wildcards are expanded into a list
	 of catalogue names. The supplied strings are intepreted by a
	 shell (/bin/tcsh if it exists, otherwise /bin/csh, otherwise
	 /bin/sh), and so may contain shell meta-characters
	 (\emph{e.g.} twiddle, \$HOME, even command substitution and
	 pipes - but pipe characters {\tt "$|$"} need to be escaped
	 using a backslash {\tt "$\backslash$"} to avoid them being 
	 interpreted as GRP editing characters).

         \sstitem
         Only the highest priority file with any give file name is included
         in the returned group. The priority of a file is determined by its
         file type. Priority decreases along the following list of file
         types: .FIT, .fit, .FITS, .fits, .GSC, .gsc, .TXT, .txt, .Txt, .sdf.
         If no file type is given by the user, the highest priority available
         file type is used. If an explicit file type is given, then that file
         type is used.

         \sstitem
         Names of catalogues stored in FITS format may include an FITS
	 extension number. For instance, {\tt
	 "/home/dsb/mydata.fit\{3\}"} refers to a catalogue stored in
	 the third extension of the FITS file {\tt mydata.fit}.

         \sstitem
         Catalogues stored in HDS format must be stored as the top level
         object within the .sdf file.

         \sstitem
         All matching files are opened in order to ensure that they are
         valid catalogues. The user is notified if there are no valid
         catalogues matching a supplied name, and they are asked to supply a
         replacement parameter value.

         \sstitem
         Each element in the returned group contains a full specification
         for a catalogue. Several other groups are created by this routine,
         and are associated with the returned group by means of a GRP
         {\emph owner-slave} relationship. These supplemental groups are
         automatically deleted when the returned group is deleted using
         GRP\_DELET. The returned group should not be altered using GRP
         directly because corresponding changes may need to be made to the
         supplemental groups. Routines \htmlref{CTG\_SETSZ}{CTG_SETSZ}, 
         \htmlref{CTG\_GTSUP}{CTG_GTSUP} and \htmlref{CTG\_PTSUP}{CTG_PTSUP}
         are provided to manipulate the entire chain of groups. The full
         chain (starting from the head) is as follows:

         \sstitemlist{
           \sstitem
               FITS extension numbers (if any)

            \sstitem
               File types

            \sstitem
               Base file names

            \sstitem
               Directory paths

            \sstitem
               Full catalogue specification (this is the returned group IGRP)
         }

         \sstitem
         If an error is reported the group is returned unaltered. If no
         group is supplied, an empty group is returned.

         \sstitem
         A null value ({\tt !}) can be given for the parameter to indicate that
         no more catalogues are to be specified. The corresponding error is
         annulled before returning unless no catalogues have been added to
         the group.

         \sstitem
         If the last character in the supplied group expression is a colon
         ({\tt :}), a list of the catalogues represented by the group expression
         (minus the colon) is displayed, but none are actually added to the
         group. The user is then re-prompted for a new group expression.
      }
   }
}
\sstroutine{
   CTG\_CATAS
}{
   Obtain a CAT identifier for an existing catalogue
}{
   \sstdescription{
      The routine returns a CAT identifier for an existing catalogue. The
      name of the catalogue is held at a given index within a given group.
      It is equivalent to \xref{CAT\_ASSOC}{sun181}{CAT_ASSOC}.
   }
   \sstinvocation{
      CALL CTG\_CATAS( IGRP, INDEX, MODE, CI, STATUS )
   }
   \sstarguments{
      \sstsubsection{
         IGRP = INTEGER (Given)
      }{
         A GRP identifier for a group holding the names of catalogues. This
         will often be created using \htmlref{CTG\_ASSOC}{CTG_ASSOC}, but groups 
         created \emph{by hand} using GRP directly (\emph{i.e.} without the 
         supplemental groups created by CTG\_ASSOC) can also be used.
      }
      \sstsubsection{
         INDEX = INTEGER (Given)
      }{
         The index within the group at which the name of the catalogue to be
         accessed is stored.
      }
      \sstsubsection{
         MODE = CHARACTER $*$ ( $*$ ) (Given)
      }{
         Type of catalogue access required: {\tt 'READ'}, or {\tt 'WRITE'}.
      }
      \sstsubsection{
         CI = INTEGER (Returned)
      }{
         catalogue identifier.
      }
      \sstsubsection{
         STATUS = INTEGER (Given and Returned)
      }{
         The global status.
      }
   }
   \sstnotes{
      \sstitemlist{

         \sstitem
         If this routine is called with STATUS set, then a value of
         CAT\_\_NOID will be returned for the CI argument, although no
         further processing will occur. The same value will also be
         returned if the routine should fail for any reason. The CAT\_\_NOID
         constant is defined in the include file {\tt CAT\_PAR}.
      }
   }
}
\sstroutine{
   CTG\_CATCR
}{
   Obtain a CAT identifier for a new catalogue
}{
   \sstdescription{
      The routine returns a CAT identifier for a new catalogue. The name
      of the new catalogue is held at a given index within a given group.
      It is equivalent to \htmlref{CAT\_CREAT}{CAT_CREAT}, except that any 
      existing catalogue with the specified name is first deleted (unless 
      the catalogue specification includes a FITS-extension specifier).
   }
   \sstinvocation{
      CALL CTG\_CATCR( IGRP, INDEX, CI, STATUS )
   }
   \sstarguments{
      \sstsubsection{
         IGRP = INTEGER (Given)
      }{
         A GRP identifier for a group holding the names of catalogues. This
         will often be created using \htmlref{CTG\_CREAT}{CTG_CREAT}, but 
         groups created \emph{by hand} using GRP directly can also be used.
      }
      \sstsubsection{
         INDEX = INTEGER (Given)
      }{
         The index within the group at which the name of the catalogue to be
         created is stored.
      }
      \sstsubsection{
         CI = INTEGER (Returned)
      }{
         Catalogue identifier.
      }
      \sstsubsection{
         STATUS = INTEGER (Given and Returned)
      }{
         The global status.
      }
   }
}
\sstroutine{
   CTG\_CREA1
}{
   Create a single new catalogue using a specified parameter
}{
   \sstdescription{
      This routine is equivalent to \htmlref{CAT\_CREAT}{CAT_CREAT} except 
      that it allows the catalogue to be specified using a GRP group 
      expression (for instance, its name may be given within a text file, 
      {\emph etc.}), and it also ensures that any existing catalogue with 
      the same name is deleted before the new one is created (so long as no 
      FITS extension number is included in the catalogue specification). The 
      first catalogue in the group expression is returned. Any other names in
      the group expression are ignored. Any modification elements in the
      supplied group expression will be treated literally.
   }
   \sstinvocation{
      CALL CTG\_CREA1( PARAM, CI, NAME, STATUS )
   }
   \sstarguments{
      \sstsubsection{
         PARAM = CHARACTER $*$ ( $*$ ) (Given)
      }{
         Name of the ADAM parameter.
      }
      \sstsubsection{
         CI = INTEGER (Returned)
      }{
         Catalogue identifier.
      }
      \sstsubsection{
         NAME = CHARACTER $*$ ( $*$ ) (Returned)
      }{
         The file specification for the catalogue.
      }
      \sstsubsection{
         STATUS = INTEGER (Given and Returned)
      }{
         The global status.
      }
   }
}
\sstroutine{
   CTG\_CREAT
}{
   Obtain the names of a group of catalogues to be created from the
   environment
}{
   \sstdescription{
      A group expression is obtained from the environment using the
      supplied parameter. The expression is parsed (using the
      facilities of the GRP routine 
      \xref{GRP\_GROUP}{sun150}{GRP_GROUP}\latex{, see SUN/150}) to produce
      a list of explicit catalogue names. These names are appended
      to the group identified by IGRP. The user is re-prompted if an
      error occurs while parsing the group expression. If IGRP has the
      value {\tt GRP\_\_NOID} on entry, then a new group is created and IGRP is
      returned holding the new group identifier.

      If IGRP0 holds a valid group identifier on entry, then the group
      identified by IGRP0 is used as the basis for any modification
      element contained in the group expression obtained from the
      environment. If IGRP0 holds an invalid identifier (such as
      {\tt GRP\_\_NOID}) on entry then modification elements are included
      literally in the output group.
   }
   \sstinvocation{
      CALL CTG\_CREAT( PARAM, IGRP0, IGRP, SIZE, FLAG, STATUS )
   }
   \sstarguments{
      \sstsubsection{
         PARAM = CHARACTER$*$($*$) (Given)
      }{
         The parameter with which to associate the group.
      }
      \sstsubsection{
         IGRP0 = INTEGER (Given)
      }{
         The GRP identifier for the group to be used as the basis for
         any modification elements. If a valid GRP identifier is
         supplied, and if the supplied group expression contains a
         modification element, then:

         \sstitemlist{

            \sstitem
            the basis token (an asterisk) is replaced by the file basename
            associated with the corresponding element of the basis group (the
            \emph{basis catalogue}); else

            \sstitem
            if no directory specification is included in the group expression,
            the directory specification associated with the basis catalogue is
            used.

         }
         The supplied group will often be created by \htmlref{CTG\_ASSOC}{CTG_ASSOC}, but
         groups created \emph{by hand} using GRP directly can also be used
         (\emph{i.e.} without the supplemental groups created by CTG). In
         this case, there are no defaults for directory path or file type,
         and the basis token ({\tt "$*$"}) in the group expression represents the
         full basis file specification supplied in IGRP0, not just the file
         basename.
      }
      \sstsubsection{
         IGRP = INTEGER (Given and Returned)
      }{
         The GRP identifier for the group to which the supplied
         files are to be appended.
      }
      \sstsubsection{
         SIZE = INTEGER (Returned)
      }{
         The total number of file names in the returned group.
      }
      \sstsubsection{
         FLAG = LOGICAL (Returned)
      }{
         If the group expression was terminated by the GRP \emph{flag}
         character, then FLAG is returned {\tt .TRUE.}. Otherwise it is
         returned {\tt .FALSE.}. Returned {\tt .FALSE.} if an error occurs.
      }
      \sstsubsection{
         STATUS = INTEGER (Given and Returned)
      }{
         The global status.
      }
   }
   \sstnotes{
      \sstitemlist{

         \sstitem
         Any FITS extensions specified in the group expression are ignored.

         \sstitem
         If an error is reported the group is returned unaltered.

         \sstitem
         A null value ({\tt !}) can be given for the parameter to indicate
         that no more catalogues are to be specified. The corresponding error
         is annulled before returning unless no catalogues have been added to
         the group.

         \sstitem
         If no file type is supplied in the group expression, then the first
         file type listed in the current value of the {\tt CAT\_FORMATS\_OUT} environment
         variable is used. If this is {\tt "$*$"} then the file type is copied from the
         corresponding input file if a modification element was used to specify
         the output file name (if the catalogue was not specified by a
         modification element, the second file type in {\tt CAT\_FORMATS\_OUT} is
         used).

         \sstitem
         If the last character in the supplied group expression is
         a colon ({\tt :}), a list of the catalogues represented by the group
         expression (minus the colon) is displayed, but none are
         actually added to the group. The user is then re-prompted for
         a new group expression.

         \sstitem
         The returned group has no associated groups holding supplemental
         information (unlike the group returned by \htmlref{CTG\_ASSOC}{CTG_ASSOC}).
      }
   }
}

\sstroutine{
   CTG\_GTSUP
}{
   Get supplemental information for a catalogue
}{
   \sstdescription{
      Returns the supplemental information associated with a given entry
      in a CTG group.
   }
   \sstinvocation{
      CALL CTG\_GTSUP( IGRP, I, FIELDS, STATUS )
   }
   \sstarguments{
      \sstsubsection{
         IGRP = INTEGER (Given)
      }{
         The CTG group as returned by CTG\_ASSOC, {\emph etc.}. This should be the last
         group in a GRP owner-slave chain.
      }
      \sstsubsection{
         I = INTEGER (Given)
      }{
         The index of the required entry.
      }
      \sstsubsection{
         FIELDS( 5 ) = CHARACTER $*$ ( $*$ ) (Returned)
      }{
         The supplemental information associated with the entry specified
         by I. Each element of the returned array contains the following:

         \sstitemlist{
            \sstitem -- FITS extension specification (\emph{e.g.}{\tt "\{3\}"}) if any
            \sstitem -- File type
            \sstitem -- Base file name
            \sstitem -- Directory path
            \sstitem -- Full catalogue specification
         }

         This information is obtained from a set of groups associated with
         the supplied group IGRP by means of a chain of GRP \emph{owner-slave}
         relationships. If any of these groups do not exist, the correponding
         elements of the above array are returned blank. Note, Element 5,
         the full catalogue specification, is obtained directly from the
         supplied group IGRP.
      }
      \sstsubsection{
         STATUS = INTEGER (Given and Returned)
      }{
         The global status.
      }
   }
}

\sstroutine{
   CTG\_PTSUP
}{
   Store suplemental information for an catalogue
}{
   \sstdescription{
      Stores the supplied items of supplemental information for a given
      entry in a CTG group. The GRP groups needed to store this
      supplemental information are created if they do not already exist,
      and associated with the supplied group by means of a chain of GRP
      \emph{owner-slave} relationships. They will be deleted automaticaly when
      the supplied group is deleted using \xref{GRP\_DELET}{sun150}{GRP_DELET}.
   }
   \sstinvocation{
      CALL CTG\_PTSUP( IGRP, I, FIELDS, STATUS )
   }
   \sstarguments{
      \sstsubsection{
         IGRP = INTEGER (Given)
      }{
         The CTG group as returned by \htmlref{CTG\_ASSOC}{CTG_ASSOC}, \emph{etc.}. 
         This should be the last group in a GRP owner-slave chain.
      }
      \sstsubsection{
         I = INTEGER (Given)
      }{
         The index of the required entry.
      }
      \sstsubsection{
         FIELDS( 5 ) = CHARACTER $*$ ( $*$ ) (Given)
      }{
         The supplemental information to be stored with the entry specified
         by I. Each element of the supplied array should contain the
         following:
         \sstitemlist{
            \sstitem -- FITS extension (\emph{e.g.}{\tt "\{3\}"}) if any
            \sstitem -- File type
            \sstitem -- Base file name
            \sstitem -- Directory path
            \sstitem -- Full catalogue specification
         }
      }
      \sstsubsection{
         STATUS = INTEGER (Given and Returned)
      }{
         The global status.
      }
   }
}

\sstroutine{
   CTG\_SETSZ
}{
   Reduces the size of a CTG group
}{
   \sstdescription{
      This routine should be used instead of \xref{GRP\_SETSZ}{sun150}{GRP_SETSZ} 
      to set the size of a group created by CTG. It sets the size of the supplied 
      group, and also sets the size of each of the supplemental groups associated 
      with the supplied group.
   }
   \sstinvocation{
      CALL CTG\_SETSZ( IGRP, SIZE, STATUS )
   }
   \sstarguments{
      \sstsubsection{
         IGRP = INTEGER (Given)
      }{
         The CTG group as returned by \htmlref{CTG\_ASSOC}{CTG_ASSOC}, \emph{etc.} 
         This should be the last group in a GRP owner-slave chain.
      }
      \sstsubsection{
         SIZE = INTEGER (Given)
      }{
         The new group size. Must be less than or equal to the size of the
         smallest group in the chain.
      }
      \sstsubsection{
         STATUS = INTEGER (Given and Returned)
      }{
         The global status.
      }
   }
}

\section{Changes Introduced in CTG Version 3.0}
\begin{itemize}
   \item It has a separate identity.  It was previously bundled into KAPLIBS.
   \item There is documentation.
\end{itemize}

\end{document}
