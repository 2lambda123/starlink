\documentclass[twoside,11pt]{article}

% ? Specify used packages
% \usepackage{graphicx}        %  Use this one for final production.
% \usepackage[draft]{graphicx} %  Use this one for drafting.
% ? End of specify used packages

\pagestyle{myheadings}

% -----------------------------------------------------------------------------
% ? Document identification
% Fixed part
\newcommand{\stardoccategory}  {Starlink User Note}
\newcommand{\stardocinitials}  {SUN}
\newcommand{\stardocsource}    {sun\stardocnumber}
\newcommand{\stardoccopyright}
{Copyright \copyright\ 2000 Council for the Central Laboratory of the Research Councils}

% Variable part - replace [xxx] as appropriate.
\newcommand{\stardocnumber}    {61.4}
\newcommand{\stardocauthors}   {R.F. Warren-Smith}
\newcommand{\stardocdate}      {12th January 2006}
\newcommand{\stardoctitle}     {TRANSFORM \latexhtml{\\[2.5ex]}{\\}
                                Coordinate Transformation Facility}
\newcommand{\stardocversion}   {Version 0.9-6}
\newcommand{\stardocmanual}    {Programmer's Guide and Reference Manual}
\newcommand{\stardocabstract}  {%

TRANSFORM provides a standard, flexible method for manipulating
coordinate transformations and for transferring information about them
between applications.  It can handle coordinate systems with any number
of dimensions and can efficiently process large
(\emph{i.e.}\ image-sized) arrays of coordinate data using a variety of
numerical precisions.  No specific support for astronomical coordinate
transformations or map projections is included -- these features are
provided by the \xref{AST}{sun210}{} library.

}

% ? End of document identification
% -----------------------------------------------------------------------------

% +
%  Name:
%     sun.tex
%
%  Purpose:
%     Template for Starlink User Note (SUN) documents.
%     Refer to SUN/199
%
%  Authors:
%     AJC: A.J.Chipperfield (Starlink, RAL)
%     BLY: M.J.Bly (Starlink, RAL)
%     PWD: Peter W. Draper (Starlink, Durham University)
%
%  History:
%     17-JAN-1996 (AJC):
%        Original with hypertext macros, based on MDL plain originals.
%     16-JUN-1997 (BLY):
%        Adapted for LaTeX2e.
%        Added picture commands.
%     13-AUG-1998 (PWD):
%        Converted for use with LaTeX2HTML version 98.2 and
%        Star2HTML version 1.3.
%      1-FEB-2000 (AJC):
%        Add Copyright statement in LaTeX
%     {Add further history here}
%
% -

\newcommand{\stardocname}{\stardocinitials /\stardocnumber}
\markboth{\stardocname}{\stardocname}
\setlength{\textwidth}{160mm}
\setlength{\textheight}{230mm}
\setlength{\topmargin}{-2mm}
\setlength{\oddsidemargin}{0mm}
\setlength{\evensidemargin}{0mm}
\setlength{\parindent}{0mm}
\setlength{\parskip}{\medskipamount}
\setlength{\unitlength}{1mm}

% -----------------------------------------------------------------------------
%  Hypertext definitions.
%  ======================
%  These are used by the LaTeX2HTML translator in conjunction with star2html.

%  Comment.sty: version 2.0, 19 June 1992
%  Selectively in/exclude pieces of text.
%
%  Author
%    Victor Eijkhout                                      <eijkhout@cs.utk.edu>
%    Department of Computer Science
%    University Tennessee at Knoxville
%    104 Ayres Hall
%    Knoxville, TN 37996
%    USA

%  Do not remove the %begin{latexonly} and %end{latexonly} lines (used by
%  LaTeX2HTML to signify text it shouldn't process).
%begin{latexonly}
\makeatletter
\def\makeinnocent#1{\catcode`#1=12 }
\def\csarg#1#2{\expandafter#1\csname#2\endcsname}

\def\ThrowAwayComment#1{\begingroup
    \def\CurrentComment{#1}%
    \let\do\makeinnocent \dospecials
    \makeinnocent\^^L% and whatever other special cases
    \endlinechar`\^^M \catcode`\^^M=12 \xComment}
{\catcode`\^^M=12 \endlinechar=-1 %
 \gdef\xComment#1^^M{\def\test{#1}
      \csarg\ifx{PlainEnd\CurrentComment Test}\test
          \let\html@next\endgroup
      \else \csarg\ifx{LaLaEnd\CurrentComment Test}\test
            \edef\html@next{\endgroup\noexpand\end{\CurrentComment}}
      \else \let\html@next\xComment
      \fi \fi \html@next}
}
\makeatother

\def\includecomment
 #1{\expandafter\def\csname#1\endcsname{}%
    \expandafter\def\csname end#1\endcsname{}}
\def\excludecomment
 #1{\expandafter\def\csname#1\endcsname{\ThrowAwayComment{#1}}%
    {\escapechar=-1\relax
     \csarg\xdef{PlainEnd#1Test}{\string\\end#1}%
     \csarg\xdef{LaLaEnd#1Test}{\string\\end\string\{#1\string\}}%
    }}

%  Define environments that ignore their contents.
\excludecomment{comment}
\excludecomment{rawhtml}
\excludecomment{htmlonly}

%  Hypertext commands etc. This is a condensed version of the html.sty
%  file supplied with LaTeX2HTML by: Nikos Drakos <nikos@cbl.leeds.ac.uk> &
%  Jelle van Zeijl <jvzeijl@isou17.estec.esa.nl>. The LaTeX2HTML documentation
%  should be consulted about all commands (and the environments defined above)
%  except \xref and \xlabel which are Starlink specific.

\newcommand{\htmladdnormallinkfoot}[2]{#1\footnote{#2}}
\newcommand{\htmladdnormallink}[2]{#1}
\newcommand{\htmladdimg}[1]{}
\newcommand{\hyperref}[4]{#2\ref{#4}#3}
\newcommand{\htmlref}[2]{#1}
\newcommand{\htmlimage}[1]{}
\newcommand{\htmladdtonavigation}[1]{}

\newenvironment{latexonly}{}{}
\newcommand{\latex}[1]{#1}
\newcommand{\html}[1]{}
\newcommand{\latexhtml}[2]{#1}
\newcommand{\HTMLcode}[2][]{}

%  Starlink cross-references and labels.
\newcommand{\xref}[3]{#1}
\newcommand{\xlabel}[1]{}

%  LaTeX2HTML symbol.
\newcommand{\latextohtml}{\LaTeX2\texttt{HTML}}

%  Define command to re-centre underscore for Latex and leave as normal
%  for HTML (severe problems with \_ in tabbing environments and \_\_
%  generally otherwise).
\renewcommand{\_}{\texttt{\symbol{95}}}

% -----------------------------------------------------------------------------
%  Debugging.
%  =========
%  Remove % on the following to debug links in the HTML version using Latex.

% \newcommand{\hotlink}[2]{\fbox{\begin{tabular}[t]{@{}c@{}}#1\\\hline{\footnotesize #2}\end{tabular}}}
% \renewcommand{\htmladdnormallinkfoot}[2]{\hotlink{#1}{#2}}
% \renewcommand{\htmladdnormallink}[2]{\hotlink{#1}{#2}}
% \renewcommand{\hyperref}[4]{\hotlink{#1}{\S\ref{#4}}}
% \renewcommand{\htmlref}[2]{\hotlink{#1}{\S\ref{#2}}}
% \renewcommand{\xref}[3]{\hotlink{#1}{#2 -- #3}}
%end{latexonly}
% -----------------------------------------------------------------------------
% ? Document specific \newcommand or \newenvironment commands.

%   General definitions for use throughout the document.
\newcommand{\name}[1]{\mbox{\small{#1}}}
\newcommand{\fortvar}[1]{\mbox{\emph{#1}}}

%   This defines "example" headings keeps count of them.
\newcounter{examplecounter}
\newcommand{\example}[1]{\addtocounter{examplecounter}{1}
                         \paragraph{\textbf{Example \theexamplecounter. #1}}}
\newcommand{\exampledone}[0]{\begin{center} \rule{6em}{0.2mm} \end{center}}

%   These are used in the Appendix on "Classifying Transformations".
\newcommand{\pdiff}[2]{\frac{\partial y_{#1}}{\partial x_{#2}}}
\newcommand{\req}[0]{$\bullet$}
\newcommand{\imp}[0]{$\circ$}
\newcommand{\pro}[0]{$\times$}
\newcommand{\irr}[0]{$-$}

\newcommand{\noteroutine}[2]{\textbf{#1}\hspace*{\fill}\nopagebreak \\
                             \hspace*{3em}\emph{#2}\hspace*{\fill}\par}

%+
%  Name:
%     SST.TEX

%  Purpose:
%     Define LaTeX commands for laying out Starlink routine descriptions.

%  Language:
%     LaTeX

%  Type of Module:
%     LaTeX data file.

%  Description:
%     This file defines LaTeX commands which allow routine documentation
%     produced by the SST application PROLAT to be processed by LaTeX and
%     by LaTeX2html. The contents of this file should be included in the
%     source prior to any statements that make of the sst commnds.

%  Notes:
%     The style file html.sty provided with LaTeX2html needs to be used.
%     This must be before this file.

%  Authors:
%     RFWS: R.F. Warren-Smith (STARLINK)
%     PDRAPER: P.W. Draper (Starlink - Durham University)

%  History:
%     10-SEP-1990 (RFWS):
%        Original version.
%     10-SEP-1990 (RFWS):
%        Added the implementation status section.
%     12-SEP-1990 (RFWS):
%        Added support for the usage section and adjusted various spacings.
%     8-DEC-1994 (PDRAPER):
%        Added support for simplified formatting using LaTeX2html.
%     {enter_further_changes_here}

%  Bugs:
%     {note_any_bugs_here}

%-

%  Define length variables.
\newlength{\sstbannerlength}
\newlength{\sstcaptionlength}
\newlength{\sstexampleslength}
\newlength{\sstexampleswidth}

%  Define a \tt font of the required size.
\latex{\newfont{\ssttt}{cmtt10 scaled 1095}}
\html{\newcommand{\ssttt}{\tt}}

%  Define a command to produce a routine header, including its name,
%  a purpose description and the rest of the routine's documentation.
\newcommand{\sstroutine}[3]{
   \goodbreak
   \rule{\textwidth}{0.5mm}
   \vspace{-7ex}
   \newline
   \settowidth{\sstbannerlength}{{\Large {\bf #1}}}
   \setlength{\sstcaptionlength}{\textwidth}
   \setlength{\sstexampleslength}{\textwidth}
   \addtolength{\sstbannerlength}{0.5em}
   \addtolength{\sstcaptionlength}{-2.0\sstbannerlength}
   \addtolength{\sstcaptionlength}{-5.0pt}
   \settowidth{\sstexampleswidth}{{\bf Examples:}}
   \addtolength{\sstexampleslength}{-\sstexampleswidth}
   \parbox[t]{\sstbannerlength}{\flushleft{\Large {\bf #1}}}
   \parbox[t]{\sstcaptionlength}{\center{\Large #2}}
   \parbox[t]{\sstbannerlength}{\flushright{\Large {\bf #1}}}
   \begin{description}
      #3
   \end{description}
}

%  Format the description section.
\newcommand{\sstdescription}[1]{\item[Description:] #1}

%  Format the usage section.
\newcommand{\sstusage}[1]{\item[Usage:] \mbox{}
\\[1.3ex]{\raggedright \ssttt #1}}

%  Format the invocation section.
\newcommand{\sstinvocation}[1]{\item[Invocation:]\hspace{0.4em}{\tt #1}}

%  Format the arguments section.
\newcommand{\sstarguments}[1]{
   \item[Arguments:] \mbox{} \\
   \vspace{-3.5ex}
   \begin{description}
      #1
   \end{description}
}

%  Format the returned value section (for a function).
\newcommand{\sstreturnedvalue}[1]{
   \item[Returned Value:] \mbox{} \\
   \vspace{-3.5ex}
   \begin{description}
      #1
   \end{description}
}

%  Format the parameters section (for an application).
\newcommand{\sstparameters}[1]{
   \item[Parameters:] \mbox{} \\
   \vspace{-3.5ex}
   \begin{description}
      #1
   \end{description}
}

%  Format the examples section.
\newcommand{\sstexamples}[1]{
   \item[Examples:] \mbox{} \\
   \vspace{-3.5ex}
   \begin{description}
      #1
   \end{description}
}

%  Define the format of a subsection in a normal section.
\newcommand{\sstsubsection}[1]{ \item[{#1}] \mbox{} \\}

%  Define the format of a subsection in the examples section.
\newcommand{\sstexamplesubsection}[2]{\sloppy
\item[\parbox{\sstexampleslength}{\ssttt #1}] \mbox{} \vspace{1.0ex}
\\ #2 }

%  Format the notes section.
\newcommand{\sstnotes}[1]{\item[Notes:] \mbox{} \\[1.3ex] #1}

%  Provide a general-purpose format for additional (DIY) sections.
\newcommand{\sstdiytopic}[2]{\item[{\hspace{-0.35em}#1\hspace{-0.35em}:}]
\mbox{} \\[1.3ex] #2}

%  Format the implementation status section.
\newcommand{\sstimplementationstatus}[1]{
   \item[{Implementation Status:}] \mbox{} \\[1.3ex] #1}

%  Format the bugs section.
\newcommand{\sstbugs}[1]{\item[Bugs:] #1}

%  Format a list of items while in paragraph mode.
\newcommand{\sstitemlist}[1]{
  \mbox{} \\
  \vspace{-3.5ex}
  \begin{itemize}
     #1
  \end{itemize}
}

%  Define the format of an item.
\newcommand{\sstitem}{\item}

%% Now define html equivalents of those already set. These are used by
%  latex2html and are defined in the html.sty files.
\begin{htmlonly}

%  sstroutine.
   \newcommand{\sstroutine}[3]{
      \subsection{#1\xlabel{#1}-\label{#1}#2}
      \begin{description}
         #3
      \end{description}
   }

%  sstdescription
   \newcommand{\sstdescription}[1]{\item[Description:]
      \begin{description}
         #1
      \end{description}
      \\
   }

%  sstusage
   \newcommand{\sstusage}[1]{\item[Usage:]
      \begin{description}
         {\ssttt #1}
      \end{description}
      \\
   }

%  sstinvocation
   \newcommand{\sstinvocation}[1]{\item[Invocation:]
      \begin{description}
         {\ssttt #1}
      \end{description}
      \\
   }

%  sstarguments
   \newcommand{\sstarguments}[1]{
      \item[Arguments:] \\
      \begin{description}
         #1
      \end{description}
      \\
   }

%  sstreturnedvalue
   \newcommand{\sstreturnedvalue}[1]{
      \item[Returned Value:] \\
      \begin{description}
         #1
      \end{description}
      \\
   }

%  sstparameters
   \newcommand{\sstparameters}[1]{
      \item[Parameters:] \\
      \begin{description}
         #1
      \end{description}
      \\
   }

%  sstexamples
   \newcommand{\sstexamples}[1]{
      \item[Examples:] \\
      \begin{description}
         #1
      \end{description}
      \\
   }

%  sstsubsection
   \newcommand{\sstsubsection}[1]{\item[{#1}]}

%  sstexamplesubsection
   \newcommand{\sstexamplesubsection}[2]{\item[{\ssttt #1}] #2}

%  sstnotes
   \newcommand{\sstnotes}[1]{\item[Notes:] #1 }

%  sstdiytopic
   \newcommand{\sstdiytopic}[2]{\item[{#1}] #2 }

%  sstimplementationstatus
   \newcommand{\sstimplementationstatus}[1]{
      \item[Implementation Status:] #1
   }

%  sstitemlist
   \newcommand{\sstitemlist}[1]{
      \begin{itemize}
         #1
      \end{itemize}
      \\
   }
%  sstitem
   \newcommand{\sstitem}{\item}

\end{htmlonly}

%  End of "sst.tex" layout definitions.
%.

% ? End of document specific commands
% -----------------------------------------------------------------------------
%  Title Page.
%  ===========
\renewcommand{\thepage}{\roman{page}}
\begin{document}
\thispagestyle{empty}

%  Latex document header.
%  ======================
\begin{latexonly}
   CCLRC / \textsc{Rutherford Appleton Laboratory} \hfill \textbf{\stardocname}\\
   {\large Particle Physics \& Astronomy Research Council}\\
   {\large Starlink Project\\}
   {\large \stardoccategory\ \stardocnumber}
   \begin{flushright}
   \stardocauthors\\
   \stardocdate
   \end{flushright}
   \vspace{-4mm}
   \rule{\textwidth}{0.5mm}
   \vspace{5mm}
   \begin{center}
   {\Huge\textbf{\stardoctitle \\ [2.5ex]}}
   {\LARGE\textbf{\stardocversion \\ [4ex]}}
   {\Huge\textbf{\stardocmanual}}
   \end{center}
   \vspace{5mm}

% ? Add picture here if required for the LaTeX version.
%   e.g. \includegraphics[scale=0.3]{filename.ps}
% ? End of picture

% ? Heading for abstract if used.
   \vspace{10mm}
   \begin{center}
      {\Large\textbf{Abstract}}
   \end{center}
% ? End of heading for abstract.
\end{latexonly}

%  HTML documentation header.
%  ==========================
\begin{htmlonly}
   \xlabel{}
   \begin{rawhtml} <H1> \end{rawhtml}
      \stardoctitle
   \begin{rawhtml} </H1> \end{rawhtml}
   \begin{rawhtml} <H3> \end{rawhtml}
      \stardocversion \\
      \stardocmanual
   \begin{rawhtml} </H3> <HR> \end{rawhtml}

% ? Add picture here if required for the hypertext version.
%   e.g. \includegraphics[scale=0.7]{filename.ps}
% ? End of picture

   \begin{rawhtml} <P> <I> \end{rawhtml}
   \stardoccategory\ \stardocnumber \\
   \stardocauthors \\
   \stardocdate
   \begin{rawhtml} </I> </P> <H3> \end{rawhtml}
      \htmladdnormallink{CCLRC / Rutherford Appleton Laboratory}
                        {http://www.cclrc.ac.uk} \\
      \htmladdnormallink{Particle Physics \& Astronomy Research Council}
                        {http://www.pparc.ac.uk} \\
   \begin{rawhtml} </H3> <H2> \end{rawhtml}
      \htmladdnormallink{Starlink Project}{http://www.starlink.ac.uk/}
   \begin{rawhtml} </H2> \end{rawhtml}
   \htmladdnormallink{\htmladdimg{source.gif} Retrieve hardcopy}
      {http://www.starlink.ac.uk/cgi-bin/hcserver?\stardocsource}\\

%  HTML document table of contents.
%  ================================
%  Add table of contents header and a navigation button to return to this
%  point in the document (this should always go before the abstract \section).
  \label{stardoccontents}
  \begin{rawhtml}
    <HR>
    <H2>Contents</H2>
  \end{rawhtml}
  \htmladdtonavigation{\htmlref{\htmladdimg{contents_motif.gif}}
        {stardoccontents}}

% ? New section for abstract if used.
  \section{\xlabel{abstract}Abstract}
% ? End of new section for abstract
\end{htmlonly}

% -----------------------------------------------------------------------------
% ? Document Abstract. (if used)
%  ==================
\stardocabstract
% ? End of document abstract

% -----------------------------------------------------------------------------
% ? Latex Copyright Statement
%  =========================
\begin{latexonly}
\newpage
\vspace*{\fill}
\stardoccopyright
\end{latexonly}
% ? End of Latex copyright statement

% -----------------------------------------------------------------------------
% ? Latex document Table of Contents (if used).
%  ===========================================
  \newpage
  \begin{latexonly}
    \setlength{\parskip}{0mm}
    \tableofcontents
    \setlength{\parskip}{\medskipamount}
    \markboth{\stardocname}{\stardocname}
  \end{latexonly}
% ? End of Latex document table of contents
% -----------------------------------------------------------------------------

\cleardoublepage
\renewcommand{\thepage}{\arabic{page}}
\setcounter{page}{1}

% ? Main text

\section{\xlabel{introduction}Introduction}

This document describes version 0.9 of the \name{TRANSFORM} coordinate
transformation facility and shows how it may be used in application programs
operating within the Starlink \name{ADAM} environment.
It is assumed that the reader is familiar with this environment and with the
Starlink Hierarchical Data System \name{HDS} (\xref{SUN/92}{sun92}{}).

The main text of this document provides a guide for programmers who have not
used the \name{TRANSFORM} facility before.
It contains an introduction to the overall capabilities of the software and
the basic concepts involved, followed by a tutorial-style description of all
the main features with examples of their use.

The Appendices at the end of the document contain further reference material
which will mainly be of value to more experienced users.
In particular, Appendix~\ref{appendix_routines} gives a description of all
the user-callable routines provided.


\subsection{\xlabel{what_is_the_nametransform_facility}What is the \name{TRANSFORM} Facility?}

The \name{TRANSFORM} facility is a library of subroutines which may be used
by application programs to process information describing the relationships
between different coordinate systems.

It provides a standard, flexible method for manipulating coordinate
transformations and for transferring information about them between
applications.
It can handle coordinate systems with any number of dimensions and can
efficiently process large (\emph{i.e.}\ image-sized) arrays of coordinate
data using a variety of numerical precisions.
No specific support for astronomical coordinate transformations or map
projections is included at present, but routines for handling these will
appear in future.
The current system provides tools for creating a wide variety of coordinate
transformations, so it should be possible to construct some of the simpler
astronomical transformations explicitly, if required, on an interim basis.

Some possible applications of \name{TRANSFORM} routines include:

\begin{itemize}
\item Defining linear and non-linear graphics coordinate systems;
\item Attaching coordinate systems to datasets (\emph{e.g.}\ relating image
pixels to sky positions);
\item Describing distortion in images and spectra;
\item Storing and applying instrumental calibration functions.
\end{itemize}

Note that the \name{TRANSFORM} facility uses the Hierarchical Data System
\name{(HDS)} to store its information in standard data structures for
interchange between applications.
These data structures may therefore be used as building blocks when
constructing larger \name{HDS} datasets and also when designing
``extensions'' to the standard Starlink \name{NDF} data structure (see
\xref{SGP/38}{sgp38}{}).

\section{\xlabel{basic_concepts_and_terminology}Basic Concepts and Terminology}

\subsection{\xlabel{transformations}Transformations}

The \name{TRANSFORM} facility stores the information which it requires for
conversion between different coordinate systems in \name{HDS} structures
called {\em transformations}.
They have an \name{HDS} type of \name{TRN\_TRANSFORM}\footnote{
Full internal details of all the \name{HDS} structures used by the
\name{TRANSFORM} facility are given in Appendix~\ref{appendix_structures}.}
and the purpose of each of these structures is to describe the relationship
which exists between two different coordinate systems.
The two systems in question are termed the \emph{input} and \emph{output}
coordinate systems and the \name{HDS} transformation structure serves to
``link'' them together as a pair.
In this role, the transformation may be regarded simply as a ``black box'',
thus:

\begin{quote}
\begin{center}
\setlength{\unitlength}{1mm}
\thicklines
\begin{picture}(60, 30)(-30,-15)
\put(-15,-5){\framebox(30,10){\textbf{Transformation}}}
\put(-15,0){\line(-1,0){10}}
\put(-25,0){\circle*{1.5}}
\put(-37.5,0){\oval(25,25)}
\put(-37.5,0){\makebox(0,0){\shortstack{{\textbf{Input}}\\
{\textbf{Coordinate}}\\{\textbf{System}}\\
$(x_{1},x_{2},\ldots x_{m})$}}}
\put(15,0){\line(1,0){10}}
\put(25,0){\circle*{1.5}}
\put(37.5,0){\oval(25,25)}
\put(37.5,0){\makebox(0,0){\shortstack{{\textbf{Output}}\\
{\textbf{Coordinate}}\\{\textbf{System}}\\
$(y_{1},y_{2},\ldots y_{n})$}}}
\end{picture}
\end{center}
\end{quote}

Positions (or \emph{points}) in each coordinate system will be specified by
appropriate sets of \emph{coordinates}, and the purpose of the transformation
is to define an ``association'' between corresponding points in each system.
Thus, given the coordinates of a point in one system, the coordinates of the
associated point in the other system may be derived by using the information
stored in the transformation structure.

\paragraph{Transformation variables.}
The inter-conversion of coordinates in this way between two different
2-dimensional systems (\emph{e.g.}\ between \emph{Equatorial} and
\emph{Galactic} sky coordinates) is a familiar concept.
In general, however, there is no need for the two coordinate systems to be
2-dimensional, nor even for both of them to have the same dimensionality.
To cater for this general case, each transformation has a set of \emph{input}
and \emph{output} \emph{variables} associated with it.
They are represented here by \mbox{$(x_{1},\ldots x_{m})$} and
\mbox{$(y_{1},\ldots y_{n})$} respectively, where the numbers of input and
output variables ($m$ and $n$) may be any positive integer -- not
necessarily equal.
These variables are akin to the ``dummy arguments'' used in Fortran
subroutines and are the means by which coordinate values are passed to and
from the transformation; they may be visualised as a set of input and output
ports attached to it, as follows:

\begin{displaymath}
\begin{array}{c}
\mbox{Forward } \longrightarrow \\

m \mbox{ input values }
\left\{ \begin{array}{c} x_{1} \\ x_{2} \\ \vdots \\ x_{m} \end{array} \right\}

\begin{array}{c}\mbox{
\setlength{\unitlength}{1mm}
\thicklines
\begin{picture}(40, 10)(-20,-5)
\put(-15,-5){\framebox(30,10){\textbf{Transformation}}}
\multiput(-15,-2.5)(0,2.5){3}{\line(-1,0){3}}
\multiput(-18,-2.5)(0,2.5){3}{\circle*{1}}
\multiput(15,-2)(0,4){2}{\line(1,0){3}}
\multiput(18,-2)(0,4){2}{\circle*{1}}
\end{picture}
} \end{array}

\left\{ \begin{array}{c} y_{1} \\ y_{2} \\ \vdots \\ y_{n} \end{array} \right\}
n \mbox{ output values}

\\ \longleftarrow \mbox{ Inverse}
\end{array}
\end{displaymath}

Any coordinate values supplied to the ports (\emph{i.e.}\ variables) at
either end of the transformation may be converted into a corresponding set
of ``transformed'' coordinates (using the information within the
transformation), to be delivered through the ports (\emph{i.e.}\ variables)
at the opposite end.

As with subroutine arguments, the names assigned to the input/output
variables have no absolute significance outside the transformation itself,
and coordinate values must be supplied in the correct order (typically
within a data array, for instance) in order to ``match up'' with the
appropriate transformation variables.
In practice, of course, when describing a transformation, it is convenient
to retain meaningful names for its variables; these may then be referred to
using the usual notation as \mbox{$(\alpha,\delta)$} or \mbox{$(x,y,z)$},
etc., so that the expected order is clear.

\subsection{\xlabel{mappings}Mappings}

The most important part of any transformation is a pair of ``numerical
recipes'' which provide a description of the precise numerical steps which
must be carried out in order to perform coordinate conversion in each
direction.
These ``recipes'' are termed the \emph{forward} and \emph{inverse}
\emph{mappings}.\footnote{
Note that the term \emph{mapping} is used here with its usual mathematical
meaning, whereas the (mathematically synonymous) term \emph{transformation}
is reserved to represent a more complex structure which may contain both a
forward and inverse mapping, as well as possible ancillary information.
The two terms should not be confused.}
It will frequently be possible to define both of these, so that coordinate
conversion may be performed in either direction.
However, this may not always be possible (or desirable), so a transformation
may also exist even if only one of its two possible mappings is defined.
In this case, coordinate conversion will only be possible in one direction.

As will be seen later, it is often convenient to extract the mappings from
a transformation and to process them as separate entities.


\paragraph{Notation.}
A general transformation with $m$ input variables and $n$ output variables,
in which both the forward and inverse mappings are defined, is denoted here
by \mbox{[$m \leftrightarrow n$]}.
The double-ended arrow `$\leftrightarrow$' indicates that coordinate
conversion is possible in either direction.
Similarly, the notation \mbox{[$m \rightarrow n$]} and \mbox{[$m \leftarrow
n$]} is used to represent transformations where only the forward or inverse
mapping (respectively) is defined, so that coordinate conversion can be
performed only in the direction indicated.
This notation may also be used to describe individual transformations by
including a list of their input and output variables.
Thus, a transformation which relates a \emph{Cartesian} coordinate system to
a \emph{Polar} system might be denoted by \mbox{[$(x,y) \leftrightarrow
(r,\theta)$]}.

To distinguish \emph{transformations} from \emph{mappings}, the notation for
the latter uses braces `\{\ldots\}' rather than square brackets, so that a
mapping which describes how to convert a set of $m$ input values into a set
of $n$ output values would be denoted by \mbox{\{$m \rightarrow n$\}}.
In this case, the arrow may only point to the right because a single
mapping, once separated from its parent transformation, can only perform
coordinate conversion in one direction.
This is always regarded as its ``forward'' direction.


\subsection{\xlabel{transformation_functions}Transformation Functions}

In principle, the value of each of a transformation's output variables may
depend on the values supplied to \textbf{all} of its input variables.
Consequently, a general transformation's forward mapping may only be
specified in full by giving a complete set of \emph{transformation functions}
which define the precise form of this dependence for each of the output
variables.
The same consideration also applies to the inverse mapping.
Thus, in general, a transformation's two mappings may be decomposed into a
set of $n$ \emph{forward transformation functions} (denoted \mbox{$F_1,\ldots
F_n$}) and a set of $m$ \emph{inverse transformation functions} (denoted
\mbox{$I_1,\ldots I_m$}) which act upon the input and output variables, as
follows:

\begin{equation}
\begin{array}{cc}
\mbox{Forward }
\left\{
\begin{array}{ccc}
y_1 & = & F_1(x_1, x_2,\ldots x_m)\\
y_2 & = & F_2(x_1, x_2,\ldots x_m)\\
\vdots & & \vdots\\
y_n & = & F_n(x_1, x_2,\ldots x_m)
\end{array}
\right.
&
\mbox{Inverse }
\left\{
\begin{array}{ccc}
x_1 & = & I_1(y_1, y_2,\ldots y_n)\\
x_2 & = & I_2(y_1, y_2,\ldots y_n)\\
\vdots & & \vdots\\
x_m & = & I_m(y_1, y_2,\ldots y_n)
\end{array}
\right.
\end{array}
\label{equation:transformationfunctions}
\end{equation}

\example{Drawing graphs.}
A simple illustration of mappings, transformation functions and the notation
used to describe them may be taken from the \mbox{[$2 \leftrightarrow 2$]}
transformation commonly used to relate ``data'' coordinates $(x_d,y_d)$ to
``graph paper'' coordinates $(x_p,y_p)$ when drawing a graph.
If the graph is linear, then the transformation's two mappings might
typically be defined by the following transformation functions:

\begin{equation}
\begin{array}{cc}
\mbox{Forward mapping } \left\{
\begin{array}{lll}
x_p & = & S_x(x_d-x_0) \\
y_p & = & S_y(y_d-y_0)
\end{array} \right.
&
\mbox{Inverse mapping } \left\{
\begin{array}{lll}
x_d & = & (x_p/S_x)+x_0 \\
y_d & = & (y_p/S_y)+y_0
\end{array} \right.
\end{array}
\end{equation}

where \mbox{$x_0$} and \mbox{$y_0$} are zero points on the two axes, and
\mbox{$S_x$} and \mbox{$S_y$} are scale factors.
In this example, \mbox{($x_d,y_d$)} have been treated as input variables,
while \mbox{($x_p,y_p$)} are output variables.
This particular choice is arbitrary, although a convention would have to be
adopted before writing software which used such a transformation.

Using the notation outlined above, this simple transformation would be
denoted by:
\begin{quote}
\begin{center}
\mbox{[$(x_d,y_d) \leftrightarrow (x_p,y_p)$]}
\end{center}
\end{quote}
\emph{i.e.}\ with the input variables appearing on the left.
The forward mapping would then be denoted by:
\begin{quote}
\begin{center}
\mbox{\{$(x_d,y_d) \rightarrow (x_p,y_p)$\}}
\end{center}
\end{quote}
and the inverse mapping by:
\begin{quote}
\begin{center}
\mbox{\{$(x_p,y_p) \rightarrow (x_d,y_d)$\}}
\end{center}
\end{quote}
\exampledone


\section{\xlabel{simple_use_of_transform_routines}Simple use of TRANSFORM Routines}


\subsection{\xlabel{formulating_a_transformation}Formulating a Transformation}

The first stage in creating a transformation is to formulate a description
of it in a form which may be used in a program.
In future, there may be a number of ways of formulating such a description
and of creating a corresponding transformation from it.
At present, however, only one method is supported, based on the explicit use
of transformation functions.

For instance, suppose you wished to create the \mbox{[$(x,y) \leftrightarrow
(r,\theta)$]} transformation relating a two-dimensional {\em Cartesian}
coordinate system \mbox{$(x,y)$} to a \emph{Polar} system
\mbox{$(r,\theta)$}.
In this case, the transformation's two mappings might be defined in terms of
the following transformation functions:

\begin{equation}
\begin{array}{cc}
\mbox{Forward } \left\{
\begin{array}{lll}
r & = & \sqrt{x^2+y^2} \\
\theta & = & \tan^{-1} \left( {y/x} \right)
\end{array}
\right.
&
\mbox{Inverse } \left\{
\begin{array}{lll}
x & = & r \cos( \theta ) \\
y & = & r \sin( \theta )
\end{array}
\right.
\end{array}
\label{equation:polar}
\end{equation}

This description of the transformation could now be used in a program by
converting it directly into character data, as follows:

\example{Formulating a Cartesian-to-Polar transformation.}
\begin{quote}
\begin{verbatim}
*  Define the number of input and output variables.
      INTEGER NVIN, NVOUT
      PARAMETER ( NVIN = 2, NVOUT = 2 )

*  Declare arrays for the forward and inverse transformation functions.
      CHARACTER * 25  FOR( NVOUT ), INV( NVIN )        [1]

*  Assign the forward transformation functions.
      FOR( 1 ) = 'r = sqrt( x * x + y * y )'           [2][3]
      FOR( 2 ) = 'theta = atan2( y, x )'

*  Assign the inverse transformation functions.
      INV( 1 ) = 'x = r * cos( theta )'
      INV( 2 ) = 'y = r * sin( theta )'

\end{verbatim}
\end{quote}

Programming notes:

\begin{enumerate}

\item  The transformation functions are assigned to the elements of two
character arrays \fortvar{FOR} and \fortvar{INV}, the number of elements in
each array being determined by the number of output and input variables,
\fortvar{NVOUT} and \fortvar{NVIN} respectively.

\item The formulae of Equation~\ref{equation:polar} have been converted into
character data for storage in these arrays by using Fortran arithmetic
operators and intrinsic functions (see Appendix~\ref{appendix_syntax} for a
full description of the syntax of transformation functions, which closely
resembles that of Fortran~77 assignment statements).

\item The names `\verb#x#', `\verb#y#', `\verb#r#' and `\verb#theta#' have
been used to represent the transformation's input and output variables.
Names such as these may be chosen freely (see
Appendix~\ref{appendix_syntax}) and are defined implicitly when they appear
on the left hand side of a transformation function.

\end{enumerate}
\exampledone

\paragraph{Unspecified mappings.}
In the above example, both the forward and inverse mappings were defined.
If only one of these were required, however, then the other could be left
unspecified.
This is done simply by omitting the right hand sides (and `$=$' signs) from
the transformation functions which define it.
Thus, if the inverse transformation functions had been specified as:

\begin{quote}
\begin{verbatim}
INV( 1 ) = 'x'
INV( 2 ) = 'y'
\end{verbatim}
\end{quote}

then the inverse mapping would not be defined.
Note that the left hand sides of the transformation functions must still
appear, however, because they define the names of the transformation's input
variables.


\subsection{\xlabel{creating_a_transformation}Creating a Transformation}
\label{section:simple:creating}

The transformation functions formulated above may be used to create an
\name{HDS} transformation structure by calling the routine \name{TRN\_NEW}
(create new transformation), as follows:

\begin{quote}\small
\begin{verbatim}
CALL TRN_NEW( NVIN, NVOUT, FOR, INV, PREC, COMM, ELOC, NAME, LOCTR, STATUS )
\end{verbatim}
\end{quote}

where:

\begin{itemize}

\item \fortvar{NVIN} and \fortvar{NVOUT} are the numbers of input and output
variables;

\item \fortvar{FOR} and \fortvar{INV} are the two character arrays
containing the forward and inverse transformation functions;

\item \fortvar{PREC} is a character expression specifying the arithmetic
precision with which the transformation functions will be evaluated;

\item \fortvar{COMM} is a comment to be stored with the transformation;

\item \fortvar{ELOC} is a locator to an existing \name{HDS} structure;

\item \fortvar{NAME} is the \name{HDS} name of the new structure component
to be created;

\item \fortvar{LOCTR} returns a locator to the newly created transformation
structure;

\item \fortvar{STATUS} is an inherited error status variable.

\end{itemize}

\example{Creating a Cartesian-to-Polar transformation.}
The following arrangement might be used to create a transformation called
\name{MYTRAN} from the \fortvar{FOR} and \fortvar{INV} arrays defined
earlier:

\begin{quote}
\begin{verbatim}
*  Declare variables.
      INCLUDE 'SAE_PAR'
      INCLUDE 'TRN_PAR'                                    [1]
      CHARACTER PREC * ( TRN__SZPRC ), COMM * 80
      CHARACTER * ( DAT__SZLOC ) ELOC, LOCTR

*  Get a locator ELOC to an existing HDS structure.
      <a call to DAT_ASSOC or DAT_FIND, for instance>

*  Specify the arithmetic precision and make a comment.
      PREC = '_REAL:'                                      [2]
      COMM = '2-d Cartesian (x,y) --> 2-d Polar (r,theta)  [3]

*  Create the transformation.
      CALL TRN_NEW( NVIN, NVOUT, FOR, INV, PREC, COMM,     [4][5]
     :              ELOC, 'MYTRAN', LOCTR, STATUS )

\end{verbatim}
\end{quote}

Programming notes:

\begin{enumerate}

\item The symbolic constant \name{TRN\_\_SZPRC} is defined in the include
file \name{TRN\_PAR}.
It is used to declare the length of the character variable \fortvar{PREC}
which will contain a precision specification.

\item \fortvar{PREC} specifies the type of arithmetic to be used when the
transformation functions are evaluated and has been assigned the value
`\verb#_REAL:#', which is recommended for general use.
This indicates that single precision (\emph{real}) arithmetic should normally
be used but that the precision may be increased if \emph{double precision}
data are being processed.
This is discussed further in Section~\ref{section:advanced:precision}.

\item The `\verb#-->#' (or `\verb#<--#') character sequence may be used in
comment strings to indicate the direction of the forward mapping.
If the transformation is subsequently \emph{inverted}
(Section~\ref{section:additional:inverting}), then the `\verb#-->#' and
`\verb#<--#' symbols will be interchanged so that the comment remains valid.
Comments may be of any length.

\item \name{TRN\_NEW} will check the transformation functions for correct
syntax and consistent use of variable names before the transformation is
created.
However, no check is performed to determine whether the forward and inverse
transformation functions actually define a pair of complementary mappings.
It is the caller's responsibility to ensure that this is so.

\item On successful exit from \name{TRN\_NEW}, the \fortvar{LOCTR} argument
returns an \name{HDS} locator associated with the newly created
transformation.

\end{enumerate}
\exampledone


\subsection{\xlabel{temporary_transformations}Temporary Transformations}

\label{section:simple:temporary}

The \name{TRN\_NEW} routine may also be used to create a temporary
transformation.
This can be very convenient for internal use by an application because it
avoids the need for an existing \name{HDS} structure to contain it.
To create such a transformation, the \fortvar{ELOC} argument to
\name{TRN\_NEW} should be replaced by a blank string (the \fortvar{NAME}
argument will then be ignored and may also be blank).
Thus, if \name{TRN\_NEW} were invoked by:

\begin{quote}\small
\begin{verbatim}
CALL TRN_NEW( NVIN, NVOUT, FOR, INV, PREC, COMM, ' ', ' ', LOCTR, STATUS )
\end{verbatim}
\end{quote}

then the locator \fortvar{LOCTR} would subsequently be associated with a
temporary transformation (\emph{i.e.}\ a temporary \name{HDS} structure of
type \name{TRN\_TRANSFORM}).


\subsection{\xlabel{compilation}Compilation}

The transformation structures created by \name{TRN\_NEW} hold information in
a form which can be easily manipulated by \name{HDS}.
However, this form of storage is not efficient if the transformation is to
be used, perhaps repeatedly, to process large arrays of data.
Therefore, before a transformation can be used to transform coordinate data,
its mapping information must first be \emph{compiled} using \name{TRN\_COMP}
(compile transformation), thus:

\begin{verbatim}
      CALL TRN_COMP( LOCTR, FORWD, ID, STATUS )
\end{verbatim}

The \name{TRN\_COMP} routine accepts a locator \fortvar{LOCTR} associated
with a transformation and checks the information within it.
It then compiles one of the transformation's mappings into a different form
which is stored internally by the \name{TRANSFORM} facility.
This internal representation is called a \emph{compiled mapping} and may
subsequently be used (for instance) to convert coordinate data from one
coordinate system to another.
The logical argument \fortvar{FORWD} is used to specify which mapping is
required -- the forward mapping is compiled if this argument is
\name{.TRUE.}\ and the inverse mapping is compiled if it is \name{.FALSE.}.
An error will be reported if the requested mapping has not been defined.

Any number of transformations may be compiled by repeatedly calling
\name{TRN\_COMP}.
To distinguish the resulting compiled mappings, a unique \emph{integer}
identifier is issued for each via the \fortvar{ID} argument, and these
identifiers are subsequently used to pass the mappings to other routines.

It is important to appreciate that a \emph{compiled mapping} is a
``uni-directional'' object and can only perform coordinate conversion in a
single direction, whereas a \emph{transformation} (which may contain up to
two mappings) is potentially ``bi-directional''.
Thus, if an [$m \leftrightarrow n$] transformation is compiled in the
forward direction (with the \fortvar{FORWD} argument of \name{TRN\_COMP} set
to \name{.TRUE.}), then a \mbox{\{$m \rightarrow n$\}} mapping will result.
Conversely, compilation in the inverse direction would yield a \mbox{\{$n
\rightarrow m$\}} mapping.
Note that in this latter case the numbers of input and output variables will
have been interchanged by the compilation process.


\subsection{\xlabel{inquiry_routines}Inquiry Routines}

An application may determine the number of input and output variables
used by a transformation by calling \name{TRN\_GTNV} (get numbers of
variables), thus:

\begin{quote}
\begin{verbatim}
CALL TRN_GTNV( LOCTR, NVIN, NVOUT, STATUS )
\end{verbatim}
\end{quote}

The information is returned via the \fortvar{NVIN} and \fortvar{NVOUT}
arguments.
Since \name{TRN\_GTNV} first checks that the locator \fortvar{LOCTR} is
associated with a valid transformation, this also provides a convenient way
of validating a transformation.

A similar routine \name{TRN\_GTNVC} (get numbers of compiled variables) is
provided for use with compiled mappings:

\begin{quote}
\begin{verbatim}
CALL TRN_GTNVC( ID, NVIN, NVOUT, STATUS )
\end{verbatim}
\end{quote}

In this case, the mapping is specified by its \emph{integer} identifier
\fortvar{ID}.


\subsection{\xlabel{transforming_1dimensional_coordinate_data}Transforming 1-Dimensional Coordinate Data}

\label{section:simple:1ddata}

A number of routines are provided for applying compiled mappings to arrays
of coordinate data (\emph{i.e.}\ lists of coordinates).
They are distinguished according to the number of variables used by the
mapping, by the way the data are stored in the arrays and by the type
(\emph{i.e.}\ numerical precision) of the data being transformed.
The common and relatively simple cases of 1- and 2-dimensional data are
discussed here and in the following Section.
Section~\ref{section:additional:generaldata} describes the use of more
general mappings.

The simplest situation arises when the input and output data points are
both 1-dimensional, so that a \mbox{\{$1 \rightarrow 1$\}} mapping is
to be applied.  Routines with names of the form \name{TRN\_TR1x} are
provided for this purpose, where x is either \textbf{I}, \textbf{R} or
\textbf{D} according to whether the data are specified by
\emph{integer}, \emph{real} or \emph{double precision} values
respectively.  Thus, to apply a compiled \mbox{\{$1 \rightarrow 1$\}}
mapping to a set of 1-dimensional data points specified by a list of
\emph{real} coordinates, the following statement might be used:

\begin{quote}
\begin{verbatim}
CALL TRN_TR1R( BAD, NX, XIN, ID, XOUT, STATUS )
\end{verbatim}
\end{quote}

where:

\begin{itemize}

\item \fortvar{BAD} is a logical value specifying whether the input
coordinates may contain \emph{bad} values (see
Section~\ref{section:additional:badvalues});

\item \fortvar{NX} specifies the number of data points to be transformed;

\item \fortvar{XIN} is a 1-dimensional array containing the coordinates of
the input data points;

\item \fortvar{ID} is an identifier associated with the compiled \mbox{\{$1
\rightarrow 1$\}} mapping to be applied;

\item \fortvar{XOUT} is a 1-dimensional array to receive the (transformed)
coordinates of the output data points;

\item \fortvar{STATUS} is an inherited error status variable.

\end{itemize}

\example{Plotting a graph of a user-specified function.}
In the following, \name{TRN\_TR1R} is used to generate data for a graph.
The function to be plotted is specified by the user via an expression
obtained through the \name{ADAM} parameter system.
For instance, if the user entered:

\begin{quote}
\begin{verbatim}
X*EXP(-4.4*X)
\end{verbatim}
\end{quote}

then a graph of the function \mbox{$y=xe^{-4.4x}$} would be plotted over the
range \mbox{$0\le x\le 1$}.

\begin{quote}
\begin{verbatim}
\verb#*  Declare variables.
      INCLUDE 'SAE_PAR'
      INTEGER NPTS
      PARAMETER ( NPTS = 1000 )
      INTEGER STATUS, ID, IPOINT
      REAL X( NPTS ), Y( NPTS )
      CHARACTER EXPRS * 80, FOR( 1 ) * 82, INV( 1 ) * 1
      CHARACTER * ( DAT__SZLOC ) LOCTR

*  Obtain an expression and formulate the transformation functions.
      CALL PAR_GET0C( 'EXPRESSION', EXPRS, STATUS )           [1]
      FOR( 1 ) = 'Y=' // EXPRS
      INV( 1 ) = 'X'

*  Create a temporary transformation and compile it.
      CALL TRN_NEW( 1, 1, FOR, INV, '_REAL:',                 [2]
                    'X --> f( X )', ' ', ' ', LOCTR, STATUS )
      CALL TRN_COMP( LOCTR, .TRUE., ID, STATUS )

*  Set up the X values, then transform to yield the Y values.
      DO 1 IPOINT = 1, NPTS
         X( IPOINT ) = REAL( IPOINT - 1 ) / REAL( NPTS - 1 )
    1 CONTINUE
      CALL TRN_TR1R( .FALSE., NPTS, X, ID, Y, STATUS )        [3]

*  Plot the graph and clean up.
      CALL GPL( NPTS, X, Y )                                  [4]
      CALL DAT_ANNUL( LOCTR, STATUS )                         [5]
      CALL TRN_ANNUL( ID, STATUS )

\end{verbatim}
\end{quote}

Programming notes:

\begin{enumerate}

\item An expression is obtained via the parameter system and used to
formulate suitable transformation functions.
Only the forward mapping is specified.

\item A temporary transformation is created (\name{TRN\_NEW}) and is then
compiled (\name{TRN\_COMP}).

\item After generating the \fortvar{X} values, \name{TRN\_TR1R} is called to
apply the compiled mapping, thereby generating the corresponding \fortvar{Y}
values.
The \fortvar{BAD} argument is set to \name{.FALSE.} since there are no {\em
bad} input coordinates.

\item A graph is plotted using the \name{GKS} polyline routine \name{GPL}.
It is assumed that a suitable coordinate system has been established.

\item Lastly, a clean up is performed by annulling the temporary
transformation (\name{DAT\_ANNUL}) and the compiled mapping
(\name{TRN\_ANNUL} -- described later in
Section~\ref{section:simple:cleanup}).

\end{enumerate}
\exampledone


\subsection{\xlabel{transforming_2dimensional_coordinate_data}Transforming 2-Dimensional Coordinate Data}

\label{section:simple:2ddata}

Transformations which inter-relate 2-dimensional coordinate systems are
common in graphical and image-processing applications, so a set of routines
is provided for applying the associated \mbox{\{$2 \rightarrow 2$\}}
mappings to coordinate data.
These routines have names of the form \name{TRN\_TR2x}, where x is either
\textbf{I}, \textbf{R} or \textbf{D} according to whether the data are
specified by
\emph{integer}, \emph{real} or \emph{double precision} values respectively.
Thus, to apply a compiled \mbox{\{$2 \rightarrow 2$\}} mapping to a set of
2-dimensional data points specified by two lists of \emph{real} coordinates
$X_{IN}$ and $Y_{IN}$, the following statement might be used:

\begin{quote}
\begin{verbatim}
CALL TRN_TR2R( BAD, NXY, XIN, YIN, ID, XOUT, YOUT, STATUS )
\end{verbatim}
\end{quote}

where:

\begin{itemize}

\item \fortvar{BAD} is a logical value specifying whether the input
coordinates may contain \emph{bad} values (see
Section~\ref{section:additional:badvalues});

\item \fortvar{NXY} specifies the number of data points to be transformed;

\item \fortvar{XIN} and \fortvar{YIN} are 1-dimensional arrays containing
the $X_{IN}$ and $Y_{IN}$ coordinates of the input data points;

\item \fortvar{ID} is an identifier associated with the compiled
\mbox{\{$2 \rightarrow 2$\}} mapping to be applied;

\item \fortvar{XOUT} and \fortvar{YOUT} are 1-dimensional arrays to receive
the (transformed) $X_{OUT}$ and $Y_{OUT}$ coordinates of the output data
points;

\item \fortvar{STATUS} is an inherited error status variable.

\end{itemize}

\example{Transforming a cursor position.}
The following shows \name{TRN\_TR2R} being used to transform 2-dimensional
coordinate data obtained by reading a cursor.
It is assumed that the identifier \fortvar{ID} is associated with a compiled
\mbox{\{$2 \rightarrow 2$\}} mapping which has been established to convert
cursor coordinates into ``user'' coordinates:

\begin{quote}
\begin{verbatim}
*  Declare variables.
      INTEGER STATUS, N, ID
      REAL XCUR( 1 ), YCUR( 1 ), XUSER( 1 ), YUSER( 1 )

*  Set up cursor choice device.
      CALL SGS_SELCH( 0 )
      CALL SGS_DEFCH( '.' )

*  Loop to read cursor positions.
      DO WHILE ( STATUS .EQ. SAI__OK )
         CALL SGS_REQCU( XCUR( 1 ), YCUR( 1 ), N )      [1]
         IF( N .NE. 0 ) GO TO 1

*  Transform the cursor coordinates.
         CALL TRN_TR2R( .FALSE., 1, XCUR, YCUR, ID,     [2]
     :                  XUSER, YUSER, STATUS )

*  Display the position in user coordinates.
         CALL MSG_SETR( 'X', XUSER( 1 ) )
         CALL MSG_SETR( 'Y', YUSER( 1 ) )
         CALL MSG_OUT( ' ', 'X = ^X ; Y = ^Y', STATUS ) [3]
      ENDDO
    1 CONTINUE

\end{verbatim}
\end{quote}

Programming notes:

\begin{enumerate}

\item The \name{SGS} cursor position is read repeatedly until the user
terminates interaction (by pressing the keyboard `.' key in this case).

\item \name{TRN\_TR2R} is used to transform each cursor position in turn
into ``user'' coordinates.
Only a single data point is transformed here, but in a non-interactive
application it would be more efficient to transform a whole set of data
points in a single call to this routine.

\item The transformed positions are displayed in ``user'' coordinates.

\end{enumerate}
\exampledone


\subsection{\xlabel{clearing_up_and_closing_down}Clearing Up and Closing Down}

\label{section:simple:cleanup}

When a compiled mapping is no longer required, the resources associated with
it should be released.
This is done by calling \name{TRN\_ANNUL} (annul compiled mapping), thus:

\begin{quote}
\begin{verbatim}
CALL TRN_ANNUL( ID, STATUS )
\end{verbatim}
\end{quote}

This causes the compiled mapping to be deleted and the \fortvar{ID} value
to be reset to \name{TRN\_\_NOID}.\footnote{
The symbolic constant \name{TRN\_\_NOID} is defined in the include file
\name{TRN\_PAR}.
It is provided as a ``null'' value, which is guaranteed never to be used as
a compiled mapping identifier.
It may be used, when necessary, to indicate that an identifier is not
currently associated with a compiled mapping.}
Finally, before an application finishes, the statement:

\begin{quote}
\begin{verbatim}
CALL TRN_CLOSE( STATUS )
\end{verbatim}
\end{quote}

should be executed.
Calling \name{TRN\_CLOSE} will first annul any compiled mappings which are
still active and will then close the \name{TRANSFORM} facility down,
recovering any resources associated with it.

Both of these cleaning up routines will attempt to execute regardless of the
value of their \fortvar{STATUS} argument.


\section{\xlabel{additional_features}Additional Features}

\subsection{\xlabel{transforming_general_coordinate_data}Transforming General Coordinate Data}

\label{section:additional:generaldata}

As well as providing routines for applying compiled mappings to 1- and
2-dimensional data (Sections~\ref{section:simple:1ddata} \&
\ref{section:simple:2ddata}), the \name{TRANSFORM} facility also has a set
of routines for applying more general mappings.
These are appropriate, for instance, when the number of input or output data
coordinates is greater than 2 (or when these numbers are unequal) or when
the number of coordinates is not known in advance.
In such cases, all the input and output coordinates must each reside in
single data arrays.

These general routines have names of the form \name{TRN\_TRNx}, where x is
either \textbf{I}, \textbf{R} or \textbf{D} according to whether the data are
specified by \emph{integer}, \emph{real} or \emph{double precision} values
respectively.
Thus, to transform a general set of data points specified by an array of
{\em real} coordinates, the routine \name{TRN\_TRNR} would be used, thus:

\begin{quote}\small
\begin{verbatim}
CALL TRN_TRNR( BAD, ND1, NCIN, NDAT, DATA, ID, NR1, NCOUT, RESULT, STATUS )
\end{verbatim}
\end{quote}

where:

\begin{itemize}

\item \fortvar{BAD} is a logical value specifying whether the input
coordinates may contain \emph{bad} values (see
Section~\ref{section:additional:badvalues});

\item \fortvar{ND1} specifies the first dimension of the \fortvar{DATA}
array;

\item \fortvar{NCIN} specifies the number of coordinates per input data
point;

\item \fortvar{NDAT} specifies the number of data points to be transformed;

\item \fortvar{DATA} is a 2-dimensional array containing a list of
coordinates for the input data points;

\item \fortvar{ID} is an identifier associated with the compiled
\mbox{\{$NCIN \rightarrow NCOUT$\}} mapping to be applied;

\item \fortvar{NR1} specifies the first dimension of the \fortvar{RESULT}
array;

\item \fortvar{NCOUT} specifies the number of coordinates per output data
point;

\item \fortvar{RESULT} is a 2-dimensional array to receive a list of
(transformed) coordinates for the output data points;

\item \fortvar{STATUS} is an inherited error status variable.

\end{itemize}

\example{Mapping 3-dimensional coordinates into 2 dimensions.}
In the following, a set of data points representing a 3-dimensional object
is converted into a related set of 2-dimensional points using
\name{TRN\_TRNR}.
The resulting points are then plotted using \name{GKS}.
With a suitable choice of mapping, such an arrangement might be used to
generate perspective pictures of 3-dimensional objects:

\begin{quote}
\begin{verbatim}
*  Define dimensions of the DATA and RESULT arrays.
      PARAMETER ( MAXPTS = 5000, MAXDIM = 3 )

*  Declare variables and arrays.
      INTEGER IPOINT, STATUS, ID
      REAL X, Y, Z, DATA( MAXPTS, MAXDIM ), RESULT( MAXPTS, MAXDIM )

*  Generate 3-d coordinates of helix.
      DO 1 IPOINT = 1, 3601
         THETA = REAL( IPOINT - 1 ) * 2.0 * 3.14159 / 360.0
         X = COS( THETA )                              [1]
         Y = SIN( THETA )
         Z = REAL( IPOINT - 1 ) / 360.0

*  Enter coordinates into the DATA array.
         DATA( IPOINT, 1 ) = X                         [2]
         DATA( IPOINT, 2 ) = Y
         DATA( IPOINT, 3 ) = Z
    1 CONTINUE

*  Transform the data points.
      CALL TRN_TRNR( .FALSE., MAXPTS, 3, 3601, DATA, ID, MAXPTS, 2,
     :               RESULT, STATUS )                  [3]

*  Plot the resulting 2-dimensional data points.
      CALL GPL( 3601, RESULT( 1, 1 ), RESULT( 1, 2 ) ) [4]

\end{verbatim}
\end{quote}

Programming notes:

\begin{enumerate}

\item The coordinates of a set of 3-dimensional data points are generated.

\item The coordinates are stored so that \fortvar{DATA( I, J )} contains the
\fortvar{J'th} coordinate of data point \fortvar{I}.
These coordinates need not fill the entire array.

\item \name{TRN\_TRNR} is called to apply the compiled mapping.
The size of the input and output coordinate arrays are specified, together
with the numbers of coordinates associated with each input and output data
point.
An error will result if the compiled mapping does not have the appropriate
numbers of input and output variables.

\item The 2-dimensional output data points are plotted.
Their coordinates are stored in the \fortvar{RESULT} array in a similar
manner to those in the \fortvar{DATA} array.
As before, these coordinates need not entirely fill the \fortvar{RESULT}
array.

\end{enumerate}
\exampledone


\subsection{\xlabel{handling_of_bad_coordinate_values}Handling of \emph{Bad} Coordinate Values}

\label{section:additional:badvalues}

During the process of transforming coordinate data, numerical errors (such
as division by zero or overflow) will inevitably occur from time to time.
There is no need to include specific protection against this when a
transformation is formulated, however, because such errors will
automatically be trapped and converted into one of the standard Starlink
{\em bad} values.\footnote{
The Starlink \emph{bad} values are symbolic constants with names of the form
\name{VAL\_\_BADx}, where x is either \textbf{D}, \textbf{R}, \textbf{I},
\textbf{W}, \textbf{UW}, \textbf{B} or \textbf{UB} according to the data
type in question.
These constants are defined in the include file \name{PRM\_PAR} (see
\xref{SGP/38}{sgp38}{}
\& \xref{SUN/39}{sun39}{} for further information).}

This form of error trapping is performed by all the routines which apply
compiled mappings to coordinate data.
The resulting ``\emph{bad} coordinates'' will, where necessary, be propagated
through each stage in the evaluation of a compiled mapping, so that all the
results which have been affected by numerical errors can later be
identified.
No error report is currently made (or \fortvar{STATUS} value set) if a
numerical error of this nature occurs.

All the routines which apply compiled mappings to coordinate data can also
recognise \emph{bad} coordinate values supplied as input, and will propagate
these if required.
Frequently, however, there will not be any \emph{bad} coordinates in the
input data stream and it may be possible to save appreciable amounts of
processing time by disabling recognition of these values (thereby
eliminating unnecessary checking), especially when large arrays are being
processed.
Each relevant routine therefore carries a \emph{logical} argument called
\fortvar{BAD}, which specifies whether recognition of \emph{bad} input data
is required (execution will generally be faster if \fortvar{BAD} is
\name{.FALSE.}).
Note that this argument only affects recognition of \emph{bad} data in the
\textbf{input} stream, while numerical errors which occur during the
computation are always converted into \emph{bad} values and correctly
propagated to the output.


\subsection{\xlabel{concatenating_transformations}Concatenating Transformations}

\label{section:additional:concatenate}

It is common to find that the relationship between two coordinate systems is
most easily expressed in terms of several transformations applied in
succession.
For instance, the transformation between positions on the sky and the pixel
coordinates of a \name{CCD} image might be divided into two stages; the
first representing the effect of imaging the sky into the focal plane of the
telescope, and the second taking account of the position, size and
orientation of the detector in the focal plane.
The \name{TRANSFORM} facility makes explicit provision for cases such as
this by allowing transformations to be \emph{concatenated}.

\begin{figure}
\begin{center}

\begin{tabular}{c}

$\begin{array}{c}

\left\{ \begin{array}{c} x_1 \\ x_2 \end{array} \right\}

\begin{array}{c}

\mbox{Forward} \longrightarrow \\

\mbox{
\setlength{\unitlength}{1mm}
\thicklines
\begin{picture}(40, 15)(-20,-6.5)
\put(-15,-5){\framebox(30,10){\begin{tabular}{c}{\textbf{Transformation}}\\
                                                {\textbf{1}}
                              \end{tabular}}}
\multiput(-15,-2)(0,4){2}{\line(-1,0){3}}
\multiput(-18,-2)(0,4){2}{\circle*{1}}
\multiput(15,-2.5)(0,2.5){3}{\line(1,0){3}}
\multiput(18,-2.5)(0,2.5){3}{\circle*{1}}
\end{picture}
}\\

\longleftarrow \mbox{Inverse}

\end{array}


\left\{ \begin{array}{c} y_1 \leftrightarrow X_1 \\
                         y_2 \leftrightarrow X_2 \\
                         y_3 \leftrightarrow X_3 \end{array} \right\}

\begin{array}{c}

\mbox{Forward} \longrightarrow \\

\mbox{
\setlength{\unitlength}{1mm}
\thicklines
\begin{picture}(40, 15)(-20,-6.5)
\put(-15,-5){\framebox(30,10){\begin{tabular}{c}{\textbf{Transformation}}\\
                                                {\textbf{2}}
                              \end{tabular}}}
\multiput(-15,-2.5)(0,2.5){3}{\line(-1,0){3}}
\multiput(-18,-2.5)(0,2.5){3}{\circle*{1}}
\multiput(15,0)(0,0){1}{\line(1,0){3}}
\multiput(18,0)(0,0){1}{\circle*{1}}
\end{picture}
}\\

\longleftarrow \mbox{Inverse}

\end{array}

\left\{ \begin{array}{c} Y_1 \end{array} \right\}

\end{array}$

\\
\setlength{\unitlength}{1mm}
\begin{picture}(22,20)(-11,-10)
\put(0,0){\oval(22,5)}
\put(0,0){\makebox(0,0){\emph{Concatenate}}}
\put(0,2.5){\line(0,1){5}}
\put(0,-2.5){\vector(0,-1){5}}
\end{picture}
\\

$\begin{array}{c}

\mbox{Forward} \longrightarrow \\

\left\{ \begin{array}{c} x_1 \\ x_2 \end{array} \right\}

\begin{array}{c}
\mbox{
\setlength{\unitlength}{1mm}
\thicklines
\begin{picture}(80, 25)(-40,-10)
\put(-35,-9.5){\framebox(70,22){}}
\put(0,9.5){\makebox(0,0){\textbf{New Transformation}}}
\thinlines
\put(-32.5,-5){\dashbox{1.25}(30,10){\begin{tabular}{c}{Transformation}\\
                                                       {1}
                                     \end{tabular}}}
\multiput(-37.5,-2)(0,4){2}{\line(1,0){5}}
\multiput(-37.5,-2)(0,4){2}{\circle*{1}}
\multiput(-2.5,-2.5)(0,2.5){3}{\line(1,0){2.5}}
\multiput(0,-2.5)(0,2.5){3}{\circle*{1}}
\multiput(0,-2.5)(0,2.5){3}{\line(1,0){2.5}}
\put(2.5,-5){\dashbox{1.25}(30,10){\begin{tabular}{c}{Transformation}\\
                                                     {2}
                                   \end{tabular}}}
\multiput(32.5,0)(0,0){1}{\line(1,0){5}}
\multiput(37.5,0)(0,0){1}{\circle*{1}}
\end{picture}
}\\

\longleftarrow \mbox{Inverse}

\end{array}

\left\{ \begin{array}{c} Y_1 \end{array} \right\}

\end{array}$

\end{tabular}

\caption{Concatenating two transformations.}
\label{figure:concatenate}

\end{center}
\end{figure}

A rather general case of concatenation is illustrated in
Figure~\ref{figure:concatenate}.
In this example, \emph{Transformation 1} relates two input variables
\mbox{$(x_1,x_2)$} to three ``intermediate'' variables
\mbox{$(y_1,y_2,y_3)$}, which are also the input variables
\mbox{$(X_1,X_2,X_3)$} of \emph{Transformation 2}.
This transformation, in turn, has a single final output variable
\mbox{$(Y_1)$}.
The concatenation process involves eliminating the three intermediate
variables and storing the two transformation definitions together in a
single new transformation.
This new transformation then has two input variables \mbox{$(x_1,x_2)$} and
a single output variable \mbox{$(Y_1)$}.
Using the `.' symbol to represent concatenation, this entire process may be
summarised as:
\begin{quote}
\begin{center}
$[2 \leftrightarrow 3].[3 \leftrightarrow 1] = [2 \leftrightarrow 1]$
\end{center}
\end{quote}
or, in general:
\begin{quote}
\begin{center}
$[i \leftrightarrow j].[j \leftrightarrow k] = [i \leftrightarrow k]$
\end{center}
\end{quote}
Note that the number of output variables from the first transformation must
be equal to the number of input variables to the second transformation.
For obvious reasons, also, it is not permitted to concatenate a
transformation in which only the forward mapping is defined with another in
which only the inverse mapping is defined (\emph{e.g.}\ \mbox{[$i \rightarrow
j$]} could not be concatenated with \mbox{[$j \leftarrow k$]}).

The result of concatenating two transformations is itself a transformation,
so the process may be repeated indefinitely, making it possible for a whole
sequence of transformations to be joined together and processed as a single
unit.
Once transformations have been combined in this way, however, they cannot
later be separated.

\paragraph{Performing concatenation.}
Concatenation of a pair of transformations is performed by the routine
\name{TRN\_JOIN} (concatenate transformations), as follows:

\begin{quote}
\begin{verbatim}
CALL TRN_JOIN( LOCTR1, LOCTR2, ELOC, NAME, LOCNEW, STATUS )
\end{verbatim}
\end{quote}

where:

\begin{itemize}

\item \fortvar{LOCTR1} and \fortvar{LOCTR2} are locators to two existing
transformations;

\item \fortvar{ELOC} is a locator to an existing \name{HDS} structure;

\item \fortvar{NAME} is the \name{HDS} name of the new structure component
to be created;

\item \fortvar{LOCNEW} returns a locator to the newly created transformation
structure;

\item \fortvar{STATUS} is an inherited error status variable.

\end{itemize}

As with the \name{TRN\_NEW} routine
(Section~\ref{section:simple:temporary}), a temporary transformation may
also be produced by leaving the \fortvar{ELOC} argument (and optionally the
\fortvar{NAME} argument) blank when \name{TRN\_JOIN} is invoked.


\subsection{\xlabel{prefixing_and_appending_transformations}Prefixing and Appending Transformations}

\label{section:additional:prefix}

In addition to producing new transformations by concatenating existing ones,
it is also possible to modify an existing transformation by \emph{prefixing}
or \emph{appending} another one to it.
For example, to prefix one transformation to another, \name{TRN\_PRFX}
(prefix transformation) would be used, thus:

\begin{quote}
\begin{verbatim}
CALL TRN_PRFX( LOCTR1, LOCTR2, STATUS )
\end{verbatim}
\end{quote}

In this case, the two transformations with locators \fortvar{LOCTR1} and
\fortvar{LOCTR2} are concatenated (as above), but the resultant
transformation replaces the second one, so that the first transformation is,
in effect, prefixed to it.
The first transformation itself is not altered.

The routine \name{TRN\_APND} (append transformation) behaves in a similar
way, except that it appends the second transformation to the first. In this
case the second transformation remains unchanged.

\example{Adjusting an astrometric calibration.}
Suppose \fortvar{LOCTRS} is a locator to a transformation which relates the
pixel coordinates \mbox{$(x,y)$} of a \name{CCD} image to positions on the
sky \mbox{$(\alpha,\delta)$}.
If the size of this image is altered during data reduction, then its pixel
coordinates will change and the astrometric calibration will need adjustment
to take account of this.

If the $x$ and $y$ pixel coordinates are reduced by (say) 19 and 25 units,
then a transformation inter-relating the old and new pixel coordinates could
be formulated as follows:

\begin{equation}
\begin{array}{cc}

\mbox{Forward } \left\{
\begin{array}{lll}
x_{old} & = & x_{new} + 19 \\
y_{old} & = & y_{new} + 25
\end{array}
\right.
&
\mbox{Inverse } \left\{
\begin{array}{lll}
x_{new} & = x_{old} - 19 \\
y_{new} & = y_{old} - 25
\end{array}
\right.

\end{array}
\end{equation}

and then created.
This transformation (with locator \fortvar{LOCTRA}, say) represents the
necessary adjustment which should be applied as a prefix to the original
transformation, thus:

\begin{quote}
\begin{verbatim}
CALL TRN_PRFX( LOCTRA, LOCTRS, STATUS )
\end{verbatim}
\end{quote}

The modified transformation (\fortvar{LOCTRS}\/) will then correctly relate
the image's \textbf{new} pixel coordinates to positions on the sky, the
necessary adjustment having been achieved through the following
concatenation process:

\begin{quote}
\begin{center}
$[(x_{new},y_{new}) \leftrightarrow (x_{old},y_{old})].[(x_{old},y_{old})
\leftrightarrow (\alpha,\delta)] = [(x_{new},y_{new}) \leftrightarrow
(\alpha,\delta)]$
\end{center}
\end{quote}

Note that it is not necessary to know anything about the nature of the
original transformation \mbox{[$(x_{old},y_{old}) \leftrightarrow
(\alpha,\delta)$]} in order to apply such a correction.
\exampledone


\subsection{\xlabel{inverting_transformations}Inverting Transformations}

\label{section:additional:inverting}

\emph{Inversion} of a transformation is a straightforward process involving
the inter-change of the forward and inverse mappings (the numbers of input
and output variables will also be inter-changed).
It is performed by \name{TRN\_INV} (invert transformation), thus:

\begin{quote}
\begin{verbatim}
CALL TRN_INV( LOCTR, STATUS )
\end{verbatim}
\end{quote}

where \fortvar{LOCTR} is a locator to the transformation to be inverted.
After such a call, compiling the transformation in the \textbf{forward}
direction will yield the same compiled mapping as would have been obtained
by compiling in the \textbf{ inverse} direction prior to calling
\name{TRN\_INV}.

The main use for \name{TRN\_INV} is in adapting transformations which have
been created or supplied the ``wrong way round'' for some purpose
(\emph{e.g.}\ concatenation with another transformation).


\subsection{\xlabel{formatting_transformation_functions}Formatting Transformation Functions}

\label{section:additional:formatting}

To simplify the process of formatting transformation functions, a set of
routines is provided which allows numerical or textual values to be inserted
into ``template'' character strings.
These routines largely eliminate the need to use \name{Fortran} \name{WRITE}
statements to construct transformation functions containing numerical
values.

The routines have names of the form \name{TRN\_STOK[x]}, where x is either
\textbf{I}, \textbf{R} or \textbf{D} according to whether the type of value
to be inserted is \emph{integer}, \emph{real} or \emph{double precision}
respectively.
The routine \name{TRN\_STOK} (\emph{i.e.}\ with x omitted altogether) is used
for making textual (rather than numerical) insertions but otherwise
functions in the same way.
These routines rely on the concept of a \emph{token},\footnote{
Tokens take the same form as the variable names used in transformation
functions -- \emph{i.e.}\ they may contain only alpha-numeric characters
(including underscore) and must begin with an alphabetic character.
Tokens may be of any length and may use mixed case (token substitution is
not case-sensitive) but embedded blanks are not allowed.}
which is placed in a character string at the point where an insertion is to
be made and which is subsequently replaced by the value to be inserted.
For example, if the character variable \fortvar{TEXT} had the value:

\begin{quote}
\begin{verbatim}
'magnitude = - 2.5 * log10( count ) + zero_point'
\end{verbatim}
\end{quote}

then \name{TRN\_STOKR} (substitute a {\em real} token value) could be used
to substitute a numerical value for the `\verb#zero_point#' token, as
follows:

\begin{quote}
\begin{verbatim}
CALL TRN_STOKR( 'zero_point', 17.7, TEXT, NSUBS, STATUS )
\end{verbatim}
\end{quote}

This causes all occurrences of the `\verb#zero_point#' token  to be replaced
with the formatted {\em real} number `\verb#17.7#', while \fortvar{NSUBS}
returns the number of substitutions made (in this case there will only be
one).
The value of \fortvar{TEXT} then becomes:

\begin{quote}
\begin{verbatim}
'magnitude = - 2.5 * log10( count ) + 17.7'
\end{verbatim}
\end{quote}

Note that the \name{TRN\_STOKx} routines will respect the syntax of
transformation functions, so that negative numerical values will be enclosed
in parentheses before a substitution is made.
For instance, if the call to \name{TRN\_STOKR} had been:

\begin{quote}
\begin{verbatim}
CALL TRN_STOKR( 'zero_point', -17.7, TEXT, NSUBS, STATUS )
\end{verbatim}
\end{quote}

then the `\verb#zero_point#' token would be replaced by `\verb#(-17.7)#' to
prevent the illegal expression \mbox{`\ldots \texttt{+ -17.7}'} from being
produced.
The need for two extra characters to accommodate these parentheses must be
remembered when declaring the size of character strings.

\example{A ``packaged'' transformation.}
This illustrates the use of \name{TRN\_STOKR} in a subroutine to
``package up'' the creation of a simple \mbox{[$1 \leftrightarrow 1$]}
transformation which contains several adjustable numerical parameters.
The same principles can be followed to write routines for creating a variety
of specialised transformations.

\begin{quote}
\begin{verbatim}
      SUBROUTINE LINEAR( SCALE, ZERO, ELOC, NAME, LOCTR, STATUS )

*  Declare variables.
      INCLUDE 'SAE_PAR'
      INTEGER STATUS, NSUBS
      REAL SCALE, ZERO, INVSCL
      CHARACTER * ( * ) ELOC, NAME, LOCTR
      CHARACTER FOR( 1 ) * 80, INV( 1 ) * 80

*  Check STATUS.
      IF ( STATUS .NE. SAI__OK ) RETURN

*  Formulate the forward transformation function.
      FOR( 1 ) = 'out = ( in - zero ) * scale'                 [1]
      CALL TRN_STOKR( 'zero', ZERO, FOR( 1 ), NSUBS, STATUS )
      CALL TRN_STOKR( 'scale', SCALE, FOR( 1 ), NSUBS, STATUS )

*  If possible, formulate the inverse transformation function.
      IF( SCALE .NE. 0.0 ) THEN
         INVSCL = 1.0 / SCALE
         INV( 1 ) = 'in = ( out * inv_scale ) + zero'          [2]
         CALL TRN_STOKR( 'zero', ZERO, INV( 1 ), NSUBS, STATUS )
         CALL TRN_STOKR( 'inv_scale', INVSCL, INV( 1 ), NSUBS, STATUS )

*  Inverse is undefined...
      ELSE
         INV( 1 ) = 'in'                                       [3]
      ENDIF

*  Create the transformation.
      CALL TRN_NEW( 1, 1, FOR, INV, '_REAL:',                  [4]
     :              'Shift and linear scaling: in --> out',
     :              ELOC, NAME, LOCTR, STATUS )

      END

\end{verbatim}
\end{quote}

Programming notes:

\begin{enumerate}

\item The forward transformation function is assigned and the numerical
parameters \fortvar{SCALE} and \fortvar{ZERO} are substituted into it.

\item If possible, the inverse transformation function is defined similarly.
Note that the reciprocal of \fortvar{SCALE} is taken so that multiplication
may be used in preference to the less efficient division operation.

\item If \fortvar{SCALE} is zero, then the inverse mapping cannot be defined
so the inverse transformation function is assigned a value to indicate this.

\item The transformation is created.

\end{enumerate}
\exampledone

\paragraph{Token delimiters.}
Tokens should normally be delimited from surrounding text (such as variable
names) by non-alphanumeric characters, otherwise they will not be
recognised.
The syntax of transformation functions ensures that this will normally be
so, but, when this is not possible, enclosing angle brackets `$<$\ldots $>$'
may also be used as token delimiters.
In this case the brackets will be replaced as if they were part of the token
itself.
By this means, values may be substituted to form part of a variable name if
required.

\example{Substituting values into variable names.}
This simple example formulates a set of transformation functions to multiply
each ordinate of a 3-dimensional coordinate system \mbox{$(X_1,X_2,X_3)$} by
2.0 to yield the coordinates \mbox{$(Y_1,Y_2,Y_3)$}:

\begin{quote}
\begin{verbatim}
       INTEGER I, NSUBS, STATUS
       CHARACTER FOR( 3 ) * 30

       DO 1 I = 1, 3
          FOR( I ) = 'Y<I> = X<I> * 2.0'
          CALL TRN_STOKI( 'I', I, FOR( I ), NSUBS, STATUS )
     1 CONTINUE
\end{verbatim}
\end{quote}
\exampledone

\paragraph{Recursive substitution.}
The routine \name{TRN\_STOK}, which allows text to be substituted in place
of a token, admits some more sophisticated possibilities, including the
insertion of text which itself contains tokens (perhaps including further
instances of the original token).
Although the substitution performed by a single call to \name{TRN\_STOK}
will not be affected by the presence of tokens within the substituted text
(\emph{i.e.}\ the substitution is not recursive), the routine may nevertheless
be invoked repeatedly to perform recursive substitution if required.
This capability can be used to construct more complex expressions which have
a suitable recursively defined form.

\example{Constructing a polynomial expression.}
The following constructs an expression representing a polynomial in $X$ with
an arbitrary number of numerical coefficients:

\begin{quote}
\begin{verbatim}
*  Declare variables.
      INTEGER NCOEFF, STATUS, I, NSUBS
      REAL COEFF( NCOEFF )
      CHARACTER * ( * ) EXPRS

*  Initialise the expression.
      EXPRS = '<next_term><coeff>'                     [1]

*  Substitute each coefficient value.
      DO 1 I = 1, NCOEFF
         CALL TRN_STOKR( 'coeff', COEFF( I ), EXPRS,   [2]
     :                   NSUBS, STATUS )

*  Expand the token representing the next term.
         IF( I .NE. NCOEFF ) THEN
            CALL TRN_STOK( 'next_term',                [3]
     :                     '(<next_term><coeff>)*X+',
     :                     EXPRS, NSUBS, STATUS )

*  Eliminate the final '<next_term>' token.
         ELSE
            CALL TRN_STOK( 'next_term', ' ',           [4]
     :                     EXPRS, NSUBS, STATUS )
         ENDIF
   1  CONTINUE

\end{verbatim}
\end{quote}

Programming notes:

\begin{enumerate}

\item The character variable \fortvar{EXPRS}, which is to contain the
expression, is initialised to the value `\verb#<next_term><coeff>#'.

\item The `\verb#<coeff>#' token is substituted with a coefficient
value (say 0.1) to give `\verb#<next_term>0.1#'.

\item The `\verb#<next_term>#' token is expanded to give
`\verb#(<next_term><coeff>)*X+0.1#'.
The process is then repeated to replace the `\verb#<next_term><coeff>#'
part of this expression using the value of the next coefficient.

\item After the last coefficient value has been substituted the remaining
`\verb#<next_term>#' token is eliminated by replacing it with a blank
string.

\end{enumerate}

The effect of this algorithm with (say) the 5 polynomial coefficients 0.1,
0.12, 1.06, $-$4.4 \& 1E$-$7, would be to produce the following expression:

\begin{quote}
\begin{verbatim}
'((((1E-7)*X+(-4.4))*X+1.06)*X+0.12)*X+0.1'
\end{verbatim}
\end{quote}

which casts the polynomial into a form for efficient evaluation using
Horner's method.
\exampledone


\section{\xlabel{more_advanced_topics}More Advanced Topics}


\subsection{\xlabel{classifying_transformations}Classifying Transformations}

\label{section:advanced:classification}

As well as holding mapping information describing how to convert from one
coordinate system to another, a transformation may also carry information
about the character of its mappings.
When plotting a graph, for instance, it may be important to know whether the
mapping being used is linear, because a more complex (and costly) algorithm
may be required if it is not.
Indeed, in many circumstances, the absence of a vital property such as
linearity could actually make it impossible for an application to proceed,
in which case it must issue an appropriate error message and abort.

There are a number of mapping characteristics, in addition to linearity,
which can influence or simplify the coding of applications in this way.
However, it is often difficult to determine by indirect means (such as
transforming test points) whether the necessary special properties are
present.
Provision has therefore been made for handling this type of information
explicitly.

\paragraph{Classification properties.}
Usually, information about any special characteristics which are present is
directly available only to the application which creates a transformation.
Consequently, this application should be responsible for declaring that such
properties are present, so that other applications may subsequently enquire
about them.
This process is termed \emph{classification} and the \emph{basic classification
properties} which may be declared are indicated in
Table~\ref{table:classification}, where each is briefly described.
These properties are defined more precisely (and mathematically) in
Appendix~\ref{appendix:classification}.

\begin{table}[h]
\begin{center}

\begin{tabular}{|l|l|l|}
\hline
\emph{Property} & \emph{Brief Description} & \emph{Symbolic Constant}
\\ \hline
LINEAR & Preserves straight lines. & TRN\_\_LIN \\
INDEPENDENT & Preserves the independence of the axes. & TRN\_\_INDEP \\
DIAGONAL & Preserves the axes themselves. & TRN\_\_DIAG \\
ISOTROPIC & Preserves angles and shapes (\emph{e.g.}\ circles). & TRN\_\_ISOT \\
POSITIVE\_DET & A component of reflection is absent. & TRN\_\_POSDT \\
NEGATIVE\_DET & A component of reflection is present. & TRN\_\_NEGDT \\
CONSTANT\_DET & The area (or volume) scale factor is constant. & TRN\_\_CONDT \\
UNIT\_DET & Areas (or volumes) are preserved. & TRN\_\_UNIDT \\
\hline
\end{tabular}

\caption{The basic classification properties which may be declared for a
transformation and the symbolic constants associated with each.
The constants are defined in the include file \name{TRN\_PAR}.}

\label{table:classification}

\end{center}
\end{table}

In most situations, a transformations's mappings are not adequately
described by any one of the basic properties alone, but require a
\emph{composite} classification comprising a set of several of these
properties.  For instance, the combination:

\begin{quote}
\begin{center}

\name{LINEAR} \emph{and} \name{ISOTROPIC} \emph{and} \name{POSITIVE\_DET}
\emph{and} \name{UNIT\_DET}

\end{center}
\end{quote}

would indicate that a mapping represents a rigid rotation about an axis (or a
point in two dimensions).
Facts such as this may not be obvious without some thought, however, so
those classifications which apply to a number of the most common types of
mapping are set out in Table~\ref{table:composite} in
Appendix~\ref{appendix:classification}.

Classification information is conveniently processed in the form of a
1-dimensional \emph{logical classification array}, each of whose elements
indicates the presence or absence of one particular property.
Each basic property therefore has an \emph{integer} symbolic constant
associated with it (see Table~\ref{table:classification}) which identifies
the array element to be used (the precise mechanism is illustrated below).
Not all possible combinations of the basic properties are permitted (see
Appendix~\ref{appendix:classification}) but the number of different
classifications possible is nevertheless still quite large.

\paragraph{Inserting classification information.}
It is important to appreciate that classifying a transformation is not
mandatory, but merely assists other applications in making effective
use of it.
The desirability of classification therefore depends largely on the type of
applications which are likely to process the transformation.
For instance, a comprehensive general-purpose application, which might be
used in conjunction with a wide range of other software, would probably take
care to classify a transformation fully, whereas a simpler application,
perhaps designed for personal use, might not.
However, a declaration of linearity, if it applies, is generally
recommended.

When a transformation is first created using \name{TRN\_NEW} it has no
classification information associated with it, so this information has to be
inserted explicitly.
This is done by calling \name{TRN\_PTCL} (put classification), as follows:

\example{Inserting classification information.}

\begin{quote}
\begin{verbatim}

*  Declare variables.
      INCLUDE 'TRN_PAR'
      LOGICAL CLASS( TRN__MXCLS )                      [1]

*  Create a (temporary) transformation.
      CALL TRN_NEW( NVIN, NVOUT, FOR, INV, PREC,       [2]
     :              ' ', ' ', LOCTR, STATUS )

*  Set up the logical classification array.
      DO 1 I = 1, TRN__MXCLS                           [3]
         CLASS( I ) = .FALSE.
    1 CONTINUE
      CLASS( TRN__LIN ) = .TRUE.                       [4]
      CLASS( TRN__DIAG ) = .TRUE.
      CLASS( TRN__ISOT ) = .TRUE.
      CLASS( TRN__POSDT ) = .TRUE.

*  Enter the classification information.
      CALL TRN_PTCL( CLASS, LOCTR, STATUS )            [5]

\end{verbatim}
\end{quote}

Programming notes:

\begin{enumerate}

\item A 1-dimensional {\em logical} array \fortvar{CLASS} is declared with
\fortvar{TRN\_\_MXCLS} elements (this symbolic constant specifies the number
of basic classification properties currently recognised and is defined in
the include file \name{TRN\_PAR}).

\item A transformation is created.
At this point it does not contain any classification information.

\item All elements of the \fortvar{CLASS} array are explicitly initialised
to \name{.FALSE.} -- this precaution is recommended so that the number of
basic classification properties may be increased in future without adversely
affecting existing software.

\item The required array elements are set to \name{.TRUE.} -- in this
example the classification describes a simple linear magnification about a
point with a positive magnification factor (see Table~\ref{table:composite}
in Appendix~\ref{appendix:classification}).
Symbolic constants (also defined in the include file \name{TRN\_PAR}) are
used to identify the array elements concerned.

\item The classification information is entered into the transformation by
calling \name{TRN\_PTCL}.

\end{enumerate}
\exampledone

\paragraph{Retrieving classification information.}
The routine \name{TRN\_GTCL} (get classification) is provided for retrieving
classification information from a transformation so that applications may
enquire about the properties which have been declared.
Unless the context dictates otherwise, all applications which import
transformations should make such an enquiry and should not assume that a
transformation has any special property unless its classification
information indicates that this is so.

\example{Retrieving classification information.}

\begin{quote}
\begin{verbatim}

*  Declare variables.
      INCLUDE 'TRN_PAR'
      LOGICAL CLASS( TRN__MXCLS ), OK                  [1]

*  Get the classification information.
      CALL TRN_GTCL( LOCTR, .TRUE., CLASS, STATUS )    [2]

*  Test for the required properties.
      OK = CLASS( TRN__LIN ) .AND.                     [3]
     :     CLASS( TRN__ISOT )

\end{verbatim}
\end{quote}

Programming notes:

\begin{enumerate}

\item A \emph{logical} array \fortvar{CLASS} is declared with
\fortvar{TRN\_\_MXCLS} elements.

\item \name{TRN\_GTCL} is called to retrieve classification information from
a transformation, returning it in the \fortvar{CLASS} array.
The second (\emph{logical}) argument to \name{TRN\_GTCL} is set \name{.TRUE.}
in this example, which indicates that information is required about the
forward mapping (as opposed to the inverse mapping).
An error would result if this mapping were not defined.

\item The appropriate elements of the \name{CLASS} array are tested to
determine whether the mapping has the required property.
In this case, \name{OK} is set \name{.TRUE.} if the forward mapping is
\name{LINEAR} and \name{ISOTROPIC}.
In two dimensions this would ensure that it preserves straight lines, angles
and shapes (\emph{e.g.}\ circles).

\end{enumerate}
\exampledone

The routine \name{TRN\_GTCLC} (get compiled classification) is also provided
to retrieve classification information from compiled mappings
(Appendix~\ref{appendix_routines}).

\paragraph{Automatic classification processing.}
The \name{TRANSFORM} software provides for a certain amount of automatic
processing to take place whenever an exchange of classification information
occurs.
The simplest form which this takes is \emph{validation}, which will detect
errors such as an attempt to declare an inconsistent set of classification
properties.
In addition, the software is capable of ``filling in'' any properties which
are missing but which can be deduced from the others supplied.
Some instances where properties can be deduced in this way are indicated by
the open circles in Table~\ref{table:composite} -- it would not be
necessary to specify these items explicitly, as \name{TRANSFORM} routines
would supply them automatically.

Automatic processing also takes place whenever transformations are
\emph{concatenated}, \emph{prefixed} or \emph{appended}
(Sections~\ref{section:additional:concatenate} \&
\ref{section:additional:prefix}).
In this case, the classification information available from each of the
contributing transformations is combined to deduce the set of
properties which apply to the result.
Note, however, that only those properties which \textbf{necessarily} follow can
be deduced in this way.
For instance, it would be possible for the non-linearities in two mappings
to cancel when their transformations are concatenated, giving an overall
linear result.
However, since this is not the case in general, it could not be deduced
automatically.
If the automatic classification processing provided should prove inadequate
(for instance, the application may have some additional information
available to it), then a new classification may be derived explicitly and
``re-declared'' by calling \name{TRN\_PTCL}, which will replace any
pre-existing classification information.


\subsection{\xlabel{arithmetic_precision}Arithmetic Precision}

\label{section:advanced:precision}

When a transformation is created, a \emph{precision specification} is
associated with it (\emph{i.e.\ via} the \fortvar{PREC} argument to
\name{TRN\_NEW} -- Section~\ref{section:simple:creating}) and this
subsequently determines the type of arithmetic (\emph{integer}, \emph{real} or
\emph{double precision}) which will be used to evaluate the transformation
functions.
The correct choice of this specification is important if the desired
behaviour is to be achieved.
Two considerations normally apply:

\begin{enumerate}

\item \textbf{The type of calculation being performed.}
For instance, \emph{integer} arithmetic might be required if effects due to
rounding were being exploited, whereas transformation functions representing
a 2-dimensional rotation would probably need to use \emph{real} arithmetic,
even if the coordinate data were to be stored as integers.
Some functions may also require \emph{double precision} arithmetic to reduce
internal rounding errors.

\item \textbf{The type of data being transformed.}
It would clearly be inappropriate to use \emph{real} arithmetic on \emph{double precision} data but, conversely, it would be inefficient to use a
higher precision than was necessary to preserve the accuracy of \emph{real}
data.

\end{enumerate}

It can be seen that an appropriate arithmetic precision cannot necessarily
be selected solely on the basis of the type of calculation to be performed,
because the type of coordinate data to be processed may also be relevant.
Unfortunately, this latter information may not be available to the
application which creates the transformation.
Furthermore, whenever transformations are concatenated
(Section~\ref{section:additional:concatenate}), each must be able to
accommodate data passed to it by neighbouring transformations although it
may not have any advance knowledge of the type of arithmetic its neighbours
will be using.

To accommodate these possibilities without unnecessary loss of data
precision, transformations are allowed some flexibility in adapting their
internal arithmetic to the external data being processed.
This is controlled by the precision specification, which is a character
string taking one of the following six values:

\begin{quote}
\begin{center}
\begin{tabular}{cc}

Fixed precisions $\left\{ \mbox{
\begin{tabular}{l}
`{\tt \_INTEGER}'\\
`{\tt \_REAL}'\\
`{\tt \_DOUBLE}'
\end{tabular} } \right.$
& Elastic precisions $\left\{ \mbox{
\begin{tabular}{l}
`{\tt \_INTEGER:}'\\
`{\tt \_REAL:}'\\
`{\tt \_DOUBLE:}'
\end{tabular} } \right.$

\end{tabular}
\end{center}
\end{quote}

\textbf{Fixed precisions} specify the type of arithmetic to be used explicitly.
In this case, the data used in the calculation (both incoming coordinates
and explicit numerical constants in the transformation functions) are first
converted to the specified numerical type (\emph{integer}, \emph{real} or
\emph{ double precision}) and the transformation functions are then evaluated
using the appropriate type of arithmetic.

\emph{N.B.\ It is currently assumed that the data types of all the input and
output coordinates are the same and all the transformation functions are
therefore evaluated using the same type of arithmetic.
This may change in future to allow coordinates to have mixed data types.
The present arrangement is designed so that changes to existing applications
and datasets will not generally be necessary.}

\textbf{Elastic precisions} are similar, except that some subsequent
adjustment is also allowed.
In this case, the precision specification indicates the \textbf{minimum}
precision of the arithmetic to be used, but this may be increased if the
data type warrants it.
Thus, a precision specification of \mbox{`{\tt \_REAL:}'} would request that
\emph{real} arithmetic be used unless \emph{double precision} data were being
processed (in which case \emph{double precision} arithmetic would be used
instead).
In general, the type of arithmetic used is related to the precision
specification and the type of data being processed as follows:

\begin{quote}
\begin{center}
\begin{tabular}{|c|c|c|c|}
\cline{2-4}
\multicolumn{1}{c}{} & \multicolumn{3}{|c|}{\em Elastic Precision
Specification}\\
\hline
{\em Data Type} & `\verb#_INTEGER:#' & `\verb#_REAL:#' & `\verb#_DOUBLE:#' \\
\hline
\name{\_INTEGER} & {\em integer} & {\em real} & {\em double precision} \\
\name{\_REAL} & {\em real} & {\em real} & {\em double precision} \\
\name{\_DOUBLE} & {\em double precision} & {\em double precision} & {\em
double precision}\\
\hline
\end{tabular}
\end{center}
\end{quote}

Once the arithmetic precision has been decided, the transformation functions
are evaluated in the same way as for a fixed precision (\emph{i.e.}\ by
converting all incoming data and constants to the appropriate data type and
then performing the arithmetic).

When several transformations have been concatenated, the type of arithmetic
to be used is determined individually for each stage in the overall
transformation (data type conversion being performed automatically between
each stage as necessary).
If an elastic precision (which is sensitive to the input data type) has been
specified, then the input data type is taken to correspond with the type of
arithmetic used in the previous mapping to be evaluated (or the input data
type itself if appropriate).
This arrangement avoids any unnecessary loss of data precision regardless of
the order in which transformations are concatenated.


\section{\xlabel{compiling_and_linking}Compiling and Linking}

The files required for compiling and linking applications which use the
\name{TRANSFORM} facility reside in the standard locations -- usually
\texttt{/star/include} for the Fortran \name{INCLUDE} files,
\texttt{/star/lib} for the object library and \texttt{/star/bin} for the
development and link scripts.

Files and routines from the \name{PRIMDAT} facility (\xref{SUN/39}{sun39}{})
are also required.
The \name{INCLUDE} file names used by both of these facilities must first
be defined by executing the commands:

\begin{quote}
\begin{verbatim}
% prm_dev
% trn_dev
\end{verbatim}
\end{quote}

Applications which contain Fortran \name{INCLUDE} statements associated with
the \name{TRANSFORM} facility may then be compiled.

ADAM applications should be linked using the link script
\texttt{trn\_link\_adam}.
This script automatically links in the required \name{PRIMDAT} library as
well as all the other required Starlink libraries.
For instance, to link an application called \name{PROG}, the \name{ADAM}
command:

\begin{quote}
\begin{verbatim}
% alink prog.f -o prog `trn_link_adam`
\end{verbatim}
\end{quote}

should be used.

\cleardoublepage
\appendix

\section{\xlabel{appendix_transformation_functions}Transformation Functions}

\label{appendix_syntax}

This Appendix describes the form and syntax of transformation functions and
the way in which they are used to define transformations.


\subsection{\xlabel{general_form}General Form}

A transformation is defined by specifying two ordered sets of transformation
functions which define its forward and inverse mappings.
These functions are stored as character data (typically in the elements of
two \name{CHARACTER} arrays) and have the general form:

\begin{quote}
$<$variable$>$ [= $<$expression$>$]
\end{quote}

where \mbox{$<$variable$>$} is a valid \emph{variable} name and
\mbox{$<$expression$>$} is an arithmetic \emph{expression}.
Transformation functions therefore resemble Fortran~77 assignment
statements, except that the contents of the square brackets \mbox{[...]} may
be omitted in appropriate circumstances (see below).

\paragraph{Variables and data coordinates.}
Variables are used in transformation functions to represent the coordinates
of data points which will be transformed.
Their names may be chosen freely to make the definition as comprehensible as
possible; they may be of any length and may contain any alphanumeric
character (including underscore) although they must start with an alphabetic
character.

A variable becomes defined when it appears on the left hand side of a
transformation function and the name of each variable must appear on the
left hand side of one (and only one) such function.
Variables which appear on the left hand side of \textbf{forward} transformation
functions are termed \textbf{output} variables, while those appearing on the
left hand side of \textbf{inverse} transformation functions are termed
\textbf{input} variables.

Variables act only as ``dummy arguments'' and their names have no
significance outside the transformation being defined.
The correspondence between variables and the coordinates of external data
points is established by the order in which the variables are defined.
In the following, for instance, the first coordinate of an input data point
would correspond with the variable `\verb#ALPHA#' and the second coordinate
with `\verb#OMEGA#', while the two coordinates of an output data point would
correspond with the variables `\verb#p#' and `\verb#q#' respectively:

\begin{center}
\begin{tabular}{cc}

\mbox{Forward } $\left\{ \mbox{
\begin{tabular}{l}
`\texttt{p = ALPHA + 2}' \\
`\texttt{q = 44 - 2 * OMEGA}' \\
\end{tabular}
}\right.$
&
\mbox{Inverse } $\left\{ \mbox{
\begin{tabular}{l}
`\texttt{ALPHA = p - 2}' \\
`\texttt{OMEGA = ( 44 - q ) / 2}'
\end{tabular}
}\right.$

\end{tabular}
\end{center}

Note that expressions appearing on the right hand side of \textbf{forward}
transformation functions may only refer to \textbf{input} variables, while
those appearing on the right hand side of \textbf{inverse} transformation
functions may only refer to \textbf{output} variables.

\paragraph{Unspecified mappings.}
Either (although not both) of a transformation's mappings may be left
unspecified by omitting the right hand sides (and `$=$' signs) from all the
relevant transformation functions.
Thus, the transformation given by:

\begin{quote}
\begin{center}
\begin{tabular}{cc}

\mbox{Forward } $\left\{ \mbox{
\begin{tabular}{l}
`\texttt{p = ALPHA + 2}' \\
`\texttt{q = 44 - 2 * OMEGA}' \\
\end{tabular}
}\right.$
&
\mbox{Inverse } $\left\{ \mbox{
\begin{tabular}{l}
`\texttt{ALPHA}' \\
`\texttt{OMEGA}'
\end{tabular}
}\right.$

\end{tabular}
\end{center}
\end{quote}

has only the forward mapping defined (the remainder of the inverse functions
serves simply to define the names of the input variables).
To define only the \textbf{inverse} mapping, the right hand sides of all the
\textbf{forward} transformation functions would be omitted instead.


\subsection{\xlabel{expression_syntax}Expression Syntax}

The right hand sides of transformation functions closely follow the syntax
of Fortran~77 arithmetic expressions.
The main differences (in addition to the less restrictive naming rules for
variables) are:

\begin{itemize}

\item Transformation functions may be written in upper, lower or mixed case
and there there is no limit on their length;

\item Only the generic form of intrinsic functions is available, but some
additional built-in functions are provided;

\item The data types of variables and constants are interpreted differently
(see below and Section~\ref{section:advanced:precision}).

\end{itemize}

The following describes particular features of the expression syntax in more
detail.

\paragraph{Constants.}
Numerical constants may be written using any of the standard Fortran~77
forms (\emph{integer}, \emph{real} or \emph{double precision}).
For positive constants, a preceding $+$ sign is optional.
Thus, all the following are valid constants:

\begin{quote}
\begin{center}
\begin{tabular}{ccccccc}

0 & $-$1 & 57 & $+$666 & 1.0 & $+$3. & 0.5438 \\
.303 & $-$.5 & 1.234d6 & $-$4.6e$-$3 & 9E4 & $+$.44D$+$19 & 3e0\\

\end{tabular}
\end{center}
\end{quote}

The special constant `$<$\verb#BAD#$>$' may also be used and represents a
\emph{bad} (\emph{i.e.}\ undefined) value; any expression containing it
evaluates to the standard Starlink \emph{bad} value for the data type being
transformed.

At present there is no distinction between the data types of constants, so
the form in which they are written does not matter and their interpretation
depends only on the type of arithmetic in use when the expression is
evaluated.
This, in turn, may depend on the type of data being transformed
(Section~\ref{section:advanced:precision}) so constant values are converted
automatically to the data type required.
For instance, a constant written as `\mbox{\texttt{2.1}}' might be interpreted as
\emph{integer}~(2), \emph{real}~(2.1E0), or \emph{double precision}~(2.1D0)
according to the type of arithmetic being used.

\emph{N.B. Handling of data types may change in future to allow implicit type
conversion and ``mixed mode'' arithmetic.
To avoid possible problems, floating-point to integer conversion of
constants should currently be avoided if the constant has a fractional part.
In practice such cases are rare.}

\paragraph{Arithmetic operations.}
The standard arithmetic operators $+$, $-$, $*$, $/$ and $**$ are available
and their use is identical to Fortran~77, thus:

\begin{quote}
\begin{center}
\begin{tabular}{lcl}

`\texttt{Q + 3.0}' & \ldots & add 3.0 to Q\\
`\texttt{-6.7 - DATA}' & \dots & subtract \name{DATA} from $-$6.7\\
`\texttt{INPUT * 14}' & \ldots & multiply \name{INPUT} by 14 \\
`\texttt{4 / V2}' & \ldots & divide 4 by \name{V2}\\
`\texttt{NUMBER ** INDEX}' & \ldots & raise \name{NUMBER} to the power
\name{INDEX}

\end{tabular}
\end{center}
\end{quote}

The normal rules of operator precedence apply.
Matching pairs of parentheses may also be used (and are recommended) to
explicitly define the order of expression evaluation; they may be nested
arbitrarily deeply.
Note that the precision with which arithmetic is performed is not determined
until an expression is actually evaluated
(Section~\ref{section:advanced:precision}).\footnote{
Expressions such as `X$**$2.0' and `X$**$2' cannot currently be
distinguished, whereas in Fortran they are different and the latter will
evaluate more efficiently.
Consequently (unless \emph{integer} arithmetic is being used) the
exponentiation operator `$**$' is an inefficient way of squaring a
number and the equivalent expression `X$*$X' is preferred.
This deficiency will be removed in future.}


\paragraph{Boolean operations.}
The standard logical/boolean operators .EQ., .NE., .GT., .LT.,
.GE., .LE., .OR., .AND. and .NOT. are available and their use
is identical to Fortran~77, except that there is no distinct logical data
type. These boolean operators have numerical operands and return numerical
values. A numerical value is considered ``true'' if it is non-zero and
``false'' if it is zero.\footnote{This is the same convention used in
the C programming language.} The above operators return a value of 1 for
``true'' and zero for ``false''.

These operators may also be specified within an expression using the
equivalent C forms:
\verb#==#,
\verb#!=#,
\verb#>#,
\verb#<#,
\verb#>=#,
\verb#<=#,
\verb#||#,
\verb#&&# and
\verb#!#.

These operators can be used with the built-in function ``QIF''. This
function has 3 arguments. It returns the value of its second argument if its
first argument is non-zero (i.e. ``true''), and returns the value of its
third argument otherwise. It can be used like the ``?'' operator in the C
programming language.

\subsection{\xlabel{builtin_functions}Built-in Functions}

\paragraph{Standard functions.}
A set of standard built-in functions is available and corresponds closely
with the Fortran~77 set of intrinsic functions
(Table~\ref{table:standardbuiltins}).
Function invocations take the same form as in Fortran and may be nested
arbitrarily deeply.

These built-in functions are all \emph{generic} and will adapt to the type of
arithmetic being used.
Some, however, do not support \emph{integer} arithmetic and a \emph{bad} value
will be returned if this is in use (see Table~\ref{table:standardbuiltins}).
There are no data type conversion functions at present.

\begin{table}
\begin{center}
\begin{tabular}{|l|c|l|} \hline
\emph{Function} &
\begin{tabular}{c}\emph{Number of} \\ \emph{arguments} \end{tabular} &
\emph{Description }\\
\hline \hline
SQRT & 1 & square root: $\sqrt{\mbox{arg}}$ \\
LOG & 1 & natural logarithm: $\ln (\mbox{arg})$ \\
LOG10 & 1 & common logarithm: $\log _{10}(\mbox{arg})$ \\
EXP & 1 & exponential: $\exp (\mbox{arg})$ \\
ABS & 1 & absolute (positive) value: $\left| \mbox{arg} \right|$ \\
NINT & 1 & nearest integer value to \mbox{arg} \\
MAX & 2 or more & maximum of arguments \\
MIN & 2 or more & minimum of arguments \\
DIM & 2 & Fortran DIM (positive difference) function \\
MOD & 2 & Fortran MOD (remainder) function \\
SIGN & 2 & Fortran SIGN (transfer of sign) function \\
SIN* & 1 & sine function: $\sin (\mbox{arg})$ \\
COS* & 1 & cosine function: $\cos (\mbox{arg})$ \\
TAN* & 1 & tangent function: $\tan (\mbox{arg})$ \\
ASIN* & 1 & inverse sine function: $\sin ^{-1}(\mbox{arg})$ \\
ACOS* & 1 & inverse cosine function: $\cos ^{-1}(\mbox{arg})$ \\
ATAN* & 1 & inverse tangent function: $\tan ^{-1}(\mbox{arg})$ \\
ATAN2* & 2 & Fortran ATAN2 (inverse tangent) function \\
SINH* & 1 & hyperbolic sine function: $\sinh (\mbox{arg})$ \\
COSH* & 1 & hyperbolic cosine function: $\cosh (\mbox{arg})$ \\
TANH* & 1 & hyperbolic tangent function: $\tanh (\mbox{arg})$ \\ \hline
\multicolumn{1}{|r}{} &
\multicolumn{2}{l|}{\footnotesize *Function does not support \emph{integer}
arithmetic.}\\
\hline
\end{tabular}

\end{center}
\caption{The standard built-in functions.  The angular arguments/results of
all trigonometric functions are in radians.}
\label{table:standardbuiltins}
\end{table}

\paragraph{Additional functions.}
Some additional built-in functions, which do not match the standard
Fortran~77 set, are also defined.
At present these simply comprise the \name{VAX}~Fortran extensions to this
set (Table~\ref{table:additionalbuiltins}).
They are also generic but do not support \emph{integer} arithmetic.

\begin{table}
\begin{center}
\begin{tabular}{|l|c|l|} \hline
\emph{Function} &
\begin{tabular}{c} \emph{Number of}\\ \emph{arguments} \end{tabular} &
\emph{Description }\\
\hline \hline
SIND & 1 & sine function: $\sin (\mbox{arg})$ \\
COSD & 1 & cosine function: $\cos (\mbox{arg})$ \\
TAND & 1 & tangent function: $\tan (\mbox{arg})$ \\
ASIND & 1 & inverse sine function: $\sin ^{-1}(\mbox{arg})$ \\
ACOSD & 1 & inverse cosine function: $\cos ^{-1}(\mbox{arg})$ \\
ATAND & 1 & inverse tangent function: $\tan ^{-1}(\mbox{arg})$ \\
ATAN2D & 2 & VAX Fortran ATAN2D (inverse tangent) function \\
\hline
\end{tabular}
\end{center}
\caption{Additional built-in functions; angular arguments/results are in
degrees.
None of these functions supports \emph{integer} arithmetic.}
\label{table:additionalbuiltins}
\end{table}

In addition, the following non-standard functions are provided:

\begin{description}
\item[IDV] - Performs integer division.
It has two arguments. The function value is formed by taking the integer
part of the two arguments, dividing them, and then returning the integer
part of the result.
\item[QIF] - Has three arguments. It returns the value of its second argument
if its first argument is non-zero (i.e. ``true''), and returns the value of
its third argument otherwise. It can be used like the ``?'' operator in the C
programming language.
\end{description}

\newpage
\section{\xlabel{classification_properties}Classification Properties}

\label{appendix:classification}

This Appendix describes the classification properties which may be declared
and associated with a transformation
(Section~\ref{section:advanced:classification} shows how this is done and
how the information may subsequently be retrieved).
In order to be precise, the definitions given here are necessarily
mathematical.
Readers who require simpler and more specific information about how to
classify a particular transformation may find Table~\ref{table:composite}
helpful.


\subsection{\xlabel{general}General}

Many of the properties described here depend on the nature of a \emph{Jacobian
matrix} associated with a transformation; there are potentially two
of these matrices, corresponding with the forward and inverse mappings.
Using the notation of Equation~\ref{equation:transformationfunctions}, the
Jacobian matrix ${\bf J}_F$ associated with the forward mapping is the
\mbox{$n \times m$} matrix of partial derivatives:

\begin{equation}
\textbf{J}_F = \left[ \begin{array}{cccc}
\pdiff{1}{1} & \pdiff{1}{2} & \cdots & \pdiff{1}{m} \\
             &              &        &              \\
\pdiff{2}{1} & \pdiff{2}{2} & \cdots & \pdiff{2}{m} \\
\vdots       & \vdots       &        & \vdots       \\
\pdiff{n}{1} & \pdiff{n}{2} & \cdots & \pdiff{n}{m}
\end{array} \right]
\end{equation}

while that associated with the inverse mapping ${\bf J}_I$ is the equivalent
\mbox{$m \times n$} matrix obtained by inter-changing input and output
variables ($x$ and $y$) throughout.

The significance of these matrices can be seen by considering a simple
linear mapping in two dimensions.
Such a mapping is capable of representing a combination of a shift of
origin, magnification, rotation, reflection and shearing deformation:

\begin{equation}
\begin{array}{lll}
y_1 & = & a x_1 + b x_2 + c \\
y_2 & = & d x_1 + e x_2 + f
\end{array}
\end{equation}

It may be re-written as the matrix equation:

\begin{equation}
\begin{array}{cc}
\left[ \begin{array}{c} y_1 \\ y_2 \end{array} \right]
= \mathbf{J} \times \left[ \begin{array}{c} x_1 \\ x_2 \end{array} \right]
+ \left[ \begin{array}{c} c \\ f \end{array} \right]
&
\mbox{where }
\mathbf{J} = \left[ \begin{array}{cc} a & b \\ d & e \end{array} \right]
\end{array}
\end{equation}

The Jacobian matrix $\mathbf{J}$ therefore contains the coefficients which
define this mapping and determine its character, apart from a shift of
origin.
The determinant of $\mathbf{J}$ (\mbox{\small$\det\mathbf{J}=ae-bd$})
is the signed ``area scale factor'' which the mapping introduces (\emph{i.e.}\
the area of the parallelogram produced when the mapping acts on a unit
square).
In more than two dimensions, $\det\mathbf{J}$ would be the equivalent ``volume
scale factor''.

If the mapping is not linear, then the Jacobian matrix will vary from point to
point.
Nevertheless, it may still be regarded as a local linear approximation to
the true mapping (apart from re-location of the origin) and \mbox{$\det
\textbf{J}$} can still be interpreted as the \textbf{local} area (or volume) scale
factor, which may now change from point to point.

\paragraph{Changing dimensionality.}
For transformations with an equal number of input and output variables
(\mbox{$m=n$}), the Jacobian matrices $\textbf{J}_F$ and $\textbf{J}_I$
associated
with the forward and inverse mappings (if specified) will both be square.
If the transformation functions are correctly formulated, then these two
matrices will be mutually inverse and will satisfy:

\begin{equation}
\textbf{J}_F \textbf{J}_I = \textbf{J}_I \textbf{J}_F = \textbf{I}
\end{equation}

where $\textbf{I}$ is an identity matrix.
Their determinants will also be related by:

\begin{equation}
\det{\textbf{J}_I} = \frac{1}{\det{\textbf{J}_F}}
\end{equation}

As a consequence of this (and the definitions of the basic classification
properties given below) any property which applies to one of a
transformation's two mappings will necessarily apply to the complementary
mapping also.

If the transformation affects a change of dimensionality, however, so that
\mbox{$m \ne n$}, then it is possible that certain properties may only apply
to one of its two mappings.
It is still acceptable to associate such properties with the transformation,
however, because the \name{TRANSFORM} software will take account of the
number of input/output variables, and will omit properties which it knows
cannot apply when information about a particular mapping is requested.
In general, therefore, a classification property may be declared for a
transformation if \underline{\textbf{either}} of its mappings has that property.


\subsection{\xlabel{basic_properties}Basic Properties}

The basic classification properties are defined as follows:

\begin{description}

\item[\textbf{LINEAR:}] A mapping has this property if all its output variables
are related to its input variables by linear arithmetic expressions.
Such a mapping will preserve straight lines.
In two dimensions, examples of \name{LINEAR} mappings include shifts of
origin, rotations, reflections, magnifications and shearing deformations.

\begin{itemize}
\item In general, a mapping is \name{LINEAR} if all its first derivatives
are constant (\emph{i.e.}\ do not change from point to point).
\end{itemize}

\item[\textbf{INDEPENDENT:}] A mapping has this property if a change in each
input variable causes a corresponding change in only a single distinct
output variable.
Such a mapping will preserve the independence of the coordinate axes.
A simple example in two dimensions would be the interchange of the two axes.

\begin{itemize}
\item In general, a mapping is \name{INDEPENDENT} if there is at most one
element in each row and column of its Jacobian matrix which is not
identically zero.
\end{itemize}

\item[\textbf{DIAGONAL:}] A mapping has this property if each output variable
depends only on the \textbf{corresponding} input variable, so that the
coordinate axes are preserved.
There are many examples of such mappings in two dimensions, including those
normally used for scaling linear (and logarithmic) graphs.
Note that a \name{DIAGONAL} mapping is more strongly constrained than an
\name{INDEPENDENT} mapping (above) in which the coordinate axes may be
interchanged.
A \name{DIAGONAL} mapping is necessarily always \name{INDEPENDENT}.

\begin{itemize}
\item In general, a mapping is \name{DIAGONAL} if its Jacobian matrix is
square and diagonal (\emph{i.e.}\ all its off-diagonal terms are identically
zero).
If the Jacobian matrix is not square, then the mapping is \name{DIAGONAL} if
$\pdiff{i}{j}$ is identically zero for all \mbox{$i \ne j$}.
\end{itemize}

\item[\textbf{ISOTROPIC:}] A mapping has this property if it locally preserves
shapes and the angles between lines.
Such a mapping may apply a local scale factor to the distances between
neighbouring points, but this factor will not depend on the orientation
of the line between the two points, although it may vary from point to
point.
In two dimensions, an \name{ISOTROPIC} mapping will convert a circle at any
point into another circle (but possibly of a different size and in a
different place), whereas a non-\name{ISOTROPIC} mapping would produce an
ellipse.
If the mapping is also \name{LINEAR} (see above) then circles of
any size will behave in this way, whereas with a non-\name{LINEAR}
mapping this may only be true for circles of infinitely small size.
Isotropy is an important property of \emph{conformal} map projections.

\begin{itemize}
\item In general, a mapping is \name{ISOTROPIC} if it introduces a distance
scale factor between neighbouring points which does not depend on the
relative orientation of the points.
This will be true if its Jacobian matrix $\textbf{J}$ satisfies:

\begin{equation}
\mathbf{\tilde{J}} \mathbf{J} = \lambda \mathbf{I}
\end{equation}

where $\mathbf{\tilde{J}}$ denotes the transpose of $\mathbf{J}$, $\lambda$ is a
constant and $\mathbf{I}$ is an identity matrix.
A mapping cannot have this property if the number of output variables is
less than the number of input variables.
A mapping with only one input and one output variable is necessarily always
\name{ISOTROPIC}.

\end{itemize}

\item[\textbf{POSITIVE\_DET:}] A mapping has this property if the determinant
of its Jacobian matrix is greater than zero at all points.
In two dimensions, such a mapping can locally represent rotations,
magnifications and shearing deformations and can globally represent
``rubber-sheet'' distortions, but it will lack any component of reflection.
A string of text subjected to such a mapping would remain legible (although
possibly highly distorted) and would not be converted into a mirror image of
itself.

\begin{itemize}
\item This property can only apply to mappings with an equal number of input
and output variables.
\end{itemize}

\item[\textbf{NEGATIVE\_DET:}] A mapping has this property if the determinant
of its Jacobian matrix is less than zero at all points.
In two dimensions, such a mapping will locally include a component of
reflection (possibly also combined with rotation, magnification and shearing
deformation) and can globally represent ``rubber-sheet'' distortion
combined with a reflection.
A string of text subjected to such a mapping would be converted into a
mirror image of itself (in addition to any other distortion present).

\begin{itemize}
\item This property can only apply to mappings with an equal number of input
and output variables.
\end{itemize}

\emph{N.B.} A mapping may not have both the \name{POSITIVE\_DET} and
\name{NEGATIVE\_DET} properties simultaneously.
It is also possible that neither of these properties may apply if the
determinant is positive at some points and negative at others.

\item[\textbf{CONSTANT\_DET:}] A mapping has this property if its area (or
volume) scale factor has the same value at all points.
If the mapping has an equal number of input and output variables, then this
will be true if the determinant of its Jacobian matrix has the same value at
all points.
Mappings which are \name{LINEAR} (see above) necessarily have the
\name{CONSTANT\_DET} property, but it can also apply to non-\name{LINEAR}
mappings and is an important property of \emph{equal area} map projections.

\begin{itemize}
\item A mapping cannot have this property if the number of output variables
is less than the number of input variables.
\end{itemize}

\item[\textbf{UNIT\_DET:}]  A mapping has this property if the absolute value of
its area (or volume) scale factor is unity (and it has the same sign) at all
points.
If the mapping has an equal number of input and output variables, then this
will be true if the determinant of its Jacobian matrix has an absolute value
of unity (and the same sign) at all points.
This is a stronger constraint than the \name{CONSTANT\_DET} property (above)
and a mapping with the \name{UNIT\_DET} property necessarily has the
\name{CONSTANT\_DET} property also.
In addition, one of the two properties \name{POSITIVE\_DET} or
\name{NEGATIVE\_DET} will apply.

\begin{itemize}
\item A mapping cannot have this property if the number of output variables
is less than the number of input variables.
\end{itemize}

\end{description}


\subsection{\xlabel{composite_properties}Composite Properties}

Many important mapping properties are \emph{composite}; \emph{i.e.}\ they
depend on the presence of several of the basic properties above in
combination.
Table~\ref{table:composite} lists the more important of these and the
following notes augment the information in this Table.
The presence of a possible shift of origin is disregarded throughout:

\begin{table}
\begin{center}

\begin{tabular}{|l|c|c|c|c|c|c|c|c|c|}
\cline{2-10}
\multicolumn{1}{c}{} & \multicolumn{9}{|c|}\emph{Type of Mapping} \\
\hline
\emph{Basic Property}
              & A    & B    & C    & D    & E    & F    & G    & H    & I    \\
\hline
LINEAR        & \req & \req & \req & \req & \pro & \req & \req & \irr & \irr \\
INDEPENDENT   & \imp & \irr & \imp & \imp & \imp & \req & \imp & \irr & \irr \\
DIAGONAL      & \req & \irr & \req & \req & \req & \pro & \req & \irr & \irr \\
ISOTROPIC     & \req & \req & \req & \irr & \irr & \req & \req & \req & \irr \\
POSITIVE\_DET & \req & \req & \req & ?    & ?    & ?    & ?    & \req & \req \\
NEGATIVE\_DET & \pro & \pro & \pro & ?    & ?    & ?    & ?    & \pro & \pro \\
CONSTANT\_DET & \imp & \imp & \imp & \imp & \irr & \imp & \imp & \irr & \req \\
UNIT\_DET     & \req & \req & \irr & \irr & \irr & \req & \req & \irr & \irr \\
\hline
\end{tabular}

\begin{footnotesize}

Key:\\
\mbox{
\begin{tabular}[t]{rc@{ -- }l}
Mapping types: & A & Shift of origin \\
               & B & Rotation about an axis \\
               & C & Magnification about a point \\
               & D & Graphical scaling (linear) \\
               & E &  Graphical scaling (non-linear) \\
               & F & Interchange of axes \\
               & G & Axis reversal \\
               & H & Conformal map projection \\
               & I & Equal area map projection \\
\end{tabular}
\begin{tabular}[t]{rc@{ -- }l}
Symbols: & \req & Required \\
         & \imp & Implied \\
         & \pro & Prohibited \\
         & \irr & Irrelevant \\
         & ?    & See note in text \\
\end{tabular}
}

\end{footnotesize}

\end{center}

\caption{Common types of mapping with their composite classification
properties.}

\label{table:composite}

\end{table}

\begin{description}

\item[\textbf{A -- Shift of origin.}] The mapping implements a simple shift of
coordinate origin, the nature of which must be determined by transforming a
test point.

\item[\textbf{B -- Rotation about an axis.}] The mapping represents a
simple rotation about an axis (a point in two dimensions) without
associated magnification or distortion.
If the \name{DIAGONAL} property also applies, then the amount of rotation
will be zero, so the mapping reduces to a shift of origin (see A above).

\item[\textbf{C -- Magnification about a point.}] The mapping applies a
simple positive magnification (a zoom) factor about a point without any
associated rotation or other form of distortion.
If the magnification factor is negative, then a component of reflection will
be introduced if the number of input/output variables is odd.
In this case the \name{POSITIVE\_DET} property should be replaced by
\name{NEGATIVE\_DET}.

\item[\textbf{D -- Graphical scaling (linear).}] This type of mapping is
commonly used to scale the axes of a graph, with different scale factors
being applied to each axis.
Either \name{POSITIVE\_DET} or \name{NEGATIVE\_DET} will also apply depending
on the sign of the scale factors in use and whether they result in a mirror
image.
\name{POSITIVE\_DET} will apply if the number of negative scale factors is
even and \name{NEGATIVE\_DET} will apply if this number is odd.

\item[\textbf{E -- Graphical scaling (non-linear).}] This type of mapping is
commonly used to non-linearly scale the axes of graphs (to produce a
log-log plot for instance).
Since the non-linear functions used are normally monotonic, either the
\name{POSITIVE\_DET} or \name{NEGATIVE\_DET} property will usually apply,
depending on the sign of the scaling functions' derivatives along each axis.
\name{POSITIVE\_DET} will apply if the number of negative derivatives is
even and \name{NEGATIVE\_DET} will apply if this number is odd.

\item[\textbf{F -- Interchange of axes.}] The mapping simply interchanges
coordinate values.
The property \name{POSITIVE\_DET} will apply if the resulting axis
permutation is \emph{cyclic} and \name{NEGATIVE\_DET} will apply if the
permutation is \emph{non-cyclic}.

\item[\textbf{G -- Axis reversal.}] The mapping reverses one or more of the
axes (\emph{i.e.}\ changes the sign of the coordinates with or without the
addition of a constant).
The \name{POSITIVE\_DET} property will apply if the number of axes reversed
is even, while \name{NEGATIVE\_DET} will apply if this number is odd.

\item[\textbf{H -- Conformal map projection.}] This implements a
conformal map projection which locally preserves shapes and angles but may
introduce a scale factor which varies from point to point.

\item[\textbf{I -- Equal area map projection.}] The mapping implements an
equal area map projection in which the area scale factor does not vary from
point to point, although shapes and the angles between lines may be
distorted.

\end{description}

\newpage
\section{\xlabel{hds_structures}HDS Structures}

\label{appendix_structures}

This Appendix describes the \name{HDS} structures used by the
\name{TRANSFORM} facility.\footnote{
The data structures described here may be subject to change in future.
However, apart from minor changes in the way data precision is handled, the
current structures will continue to be supported by future versions of
user-level \name{TRANSFORM} routines.
}
There are three of these, distinguished by their HDS type:

\begin{description}

\item[\textbf{TYPE = \name{TRN\_TRANSFORM}}] structures are used to hold data
defining a complete (possibly multi-stage) \emph{transformation} and are
handled directly by user-level \name{TRANSFORM} routines.

\item[\textbf{TYPE = \name{TRN\_MODULE}}] structures are used to hold
\emph{transformation module} information which represents a simple 1-stage
transformation without associated classification information.
These structures are not handled directly by user-level \name{TRANSFORM}
routines.

\item[\textbf{TYPE = \name{TRN\_CLASS}}] structures are used to hold
{\em classification} information about the properties of a transformation.
These structures are also not handled directly by the user-level
\name{TRANSFORM} routines.

\end{description}

The following Sections describe these structures in detail.

\subsection{\xlabel{the_nametrn_transform_structure}The \name{TRN\_TRANSFORM} Structure}

An \name{HDS} structure of type \name{TRN\_TRANSFORM} contains the complete
specification of a (possibly multi-stage) transformation with its associated
classification information.
It is defined as follows:

\begin{center}

\begin{tabular}{|llc|c|}
\hline
\multicolumn{4}{|c|}{\raisebox{0ex}[3.5ex][1.5ex]{\bf Components of a
\name{TRN\_TRANSFORM} Structure}} \\
\hline
\hline
{\em Component Name} & {\em \name{HDS} Type} & {\em Typical Value} & {\em
Optional?} \\
\hline
TRN\_VERSION & \_REAL & 0.9 & no \\
FORWARD & \_CHAR & `\verb#DEFINED#' & no \\
INVERSE & \_CHAR & `\verb#UNDEFINED#' & no \\
MODULE\_ARRAY() & TRN\_MODULE & $<$array of structures$>$ & no \\
CLASSIFICATION & TRN\_CLASS & $<$structure$>$ & yes \\
\hline
\end{tabular}

\end{center}

The structure components have the following meanings and restrictions:

\begin{description}

\item[\textbf{\name{TRN\_VERSION}}] is a mandatory scalar \name{\_REAL} component
containing the version number of the \name{TRANSFORM} software which created
or last modified the \name{TRN\_TRANSFORM} structure or any of its
components.
This information allows programs linked with old versions of
\name{TRANSFORM} software to detect if they are processing data structures
created by more recent versions, with which they may not be compatible.

\item[\textbf{\name{FORWARD}}] is a mandatory scalar \name{\_CHAR} component which
must contain one of the two values `\verb#DEFINED#' or `\verb#UNDEFINED#' (case
insensitive).
It specifies whether the transformation's \textbf{forward} mapping is defined
or undefined.

\item[\textbf{\name{INVERSE}}] is a mandatory scalar \name{\_CHAR} component which
must contain one of the two values `\verb#DEFINED#' or `\verb#UNDEFINED#' (case
insensitive).
It specifies whether the transformation's \textbf{inverse} mapping is defined
or undefined.
It is not permitted for both the \name{FORWARD} and \name{INVERSE}
components to have the value `\verb#UNDEFINED#'.

\item[\textbf{\name{MODULE\_ARRAY()}}] is a mandatory 1-dimensional array of
structures of type \name{TRN\_MODULE}.
It holds a sequence of transformation modules which are to be used in
succession to define the overall transformation.
If the \name{FORWARD} component has the value `\verb#DEFINED#', then all the
modules in this array must have their \textbf{forward} mappings defined.
Similarly, if the \name{INVERSE} component has the value `\verb#DEFINED#', then
all the modules must have their \textbf{inverse} mappings defined.
Adjacent modules in this array must have matching numbers of output and input
variables.
The number of input variables for the transformation as a whole is
determined by the number of input variables for the first module in this
array, while the number of output variables is determined by the number of
output variables for the final module.

\item[\textbf{\name{CLASSIFICATION}}] is an optional scalar structure component
of type \name{TRN\_CLASS}.  If present, it holds the classification
information for the entire transformation.  If absent, a default set of
classification properties applies, which corresponds to all elements of the
transformation's logical classification array being set to \name{.FALSE.}.

\end{description}

\subsection{\xlabel{the_nametrn_module_structure}The \name{TRN\_MODULE} Structure}

A structure of type \name{TRN\_MODULE} contains transformation module
information, which represents the simplest form of 1-stage transformation
without associated classification information.
More complex transformations may be built up by joining several of these
modules together, as is done in the \name{MODULE\_ARRAY} component of a
\name{TRN\_TRANSFORM} structure.
The \name{TRN\_MODULE} structure is defined as follows:

\begin{center}

\begin{tabular}{|llc|c|}
\hline
\multicolumn{4}{|c|}{\raisebox{0ex}[3.5ex][1.5ex]{\textbf{Components of a
\name{TRN\_MODULE} Structure}}} \\
\hline
\hline
\emph{Component Name} & \emph{\name{HDS} Type} & \emph{Typical Value} &
\emph{Optional?} \\ \hline
NVAR\_IN & \_INTEGER & 2 & no \\
NVAR\_OUT & \_INTEGER & 2 & no \\
COMMENT & \_CHAR &         `\verb#2-d Cartesian --> 2-d Polar#' & yes \\
PRECISION & \_CHAR &       `\verb#_REAL:#' & no \\
FORWARD\_FUNC() & \_CHAR & `\verb#R=SQRT(X*X+Y*Y)#',`\verb#THETA=A...#' & no \\
INVERSE\_FUNC() & \_CHAR & `\verb#X=R*COS(THETA)#',`\verb#Y=R*SIN(...#' & no \\
\hline
\end{tabular}

\end{center}

The structure components have the following meanings and restrictions:

\begin{description}

\item[\textbf{\name{NVAR\_IN}}] is a mandatory scalar \name{\_INTEGER}
component with a positive value specifying the number of \textbf{input}
variables for the transformation which the module describes.

\item[\textbf{\name{NVAR\_OUT}}] is a mandatory scalar \name{\_INTEGER}
component with a positive value specifying the number of \textbf{output}
variables for the transformation which the module describes.

\item[\textbf{\name{COMMENT}}] is an optional scalar \name{\_CHAR} component of
arbitrary length which contains a comment associated with the transformation
module.
This comment should describe (in English) the type of transformation the
module performs.
The contents of the \name{COMMENT} component are only intended to be
human-readable and should not be machine-processed, except that the special
three-character sequences `\verb#-->#' and `\verb#<--#' (if present) may be
automatically interchanged to reflect the effect of inverting the
transformation module.

\item[\textbf{\name{PRECISION}}] is a mandatory scalar \name{\_CHAR} component
of arbitrary length which contains a valid precision specification for the
transformation module, according to the syntax described in
Section~\ref{section:advanced:precision}.

\item[\textbf{\name{FORWARD\_FUNC()}}] is a mandatory 1-dimensional
\name{\_CHAR} array, with elements of arbitrary length, containing
transformation functions which define the \textbf{forward} mapping (according
to the syntax in Appendix~\ref{appendix_syntax}).
The number of array elements must match the value stored in the \name{NVAR\_OUT}
component.

\item[\textbf{\name{INVERSE\_FUNC()}}] is a mandatory 1-dimensional
\name{\_CHAR} array, with elements of arbitrary length, containing
transformation functions which define the \textbf{inverse} mapping
(according to the syntax in Appendix~\ref{appendix_syntax}).
The number of array elements must match the value stored in the
\name{NVAR\_IN} component.

\end{description}

\subsection{\xlabel{the_nametrn_class_structure}The \name{TRN\_CLASS} Structure}

A structure of type \name{TRN\_CLASS} contains information specifying the
set of classification properties associated with a transformation.
These properties are described in Appendix~\ref{appendix:classification}.
The \name{TRN\_CLASS} structure is defined as follows:

\begin{center}

\begin{tabular}{|llc|c|}
\hline
\multicolumn{4}{|c|}{\raisebox{0ex}[3.5ex][1.5ex]{\bf Components of a
\name{TRN\_CLASS} Structure}} \\
\hline
\hline
{\em Component Name} & {\em \name{HDS} Type} & {\em Typical Value} & {\em
Optional?} \\ \hline
LINEAR & \_LOGICAL & TRUE & yes \\
INDEPENDENT & \_LOGICAL & TRUE & yes \\
DIAGONAL & \_LOGICAL & TRUE & yes \\
ISOTROPIC & \_LOGICAL & FALSE & yes \\
POSITIVE\_DET & \_LOGICAL & TRUE & yes \\
NEGATIVE\_DET & \_LOGICAL & FALSE & yes \\
CONSTANT\_DET & \_LOGICAL & TRUE & yes \\
UNIT\_DET & \_LOGICAL & FALSE & yes \\
\hline
\end{tabular}

\end{center}

All the structure components are scalar \name{\_LOGICAL} objects, whose
presence is optional.
Their names correspond with the classification properties described in
Appendix~\ref{appendix:classification} and their logical values indicate
whether the transformation has the associated property. If any component is
absent, its value defaults to \name{.FALSE.}

\newpage
\section{\xlabel{routine_descriptions}Routine Descriptions}
\label{appendix_routines}

\subsection{\xlabel{routine_list}Routine List}

The following is a complete list of user-level \name{TRANSFORM} routines
with a brief description of their purpose.
Full specifications are given in the next Section.

\noteroutine{TRN\_ANNUL( ID, STATUS )}
            { Annul compiled mapping}
\noteroutine{TRN\_APND( LOCTR1, LOCTR2, STATUS )}
            { Append transformation}
\noteroutine{TRN\_CLOSE( STATUS )}
            { Close the \name{TRANSFORM} facility}
\noteroutine{TRN\_COMP( LOCTR, FORWD, ID, STATUS )}
            { Compile transformation}
\noteroutine{TRN\_GTCL( LOCTR, FORWD, CLASS, STATUS )}
            { Get classification}
\noteroutine{TRN\_GTCLC( ID, CLASS, STATUS )}
            { Get compiled classification}
\noteroutine{TRN\_GTNV( LOCTR, NVIN, NVOUT, STATUS )}
            { Get numbers of variables}
\noteroutine{TRN\_GTNVC( ID, NVIN, NVOUT, STATUS )}
            { Get numbers of compiled variables}
\noteroutine{TRN\_INV( LOCTR, STATUS )}
            { Invert transformation}
\noteroutine{TRN\_JOIN( LOCTR1, LOCTR2, ELOC, NAME, LOCTR, STATUS )}
            { Concatenate transformations}
\noteroutine{TRN\_NEW( NVIN, NVOUT, FOR, INV, PREC, COMM, ELOC, NAME, LOCTR, STATUS )}
            { Create new transformation}
\noteroutine{TRN\_PRFX( LOCTR1, LOCTR2, STATUS )}
            { Prefix transformation}
\noteroutine{TRN\_PTCL( CLASS, LOCTR, STATUS )}
            { Put classification}
\noteroutine{TRN\_STOK[x]( TOKEN, VALUE, TEXT, NSUBS, STATUS )}
            { Substitute token}
\noteroutine{TRN\_TR1x( BAD, NX, XIN, ID, XOUT, STATUS )}
            { Transform 1-dimensional data}
\noteroutine{TRN\_TR2x( BAD, NXY, XIN, YIN, ID, XOUT, YOUT, STATUS )}
            { Transform 2-dimensional data}
\noteroutine{TRN\_TRNx( BAD, ND1, NCIN, NDAT, DATA, ID, NR1, NCOUT, RESULT, STATUS )}
            { Transform general data}

\newpage
\subsection{\xlabel{full_routine_specifications}Full Routine Specifications}

This Section gives full specifications for the user-level \name{TRANSFORM}
routines.
The following notation is used to specify the types of routine arguments:

\begin{center}
\begin{tabular}{ccp{30em}}
\textbf{Notation} & : & \textbf{Fortran type} \\
\hline
\textbf{L} & : & \name{LOGICAL}\\
\textbf{I} & : & \name{INTEGER}\\
\textbf{R} & : & \name{REAL}\\
\textbf{C} & : & CHARACTER$*$($*$)\\
\textbf{?} & : & Unspecified, dependent on the data type being processed.\\
\textbf{()} & : & A 1-dimensional array of one of the above types.\\
\textbf{(,)} & : & A 2-dimensional array of one of the above types.\\
\textbf{L$_{CLS}$} & : & A 1-dimensional \name{LOGICAL} array, with
                  \name{TRN\_\_MXCLS}\footnotemark\ elements, to
                  contain classification information.\\
\textbf{C$_{PRC}$} & : & A \name{CHARACTER} argument to contain a precision
                  specification of maximum length
                  \name{TRN\_\_SZPRC}.\addtocounter{footnote}{-1}\footnotemark\\
\textbf{C$_{LOC}$} & : & A \name{CHARACTER$*$(DAT\_\_SZLOC)\footnotemark} argument
                  containing an \name{HDS} locator.  In some routines where
                  this is an input argument, a blank string (of any length)
                  may be substituted to specify different routine behaviour;
                  the individual routine descriptions give details.\\
\textbf{C$_{NAM}$} & : & A \name{CHARACTER} argument specifying an \name{HDS} name
                  string of maximum length
                  \name{DAT\_\_SZNAM}.\addtocounter{footnote}{-1}\footnotemark\\
\hline
\end{tabular}
\end{center}

\addtocounter{footnote}{-1}
\footnotetext{The symbolic constants \name{TRN\_\_MXCLS} and
              \name{TRN\_\_SZPRC} are defined in the include file
              \name{TRN\_PAR}.}
\addtocounter{footnote}{1}
\footnotetext{The symbolic constants \name{DAT\_\_SZLOC} and
              \name{DAT\_\_SZNAM} are defined by \name{HDS}.}

\newpage

\sstroutine{
   TRN\_ANNUL
}{
Annul compiled mapping
}{
   \sstdescription{
      Annul the compiled mapping associated with the identifier supplied.
      Resources associated with the compiled mapping are released and
      the identifier is reset to \name{TRN\_\_NOID}.
   }
   \sstinvocation{
      CALL TRN\_ANNUL( ID, STATUS )
   }
   \sstarguments{
      \sstsubsection{
         ID = I
      }{
         Compiled mapping identifier to be annulled.
      }
      \sstsubsection{
         STATUS = I
      }{
         The global status
      }
   }
   \sstnotes{
      \sstitemlist{
         \sstitem
         The symbolic constant \name{TRN\_\_NOID} is defined in the include
         file \name{TRN\_PAR}.

         \sstitem
         This routine attempts to execute even if \fortvar{STATUS} is set on
         entry, although no error report will be made if it subsequently
         fails under these circumstances.
      }
   }
}
\sstroutine{
   TRN\_APND
}{
   Append transformation
}{
   \sstdescription{
      Concatenate two transformations; the first transformation is
      altered by appending the second one to it.
      The second transformation is not altered.
   }
   \sstinvocation{
      CALL TRN\_APND( LOCTR1, LOCTR2, STATUS )
   }
   \sstarguments{
      \sstsubsection{
         LOCTR1 = C$_{LOC}$
      }{
         Locator to the first transformation (to be altered).
      }
      \sstsubsection{
         LOCTR2 = C$_{LOC}$
      }{
         Locator to the second transformation (to be appended).
      }
      \sstsubsection{
         STATUS = I
      }{
         Inherited error status.
      }
   }
   \sstnotes{
      Before two transformations may be concatenated, the following conditions
      must apply:
      \sstitemlist{
         \sstitem
         The number of output variables from the first transformation must
         match the number of input variables to the second transformation.

         \sstitem
         The transformations must have at least one mapping (forward or
         inverse) in common.
      }
   }
}

\sstroutine{
   TRN\_CLOSE
}{
   Close the \name{TRANSFORM} facility
}{
   \sstdescription{
      Close the \name{TRANSFORM} facility down, annulling all compiled
      mappings and releasing all resources.  No action is taken if the
      facility is already closed.
   }
   \sstinvocation{
      CALL TRN\_CLOSE( STATUS )
   }
   \sstarguments{
      \sstsubsection{
         STATUS = I
      }{
         Inherited error status.
      }
   }
   \sstnotes{
      This routine attempts to execute even if the \fortvar{STATUS} value
      is set on entry, although no error report will be made if it
      subsequently fails under these circumstances.
   }
}

\sstroutine{
   TRN\_COMP
}{
   Compile transformation
}{
   \sstdescription{
      Compile a transformation and return an identifier for the
      resulting compiled mapping.
   }
   \sstinvocation{
      CALL TRN\_COMP( LOCTR, FORWD, ID, STATUS )
   }
   \sstarguments{
      \sstsubsection{
         LOCTR = C$_{LOC}$
      }{
         Locator to the transformation.
      }
      \sstsubsection{
         FORWD = L
         }{
         Specifies which mapping is to be compiled; \name{.TRUE.}
                  for forward, \name{.FALSE.} for inverse.
      }
      \sstsubsection{
         ID = I
         }{
         Identifier for the compiled mapping.
      }
      \sstsubsection{
         STATUS = I
         }{
         Inherited error status.
      }
   }
}

\sstroutine{
   TRN\_GTCL
}{
   Get classification
}{
   \sstdescription{
      Obtain an array of {\em logical} classification values from a
      transformation.
      The classification information is fully validated before it is
      returned.
   }
   \sstinvocation{
      CALL TRN\_GTCL( LOCTR, FORWD, CLASS, STATUS )
   }
   \sstarguments{
      \sstsubsection{
         LOCTR = C$_{LOC}$
      }{
         Locator to the transformation.
      }
      \sstsubsection{
         FORWD = L
      }{
         Specifies which mapping the information is required for;
         \name{.TRUE.} for forward, \name{.FALSE.} for inverse.
      }
      \sstsubsection{
         CLASS = L$_{CLS}$
      }{
         Classification array.
      }
      \sstsubsection{
         STATUS = I
      }{
         Inherited error status.
      }
   }
}

\sstroutine{
   TRN\_GTCLC
}{
   Get compiled classification
}{
   \sstdescription{
      Obtain an array of \emph{logical} classification values from a
      compiled mapping.
   }
   \sstinvocation{
       CALL TRN\_GTCLC( ID, CLASS, STATUS )
   }
   \sstarguments{
      \sstsubsection{
         ID = I
      }{
         Identifier for the compiled mapping.
      }
      \sstsubsection{
         CLASS = L$_{CLS}$
      }{
         Classification array.
      }
      \sstsubsection{
         STATUS = I
      }{
         Inherited error status.
      }
   }
}

\sstroutine{TRN\_GTNV}{Get numbers of variables}
{
   \sstdescription{Obtain the numbers of input and output variables for a
          transformation.}
   \sstinvocation{CALL TRN\_GTNV( LOCTR, NVIN, NVOUT, STATUS )}
   \sstarguments{
      \sstsubsection{LOCTR = C$_{LOC}$}{Locator to the transformation.}
      \sstsubsection{NVIN = I}{Number of input variables.}
      \sstsubsection{NVOUT = I}{Number of output variables.}
      \sstsubsection{STATUS = I}{Inherited error status.}
   }
}

\sstroutine{TRN\_GTNVC}{Get numbers of compiled variables}
{
   \sstdescription{Obtain the numbers of input and output variables for a compiled mapping.}
   \sstinvocation{CALL TRN\_GTNVC( ID, NVIN, NVOUT, STATUS )}
   \sstarguments{
      \sstsubsection{ID = I}{Identifier for the compiled mapping.}
      \sstsubsection{NVIN = I}{Number of input variables.}
      \sstsubsection{NVOUT = I}{Number of output variables.}
      \sstsubsection{STATUS = I}{Inherited error status.}
   }
}

\sstroutine{TRN\_INV}{Invert transformation}
{
   \sstdescription{
      Invert a transformation.  This interchanges the definitions of the
      forward and inverse mappings.
   }
   \sstinvocation{
      CALL TRN\_INV( LOCTR, STATUS )
   }
   \sstarguments{
      \sstsubsection{
         LOCTR = C$_{LOC}$
      }{
         Locator to the transformation to be inverted.
      }
      \sstsubsection{
         STATUS = I
      }{
         Inherited error status.
      }
   }
}

\sstroutine{
   TRN\_JOIN
}{
   Concatenate transformations
}{
   \sstdescription{
      Concatenate two transformations to produce a combined transformation
      which is stored as a new component in an existing \name{HDS} structure.
      The combined transformation may also be assigned to a temporary
      \name{HDS} object if required.
   }
   \sstinvocation{
      CALL TRN\_JOIN( LOCTR1, LOCTR2, ELOC, NAME, LOCTR, STATUS )
   }
   \sstarguments{
      \sstsubsection{
         LOCTR1 = C$_{LOC}$
      }{
         Locator to the first transformation.
      }
      \sstsubsection{
         LOCTR2 = C$_{LOC}$
      }{
         Locator to the second transformation.
      }
      \sstsubsection{
         ELOC = C$_{LOC}$
      }{
         Locator to an enclosing structure to contain the new object.
      }
      \sstsubsection{
         NAME = C$_{NAM}$
      }{
         \name{HDS} name of the new structure component to be created.
      }
      \sstsubsection{
         LOCTR = C$_{LOC}$
      }{
         Locator to the newly created transformation.
      }
      \sstsubsection{
         STATUS = I
      }{
         Inherited error status.
      }
   }
\newpage
   \sstnotes{
      \sstitemlist{
         \sstitem
         If the enclosing structure locator \fortvar{ELOC} is supplied as a
         blank string, then a temporary transformation will be created.
         The \fortvar{NAME} argument is then not used and may also be blank.

         \sstitem
         Before two transformations may be concatenated, the following
         conditions must apply:

         \sstitemlist{
            \sstitem
            The number of output variables from the first transformation
            must match the number of input variables to the second
            transformation.

            \sstitem
            The transformations must have at least one mapping (forward or
            inverse) in common.
         }
      }
   }
}

\sstroutine{
   TRN\_NEW
}{
   Create new transformation
}{
   \sstdescription{
       Create a new transformation and return a locator to it.  The
       transformation is stored as a new component in an existing HDS
       structure, or may be assigned to a temporary object if
       required.
   }
   \sstinvocation{
      CALL TRN\_NEW( NVIN, NVOUT, FOR, INV, PREC, COMM, ELOC, \\
           NAME, LOCTR, STATUS )
   }
   \sstarguments{
      \sstsubsection{
         NVIN = I
      }{
         Number of input variables.
      }
      \sstsubsection{
         NVOUT = I
      }{
         Number of output variables.
      }
      \sstsubsection{
         FOR = C({\small NVOUT})
      }{
         Array of forward transformation functions.
      }
      \sstsubsection{
         INV = C({\small NVIN})
      }{
         Array of inverse transformation functions.
      }
      \sstsubsection{
         PREC = C$_{PRC}$
      }{
         Precision specification.
      }
      \sstsubsection{
         COMM = C
      }{
         Comment string.
      }
      \sstsubsection{
         ELOC = C$_{LOC}$
      }{
         Locator to an enclosing structure to contain the new object.
      }
      \sstsubsection{
         NAME = C$_{NAM}$
      }{
         \name{HDS} name of the new structure component.
      }
      \sstsubsection{
         LOCTR = C$_{LOC}$
      }{
         Locator to the newly created transformation.
      }
      \sstsubsection{
         STATUS = I
      }{
         Inherited error status.
      }
   }
   \sstnotes{
      \sstitemlist{
	 \sstitem The transformation functions will be fully validated
         before use.

         \sstitem
         If the enclosing structure locator \fortvar{ELOC} is supplied as a
         blank string, then a temporary transformation will be created.
         The \fortvar{NAME} argument is then not used and may also be blank.
      }
   }
}

\sstroutine{TRN\_PRFX}{Prefix transformation}
{
   \sstdescription{
      Concatenate two transformations; the second transformation is
      altered by prefixing the first one to it.  The first transformation
      is not altered.}
   \sstinvocation{CALL TRN\_PRFX( LOCTR1, LOCTR2, STATUS )}
   \sstarguments{
      \sstsubsection{LOCTR1 = C$_{LOC}$}{
         Locator to the first transformation (to be prefixed).}
      \sstsubsection{LOCTR2 = C$_{LOC}$}{
         Locator to the second transformation (to be altered).}
      \sstsubsection{STATUS = I}{
         Inherited error status.}
   }
   \sstnotes{
      Before two transformations may be concatenated, the following conditions
      must apply:
      \sstitemlist{
         \sstitem
         The number of output variables from the first transformation must
         match the number of input variables to the second transformation.

         \sstitem
         The transformations must have at least one mapping (forward or
         inverse) in common.
      }
   }
}

\sstroutine{TRN\_PTCL}{Put classification}
{
   \sstdescription{
       Enter classification information into a transformation.  The
       information is supplied as an array of \emph{logical} values and
       is validated before use.  If the transformation already contains
       such information, it is over-written by this routine.
   }
   \sstinvocation{CALL TRN\_PTCL( CLASS, LOCTR, STATUS )}
   \sstarguments{
      \sstsubsection{CLASS = L$_{CLS}$}{
         Array containing the classification information.}
      \sstsubsection{LOCTR = C$_{LOC}$}{
         Locator to the transformation.}
      \sstsubsection{STATUS = I}{
         Inherited error status.}
   }
}

\sstroutine{TRN\_STOK[x]}{Substitute token}
{
   \sstdescription{
       Substitute a value for a token in a text string.
       The \fortvar{VALUE} supplied is formatted as a character string
       (if necessary) and used to replace all valid occurrences of the
       \fortvar{TOKEN} sub-string in the \fortvar{TEXT} supplied.
       The substitution is not recursive.
   }
   \sstinvocation{CALL TRN\_STOK[x]( TOKEN, VALUE, TEXT, NSUBS, STATUS )}
   \sstarguments{
      \sstsubsection{TOKEN = C}{The token string.}
      \sstsubsection{VALUE = ?}{The value to substitute.}
      \sstsubsection{TEXT = C}{The text to be processed.}
      \sstsubsection{NSUBS = I}{Number of substitutions made.}
      \sstsubsection{STATUS = I}{Inherited error status.}
   }
   \sstnotes{
      \sstitemlist{
         \sstitem
         There is a routine for each standard type x.
         Replace x by \textbf{I} (\emph{integer}), \textbf{R} (\emph{real})
         or \textbf{D} (\emph{double precision}).
         Omit x altogether if a character string is being supplied for the
         \fortvar{VALUE} argument.

         \sstitem
         When \fortvar{VALUE} is a numerical quantity, the token replacement
         will be enclosed in parentheses if it is negative.

         \sstitem
         To be replaced, a token sub-string within \fortvar{TEXT} must be
         correctly delimited (\emph{i.e.}\ surrounded by non-alphanumeric
         characters).
         It may also be enclosed in angle brackets (\emph{e.g.}\
         \name{$<$a\_token$>$}), in which case the brackets will be regarded
         as part of the token and will also be replaced.

         \sstitem
         To be valid, a token must begin with an alphabetic character and
         contain only alphanumeric characters (including underscore).
         It may be of any length.
         No embedded blanks are allowed.
      }
   }
}

\sstroutine{TRN\_TR1x}{Transform 1-dimensional data}
{
   \sstdescription{
      Apply a compiled \mbox{\{$1 \rightarrow 1$\}} mapping to a set of
      1-dimensional data points specified by an array of $X_{IN}$ coordinates.}
   \sstinvocation{CALL TRN\_TR1x( BAD, NX, XIN, ID, XOUT, STATUS )}
   \sstarguments{
      \sstsubsection{BAD = L}{
         Whether the input coordinates may be {\em bad}.}
      \sstsubsection{NX = I}{
         The number of data points to transform.}
      \sstsubsection{XIN = ?(\small{NX})}{
         Array of input $X_{IN}$ coordinates.}
      \sstsubsection{ID = I}{
         Identifier for the compiled \mbox{\{$1 \rightarrow 1$\}} mapping
         to be applied.}
      \sstsubsection{XOUT = ?(\small{NX})}{
         Array to receive the transformed $X_{OUT}$ coordinates.}
      \sstsubsection{STATUS = I}{Inherited error status.}
   }
   \sstnotes{
      \sstitemlist{
         \sstitem
         There is a routine for each standard numerical data type x.
         Replace x by \textbf{I} (\emph{integer}), \textbf{R} (\emph{real})
         or \textbf{D} (\emph{double precision}).

         \sstitem
         The number of input and output variables for the compiled
         mapping must both be equal to 1.
      }
   }
}

\sstroutine{TRN\_TR2x}{Transform 2-dimensional data}
{
   \sstdescription{
      Apply a compiled \mbox{\{$2 \rightarrow 2$\}} mapping to a set of
      2-dimensional data points specified by separate arrays of $X_{IN}$
      and $Y_{IN}$ coordinates.
   }
   \sstinvocation{CALL TRN\_TR2x( BAD, NXY, XIN, YIN, ID, XOUT, YOUT, STATUS )}
   \sstarguments{
      \sstsubsection{BAD = L}{
         Whether the input coordinates may be {\em bad}.}
      \sstsubsection{NXY = I}{
         The number of data points to transform.}
      \sstsubsection{XIN = ?({\small NXY})}{
         Array of input $X_{IN}$ coordinates.}
      \sstsubsection{YIN = ?({\small NXY})}{
         Array of input $Y_{IN}$ coordinates.}
      \sstsubsection{ID = I}{
         Identifier for the compiled \mbox{\{$2 \rightarrow 2$\}} mapping
         to be applied.}
      \sstsubsection{XOUT = ?({\small NXY})}{
         Array to receive the transformed $X_{OUT}$ coordinates.}
      \sstsubsection{YOUT = ?({\small NXY})}{
         Array to receive the transformed $Y_{OUT}$ coordinates.}
      \sstsubsection{STATUS = I}{Inherited error status.}
   }
   \sstnotes{
      \sstitemlist{
         \sstitem
         There is a routine for each standard numerical data type x.
         Replace x by \textbf{I} (\emph{integer}), \textbf{R} (\emph{real})
         or \textbf{D} (\emph{double precision}).

         \sstitem
         The number of input and output variables for the compiled
         mapping must both be equal to 2.
      }
   }
}

\sstroutine{TRN\_TRNx}{Transform general data}
{
   \sstdescription{
       Apply a general compiled mapping (with an arbitrary number of
       input/output variables) to a set of data points.
   }
   \sstinvocation{
      CALL TRN\_TRNx( BAD, ND1, NCIN, NDAT, DATA, ID, NR1, NCOUT, \\
           RESULT, STATUS )}
   \sstarguments{
      \sstsubsection{BAD = L}{
         Whether the input coordinates may be {\em bad}.}
      \sstsubsection{ND1 = I}{
         First dimension of the \fortvar{DATA} array (as declared in the
         calling routine).}
      \sstsubsection{NCIN = I}{
         Number of coordinates for each input data point.}
      \sstsubsection{NDAT = I}{
         Number of data points to transform.}
      \sstsubsection{DATA = ?({\small ND1,NCIN})}{
         Array of coordinates for the input data points; they need not fill
         the entire array.}
      \sstsubsection{ID = I}{
         Identifier for the compiled \mbox{\{$NCIN \rightarrow NCOUT$\}}
         mapping to be applied.}
      \sstsubsection{NR1 = I}{
         First dimension of the \fortvar{RESULT} array (as declared in the
         calling routine).}
      \sstsubsection{NCOUT = I}{
         Number of coordinates for each output data point.}
      \sstsubsection{RESULT = ?({\small NR1,NCOUT})}{
         Array to receive the (transformed) coordinates of the output
         data points.}
      \sstsubsection{STATUS = I}{
         Inherited error status.}
   }
   \sstnotes{
      \sstitemlist{
         \sstitem
         There is a routine for each standard numerical data type x.
         Replace x by \textbf{I} (\emph{integer}), \textbf{R} (\emph{real})
         or \textbf{D} (\emph{double precision}).

         \sstitem
         \fortvar{ND1} and \fortvar{NR1} must both be at least equal to the
         number of data points \fortvar{NDAT}.

         \sstitem
         \fortvar{DATA(I,J)} should contain the value of the \fortvar{J'th}
         coordinate for the \fortvar{I'th} input data point.
         Coordinate values are returned in the \fortvar{RESULT} array in the
         same order.

         \sstitem
         The values of \fortvar{NCIN} and \fortvar{NCOUT} must match the
         numbers of input and output variables for the compiled
         mapping being applied.
      }
   }
}

\newpage
\section {Error Handling}

\label{appendix:error}


\subsection{\xlabel{the_status_argument_and_error_reporting}%
The \fortvar{STATUS} Argument and Error Reporting}

All \name{TRANSFORM} routines carry a final \emph{integer} \fortvar{STATUS}
argument and adhere to the \name{ADAM} inherited error handling strategy.
Unless otherwise indicated, a routine which finds that \fortvar{STATUS} is
not set to \name{SAI\_\_OK}\footnote{
\name{SAI\_\_OK} is a symbolic constant defined in the include file
\name{SAE\_PAR}.
}
on entry will assume an error has occurred in a previous routine and will
return immediately without action and without accessing other arguments
(which may not be defined under error conditions).
The value of the \fortvar{STATUS} argument will not be changed.
This behaviour usually allows tests of the \fortvar{STATUS} value to be
deferred until after several routine calls have been made.

If a \name{TRANSFORM} routine is called with \fortvar{STATUS} set to
\name{SAI\_\_OK}, then it will attempt to execute.
If it subsequently encounters an error, it will first perform any ``cleaning
up'' which is possible and will then exit with its \fortvar{STATUS} argument
set to one of the error codes described in the next Section.
When a \fortvar{STATUS} value is set by a \name{TRANSFORM} routine, an
associated error report will always be made by calling the \name{ERR}
routine \name{ERR\_REP}.
The report will normally contain additional information about the
circumstances of the error.
Transmission of the report to the user will be deferred by the \name{ERR}
facility pending action by the caller of the \name{TRANSFORM} routine
(such as a call to \name{ERR\_FLUSH} to output the error, or
\name{ERR\_ANNUL} to ignore it).
The documentation for the \name{ERR} facility should be consulted for
further details.

\paragraph{Exceptions to the rule.}
The routines \name{TRN\_ANNUL} and \name{TRN\_CLOSE} are exceptions to the
above rules.
Both of these perform ``cleaning up'' operations and will therefore attempt
to execute regardless of the \fortvar{STATUS} value set on entry.
If either of these routines fail, they will only set a new \fortvar{STATUS}
value and make an error report if the value of \fortvar{STATUS} was
\name{SAI\_\_OK} on entry.
If \fortvar{STATUS} was not set to \name{SAI\_\_OK} when they were called,
then these two routines will assume that a previous error has occurred and
will fail ``silently'' so that the initial \fortvar{STATUS} value and error
report are preserved.


\subsection{\xlabel{error_codes}Error Codes}

The value returned via the \fortvar{STATUS} argument of \name{TRANSFORM}
routines under error conditions may be:

\begin{itemize}

\item Any of the error codes returned by \name{HDS} routines.

\item Any of the error codes described below.

\end{itemize}

If it is necessary to test for specific error conditions, then symbolic
names (defined by Fortran \name{PARAMETER} statements) should be used to
identify the associated error codes.
Symbolic names for the \name{TRANSFORM} error codes are defined in the
include file \name{TRN\_ERR} and may be incorporated into an application
with the statement:

\begin{quote}
\begin{verbatim}
INCLUDE 'TRN_ERR'
\end{verbatim}
\end{quote}

The following list gives the names of these error codes, the associated
error messages and an explanation of each error condition:

\begin{description}

\item \textbf{TRN\_\_CLSIN, classification information invalid}\\
The classification information associated with a transformation is invalid.
This may be because conflicting classification properties are being
specified (\emph{e.g.}\ \name{POSITIVE\_DET} and \name{NEGATIVE\_DET}) or
because a property is being specified in circumstances where it cannot be
adequately defined (because the numbers of input and output variables
are unequal, for instance).

\item \textbf{TRN\_\_CMPER, compilation error}\\
An error has been detected during compilation of the right hand side of a
transformation function.
The associated error report will give further diagnostic information.

\item \textbf{TRN\_\_CMTOF, compiled mapping table overflow}\\
The maximum number of simultaneously active compiled mappings has
been exceeded.
This error should not be encountered because the permitted number of
compiled mappings is very large.

\item \textbf{TRN\_\_CONIN, constant syntax invalid}\\
The right hand side of a transformation function contains a numerical
constant whose syntax is invalid.

\item \textbf{TRN\_\_DELIN, delimiting comma invalid}\\
The right hand side of a transformation function contains an invalid comma
(commas are only used to separate the arguments of built-in functions).

\item \textbf{TRN\_\_DIMIN, dimensions invalid}\\
An \name{HDS} object has inappropriate dimensions for its purpose.

\item {\bf TRN\_\_DSTIN, definition status invalid}\\
The \name{FORWARD} or \name{INVERSE} component of an \name{HDS}
transformation structure contains an invalid value.
Only the values \name{`DEFINED'} and \name{`UNDEFINED'} (case insensitive)
are allowed.

\item \textbf{TRN\_\_DUVAR, duplicate variable name}\\
A variable name is defined (\emph{i.e.}\ appears on the left hand side of a
transformation function) more than once.

\item \textbf{TRN\_\_EXPUD, expression undefined}\\
The expression on the right hand side of a transformation function is
missing.

\item \textbf{TRN\_\_ICDIR, incompatible transformation directions}\\
An attempt to concatenate two transformations has failed because only the
forward mapping was defined within one of them and only the inverse mapping
was defined within the other.

\item \textbf{TRN\_\_MAPUD, mapping undefined}\\
The mapping requested from a transformation (either forward or inverse) has
not been defined.

\item \textbf{TRN\_\_MIDIN, compiled mapping identifier invalid}\\
A compiled mapping identifier is not valid (\emph{i.e.}\ it is not currently
associated with a compiled mapping).

\item \textbf{TRN\_\_MIOPA, missing or invalid operand}\\
During compilation of the right hand side of a transformation function, an
operand (\emph{i.e.}\ a constant, variable or expression) was expected but was
not found.

\item \textbf{TRN\_\_MIOPR, missing or invalid operator}\\
During compilation of the right hand side of a transformation function, an
operator (or delimiter) was expected but was not found.

\item \textbf{TRN\_\_MISVN, missing variable name}\\
The variable name is missing from the left hand side of a transformation
function.

\item \textbf{TRN\_\_MLPAR, missing left parenthesis}\\
The right hand side of a transformation function has a left parenthesis
missing.

\item \textbf{TRN\_\_MRPAR, missing right parenthesis}\\
The right hand side of a transformation function has a right parenthesis
missing.

\item \textbf{TRN\_\_NDCMM, number of data coordinates mis-matched}\\
The number of input or output coordinates specified for a set of data points
does not match the corresponding number of variables associated with the
compiled mapping being used to transform the points.

\item \textbf{TRN\_\_NMVMM, number of module variables mis-matched}\\
Two adjacent transformation modules in the \name{MODULE\_ARRAY} component of
an \name{HDS} transformation structure are mis-matched because the number of
output variables from one module is not equal to the number of input
variables for the one which follows.

\item \textbf{TRN\_\_NTVMM, number of transformation variables mis-matched}\\
The number of output variables from the first of two transformations being
concatenated is not equal to the number of input variables to the second
transformation.

\item \textbf{TRN\_\_NVRIN, number of variables invalid}\\
The number of variables specified in the \name{NVAR\_IN} or \name{NVAR\_OUT}
component of an \name{HDS} transformation module (\name{TRN\_MODULE})
structure is invalid (\emph{i.e.}\ it is not positive).

\item \textbf{TRN\_\_OPCIN, operation code invalid}\\
An invalid internal arithmetic operation code has been encountered while
evaluating a transformation function which is part of a compiled mapping.
This is a serious error indicating internal inconsistency within the
\name{TRANSFORM} software.
It should be reported immediately.

\item \textbf{TRN\_\_PRCIN, precision invalid}\\
An invalid precision specification has been supplied when creating a new
transformation.

\item \textbf{TRN\_\_TOKIN, token name invalid}\\
An attempt to substitute a value for a token in a character string has failed
because an invalid token name has been specified.

\item \textbf{TRN\_\_TRNUD, transformation undefined}\\
A set of transformation functions is incomplete or is otherwise insufficient
to fully define a transformation.

\item \textbf{TRN\_\_TRUNC, character string truncated}\\
A character string has been truncated because the \name{CHARACTER} variable
supplied is not of sufficient length to accommodate it.

\item \textbf{TRN\_\_TYPIN, type invalid}\\
An \name{HDS} object has a type inappropriate for its purpose.

\item \textbf{TRN\_\_VARIN, variable name invalid}\\
A variable used in a transformation function has an invalid name.

\item \textbf{TRN\_\_VARUD, variable name undefined}\\
A variable used on the right hand side of a transformation function is
undefined because it does not appear on the left hand side of one of the
transformation functions which define the complementary mapping.

\item \textbf{TRN\_\_VERMM, software version mis-match}\\
The version number of the \name{TRANSFORM} software used to create or last
modify an \name{HDS} transformation structure exceeds the version number of
the software with which the current application is linked.

\item \textbf{TRN\_\_WRNFA, wrong number of function arguments}\\
An invalid number of arguments has been supplied for a built-in function
invoked from the right hand side of a transformation function.

\end{description}

\end{document}

% ? End of main text
\end{document}
