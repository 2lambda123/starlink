\documentstyle[11pt]{article}
\pagestyle{myheadings}

% -----------------------------------------------------------------------------
% ? Document identification
\newcommand{\stardoccategory}  {Starlink User Note}
\newcommand{\stardocinitials}  {SUN}
\newcommand{\stardocsource}    {sun124.10}
\newcommand{\stardocnumber}    {124.10}
\newcommand{\stardocauthors}   {P.\,T.\,Wallace}
\newcommand{\stardocdate}      {13th June 1995}
\newcommand{\stardoctitle}     {HLP --- Interactive Help System}
\newcommand{\stardocversion}   {v3.3}
\newcommand{\stardocmanual}    {User Guide}
% ? End of document identification
% -----------------------------------------------------------------------------

\newcommand{\stardocname}{\stardocinitials /\stardocnumber}
\markright{\stardocname}
\setlength{\textwidth}{160mm}
\setlength{\textheight}{230mm}
\setlength{\topmargin}{-2mm}
\setlength{\oddsidemargin}{0mm}
\setlength{\evensidemargin}{0mm}
\setlength{\parindent}{0mm}
\setlength{\parskip}{\medskipamount}
\setlength{\unitlength}{1mm}

% -----------------------------------------------------------------------------
%  Hypertext definitions.
%  ======================
%  These are used by the LaTeX2HTML translator in conjunction with star2html.

%  Comment.sty: version 2.0, 19 June 1992
%  Selectively in/exclude pieces of text.
%
%  Author
%    Victor Eijkhout                                      <eijkhout@cs.utk.edu>
%    Department of Computer Science
%    University Tennessee at Knoxville
%    104 Ayres Hall
%    Knoxville, TN 37996
%    USA

%  Do not remove the %begin{latexonly} and %end{latexonly} lines (used by 
%  star2html to signify raw TeX that latex2html cannot process).
%begin{latexonly}
\makeatletter
\def\makeinnocent#1{\catcode`#1=12 }
\def\csarg#1#2{\expandafter#1\csname#2\endcsname}

\def\ThrowAwayComment#1{\begingroup
    \def\CurrentComment{#1}%
    \let\do\makeinnocent \dospecials
    \makeinnocent\^^L% and whatever other special cases
    \endlinechar`\^^M \catcode`\^^M=12 \xComment}
{\catcode`\^^M=12 \endlinechar=-1 %
 \gdef\xComment#1^^M{\def\test{#1}
      \csarg\ifx{PlainEnd\CurrentComment Test}\test
          \let\html@next\endgroup
      \else \csarg\ifx{LaLaEnd\CurrentComment Test}\test
            \edef\html@next{\endgroup\noexpand\end{\CurrentComment}}
      \else \let\html@next\xComment
      \fi \fi \html@next}
}
\makeatother

\def\includecomment
 #1{\expandafter\def\csname#1\endcsname{}%
    \expandafter\def\csname end#1\endcsname{}}
\def\excludecomment
 #1{\expandafter\def\csname#1\endcsname{\ThrowAwayComment{#1}}%
    {\escapechar=-1\relax
     \csarg\xdef{PlainEnd#1Test}{\string\\end#1}%
     \csarg\xdef{LaLaEnd#1Test}{\string\\end\string\{#1\string\}}%
    }}

%  Define environments that ignore their contents.
\excludecomment{comment}
\excludecomment{rawhtml}
\excludecomment{htmlonly}

%  Hypertext commands etc. This is a condensed version of the html.sty
%  file supplied with LaTeX2HTML by: Nikos Drakos <nikos@cbl.leeds.ac.uk> &
%  Jelle van Zeijl <jvzeijl@isou17.estec.esa.nl>. The LaTeX2HTML documentation
%  should be consulted about all commands (and the environments defined above)
%  except \xref and \xlabel which are Starlink specific.

\newcommand{\htmladdnormallinkfoot}[2]{#1\footnote{#2}}
\newcommand{\htmladdnormallink}[2]{#1}
\newcommand{\htmladdimg}[1]{}
\newenvironment{latexonly}{}{}
\newcommand{\hyperref}[4]{#2\ref{#4}#3}
\newcommand{\htmlref}[2]{#1}
\newcommand{\htmlimage}[1]{}
\newcommand{\htmladdtonavigation}[1]{}
\newcommand{\latexhtml}[2]{#1}
\newcommand{\html}[1]{}

%  Starlink cross-references and labels.
\newcommand{\xref}[3]{#1}
\newcommand{\xlabel}[1]{}

%  LaTeX2HTML symbol.
\newcommand{\latextohtml}{{\bf LaTeX}{2}{{\tt HTML}}}

%  Define command to re-centre underscore for Latex and leave as normal
%  for HTML (severe problems with \_ in tabbing environments and \_\_
%  generally otherwise).
\newcommand{\latex}[1]{#1}
\newcommand{\setunderscore}{\renewcommand{\_}{{\tt\symbol{95}}}}
\latex{\setunderscore}

%  Redefine the \tableofcontents command. This procrastination is necessary 
%  to stop the automatic creation of a second table of contents page
%  by latex2html.
\newcommand{\latexonlytoc}[0]{\tableofcontents}

% -----------------------------------------------------------------------------
%  Debugging.
%  =========
%  Remove % on the following to debug links in the HTML version using Latex.

% \newcommand{\hotlink}[2]{\fbox{\begin{tabular}[t]{@{}c@{}}#1\\\hline{\footnotesize #2}\end{tabular}}}
% \renewcommand{\htmladdnormallinkfoot}[2]{\hotlink{#1}{#2}}
% \renewcommand{\htmladdnormallink}[2]{\hotlink{#1}{#2}}
% \renewcommand{\hyperref}[4]{\hotlink{#1}{\S\ref{#4}}}
% \renewcommand{\htmlref}[2]{\hotlink{#1}{\S\ref{#2}}}
% \renewcommand{\xref}[3]{\hotlink{#1}{#2 -- #3}}
%end{latexonly}
% -----------------------------------------------------------------------------
% ? Document specific \newcommand or \newenvironment commands.

\newcommand{\qt}[1]{``{\tt{#1}}''}
\begin{htmlonly}
   \newcommand{\qt}[1]{{\tt{"#1"}}}
\end{htmlonly}

\newcommand{\fstring}[1]{\hbox{\hspace{0.05em}{\qt{#1}}\hspace{0.05em}}}

% ? End of document specific commands
% -----------------------------------------------------------------------------
%  Title Page.
%  ===========
\renewcommand{\thepage}{\roman{page}}
\begin{document}
\thispagestyle{empty}

%  Latex document header.
%  ======================
\begin{latexonly}
   CCLRC / {\sc Rutherford Appleton Laboratory} \hfill {\bf \stardocname}\\
   {\large Particle Physics \& Astronomy Research Council}\\
   {\large Starlink Project\\}
   {\large \stardoccategory\ \stardocnumber}
   \begin{flushright}
   \stardocauthors\\
   \stardocdate
   \end{flushright}
   \vspace{-4mm}
   \rule{\textwidth}{0.5mm}
   \vspace{5mm}
   \begin{center}
   {\Huge\bf  \stardoctitle \\ [2.5ex]}
   {\LARGE\bf \stardocversion \\ [4ex]}
   {\Huge\bf  \stardocmanual}
   \end{center}
   \vspace{5mm}

% ? Heading for abstract if used.
  \vspace{10mm}
  \begin{center}
     {\Large\bf Abstract}
  \end{center}
% ? End of heading for abstract.
\end{latexonly}

%  HTML documentation header.
%  ==========================
\begin{htmlonly}
   \xlabel{}
   \begin{rawhtml} <H1> \end{rawhtml}
      \stardoctitle\\
      \stardocversion\\
      \stardocmanual
   \begin{rawhtml} </H1> \end{rawhtml}

% ? Add picture here if required.
% ? End of picture

   \begin{rawhtml} <P> <I> \end{rawhtml}
   \stardoccategory \stardocnumber \\
   \stardocauthors \\
   \stardocdate
   \begin{rawhtml} </I> </P> <H3> \end{rawhtml}
      \htmladdnormallink{CCLRC}{http://www.cclrc.ac.uk} /
      \htmladdnormallink{Rutherford Appleton Laboratory}
                        {http://www.cclrc.ac.uk/ral} \\
      \htmladdnormallink{Particle Physics \& Astronomy Research Council}
                        {http://www.pparc.ac.uk} \\
   \begin{rawhtml} </H3> <H2> \end{rawhtml}
      \htmladdnormallink{Starlink Project}{http://star-www.rl.ac.uk/}
   \begin{rawhtml} </H2> \end{rawhtml}
   \htmladdnormallink{\htmladdimg{source.gif} Retrieve hardcopy}
      {http://star-www.rl.ac.uk/cgi-bin/hcserver?\stardocsource}\\

%  HTML document table of contents. 
%  ================================
%  Add table of contents header and a navigation button to return to this 
%  point in the document (this should always go before the abstract \section). 
  \label{stardoccontents}
  \begin{rawhtml} 
    <HR>
    <H2>Contents</H2>
  \end{rawhtml}
  \renewcommand{\latexonlytoc}[0]{}
  \htmladdtonavigation{\htmlref{\htmladdimg{contents_motif.gif}}
        {stardoccontents}}

% ? New section for abstract if used.
\section{\xlabel{abstract}Abstract}
% ? End of new section for abstract
\end{htmlonly}

% -----------------------------------------------------------------------------
% ? Document Abstract. (if used)
%   ==================
The Starlink HLP system
allows an application program to retrieve named items from
a hierarchically-arranged library of text.
The facility resembles VAX/VMS HELP.  It consists
of a library plus
utility programs for creating, listing and reading help
libraries.
% ? End of document abstract
% -----------------------------------------------------------------------------
% ? Latex document Table of Contents (if used).
%  ===========================================
 \newpage
 \begin{latexonly}
   \setlength{\parskip}{0mm}
   \latexonlytoc
   \setlength{\parskip}{\medskipamount}
   \markright{\stardocname}
 \end{latexonly}
% ? End of Latex document table of contents
% -----------------------------------------------------------------------------
\newpage
\renewcommand{\thepage}{\arabic{page}}
\setcounter{page}{1}

\section{INTRODUCTION}
The Starlink HLP system is a set of subprograms and utilities which
allows an application program to retrieve named items from
a hierarchically-arranged library of text.

The facility is functionally
very similar to the VAX/VMS Help system.  The major differences
are that the Starlink HLP system (i)~is implemented in
a portable way and is not tied to the VAX, and (ii)~allows
independent creation of multiple libraries which
are bound together at run-time and appear to the user as a single
``tree''.  The system is written in a free-standing manner and does not call
any other Starlink packages.

The present document will be of most interest to application programmers,
though users of application packages which incorporate the HLP system
may find Section~2 of some
value.  Section~5 is
relevant only to those providing software support for the package,
or others who are interested in the internals of the system.

\subsection{Comparison with VMS Help}
Readers already familiar with the VMS Help system may find it
most convenient to begin by looking at how the Starlink HLP
system differs.  In the
list, below, the most useful or significant features are
described first, with less important differences later.
\begin{itemize}
\item The package is portable, and available from Starlink in
source form.
\item The Starlink system allows different parts of the help text
tree to be stored in separate files which are joined
at run-time, transparently.  It is possible to refer to a file
from more than one point in the hierarchy.
\item If no keyword is supplied, the VMS Help system delivers
the help information for the \fstring{HELP} topic, though this
behaviour can be suppressed.  In the Starlink HLP
system each library has a unique top-level topic which is
returned under these circumstances.
\item In the VAX version, the \fstring{...} ellipsis feature
only works for top-level topics.  In the Starlink HLP system
ellipsis can be used from any level.
\item The keyword matching capabilities are slightly more
elaborate in the Starlink version: a keyword can consist
of multiple words, separated by
underscores, each of which can be individually abbreviated.
For example, the keyword \fstring{ACTIVE\_GALACTIC\_NUCLEI} could
be abbreviated \fstring{A\_G\_N} (or \fstring{A} {\it etc}).  
\item In both systems, \fstring{\%} and \fstring{*}
wildcard characters can be used
when specifying the keyword to be searched for.  In the VMS system
these characters
must not form part of a keyword in a library.  In the Starlink
system it is possible (though not recommended) to use such
characters, their special function being
suppressed by preceding them with a \fstring{$\backslash$}
``escape'' character in the search string.
\item In the Starlink system there is no capability for specifying
different help libraries on the command line.  Any such facilities
must be provided by the application.
\item The VAX system treats a keyword beginning with the character
\fstring{/} as a special case, signifying a command qualifier.  This feature
does not exist in the Starlink HLP system;  all topics and subtopics
begin with a level number.
\item Like the VAX Help system, the Starlink HLP system ignores
records beginning with \fstring{!}.  However, the Starlink
system allows exclamation marks to appear in HLP text, whereas
the VAX system would interpret them as end-of-line comments.
\item The package includes subprogram interfaces and utility
programs only;  no equivalent of the DCL HELP command
has been implemented.
\item The Starlink system offers fewer library maintenance facilities.
In particular, a library has to be rebuilt in its entirety from source
when a single topic is changed (but note that pieces of library
can be separate files in the Starlink system).
\item The VMS routine LBR\$OUTPUT\_HELP supports five options
(through its {\it flags}\, argument) that are irrelevant for the
Starlink HLP\_HELP routine.  The only option common to both is
the one which enables and disables interactive prompting.
\item In the Starlink HLP system a series of keywords is
separated by spaces;  in the VMS Help system, \fstring{/} can also
be used.
\item There are minor differences in the handling of unusual
and unimportant conditions, {\it e.g.}\ duplicate keywords in
a library, use of punctuation characters in keywords.
\item In the VAX system, top-level topics always appear in
alphabetical order.  In the Starlink system they appear in the order
specified in the help source, as do subtopics in both systems.
\item There are small differences in the way blank lines
are handled.  In particular, multiple
blank lines preceding and following
the text for a given topic are ignored in the Starlink version.
\item There are unimportant differences in the formatting of
output text -- where blank lines occur, what keywords are
converted to uppercase, {\it etc}.
\end{itemize}

\subsection{Portability}
Although the Starlink HLP facility is
functionally similar to the one provided
with VAX/VMS, it is not tied to the VAX, makes no use of the VMS
Librarian utility and Run-Time Library and does not depend (to any
significant extent) on DEC Fortran extensions.  Written almost entirely
in ANSI-standard Fortran, the few unavoidable machine and operating
system dependencies are isolated within a small number of routines which
are supplied in different forms for different platforms.  At present,
VAX/VMS, Sun/SunOS, Sun/Solaris, DECstation/Utrix and DEC Alpha/OSF-1
variants are available (there is also a private PC/MS-Fortran version
available from the author), plus
functionally inferior but machine-independent variants.

The present document describes only the VAX and Unix versions.  The
VAX release contains also the PC versions.  More information
on portability issues can be found in Section~5.2.

\subsection{Package Contents}
The main components of the Starlink HLP system are as follows:
\begin{itemize}
\item A utility program CREHLP,
which reads a file of help text and writes a help library;
\item A command procedure, HLIB, which front-ends the
CREHLP program.
\item An object library containing, among other things:
   \begin{itemize}
   \item a subprogram, HLP\_HELP, which executes an interactive
   help session; and
   \item a subprogram, HLP\_ERRMES, which translates internal error
   codes into message strings.
   \end{itemize}
For VAX/VMS and some Unix platforms two shareable libraries are supplied
in addition to the non-shared library.  One
(HLP\_IMAGE\_ADAM.EXE on VMS) is for use with applications which run
under the ADAM software environment; the other
(HLP\_IMAGE.EXE on VMS) is for use with stand-alone applications.
There are associated link option files (HLP\_LINK.OPT and HLP\_LINK\_ADAM.OPT
for VMS, hlp\_link and hlp\_link\_adam for Unix).
\end{itemize}
Other useful items include:
\begin{itemize}
\item An example application program, TSTHLP, which allows a help
library to be interrogated.
\item An example help library, {\tt demo.hlp} (source)
and {\tt demo.shl} (library format).
\item Example implementations of the user-supplied routines which
(a)~output a line of text, (b)~obtain an interactive response and
(c)~translate a library name into a filename.  The
names of these examples are HLP\_OUTSUB, HLP\_INSUB and
HLP\_NAMETR respectively.
\item The HLP\_CREH subprogram, which carries out the translation of
a file of help text into the library format.  (The CREHLP program
is just a front-end for HLP\_CREH.)
\end{itemize}
Also supplied, but of less interest, are:
\begin{itemize}
\item A number of internal subprograms, some of which may have uses outside
the HLP system but are not officially part of the published interface
to the system,
{\it e.g.}\ some of the
character-string handling and I/O routines.
\item Command procedures for managing development of the software.
\item A utility program, LSTHLP, that lists a help library
file for diagnostic purposes, showing the various index pointers {\it etc.}
Section 5.3 includes an example of such a listing.
\end{itemize}

\section{USING HELP}
The example program TSTHLP, together with the sample library {\tt demo.shl},
shows a typical help session.
Type the following command:

\begin{tabular}{ll}
\\
{\it VAX/VMS} & {\it UNIX} \\ \\
{\tt \$ RUN HLP\_DIR:TSTHLP} &{\tt \% /star/bin/tsthlp} \\ \\
\end{tabular}

This will produce an announcement, followed by the question:
\begin{verse}
{\tt Name of help library?}
\end{verse}
Give the reply:

\begin{tabular}{ll}
\\
{\it VAX/VMS} & {\it UNIX} \\ \\
{\tt HLP\_DIR:DEMO.SHL} & {\tt /star/bin/examples/hlp/demo.shl} \\ \\
\end{tabular}

This will produce a \fstring{:} prompt; enter one of the following:
\begin{itemize}
\item $<$CR$>$ to display the top-level help text and subtopics,
\item a topic name to display the help text for the nominated topic, or
\item a period to terminate the program.
\end{itemize}
At other prompts, enter:
\begin{itemize}
\item a subtopic name to display the text for that subtopic,
\item $<$CR$>$ to move back one level in the hierarchy,
\item question mark \fstring{?} to
redisplay the text for the current topic, or
\item ellipsis \fstring{...} to display all
the text below the current point in the hierarchy.
\end{itemize}
The topic and subtopic names in the library are called
``keywords''.  When searching for a given keyword, it is unnecessary to
give the full name.  The rules for abbreviating keywords are as follows:
\begin{enumerate}
\item The simplest form of abbreviation is to give only the first few
characters.  Thus \fstring{FRED} is a valid
abbreviation of \fstring{FREDERIC}.
\item A keyword can be made up of one or more ``words'' separated by an
underscore.  Each such word can be abbreviated individually.  Thus
\fstring{A\_D} is a valid abbreviation of \fstring{ANNO\_DOMINI}.
\item The characters \fstring{\%} and \fstring{*}
match, respectively, any single
non-space character or any sequence of such characters.
\item The character \fstring{$\backslash$}
in the entered string is itself ignored but
causes the next character to be accepted literally (but not space).
This means that the various special characters can appear in keywords;
however, exploiting this feature is not recommended.
\item If a word from the entered string contains but does not end with a
\fstring{*} wildcard, it must not be truncated.
Thus, although \fstring{SAM} and \fstring{S*E} are both valid
matches for \fstring{SAMPLE}, \fstring{S*PL} is not.
This feature allows abbreviations
to be used which unambiguously specify the end of the string as well as
the beginning.
\item Neither the keyword in the library nor the string entered may
contain leading or embedded spaces.
\end{enumerate}
When the entered string is a valid abbreviation of
more than one topic but is not an
exact match for any topic, all the matched
topics are reported.  Thus \fstring{*  *  *} would display all the level-three
topics, and \fstring{...} would display every topic in the library.
When the entered string is an exact match for a topic, no further
matches are reported once the exact match has been found.

Keyword comparisons are not case-sensitive: \fstring{aBc} is a valid match for
\fstring{ABC} for example.

\section{CREATING A HELP LIBRARY}
Help files exist in two forms: (i)~the {\it source}\, form, which you
create using a text editor, and (ii)~the {\it library}\, form, which
is read by the application program when it
calls the HLP\_HELP subprogram.  The source form is read
sequentially (a two-pass operation) by the HLP\_CREH routine
to write the randomly-addressable indexed library file which
can be interrogated by means of the HLP\_HELP routine.

No facilities exist for inserting a single topic into an existing
help library.

\subsection{The Source File}
Each source file contains one hierarchy of help text.  Embedded
within the file may be pointers to other help libraries, though
it is perfectly possible and normal for all the help text for a given
application to be stored in a single source file.  Whether or not
multiple files are used, the result always appears to the user as a
single ``tree'' of help information.

Each topic within a help source file consists of lines of plain
text, preceded by a special line of text containing the {\it level
number} and the {\it keyword}.  The level number, a single decimal
digit in the first character position, can be from 0 to 9, and shows
the hierarchical level of the help text which follows.  The keyword
is a unique name for the topic, which will be specified by the
user to retrieve the text for that topic or one of its subtopics.

The source file comprises four sorts of record;  COMMENT, KEYWORD,
TEXT and END records.  All are ordinary formatted alphanumeric
records, as produced by a text editor.  Up to 132 characters
are accepted, though a maximum of 80 characters is recommended.

COMMENT records have \fstring{!} as their first character, and are ignored.

The format of a KEYWORD record is either:
\begin{verse}
{\it n~~~keyword}
\end{verse}
where {\it n}\, is the level number, or:
\begin{verse}
{\it @library~~~n~~~keyword}
\end{verse}
where {\it library} is the name of another (different) help library and
follows the \fstring{@} with no intervening spaces.  The {\it @library},
{\it n}\, and {\it keyword}\, fields are separated by one or more spaces.
Spaces before the {\it n}\, or \fstring{@} are
not allowed.  More details on level numbers and keywords are
given later.

TEXT records are just plain text, and apply to the preceding keyword.
Any that precede the first keyword are
ignored.  Blank lines before and after each group of TEXT records
are ignored.  Note that the rules for recognizing other sorts of
record mean that text records cannot begin with a decimal digit,
\fstring{@} or \fstring{END} {\it etc.}  It
is also recommended that the character \fstring{/} is avoided;
\fstring{/} is reserved for future use by Starlink and also could
cause incompatibilities with VMS Help.

The optional END record consists of the three characters \fstring{END}, in
any mixture of uppercase and lowercase.

The rules for {\it level numbers}\, are as follows:
\begin{enumerate}
\item Higher-level topics have smaller level numbers.  Thus a topic at
level $n$ embraces any material at levels $n+1$, $n+2$, {\it etc.}.
\item The order in which the keyword records appear in the source file
defines the hierarchy.  A level number one larger than the previous
one means a subtopic; a level number less than or equal to the previous one
means that the current branch has finished and we have moved
to the next branch of the given level.
\item A level number can be less than the previous one, the same,
or one more.  Increases greater than one are not allowed.
\item The first keyword record in the source file defines the
highest level topic for that file.  Each file must have exactly
one such record:  no further keyword records at that level (or higher)
are permitted.
\item Though it is conventional to start each source file with a
zero-level topic, it is permissible to start at any level; apart
from limiting the number of levels which can appear in that
library, it makes no difference.
\item Where one help library refers to another it is not necessary for
the levels to match.  It is possible, for example, for the same
library to appear twice in a tree at two different levels.
\item Irrespective of the range of levels in the
individual libraries, the complete help tree must not go beyond
level~9.
\end{enumerate}

The rules for {\it keywords}\, are as follows:
\begin{enumerate}
\item The maximum length of a keyword is 64 characters.  In practice,
keywords longer than about 20 characters are inconvenient to use
and are discouraged.
\item Keywords remain in the format in which they were entered, with
use of uppercase and lowercase preserved.  Character case is, however,
ignored when matching.
\item The wildcard, escape and abbreviation features available during
text retrieval mean that there is considerable freedom in choosing
keywords.  However, it is strongly recommended that only alphabetic
characters (uppercase and lowercase), decimal digits, dollar sign,
underscore and hyphen be used.  Prohibited or deprecated are
asterisk, percent sign, ellipsis (three or more consecutive periods), at
sign, slashes (forward or back), parentheses (left or right), and quote
marks (single or double).
\item Accessing topics whose names are valid abbreviations
of any that precede them can be awkward: it would be
unwise to have a topic PLOT following a topic PLOTXY for example.
Topics arranged in alphabetical order automatically comply with
this recommendation.
\end{enumerate}

Here is an example help source file:
\vspace{-1mm}
\begin{small}
\begin{verbatim}
   !
   !  Example help library
   !
   0 PROGRAMMING_LANGUAGES

   Programming via front-panel switches, or by plugboards, is no longer
   in fashion.  Even macho programmers now resort to describing what the
   computer is to do in terms of text which is assembled or compiled into
   machine code, or which is interpreted and executed line by line.

   1 Assemblers

   One line of assembly language used to turn into one machine instruction,
   but these days you're never quite sure.

   1 Compilers

   A compiler turns high-level code which is supposed to be machine-
   independent but isn't into machine code which definitely isn't .

   2 Fortran

   An archaic language, a fossil remnant of 1950s IBM machines.  Used
   to excellent effect by hordes of programmers round the world.  Produces
   more efficient code than anything except assembler.  Its imminent
   demise has been announced annually since about 1963.

   2 PASCAL

   Used for teaching structured programming.  Comes in various toxic
   vendor-specific flavours.

   2 C

   The most successful computer virus to date.  Great to write in.
   Produces really impressive gibberish code.  Goes wrong in all sorts
   of fun ways.

   1 Interpreters

   There's nothing quite like changing a line of code and instantly
   seeing the result.

   2 BASIC

   Revolting old-fashioned language which lots of people understand
   and use, and which runs surprisingly fast on lots of computers.

   2 Forth

   Forth combines fast execution with compact code and rapid program
   development turnaround.  Other benefits are really sensational
   gibberish code which no-one can ever understand, and a propensity
   to spectacular crashes.

   END
\end{verbatim}
\end{small}

The file {\tt HLP\_DIR:DEMO.HLP} on VAX/VMS or
{\tt /star/bin/examples/hlp/demo.shl} on Unix platforms
contains a more elaborate example.

\subsection{Creating the Library File}
To translate one or more files of help source into a library which can be
read by an application program, execute the {\tt hlib} command script:

\begin{tabular}{ll}
\\
{\it VAX/VMS} & {\it UNIX} \\ \\
{\tt \$ @HLP\_DIR:HLIB}~~{\it source~~library} &
   {\tt \% hlib}~~{\it source\_1~~source\_2}~~\ldots \\ \\
\end{tabular}

where {\it source..}\ is a file of help text to be
input and (on VAX/VMS
platforms only) {\it library}\, is the library file to be written.

On {\bf VAX/VMS}, the default file extensions are {\tt .HLP} for the
source file and {\tt .SHL} for the library file.  A wildcard in the name
of the source file will cause each such file to be processed
individually.  If the first argument is omitted it defaults to
\fstring{*}.  If the second argument is omitted then
{\it library}\, = {\it source}\, with
extension changed to {\tt .SHL}.  A wildcarded name in the second argument
defaults to the name field of the first filename.  Typical uses are
as follows.

To translate source {\tt EXAMPLE.HLP} into library file {\tt EXAMPLE.SHL}:
\begin{verse}
{\tt \$ @HLP\_DIR:HLIB EXAMPLE}
\end{verse}
To translate all {\tt *.HLP} source files in the current directory
into {\tt *.SHL} library files:
\begin{verse}
{\tt \$ @HLP\_DIR:HLIB}
\end{verse}
To translate all {\tt *.HLP} source files in the current directory into
{\tt *.SHL} library files in directory {\tt [.SUB]}:
\begin{verse}
{\tt \$ @HLP\_DIR:HLIB * [.SUB]*}
\end{verse}

On {\bf Unix} platforms, less flexibility is provided.  The library
file always has file extension {\tt .shl}.  A series of source file names
may be specified, perhaps using a wildcard, and each one will be
translated into an appropriately-named {\tt .shl} file.  Typical uses are
as follows.

To translate source {\tt example.hlp} into library file {\tt example.shl}:
\begin{verse}
{\tt \% hlib example.hlp}
\end{verse}
To translate all {\tt *.hlp} source files in the current directory
into {\tt *.shl} library files:
\begin{verse}
{\tt \% hlib *.hlp}
\end{verse}

Programmers wishing to integrate help library creation into their
application packages may use the HLP\_CREH subprogram:
\begin{verse}
{\tt  CALL HLP\_CREH (NAMETR, LUIN, SOURCE, LUOUT, LIB, LUERR, EOS, JSTAT)}
\end{verse}
where the arguments are as follows:

\begin{tabular}{llll}
\\
Given: \\
& {\tt NAMETR} & {\tt EXTERNAL} &
              subroutine to translate library names into filenames \\
& {\tt LUIN} & {\tt INTEGER} &
                           I/O unit number for reading help source \\
& {\tt SOURCE} & {\tt CHARACTER*(*)} &
                               filename for help source, or spaces \\
& {\tt LUOUT} & {\tt INTEGER} &
                          I/O unit number for writing help library \\
& {\tt LIB} & {\tt CHARACTER*(*)} &
                              filename for help library, or spaces \\
& {\tt LUERR} & {\tt INTEGER} &
                                I/O unit number for error messages \\
& {\tt EOS} & {\tt CHARACTER*1} &
                                 character to use as end-of-string \\ \\
Returned: \\
& {\tt JSTAT} & {\tt INTEGER} &
                                         status: $0=$OK, $-9$=fail \\ \\
\end{tabular}

The conventional EOS value in all present implementations is {\tt CHAR(0)}.

\section{RETRIEVING HELP TEXT}
Once a help library has been created, it can be interrogated by an
application program through the HLP\_HELP subprogram.  The programmer
supplies the following four things:
\begin{itemize}
\item The call to HLP\_HELP.
\item A subprogram which handles one line of retrieved help text,
typically displaying it in some way.
\item A subprogram which provides one line of interactive input.
\item A subprogram which translates a help library name into a filename.
\end{itemize}

The HLP package includes a demonstration application TSTHLP.FOR,
together with subprograms HLP\_OUTSUB.FOR, HLP\_INSUB.FOR and HLP\_NAMETR.FOR.

\subsection{The HLP\_HELP Subprogram}
The HLP\_HELP routine takes a line of text containing commands
or keywords and enters an interactive help session, using supplied
application-specific routines to obtain further command and keyword
input and to handle retrieved text.  Once the help session is complete,
control passes back to the calling program.

The usual way to generate the initial line of input will be to
interpret the application's own help command and to present the
arguments appropriately formatted.  For example, the application may
respond to a command \fstring{HELP COMP C} by calling HLP\_HELP with
\fstring{COMP C} as the initial response string.  The HLP\_HELP routine
will then search for the subtopic \fstring{C} of topic \fstring{COMP},
hand back to the application any text it finds, request further lines
of input, and allow the user to explore the help tree before finally
terminating and returning to the application.

The response strings accepted at each stage are given in
Section~2.

The HLP\_HELP routine is an integer function subprogram, which
returns the number $+1$ to indicate successful completion and a
range of negative integers to indicate various error conditions
(see Section~4.6).  The call is as follows:
\begin{verse}
{\tt ISTAT = HLP\_HELP (OUTSUB, LOUT, INLINE, LU, LIB, JFLAGS, INSUB, NAMETR)}
\end{verse}
where the arguments are:

\begin{tabular}{llll}
\\
Given: \\
& {\tt OUTSUB} & {\tt EXTERNAL} &
                   user-supplied output subroutine (note 2, below) \\
& {\tt LOUT} & {\tt INTEGER} &
                          maximum record length accepted by OUTSUB \\
& {\tt INLINE} & {\tt CHARACTER*(*)} &
                     string specifying required help text (note 3) \\
& {\tt LU} & {\tt INTEGER} &
                     I/O unit number for reading help library file \\
& {\tt LIB} & {\tt CHARACTER*(*)} &
                                name of help library file (note 4) \\
& {\tt JFLAGS} & {\tt INTEGER} &
                                                    flags (note 5) \\
& {\tt INSUB} & {\tt EXTERNAL} &
                  user-supplied interactive input routine (note 6) \\
& {\tt NAMETR} & {\tt EXTERNAL} &
                   user-supplied name translation routine (note 7) \\ \\
Returned: \\
& {\tt HLP\_HELP} & {\tt INTEGER} &
                              status:  $+1=$ OK \\
& & & \hspace{2.6em}               $-11=$ illegal current level \\
& & & \hspace{2.6em}               $-12=$ OUTSUB reported error \\
& & & \hspace{2.6em}               $-13=$ INSUB reported error \\
& & & \hspace{2.75em}             else = other errors (note 8) \\ \\
\end{tabular}

Notes:
\nopagebreak
\begin{enumerate}
\item This routine is similar, but not identical, in its argument list
and action to the VAX/VMS routine LBR\$OUTPUT\_HELP.
\item The user-supplied OUTSUB routine is responsible for
knowing where to
write the information, how to handle pagination, and so on.  Details
are given in Section~4.2, below.
\item The INLINE string contains the initial series of help keywords,
separated by spaces.  The interactive help session will begin at the
place in the help tree so specified.  INLINE may contain leading as well as
embedded and trailing spaces.  The command itself, for example
\fstring{HELP}, is not included in this string, nor any command
qualifiers which the application supports.  A maximum of 9 keywords is
accepted;  any more are ignored.  Keywords longer than 64 characters
are truncated.
\item The help library specified by LIB is in a special format, produced by
the HLP\_CREH routine
(as described in Section~3.2).  The system is not compatible with
VMS .HLB files, though the source form of the library is very
similar (see Section~3.1).  The name is subject to environment-dependent
translation at open time through the NAMETR routine (see
section 4.4).
\item At present, JFLAGS controls only one option, though more
may be added in the future.  JFLAGS=1 means that
interactive help prompting is in effect, while JFLAGS=0
value means that the help text for the topic specified in
INLINE is looked up but no interactive dialogue ensues.
\item The user-supplied INSUB routine is responsible for
carrying out a prompted read from the interactive command device.
Details are given in Section~4.3, below. (Note that if JFLAGS=0
INSUB is never called.)
\item The user-supplied NAMETR routine is responsible for translating
a help library name into a filename.  Details are given in Section 4.4,
below.
\item The status values returned by this routine may be translated into
text by means of the routine HLP\_ERRMES.  See Section 4.6.
\end{enumerate}

\subsection{The Output Subprogram}
The user-supplied output routine, which is named OUTSUB in the above
example call to HLP\_HELP but can have any name,
is an integer function subprogram which
accepts one argument, the string to be output, and returns a
status of $+1$ if OK.  The output routine is responsible for knowing where to
write the information, how to handle pagination, and so on.

The HLP package contains a simple example of an output routine.  It
is called HLP\_OUTSUB and is the one used in the TSTHLP demonstration
program.

Note that there is no obligation for applications simply to display
the lines of text retrieved by the help system.  For example,
information could be encoded into the help text which allows the
application to plot graphs, log usage information, change the
display colour {\it etc}.

\subsection{The Input Subprogram}
The user-supplied interactive input routine, which has the name INSUB
in the above
example call to HLP\_HELP but which can have any name, is an
integer function subprogram with arguments STRING, PROMPT, L.
The argument STRING
receives the line input, PROMPT is the string to output prior to
reading the line, and L the number of characters input.  If the
call is successful, a function value of $+1$ is returned.

The HLP package contains a simple example of an input routine.  It
is called HLP\_INSUB and is the one used in the TSTHLP demonstration
program.

\subsection{The Name Translation Subprogram}
Help libraries have two names:
\begin{itemize}
\item The name specified either through the {\tt LIB} argument of the
HLP\_HELP call or following \fstring{@} in the help source file.
\item The filename required by Fortran OPEN statements.
\end{itemize}
It is the job of the user-supplied
NAMETR routine to translate the first form into the second.

The call is as follows:
\begin{verse}
{\tt CALL nametr (KMD, INSTR, OUTSTR, JSTAT)}
\end{verse}
where the arguments are:

\begin{tabular}{llll}
\\
Given: \\
& {\tt KMD} & {\tt CHARACTER*(*)} & function number (see below) \\
& {\tt INSTR} & {\tt CHARACTER*(*)} & given string \\ \\
Returned: \\
& {\tt OUTSTR} & {\tt CHARACTER*(*)} & returned string \\
& {\tt JSTAT} & {\tt INTEGER} & status:  $0=$ OK \\
& & & \hspace{1.7em}         $-16=$ a string had to be truncated \\
& & & \hspace{1.7em}         $-17=$ other error \\ \\
\end{tabular}

The function number {\tt KMD} is always zero when the NAMETR routine
is called by the HLP system, and means ``{\tt INSTR} is a help
library name; translate it into a filename {\tt OUTSTR}''.  Non-zero
{\tt KMD} values are available to implementors for their own
purposes.

The HLP package contains a suitable NAMETR routine, called
HLP\_NAMETR.  The translation it provides is to
enclose the supplied name within a prefix and suffix.  This
simple transformation adequately supports applications
which must run on multiple platforms.
The prefix and
suffix are originally spaces, which HLP\_NAMETR treats as meaning
``no translation required''.  The full set of options supported by
HLP\_NAMETR is as follows:

\begin{tabular}{llll}
\\
{\it KMD} & {\it INSTR} & {\it OUTSTR} & {\it action} \\ \\
~~0 & library name & filename & translate library name to filename \\
~~1 & prefix & -- & specify prefix \\
~~2 & suffix & -- & specify suffix \\
~~3 & -- & prefix & enquire prefix \\
~~4 & -- & suffix & enquire suffix \\ \\
\end{tabular}

This is what the application must do:
\begin{verbatim}
        :
*  Routines used by HLP_HELP
      EXTERNAL ...,hlp_NAMETR
        :
        :
*  Define prefix and suffix for help name translations
      CALL HLP_NAMETR(1,PREFIX,DUMMY,J)
      IF (J.NE.0) ... error
      CALL HLP_NAMETR(2,SUFFIX,DUMMY,J)
      IF (J.NE.0) ... error
        :
        :
*  Perform help session
      CALL HLP_HELP(...,HLP_NAMETR)
        :
\end{verbatim}
An example prefix and suffix might be \fstring{/star/help/tpoint/} and
\fstring{.shl} respectively.  A library name \fstring{model} would
then become the filename \fstring{/star/help/tpoint/model.shl}.

If this simple prefix+name+suffix model is not adequate, the programmer
is free to supply his own translation routine, perhaps using
environment variables or an internal table.

\subsection{Terminal Handling}
The example application TSTHLP works in the simplest way possible.  Most
real-life applications will be more sophisticated:
\begin{itemize}
\item The output routine HLP\_OUTSUB used in
TSTHLP outputs lines of text to the terminal without
regard to screen management.  The output routines of real
applications may count lines and issue a ``press return for more''
message after a certain number, then wait for a response before going
on; on appropriate terminals there may be screen erasures; and
so on.
\item Some sort of ``abort'' feature is useful, to provide a
quick exit from the help session.  On the VAX,
detection of $<$CTRL$>$/Z in the
application's input routine (by means of the {\tt END=} feature
of the {\tt READ} statement), followed by simulation of multiple
$<$CR$>$ responses and ignoring of lines supplied to the output routine, is
one way to arrange this.
\item In the VMS DCL command HELP, a \fstring{?} response requires no
$<$CR$>$, and it is possible an application may wish to provide this
feature.  (The demonstration program TSTHLP does not.)
\end{itemize}

\subsection{Error Codes}
The HLP\_CREH routine and all internal subprograms
return status information through an integer argument using the
following codes:

\begin{tabular}{rl}
\\
   $0=$ & OK \\
  $-1=$ & Help system in wrong state \\
  $-2=$ & Help library error on OPEN \\
  $-3=$ & Help library error on WRITE \\
  $-4=$ & Help library error on READ \\
  $-5=$ & Help file error on CLOSE \\
  $-6=$ & Attempt to WRITE outside HELP library \\
  $-7=$ & Attempt to READ outside HELP library \\
  $-8=$ & Help record overflows supplied string \\
  $-9=$ & Help library creation failure \\
 $-10=$ & Unused \\
 $-11=$ & HLP\_HELP internal error \\
 $-12=$ & Line output failure \\
 $-13=$ & Line input failure \\
 $-14=$ & Invalid index entry \\
 $-15=$ & Attempted switch to current library \\
 $-16=$ & String too small for file name \\
 $-17=$ & Name translation failed \\ \\
\end{tabular}

HLP\_HELP returns a function value of $+1$ rather than 0 to indicate
success, for compatibility with the VMS HELP system.  HLP\_HELP may also
return certain of the error values from the above list.

Most of these error conditions can only occur as a result of
bugs in the HLP software.  The most common one to arise during
normal use is \fstring{HELP library error on OPEN} -- for example where
an incorrect library name has been supplied.

A status value {\tt J} returned by the HLP routines may be translated
into a message string {\tt MES} by means of the HLP\_ERRMES routine:
\begin{verse}
{\tt CALL hlp\_ERRMES(J,MES)}
\end{verse}
The length of the {\tt MES} string should be at least 50~characters.

\subsection{Linking}
ADAM tasks will be linked with the HLP library automatically.
Other applications which call HLP\_HELP, HLP\_ERRMES, HLP\_CREH,
or any other of the other public
routines in the HLP package may be linked as follows.

On VAX/VMS:
\begin{verse}
{\tt \$ LINK program, HLP\_LINK/OPT}
\end{verse}

On the Unix platforms:\begin{verse}
{\tt \% f77 program.o -L/star/lib `hlp\_link` -o program.out}
\end{verse}
(The above assumes that all Starlink directories
have been added to the environment variables {\tt PATH} and
{\tt LD\_LIBRARY\_PATH} as described in SUN/118.)

\section{SOFTWARE SUPPORT GUIDE}
This section is intended for support programmers and those interested
in the internal workings of the HLP package.  It is unlikely to
interest application programmers or users.

\subsection{Release Contents}
The complete HLP system, for all platforms,
is stored in a single VAX/VMS directory.
The following files are included:

\begin{tabular}{ll}
\\
 *.FOR & Fortran source (machine-independent) \\
 *.BAT & various PC MS/DOS command procedures (PC only)\\
 *.COM & various VAX/VMS DCL command procedures \\
 *.OPT & various VAX/VMS link option files \\
 *.MAR & transfer vector for VMS shareable images \\
 *.IND & Fortran source (machine-independent) \\
 *.VAX & Fortran source (VAX/VMS etc) \\
 *.PCM & Fortran source (PC -- Microsoft Fortran under MS/DOS) \\
 *.SUN4 & Fortran source (Sun SPARC etc) \\
 *.EXE & Utility programs -- VAX executables \\
 *.OLB & VAX object module library \\
 READ.ME & General information \\
 HLP.NEWS & NEWS item for current release \\
 *.TEX,*.TOC & \LaTeX\ source for document \\
 MAKEFILE. & make file for Unix platforms \\
 MK. & mk script, which sets environment variables and runs make \\
 HLIB. & Unix shell script for running CREHLP \\
 HLP\_LINK*. & Part of Unix link procedure. \\
 VAX\_TO\_UNIX.USH & Unix shell script which accepts the release. \\ \\
\end{tabular}

\subsection{Portability Issues}
The bulk of the system is coded in ANSI-standard Fortran.  The only
VAX extensions used are the SGP/16-sanctioned
INCLUDE, DO~WHILE, DO~...~END~DO and names longer than
6 characters.  Full ANSI-compliance would be achieved by including
the INCLUDEd text, recoding
all the DOs and stripping the prefix \fstring{HLP\_} from
external names.

Specific machine dependencies are as follows.  Filenames are given
in their uppercase VAX/VMS forms, but are always lowercase on
the Unix platforms.
\begin{itemize}
\item COMIC -- Two versions of this INCLUDE file are supplied.
{\tt COMIC.FOR} is ANSI standard except that underscore, percent and
backslash appear in character strings.  {\tt COMIC.SUN4} has a double
backslash to override the ``escape'' mechanism used by the Fortran
compilers found on most Unix platforms.
\item FOPR -- Two versions are supplied.  {\tt FOPR.VAX} uses
the OPEN keyword {\tt READONLY} to avoid requiring
write access to the help library;  it is used on the VAX and
the DECstation.  The PC version uses the OPEN option
{\tt MODE='READ'} for the same
purpose.  The machine-independent version {\tt FOPR.IND} uses a
standard OPEN;  this is the version used on the Sun.
\item INSUB -- Three versions are supplied.  The VAX version
{\tt INSUB.VAX} uses
the format descriptor {\tt \$} to suppress the newline after the
prompt, and begins each line with a blank to act as a printer
control code.  The PC version {\tt INSUB.PCM} is similar, but uses the
format descriptor $\backslash$ to suppress the newline.
{\tt INSUB.SUN4}, which works on both the Sun and DECstation,
like the VAX version uses {\tt \$}, but does not output
a blank.
With the machine-independent version {\tt INSUB.IND}, a newline (unavoidably)
follows the prompt.
\item CREHLP -- Three versions are supplied.  {\tt CREHLP.VAX} is essentially
VAX-only, and works in conjunction with the {\tt HLIB.COM} command procedure.
{\tt CREHLP.IND} works on the PC and may work on other machines;  however,
it uses explicit I/O unit numbers, which may need to be changed.
{\tt CREHLP.SUN4} works on the Sun and DECstations and may work on other Unix
platforms;  however, it uses explicit I/O unit numbers and also the
filenames {\tt fort.1} and {\tt fort.2} (as required by the {\tt hlib}
script).
\item LSTHLP -- An explicit I/O unit number is used, which may need to
be changed on some machines.
\item OUTSUB -- Two versions are supplied.
{\tt OUTSUB.IND}, which is used on VAX and PC, outputs
a blank printer control code. {\tt OUTSUB.SUN4} doesn't, and is
used on the Sun and DECstation.
In all versions, an explicit I/O unit number is used,
which may need to be changed on some machines.
\item TSTHLP -- An explicit I/O unit number is used, which may need to
be changed on some machines.
\item UPCASE -- The algorithm depends on ASCII coding, in that
a--z is assumed to exist and be in the same collating order as A--Z.
\item NAMETR -- The routine supplied, HLP\_NAMETR, either leaves
the name unchanged or adds a prefix and suffix.  Some
other behaviour may be more approriate or convenient.
\end{itemize}

The VAX, Sun SPARCstation and DECstation versions of the HLP system
are supported by the Starlink Project.  Versions for other Unix
platforms will be produced if and when they are needed and will
also be Starlink-supported.  The PC version is supported by the author.

\subsection{Library Format}
A help library file (as opposed to a help source file)
consists of three regions:
\begin{enumerate}
\item The {\it header}\, region, a single record, gives the size of the
file, in characters, as a decimal number in {\tt I12.12} format.
\item The {\it index}\, region, terminated by an empty record.
\item The {\it data}\, region, lines of help
text interspersed with keyword records.  The text region, and
the whole file, are terminated by an empty record.
\end{enumerate}

For a more detailed picture, it is best to look at an
actual file, and one is listed below.  It is the
example help source file given in Section~3.1, translated
into a help library with the HLIB procedure, and listed using the
LSTHLP utility.
The first column of the listing is the character address.  All
records are terminated with the chosen ``EOS'' character,
conventionally the null {\tt CHAR(0)}.

The file begins with the header record, in this case giving the
total length of the file as 1937 characters.
The records at character 13, 67, \ldots\ 348 are the index
region of the file, terminated by the empty record at character 386.
The records at character 387, 411, \ldots\ 1912 are the
data region of the file, terminated by the
empty record at character 1936.

The various fields of each index record are separated by spaces.
Each record begins with three pointers in {\tt I9.9} format.
The first pointer indicates the position of
the item in the data region of the file.  The second pointer
indicates the position of
the next index entry down the branch of the help tree.  The
third pointer indicates the position of the next entry of the
same level.
In this simple library, which does not contain references to
other libraries, the three pointers are followed by the
level number and the keyword.  Where the library refers to
another, the name of the referred-to library, prefixed with an \fstring{@},
is included prior to the level number;  an example of this is
given later.

\begin{small}
\begin{verbatim}
    0   000000001937
   13   000000387 000000067 000000386 0 PROGRAMMING_LANGUAGES
   67   000000686 000000110 000000110 1 Assemblers
  110   000000812 000000197 000000152 1 Compilers
  152   000001473 000000310 000000386 1 Interpreters
  197   000000955 000000237 000000237 2 Fortran
  237   000001223 000000276 000000276 2 PASCAL
  276   000001324 000000152 000000152 2 C
  310   000001572 000000348 000000348 2 BASIC
  348   000001709 000000386 000000386 2 Forth
  386   
  387   0 PROGRAMMING_LANGUAGES
  411   Programming via front-panel switches, or by plugboards, is no longer
  480   in fashion.  Even macho programmers now resort to describing what the
  550   computer is to do in terms of text which is assembled or compiled into
  621   machine code, or which is interpreted and executed line by line.
  686   1 Assemblers
  699   One line of assembly language used to turn into one machine instruction,
  772   but these days you're never quite sure.
  812   1 Compilers
  824   A compiler turns high-level code which is supposed to be machine-
  890   independent but isn't into machine code which definitely isn't .
  955   2 Fortran
  965   An archaic language, a fossil remnant of 1950s IBM machines.  Used
 1032   to excellent effect by hordes of programmers round the world.  Produces
 1104   more efficient code than anything except assembler.  Its imminent
 1170   demise has been announced annually since about 1963.
 1223   2 PASCAL
 1232   Used for teaching structured programming.  Comes in various toxic
 1298   vendor-specific flavours.
 1324   2 C
 1328   The most successful computer virus to date.  Great to write in.
 1392   Produces really impressive gibberish code.  Goes wrong in all sorts
 1460   of fun ways.
 1473   1 Interpreters
 1488   There's nothing quite like changing a line of code and instantly
 1553   seeing the result.
 1572   2 BASIC
 1580   Revolting old-fashioned language which lots of people understand
 1645   and use, and which runs surprisingly fast on lots of computers.
 1709   2 Forth
 1717   Forth combines fast execution with compact code and rapid program
 1783   development turnaround.  Other benefits are really sensational
 1846   gibberish code which no-one can ever understand, and a propensity
 1912   to spectacular crashes.
 1936    
\end{verbatim}
\end{small}

Where a help library includes references to other libraries, index
entries like the following one appear:

\begin{verbatim}
   91   000002791 000000149 000000149 @cmnds 1 Commands
\end{verbatim}

The corresponding keyword entry in the data region of the file is
normal in appearance except that there are no text records.

\subsection{The INCLUDE files}
The files
{\tt helpic} and {\tt comic}
(on the Unix platforms -- {\tt HELPIC.FOR} and {\tt COMIC.FOR} on the VAX,
{\tt HELPIC} and {\tt COMIC} on the PC) contain definitions and
COMMON declarations.  They are included through \fstring{INCLUDE 'helpic'}
and \fstring{INCLUDE 'comic'} statements, the 
interpretation of which is machine-dependent.
The {\tt comic} INCLUDE is used only by the \fstring{HLP\_COMSTR}
routine and exists in different forms for different platforms.

\subsection{Subprograms}
This section is a classified summary of the subprograms
which make up the HLP package.  For more complete information,
including argument data types and character string lengths, it will
be necessary to refer to the source code itself.

\subsubsection{Supported interfaces}
The following three routines, plus the templates described in the
next section, constitute the supported interface
to the HLP package.  Use of any other HLP\_ routines, or of the
COMMON blocks used by the HLP system, is at the
programmer's own risk.
\begin{verse}
{\tt SUBROUTINE HLP\_ERRMES(J,MES)}
\end{verse}
HLP\_ERRMES translates a HLP system error code {\tt J}
into a message string {\tt MES}.  See Section~4.6.

\begin{verse}
{\tt SUBROUTINE HLP\_CREH(NAMETR,LUIN,SOURCE,LUOUT,LIB,LUERR,EOS,JSTAT)}
\end{verse}
HLP\_CREH translates a file of help source {\tt SOURCE},
on I/O unit {\tt LUIN}, into a help library {\tt LIB}, subject
to name translation by the routine {\tt NAMETR}, on I/O
unit {\tt LUOUT}.  Any error messages go to I/O unit {\tt LUERR}.
{\tt EOS} is the end-of-string character, conventionally {\tt CHAR(0)}.
{\tt JSTAT} is the status.  See Section~3.2.

\begin{verse}
{\tt INTEGER FUNCTION HLP\_HELP(OUTSUB,LOUT,INLINE,LU,LIB,JFLAGS,INSUB)}
\end{verse}
HLP\_HELP is the main interface between an application and the
HLP system, and carries out a complete interactive help session.
Full details are given in Section~4.1.

\subsubsection{Templates}
\begin{verse}
{\tt SUBROUTINE HLP\_NAMETR(KMD,INSTR,OUTSTR,JSTAT)}
\end{verse}
This implementation of the user-supplied name translation routine is
suitable for use in a multiple-platform environment.  It forms each
filename by sandwiching the library name within a prefix and a suffix.
HLP\_NAMETR, or a user-supplied equivalent, is called by the
HLP system prior to opening any help library, always
with {\tt KMD=0}.  The
library name to be translated is supplied in {\tt INSTR}, and
the filename is returned in {\tt OUTSTR} ready for use in a
Fortran OPEN statement.
See Section~4.4 for further details of the HLP\_NAMETR routine
supplied, which uses other {\tt KMD} values to set and enquire the
prefix and suffix strings.

\begin{verse}
{\tt INTEGER FUNCTION HLP\_INSUB(STRING,PROMPT,L)}
\end{verse}
This is a simple example of the user-supplied input
routine which is to be called by the HLP\_HELP routine each
time an interactive response is required.  {\tt STRING} is the response
received. {\tt PROMPT} is the prompt string,
output before {\tt STRING} is solicited.  {\tt L} is the length
of {\tt STRING}, excluding any trailing spaces.  The function
returns the value $+1$ to indicate success.  Note that the order of
the arguments does not comply with Starlink standards (see SGP/16),
a consequence of maintaining a family-resemblance with VMS Help.  See
Section~4.3.  As well as being a template, HLP\_INSUB is used in
the LSTHLP and TSTHLP programs.

\begin{verse}
{\tt INTEGER FUNCTION HLP\_OUTSUB(STRING)}
\end{verse}
This is a simple example of the user-supplied output routine which is
to be called by the HLP\_HELP routine each time a line of help text
is to be output.
{\tt STRING} is the string to be output.  The function
returns the value $+1$ to indicate success.  See Section~4.2.
As well as being a template, HLP\_OUTSUB is called in
the LSTHLP and TSTHLP programs.

\subsubsection{String handling}
The following routines overlap in function with ones found
in other Starlink packages, but are included in the HLP package for
independence.

\begin{verse}
{\tt LOGICAL FUNCTION HLP\_COMSTR(FULSTR,STR)}
\end{verse}
HLP\_COMSTR compares two keywords for agreement, under the HLP
package's rules for abbreviation (see Section~2).  {\tt FULSTR}
is the full keyword read from the HLP library; {\tt STR} is
the abbreviated string, including wildcards {\it etc}, supplied
by the user.  A .TRUE.\, response is returned if the
two strings match.

\begin{verse}
{\tt SUBROUTINE HLP\_DEC(STRING,IPTR,NUM)}
\end{verse}
HLP\_DEC decodes a decimal integer {\tt NUM}
from a string {\tt STRING}, starting at position {\tt IPTR}.
The pointer {\tt IPTR} is incremented ready for the next call.

\begin{verse}
{\tt INTEGER FUNCTION HLP\_LENGTH(STRING)}
\end{verse}
The HLP\_LENGTH function returns the
length of a string {\tt STRING} excluding any
trailing spaces.  An all-blank string is length 1.  The routine is
optimized for speed in the case where the string is mostly trailing
spaces.

\begin{verse}
{\tt SUBROUTINE HLP\_SPLIT(STRING,ISTART,IFROM,ITO)}
\end{verse}
HLP\_SPLIT splits up a sentence into words.
{\tt STRING} consists of words separated by spaces.  Starting
at character position {\tt ISTART}, this routine locates the
beginning and end of the next word {\tt IFROM} and {\tt ITO}.

\begin{verse}
{\tt SUBROUTINE HLP\_UPCASE(STRING)}
\end{verse}
HLP\_UPCASE converts a string {\tt STRING} to uppercase, in place.

\subsubsection{Variable-length random-access file package}
help libraries are standard Fortran direct-access files, which have
a fixed record length, on which the system implements a
variable-length record scheme.  The following routines manage
these files.  They communicate in part through labelled
COMMON.

\begin{verse}
{\tt SUBROUTINE HLP\_HCLOSE(JSTAT)}
\end{verse}
HLP\_HCLOSE closes the current help library.  {\tt JSTAT} is the
status return.

\begin{verse}
{\tt SUBROUTINE HLP\_HDREAD(IADR,STRING,NC,JSTAT)}
\end{verse}
HLP\_HDREAD performs a direct-access read from the current help library.
{\tt IADR} is the character address from which the read is to
begin.  {\tt STRING} is the record that is found there, and
{\tt NC} its length.  {\tt JSTAT} is the status.

\begin{verse}
{\tt SUBROUTINE HLP\_HDWRIT(STRING,IADR,JSTAT)}
\end{verse}
HLP\_HDWRIT performs a direct-access write to the current help library.
{\tt IADR} is the character address from which the write is to begin.
{\tt STRING} is the record to be written.  {\tt JSTAT} is the status.

\begin{verse}
{\tt SUBROUTINE HLP\_HINIT(LU,FNAME,EOS)}
\end{verse}
HLP\_HINIT initializes the HLP system labelled COMMON blocks.
{\tt LU} is the I/O unit number to be used for accessing the
help library file.  {\tt FNAME} is the initial help library name.
{\tt EOS} is the special character that will signify end-of-string,
conventionally {\tt CHAR(0)}.

\begin{verse}
{\tt SUBROUTINE HLP\_HOPENR(NAMETR,JSTAT)}
\end{verse}
HLP\_HOPENR opens or re-opens the current help library for reading,
with the assistance of the library name to filename translation
routine {\tt NAMETR}.
{\tt JSTAT} is the status.

\begin{verse}
{\tt SUBROUTINE HLP\_HOPENW(NAMETR,NCHARS,JSTAT)}
\end{verse}
HLP\_HOPENW opens the current help library for writing,
with the assistance of the library name to filename translation
routine {\tt NAMETR}.
{\tt NCHARS}
is the length of the file, which must be known before the call.
{\tt JSTAT} is the status.

\subsubsection{Indexed-sequential package}
The HLP system uses its variable-length direct-access file
system, described above,
to implement the indexed-sequential structure described
in Section~5.3.  The file has an index region and a
data region.  The index region contains three sets of pointers,
to the data, to the next index entry down the branch, and the
next index entry at the same level in the current branch.

\begin{verse}
{\tt SUBROUTINE HLP\_HREADD(STRING,NC,JSTAT)}
\end{verse}
HLP\_HREADD performs a sequential-access read from the data
region of the help library.  The returned string
{\tt STRING} has length {\tt NC}.  {\tt JSTAT} is the status.

\begin{verse}
{\tt SUBROUTINE HLP\_HREADX(NAMETR,NAVIG,STRING,NC,JSTAT)}
\end{verse}
HLP\_HREADX reads the help library index, leaving
the sequential-access addresses pointing to the indexed record.
If it is necessary to translate a library name into a filename,
the routine {\tt NAMETR} will be called.
{\tt NAVIG} controls which track through the library is
required -- down the branch or along a level.  {\tt STRING} is
the record input and {\tt NC} its length.  {\tt JSTAT} is the
status.

\begin{verse}
{\tt SUBROUTINE HLP\_HSEEKX(FNAME,IADRX,LOGLEV)}
\end{verse}
HLP\_HSEEKX positions the index of the help library {\tt FNAME}
for a sequential
access (using the HLP\_HREADX routine) starting at character address
{\tt IADRX}.  The caller also supplies {\tt LOGLEV},
the {\it logical level}, which specifies the hierarchical level,
on this occasion, of the help subtree contained in the library.

\begin{verse}
{\tt SUBROUTINE HLP\_HTELLX(FNAME,IADRX,LOGLEV)}
\end{verse}
HLP\_HTELLX inquires the help library's current name, index address
and logical level.

\subsubsection{Internal}
The following are highly specialized internal routines, with no
significant potential use outside the HLP system (and which
are in any case
inaccessible to users of the shareable libraries on VMS).
For further details, see the source code.
\begin{verse}
{\tt SUBROUTINE HLP\_HCHKL(RECORD,LEVEL,NAME)}
\end{verse}
\begin{verse}
{\tt SUBROUTINE HLP\_HLEAP(NAMETR,STRING,FNAME,IADR,LOGL,JSTAT)}
\end{verse}
\begin{verse}
{\tt SUBROUTINE HLP\_LINOUT(OUTSUB,LOUT,INDENT,BUFFER,JSTAT)}
\end{verse}
\begin{verse}
{\tt SUBROUTINE HLP\_REPSUB(NAMETR,OUTSUB,LOUT,LEVCUR,NAME,OUTBUF,HLPBUF,ISTAT)}
\end{verse}
\begin{verse}
{\tt SUBROUTINE HLP\_FOPR (NAMETR,LU,FILE,LREC,JSTAT)}
\end{verse}

\subsection{Rebuilding the System}
On the VAX, the DCL command procedure CREATE.COM in the release
directory rebuilds the object
library, compiles and links the executable programs, generates
the demonstration help library from source, and produces the
Unix release.  The latter consists of an archive file, a {\tt makefile},
and a script {\tt mk} which sets up environment variables and
then runs make.  For details about the targets supported, execute
{\tt mk help}.

\subsection{VAX/VMS Logical Names}
On VAX/VMS, in order to run applications using HLP the following
logical names must be assigned:

\begin{tabular}{lll}
\\
 {\it logical name} & {\it points to} & {\it typical assignment} \\ \\
 HLP\_DIR & HLP software & {\tt STARDISK:[STARLINK.LIB.HELP]} \\
 HLP\_IMAGE & shareable library (non-ADAM) &
                               {\tt HLP\_DIR:HLP\_IMAGE.EXE} \\
 HLP\_IMAGE\_ADAM & shareable library (ADAM) &
                                 {\tt HLP\_DIR:HLP\_IMAGE\_ADAM.EXE} \\ \\
\end{tabular}

In order to link a non-ADAM application with the shared library, the
following additional assignment is required:

\begin{tabular}{lll}
\\
 HLP\_LINK & link options file & {\tt HLP\_DIR:HLP\_LINK.OPT} \\ \\
\end{tabular}

\section{ACKNOWLEDGMENTS}
Early versions of the HLP system drew on the code developed by
Roger~Noble (Jodrell~Bank) which runs on the Alliant machine as part of
the OLAF system.  Additional guidance came from the MRAO Cambridge HELP
system, which runs on Norsk Data computers;  listings of this software
were kindly supplied by David~Titterington.  Development of the
HLP\_COMSTR string-comparison routine was done in collaboration with
Paul~Rees (Starlink Project).  Brian~McIlwrath and David~Terrett
(Starlink Project) wrote the HLIB command procedure and the {\tt hlib}
shell script respectively.  The port to Unix was carried out
with the help of David~Terrett, Paul~Rees and Chris~Clayton.  Further
work, concerning use in the ADAM environment, the provision
of shared libraries, and the use of the Starlink template makefile
was carried out by Alan~Chipperfield and Peter~Allan (Starlink
Project).

The VAX/VMS Help system was an invaluable guide in the development of
the specification for the Starlink HLP package.
\end{document}
