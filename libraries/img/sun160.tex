\documentclass[twoside,11pt]{article}

% ? Specify used packages
% \usepackage{graphicx}        %  Use this one for final production.
% \usepackage[draft]{graphicx} %  Use this one for drafting.
% ? End of specify used packages

\pagestyle{myheadings}

% -----------------------------------------------------------------------------
% ? Document identification
% Fixed part
\newcommand{\stardoccategory}  {Starlink User Note}
\newcommand{\stardocinitials}  {SUN}
\newcommand{\stardocsource}    {sun\stardocnumber}
\newcommand{\stardoccopyright}
{Copyright \copyright\ 2000-2003 Council for the Central Laboratory of the Research Councils}

% Variable part - replace [xxx] as appropriate.
\newcommand{\stardocnumber}    {160.6}
\newcommand{\stardocauthors}   {P.W. Draper \\
                                R.F. Warren-Smith}
\newcommand{\stardocdate}      {3 March 2003}
\newcommand{\stardoctitle}     {IMG \\ [1ex]
                                Simple Image Data Access}
\newcommand{\stardocversion}   {Version 1.3}
\newcommand{\stardocmanual}    {Subroutine Library}
\newcommand{\stardocabstract}  {
IMG is a subroutine library for accessing astronomical image data and
associated header information. It is designed to be easy to use and
understand.}

% ? End of document identification
% -----------------------------------------------------------------------------

% +
%  Name:
%     sun.tex
%
%  Purpose:
%     Template for Starlink User Note (SUN) documents.
%     Refer to SUN/199
%
%  Authors:
%     AJC: A.J.Chipperfield (Starlink, RAL)
%     BLY: M.J.Bly (Starlink, RAL)
%     PWD: Peter W. Draper (Starlink, Durham University)
%
%  History:
%     17-JAN-1996 (AJC):
%        Original with hypertext macros, based on MDL plain originals.
%     16-JUN-1997 (BLY):
%        Adapted for LaTeX2e.
%        Added picture commands.
%     13-AUG-1998 (PWD):
%        Converted for use with LaTeX2HTML version 98.2 and
%        Star2HTML version 1.3.
%      1-FEB-2000 (AJC):
%        Add Copyright statement in LaTeX
%     {Add further history here}
%
% -

\newcommand{\stardocname}{\stardocinitials /\stardocnumber}
\markboth{\stardocname}{\stardocname}
\setlength{\textwidth}{160mm}
\setlength{\textheight}{230mm}
\setlength{\topmargin}{-2mm}
\setlength{\oddsidemargin}{0mm}
\setlength{\evensidemargin}{0mm}
\setlength{\parindent}{0mm}
\setlength{\parskip}{\medskipamount}
\setlength{\unitlength}{1mm}

% -----------------------------------------------------------------------------
%  Hypertext definitions.
%  ======================
%  These are used by the LaTeX2HTML translator in conjunction with star2html.

%  Comment.sty: version 2.0, 19 June 1992
%  Selectively in/exclude pieces of text.
%
%  Author
%    Victor Eijkhout                                      <eijkhout@cs.utk.edu>
%    Department of Computer Science
%    University Tennessee at Knoxville
%    104 Ayres Hall
%    Knoxville, TN 37996
%    USA

%  Do not remove the %begin{latexonly} and %end{latexonly} lines (used by
%  LaTeX2HTML to signify text it shouldn't process).
%begin{latexonly}
\makeatletter
\def\makeinnocent#1{\catcode`#1=12 }
\def\csarg#1#2{\expandafter#1\csname#2\endcsname}

\def\ThrowAwayComment#1{\begingroup
    \def\CurrentComment{#1}%
    \let\do\makeinnocent \dospecials
    \makeinnocent\^^L% and whatever other special cases
    \endlinechar`\^^M \catcode`\^^M=12 \xComment}
{\catcode`\^^M=12 \endlinechar=-1 %
 \gdef\xComment#1^^M{\def\test{#1}
      \csarg\ifx{PlainEnd\CurrentComment Test}\test
          \let\html@next\endgroup
      \else \csarg\ifx{LaLaEnd\CurrentComment Test}\test
            \edef\html@next{\endgroup\noexpand\end{\CurrentComment}}
      \else \let\html@next\xComment
      \fi \fi \html@next}
}
\makeatother

\def\includecomment
 #1{\expandafter\def\csname#1\endcsname{}%
    \expandafter\def\csname end#1\endcsname{}}
\def\excludecomment
 #1{\expandafter\def\csname#1\endcsname{\ThrowAwayComment{#1}}%
    {\escapechar=-1\relax
     \csarg\xdef{PlainEnd#1Test}{\string\\end#1}%
     \csarg\xdef{LaLaEnd#1Test}{\string\\end\string\{#1\string\}}%
    }}

%  Define environments that ignore their contents.
\excludecomment{comment}
\excludecomment{rawhtml}
\excludecomment{htmlonly}

%  Hypertext commands etc. This is a condensed version of the html.sty
%  file supplied with LaTeX2HTML by: Nikos Drakos <nikos@cbl.leeds.ac.uk> &
%  Jelle van Zeijl <jvzeijl@isou17.estec.esa.nl>. The LaTeX2HTML documentation
%  should be consulted about all commands (and the environments defined above)
%  except \xref and \xlabel which are Starlink specific.

\newcommand{\htmladdnormallinkfoot}[2]{#1\footnote{#2}}
\newcommand{\htmladdnormallink}[2]{#1}
\newcommand{\htmladdimg}[1]{}
\newcommand{\hyperref}[4]{#2\ref{#4}#3}
\newcommand{\htmlref}[2]{#1}
\newcommand{\htmlimage}[1]{}
\newcommand{\htmladdtonavigation}[1]{}

\newenvironment{latexonly}{}{}
\newcommand{\latex}[1]{#1}
\newcommand{\html}[1]{}
\newcommand{\latexhtml}[2]{#1}
\newcommand{\HTMLcode}[2][]{}

%  Starlink cross-references and labels.
\newcommand{\xref}[3]{#1}
\newcommand{\xlabel}[1]{}

%  LaTeX2HTML symbol.
\newcommand{\latextohtml}{\LaTeX2\texttt{HTML}}

%  Define command to re-centre underscore for Latex and leave as normal
%  for HTML (severe problems with \_ in tabbing environments and \_\_
%  generally otherwise).
\renewcommand{\_}{\texttt{\symbol{95}}}

% -----------------------------------------------------------------------------
%  Debugging.
%  =========
%  Remove % on the following to debug links in the HTML version using Latex.

% \newcommand{\hotlink}[2]{\fbox{\begin{tabular}[t]{@{}c@{}}#1\\\hline{\footnotesize #2}\end{tabular}}}
% \renewcommand{\htmladdnormallinkfoot}[2]{\hotlink{#1}{#2}}
% \renewcommand{\htmladdnormallink}[2]{\hotlink{#1}{#2}}
% \renewcommand{\hyperref}[4]{\hotlink{#1}{\S\ref{#4}}}
% \renewcommand{\htmlref}[2]{\hotlink{#1}{\S\ref{#2}}}
% \renewcommand{\xref}[3]{\hotlink{#1}{#2 -- #3}}
%end{latexonly}
% -----------------------------------------------------------------------------
% ? Document specific \newcommand or \newenvironment commands.
% In-line verbatim. Note that special characters within this still need
% escaping.
\newcommand{\myverb}[1]{{\texttt{#1}}}
%
%  Notes are:
\newcommand{\mynote}{The following notes refer to the numbered statements:}
%
%  Start of code section.
\newenvironment{code}{\begin{small} \begin{quote}}
                     {\end{quote} \end{small}}


% Latex only sections, subsections etc. Surround these with a latexonly
% environment.
\newcommand{\latexonlysection}[1]{\section{#1}}
\newcommand{\latexonlysubsection}[1]{\subsection{#1}}
\newcommand{\latexonlysubsubsection}[1]{\subsubsection{#1}}
\begin{htmlonly}
   \newcommand{\latexonlysection}[1]{#1}
   \newcommand{\latexonlysubsection}[1]{#1}
   \newcommand{\latexonlysubsubsection}[1]{#1}
\end{htmlonly}

% Enumeration of notes.
\newenvironment{enumnotes}
{
   \renewcommand{\labelenumi}{\myverb{[\theenumi]}}
   \begin{enumerate}
}{
   \end{enumerate}
   \renewcommand{\labelenumi}{\theenumi}
}
\begin{htmlonly}
  \renewenvironment{enumnotes}
  {
    \begin{enumerate}
  }{
    \end{enumerate}
  }
\end{htmlonly}

%  SST quotes.
\newcommand{\qt}[1]{\texttt{"}#1\texttt{"}}
\newcommand{\qs}[1]{\texttt{'}#1\texttt{'}}

%+
%  Name:
%     SST.TEX

%  Purpose:
%     Define LaTeX commands for laying out Starlink routine descriptions.

%  Language:
%     LaTeX

%  Type of Module:
%     LaTeX data file.

%  Description:
%     This file defines LaTeX commands which allow routine documentation
%     produced by the SST application PROLAT to be processed by LaTeX and
%     by LaTeX2html. The contents of this file should be included in the
%     source prior to any statements that make of the sst commnds.

%  Notes:
%     The style file html.sty provided with LaTeX2html needs to be used.
%     This must be before this file.

%  Authors:
%     RFWS: R.F. Warren-Smith (STARLINK)
%     PDRAPER: P.W. Draper (Starlink - Durham University)

%  History:
%     10-SEP-1990 (RFWS):
%        Original version.
%     10-SEP-1990 (RFWS):
%        Added the implementation status section.
%     12-SEP-1990 (RFWS):
%        Added support for the usage section and adjusted various spacings.
%     8-DEC-1994 (PDRAPER):
%        Added support for simplified formatting using LaTeX2html.
%     {enter_further_changes_here}

%  Bugs:
%     {note_any_bugs_here}

%-

%  Define length variables.
\newlength{\sstbannerlength}
\newlength{\sstcaptionlength}
\newlength{\sstexampleslength}
\newlength{\sstexampleswidth}

%  Define a \tt font of the required size.
\latex{\newfont{\ssttt}{cmtt10 scaled 1095}}
\html{\newcommand{\ssttt}{\tt}}

%  Define a command to produce a routine header, including its name,
%  a purpose description and the rest of the routine's documentation.
\newcommand{\sstroutine}[3]{
   \goodbreak
   \rule{\textwidth}{0.5mm}
   \vspace{-7ex}
   \newline
   \settowidth{\sstbannerlength}{{\Large {\bf #1}}}
   \setlength{\sstcaptionlength}{\textwidth}
   \setlength{\sstexampleslength}{\textwidth}
   \addtolength{\sstbannerlength}{0.5em}
   \addtolength{\sstcaptionlength}{-2.0\sstbannerlength}
   \addtolength{\sstcaptionlength}{-5.0pt}
   \settowidth{\sstexampleswidth}{{\bf Examples:}}
   \addtolength{\sstexampleslength}{-\sstexampleswidth}
   \parbox[t]{\sstbannerlength}{\flushleft{\Large {\bf #1}}}
   \parbox[t]{\sstcaptionlength}{\center{\Large #2}}
   \parbox[t]{\sstbannerlength}{\flushright{\Large {\bf #1}}}
   \begin{description}
      #3
   \end{description}
}

%  Format the description section.
\newcommand{\sstdescription}[1]{\item[Description:] #1}

%  Format the usage section.
\newcommand{\sstusage}[1]{\item[Usage:] \mbox{}
\\[1.3ex]{\raggedright \ssttt #1}}

%  Format the invocation section.
\newcommand{\sstinvocation}[1]{\item[Invocation:]\hspace{0.4em}{\tt #1}}

%  Format the arguments section.
\newcommand{\sstarguments}[1]{
   \item[Arguments:] \mbox{} \\
   \vspace{-3.5ex}
   \begin{description}
      #1
   \end{description}
}

%  Format the returned value section (for a function).
\newcommand{\sstreturnedvalue}[1]{
   \item[Returned Value:] \mbox{} \\
   \vspace{-3.5ex}
   \begin{description}
      #1
   \end{description}
}

%  Format the parameters section (for an application).
\newcommand{\sstparameters}[1]{
   \item[Parameters:] \mbox{} \\
   \vspace{-3.5ex}
   \begin{description}
      #1
   \end{description}
}

%  Format the examples section.
\newcommand{\sstexamples}[1]{
   \item[Examples:] \mbox{} \\
   \vspace{-3.5ex}
   \begin{description}
      #1
   \end{description}
}

%  Define the format of a subsection in a normal section.
\newcommand{\sstsubsection}[1]{ \item[{#1}] \mbox{} \\}

%  Define the format of a subsection in the examples section.
\newcommand{\sstexamplesubsection}[2]{\sloppy
\item[\parbox{\sstexampleslength}{\ssttt #1}] \mbox{} \vspace{1.0ex}
\\ #2 }

%  Format the notes section.
\newcommand{\sstnotes}[1]{\item[Notes:] \mbox{} \\[1.3ex] #1}

%  Provide a general-purpose format for additional (DIY) sections.
\newcommand{\sstdiytopic}[2]{\item[{\hspace{-0.35em}#1\hspace{-0.35em}:}]
\mbox{} \\[1.3ex] #2}

%  Format the implementation status section.
\newcommand{\sstimplementationstatus}[1]{
   \item[{Implementation Status:}] \mbox{} \\[1.3ex] #1}

%  Format the bugs section.
\newcommand{\sstbugs}[1]{\item[Bugs:] #1}

%  Format a list of items while in paragraph mode.
\newcommand{\sstitemlist}[1]{
  \mbox{} \\
  \vspace{-3.5ex}
  \begin{itemize}
     #1
  \end{itemize}
}

%  Define the format of an item.
\newcommand{\sstitem}{\item}

%% Now define html equivalents of those already set. These are used by
%  latex2html and are defined in the html.sty files.
\begin{htmlonly}

%  sstroutine.
   \newcommand{\sstroutine}[3]{
      \subsubsection{#1\xlabel{#1}-\label{#1}#2}
      \begin{description}
         #3
      \end{description}
   }

%  sstdescription
   \newcommand{\sstdescription}[1]{\item[Description:]
      \begin{description}
         #1
      \end{description}
      \\
   }

%  sstusage
   \newcommand{\sstusage}[1]{\item[Usage:]
      \begin{description}
         {\ssttt #1}
      \end{description}
      \\
   }

%  sstinvocation
   \newcommand{\sstinvocation}[1]{\item[Invocation:]
      \begin{description}
         {\ssttt #1}
      \end{description}
      \\
   }

%  sstarguments
   \newcommand{\sstarguments}[1]{
      \item[Arguments:] \\
      \begin{description}
         #1
      \end{description}
      \\
   }

%  sstreturnedvalue
   \newcommand{\sstreturnedvalue}[1]{
      \item[Returned Value:] \\
      \begin{description}
         #1
      \end{description}
      \\
   }

%  sstparameters
   \newcommand{\sstparameters}[1]{
      \item[Parameters:] \\
      \begin{description}
         #1
      \end{description}
      \\
   }

%  sstexamples
   \newcommand{\sstexamples}[1]{
      \item[Examples:] \\
      \begin{description}
         #1
      \end{description}
      \\
   }

%  sstsubsection
   \newcommand{\sstsubsection}[1]{\item[{#1}]}

%  sstexamplesubsection
   \newcommand{\sstexamplesubsection}[2]{\item[{\ssttt #1}] #2}

%  sstnotes
   \newcommand{\sstnotes}[1]{\item[Notes:] #1 }

%  sstdiytopic
   \newcommand{\sstdiytopic}[2]{\item[{#1}] #2 }

%  sstimplementationstatus
   \newcommand{\sstimplementationstatus}[1]{
      \item[Implementation Status:] #1
   }

%  sstitemlist
   \newcommand{\sstitemlist}[1]{
      \begin{itemize}
         #1
      \end{itemize}
      \\
   }
%  sstitem
   \newcommand{\sstitem}{\item}

\end{htmlonly}

%  End of "sst.tex" layout definitions.
%.

% ? End of document specific commands
% -----------------------------------------------------------------------------
%  Title Page.
%  ===========
\renewcommand{\thepage}{\roman{page}}
\begin{document}
\thispagestyle{empty}

%  Latex document header.
%  ======================
\begin{latexonly}
   CCLRC / \textsc{Rutherford Appleton Laboratory} \hfill \textbf{\stardocname}\\
   {\large Particle Physics \& Astronomy Research Council}\\
   {\large Starlink Project\\}
   {\large \stardoccategory\ \stardocnumber}
   \begin{flushright}
   \stardocauthors\\
   \stardocdate
   \end{flushright}
   \vspace{-4mm}
   \rule{\textwidth}{0.5mm}
   \vspace{5mm}
   \begin{center}
   {\Huge\textbf{\stardoctitle \\ [2.5ex]}}
   {\LARGE\textbf{\stardocversion \\ [4ex]}}
   {\Huge\textbf{\stardocmanual}}
   \end{center}
   \vspace{5mm}

% ? Add picture here if required for the LaTeX version.
%   e.g. \includegraphics[scale=0.3]{filename.ps}
% ? End of picture

% ? Heading for abstract if used.
   \vspace{10mm}
   \begin{center}
      {\Large\textbf{Abstract}}
   \end{center}
% ? End of heading for abstract.
\end{latexonly}

%  HTML documentation header.
%  ==========================
\begin{htmlonly}
   \xlabel{}
   \begin{rawhtml} <H1> \end{rawhtml}
      \stardoctitle\\
      \stardocversion\\
      \stardocmanual
   \begin{rawhtml} </H1> <HR> \end{rawhtml}

% ? Add picture here if required for the hypertext version.
%   e.g. \includegraphics[scale=0.7]{filename.ps}
% ? End of picture

   \begin{rawhtml} <P> <I> \end{rawhtml}
   \stardoccategory\ \stardocnumber \\
   \stardocauthors \\
   \stardocdate
   \begin{rawhtml} </I> </P> <H3> \end{rawhtml}
      \htmladdnormallink{CCLRC / Rutherford Appleton Laboratory}
                        {http://www.cclrc.ac.uk} \\
      \htmladdnormallink{Particle Physics \& Astronomy Research Council}
                        {http://www.pparc.ac.uk} \\
   \begin{rawhtml} </H3> <H2> \end{rawhtml}
      \htmladdnormallink{Starlink Project}{http://www.starlink.ac.uk/}
   \begin{rawhtml} </H2> \end{rawhtml}
   \htmladdnormallink{\htmladdimg{source.gif} Retrieve hardcopy}
      {http://www.starlink.ac.uk/cgi-bin/hcserver?\stardocsource}\\

%  HTML document table of contents.
%  ================================
%  Add table of contents header and a navigation button to return to this
%  point in the document (this should always go before the abstract \section).
  \label{stardoccontents}
  \begin{rawhtml}
    <HR>
    <H2>Contents</H2>
  \end{rawhtml}
  \htmladdtonavigation{\htmlref{\htmladdimg{contents_motif.gif}}
        {stardoccontents}}

% ? New section for abstract if used.
  \section{\xlabel{abstract}Abstract}
% ? End of new section for abstract
\end{htmlonly}

% -----------------------------------------------------------------------------
% ? Document Abstract. (if used)
%  ==================
\begin{center}
\textsl{\stardocabstract}
\end{center}

% ? End of document abstract

% -----------------------------------------------------------------------------
% ? Latex Copyright Statement
%  =========================
\begin{latexonly}
\newpage
\vspace*{\fill}
\stardoccopyright
\end{latexonly}
% ? End of Latex copyright statement

% -----------------------------------------------------------------------------
% ? Latex document Table of Contents (if used).
%  ===========================================
\newpage
\begin{latexonly}
  \markboth{\stardocname}{\stardocname}
  \vspace{5mm}
  \begin{center}
     \rule{80mm}{0.5mm} \\ [1ex]
     {\Large\bf \stardoctitle \\ [2.5ex]
                \stardocversion} \\ [2ex]
     \rule{80mm}{0.5mm}
  \end{center}
  \vspace{30mm}
  \setlength{\parskip}{0mm}
  \tableofcontents
  \setlength{\parskip}{\medskipamount}
  \markboth{\stardocname}{\stardocname}
\end{latexonly}
% ? End of Latex document table of contents
% -----------------------------------------------------------------------------

\cleardoublepage
\renewcommand{\thepage}{\arabic{page}}
\setcounter{page}{1}

\section{INTRODUCTION}
This document describes the IMG library and how to use it to access
astronomical images and header information from within your
programs. The library has been designed to be easy to use and
understand and will be of most interest to astronomers who write their
own software and require uncomplicated access to their data.  In this
context, an ``image'' may be an ordinary (2-D) image, but could also be
a spectrum (1-D) or ``data cube'' (3-D).

The data format used by the IMG library will normally be the Starlink
``Extensible N-Dimensional Data Format'' or NDF (described in
\xref{SUN/33}{sun33}{}), which is typically stored in files with a
\myverb{.sdf} extension.  However, IMG can also \htmlref{access other
astronomical formats}{IMGFOREIGNDATAFORMATS} (such as IRAF, disk FITS
or FIGARO) and new ones can easily be added, so using IMG gives your
programs access to a wide range of astronomical data in a very
straightforward way\latexonly{ (see \S\ref{IMGFOREIGNDATAFORMATS})}.

IMG can be used from the Fortran-77 and C programming languages and some
familiarity with one of these is assumed, as is knowledge of
the UNIX C-shell. If you want to use IMG from C then you should
briefly study the earlier sections (which are Fortran specific, but
nevertheless show the appropriate calling sequences and discuss more
general issues) and then look at
\hyperref{this later section}{appendix }{}{IMGCinterface}.

\section{ACCESSING IMAGE DATA \label{USINGIMG}}
The easiest way to understand IMG is to look at how simple example
programs work. That's what this section does. The examples are generally
snippets of more complete programs that can be found in the
directory
\htmladdnormallink{\myverb{/star/bin/examples/img}}{../../bin/examples/img}
(on non-Starlink machines replace \myverb{/star} with wherever you
have the software installed).  You are encouraged to copy and modify
these for your own use.

\xref{SUN/101}{sun101}{} also contains examples of programming
techniques that you might want to to use with IMG. You should ideally
read these two documents together, although this isn't absolutely
necessary.

\subsection{Accessing \label{IMGMEAN}an existing image}
The first example is a complete program. It gets an existing image,
works out its mean value and then writes it out:
\begin{code}
\begin{verbatim}
      SUBROUTINE MEAN( ISTAT )                                [1]

*  Access an input image.
      CALL IMG_IN( 'IN', NX, NY, IP, ISTAT )                  [2]

*  Derive the mean and write it out.
      CALL DOSTAT( %VAL( IP ), NX, NY, ISTAT )                [3]

*  Free the input image.
      CALL IMG_FREE( 'IN', ISTAT )                            [4]
      END

      SUBROUTINE DOSTAT( IMAGE, NX, NY, ISTAT )
      INCLUDE 'SAE_PAR'                                       [5]
      REAL IMAGE( NX, NY )                                    [6]

      IF ( ISTAT .NE. SAI__OK ) RETURN                        [7]

*  Initialise the sum and loop over all elements of the image.
      SUM = 0.0
      DO 1 J = 1, NY
         DO 2 I = 1, NX
            SUM = SUM + IMAGE( I, J )
 2       CONTINUE
 1    CONTINUE

*  Write out the mean value.
      WRITE( *, * ) 'Mean = ', SUM / REAL( NX * NY )          [8]

      END
\end{verbatim}
\end{code}

\mynote
%\renewcommand{\labelenumi}{\myverb{[\theenumi]}}
\begin{enumnotes}
% 1
\item Programs that use IMG should {\em always} have a main subroutine
with an \myverb{INTEGER} argument. This subroutine should be named
after the program and be stored in a file with the same name
(\myverb{mean} and \htmladdnormallink{\myverb{mean.f}}{../../bin/examples/img/mean.f} in this
case). You should consider this subroutine as a replacement for the
file that normally has a \myverb{PROGRAM} statement in it.

% 2
\item The call to \htmlref{\myverb{IMG\_IN}}{IMG_IN_n__x_} gets the input
image. The image is associated with the ``parameter'' \myverb{'IN'}
which usually means that you will be prompted for the name of an image
data-file.  The subroutine then returns the size of the image --
\myverb{NX}~$\times$~\myverb{NY} -- and a pointer \myverb{IP} to its
data (don't panic -- hold on for the next item).

% 3
\item A call to the subroutine \myverb{DOSTAT}, which performs the
real work, is now made. Since any realistically useful program
needs to be able to handle images of any size, a cheat is necessary
-- this is the \myverb{\%VAL(IP)} argument. Basically, this makes the
pointer to the image data, \myverb{IP}, look just like an array
declaration in the calling routine (i.e.\ like say \myverb{REAL
IMAGE(NX,NY)}).  This isn't standard Fortran, but should work on any
machine that has IMG installed. Pointers are stored in
\myverb{INTEGER} variables.

% 4
\item A call to \htmlref{\myverb{IMG\_FREE}}{IMG_FREE} is made for
each image parameter to tidy up before a program ends.

% 5
\item The \myverb{INCLUDE} statement in the subroutine \myverb{DOSTAT}
inserts the contents of the file \myverb{`SAE\_PAR'}. This defines
some standard Fortran parameters for you, such as \myverb{SAI\_\_OK}
(note the double underscore). Include this file as standard as the
parameters it defines are usually necessary.

% 6
\item The image is now available as an adjustable 2-dimensional array
within the subroutine \myverb{DOSTAT}.

% 7
\item The value of the \myverb{ISTAT} argument is tested against
\myverb{SAI\_\_OK} at the start of \myverb{DOSTAT}. This is done in case an
error has occurred earlier on (in which case \myverb{ISTAT} will not
be equal to \myverb{SAI\_\_OK}, so the remainder of \myverb{DOSTAT}
will not execute).

Doing this means that the program will not crash if an earlier error
results in (say) the \myverb{IMAGE} array containing rubbish, and it
makes sure that you get a sensible error message at the end telling
you what went wrong.

% 8
\item The calculation result is written out. In a more sophisticated
program you might want to use routines from the MSG library
(\xref{SUN/104}{sun104}{}) to do this.
\end{enumnotes}

To complete the program another file is required. This is an
``interface'' file (which should be named after the program but with a
file extension of \myverb{.ifl}). It contains a description of each
parameter associated with an image.  The following will do
for this example program:
\begin{code}
\begin{verbatim}
interface MEAN
   parameter IN
      prompt 'Input image'
   endparameter
endinterface
\end{verbatim}
\end{code}
The main point is that it is necessary to include a \myverb{parameter}
\ldots \myverb{endparameter} statement for each input or output image.
\xref{SUN/101}{sun101}{} and \xref{SUN/115}{sun115}{} describe
numerous other parameter options if you need more sophisticated
control.

Once you have the two files
\htmladdnormallink{\myverb{mean.f}}{../../bin/examples/img/mean.f}
and
\htmladdnormallink{\myverb{mean.ifl}}{../../bin/examples/img/mean.ifl}
you can compile and run the program by following the instructions
\hyperref{given elsewhere}{in \S}{}{IMGCOMPILING}.

{\bf To recap:} The most important IMG concepts to remember
from this example are:

\begin{enumerate}
\item {\em Images are associated with a label known as a ``parameter''.}

\item {\em Image data is accessed by a ``pointer'' that needs to be passed
           to a subroutine using the \myverb{\%VAL} mechanism.}

\item {\em It is necessary to call \htmlref{\myverb{IMG\_FREE}}{IMG_FREE} to free
           images before the program ends.}
\end{enumerate}

\subsection{Creating a new image}
This second example (taken from the program
\htmladdnormallink{\myverb{flat.f}}{../../bin/examples/img/flat.f}
in the example directory\latexonly{ - see \S\ref{USINGIMG}}), shows how to
create a new image and then write values to it:
\begin{code}
\begin{verbatim}

*  Create a new image.
      CALL IMG_NEW( 'OUT', 416, 578, IP, ISTAT )              [1]

*  Fill the array with the value 1.
      CALL DOFILL( %VAL( IP ), 416, 578, ISTAT )              [2]

*  Free the image.
      CALL IMG_FREE( 'OUT', ISTAT )                           [3]
      END

      SUBROUTINE DOFILL( IMAGE, NX, NY, ISTAT )
      INCLUDE 'SAE_PAR'
      REAL IMAGE( NX, NY )
      IF ( ISTAT .NE. SAI__OK ) RETURN

*  Loop over all the elements of the image setting them to 1.0.
      DO 1 J = 1, NY
         DO 2 I = 1, NX
            IMAGE( I, J ) = 1.0
 2       CONTINUE
 1    CONTINUE
      END

\end{verbatim}
\end{code}
\mynote
\begin{enumnotes}
%1
\item The call to \htmlref{\myverb{IMG\_NEW}}{IMG_NEW_n__x_} creates the new image dataset
(usually prompting you for a name). The image size in this example is
416~$\times$~578 pixels.

%2
\item The newly created image is passed to the subroutine
\myverb{DOFILL} where all its elements are set to the value $1.0$.

%3
\item Calling \htmlref{\myverb{IMG\_FREE}}{IMG_FREE} ensures that the output image is
correctly freed (avoiding possible loss of the data you have written).
\end{enumnotes}

\subsection{Modifying images}
One of the most common things that programs do is modify an existing
image or create an image that is a modification of an existing one. The
examples shown here deal with both these cases.

\subsubsection{Modifying an image}

This snippet of Fortran shows an existing image being accessed so that
it can be modified in place (that is, without creating a new copy):
\begin{code}
\begin{verbatim}
*  Access an existing image.
      CALL IMG_MOD( 'IN', NX, NY, IP, ISTAT )

*  Fill the image with a value.
      CALL DOFILL( %VAL( IP ), NX, NY, ISTAT )

*  Free the image.
      CALL IMG_FREE( 'IN', ISTAT )
\end{verbatim}
\end{code}

\subsubsection{Modifying a copy of an image}

This example copies the input image first and then modifies the
copy. This is essential if the input image needs to be kept.
A complete program called
\htmladdnormallink{\myverb{add.f}}{../../bin/examples/img/add.f}
exists\latexonly{ (see \S\ref{USINGIMG})}.
\begin{code}
\begin{verbatim}
*  Access an existing image.
      CALL IMG_IN( 'IN', NX, NY, IPIN, ISTAT )

*  Create a new output image by copying the input image.
      CALL IMG_OUT( 'IN', 'OUT', IPOUT, ISTAT )               [1]

*  Modify the output image.
      CALL DOFILL( %VAL( IPOUT ), NX, NY, ISTAT )

*  Free the input and output images.
      CALL IMG_FREE( '*', ISTAT )                             [2]
\end{verbatim}
\end{code}
\mynote
\begin{enumnotes}
\item The call to \htmlref{\myverb{IMG\_OUT}}{IMG_OUT_x_} creates a new output image
associated with the parameter \myverb{'OUT'} by copying the one
associated with the parameter \myverb{'IN'}. It returns a pointer
\myverb{IPOUT} for the output image data.

\item Notice that the names of the image parameters are not shown
explicitly in this call to \htmlref{\myverb{IMG\_FREE}}{IMG_FREE}. Using a \myverb{'*'}
indicates that all known images should be freed.
\end{enumnotes}

\section{ACCESSING HEADER INFORMATION}
Most astronomical images have collections of numeric, character or
logical values associated with them. Typically these consist of
important calibration values, but they are also often used to describe
the history or status of the data. Collections of such data, which are
not actually part of the image array, are known as ``header''
information (this name derives from the fact that FITS\footnote{FITS,
the ``Flexible Image Transport System'' is the {\em de facto} standard
for the interchange of astronomical data.}  stores these values at the
beginning -- head -- of a file). IMG refers to a single piece of such
information as a ``header item'' and a separate set of subroutines
exists to read, write, copy and make enquiries about them.

\subsection{Reading header items}
This example (taken from a complete program called
\htmladdnormallink{\myverb{hdrread.f}}{../../bin/examples/img/hdrread.f}
\latexonly{-- see \S \ref{USINGIMG}}) shows how to read the value of a
header item from an image associated with the parameter \myverb{'IN'}:
\begin{code}
\begin{verbatim}
*  Read the header item.
      CALL HDR_IN( 'IN', ' ', 'OBSERVER', 1, VALUE, ISTAT )
\end{verbatim}
\end{code}
The call to \htmlref{\myverb{HDR\_IN}}{HDR_IN_x_} specifies the image by using its
parameter name -- \myverb{'IN'} (argument 1). If you have not
previously accessed this image you will probably be prompted for its
file name at this point.

The 2nd, blank argument, indicates that an ``ordinary'' FITS header
item is required (there are other possible ``sources'' of header information
which are described
\hyperref{elsewhere}{later -- \S}{}{IMGNAMEDEXTENSIONS}).

The 3rd argument \myverb{'OBSERVER'} specifies the name of the header
item required and the 4th argument \myverb{1} indicates that you want
the first occurrence of that item (just in case it occurs several
times, which some items are allowed to do).

The 5th argument \myverb{VALUE} returns the header item value as a
character string. If the header item you requested doesn't exist, then
\myverb{VALUE} is returned unchanged, so you may want to set it to a
sensible value (say \myverb{'$<$unknown$>$'}) beforehand.

\subsection{Writing header items}
This example (taken from a complete program
\htmladdnormallink{\myverb{hdrwrite.f}}{../../bin/examples/img/hdrwrite.f}\latexonly{
-- see \S \ref{USINGIMG}}) shows how to write the value of a
header item to an image associated with the parameter \myverb{'OUT'}:
\begin{code}
\begin{verbatim}
*  Write the new header item.
      CALL HDR_OUT( 'OUT', ' ', 'OBSERVER', 'The observer',   [1]
     :              'Fred Bloggs', ISTAT )

*  Free the image.
      CALL IMG_FREE( 'OUT', ISTAT )                           [2]
\end{verbatim}
\end{code}
\mynote
\begin{enumnotes}
\item The call to \htmlref{\myverb{HDR\_OUT}}{HDR_OUT_x_} specifies the image by using its
parameter name -- \myverb{'OUT'} (argument 1). If you have not
previously accessed this image you will probably be prompted for its
file name at this point. If you have already accessed the image then
it should be either an output image or one accessed for modification,
as you cannot write items to images accessed by \htmlref{\myverb{IMG\_IN}}{IMG_IN_n__x_}.

The 2nd, blank argument, indicates that an ``ordinary'' FITS header
item is to be written.

The 3rd argument (\myverb{'OBSERVER'}) specifies the name of the header
item and the 4th argument \myverb{'The observer'} is a comment
to store with the item.

The 5th argument (\myverb{'Fred Bloggs'}) is the header item value.

\item The image is freed. If this wasn't done, the new header item
would be lost.
\end{enumnotes}

\section{ADDITIONAL FEATURES}
The IMG library has more capabilities than have been covered in the
preceding descriptions, which have been kept as simple as possible.
You should read the following  if you need to do more than has been
covered so far.

\subsection{More advanced image access}
\subsubsection{Getting workspace}
Any reasonably complex program sooner or later needs to be able to
store image information between processing stages. This program
snippet (from an example called
\htmladdnormallink{\myverb{shadow.f}}{../../bin/examples/img/shadow.f}
\latexonly{-- see \S\ref{USINGIMG}})
shows you how to create a temporary image for this purpose:
\begin{code}
\begin{verbatim}
*  Access the input image.
      CALL IMG_IN( 'IN', NX, NY, IPIN, ISTAT )
      ...
*  Get an image-sized piece of workspace.
      CALL IMG_TMP( 'TEMP', NX, NY, IPTEMP, ISTAT )           [1]
      ...
*  Free all the images (this deletes the temporary image).
      CALL IMG_FREE( '*', ISTAT )                             [2]
\end{verbatim}
\end{code}
\mynote
\begin{enumnotes}
\item The call to \htmlref{\myverb{IMG\_TMP}}{IMG_TMP_n__x_} creates a temporary image of
the size you specify (in this case the same size as the input image).
You will {\em not} be prompted for a file name for a temporary image,
and {\em no } entry is needed for it in the program's interface file.

\item The call to \htmlref{\myverb{IMG\_FREE}}{IMG_FREE} will delete the temporary image.
\end{enumnotes}

\subsubsection{Using ``images'' which are not 2-dimensional}
IMG subroutines can access ``image'' data with between 1 and 3
dimensions. This allows you to handle spectra (1-D) and cubes (3-D) as
well as traditional 2-dimensional images. The subroutine calls needed
are almost identical to the normal ones \latexonly{(above)} that assume 2
dimensions. They differ only in having more or less arguments for the
dimension sizes. The following illustrates the possibilities:
\begin{code}
\begin{verbatim}
      CALL IMG_IN1( 'SPECTRUM', NX, IP, ISTAT )
      CALL IMG_IN2( 'IMAGE', NX, NY, IP, ISTAT )
      CALL IMG_IN3( 'CUBE', NX, NY, NZ, IP, ISTAT )
\end{verbatim}
\end{code}
Note how the number of dimensions is specified by appending a number
to the routine name, and how \myverb{IMG\_IN2} is just a synonym for
\htmlref{\myverb{IMG\_IN}}{IMG_IN_n__x_} (2 dimensions are assumed if you do not say
otherwise).

As well as accessing input images, there are equivalent routines for
performing most other IMG operations on images with between 1 and 3
dimensions. For instance, the following would create a new
3-dimensional image:
\begin{code}
\begin{verbatim}
      CALL IMG_NEW3( 'CUBE', NX, NY, NZ, IP, ISTAT )
\end{verbatim}
\end{code}

When accessing an existing image with (say) 2 dimensions, you will
also be able to access an appropriate slice from (say) a 3-dimensional
image if you specify this when you are prompted for the image name
(see \hyperref{elsewhere}{\S}{}{IMGREGIONS}).

\subsubsection{Accessing \label{IMGDIFFERENTDATATYPES} images using different data types}
IMG also allows access to images using data types other than
\myverb{REAL} (which is the default). The most useful of these are
\myverb{INTEGER}, \myverb{INTEGER*2}, and \myverb{DOUBLE
PRECISION}. The subroutine calls needed are almost identical to the
normal ones except that you should append a character code to the
routine name to indicate the data type you require.  The character
codes (and their data types) are:
\begin{small}
\begin{center}
\begin{tabular}{||l|l|l||}
\hline
Character code & Fortran-77 data type      &  Description      \\
\hline
 \myverb{R}    & \myverb{REAL}             &  Single precision \\
 \myverb{D}    & \myverb{DOUBLE PRECISION} &  Double precision \\
 \myverb{I}    & \myverb{INTEGER}          &  Integer          \\
\hline
  \myverb{W}   & \myverb{INTEGER*2}        &  Word             \\
  \myverb{UW}  & \myverb{INTEGER*2}        &  Unsigned word    \\
  \myverb{B}   & \myverb{BYTE}             &  Byte             \\
  \myverb{UB}  & \myverb{BYTE}             &  Unsigned byte    \\
\hline
\end{tabular}
\end{center}
\end{small}
so among the possibilities are the calls:
\begin{code}
\begin{verbatim}
      CALL IMG_IND( 'IMAGE', NX, NY, IP, ISTAT )              [1]
      CALL IMG_INI( 'IMAGE', NX, NY, IP, ISTAT )              [2]
      CALL IMG_INR( 'IMAGE', NX, NY, IP, ISTAT )              [3]
\end{verbatim}
\end{code}
\mynote
\begin{enumnotes}
\item Access an image using \myverb{DOUBLE PRECISION}.

\item Access an image using \myverb{INTEGER}.

\item Access an image using \myverb{REAL}. Note that
this is a synonym for \htmlref{\myverb{IMG\_IN}}{IMG_IN_n__x_}, since it also gets a 2-D
\myverb{REAL} image.
\end{enumnotes}

You should declare the image array in your program to have the
corresponding Fortran data type, as in the following example that
accesses and modifies an image using \myverb{INTEGER} format:
\begin{code}
\begin{verbatim}

*  Access an input image, allowing it to be modified.
      CALL IMG_MODI( 'IN', NX, NY, IP, ISTAT )                [1]

*  Fill the array with zeros.
      CALL DOZERO( %VAL( IP ), NX, NY, ISTAT )

*  Free the image.
      CALL IMG_FREE( 'IN', ISTAT )
      END

      SUBROUTINE DOZERO( IMAGE, NX, NY, ISTAT )
      INCLUDE 'SAE_PAR'
      INTEGER IMAGE( NX, NY )                                 [2]

      IF ( ISTAT .NE. SAI__OK ) RETURN
      DO 1 J = 1, NY
         DO 2 I = 1, NX
            IMAGE( I, J ) = 0
 2       CONTINUE
 1    CONTINUE
      END
\end{verbatim}
\end{code}
\mynote
\begin{enumnotes}
\item An \myverb{INTEGER} input image is accessed for modification

\item In the subroutine \myverb{DOZERO} the image is declared as an
adjustable dimension \myverb{INTEGER} array.
\end{enumnotes}

All IMG subroutines that access images have variants that allow you to
use different data types. There are also versions that allow these to
be mixed with different numbers of data dimensions. Some possibilities
are:
\begin{code}
\begin{verbatim}
      CALL IMG_IN1D( 'SPECTRUM', NX, IP, ISTAT )              [1]
      CALL IMG_MOD2I( 'IMAGE', NX, NY, IP, ISTAT )            [2]
      CALL IMG_NEW3W( 'CUBE', NX, NY, NZ, IP, ISTAT )         [3]
      CALL IMG_IN2R( 'IMAGE', NX, NY, IP, ISTAT )             [4]
\end{verbatim}
\end{code}
\mynote
\begin{enumnotes}
\item This call accesses a 1-D image for reading using \myverb{DOUBLE
PRECISION}.

\item This call accesses a 2-D image for modification using the
\myverb{INTEGER} data type.

\item This call creates a new 3-D data cube with the word (Fortran
\myverb{INTEGER*2}) data type.

\item This call is a synonym for \htmlref{\myverb{IMG\_IN}}{IMG_IN_n__x_}, since it gets a 2-D
\myverb{REAL} image.
\end{enumnotes}

If your image isn't stored with the type that you ask for, then it
will be converted from the stored type (if possible). If you modify
the data it will be re-converted to the storage type when the image is
freed.  New images are stored using the data type you ask for.

The data types unsigned word (UW), signed byte (B) and unsigned byte
(UB) are less commonly used. Indeed, the unsigned data types have no
equivalents in Fortran and should be manipulated using the
PRIMDAT library (\xref{SUN/39}{sun39}{}).


\subsubsection{Accessing multiple images}
IMG can access more than one image at a time using a single subroutine
call. An example of this is:
\begin{code}
\begin{verbatim}

*  Declare pointers.
      INTEGER IP( 3 )

*  Access images.
      CALL IMG_IN( 'BIAS,FLAT,RAW', NX, NY, IP, ISTAT )       [1]

*  Create a new output image by copying the RAW input image.
      CALL IMG_OUT( 'RAW', 'PROC', IPPROC, ISTAT )            [2]

*  Debias and flatfield data.
      CALL DOPROC( %VAL( IP( 1 ) ), %VAL( IP( 2 ) ),          [3]
     :             %VAL( IP( 3 ) ), NX, NY, %VAL( IPPROC ),
     :             ISTAT )

*  Free all the images.
      CALL IMG_FREE( '*', ISTAT )
      END

      SUBROUTINE DOPROC( BIAS, FLAT, RAW, NX, NY, PROC, ISTAT )
      INCLUDE 'SAE_PAR'
      REAL BIAS( NX, NY ), FLAT( NX, NY ), RAW( NX, NY ),     [4]
     :     PROC( NX, NY )

      IF ( ISTAT .NE. SAI__OK ) RETURN
      DO 1 J = 1, NY
         DO 2 I = 1, NX
            PROC( I, J ) = ( RAW( I, J ) - BIAS( I, J ) ) / FLAT( I, J )
 2       CONTINUE
 1    CONTINUE
      END
\end{verbatim}
\end{code}
\mynote
\begin{enumnotes}
% 1
\item
This call accesses three images \myverb{'BIAS'}, \myverb{'FLAT'} and
\myverb{'RAW'}. Prompts for the actual image files will be made for
each of these names and pointers to the image data will returned in IP,
which should now be an array of size 3.

% 2
\item
The image \myverb{'RAW'} is copied to a new file. This also copies any
header information, so is usually preferred to using \htmlref{\myverb{IMG\_NEW}}{IMG_NEW_n__x_}.

% 3
\item
The individual pointers are now passed to a subroutine so that the
\myverb{'BIAS'}, \myverb{'FLAT'} and \myverb{'RAW'} images together
with the copy of \myverb{'RAW'} can be accessed.

% 4
\item
The image arrays are declared as normal.
\end{enumnotes}

There are two advantages to accessing multiple images in one call in
this way:
\begin{enumerate}
\item
All the images are made available to the program as if they were the
same size, regardless of their actual sizes (the largest region that
is common to all the images is selected).

\item
All the images are made available with the same data type
(\myverb{REAL} in this case).
\end{enumerate}
This means that corresponding values in each of the images can be
directly inter-compared by your program, even if the images themselves
have different sizes or data types.

\subsubsection{Creating multiple images}
IMG can \latexonly{also} create more than one image using a single
subroutine call. Some possibilities are:
\begin{code}
\begin{verbatim}
      CALL IMG_NEW( 'MODEL1,MODEL2', 1024, 1024, IP, ISTAT )  [1]
      CALL IMG_OUT( 'TEMPLATE', 'OUT1,OUT2', IP, ISTAT )      [2]
      CALL IMG_TMP( 'T1,T2,T3,T4', NX, NY, IP, ISTAT )        [3]
\end{verbatim}
\end{code}
\mynote
\begin{enumnotes}
% 1
\item The call to \htmlref{\myverb{IMG\_NEW}}{IMG_NEW_n__x_} creates two new images with size
1024~$\times$~1024. The variable \myverb{IP} should be declared as
an \myverb{INTEGER} array of size 2.

% 2
\item The image associated with parameter \myverb{'TEMPLATE'} is
copied to two new images \myverb{'OUT1'} and \myverb{'OUT2'}.
Again, the pointers are returned in an array \myverb{IP(2)}.

%3
\item Four temporary images of the same size and type are created.
\end{enumnotes}

\subsubsection{Handling \label{IMGBADDATA}``bad'' data}

In practice, astronomical images will often contain values that are
unreliable (or unknown) and should be ignored. For example, you might
not be able to compute a result for every element of an image array
(say where a divide by zero would occur), or your detector system may
not generate sensible values everywhere.  These missing data are
usually flagged by setting them to a unique number that allows them to
be recognised, and this is known as the ``bad'' data
value.\footnote{It may also be known as the ``invalid'', ``flagged''
or even ``magic'' value, but the meaning is the same.}

Unfortunately, unless you have complete control over your data, you
cannot usually ignore the possibility that an image may contain some
of these ``bad'' values.  If you did, and processed the numbers as if
they were valid data, you would at best get the wrong answer. More
probably, your program would crash.

To prevent you being baffled by this pitfall, IMG will normally check
whether your input images contain any bad values and will issue a
warning if any are found. Then at least you know {\em why} your
program crashed!

You can deal with bad data values in one of two ways. The simplest is
to use a suitable program to detect them and replace them with a
sensible alternative value (for example, the KAPPA -
\xref{SUN/95}{sun95}{} - program \xref{NOMAGIC}{sun95}{NOMAGIC} will
do this). Alternatively, you can modify your IMG program to take
proper account of them.

If you decide to modify your program, you'll probably want to inhibit
the checks that IMG makes. This is done by appending an exclamation
mark (!) to the parameter name of each affected image. So, for
instance:
\begin{code}
\begin{verbatim}
      CALL IMG_IN( 'BIAS!,FLAT!,RAW!', NX, NY, IP, ISTAT )
\end{verbatim}
\end{code}
inhibits bad data value checking for all the input images.

The following example is a version of the
\htmlref{\myverb{mean.f}}{IMGMEAN} program \latexonly{(see
\S\ref{IMGMEAN})} that shows the sort of changes needed to deal with
bad data:
\begin{code}
\begin{verbatim}
      SUBROUTINE MEAN( ISTAT )

*  Access an input image. Inhibit checks for bad pixels.
      CALL IMG_IN( 'IN!', NX, NY, IP, ISTAT )                 [1]

*  Derive the mean and write it out.
      CALL DOSTAT( %VAL( IP ), NX, NY, ISTAT )

*  Free the input image.
      CALL IMG_FREE( 'IN', ISTAT )
      END

      SUBROUTINE DOSTAT( IMAGE, NX, NY, ISTAT )
      INCLUDE 'SAE_PAR'
      INCLUDE 'PRM_PAR'                                       [2]
      REAL IMAGE( NX, NY )

      IF ( ISTAT .NE. SAI__OK ) RETURN

*  Initialise the sum and loop over all elements of the image, checking
*  that every data value is not bad.
      SUM = 0.0
      N = 0
      DO 1 J = 1, NY
         DO 2 I = 1, NX
            IF ( IMAGE( I, J ) .NE. VAL__BADR ) THEN          [3]
               SUM = SUM + IMAGE( I, J )
               N = N + 1
            END IF
 2       CONTINUE
 1    CONTINUE

*  Write out the mean value.
      IF ( N .GT. 0 ) THEN
         WRITE( *, * ) 'Mean of ', N, ' values = ', SUM / REAL( N )
      ELSE
         WRITE( *, * ) 'Error: all data values are bad'
      END IF
      END
\end{verbatim}
\end{code}
\mynote
\begin{enumnotes}
%1
\item
An input image is accessed without performing any checks for the
presence of bad data.

%2
\item
The file `\myverb{PRM\_PAR}' is included. This defines Fortran
parameters for the values used to flag bad data (it should be used by
all programs that handle bad data). The parameter names are
type-dependent and have the form \myverb{VAL\_\_BAD[x]}, where
\myverb{[x]} is replaced by the character code for the data type
you're processing. These codes are the same as those used when
accessing images of different data types -- see
\hyperref{here}{\S}{}{IMGDIFFERENTDATATYPES}.  So, for instance, with
\myverb{REAL} data you would use \myverb{VAL\_\_BADR}, while with
\myverb{INTEGER} data you would use \myverb{VAL\_\_BADI}, etc. Before
compiling a program that includes this file you should execute the
command \myverb{prm\_dev}, which creates a soft link to the include
file.

%3
\item
Since every input value may now potentially be bad, each one must be
checked before use. As the data type being processed is \myverb{REAL},
the bad data value we test against is \myverb{VAL\_\_BADR}.
\end{enumnotes}

If your program accesses more than one input image, you must be
careful to check each of them. For instance if you were adding two
images together you'd need to do something like the following:
\begin{code}
\begin{verbatim}
      DO 1 I = 1, NX
         IF ( A( I ) .NE. VAL__BADI .AND. B( I ) .NE. VAL__BADI ) THEN
            C( I ) = A( I ) + B( I )
         ELSE
            C( I ) = VAL__BADI
         END IF
 1    CONTINUE
\end{verbatim}
\end{code}

Where \myverb{A} and \myverb{B} are the ``images'' to be added and
\myverb{C} is the image to store the result (in this case the images
are really 1-D \myverb{INTEGER} spectra).
Note how a bad value is assigned to the output image if we are unable
to calculate a result.

If you need to know more about how to handle bad data you should
consult \xref{SUN/33}{sun33}{bad_pixels}.

\subsection{More advanced header access}
\subsubsection{Accessing header items using different data types}
Header items can come in more than one data type, so extended HDR
subroutines (as for IMG) are provided to access them. If the item
isn't of the type you want, then a format conversion will be
attempted. The way you specify the type you require is similar to
that used for IMG routines -- you just append the necessary character
code to the routine name:
\begin{small}
\begin{center}
\begin{tabular}{||l|l|l||}
\hline
Character code & Fortran-77 data type      &  Description      \\
\hline
 \myverb{C}    & \myverb{CHARACTER}        &  Character string \\
 \myverb{D}    & \myverb{DOUBLE PRECISION} &  Double precision \\
 \myverb{I}    & \myverb{INTEGER}          &  Integer          \\
 \myverb{L}    & \myverb{LOGICAL}          &  Logical          \\
 \myverb{R}    & \myverb{REAL}             &  Single precision \\
\hline
\end{tabular}
\end{center}
\end{small}
Since all header item values can be converted to a character
representation, this is the safest method to access an item of unknown
type (which is why it is the method used by the ordinary
\htmlref{\myverb{HDR\_IN}}{HDR_IN_x_} subroutine). Some of the possible calls are:
\begin{code}
\begin{verbatim}
     CALL HDR_INR( 'IN', ' ', 'BSCALE', 1, BSCALE, ISTAT )    [1]
     CALL HDR_INI( 'IN', ' ', 'BINNED', 1, IBFACT, ISTAT )    [2]
     CALL HDR_OUTL( 'OUT', ' ', 'CHECKED',                    [3]
     :              'Data checked for C/Rs', .TRUE., ISTAT )
\end{verbatim}
\end{code}
\mynote
\begin{enumnotes}
% 1
\item This example reads in a \myverb{REAL} value \myverb{BSCALE}.

% 2
\item This example reads in an \myverb{INTEGER} value \myverb{BINNED}.

% 1
\item This example writes out a \myverb{LOGICAL} value
\myverb{CHECKED} with the comment \myverb{'Data checked for C/Rs'}.
\end{enumnotes}

\subsubsection{Using \label{IMGNAMEDEXTENSIONS}header items from different sources}
So far, header items have been assumed to originate only from one
source (FITS). In reality, an image may have more than one
set of header information, and these are distinguished by giving them
different names. FITS headers are normally used for storing
information that may be widely distributed, perhaps to people using
different data reduction systems, whereas more private collections of
header information may be created by using a name of your own
choice. Private header information has the advantage that it is
unlikely to be trampled on by other software (that believes it knows
what the headers mean -- rightly or wrongly) and is typically used for
information needed only within a particular software
package.\footnote{The Starlink application packages CCDPACK
(\xref{SUN/139}{sun139}{}) and POLPACK (\xref{SUN/223}{sun223}{}) make
extensive use of these facilities.}

Using header items from other sources is achieved simply by replacing
\myverb{' '} (which is a synonym for 'FITS') with the appropriate
name.  Storing information in a named source solely used by your
programs is encouraged. This is better than trusting other programs
not to modify values that you have set. \myverb{'MYSOURCE'} is used in
the following examples:
\begin{code}
\begin{verbatim}
*  Write a new header item.
      CALL HDR_OUT( 'OUT', 'MYSOURCE', ITEM, ' ', VALUE, ISTAT )
      ...
*  Read back the value.
      CALL HDR_IN( 'IN', 'MYSOURCE', ITEM, 1, VALUE, ISTAT )
\end{verbatim}
\end{code}

\subsubsection{Accessing header items by index \label{HDRINDEXING}}
There are two HDR routines that allow you to access header items using
an index number. Header item index numbers run from 1 through to the
maximum number of items available. These facilities can be used to
list all the header items, or to test for the existence of items with
particular names.

The following program snippet shows how to list all the header items
associated with an image.
It uses the \htmlref{\myverb{HDR\_NUMB}}{HDR_NUMB} subroutine to query the number of
items. The complete example is called
\htmladdnormallink{\myverb{hdrlist.f}}
{../../bin/examples/img/hdrlist.f}\latexonly{ -- see \S \ref{USINGIMG}}.
\begin{code}
\begin{verbatim}
*  See how many header items are available.
      CALL HDR_NUMB( 'IN', ' ', '*', N, ISTAT )            [1]
      DO 1 I = 1, N

*  Get the name of the I'th header item.
         CALL HDR_NAME( 'IN', ' ', I, ITEM, ISTAT )        [2]

*  Get its value.
         CALL HDR_IN( 'IN', ' ', ITEM, 1, VALUE, ISTAT )   [3]

*  And write it out.
         WRITE( *, '( 1X, 3A )' ) ITEM ,' = ', VALUE       [4]
 1    CONTINUE
\end{verbatim}
\end{code}
\mynote
\begin{enumnotes}
% 1
\item
The subroutine \htmlref{\myverb{HDR\_NUMB}}{HDR_NUMB} counts the number of header items
or the number of occurrences of an item. The \myverb{'*'} argument indicates
that all the header items are to be counted. If an
item name was given the number of occurrences of that item would be
returned (zero if none exist).

% 2
\item
Item names may be queried by using an index number. This provides a
method of getting all the names when the exact contents of a header
item source are not known.

% 3
\item
\htmlref{\myverb{HDR\_IN}}{HDR_IN_x_} returns the value of the header item as a
character string.

% 4
\item
The name and value of the header item are written out.
\end{enumnotes}

\subsubsection{Special behaviour of FITS headers}
A complication that we have glossed over so far is the fact that
certain FITS header items can occur more than once within an image.
Strictly, this is limited to the keywords \myverb{COMMENT},
\myverb{HISTORY} and \myverb{' '} (blank) which are often used to
construct multi-line descriptions of the contents or history of a FITS
file. To handle this, various of the HDR subroutines are ``FITS
occurrence aware''. For instance, the fourth argument of the
\htmlref{\myverb{HDR\_IN}}{HDR_IN_x_} subroutine specifies the
occurrence of the item to be read, and
\htmlref{\myverb{HDR\_OUT}}{HDR_OUT_x_} will append (rather than
overwriting) items with these special names.

FITS items can also have a comment associated with them (to be read by
the recipient when a FITS file is sent to another astronomical
institution). The comment string is given by the fourth argument of
the \myverb{HDR\_OUT} subroutine, but this facility isn't available
for non-FITS header sources where this argument will simply be
ignored.


\subsubsection{Hierarchical header items}

Header items can be stored in hierarchical structures.\footnote{ {\em
Warning:} you should not write hierarchical FITS header items as this
violates the FITS standard (although data from outside sources may
sometimes contain them).}  What this actually means is that a header
item can be used as a ``container'' to store other items (which are
known as components). Hierarchical items are distinguished by using a
period sign in the name to separate their various components. So, for
instance, if you had a series of values that described a useful region
on an image you might write these out using something like:
\begin{code}
\begin{verbatim}
*  Store the useful region of this image.
      CALL HDR_OUTI( 'OUT', 'MYSOURCE', 'USEFUL.MINX', ' ',IXLOW, ISTAT )
      CALL HDR_OUTI( 'OUT', 'MYSOURCE', 'USEFUL.MAXX', ' ',IXHIGH, ISTAT )
      CALL HDR_OUTI( 'OUT', 'MYSOURCE', 'USEFUL.MINY', ' ',IYLOW, ISTAT )
      CALL HDR_OUTI( 'OUT', 'MYSOURCE', 'USEFUL.MAXY', ' ',IYHIGH, ISTAT )
\end{verbatim}
\end{code}
The number of sub-components and the level of hierarchy isn't
restricted, so more elaborate structures are possible (but not
necessarily desirable).

You can examine the header items in an image by using the HDSTRACE program
(\xref{SUN/102}{sun102}{}). If your image file is called
\myverb{image} then you would use the command:
\begin{code}
\begin{verbatim}
%  hdstrace image.more.mysource
\end{verbatim}
\end{code}
to see the contents of the source \myverb{'MYSOURCE'}. To list the
contents of a FITS source the most convenient method is to use the
KAPPA program \xref{FITSLIST}{sun95}{FITSLIST}
(\xref{SUN/95}{sun95}{}).

If you have any hierarchical FITS headers (these were once used
extensively by UK observatories), then they should look something
like:
\begin{code}
\begin{verbatim}
        ING DETHEAD = 'CCD-RCA2'       /Type of detector head
\end{verbatim}
\end{code}
when examined by \xref{FITSLIST}{sun95}{FITSLIST}.  To access this
item using IMG you should call it \myverb{'ING.DETHEAD'}.
The value of this item is shown being read by the example
program
\htmladdnormallink{\myverb{hdrread.f}}{../../bin/examples/img/hdrread.f}
\latexonly{(see \S \ref{USINGIMG})} next:
\begin{code}
\begin{verbatim}
% hdrread
ITEM - Header item > ING.DETHEAD
IN - Input image > image
The header item 'ING.DETHEAD' has a value of CCD-RCA2.
%
\end{verbatim}
\end{code}
Hierarchical items are processed as normal by the HDR
\htmlref{indexing
routines}{HDRINDEXING}\latexonly{ (\S\ref{HDRINDEXING})}. Each
component item counts only once and no significance is attached to the
fact it may be part of a hierarchy. When retrieving an indexed header
item, its name is returned in full, including all the necessary
periods.

\section{USING THE NDF LIBRARY TO DO MORE}
IMG is really just a simplified interface to the NDF library
(\xref{SUN/33}{sun33}{}), so all the facilities of the NDF format and
library can be used along with IMG (and HDR) routines, if
required. The following outlines some of the more important
facilities this provides.

\subsection{Using \label{IMGREGIONS}regions and slices of images}
The NDF library allows you to specify that a rectangular region or
slice of an image should be used, rather than the whole. You identify
the part you want by appending a comma separated list of ranges, in
parentheses, to the name of the image.  For instance if you wanted to
access a square region of an image, you might use (the program running
here is the \htmlref{\myverb{mean}}{IMGMEAN} example\latexonly{ from
\S\ref{IMGMEAN}}):
\begin{code}
\begin{verbatim}
   % mean
   IN - Input image > image(100:200,100:200)
\end{verbatim}
\end{code}
and the square that covers the region from \myverb{100} to
\myverb{200} pixels in both dimensions will be used. The rest of the image is
ignored. Another way that you could get a similar effect would be to
use:
\begin{code}
\begin{verbatim}
   % mean
   IN - Input image > image(155~100,375~100)
\end{verbatim}
\end{code}
which selects a square of side \myverb{100} centred on
\myverb{155,375}.

Missing out a ``range'' results in the whole of that dimension being
used:
\begin{code}
\begin{verbatim}
   % mean
   IN - Input image > image(,100:200)
\end{verbatim}
\end{code}
This uses a rectangle that extends from \myverb{100} to \myverb{200} in the
second dimension and that spans the whole of the first dimension.

If you had a data cube and you wanted to process a plane from it,
you might use:
\begin{code}
\begin{verbatim}
   IN - Input image > cube(,,10)
\end{verbatim}
\end{code}
This would use the 2-D image stored in the tenth plane of the cube.

If the region of the image that you specify doesn't exist, then the
program will still be supplied with an image array of the requested
size, but the ``non-existent'' parts will be set to the bad value (see
\hyperref{elsewhere}{\S}{}{IMGBADDATA}).  A complete description of
how to use image ``sections'' (as they are called by the NDF library)
is given in \xref{SUN/33}{sun33}{}.

\subsection{Accessing \label{IMGFOREIGNDATAFORMATS} foreign data formats}
IMG can be used to access data in formats other than NDF by using the
NDF library's `on-the-fly' data conversion capabilities.  If you have
the CONVERT package (\xref{SUN/55}{sun55}{}) available on your system,
you can use it to give your IMG programs access to several important
\xref{additional astronomical data formats}{sun55}{sect_auto}
(including IRAF, disk FITS and FIGARO). All you need to do is issue
the package startup command:
\begin{code}
\begin{verbatim}
% convert
\end{verbatim}
\end{code}
before running your IMG program. It will then be able to access these other
formats in the same way as its ``native'' NDF format.

If you have copies of the same image in more than one format, you may
need to add the appropriate file type extension -- \myverb{.imh},
\myverb{.fit} and \myverb{.dst} for the IRAF, FITS and FIGARO formats
respectively -- as part of the image name, in order to distinguish
them.  So a typical session in which an IRAF data frame is read might
be (the program is the \htmlref{\myverb{mean}}{IMGMEAN}
example\latexonly{ from \S\ref{IMGMEAN}}):
\begin{code}
\begin{verbatim}
% convert
% mean
IN - Input image > myimage.imh
Mean = 108.3154
%
\end{verbatim}
\end{code}

One point of particular relevance to IMG is the transfer of header
information between the different data formats. Unless you take
special action, only FITS headers will generally be transferred. So,
unless you intend to make sole use of the NDF format, having sources
of header information other than FITS may cause problems and is
probably best avoided for the sake of simplicity.

If you want to process data in formats other than those provided by
CONVERT, then you can define your own conversions. You should consult
\xref{SSN/20}{ssn20}{} about how to do this.

\subsection{Using the NDF library directly}
When all else fails, the full power of the NDF library
(\xref{SUN/33}{sun33}{}) can be made available using the call:
\begin{code}
\begin{verbatim}
      CALL IMG_INDF( 'IN', ID, ISTAT )
\end{verbatim}
\end{code}
This returns an NDF identifier (\myverb{ID}) to your image dataset
(this shouldn't be confused with the pointers to images which we have
used so far). This allows you to get at the other components of the
NDF (our ``images'' are really the main NDF data array) such as its
variance, quality or world coordinate system. It also allows you to
access NDF extensions (our ``sources'' of header information) in more
sophisticated ways.  As a very simple example the next program snippet
shows how to write out the name of an image dataset
(also see \htmlref{IMG\_NAME}{IMG\_NAME}):
\begin{code}
\begin{verbatim}
      CALL IMG_INDF( 'IN', ID, ISTAT )                        [1]
      CALL NDF_MSG( 'NAME', ID )                              [2]
      CALL MSG_OUT( ' ', 'Name of image = ^NAME', ISTAT )     [3]
      CALL NDF_ANNUL( ID, ISTAT )                             [4]
\end{verbatim}
\end{code}
\mynote
\begin{enumnotes}
% 1
\item
\htmlref{\myverb{IMG\_INDF}}{IMG_INDF} returns an NDF identifier \myverb{ID}.

% 2
\item
The NDF subroutine \myverb{NDF\_MSG} sets a message token \myverb{'NAME'}
(see \xref{SUN/104}{sun104}{}).

% 3
\item
The name is written out.

% 4
\item
The NDF identifier is annulled. It is important that you remember to
do this once the identifier has been finished with.
\end{enumnotes}

\section{COMPILING \label{IMGCOMPILING}AND RUNNING PROGRAMS}
Compiling and linking a simple program is very straight-forward. If
your program source file is named \myverb{prog.f}, then the commands:
\begin{code}
\begin{verbatim}
% star_dev
% alink prog.f -L/star/lib `img_link_adam`
\end{verbatim}
\end{code}
will compile the source and link it against IMG (you should replace
the \myverb{/star} with where your system is installed on non-Starlink
machines). The final product is an executable called \myverb{prog},
that you run by typing:
\begin{code}
\begin{verbatim}
% prog
\end{verbatim}
\end{code}
(note the \myverb{\%} is the shell-prompt and should not be typed, and
the \myverb{``} are grave signs that execute the command
\myverb{img\_link\_adam}).  If you have a program that has several
modules that are compiled independently then you could use a
command like:
\begin{code}
\begin{verbatim}
% alink prog.f mysub1.o mysub2.o `img_link_adam`
\end{verbatim}
\end{code}
The \myverb{star\_dev} command only needs to be run once, as it creates
a permanent soft link for accessing the include file \myverb{SAE\_PAR}.
The \xref{\myverb{alink}}{sun144}{ADAM_link_scripts}
command is described more fully in \xref{SUN/144}{sun144}{}.

%%%%%%%%%%%%%%%%%%%%%%%%%%%%%%%%%%%%%%%%%%%%%%%%%%%%%%%%%%%%%%%%%%%%%%%%%%%%%%%
\appendix
\xlabel{appendix}

%begin{latexonly}
  \newpage
  \latexonlysection{ALPHABETICAL LIST OF FORTRAN SUBROUTINES}

% List of subroutines:
% =================

% Command for displaying routines in routine lists:
% =================================================

  \newcommand{\noteroutine}[3]{\item {\small\texttt{#1#2}} \nopagebreak \\
                                \hspace*{3em}\textit{#3} }
  {\small
  \latexonlysubsection{IMG Fortran subroutines}
  \begin{description}
    \noteroutine{IMG\_CANCL} {( PARAM, STATUS )}
       {Cancel an image/parameter association}
    \noteroutine{IMG\_DELET} {( PARAM, STATUS )}
       {Delete an image}
    \noteroutine{IMG\_FREE} {( PARAM, STATUS )}
       {Free an image}
    \noteroutine{IMG\_IN[n][x]} {( PARAM, NX, [NY], [NZ], IP, STATUS )}
       {Access an existing image for reading}
    \noteroutine{IMG\_INDF} {( PARAM, INDF, STATUS )}
       {Obtain an NDF identifier for an image}
    \noteroutine{IMG\_MOD[n][x]} {( PARAM, NX, [NY], [NZ], IP, STATUS )}
       {Access an image for modification}
    \noteroutine{IMG\_NAME} {( PARAM, VALUE, STATUS )}
       {Return the image name}
    \noteroutine{IMG\_NEW[n][x]} {( PARAM, NX, [NY], [NZ], IP, STATUS )}
       {Create a new image}
    \noteroutine{IMG\_OUT[x]} {( PARAM1, PARAM2, IP, STATUS )}
       {Create an output image}
    \noteroutine{IMG\_TMP[n][x]} {( PARAM, NX, [NY], [NZ], IP, STATUS )}
       {Create a temporary image}
  \end{description}

  \latexonlysubsection{HDR Fortran subroutines}
  \begin{description}
    \noteroutine{HDR\_COPY}{( PARAM1, XNAME1, PARAM2, XNAME2, STATUS )}
       {Copy header information from one image to another}
    \noteroutine{HDR\_DELET}{( PARAM, XNAME, ITEM, COMP, STATUS )}
       {Delete a header item}
    \noteroutine{HDR\_IN[x]}{( PARAM, XNAME, ITEM, COMP, VALUE, STATUS )}
       {Read a header item}
    \noteroutine{HDR\_MOD}{( PARAM, STATUS )}
       {Open an image allowing modification of any header items}
    \noteroutine{HDR\_NAME}{( PARAM, XNAME, N, ITEM, STATUS )}
       {Return a header item name}
    \noteroutine{HDR\_NUMB}{( PARAM, XNAME, ITEM, N, STATUS )}
       {Return a header item count}
    \noteroutine{HDR\_OUT[x]}{( PARAM, XNAME, ITEM, COMMEN, VALUE, STATUS )}
       {Write a header item}
  \end{description}
  } % end of \small
%end{latexonly}

\newpage
\section{FULL \label{IMGDESCRIPTIONS} FORTRAN SUBROUTINE DESCRIPTIONS}

\subsection{IMG Fortran subroutines}

\sstroutine{IMG\_CANCL}{
   Cancel an image/parameter association
}{
   \sstdescription{
      This subroutine may be used to free the resources associated with
      an image. Its behaviour is similar to that of
      \htmlref{IMG\_FREE}{IMG\_FREE}, except
      that it also removes the association between an image and a
      parameter. If the same parameter is later used to access an
      image, then a new image will be obtained (whereas if
      IMG\_FREE is used the original image will be re-accessed).
   }
   \sstinvocation{
      CALL IMG\_CANCL( PARAM, STATUS )
   }
   \sstarguments{
      \sstsubsection{
         PARAM = CHARACTER $*$ ( $*$ ) (Given)
      }{
         Parameter name, specifying either the individual
         image/parameter association to be cancelled, or \qs{$*$},
         indicating that all such associations should be cancelled.
      }
      \sstsubsection{
         STATUS = INTEGER (Given and Returned)
      }{
         The global status.
      }
   }

   \sstnotes{
      \sstitemlist{

         \sstitem
         This subroutine attempts to execute even if STATUS is set on entry,
         although no further error report will be made if it subsequently
         fails under these circumstances.

         \sstitem
         It is possible to cancel more than one association per call
         using multiple parameter names. Multiple parameter names are
         specified using a comma separated list of names (i.e.
         \qs{IMAGE1,IMAGE2}), or by using \qt{wild-card}
         characters as part of a parameter name. In this context, a
         \qs{$*$} will match any set of characters, while
         \qs{\%} will match any single character.  Note that
         only those parameter names previously used to access images
         via IMG subroutines are considered as potential matches.

         \sstitem Identifiers obtained using
         \htmlref{IMG\_INDF}{IMG\_INDF} are not released by this routine.
      }
   }
}
\sstroutine{IMG\_DELET}{
   Delete an image
}{
   \sstdescription{
      This subroutine deletes an image and frees any resources associated
      with it. It should be used as an alternative to
      \htmlref{IMG\_FREE}{IMG\_FREE} for any image which is not to be kept.

   }
   \sstinvocation{
      CALL IMG\_DELET( PARAM, STATUS )
   }
   \sstarguments{
      \sstsubsection{
         PARAM = CHARACTER $*$ ( $*$ ) (Given)
      }{
         Parameter name specifying the image to be deleted.
      }
      \sstsubsection{
         STATUS = INTEGER (Given and Returned)
      }{
         The global status.
      }
   }
   \sstnotes{
      \sstitemlist{
         \sstitem Only output or images accessed for modification
         may be deleted.

         \sstitem
         It is possible to delete more than one image per call using
         multiple parameter names. Multiple parameter names are
         specified using a comma separated list of names (i.e.
         \qs{IMAGE1,IMAGE2}). A wildcard capability is not
         supplied for this subroutine.

      }
   }
}
\sstroutine{IMG\_FREE}{
   Free an image
}{
   \sstdescription{
      This subroutine should be used to free the resources associated
      with an image when it has been finished with. This should
      always be done before the end of a program.
   }
   \sstinvocation{
      CALL IMG\_FREE( PARAM, STATUS )
   }
   \sstarguments{
      \sstsubsection{
         PARAM = CHARACTER $*$ ( $*$ ) (Given)
      }{
         Parameter name, specifying either the individual image to be
         freed or \qs{$*$}, indicating that all images should be
         freed.
      }
      \sstsubsection{
         STATUS = INTEGER (Given and Returned)
      }{
         The global status.
      }
   }
   \sstnotes{
      \sstitemlist{

         \sstitem
         This subroutine attempts to execute even if STATUS is set on entry,
         although no further error report will be made if it subsequently
         fails under these circumstances.

         \sstitem
         It is possible to free more than one image per call using
         multiple parameter names. Multiple parameter names are
         specified using a comma separated list of names (i.e.
         \qs{IMAGE1,IMAGE2}), or by using \qt{wild-card}
         characters as part of a parameter name. In this context, a
         \qs{$*$} will match any set of characters, while
         \qs{\%} will match any single character.  Note that
         only those parameter names previously used to access images
         via IMG subroutines are considered as potential matches.

         \sstitem Identifiers obtained using
         \htmlref{IMG\_INDF}{IMG\_INDF} are not released by this routine.
      }
   }
}
\sstroutine{IMG\_IN[n][x]}{
   Access an existing image for reading
}{
   \sstdescription{
      This subroutine accesses an input image for reading. It returns
      the size of the image and a pointer to its data.

      The [n] and [x] parts of the name are optional. If used [n]
      indicates the number of dimensions of the data (1 to 3, the
      default is 2) and [x] the data type (one of R, D, I, W, UW, B or
      UB, the default is R). So for instance if you wanted to use 1-D
      data with a data type of DOUBLE PRECISION, the subroutine that you
      should call is IMG\_IN1D.
   }
   \sstinvocation{ \vspace{1ex} \\
      CALL IMG\_IN( PARAM, NX, NY, IP, STATUS ) \\
      CALL IMG\_IN1[x]( PARAM, NX, IP, STATUS ) \\
      CALL IMG\_IN2[x]( PARAM, NX, NY, IP, STATUS ) \\
      CALL IMG\_IN3[x]( PARAM, NX, NY, NZ, IP, STATUS )
   }
   \sstarguments{
      \sstsubsection{
         PARAM = CHARACTER $*$ ( $*$ ) (Given)
      }{
         Parameter name (case insensitive).
      }
      \sstsubsection{
         NX = INTEGER (Returned)
      }{
         First dimension of the image (in pixels).
      }
      \sstsubsection{
         NY = INTEGER (Returned)
      }{
         Second dimension of the image (in pixels).
      }
      \sstsubsection{
         NZ = INTEGER (Returned)
      }{
         Third dimension of the image (in pixels).
      }
      \sstsubsection{
         IP = INTEGER (Returned)
      }{
         Pointer to image data.
      }
      \sstsubsection{
         STATUS = INTEGER (Given and Returned)
      }{
         The global status.
      }
   }
   \sstnotes{
      \sstitemlist{

         \sstitem
         It is possible to access more than one image per call
         using multiple parameter names. Multiple parameter names are
         specified by supplying a comma separated list of names
         (i.e. \qs{DATA,BIAS,FLAT}).  A pointer to the data
         of each image is then returned (in this case the IP argument
         should be passed as an array of size at least the number of
         parameter names). The advantage of obtaining a sequence of
         images in this manner is that the images are guaranteed to
         have the same shape and the same data type.

         \sstitem
         The message which complains when \qt{bad} (undefined) pixels
         are present in the input data can be stopped by following each
         parameter name by the character \qt{!},
         i.e. \qs{DATA!,BIAS!,FLAT} will inhibit checking the images
         associated with parameters \qs{DATA} and \qs{BIAS},
         but will check the image \qs{FLAT}. \qt{Bad} pixels
         have the symbolic values VAL\_\_BAD[x] which are defined
         in the include file PRM\_PAR.
      }
   }
}
\sstroutine{IMG\_INDF}{
   Return an image NDF identifier
}{
   \sstdescription{
      This subroutine returns an NDF identifier for an image. This
      identifier may be passed to subroutines from the NDF library
      (see \xref{SUN/33}{sun33}{}) to perform lower-level operations on the data which
      cannot be done using IMG subroutines.
   }
   \sstinvocation{
      CALL IMG\_INDF( PARAM, INDF, STATUS )
   }
   \sstarguments{
      \sstsubsection{
         PARAM = CHARACTER $*$ ( $*$ ) (Given)
      }{
         Parameter name (case insensitive).
      }
      \sstsubsection{
         INDF = INTEGER (Returned)
      }{
         NDF identifier for the image.
      }
      \sstsubsection{
         STATUS = INTEGER (Given and Returned)
      }{
         The global status.
      }
   }
   \sstnotes{
      \sstitemlist{

         \sstitem
         The NDF identifier returned by this subroutine should be annulled
         (e.g. using NDF\_ANNUL) when it is no longer required.  The IMG
         system will not perform this operation itself.

         \sstitem
         If this subroutine is called with STATUS set, then a value of
         NDF\_\_NOID will be returned for the INDF argument, although no
         further processing will occur. The same value will also be
         returned if the subroutine should fail for any reason.  The
         NDF\_\_NOID constant is defined in the NDF\_PAR include file.
      }
   }
}
\sstroutine{IMG\_MOD[n][x]}{
   Access an image for modification
}{
   \sstdescription{
      This routine provides access to an input image. It returns the
      size of the image and a pointer to its data. Existing values in
      the image may be modified.

      The [n] and [x] parts of the name are optional. If used [n]
      indicates the number of dimensions of the data (1 to 3, the
      default is 2) and [x] the data type (one of R, D, I, W, UW, B or
      UB, the default is R). So for instance if you wanted to use 3-D
      data with a data type of DOUBLE PRECISION, the subroutine that you
      should call is IMG\_MOD3D.
   }
   \sstinvocation{ \vspace{1ex} \\
      CALL IMG\_MOD( PARAM, NX, NY, IP, STATUS ) \\
      CALL IMG\_MOD1[x]( PARAM, NX, IP, STATUS ) \\
      CALL IMG\_MOD2[x]( PARAM, NX, NY, IP, STATUS ) \\
      CALL IMG\_MOD3[x]( PARAM, NX, NY, NZ, IP, STATUS ) \\
   }
   \sstarguments{
      \sstsubsection{
         PARAM = CHARACTER $*$ ( $*$ ) (Given)
      }{
         Parameter name (case insensitive).
      }
      \sstsubsection{
         NX = INTEGER (Returned)
      }{
         First dimension of the image (in pixels).
      }
      \sstsubsection{
         NY = INTEGER (Returned)
      }{
         Second dimension of the image (in pixels).
      }
      \sstsubsection{
         NZ = INTEGER (Returned)
      }{
         Third dimension of the image (in pixels).
      }
      \sstsubsection{
         IP = INTEGER (Returned)
      }{
         Pointer to image data.
      }
      \sstsubsection{
         STATUS = INTEGER (Given and Returned)
      }{
         The global status.
      }
   }
\newpage
   \sstnotes{
      \sstitemlist{
         \sstitem
         Access to multiple image data can also be provided by this
         routine. Multiple parameter names are specified by supplying a
         comma separated list of names (i.e. \qs{DATA,BIAS,FLAT}). A pointer
         to the data of each image is then returned (in this case the IP
         argument should be passed as an array of size at least the number
         of parameter names). The advantage of obtaining a sequence of
         images in this manner is that the images are guaranteed to have
         the same shape and the same data type.

         \sstitem
         The message which complains when \qt{bad} (undefined) pixels are
         present in the input data can be stopped by following each
         parameter name by the character \qt{!}, i.e. \qs{DATA!,BIAS!,FLAT}
         will
         inhibit checking the images associated with parameters
         \qs{DATA} and
         \qs{BIAS}, but will check the image \qs{FLAT}.
         \qt{Bad} pixels have the
         symbolic value VAL\_\_BAD[x] which is defined in the include file
         PRM\_PAR.

         \sstitem
         The input image is accessed in the type that you ask for
         regardless of its storage type. This may occasionally cause
         problems if these two types differ. This arises because when
         the image is converted to and from its storage type it is
         not always possible to guarantee that a valid value will result
         (for instance many numbers that you may store in a REAL
         variable cannot be stored in an INTEGER variable). When an
         image element cannot be converted it is assigned the
         appropriate bad value.
      }
   }
}

\sstroutine{
   IMG\_NAME
}{
   Return the image name
}{
   \sstdescription{
      This subroutine returns the name of the input image as a character
      string.
   }
   \sstinvocation{
      CALL IMG\_NAME( PARAM, VALUE, STATUS )
   }
   \sstarguments{
      \sstsubsection{
         PARAM = CHARACTER $*$ ( $*$ ) (Given)
      }{
         Parameter name of the image (case insensitive).
      }
      \sstsubsection{
         VALUE = CHARACTER $*$ ( $*$ ) (Returned)
      }{
         The name of the image.
      }
      \sstsubsection{
         STATUS = INTEGER (Given and Returned)
      }{
         The global status.
      }
   }
   \sstnotes{
      \sstitemlist{

         \sstitem
         The character string VALUE should be made large enough to
           contain the full name of the image.

         \sstitem
         This subroutine will directly access an image if an association
           has not already been made. Note that it will be opened for
           read-only access and you will not be able to write or delete any
           header items. If you need to be able to do this then access the
           image first, either with one of the
           \htmlref{IMG\_MOD[n][x]}{IMG\_MOD[n][x]} subroutines
           (if you intend to process the image data), or the
           \htmlref{HDR\_MOD}{HDR\_MOD} subroutine.
      }
   }
}


\sstroutine{IMG\_NEW[n][x]}{
   Create a new image
}{
   \sstdescription{
      This subroutine creates a new image and returns a pointer to its data.

      The [n] and [x] parts of the name are optional. If used [n]
      indicates the number of dimensions of the data (1 to 3, the
      default is 2) and [x] the data type (one of R, D, I, W, UW, B or
      UB, the default is R). So for instance if you wanted to use 3-D
      data with a data type of INTEGER, the subroutine that you should call
      is IMG\_NEW3I.

   }
   \sstinvocation{ \vspace{1ex} \\
      CALL IMG\_NEW( PARAM, NX, NY, IP, STATUS ) \\
      CALL IMG\_NEW1[x]( PARAM, NX, IP, STATUS ) \\
      CALL IMG\_NEW2[x]( PARAM, NX, NY, IP, STATUS ) \\
      CALL IMG\_NEW3[x]( PARAM, NX, NY, NZ, IP, STATUS )
   }
   \sstarguments{
      \sstsubsection{
         PARAM = CHARACTER $*$ ( $*$ ) (Given)
      }{
         Parameter name (case insensitive).
      }
      \sstsubsection{
         NX = INTEGER (Given)
      }{
         First dimension of the image (in pixels).
      }
      \sstsubsection{
         NY = INTEGER (Given)
      }{
         Second dimension of the image (in pixels).
      }
      \sstsubsection{
         NZ = INTEGER (Given)
      }{
         Third dimension of the image (in pixels).
      }
      \sstsubsection{
         IP = INTEGER (Returned)
      }{
         Pointer to the image data.
      }
      \sstsubsection{
         STATUS = INTEGER (Given and Returned)
      }{
         The global status.
      }
   }
   \sstnotes{
      \sstitemlist{

         \sstitem
         It is possible to create more than one image per call using
         multiple parameter names. Multiple names are specified by
         supplying a comma separated list (i.e. \qs{NEW1,NEW2}).
         A pointer to the data of each image is then returned (in
         this case the IP argument must be passed as an array of size
         at least the number of parameter names). An advantage of this
         method is that multiple images can be made using a single
         invocation of this subroutine.
      }
   }
}
\sstroutine{IMG\_OUT[x]}{
   Create an output image
}{
   \sstdescription{
      This subroutine creates a new output image by duplicating an input
      image.  A pointer is returned to the output image data.

      The [x] part of the name is optional. If used [x] indicates the
      data type (one of R, D, I, W, UW, B or UB, the default is R). So for
      instance if you wanted to make an output image with type REAL, the
      subroutine which you should call would be IMG\_OUT.
   }
   \sstinvocation{
      CALL IMG\_OUT[x]( PARAM1, PARAM2, IP, STATUS )
   }
   \sstarguments{
      \sstsubsection{
         PARAM1 = CHARACTER $*$ ( $*$ ) (Given)
      }{
         Parameter name for the input image (case insensitive).
      }
      \sstsubsection{
         PARAM2 = CHARACTER $*$ ( $*$ ) (Given)
      }{
         Parameter name for the new output image (case insensitive).
      }
      \sstsubsection{
         IP = INTEGER (Returned)
      }{
         Pointer to the mapped output data.
      }
      \sstsubsection{
         STATUS = INTEGER (Given and Returned)
      }{
         The global status.
      }
   }
   \sstnotes{
      \sstitemlist{

         \sstitem
         It is possible to copy the input image to more than one
         output image using multiple parameter names. Multiple
         parameter names are specified by supplying a comma separated
         list of names (i.e. \qs{OUT1,OUT2}). A pointer to
         the data of each image is then returned (in this case the IP
         argument must be passed as an array of size at least the
         number of parameter names). The advantage of using this
         method is that multiple copies of an input image can be made
         using a single invocation of this subroutine. Multiple input
         image names are not allowed.

         \sstitem
         The input image must have already been associated with a parameter
         before calling this subroutine.

         \sstitem
         The output image is created with the storage data type of the
         input image and is then accessed in the type that you ask for
         (\qt{[x]}). This may occasionally cause problems if these two
         types differ. This arises because when the output image is
         converted to and from its storage type it is not always
         possible to guarantee that a valid value will result (for
         instance many numbers that you may store in a REAL variable
         cannot be stored in an INTEGER variable). When an image
         element cannot be converted it is assigned the appropriate
         bad value.
      }
   }
}
\sstroutine{IMG\_TMP[n][x]}{
   Create a temporary image
}{
   \sstdescription{
      This subroutine creates a temporary image for use as workspace
      and returns a pointer to its data. The image will be deleted
      automatically when it is freed later (e.g. by calling
      \htmlref{IMG\_FREE}{IMG\_FREE}).

      The [n] and [x] parts of the name are optional. If used [n]
      indicates the number of dimensions of the data (1 to 3, the
      default is 2) and [x] the data type (one of R, D, I, W, UW, B or
      UB, the default is R). So for instance if you wanted to use 3-D
      data with a data type of INTEGER, the subroutine that you should call
      is IMG\_TMP3I.
   }
   \sstinvocation{ \\
      CALL IMG\_TMP( PARAM, NX, NY, IP, STATUS )\\
      CALL IMG\_TMP1[x]( PARAM, NX, IP, STATUS )\\
      CALL IMG\_TMP2[x]( PARAM, NX, NY, IP, STATUS )\\
      CALL IMG\_TMP3[x]( PARAM, NX, NY, NZ, IP, STATUS )
   }
   \sstarguments{
      \sstsubsection{
         PARAM = CHARACTER $*$ ( $*$ ) (Given)
      }{
         Parameter name (case insensitive).
      }
      \sstsubsection{
         NX = INTEGER (Given)
      }{
         First dimension of the image (in pixels).
      }
      \sstsubsection{
         NY = INTEGER (Given)
      }{
         Second dimension of the image (in pixels).
      }
      \sstsubsection{
         NZ = INTEGER (Given)
      }{
         Third dimension of the image (in pixels).
      }
      \sstsubsection{
         IP = INTEGER (Returned)
      }{
         Pointer to the image data.
      }
      \sstsubsection{
         STATUS = INTEGER (Given and Returned)
      }{
         The global status.
      }
   }
   \sstnotes{
      \sstitemlist{

         \sstitem
         It is possible to create more than one temporary image per
         call using multiple parameter names. These are specified by
         supplying a comma separated list of names (i.e.
         \qs{WORK1, WORK2, WORK3}). A pointer to the data of each
         image is then returned (in this case the IP argument should
         be passed as an array of size at least the number of
         parameter names).
      }
   }
}

\newpage
\subsection{HDR Fortran subroutines}

\sstroutine{
   HDR\_COPY
}{
   Copy header information from one image to another
}{
   \sstdescription{
      This routine copies a compatible source of header information
      from one image to another. FITS headers may only be copied to
      FITS headers, other sources may be copied without restriction.
   }
   \sstinvocation{
      CALL HDR\_COPY( PARAM1, XNAME1, PARAM2, XNAME2, STATUS )
   }
   \sstarguments{
      \sstsubsection{
         PARAM1 = CHARACTER $*$ ( $*$ ) (Given)
      }{
         Parameter name of the image containing the input source of
         header information (case insensitive).
      }
      \sstsubsection{
         XNAME1 = CHARACTER $*$ ( $*$ ) (Given)
      }{
         The name of the extension to be copied (\qs{FITS} or \qs{ }
         for FITS).
      }
      \sstsubsection{
         PARAM2 = CHARACTER $*$ ( $*$ ) (Given)
      }{
         Parameter name of the image that you want to copy a header
         source into (case insensitive).
      }
      \sstsubsection{
         XNAME2 = CHARACTER $*$ ( $*$ ) (Given)
      }{
         The name of the destination header source (\qs{FITS} or
         \qs{ } for FITS, must be FITS if XNAME1 is FITS).
      }
      \sstsubsection{
         STATUS = INTEGER (Given and Returned)
      }{
         The global status. If a header source or destination is FITS
         and the other isn't then IMG\_\_BDEXT will be returned.
      }
   }
   \sstnotes{
      \sstitemlist{

         \sstitem
         Modified header items associated with the input source will copied
         to the new image.

         \sstitem
         This routine may be used to copy the same header source (from a
           single input image), to more than one image at a time by using
           multiple parameter names for PARAM2. Multiple parameter names are
           provided as a comma separated list (i.e. \qs{OUT1,OUT2,OUT3}).
      }
   }
}

\sstroutine{HDR\_DELET}{
   Delete a header item
}{
   \sstdescription{
      This subroutine deletes a header item.
   }
   \sstinvocation{
      CALL HDR\_DELET( PARAM, XNAME, ITEM, COMP, STATUS )
   }
   \sstarguments{
      \sstsubsection{
         PARAM = CHARACTER $*$ ( $*$ ) (Given)
      }{
         Parameter name of the image (case insensitive).
      }
      \sstsubsection{
         XNAME = CHARACTER $*$ ( $*$ ) (Given)
      }{
         Source of the header information
         (\qs{FITS} or \qs{ } for FITS headers).
      }
      \sstsubsection{
         ITEM = CHARACTER $*$ ( $*$ ) (Given)
      }{
         Name of the header item.
      }
      \sstsubsection{
         COMP = INTEGER (Given)
      }{
         The component of a multiple FITS header item which is to be
         deleted, usually 1.
      }
      \sstsubsection{
         STATUS = INTEGER (Given and Returned)
      }{
         The global status.
      }
   }
   \sstnotes{
      \sstitemlist{

         \sstitem
         Only header items in output images or those accessed for
         modification will be deleted by this routine.

         \sstitem
         This subroutine will directly access an image if an
         association has not already been made.

         \sstitem The FITS items \qs{HISTORY}, \qs{COMMENT}
         and \qs{ } can have more than one occurrence. These can be deleted
         one-by-one by repetitive calls to this routine with a value of
         1 for the COMP argument (since if you delete the first component
         what was the second component now becomes the first). A specific
         occurrence can be deleted by giving the correct component number.
         The number of occurrences may be queried using the
         \htmlref{HDR\_NUMB}{HDR\_NUMB} subroutine.

         \sstitem
         Item names from any source may be hierarchical
         (i.e. ING.DETHEAD deletes the FITS header item \qt{ING
         DETHEAD}; BOUNDS.MAXX deletes the MAXX component of BOUNDS in
         a non-FITS source).

         \sstitem
         This subroutine may be used to delete an item in the same
         source of more than one image dataset at a time by using
         multiple parameter names. Multiple parameter names are
         provided as a comma separated list (i.e. \qs{IN1,IN2,IN3}).
      }
   }
}
\sstroutine{HDR\_IN[x]}{
   Read a header item
}{
   \sstdescription{
      This subroutine returns the value of a header item.

      The [x] part of the name is optional. If used [x] indicates the
      data type (one of R, D, I, L or C, the default is C) the
      item is to be returned in. So for instance if you want an item
      with type REAL, the subroutine that you should call is HDR\_INR.
   }
   \sstinvocation{
      CALL HDR\_IN[x]( PARAM, XNAME, ITEM, COMP, VALUE, STATUS )
   }
   \sstarguments{
      \sstsubsection{
         PARAM = CHARACTER $*$ ( $*$ ) (Given)
      }{
         Parameter name of the image (case insensitive).
      }
      \sstsubsection{
         XNAME = CHARACTER $*$ ( $*$ ) (Given)
      }{
         Source of the header information
         (\qs{FITS} or \qs{ } for FITS headers).
      }
      \sstsubsection{
         ITEM = CHARACTER $*$ ( $*$ ) (Given)
      }{
         Name of the header item.
      }
      \sstsubsection{
         COMP = INTEGER (Given)
      }{
         The component of a multiple FITS header item (\qs{HISTORY} and
         \qs{COMMENT} items often have many occurrences). The number of
         occurrences may be queried using the \htmlref{HDR\_NUMB}{HDR\_NUMB}
         subroutine.
      }
      \sstsubsection{
         VALUE = ? (Given and Returned)
      }{
         The value. This is unmodified if the item doesn't exist.
      }
      \sstsubsection{
         STATUS = INTEGER (Given and Returned)
      }{
         The global status.
      }
   }
   \sstnotes{
      \sstitemlist{

         \sstitem
         This subroutine will directly access an image if an
         association has not already been made. Note that it will be
         opened for read-only access and you will not be able to write
         or delete any header items. If you need to be able to do this
         then access the image first, either with one of the
         \htmlref{IMG\_MOD[n][x]}{IMG\_MOD[n][x]} subroutines
         (if you intend to process the image data), or the
         \htmlref{HDR\_MOD}{HDR\_MOD} subroutine.

         \sstitem
         Item names from any source may be hierarchical
         (i.e. ING.DETHEAD gets the value of the FITS header
         \qt{ING DETHEAD}; BOUNDS.MAXX gets the value of the
         MAXX component of BOUNDS in a non-FITS source).

         \sstitem
         This subroutine may be used to read the value of an item from
         the same source of more than one image dataset at a time by
         using multiple parameter names. Multiple parameter names are
         provided as a comma separated list (i.e. \qs{IN1,IN2,IN3}).
         The header source specified by XNAME must exist in all images
         and the argument VALUE must be declared as a dimension of
         size at least the number of parameters in the list; if this
         option is used.

         \sstitem
         If a header item is not found its associated element of the
         VALUE argument will remain unchanged. It is therefore important
         that suitable defaults are assigned to VALUE before calling this
         subroutine. The source must exist.
      }
   }
}

\sstroutine{HDR\_MOD}{
   Open an image allowing modification of any header items
}{
   \sstdescription{
      This subroutine opens an image and allows the header information
      to be modified by other HDR routines.
   }
   \sstinvocation{
      CALL HDR\_MOD( PARAM, STATUS )
   }
   \sstarguments{
      \sstsubsection{
         PARAM = CHARACTER $*$ ( $*$ ) (Given)
      }{
         Parameter name of the image (case insensitive).
      }
      \sstsubsection{
         STATUS = INTEGER (Given and Returned)
      }{
         The global status.
      }
   }
\newpage
   \sstnotes{
      \sstitemlist{

         \sstitem
         This subroutine may be used to obtain modification access to more
         than one image at a time by using multiple parameter
         names. Multiple parameter names are provided as a comma separated
         list (i.e. \qs{IN1,IN2,IN3}).

         \sstitem
         This subroutine should only be used in programs that only
         modify header items. If you intend to also modify the image
         data  then use an \htmlref{IMG\_MOD[n][x]}{IMG\_MOD[n][x]}
         subroutine.

      }
   }
}

\sstroutine{HDR\_NAME}{
   Return a header item name
}{
   \sstdescription{
      This subroutine returns the name of a header item using an index
      of its relative position within a named source. By incrementing
      index N all the names in a source may be queried.
   }
   \sstinvocation{
      CALL HDR\_NAME( PARAM, XNAME, N, ITEM, STATUS )
   }
   \sstarguments{
      \sstsubsection{
         PARAM = CHARACTER $*$ ( $*$ ) (Given)
      }{
         Parameter name of the image (case insensitive).
      }
      \sstsubsection{
         XNAME = CHARACTER $*$ ( $*$ ) (Given)
      }{
         Source of the header information
         (\qs{FITS} or \qs{ } for FITS headers).
      }
      \sstsubsection{
         N = INTEGER (Given)
      }{
         The index of the item.
      }
      \sstsubsection{
         ITEM = CHARACTER $*$ ( $*$ ) (Returned)
      }{
         The name of the header item (blank when no item with the
         given index exists).
      }
      \sstsubsection{
         STATUS = INTEGER (Given and Returned)
      }{
         The global status.
      }
   }
   \sstnotes{
      \sstitemlist{

         \sstitem
         This subroutine will directly access an image if an
         association has not already been made. Note that it will be
         opened for read-only access and you will not be able to write
         or delete any header items. If you need to be able to do this
         then access the image first, either with one of the
         \htmlref{IMG\_MOD[n][x]}{IMG\_MOD[n][x]} subroutines
         (if you intend to process the image data), or the
         \htmlref{HDR\_MOD}{HDR\_MOD} subroutine.

         \sstitem
         The order in which header item names are returned may change if
         the source is modified (by deletion or by writing).

         \sstitem
         The header item name will be returned in a hierarchical format
         if necessary.

         \sstitem
         The total number of items from a source can be found using
         \htmlref{HDR\_NUMB}{HDR\_NUMB}.
      }
   }
}
\sstroutine{HDR\_NUMB}{
   Return a header item count
}{
   \sstdescription{
      This subroutine returns the number of header items or the number
      of occurrences of a specific item.  The numbers returned can be
      used as limits when indexing using \htmlref{HDR\_NAME}{HDR\_NAME} or when
      deleting/reading items with more than one occurrence (such as
      FITS \qs{HISTORY} or \qs{COMMENTS}) or to test the existence of
      an item.
   }
   \sstinvocation{
      CALL HDR\_NUMB( PARAM, XNAME, ITEM, N, STATUS )
   }
   \sstarguments{
      \sstsubsection{
         PARAM = CHARACTER $*$ ( $*$ ) (Given)
      }{
         Parameter name of the image (case insensitive).
      }
      \sstsubsection{
         XNAME = CHARACTER $*$ ( $*$ ) (Given)
      }{
         Source of the header information
         (\qs{FITS} or \qs{ } for FITS headers).
      }
      \sstsubsection{
         ITEM = CHARACTER $*$ ( $*$ ) (Given)
      }{
         The name of an item or \qs{$*$}. If this is \qs{$*$} then a
         count of all the items is returned, otherwise the
         number of occurrences of the named item is returned.
      }
      \sstsubsection{
         N = INTEGER (Returned)
      }{
         The number of header items or occurrences  of an item.
      }
      \sstsubsection{
         STATUS = INTEGER (Given and Returned)
      }{
         The global status.
      }
   }
   \sstnotes{
      \sstitemlist{

         \sstitem
         This subroutine will directly access an image if an
         association has not already been made. Note that it will be
         opened for read-only access and you will not be able to write
         or delete any header items. If you need to be able to do this
         then access the image first, either with one of the
         \htmlref{IMG\_MOD[n][x]}{IMG\_MOD[n][x]} subroutines
         (if you intend to process the image data), or the
         \htmlref{HDR\_MOD}{HDR\_MOD} subroutine.

         \sstitem
         This subroutine may be used to query the number of items or
         occurrences of an item in the same source of more than one
         image dataset at a time by using multiple parameter names.
         Multiple parameter names are provided as a comma separated list
         (i.e. \qs{IN1,IN2,IN3}). The source must exist in all images and
         the argument N will be returned as the maximum number of header
         items or occurrences located (from each source).

         \sstitem
         If the number of header items is zero then this will be
         returned, no error will be reported.

         \sstitem
         The number of occurrences of an item will usually be 1, except
         for the special FITS items \qs{COMMENT}, \qs{HISTORY} and
         \qs{ }. This ability is therefore most useful for testing the
         existence of items (0 is returned if an item isn't found) or
         for manipulation of the occurrences of the special FITS items.
      }
   }
}
\sstroutine{HDR\_OUT[x]}{
   Write a header item
}{
   \sstdescription{
      This subroutine writes a header item.

      The [x] part of the name is optional. If used [x] indicates the
      data type (one of R, D, I, L or C, the default is C) of
      the item. So for instance if you want an item with type REAL,
      the subroutine that you should call is HDR\_OUTR.
   }
   \sstinvocation{
      CALL HDR\_OUT[x]( PARAM, XNAME, ITEM, COMMEN, VALUE, STATUS )
   }
   \sstarguments{
      \sstsubsection{
         PARAM = CHARACTER $*$ ( $*$ ) (Given)
      }{
         Parameter name of the image (case insensitive).
      }
      \sstsubsection{
         XNAME = CHARACTER $*$ ( $*$ ) (Given)
      }{
         Source of the header information
         (\qs{FITS} or \qs{ } for FITS headers).
      }
      \sstsubsection{
         ITEM = CHARACTER $*$ ( $*$ ) (Given)
      } {
         Name of the header item.
      }
      \sstsubsection{
         COMMEN = CHARACTER $*$ ( $*$ ) (Given)
      }{
         If XNAME is \qs{FITS} then this is used as a comment, otherwise
         it is not used.
      }
      \sstsubsection{
         VALUE = ? (Given)
      }{
         The value.
      }
      \sstsubsection{
         STATUS = INTEGER (Given and Returned)
      }{
         The global status.
      }
   }
   \sstnotes{
      \sstitemlist{

         \sstitem
         This subroutine will directly access an image if an
         association has not already been made. If an association has
         already been made then the image must be opened for write or
         modification access (by \htmlref{IMG\_MOD[n][x]}{IMG\_MOD[n][x]}
         or \htmlref{IMG\_OUT[x]}{IMG\_OUT[x]}, if you also
         want to process the image data and \htmlref{HDR\_MOD}{HDR\_MOD}
         otherwise).

         \sstitem
         Item names from any source may be hierarchical
         (i.e. ING.DETHEAD writes the FITS header \qt{ING DETHEAD};
         BOUNDS.MAXX the value of the MAXX component of BOUNDS in a
         non-FITS source). Writing hierarchical records in FITS
         records is strongly discouraged (as it violates the standard).

         \sstitem
         This subroutine may be used to write the value of items in
         the same source of more than one image dataset at a time by
         using multiple parameter names. Multiple parameter names are
         provided as a comma separated list (i.e. \qs{IN1,IN2,IN3}). Note
         that the argument VALUE must be declared as a dimension of size
         at least the number of parameters in the list, if this option is
         used.
      }
   }
}

\newpage

\section{USING \xlabel{IMGCinterface}IMG FROM C\label{IMGCinterface}}

C functions, that are equivalent to the Fortran subroutines, are
available for those who prefer this language (indeed using IMG from C
can seem more natural as IMG returns pointers to data).  The major
changes in the interface (compared with the Fortran specifications)
are that the names of functions change from \myverb{img\_something} or
\myverb{hdr\_something} to \myverb{imgSomething} and
\myverb{hdrSomething} and that ``strings'' which are returned or can
be passed as arrays also have a length argument.

\newcommand{\cnoteroutine}[4]{
   \item {\small \texttt{#1~#2 #3}} \nopagebreak \\ {\em #4 }
}
\subsection{IMG C functions}
\begin{latexonly}
{\small
\begin{description}
  \cnoteroutine{void}{imgCancl}{(char *param, int *status)}
     {Cancel an image/parameter association}
  \cnoteroutine{void}{imgDelet}{(char *param, int *status)}
     {Delete an image}
  \cnoteroutine{void}{imgFree}{(char *param, int *status)}
     {Free an image}
  \cnoteroutine{void}{imgIn[N][X]}
     {(char *param, int *nx, [int *ny,] [int *nz,] TYPE **ip, int *status)}
     {Access an existing image for reading}
  \cnoteroutine{void}{imgIndf}{(char *param, int *indf, int *status)}
     {Obtain an NDF identifier for an image}
  \cnoteroutine{void}{imgMod[N][X]}
     {(char *param, int *nx, [int *ny,] [int nz,] TYPE **ip, int *status)}
     {Access an image for modification}
  \cnoteroutine{void}{imgName[N][X]}
     {(char *param, char *value, int value\_length, int *status)}
     {Return the image name}
  \cnoteroutine{void}{imgNew[N][X]}
     {(char *param, int nx, [int ny,] [int nz,] TYPE **ip, int *status)}
     {Create a new image}
  \cnoteroutine{void}{imgOut[X]}
     {(char *param1, char *param2, TYPE **ip, int *status)}
     {Create an output image}
  \cnoteroutine{void}{imgTmp[N][X]}
     {(char *param, int nx, [int ny,] [int nz,] TYPE **ip, int *status)}
     {Create a temporary image}
\end{description}
}
\end{latexonly}
\begin{rawhtml}
<PRE>
<A NAME="xref_imgCancl"></A>
  void <B>imgCancl</B>(char *param, int *status)
     Cancel an image/parameter association
<A NAME="xref_imgDelet"></a>
  void <B>imgDelet</B>(char *param, int *status)
     Delete an image
<A NAME="xref_imgFree"></A>
  void <B>imgFree</B>(char *param, int *status)
     Free an image
<A NAME="xref_imgIn[N][X]"></A>
  void <B>imgIn[N][X]</B>(char *param, int *nx, [int *ny,] [int *nz,] TYPE **ip, int *status)
     Access an existing image for reading
<A NAME="xref_imgIndf"></A>
  void <B>imgIndf</B>(char *param, int *indf, int *status)
     Obtain an NDF identifier for an image
<A NAME="xref_imgMod[N][X]"></A>
  void <B>imgMod[N][X]</B>(char *param, int *nx, [int *ny,] [int nz,] TYPE **ip, int *status)
     Access an image for modification
<A NAME="xref_imgNew[N][X]"></A>
  void <B>imgNew[N][X]</B>(char *param, int nx, [int ny,] [int nz,] TYPE **ip, int *status)
     Create a new image
<A NAME="xref_imgName"></A>
  void <B>imgName</B>(char *param, char *value, int value_length, int *status)
     Return the image name
<A NAME="xref_imgOut[X]"></A>
  void <B>imgOut[X]</B>(char *param1, char *param2, TYPE **ip, int *status)
     Create an output image
<A NAME="xref_imgTmp[N][X]"></A>
  void <B>imgTmp[N][X]</B>(char *param, int nx, [int ny,] [int nz,] TYPE **ip, int *status)
     Create a temporary image
</PRE>
\end{rawhtml}

Where the parts in \texttt{"[]"} are optional or not always available.
\texttt{[N]} indicates the number of dimensions of the data you want to access
and the necessary \texttt{nx}, \texttt{ny} and \texttt{nz} arguments should
be passed.
\begin{small}
\begin{center}
\begin{tabular}{||l|l||}
\hline
Number of dimensions \texttt{[N]}    &    \texttt{nx}, \texttt{ny} \& \texttt{nz} required \\
\hline
     \texttt{1}                      &    \texttt{nx}       \\
     \texttt{2}                      &    \texttt{nx} \& \texttt{ny} \\
     \texttt{3}                      &    \texttt{nx}, \texttt{ny} \& \texttt{nz} \\
\hline
\end{tabular}
\end{center}
\end{small}
The default number of dimensions are two (in which case do not replace
\texttt{[N]} by anything).

The ordering of array dimensions is different in C to Fortran, so for
instance the Fortran array \myverb{ARRAY(NX,NY,NZ)} is equivalent to
the C array \myverb{ARRAY[NZ][NY][NX]}, but usually this is of little
importance since images are not of fixed sizes (so you're unlikely to
hard code any array dimensions) and you're far more likely to use the
data via a pointer to the first element. Since C arrays also always start
at 0 (rather than 1 as is the default in Fortran) the index of an
element say \texttt{[y][x]} is
\texttt{nx}~$\times$~\texttt{y}~$+$~\texttt{x}. In 3-D this becomes
\texttt{[z][y][x]} ---
\texttt{nx}~$\times$~\texttt{ny}~$\times$~\texttt{z}~$+$~\texttt{nx}~$\times$~\texttt{y}~$+$~\texttt{x}.

\label{IMGCdatatypes}
The data type of the image is specified by replacing the \texttt{[X]} part
of the name by one of the following character codes:
\begin{small}
\begin{center}
\begin{tabular}{||l|l|l||}
\hline
Character code {\tt [X]} & C data type (\texttt{TYPE}) &  Description \\
\hline
 \texttt{F}       & \texttt{float}          &   \\
 \texttt{D}       & \texttt{double}         &  \\
 \texttt{I}       & \texttt{int}            &  \\
 \texttt{S}       & \texttt{short}          &  \\
 \texttt{US}      & \texttt{unsigned short} &  \\
 \texttt{B}       & \texttt{signed char}    &  Signed byte      \\
 \texttt{UB}      & \texttt{unsigned char}  &  Unsigned byte    \\
\hline
\end{tabular}
\end{center}
\end{small}
You will also need to replace \texttt{TYPE} by the related data type.
The default data type is \texttt{float} (in which case do not replace
\texttt{[X]} by anything and set \texttt{TYPE} to \texttt{float}).

\subsection{HDR C functions}
\begin{latexonly}
{\small
\begin{description}
  \cnoteroutine{void}{hdrCopy}
     {( char *param1, char *xname1, char *param2, char *xname2, int *status )}
     {Copy header information from one image to another}
  \cnoteroutine{void}{hdrDelet}
     {(char *param, char *xname, char *item, int comp, int *status)}
     {Delete a header item}
  \cnoteroutine{void}{hdrIn}
     {(char *param, char *xname, char *item, int comp, char *value,
       \\ \hspace*{3em} int~value\_length, int~*status)}
     {Read a character header item}
  \cnoteroutine{void}{hdrInC}
     {(char *param, char *xname, char *item, int comp, char *value,
       \\ \hspace*{3em} int~value\_length, int~*status)}
     {Read a character header item}
  \cnoteroutine{void}{hdrIn[X]}
     {(char *param, char *xname, char *item, int comp, TYPE *value,
       \\ \hspace*{3em} int *status)}
     {Read a header item}
  \cnoteroutine{void}{hdrMod}
     {(char *param, int *status)}
     {Open an image allowing modification of any header items}
  \cnoteroutine{void}{hdrName}
     {(char *param, char *xname, int n, char *item, int item\_length,
       \\ \hspace*{3em} int *status)}
     {Return a header item name}
  \cnoteroutine{void}{hdrNumb}
     {(char *param, char *xname, char *item, int *n, int *status)}
     {Return a header item count}
  \cnoteroutine{void}{hdrOut}
     {(char *param, char *xname, char *item, char *commen,
       char *value, \\ \hspace*{3em} int~value\_length, int~*status)}
     {Write a character header item}
  \cnoteroutine{void}{hdrOutC}
     {(char *param, char *xname, char *item, char *commen,
       char *value, \\ \hspace*{3em} int~value\_length, int~*status)}
     {Write a character header item}
  \cnoteroutine{void}{hdrOut[X]}
     {(char *param, char *xname, char *item, char *commen,
       TYPE *value, \\ \hspace*{3em} int *status)}
     {Write a header item}
\end{description}
}
\end{latexonly}
\begin{rawhtml}
<PRE>
<A NAME="xref_hdrDelet"></A>
  void <B>hdrDelet</B>(char *param, char *xname, char *item, int comp, int *status)
     Delete a header item
<A NAME="xref_hdrIn"></A>
  void <B>hdrIn</B>(char *param, char *xname, char *item, int comp, char *value, int value_length, int *status)
     Read a character header item
<A NAME="xref_hdrInc"></A>
  void <B>hdrInC</B>(char *param, char *xname, char *item, int comp, char *value, int value_length, int *status)
     Read a character header item
<A NAME="xref_hdrIn[X]"></A>
  void <B>hdrIn[X]</B>(char *param, char *xname, char *item, int comp, TYPE *value, int *status)
     Read a header item
<A NAME="xref_hdrMod"></A>
  void <B>hdrMod</B>(char *param, int *status)
     Open an image allowing modification of any header items
<A NAME="xref_hdrName"></A>
  void <B>hdrName</B>(char *param, char *xname, int n, char *item, int item_length, int *status)
     Return a header item name
<A NAME="xref_hdrNumb"></A>
  void <B>hdrNumb</B>(char *param, char *xname, char *item, int *n, int *status)
     Return a header item count
<A NAME="xref_hdrOut"></A>
  void <B>hdrOut</B>(char *param, char *xname, char *item, char *commen, char *value,   int value_length, int *status)
     Write a character header item
<A NAME="xref_hdrOutC"></A>
  void <B>hdrOutc</B>(char *param, char *xname, char *item, char *commen, char *value,   int value_length, int *status)
     Write a character header item
<A NAME="xref_hdrOut[X]"></A>
  void <B>hdrOut[X]</B>(char *param, char *xname, char *item, char *commen, TYPE *value,   int *status)
     Write a header item
</PRE>
\end{rawhtml}
\label{HDRCdatatypes}
Where the \texttt{"[X]"} indicate the data type of the header items.
The possible values of \texttt{[X]} are:
\begin{small}
\begin{center}
\begin{tabular}{||l|l|l||}
\hline
Character code[X] & C data type (\texttt{TYPE)} &  Description      \\
\hline
 \texttt{C}          & \texttt{char}               &  Character string \\
 \texttt{D}          & \texttt{double}             &  \\
 \texttt{I}          & \texttt{int}                &  \\
 \texttt{L}          & \texttt{int}                &  Boolean          \\
 \texttt{F}          & \texttt{float}              &  \\
\hline
\end{tabular}
\end{center}
\end{small}

\subsection{Examples of using IMG from C}
\subsubsection{Accessing an existing image}
This section shows a C version of the \myverb{mean} program from
\hyperref{elsewhere}{\S}{}{IMGMEAN}.
\begin{code}
\begin{verbatim}
#include <stdio.h>
#include "img.h"                                              [1]

void mean_(int *istat)                                        [2]
{
  /* Local variables: */
  float *ip, sum;
  int nx, ny, i;

  /*  Access the input image. */
  imgIn( "in", &nx, &ny, &ip, istat );

  /*  Derive the mean and write it out. */
  sum = 0.0f;
  for( i=0; i < nx*ny; i++ ) sum += ip[i];
  printf ("Mean value = %f\n", sum/(nx*ny) );

  /*  Free the input image. */
  imgFree( "in", istat );
}
\end{verbatim}
\end{code}

\mynote

\begin{enumnotes}
% 1
\item The file \myverb{img.h} contains prototypes of all the IMG
and HDR functions.
% 2
\item As in the Fortran case this program needs to be created as a
function with an \myverb{int *} argument. This is a replacement for
the \myverb{main} function. Note that the function is named the same
as the eventual program {\em except} for the trailing underscore. The
underscore is required as this function will be called from a Fortran
subroutine (you should also note that this ``trick'' may not be the
same from machine to machine, to do this portably you should use the
\xref{CNF}{sun209}{} (SUN/209) macros as in the full example of
\htmladdnormallink{\myverb{mean.c}}{../../bin/examples/img/mean.c}).
\end{enumnotes}

As before an interface file -
\htmladdnormallink{\myverb{mean.ifl}}{../../bin/examples/img/mean.ifl}
- is also required to complete the program.

\subsubsection{Creating a new image}

This example
(\htmladdnormallink{\myverb{flat.c}}{../../bin/examples/img/flat.c})
shows how to create a new image. It also sets all the elements of the
image to 1.0.

\begin{code}
\begin{verbatim}
#include <stdio.h>
#include "img.h"

void flat_(int *istat)
{


  /*  Local variables: */
  float *ip;
  int i;

  /*  Create a new image. */
  imgNew( "out", 416, 578, &ip, istat );

  /*  Set all its elements to the value 1.0. */
  for( i=0; i < 416*578; i++ ) {
    ip[i] = 1.0f;
  }

  /*  Free the new image. */
  imgFree( "out", istat );
}
\end{verbatim}
\end{code}

\subsubsection{Modifying an image}
This example shows how to access an existing image and modify
its values.
\begin{code}
\begin{verbatim}
   /* Access an existing image. */
   imgMod( "in", &nx, &ny, &ip, istat );

   /* Set all elements with a value above 32000 to 32000. */
   for( i=0; i <nx*ny; i++ ) {
       if ( ip[i] > 32000.0f ) {
          ip[i] = 32000.0f;
       }
   }

   /* Free the image */
   imgFree( "in", istat );
\end{verbatim}
\end{code}

\subsubsection{Modifying a copy of an image}
This example
(\htmladdnormallink{\myverb{add.c}}{../../bin/examples/img/add.c})
copies the input image and then modifies the copy.
\begin{code}
\begin{verbatim}
  /*  Access an existing image */
  imgIn( "in", &nx, &ny, &ptrIn, status );

  /*  Copy this to an output image */
  imgOut( "in", "out", &ptrOut, status );

  /*  Get the value to add. */
  printf("Value to add to image: ");
  scanf( "%f", &value );

  /*  And do the work. */
  for( i=0; i <nx*ny; i++ ) {
    ptrOut[i] = value + ptrIn[i];
  }

  /*  Free the input and output images. */
  imgFree( "*", status );
\end{verbatim}
\end{code}

\subsubsection{Getting images as workspace}
This example
(\htmladdnormallink{\myverb{shadow.c}}{../../bin/examples/img/shadow.c})
shows how to create temporary images.
The \xref{HDS}{sun92}{} (SUN/92) manual describes
\xref{some reasons}{sun92}{HDS_MAP_tuning_parameter}
why you might consider using temporary images rather than
``malloc'd'' memory in your programs.
\begin{code}
\begin{verbatim}
  /*  Access the input image. */
  imgIn( "in", &nx, &ny, &ipin, istat );

  /*  Get a same sized temporary image as workspace. */
  imgTmp( "temp", nx, ny, &iptemp, istat );

  /*  Free all images (this deletes the temporary image). */
  imgFree( "*", istat );
\end{verbatim}
\end{code}

\subsubsection{Using ``images'' which are not 2-dimensional}
This section just shows some example calls that access spectra and
data-cubes.
\begin{code}
\begin{verbatim}
   imgIn1( "spectrum", &nx, &ip, istat );
   imgIn3( "cube", &nx, &ny, &nz, &ip, istat );
\end{verbatim}
\end{code}

\subsubsection{Accessing images using different data types}
So far all the example shown access the image data assuming a type of
\myverb{float}. IMG can also access data in other useful types. A list
of these is shown \hyperref{elsewhere}{in \S}{}{IMGCdatatypes}. Among
the possible calls are:
\begin{code}
\begin{verbatim}
   double *dPtr;
   imgInD( "image", &nx, &ny, &dPtr, istat );
   float *fPtr;
   imgInF( "image", &nx, &ny, &fPtr, istat );
   short *sPtr;
   imgInS( "image", &nx, &ny, &sPtr, istat );
   int *iPtr;
   imgInI( "image", &nx, &ny, &iPtr, istat );
\end{verbatim}
\end{code}

Requirements for data types and dimensionalities can be mixed as in:
\begin{code}
\begin{verbatim}
   imgIn1D( 'spectrum', &nx, &ip, istat )
   imgMod2I( 'image', &nx, &ny, &ip, istat )
   imgNew3S( 'cube', &nx, &ny, &nz, ip, istat )
   imgIn2F( 'image', &nx, &ny, &ip, istat )
\end{verbatim}
\end{code}

\subsubsection{Accessing multiple images}
This example
(\htmladdnormallink{\myverb{proc.c}}{../../bin/examples/img/proc.c})
shows how IMG can access more than one image per function call:
\begin{code}
\begin{verbatim}
   /*  Declare pointers for images. */
   float *ip[3], *ipproc, *ipbias, *ipflat, *ipraw ;              [1]
   int nx, ny, i;

   /*  Access input images. */
   imgIn( "bias,flat,raw", &nx, &ny, ip, istat );                 [2]
   ipbias = ip[0];
   ipflat = ip[1];
   ipraw = ip[2];

   /*  Create a new output image by copying the raw input image. */
   imgOut( "raw", "proc", &ipproc, istat );

   /*  Debias and flatfield data. */
   for( i=0; i <nx*ny; i++ ) {
      *ipproc++ = ( *ipraw++ - *ipbias++ ) / *ipflat++;
   }

   /*  Free all the images.*/
   imgFree( "*", istat );
\end{verbatim}
\end{code}
\mynote
\begin{enumnotes}
% 1
\item Note that the variable \myverb{ip} is declared as an array of
pointers to \myverb{float}.
%2
\item After this call each of the elements of \myverb{ip} is set to
point to a different image. The advantage of such a call is that all
the images are guaranteed to be the same shape and data type.
\end{enumnotes}

\subsubsection{Handling ``bad'' data}
The way that you handle ``bad'' values in your data is similar to that
used in Fortran, see
\hyperref{here}{\S}{}{IMGBADDATA},
however, the bad data values themselves are now defined in the
\myverb{img.h} file and have names to reflect the C data types.
\begin{small}
\begin{center}
\begin{tabular}{||l|l|l||}
\hline
Bad value macro  & C data type          \\
\hline
 \texttt{VAL\_\_BADF}   & \texttt{float}          \\
 \texttt{VAL\_\_BADD}   & \texttt{double}         \\
 \texttt{VAL\_\_BADI}   & \texttt{int}            \\
 \texttt{VAL\_\_BADS}   & \texttt{short}          \\
 \texttt{VAL\_\_BADUS}  & \texttt{unsigned short} \\
 \texttt{VAL\_\_BADB}   & \texttt{signed char}    \\
 \texttt{VAL\_\_BADUB}  & \texttt{unsigned char}  \\
\hline
\end{tabular}
\end{center}
\end{small}

\subsection{Examples of using HDR from C}

\subsubsection{Reading header items}

This example
(\htmladdnormallink{\myverb{hdrread.c}}{../../bin/examples/img/hdrread.c})
shows how to read the value of a header item.
\begin{code}
\begin{verbatim}

  /*  Local Variables: */
  char item[] = "OBSERVER";
  char value[40];                                 [1]

  /*  Try to read the value. */
  hdrIn( "in", " ", item, 1, value, 40, istat );  [2]

  /*  And write it out */
  printf( "This data was taken by %s.\n", value);

  /*  Free the image */
  imgFree( "in", istat );
\end{verbatim}
\end{code}
\mynote
\begin{enumnotes}
% 1
\item A character ``string'' is declared to contain the value of the
header item.
% 2
\item The size of the string is passed as an additional argument when
calling \myverb{hdrIn}
(this argument is not present in the Fortran version).
\end{enumnotes}

\subsubsection{Writing header items}
This example
(\htmladdnormallink{\myverb{hdrwrite.c}}{../../bin/examples/img/hdrwrite.c})
shows how to write a header item to an image.
\begin{code}
\begin{verbatim}
  /*  Write the header item. */
  hdrOut( "out", " ", "OBSERVER", "The observer.", "Fred Bloggs", 0, istat ); [1]

  /*  Free the image. */
  imgFree( "out", istat );
\end{verbatim}
\end{code}
\mynote
\begin{enumnotes}
% 1
\item Note the extra argument which is set to 0. This indicates that a
single string (\myverb{"Fred Bloggs"}) has been passed. This could also
have been set to the value 12, the string length. This argument is of
more significance when dealing with multiple images.
\end{enumnotes}

\subsubsection{Accessing header items using different data types}
All the possible types of header items that can be accessed are
described \hyperref{here}{in \S}{}{HDRCdatatypes}. Among the
possible calls are:
\begin{code}
\begin{verbatim}
   float bscale;
   hdrInF( "in", " ", "bscale", 1, &bscale, istat );
   int ibfact;
   hdrInI( "in", " ", "binned", 1, &ibfact, istat );
   #define TRUE 1
   hdrOutL( "out", " ", "checked",  "Data checked for C/Rs", TRUE, istat );
\end{verbatim}
\end{code}

\subsubsection{Accessing header items by index}
This example
(\htmladdnormallink{\myverb{hdrlist.c}}{../../bin/examples/img/hdrlist.c})
shows how to access header items by index.
\begin{code}
\begin{verbatim}
  /*  Local Variables: */
  char value[80], item[30];
  int nitem, i;

  /*  See how many items are present (this also accesses the image). */
  hdrNumb( "in", " ", "*", &nitem, istat );
  for( i=1; i<=nitem; i++ ) {

    /*  Get the name of the I'th header item. */
    hdrName( "in", source, i, item, 30, istat );                     [1]

    /*  Get its value and print out the item name and value. */
    hdrInC( "in", source, item, 1, value, 80, istat );               [1]
    printf( "%s = %s \n", item, value );
  }
\end{verbatim}
\end{code}
\mynote
\begin{enumnotes}
% 1
\item Note the extra arguments (\texttt{30} \& \texttt{80}) that pass the
      maximum length of the output strings.
\end{enumnotes}

\subsubsection{Reading and writing header items from/to many images}
It is possible to read and write header items from/to many images in
one call and some examples follow. Note that arrays of strings can
only be passed using fixed string length. It is not possible to use
arrays of pointers to \texttt{char} forming a ragged array. When
using a character array you should pass a pointer to the first element
not the actual array (as the HDR routines expect to see \texttt{char *}).

This example shows how to read the same header item from two images
at the same time.
\begin{code}
\begin{verbatim}
   char RA[2][80];
   char DEC[2][80];
   hdrIn( "IN1,IN2", " ", "RA", 1, (char *) RA, 80, status );         [1]
   hdrIn( "IN1,IN2", " ", "DEC", 1, (char *) DEC, 80, status );       [1]
   printf( "The RA and DEC of IN1 are: %s, %s\n", RA[0], DEC[0] );
   printf( "and the RA and DEC of IN2 are: %s, %s\n", RA[1], DEC[1] );
   imgFree( "*", status );
\end{verbatim}
\end{code}
\mynote
\begin{enumnotes}
% 1
\item The address to the first element of the \texttt{RA} and \texttt{DEC}
arrays is passed. This could have been written as
\texttt{\&RA[0][0]} and \texttt{\&DEC[0][0]}.
\end{enumnotes}

This example shows how to write values to the same header item of two images.
\begin{code}
\begin{verbatim}
   char obstype[2][5] = {"BIAS","FLAT"};
   hdrOut( "BIAS,FLAT", " ", "OBSTYPE", "Type of observation",
           (char *) obstype, 5, istat );
   imgFree( "*", istat );
\end{verbatim}
\end{code}

\subsection{Compiling and linking C programs}
Compiling and linking a C program is very similar to the methods
described
(\hyperref{in the Fortran section}{in \S}{}{IMGCOMPILING}), except
that you must compile the source code before the linking stage.

If your program source file is named \myverb{prog.c}, and it only
depends on IMG, then the commands:
\begin{code}
\begin{verbatim}
% cc -c -I/star/include prog.c
% alink prog.o -L/star/lib `img_link_adam`
\end{verbatim}
\end{code}
will compile the source and link the program.

\section{Changes in release (1.1)}

This version introduces and documents the C interface.

\section{Changes in release (1.2)}

In this release several bugs in the Linux version of IMG have been
corrected. These affected the release of headers when using the
\htmlref{\myverb{IMG\_CANCL}}{IMG\_CANCL} routine and counting of
header items by \htmlref{\myverb{HDR\_NUMB}}{HDR\_NUMB}. A new routine
has also been added -- \htmlref{\myverb{IMG\_NAME}}{IMG\_NAME}. This
returns the name of the image data file for use when producing logfile
information etc.

\section{Changes in release (1.3)}
The documentation has been updated to reinforce the fact that NDF
identifiers obtained using \htmlref{\myverb{IMG\_INDF}}{IMG\_INDF} are not
released by any IMG calls (you must do this for yourself using
\myverb{NDF\_ANNUL}).

A new routine for copying sources of header information
(\htmlref{\myverb{HDR\_COPY}}{HDR\_COPY}) has been added.

A bug in the FITS header writer that added NULL characters after the
END card has been corrected. This should have only affected output
FITS headers that contained some initially empty cards.

\section{Changes in this release (1.3-1)}

A bug that returned all images as the same size, even from different
calls to \htmlref{\myverb{IMG\_IN}}{IMG_IN_n__x_}, has been fixed.

FITS character strings may now be free-format (i.e.\
the terminating quote may be before the 20th column of a header card).

\end{document}
% $Id$
