\documentclass[twoside,11pt,nolof]{starlink}

% ? Specify used packages
% ? End of specify used packages

% -----------------------------------------------------------------------------
% ? Document identification
% Fixed part
\stardoccategory    {Starlink User Note}
\stardocinitials    {SUN}
\stardocsource      {sun\stardocnumber}
\stardoccopyright
{Copyright
\copyright\ 2000 Council for the Central Laboratory of the Research Councils}

% Variable part - replace [xxx] as appropriate.
\stardocnumber      {241.0}
\stardocauthors   {B D Kelly (ROE)\\
                                A J Chipperfield (RAL)}
\stardocdate        {16 August 2001}
\stardoctitle       {AMS\\The Unix ADAM Message System}
\stardocversion     {2.0}
\stardocmanual      {Programmer's Manual}
\stardocabstract  {The ADAM Message System (AMS) library, which
implements the ADAM inter-task communications protocol under Unix, is
described, along with its Fortran-callable interface (FAMS).

The description of AMS is distinguished from the current implementation which
uses the Message System Primitives (MSP).}
% ? End of document identification
% -----------------------------------------------------------------------------
% ? Document specific \providecommand or \newenvironment commands.
% ? End of document specific commands
% -----------------------------------------------------------------------------
%  Title Page.
%  ===========
\begin{document}
\scfrontmatter

% ? Main text
\section{Introduction}
The AMS library enables communications to be opened between ADAM tasks, and
commands and acknowledgements to be sent and received. Provided suitable ADAM
network (ADAMNET) processes are loaded, communications can also take place
across networks.
In the absence of ADAMNET processes, communication is restricted to ADAM
tasks on a single machine.

The
\htmlref{example}{example} (see Section \ref{example})
shows that communicating programs do not have to be strictly ADAM tasks but
the term `task' is used throughout this document to mean either end of a
communications link.
More details on the way ADAM uses AMS can be found in
\xref{SSN/77}{ssn77}{}

AMS is currently implemented using the Message System Primitives (MSP) and the
ADAM Timer (ATIMER) libraries.
AMS, MSP and ATIMER are all written in C and included in the Parameter and
Communication Subsystems
\xref{PCS}{ssn29}{}
Starlink Software Item. Fortran interfaces (FAMS and FATIMER) are also provided
for AMS and ATIMER respectively.

This document describes AMS and its current implementation, keeping the two
separate as far as possible in order to clarify the distinction whilst giving
readers  a feel for the way the whole system currently works.

\section{Transactions}
AMS communications are carried out as a series of `transactions'. A
transaction consists of an initial message and a number of further messages
(replies) in either direction, associated with the initial message.
Separate transactions are used to set up a communications path and to carry
out the business of obeying a command.
A command transaction is started by calling
\texttt{ams\_send}
and is terminated as described under
`\htmlref{Getting Expected Replies}
{getting_expected_replies}'\latex{ (see Section \ref{getting_expected_replies})}.

When sending a message, the user specifies a \texttt{message\_function} and a
\texttt{message\_status} which are passed as arguments to functions
\texttt{ams\_send}
or
\texttt{ams\_reply}.

The value of \texttt{message\_function} may be:
\begin{description}
\item[MESSYS\_\_INIT] Used to ask for a communications link to another task.
(\texttt{ams\_send} only)\footnote{This function of \texttt{ams\_send} is
rendered obsolete by \texttt{ams\_path}.}.
\item[MESSYS\_\_DE\_INIT] Used to close a communications link to another task.
\item[MESSYS\_\_MESSAGE] Used for all other purposes and qualified by the
\texttt{message\_status} and possibly other arguments.
\end{description}

\section{Task Initialisation}
A task initialises AMS by calling
\htmlref{\texttt{ams\_init()}}{AMS_INIT}
specifying the name by which the task is to be known to the
message system. The name is registered and an exit handler set up.
Sometimes it is not desirable to set up an exit handler -- in that case
\htmlref{\texttt{ams\_initeh()}}{AMS_INITEH}
should be used.

A controlling task will need to know the name by which the subsidiary task is
known to the message system.
(For ADAM tasks this is done by having the user interface set
environment variable \texttt{ICL\_TASK\_NAME} to the required name.
This also serves to tell the task that it is indeed being run via the ADAM
message system and not directly from the Unix shell.)

\section{Task Exit}
When a task receives a signal causing it to exit, the message system exit
handler is invoked (assuming it has been set up).
The result is that messages are sent to any other tasks which have been in
communication with the exiting task, informing them of its removal, and the
task name is de-registered from the message system.

\section{Opening Communications}
Once a task has initialised itself successfully, it can open communications
with another initialised task by using
\htmlref{\texttt{ams\_path()}}{AMS_PATH},
specifying the name by which the other task is registered.
The path number returned can then be used for further communications with the
other task.

A short transaction consisting of a connection request message and an
acknowledgement from the slave task is carried out.
If the slave task is
\htmlref{waiting to receive a command}
{receiving_a_command}\latex{ (see Section \ref{receiving_a_command})},
the acknowledgement is sent automatically from within
\htmlref{\texttt{ams\_receive()}}{AMS_RECEIVE}.
The acknowledgement is expected within \texttt{MESSYS\_\_INIT\_WAIT\_TIME}
(currently 30000) milliseconds.

\section{\label{sending_a_command}Sending a Command}
Having obtained a path,
\htmlref{\texttt{ams\_send()}}{AMS_SEND}
can be used to send a command message to the task identified by that path --
the \texttt{message\_function} argument is set to \texttt{MESSYS\_\_MESSAGE}.
If \texttt{ams\_send()} succeeds, it starts a new transaction and returns
a transaction id (\texttt{messid}) which remains valid for the duration of the
transaction. The task which sends the message becomes the `master' and
the one which receives it becomes the `slave' for the transaction.

Command messages are designed for use with ADAM tasks but this does not
preclude their use for other purposes (see the
\htmlref{example}{example}\latex{ in Section~\ref{example}}).
The other arguments to \texttt{ams\_send} are packed into the message and are
unpacked by AMS at the other end.
What the slave task does with them is up to it -- AMS generally has no
interest in the other arguments of \texttt{ams\_send()}.

\emph{There is one exception to this -- if the \texttt{message\_context} is
\texttt{OBEY}, the receiving end will allow the transaction to include further
messages from the master task  as part of the same transaction.
This enables the slave to reply with requests (prompts) for parameter values
and receive replies from the master.}

ADAM use of the other arguments is as follows:
\begin{description}
\item[\texttt{message\_status}] Should be SAI\_\_OK.
\item[\texttt{message\_context}] One of the following constants, defined in
\texttt{adamdefns.h}:
\begin{description}
\item[\texttt{GET}] To request a parameter value.
\item[\texttt{SET}] To set a parameter value.
\item[\texttt{OBEY}] To obey an action in the task.
\item[\texttt{CANCEL}] To cancel an action in the task.
\item[\texttt{CONTROL}] To control/enquire the task environment.
\end{description}
\item[\texttt{message\_name}] Name of task action or parameter.
\item[\texttt{message\_length}] Length of \texttt{message\_value}.
\item[\texttt{message\_value}] Context-dependent values.
\end{description}

\section{\label{getting_expected_replies}Getting Expected Replies}
Once a command is in progress between two tasks, they can receive replies
specific to that command (as identified by \texttt{path,messid}) by using
\htmlref{\texttt{ams\_getreply()}}{AMS_GETREPLY}.

\htmlref{\texttt{ams\_getreply()}}{AMS_GETREPLY}
ignores all messages not associated with the specified
(\texttt{path,messid}) except those generated by the
\htmlref{\texttt{ams\_extint()}}{AMS_EXTINT}
function.

For ADAM tasks, expected values of \texttt{message\_status} are:
\begin{small}
\begin{description}
\item[MESSYS\_\_PARAMREQ] Request a parameter value.
\item[MESSYS\_\_PARAMREP] Reply to a PARAMREQ.
\item[MESSYS\_\_INFORM] Message to be displayed
\item[MESSYS\_\_SYNC] Synchronisation request
\item[MESSYS\_\_SYNCREP] Synchronisation reply
\item[MESSYS\_\_TRIGGER] Trigger an action in the master task
\item[DTASK\_\_ACTSTART]\footnote{This reference to DTASK, the main part of the
ADAM task fixed part, is for historical reasons.} Acknowledge a GSOC command.
\end{description}
\end{small}

Completion of the command transaction and freeing of the associated
\texttt{messid} occurs automatically if the task returns a
\texttt{message\_status} which is not one of the above.  Apart from that, the
other arguments returned by \texttt{ams\_getreply} are of no significance to
AMS.

\section{\label{receiving_a_command}Receiving a Command}
Any message, including commands or replies associated with an existing command,
can be received using
\htmlref{\texttt{ams\_receive()}}{AMS_RECEIVE}.
When a new command message is received a new transaction (\texttt{messid})
will be set up in the receiving task.
The \texttt{path} and \texttt{messid} associated with the command/reply are
returned.

See
\htmlref{`Sending a Command'}{sending_a_command}\latex{ (Section
\ref{sending_a_command})} for a discussion of the significance of the other
arguments of
\texttt{ams\_receive}.

\section{\label{sending_expected_replies}Sending Expected Replies}
A reply associated with an existing command has to be sent using
\htmlref{\texttt{ams\_reply()}}{AMS_REPLY},
specifying the relevant (\texttt{path,messid}).
The command transaction is terminated automatically if a reply is sent with a
\texttt{message\_status} other than those listed in
\htmlref{`Getting Expected Replies'}{getting_expected_replies}\latex{ (Section
\ref{getting_expected_replies})}

See also
\htmlref{`Sending a Command'}{sending_a_command}\latex{ (Section
\ref{sending_a_command})} for a discussion of the significance of the other
arguments of
\texttt{ams\_reply}.

\section{Sending Internal Messages}
A set of routines are provided to enable a task to send messages to itself.
\htmlref{\texttt{ams\_kick()}}{AMS_KICK}
is intended to enable main-line code to queue a message which it can
subsequently collect by using
\htmlref{\texttt{ams\_receive()}}{AMS_RECEIVE}.

The other functions are intended only for calling from within signal handlers.
\htmlref{\texttt{ams\_resmsg()}}{AMS_RESMSG},
\htmlref{\texttt{ams\_astmsg()}}{AMS_ASTMSG}
and
\htmlref{\texttt{ams\_astint()}}{AMS_ASTINT}
generate messages to be read by
\htmlref{\texttt{ams\_receive()}}{AMS_RECEIVE}
but ignored by
\htmlref{\texttt{ams\_getreply()}}{AMS_GETREPLY}.

\htmlref{\texttt{ams\_extint()}}{AMS_EXTINT}
generates a message which can be received by either
\htmlref{\texttt{ams\_receive()}}{AMS_RECEIVE}
or
\htmlref{\texttt{ams\_\-getreply()}}{AMS_GETREPLY}
(its main purpose is to allow user interfaces to be implemented).

At the receiving end, these messages result in the appropriate value of
\texttt{message\_status}, one of:
\begin{description}
\item[MESSYS\_\_KICK]
\item[MESSYS\_\_RESCHED]
\item[MESSYS\_\_ASTINT]
\item[MESSYS\_\_EXTINT]
\end{description}
\section{Implementation}

\subsection{AMS Messages}
AMS messages may be transported within other structures such as MSP messages
but the AMS message itself consists of a message type, followed by a structure
dependent upon the type. The types are defined in \texttt{ams\_sys.h} and
the associated structures in \texttt{ams\_struc.h}.
The message type name has three elements separated by `\_':
\begin{description}
\item[LOC/REM] Whether local or remote (this or other machine).
\item[type]
 \begin{description}
 \item[ACK]     Acknowledge
 \item[GSOC\_START]  Start  GET/SET/OBEY/CANCEL/CONTROL
 \item[GSOC\_END]    End GET/SET/OBEY/CANCEL/CONTROL
 \item[MSG]     Message.
 \item[DEINIT]  De-initialise
 \item[INIT]    Initialise
 \item[CALL]    Request remote connection (REM only)
 \item[ACCEPT]  Accept remote connection (REM only)
\end{description}
\item[IN/OUT] Whether message is in or out
\end{description}

\textit{E.g.} LOC\_ACK\_IN, REM\_CALL\_OUT.

Internal messages are of type LOC\_MSG\_IN/OUT

The AMS functions will automatically send the right type of message, in the
case of \texttt{ams\_send()} and \texttt{ams\_reply()} this will depend upon
their \texttt{message\_function} and \texttt{message\_status} arguments.

\subsection{MSP, Sockets and Queues}
MSP uses a single STREAM socket (path) in the INET domain to communicate
between any two tasks and a STREAM socket pair in the UNIX domain to
send/receive local (internal) messages. Messages on the sockets specify a
`queue' in the receiving task for which they are intended and a queue
in the sending task to which replies should be sent.

The MSP message reading function accepts a list of `receive' queues on which it
should look for messages and the MSP message sending function accepts a `send'
queue which specifies a socket and a receive queue in the other task.
N.B. the reply queue may be given as \texttt{MSP\_\_NULL\_SENDQ} (it is a send
queue in the task which has to reply).

AMS uses separate MSP queues within each task for:
\begin{itemize}
\item Receiving unexpected messages from other tasks (the command queue).
\item Sending unsolicited messages to other tasks (other task's command queue).
\item Receiving replies to a sent message which initiates a transaction -- each
transaction has its own queue (reply queue).
\item Sending replies as part of a transaction (other task's reply queue)
\item Sending local (internal) messages (one for each internal message type).
\begin{description}
\item[sigast\_q]
\item[sigext\_q]
\item[sigkick\_q]
\item[sigresch\_q]
\item[sigtimeout\_q]
\end{description}
\item Receiving local (internal) messages (one for each internal message type).
\begin{description}
\item[astint\_q]
\item[extint\_q]
\item[kick\_q]
\item[resched\_q]
\item[timeout\_q]
\end{description}
\end{itemize}
The AMS user will not need to know about MSP queues.

\subsection{Communications Directory}
MSP communications are opened between tasks by name.
The name lookup is provided by
each task making an entry in a directory pointed to by the environment
variable \texttt{ADAM\_USER}. If \texttt{ADAM\_USER} is not defined when a task
attempts to register itself with the message system, directory \texttt{\~{}/adam}
will be used and, if the specified directory does not exist, an attempt will
be made to create it.

This results in a file being created in the \texttt{ADAM\_USER} directory
with a name compounded of the task name and an identifying number
(\textit{e.g} \texttt{\$ADAM\_USER/slave\_5001}).
Another task can then open communications by searching \texttt{ADAM\_USER} for
the right name and using the identifying number.

When a task exits, the AMS exit handler de-registers the task from MSP and
the task's file is removed from \texttt{ADAM\_USER}.
If the task does not exit normally for some reason, the file in
\texttt{ADAM\_USER} may get left behind.
In this case the file must be deleted explicitly; otherwise the
task will refuse to load next time.

\subsection{MSP Messages}
MSP messages are passed between processes and also added to a queue at the
receiving end. Some of the structure components are only relevant when it is
being passed, or when it is on a queue.

A received MSP message contains the following information:
\begin{itemize}
\item Target queue in this task
\item Reply queue in other task
\item Actual size of message body
\item Message body
\item Pointer to next message in the queue
\end{itemize}
The content of the message body is only understood at higher levels (AMS in
this case) -- it is immaterial to MSP.

\newpage
\appendix
\section{\label{example}Example}
The example consists of a pair of C programs called \texttt{master} and
\texttt{slave}.
They should be run in the background by:
\begin{quote}
\begin{terminalv}
% slave &
% master &
\end{terminalv}
\end{quote}

The code for \texttt{master} is:

\begin{small}
\begin{quote}
\begin{terminalv}
/* amsmaster
 * A test of ams - run in conjunction with amsslave
 *   % amsslave &
 *   % amsmaster
*/
#include <stdio.h>
#include <sys/types.h>
#include <sys/time.h>
#include <string.h>
#include <signal.h>
#include <errno.h>
#include <unistd.h>

#include "sae_par.h"
#include "adam_defns.h"
#include "messys_len.h"
#include "messys_par.h"

#include "ams.h"

int main()
{
   int outmsg_status;
   int outmsg_function;
   int outmsg_context;
   int outmsg_length;
   char outmsg_name[32];
   char outmsg_value[MSG_VAL_LEN];
   int inmsg_status;
   int inmsg_context;
   int inmsg_length;
   char inmsg_name[32];
   char inmsg_value[MSG_VAL_LEN];
   int status;
   int path;
   int messid;
   int j;

   status = 0;

/* Set up components of a GSOC OBEY message. The slave does not care about
 * the name component of the message */
   outmsg_status = SAI__OK;
   outmsg_function = MESSYS__MESSAGE;
   outmsg_context = OBEY;
   outmsg_length = 16;

   strcpy ( outmsg_name, "junk" );
   strcpy ( outmsg_value, "master calling" );

/* Register as "master" with the message system */
   ams_init ( "master", &status );
   if ( status != SAI__OK )
   {
      printf ( "master - bad status after ams_init\n" );
   }

/* Get a path to "slave" and report */
   ams_path ( "slave", &path, &status );
   if ( status != SAI__OK )
   {
      printf ( "master - bad status after ams_path\n" );
   }
   else
   {
      printf ( "master - path set up ok\n" );
   }

/* Perform 1000 identical transactions - send a GSOC obey message and
 * await an initial acknowledgement (message_status = DTASK__ACTSTART)
 * and a completion message (message_status = SAI__OK) */
   for ( j=0; j<1000; j++ )
   {
/* Send the OBEY command */
      ams_send ( path, outmsg_function, outmsg_status, outmsg_context,
        outmsg_name, outmsg_length, outmsg_value, &messid, &status );
/* Get the acknowledement reply - content not checked */
      ams_getreply ( MESSYS__INFINITE, path, messid, 32, MSG_VAL_LEN,
        &inmsg_status, &inmsg_context, inmsg_name, &inmsg_length,
        inmsg_value, &status );
/* Get the completion reply - content not checked.
 * AMS will terminate the transaction if it is the expected message status
 * SAI__OK */
      ams_getreply ( MESSYS__INFINITE, path, messid, 32, MSG_VAL_LEN,
        &inmsg_status, &inmsg_context, inmsg_name, &inmsg_length,
        inmsg_value, &status );
   }

/* If all OK, display the last received message value;
 * otherwise display the error status */
   if ( status != 0 )
   {
      printf ( "master: bad status = %d\n", status );
   }
   else
   {
      printf ( "master: received - %s\n", inmsg_value );
   }

   return 0;
}
\end{terminalv}
\end{quote}
\end{small}

The code for \texttt{slave} is:
\begin{small}
\begin{quote}
\begin{terminalv}
/* amsslave
 * A test of ams - run in conjunction with amsmaster
 *   % amsslave &
 *   % amsmaster
*/
#include <stdio.h>
#include <time.h>
#include <sys/types.h>
#include <sys/time.h>
#include <string.h>
#include <signal.h>
#include <errno.h>
#include <unistd.h>

#include "sae_par.h"
#include "adam_defns.h"
#include "dtask_err.h"             /* dtask error codes */

#include "messys_len.h"
#include "messys_par.h"

#include "ams.h"

int main()
{
   int outmsg_status;
   int outmsg_function;
   int outmsg_context;
   int outmsg_length;
   char outmsg_name[32];
   char outmsg_value[MSG_VAL_LEN];
   int inmsg_status;
   int inmsg_context;
   int inmsg_length;
   char inmsg_name[32];
   char inmsg_value[MSG_VAL_LEN];

   int status;
   int path;
   int messid;
   int j;

/* Set up components of a reply to a GSOC OBEY message. The master does not
 * care about the name component of the message */
   status = 0;
   outmsg_status = SAI__OK;
   outmsg_function = MESSYS__MESSAGE;
   outmsg_context = OBEY;
   outmsg_length = 16;

   strcpy ( outmsg_name, "junk" );
   strcpy ( outmsg_value, "slave replying" );

/* Register as "slave" with the message system */
   ams_init ( "slave", &status );
   if ( status != 0 )
   {
      printf ( "slave: failed init\n" );
   }

/* Receive 1000 commands, sending an initial acknowledgement and a completion
 * message in reply to each */
   for ( j=0; j<1000; j++ )
   {
/* Await a command message */
      ams_receive ( MESSYS__INFINITE, 32, MSG_VAL_LEN, &inmsg_status,
        &inmsg_context, inmsg_name, &inmsg_length, inmsg_value, &path,
        &messid, &status );

/* Send an initial acknowledgement (message_status = DTASK_ACTSTART). */
      outmsg_status = DTASK__ACTSTART;
      ams_reply ( path, messid, outmsg_function, outmsg_status,
        outmsg_context, outmsg_name, outmsg_length, outmsg_value,
        &status );

/* Send a completion message (message_status = SAI__OK) - this will terminate
   the transaction at both ends */
      outmsg_status = SAI__OK;
      ams_reply ( path, messid, outmsg_function, outmsg_status,
        outmsg_context, outmsg_name, outmsg_length, outmsg_value,
        &status );

/* If there was a failure, exit the loop */
      if ( status != SAI__OK ) break;

   }

/* If all OK, display the last received message value;
 * otherwise display the error status */
   if ( status != 0 )
   {
      printf ( "slave: bad status = %d\n", status );
   }
   else
   {
      printf ( "slave: received - %s\n", inmsg_value );
   }

   return 0;
}
\end{terminalv}
\end{quote}
\end{small}
\newpage
\section{Function Descriptions}
\begin{small}
\sstroutine{
   AMS\_ASTINT
}{
   Send an ASTINT message to this task
}{

   \sstinvocation{
      (void) ams\_astint( *status )
   }

   \sstarguments{
      \sstsubsection{
         status = int * (given and returned)
      }{
         global status
      }
   }
   \sstdescription{
      Causes the task to send an ASTINT message to itself. This should be
      called from a signal handler
   }

   \sstimplementation{
      Send an OBEY message to the astint queue.
   }
}

\sstroutine{
   AMS\_ASTMSG
}{
   Send an ASTMSG to this task
}{

   \sstinvocation{
      (void) ams\_astmsg( name, length, value, status )
   }

   \sstarguments{
      \sstsubsection{
         name = char * (given)
      }{
         name of the action to be rescheduled
      }
      \sstsubsection{
         length = int (given)
      }{
         number of significant bytes in value
      }
      \sstsubsection{
         value = char * (given)
      }{
         message to be passed to main-line code
      }
      \sstsubsection{
         status = int * (given and returned)
      }{
         global status
      }
   }

   \sstdescription{
      Causes the task to send specified ASTINT message 'value', qualified by
      'name' to itself. This should be called from a signal handler.
   }

   \sstimplementation{
      Send an OBEY message to the astint queue.
   }
}

\sstroutine{
   AMS\_EXTINT
}{
   Send an EXTINT message to this task
}{

   \sstinvocation{
      (void)ams\_extint( status )
   }

   \sstarguments{
      \sstsubsection{
         status = int * (given and returned)
      }{
         global status
      }

   }

   \sstdescription{
      Causes the task to send an EXTINT message to itself. This should be
      called from a signal handler when an external interrupt has occurred.

      N.B. This is not be used within a normal ADAM task - it is intended
      for use in user interfaces.
   }

   \sstimplementation{
      Send an OBEY message to the extint queue.
   }

}


\sstroutine{
   AMS\_GETREPLY
}{
   Receive a message on a specified path, messid
}{

   \sstinvocation{
      (void)ams\_getreply( timeout, path, messid, message\_name\_s,\\
                                message\_value\_s, message\_status,
                                message\_context, message\_name,\\
                                message\_length, message\_value, status )
   }

   \sstarguments{
      \sstsubsection{
         timeout = int (given)
      }{
         timeout time in milliseconds
      }
      \sstsubsection{
         path = int (given)
      }{
         pointer to the path
      }

      \sstsubsection{
         messid = int (given)
      }{
         message number of incoming message
      }
      \sstsubsection{
         message\_name\_s = int (given)
      }{
         maximum space for name (bytes)
      }

      \sstsubsection{
         message\_value\_s = int (given)
      }{
         maximum space for value (bytes)
      }
      \sstsubsection{
         message\_status = int * (returned)
      }{
         message status
      }
      \sstsubsection{
         message\_context = int * (returned)
      }{
         message context
      }
      \sstsubsection{
         message\_name = char * (returned)
      }{
         message name
      }

      \sstsubsection{
         message\_length = int * (returned)
      }{
         length of value
      }

      \sstsubsection{
         message\_value = char * (returned)
      }{
         message value
      }

      \sstsubsection{
         status = int * (given and returned)
      }{
         global status
      }

   }

   \sstdescription{
      The application has sent a message on path 'path' as part of
      transaction 'messid' and wishes to obtain the reply within 'timeout'
      milliseconds. A timeout value of MESSYS\_\_INFINITE indicates no time
      limit.

      Any received message is unpacked appropriately and the contents
      returned to the calling routine. Only those arguments relevant to the
      particular message type will be returned.

      Note that the received message may be a TIMEOUT message as a result of
      the timer being set.
   }

   \sstimplementation{
      The function first checks the transaction (messid) is legally identified
      and that there exists an acknowledge queue for that transaction.

      If 'timeout' is not MESSYS\_\_INFINITE, it then sets the timer clock
      going so that we get a timeout if there is no response within timeout
      milliseconds. The ATIMER package is used to handle timers.

      The function then looks for a message on either the
      external interrupt queue, the transaction acknowledge queue or the
      timeout queue.
   }
}


\sstroutine{
   AMS\_INIT
}{
   Initialise ams and register an exit handler
}{

   \sstinvocation{
      (void)ams\_init( own\_name, status )
   }

   \sstarguments{
      \sstsubsection{
         own\_name = char * (given)
      }{
         name of this task
      }

      \sstsubsection{
         status = int * (given and returned)
      }{
         global status
      }
   }
   \sstdescription{
      Initialise ams and register an exit handler
   }

   \sstimplementation{
      Call ams\_initeh, requesting that an exit handler be set up.
   }
}

\sstroutine{
   AMS\_INITEH
}{
   Initialise ams and optionally register an exit handler
}{

   \sstinvocation{
      (void)ams\_initeh( own\_name, eh, status )
   }
   \sstarguments{
      \sstsubsection{
         own\_name = char * (given)
      }{
         name of this task
      }

      \sstsubsection{
         eh = int (given)
      }{
         whether to register exit handler
      }

      \sstsubsection{
         status = int * (given and returned)
      }{
         global status
      }
   }

   \sstdescription{
      Initialise AMS optionally registering an exit handler
   }

   \sstimplementation{
      Initialise the internal data structures.

      Register with msp, obtain the command queue for incoming messages,
      then create the queues used for this task sending messages to
      itself.

      Finally set up the signal handler if so requested.
   }
}
\sstroutine{
   AMS\_KICK
}{
   Send a KICK message to this task.
}{
   \sstinvocation{
      (void)ams\_kick( name, length, value, status )
   }
   \sstarguments{
      \sstsubsection{
         name = char * (given)
      }{
         name of the action to be rescheduled
      }

      \sstsubsection{
         length = int (given)
      }{
         number of significant bytes in value
      }

      \sstsubsection{
         value = char * (given)
      }{
         message to be passed to application code
      }

      \sstsubsection{
         status = int * (given and returned)
      }{
         global status
      }

   }
   \sstdescription{
      Sends a KICK message to this task, specifying an action to be obeyed
      and a message (command line) to be passed to the application code.
   }

   \sstimplementation{
      Send a soft kick interrupt message 'value' qualified by 'name' to
      kick queue.
   }
}

\sstroutine{
   AMS\_PATH
}{
   Get a communications path to another task
}{

   \sstinvocation{
      (void)ams\_path( other\_task\_name, path, status )
   }

   \sstarguments{
      \sstsubsection{
         other\_task\_name = char * (given)
      }{
         name of task to which path is required
      }

      \sstsubsection{
         path = int * (returned)
      }{
         the path number
      }

      \sstsubsection{
         status = int * (given and returned)
      }{
         global status
      }

   }

   \sstdescription{
      Open a path to the task whose name is 'other\_task\_name' and return
      the path index in 'path'. The other task may be local or remote,
      indicated by a name of the form machine::name, where :: may be any of
      the permitted separator pairs and defines the ADAMNET process to be
      used.
   }

   \sstimplementation{
      A temporary transaction acknowledge queue is obtained, a MESSYS\_\_INIT
      message sent via the path just obtained, and the reply obtained.

      If this short transaction fails to complete, the path and any
      associated transactions are freed; otherwise the path (index) is
      returned.
   }
}

\sstroutine{
   AMS\_PLOOKUP
}{
   Look up a taskname given a path to it
}{

   \sstinvocation{
      (void)ams\_plookup( path, name, status )
   }


   \sstarguments{
      \sstsubsection{
         path = int (given)
      }{
         the path number
      }

      \sstsubsection{
         name = char * (returned)
      }{
         the task name
      }

      \sstsubsection{
         status = int * (given and returned)
      }{
         global status
      }

   }

   \sstdescription{
      Returns the name of the task connected via the given path.

      If the task is remote, a name of the form xxxxx::yyyyyy is returned,
      where xxxxxx:: is the machine and ADAMNET indicator, and yyyyyy is the
      task name.
   }

   \sstimplementation{
      Given a path 'path', we check that the path is legal and then use
      it to ascertain whether the path is linked to a remote task or a local
      task. A name of the appropriate form is returned.
   }
}

\sstroutine{
   AMS\_RECEIVE
}{
   Receive any incoming message
}{

   \sstinvocation{
      (void)ams\_receive( timeout, message\_name\_s, message\_value\_s,\\
                             message\_status, message\_context, message\_name,
                             message\_length,\\
                             message\_value, path, messid, status )
   }


   \sstarguments{
      \sstsubsection{
         timeout = int (given)
      }{
         timeout time in milliseconds
      }

      \sstsubsection{
         message\_name\_s = int (given)
      }{
         maximum space for name (bytes)
      }

      \sstsubsection{
         message\_value\_s = int (given)
      }{
         maximum space for value (bytes)
      }

      \sstsubsection{
         message\_status = int * (returned)
      }{
         message status
      }

      \sstsubsection{
         message\_context = int * (returned)
      }{
         message\_context
      }

      \sstsubsection{
         message\_name = char * (returned)
      }{
         message name
      }

      \sstsubsection{
         message\_length = int * (returned)
      }{
         length of value
      }

      \sstsubsection{
         message\_value = char * (returned)
      }{
         message value
      }

      \sstsubsection{
         path = int * (returned)
      }{
         path on which message received
      }

      \sstsubsection{
         messid = int * (returned)
      }{
         message number of incoming message
      }

      \sstsubsection{
         status = int * (given and returned)
      }{
         global status
      }

   }

   \sstdescription{
      Looks for a message to this task from any source for 'timeout'
      milliseconds. A timeout value of MESSYS\_\_INFINITE indicates no time
      limit.

      Any received message is unpacked appropriately and the contents
      returned to the calling routine.  Only those arguments relevant to the
      particular message type will be returned.

      Note that the received message may be a TIMEOUT message as a result of
      the timer being set.
   }

   \sstimplementation{

      If 'timeout' is not MESSYS\_\_INFINITE, the function sets the timer clock
      going so that we get a timeout if there is no response within 'timeout'
      milliseconds. The ATIMER package is used to handle timers.

      The function then looks for a message on any of this task's receive
      queues and returns the message components.
   }
}

\sstroutine{
   AMS\_REPLY
}{
   Send a message on a given (path,messid)
}{

   \sstinvocation{
      (void)ams\_reply( path, messid, message\_function, message\_status,\\
                             message\_context, message\_name, message\_length,
                             message\_value, status )
   }

   \sstarguments{
      \sstsubsection{
         path = int (given)
      }{
         the path number for communicating with the other task
      }

      \sstsubsection{
         messid = int (given)
      }{
         the number identifying the transaction
      }

      \sstsubsection{
         message\_function = int (given)
      }{
         message function
      }

      \sstsubsection{
         message\_status = int (given)
      }{
         message status
      }

      \sstsubsection{
         message\_context = int (given)
      }{
         message context
      }

      \sstsubsection{
         message\_name = char * (given)
      }{
         message name
      }

      \sstsubsection{
         message\_length = int (given)
      }{
         length of value
      }

      \sstsubsection{
         message\_value = char * (given)
      }{
         message value
      }

      \sstsubsection{
         status = int * (given and returned)
      }{
         global status
      }

   }

   \sstdescription{

      As part of transaction 'messid' on path 'path', the user wishes to
      send the message AS A REPLY to a previously received message from
      the other end of the path. The user can ONLY reply with a
      MESSYS\_\_DE\_INIT message (something has gone wrong) or with a normal
      MESSY\_\_MESSAGE message. Any other value of message\_function will
      result in status MESSYS\_\_MSGFUNC being returned.

      A MESSYS\_\_MESSAGE message may be a GSOC\_END, terminating a transaction
      or one of the various other types, MESSYS\_\_INFORM, MESSYS\_\_PARAMREQ
      etc.
   }

   \sstimplementation{
      We first check that the path is open and that the
      transaction is in use. If both these are OK we
      check the 'function' part of the external form and check exactly
      what is being sent.  If the process ends up sending a GSOC\_END message
      the transaction is also closed at this end.
   }

}

\sstroutine{
   AMS\_RESMSG
}{
   Send a RESCHEDULE message to this task
}{

   \sstinvocation{
     (void)ams\_resmsg( length, value, status )
   }
   \sstarguments{
      \sstsubsection{
         length = int (given)
      }{
         number of significant bytes in value
      }

      \sstsubsection{
         value = char * (given)
      }{
         message to be passed to main-line code
      }

      \sstsubsection{
         status = int * (given and returned)
      }{
         global status
      }

   }


   \sstdescription{
      Causes the task to send specified RESCHED message 'value', qualified by
      'name' to itself.
   }

   \sstimplementation{
      Send an OBEY message to the reschedule queue.
   }
}

\sstroutine{
   AMS\_SEND
}{
   Send a message on a given path
}{

   \sstinvocation{
      (void)ams\_send( path, message\_function, message\_status,\\
                       message\_context, message\_name, message\_length,
                       message\_value, messid,\\
                       status )}
   \sstarguments{
      \sstsubsection{
         path = int (given)
      }{
         pointer to the path
      }

      \sstsubsection{
         message\_function = int (given)
      }{
         message function
      }

      \sstsubsection{
         message\_status = int (given)
      }{
         message status
      }

      \sstsubsection{
         message\_context = int (given)
      }{
         message context
      }

      \sstsubsection{
         message\_name = char * (given)
      }{
         message name
      }

      \sstsubsection{
         message\_length = int (given)
      }{
         length of value
      }

      \sstsubsection{
         message\_value = char * (given)
      }{
         message value
      }

      \sstsubsection{
         messid = int * (returned)
      }{
         message number issued by this task (returned)
      }

      \sstsubsection{
         status = int * (given and returned)
      }{
         global status
      }

   }

   \sstdescription{
      This function is used by the application code to send a message
      to another task on path 'path'.

      The expectation is that this is one of a DEINIT message, an INIT
      message or the first message of a NEW transaction whose transaction
      number is to be set into '*messid'.
   }

   \sstimplementation{
      When used for the first message of a transaction, ams\_getfreetrans()
      is used to obtain a free transaction and the MESSYS\_\_MESSAGE
      is sent to the other task's command queue (using ams\_sendgsocstart()).

      When used to send an INIT message, ams\_getfreetrans()
      is used to obtain a free temporary transaction and the MESSYS\_\_INIT
      is sent to the other task's command queue (using ams\_sendinit()).

      When used to send a DEINIT, ams\_senddeinit() is used to
      send the MESSYS\_\_DE\_INIT message to the other task's command queue.
   }
}
\end{small}

% ? End of main text
\end{document}
