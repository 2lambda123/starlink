\documentclass[twoside,11pt]{article}

% ? Specify used packages
% \usepackage{graphicx}        %  Use this one for final production.
% \usepackage[draft]{graphicx} %  Use this one for drafting.
% ? End of specify used packages

\pagestyle{myheadings}

% -----------------------------------------------------------------------------
% ? Document identification
% Fixed part
\newcommand{\stardoccategory}  {Starlink System Note}
\newcommand{\stardocinitials}  {SSN}
\newcommand{\stardocsource}    {ssn\stardocnumber}
\newcommand{\stardoccopyright} 
{Copyright \copyright\ 2000 Council for the Central Laboratory of the Research Councils}

% Variable part - replace [xxx] as appropriate.
\newcommand{\stardocnumber}    {29.5}
\newcommand{\stardocauthors}   {A J Chipperfield}
\newcommand{\stardocdate}      {20 August 2001}
\newcommand{\stardoctitle}     {PCS\\[1ex]
                                The Parameter and Communication Subsystems}
\newcommand{\stardocabstract}  {Application programs often need to obtain
parameter values from a variety of sources and to communicate with other
programs. 
The Parameter and Communication Subsystems (PCS) are a set of closely-related 
subroutine libraries which provide these facilities for many Starlink 
applications and the associated user-interfaces.
\par
The PCS libraries will not generally be called directly by application programs,
but form a basic part of the Starlink Software Environment which is 
described in
\xref{Starlink Guide SG/4}{sg4}{}.
Additional notes on using it under Unix are given in
\xref{Starlink User Note SUN/144}{sun144}{}.}
% ? End of document identification
% -----------------------------------------------------------------------------

% +
%  Name:
%     ssn.tex
%
%  Purpose:
%     Template for Starlink System Note (SSN) documents.
%     Refer to SUN/199
%
%  Authors:
%     AJC: A.J.Chipperfield (Starlink, RAL)
%     BLY: M.J.Bly (Starlink, RAL)
%     PWD: Peter W. Draper (Starlink, Durham University)
%
%  History:
%     17-JAN-1996 (AJC):
%        Original with hypertext macros, based on MDL plain originals.
%     16-JUN-1997 (BLY):
%        Adapted for LaTeX2e.
%     13-AUG-1998 (PWD):
%        Converted for use with LaTeX2HTML version 98.2 and
%        Star2HTML version 1.3.
%      1-FEB-2000 (AJC):
%        Add Copyright statement in LaTeX
%     {Add further history here}
%
% -

\newcommand{\stardocname}{\stardocinitials /\stardocnumber}
\markboth{\stardocname}{\stardocname}
\setlength{\textwidth}{160mm}
\setlength{\textheight}{230mm}
\setlength{\topmargin}{-2mm}
\setlength{\oddsidemargin}{0mm}
\setlength{\evensidemargin}{0mm}
\setlength{\parindent}{0mm}
\setlength{\parskip}{\medskipamount}
\setlength{\unitlength}{1mm}

% -----------------------------------------------------------------------------
%  Hypertext definitions.
%  ======================
%  These are used by the LaTeX2HTML translator in conjunction with star2html.

%  Comment.sty: version 2.0, 19 June 1992
%  Selectively in/exclude pieces of text.
%
%  Author
%    Victor Eijkhout                                      <eijkhout@cs.utk.edu>
%    Department of Computer Science
%    University Tennessee at Knoxville
%    104 Ayres Hall
%    Knoxville, TN 37996
%    USA

%  Do not remove the %begin{latexonly} and %end{latexonly} lines (used by 
%  LaTeX2HTML to signify text it shouldn't process).
%begin{latexonly}
\makeatletter
\def\makeinnocent#1{\catcode`#1=12 }
\def\csarg#1#2{\expandafter#1\csname#2\endcsname}

\def\ThrowAwayComment#1{\begingroup
    \def\CurrentComment{#1}%
    \let\do\makeinnocent \dospecials
    \makeinnocent\^^L% and whatever other special cases
    \endlinechar`\^^M \catcode`\^^M=12 \xComment}
{\catcode`\^^M=12 \endlinechar=-1 %
 \gdef\xComment#1^^M{\def\test{#1}
      \csarg\ifx{PlainEnd\CurrentComment Test}\test
          \let\html@next\endgroup
      \else \csarg\ifx{LaLaEnd\CurrentComment Test}\test
            \edef\html@next{\endgroup\noexpand\end{\CurrentComment}}
      \else \let\html@next\xComment
      \fi \fi \html@next}
}
\makeatother

\def\includecomment
 #1{\expandafter\def\csname#1\endcsname{}%
    \expandafter\def\csname end#1\endcsname{}}
\def\excludecomment
 #1{\expandafter\def\csname#1\endcsname{\ThrowAwayComment{#1}}%
    {\escapechar=-1\relax
     \csarg\xdef{PlainEnd#1Test}{\string\\end#1}%
     \csarg\xdef{LaLaEnd#1Test}{\string\\end\string\{#1\string\}}%
    }}

%  Define environments that ignore their contents.
\excludecomment{comment}
\excludecomment{rawhtml}
\excludecomment{htmlonly}

%  Hypertext commands etc. This is a condensed version of the html.sty
%  file supplied with LaTeX2HTML by: Nikos Drakos <nikos@cbl.leeds.ac.uk> &
%  Jelle van Zeijl <jvzeijl@isou17.estec.esa.nl>. The LaTeX2HTML documentation
%  should be consulted about all commands (and the environments defined above)
%  except \xref and \xlabel which are Starlink specific.

\newcommand{\htmladdnormallinkfoot}[2]{#1\footnote{#2}}
\newcommand{\htmladdnormallink}[2]{#1}
\newcommand{\htmladdimg}[1]{}
\newcommand{\hyperref}[4]{#2\ref{#4}#3}
\newcommand{\htmlref}[2]{#1}
\newcommand{\htmlimage}[1]{}
\newcommand{\htmladdtonavigation}[1]{}

\newenvironment{latexonly}{}{}
\newcommand{\latex}[1]{#1}
\newcommand{\html}[1]{}
\newcommand{\latexhtml}[2]{#1}
\newcommand{\HTMLcode}[2][]{}

%  Starlink cross-references and labels.
\newcommand{\xref}[3]{#1}
\newcommand{\xlabel}[1]{}

%  LaTeX2HTML symbol.
\newcommand{\latextohtml}{\LaTeX2\texttt{HTML}}

%  Define command to re-centre underscore for Latex and leave as normal
%  for HTML (severe problems with \_ in tabbing environments and \_\_
%  generally otherwise).
\renewcommand{\_}{\texttt{\symbol{95}}}

% -----------------------------------------------------------------------------
%  Debugging.
%  =========
%  Remove % on the following to debug links in the HTML version using Latex.

% \newcommand{\hotlink}[2]{\fbox{\begin{tabular}[t]{@{}c@{}}#1\\\hline{\footnotesize #2}\end{tabular}}}
% \renewcommand{\htmladdnormallinkfoot}[2]{\hotlink{#1}{#2}}
% \renewcommand{\htmladdnormallink}[2]{\hotlink{#1}{#2}}
% \renewcommand{\hyperref}[4]{\hotlink{#1}{\S\ref{#4}}}
% \renewcommand{\htmlref}[2]{\hotlink{#1}{\S\ref{#2}}}
% \renewcommand{\xref}[3]{\hotlink{#1}{#2 -- #3}}
%end{latexonly}
% -----------------------------------------------------------------------------
% ? Document specific \newcommand or \newenvironment commands.
\newcommand{\ROEURL}{http://www.roe.ac.uk/}
\newcommand{\STARURL}{http://www.starlink.ac.uk/}
\newcommand{\RALURL}{http://www.clrc.ac.uk/} 
\newcommand{\AAOURL}{http://www.aao.gov.au/}
\newcommand{\JACHURL}{http://www.jach.hawaii.edu/}
% ? End of document specific commands
% -----------------------------------------------------------------------------
%  Title Page.
%  ===========
\renewcommand{\thepage}{\roman{page}}
\begin{document}
\thispagestyle{empty}

%  Latex document header.
%  ======================
\begin{latexonly}
   CCLRC / \textsc{Rutherford Appleton Laboratory} \hfill \textbf{\stardocname}\\
   {\large Particle Physics \& Astronomy Research Council}\\
   {\large Starlink Project\\}
   {\large \stardoccategory\ \stardocnumber}
   \begin{flushright}
   \stardocauthors\\
   \stardocdate
   \end{flushright}
   \vspace{-4mm}
   \rule{\textwidth}{0.5mm}
   \vspace{5mm}
   \begin{center}
   {\Large\textbf{\stardoctitle}}
   \end{center}
   \vspace{5mm}

% ? Heading for abstract if used.
   \vspace{10mm}
   \begin{center}
      {\Large\textbf{Abstract}}
   \end{center}
% ? End of heading for abstract.
\end{latexonly}

%  HTML documentation header.
%  ==========================
\begin{htmlonly}
   \xlabel{}
   \begin{rawhtml} <H1> \end{rawhtml}
      \stardoctitle
   \begin{rawhtml} </H1> <HR> \end{rawhtml}

   \begin{rawhtml} <P> <I> \end{rawhtml}
   \stardoccategory\ \stardocnumber \\
   \stardocauthors \\
   \stardocdate
   \begin{rawhtml} </I> </P> <H3> \end{rawhtml}
      \htmladdnormallink{CCLRC / Rutherford Appleton Laboratory}
                        {http://www.cclrc.ac.uk} \\
      \htmladdnormallink{Particle Physics \& Astronomy Research Council}
                        {http://www.pparc.ac.uk} \\
   \begin{rawhtml} </H3> <H2> \end{rawhtml}
      \htmladdnormallink{Starlink Project}{http://www.starlink.ac.uk/}
   \begin{rawhtml} </H2> \end{rawhtml}
   \htmladdnormallink{\htmladdimg{source.gif} Retrieve hardcopy}
      {http://www.starlink.ac.uk/cgi-bin/hcserver?\stardocsource}\\

%  HTML document table of contents. 
%  ================================
%  Add table of contents header and a navigation button to return to this 
%  point in the document (this should always go before the abstract \section). 
  \label{stardoccontents}
  \begin{rawhtml} 
    <HR>
    <H2>Contents</H2>
  \end{rawhtml}
  \htmladdtonavigation{\htmlref{\htmladdimg{contents_motif.gif}}
        {stardoccontents}}

% ? New section for abstract if used.
  \section{\xlabel{abstract}Abstract}
% ? End of new section for abstract

\end{htmlonly}

% -----------------------------------------------------------------------------
% ? Document Abstract. (if used)
%  ==================
\stardocabstract
% ? End of document abstract

% -----------------------------------------------------------------------------
% ? Latex Copyright Statement
%  =========================
\begin{latexonly}
\newpage
\vspace*{\fill}
\stardoccopyright
\end{latexonly}
% ? End of Latex copyright statement

% -----------------------------------------------------------------------------
% ? Latex document Table of Contents (if used).
%  ===========================================
  \newpage
  \begin{latexonly}
    \setlength{\parskip}{0mm}
    \tableofcontents
    \setlength{\parskip}{\medskipamount}
    \markboth{\stardocname}{\stardocname}
  \end{latexonly}
% ? End of Latex document table of contents
% -----------------------------------------------------------------------------
\cleardoublepage
\renewcommand{\thepage}{\arabic{page}}
\setcounter{page}{1}

\section{\xlabel{the_parameter_subsystem}The Parameter Subsystem}
The Starlink parameter system interface library,
\xref{PAR}{sun114}{abstract}\latexonly{ (see SUN/114)},
provides an interface between application programs and an underlying system
for handling program parameters. 

The current underlying parameter subsystem allows parameters to be obtained 
from a number of different sources such as the user (via prompts) or values 
generated by other applications. 
Facilities are also provided for dynamic generation of default values and 
value limit checking.
Programs may also save parameter values for use in later invocations or by
other programs.

The system is implemented by five libraries in the PCS package.
\begin{description}
\item[SUBPAR] The top level of the underlying parameter system. Many
of the basic PAR subroutines are almost straight-through calls to the 
corresponding SUBPAR subroutines.
\item[PARSECON] Parsing of the interface files associated with SUBPAR.
The package also includes the interface file compiler, COMPIFL.
\item[LEX] A lexical analyser used by SUBPAR in analysing command lines
\textit{etc}.
\item[STRING] Fortran string manipulation subroutines. These subroutines are
also used by software items outside the parameter system, but the library
has no published interface.
\item[MISC] Miscellaneous routines which do not fit into other libraries.
Two handle terminal I/O for programs running directly from a Unix shell
and others provide a Fortran interface into the C library for platforms 
(notably Linux) which do not include them as part of the system.
\end{description}

\section{\xlabel{the_communication_subsystems}The Communication Subsystems}
These libraries provide the system which is currently used to construct
programs capable of communicating with each other using the ADAM message
protocol. Messages can control the actions of programs or convey information.

\xref{SG/4}{sg4}{}
describes the simple use of the system (which is usually all that is needed for
data analysis programs) whilst 
\xref{SUN/134}{sun134}{} 
describes more complicated use in instrumentation control systems.
\begin{description}
\item[MSP] This library provides an inter-program communication system based 
upon a system of message `queues'.
\item[SOCK] This library provides a Unix-socket-based message transport system 
for the MSP system.
\item[AMS] This library implements the ADAM message protocol on top of MSP.
AMS is written in C but a Fortran interface, FAMS, is provided.
\xref{SUN/241}{sun241}{}
is the programmers manul for AMS.
\item[DTASK] This provides an application program structure which allows the 
program to be run directly from the shell or to respond to a specified set of 
control messages from other programs using the ADAM message protocol.
The application may consist of multiple `actions' which can be controlled
separately.
DTASK provides the main routine of the program and applications are written 
as subroutines which are called by the DTASK layer after the communication and
parameter systems have been initialised. This is described for system
programmers in
\xref{SSN/77}{ssn77}{}
The package also includes shell scripts to link such applications with PCS and
the other Starlink subroutine libraries required.
\item[TASK] This library provides an interface between the application code
and the DTASK layer so that an application can find out some information
about its own status. It also enables programs to control other co-operating
programs using the ADAM message protocol and includes subroutines for encoding
and decoding data values in messages.
For more details, see 
\xref{SUN/134}{sun134}{}.
\item[ATIMER] This library provides a system of millisecond interval timers 
used by the message system and DTASK. 
Each timer has an associated handler which is invoked when the timer expires.
The library is written in C but a Fortran interface, FATIMER, is provided.
\end{description}

\section{\xlabel{hdspar}HDSPAR}
This library is a slight anomaly and should probably be a separate item.
It provides a link between the parameter system and the
\xref{Hierarchical Data System (HDS)}{sun92}{abstract},
enabling object names to be specified by program parameters.
Unlike other PCS subroutines, the HDSPAR routines, DAT\_ASSOC \textit{etc.}, are
expected to be called directly from application programs.

The HDSPAR library is described in
\xref{SUN/224}{sun224}{}.

\section{\xlabel{obsolete_components}Obsolete Components}

\begin{description}
\item[ADAM] This library sends and receives messages using MESSYS.
The message content is assembled/disassembled from/to its constituent parts.
\item[MESSYS] This level of the message system provides compatibility with
earlier systems, thus removing the necessity for wholesale re-writing of 
higher-level libraries. It consists mainly of the corresponding calls to AMS
via the AMS Fortran interface. Many of the important parameters of the message 
system are defined here.
\end{description}

In PCS V4.0 use of these two libraries has been completely replaced by
calls direct to the Fortran interface of AMS. However, include files from both
packages are still required so the components are still included in PCS but
the subroutine libraries are not built or installed. 

\emph{This situation needs re-organisation}

\section{\xlabel{origins}Origins}
The PCS libraries have been developed over many years at several different 
establishments. Initially developed as part of the ADAM environment for 
instrument control at the Royal Observatories, notably 
\htmladdnormallink{ROE}{\ROEURL}, 
the environment was adopted in 1986 by the 
\htmladdnormallink{Starlink}{\STARURL}
project at the
\htmladdnormallink{Rutherford Appleton Lab}{\RALURL} 
to support data analysis programs.
Since then further developments have been made by ROE with substantial support 
and further developments from Starlink and additional support from 
\htmladdnormallink{AAO}{\AAOURL}
and 
\htmladdnormallink{JACH}{\JACHURL}.

\section{\xlabel{references}References}
{\em Note}: Only the first author is listed here.

\begin{latexonly}
\begin {tabular}{lll}
Lawden, M.D. & \xref{SG/4}{sg4}{} 
& ADAM -- The Starlink Software Environment.\\
Chipperfield, A.J. & \xref{SUN/144}{sun144}{}
& ADAM -- Unix Version.\\
Currie, M.J. & \xref{SUN114}{sun114}{}
& PAR -- Interface to the ADAM Parameter System.\\
Kelly, B.D. & \xref{SUN/134}{sun134}{}
& ADAM -- Guide to Writing Instrumentation Tasks.\\
Kelly, B.D. & \xref{SUN/241}{sun241}{}
& AMS -- The Unix ADAM Message System.\\
Chipperfield, A.J. & \xref{SSN/77}{ssn77}{}
& ADAM -- The Control Subsystem.\\
Warren-Smith, R.F. & \xref{SUN/92}{sun92}{}
& HDS -- Hierarchical Data System.
\end {tabular}
\end{latexonly}
\begin{htmlonly}
Lawden, M.D. : \xref{SG/4}{sg4}{} :
ADAM - The Starlink Software Environment.\\
Chipperfield, A.J. : \xref{SUN/144}{sun144}{} :
ADAM - Unix Version.\\
Currie, M.J. : \xref{SUN114}{sun114}{} :
PAR - Interface to the ADAM Parameter System.\\
Kelly, B.D. : \xref{SUN/134}{sun134}{} :
ADAM - Guide to Writing Instrumentation Tasks.\\
Kelly, B.D. & \xref{SUN/241}{sun241}{} :
AMS -- The Unix ADAM Message System.\\
Chipperfield, A.J. & \xref{SSN/77}{ssn77}{} :
ADAM -- The Control Subsystem.\\
Warren-Smith, R.F. : \xref{SUN/92}{sun92}{} :
HDS - Hierarchical Data System.
\end{htmlonly}

\end{document}
