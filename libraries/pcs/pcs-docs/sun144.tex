\documentclass[twoside,11pt,nolof]{starlink}

% ? Specify used packages
% ? End of specify used packages

% -----------------------------------------------------------------------------
% ? Document identification
% Fixed part
\stardoccategory    {Starlink User Note}
\stardocinitials    {SUN}
\stardocsource      {sun\stardocnumber}
\stardoccopyright
{Copyright \copyright\ 2000 Council for the Central Laboratory of the Research Councils}

% Variable part - replace [xxx] as appropriate.
\stardocnumber      {144.15}
\stardocauthors     {A J Chipperfield}
\stardocdate        {17 August 2001}
\stardoctitle       {ADAM}
\stardocversion     {Unix Version 4.0}
\stardocabstract  {
This document describes the use of the Starlink Software Environment, ADAM,
on Unix. It is primarily of use to programmers but the early sections contain
information useful to any user.
\par
It is assumed that the reader is familiar with the concepts of ADAM
programming and that the Starlink software is installed in the standard way.}
% ? End of document identification
% -----------------------------------------------------------------------------

% ? Document specific \providecommand or \newenvironment commands.
\providecommand{\ROEURL}{http://www.roe.ac.uk/}
\providecommand{\IRAFURL}{http://www.starlink.ac.uk/iraf/web/iraf-homepage.html}
% ? End of document specific commands
% -----------------------------------------------------------------------------
%  Title Page.
%  ===========
\begin{document}
\scfrontmatter


% ? Main text

\section{\xlabel{introduction}\label{introduction}Introduction}

The Starlink Software Collection provides an infrastructure of facilities
likely to be required by any astronomical application package.
The Starlink Software Environment (ADAM) is a particular way of combining the
elements of the collection to provide an integrated system with a common look
and feel across various packages.
Originally developed for VMS at
\htmladdnormallink{ROE}{\ROEURL},
the system is now supported by Starlink and has been ported to Unix.

Normal data analysis programs for the Starlink Environment are known as A-tasks
and it is possible to combine many A-tasks into a single monolithic executable
file (an A-task monolith) for efficiency -- the tasks may still be invoked
separately.
The Environment also supports programs for instrumentation control. These need
to behave differently from A-tasks and are known as I-tasks,

Programs written for the Starlink Software Environment may be run directly
from a Unix shell or from a variety of other user-interfaces including
the original ADAM user-interface,
\xref{ICL}{sg5}{abstract} (see SG/5),
and
\htmladdnormallink{IRAF}{\IRAFURL}
cl (see
\xref{SUN/217}{sun217}{}).

The first five sections of this document are relevant for any user but the
remaining sections will only be of interest to people writing software to run
in the environment.
For an introduction to these topics and details of how to write ADAM
programs, see
\xref{SG/4}{sg4}{}
for A-tasks and
\xref{SUN/134}{sun134}{}
for I-tasks.

Fine detail about methods of controlling the behaviour of ADAM
programs and ICL using environment variables is also
given in the appendices.

\section{\xlabel{running_from_the_shell}\label{running_from_the_shell}Running from the shell}
Most Starlink application programs are part of a `package'. The documentation
for the package will tell you how to start it up and run its applications.

Stand-alone ADAM programs are run from the shell in the same way as any Unix
program by typing:
\begin{small}
\begin{quote}
\texttt{\% \textit{program\_name}}
\end{quote}
\end{small}
where \texttt{\% \textit{program\_name}} may be a full pathname.

The Interface File, which defines the program's parameters, is usually in the
same directory as the executable but an alternative search path for Interface
Files may be specified in environment variable \texttt{ADAM\_IFL}.
If the variable is undefined, or the search is unsuccessful, the directory in
which the executable was found is assumed.
A file with the same name as the executable and with extension
\texttt{.ifc} or, failing that, \texttt{.ifl} is sought.

If the task is built into a monolith, a link with the name of the task must be
made to the monolithic executable and an individual Interface File for the task
provided. The link can then be executed as if it were a simple program.
(This is all usually set up by
\xref{package startup
scripts}{ssn64}{shell_package_startup_scripts} -- see SSN/46.)

When running tasks directly from a shell, the normal Unix rules for the
use of meta-characters on the command line will apply -- these sometimes
conflict with the ADAM parameter definition syntax.
Characters to be particularly wary of are `{\texttt{"}}',
`{\texttt{$\backslash$}}' and `{\texttt{\$}}'.

\section{\xlabel{icl_for_unix}ICL for Unix}
ICL for Unix behaves very like the VMS version described in
\xref{SG/5}{sg5}{},
however there are differences.
The ICL HELP command will give more up-to-date and Unix
oriented information than SG/5.

ICL is started from a Unix shell by a command of the form:
\begin{quote}
\texttt{\% icl} [\textit{ICL\_options}\/] [\textit{command\_filenames...}]
\end{quote}
Where:
\begin{description}
\item[\textit{ICL\_options}] (optional) have not yet been fully developed
and should not be used without advice. ICL options start
with `{\texttt{-}}' and must appear before any command-file names.
\item[\textit{command\_filenames...}] (optional) are the names of any files
containing ICL commands which are to be obeyed by ICL before the ICL prompt
appears.
They are loaded \emph{after} any ICL login files (see Appendix \ref{iclvars}).
A file extension of \texttt{.icl} is assumed if no extension is specified.
All filenames must appear after any ICL options.
\end{description}
Examples:
\begin{terminalv}
% icl
\end{terminalv}
This will be the normal invocation. Only defined ICL login files
will be loaded before the ICL prompt is output.

\begin{terminalv}
% icl -io test_io comfile1 comfile2
\end{terminalv}
which would set the value of the ICL option \texttt{-io} to \texttt{test\_io}
and the ICL command files \texttt{comfile1.icl} and \texttt{comfile2.icl}
would be automatically loaded into ICL (after any defined ICL login files)
before the ICL prompt was output.

\section{\xlabel{input_line_editing}\label{input_line_editing}Input-line Editing}
When you reply to prompts from either an ADAM program or ICL, previous input can
be recalled and edited -- automatic filename completion is also possible in
much the same way as it would be on the shell command line.
At prompts for ADAM program parameters, you can also paste the suggested value
into the input buffer and edit it. The maximum length of an input line is 256
characters -- the terminal will beep if you attempt to input more.

Line recall and editing for ADAM uses virtually the same keys as \texttt{tcsh}
with the \texttt{emacs} key bindings.

\subsection{Input-line Recall}
This is achieved using the up and down arrow keys -- the recalled line can
then be edited in the normal way.
Up to a maximum of 100 input lines will be remembered but in the case of
programs run from the shell, only responses to prompts in the current
invocation of the program can be recalled.

\subsection{Suggested Value Recall}
If there are no characters on the input line, the \texttt{TAB} key (or
\texttt{ESC},\texttt{ESC}) will cause the suggested value (if any) to be
inserted as the  input line. It can then be edited in the normal way.
(There will be no suggested value at the \texttt{ICL>} prompt.)

See
\htmlref{Filename Completion}{filename_completion}
(Section \ref{filename_completion}) for the effect of these
characters if there is already some input on the line.

\subsection{\label{filename_completion}Filename completion}
If there are currently characters on the input line, the \texttt{TAB} key (or
\texttt{ESC},\texttt{ESC}) will cause filename completion to be attempted on
the word preceding the cursor.
If no match is found, ``No match.'' will be printed and the terminal will beep;
if more than one match is found, ``Multiple matches.'' will be printed and the
terminal will beep -- the input line will be set to the longest common
prefix.
A list of all possible matches can be displayed by typing
\texttt{ESC},\texttt{CNTL/D}. (At the end of line, just \texttt{CNTL/D} is
sufficient -- elsewhere \texttt{CNTL/D} will delete the character at the
cursor.)

To overcome the problem of Starlink NDF and HDS filenames usually being
required without the \texttt{.sdf} extension, \texttt{.sdf} will be omitted
from the end of any completed filename.
This behaviour may be altered by setting environment variable
ADAM\_EXTN to a comma-separated list of extensions (in fact any strings)
which are to be omitted from the end of completed filenames.
If no truncation of filenames is required, ADAM\_EXTN should be set to a null
string.

When a single match is found, the filename is truncated if required and copied
to the input line followed by a single space.

For example, suppose the default for filename truncation (\texttt{.sdf}) is in
use and the  current directory contains two files \texttt{file.dat} and
\texttt{file.sdf}.
The dialogue might go as follows (\verb!<>! indicates typing by the user):
\begin{small}
\begin{terminalv}
Give NDF name > <x><TAB>
No match.[beep]
Give NDF name > <f><TAB>
Multiple matches.[beep]
Give NDF name > file.<CNTL/D>
file.dat    file.sdf
Give NDF name > file.<s><TAB>
Give NDF name > file <Return>
Give auxiliary data file name > <file.d><TAB>
Give auxiliary data file name > file.dat <Return>
\end{terminalv}
\end{small}

(The last four lines would appear as two lines on the terminal, the second and
fourth overwriting the first and third respectively.)

\subsection{Editing the Input Line}
The left and right arrows and the delete key may be used as expected to edit the
current input line. Input is always in `insert' rather than `overwrite' mode.

Key sequences for more complex line editing are given in Appendix
\ref{edit_keys}.

\subsection{\xlabel{other_special_keys}\label{other_special_keys}Other Special Keys}
\begin{description}
\item[\texttt{CNTL/Z}] This will suspend the program in the normal way. It may
be re-started with the \texttt{fg} command.
\item[\texttt{CNTL/C}]
This will abort a task running from the shell.
If running ICL, the effect depends on what is happening at the time.
Generally, ICL itself will keep running but the current activity will be
aborted (see
\xref{SG/5}{sg5}{errors_and_exceptions}
for information on ICL exception handling).
\end{description}

\section{\xlabel{the_adam_user_directory}\xlabel{ADAM_USER_directory}
\label{adam_user}The ADAM\_USER Directory}
When any ADAM application, or
\xref{ICL}{sg5}{},
is run, a directory known as the
\textbf{ADAM\_USER} directory is used to hold various automatically-created files.
These are: program parameter files, the global parameter file (GLOBAL.sdf)
and task identifying files (used by the ADAM inter-task message system).
The directory is also used by the
\xref{MAG}{sun171}{} library as the default to contain
information about the user's tape devices.

By default, subdirectory \texttt{adam} in the user's HOME directory
is used. If you want to use some other directory, set its name in
environment variable ADAM\_USER.

Whichever directory is used, it will be created if necessary (and possible).
The directory and/or its contents can be deleted when not in use, but this
may remove memory of parameter values last used.

\section{\xlabel{compiling_and_linking}Compiling and Linking}
Originally ADAM programs were always written in Fortran, the top-level module
being written as a subroutine with a single \texttt{INTEGER} argument, the
ADAM status.
The
\htmlref{link scripts}{link_scripts}
now also accept C functions. If the top-level function is a C function, it
must have a single argument of type \texttt{int *}. If it is not written as a
Fortran-callable routine, the source file must be presented to the link script.

Note that the complete set of C interfaces for the Starlink libraries
is not yet available so it may not be possible to write your program entirely
in C without the aid of something like the
\xref{CNF package}{sun209}{} (see SUN/209).

To compile and link ADAM programs, it is necessary to add \texttt{/star/bin} to
your PATH environment variable. This is done if you have `sourced'
\texttt{/star/etc/login} to set up for Starlink software generally.

\subsection{\xlabel{include_files}\label{incs}Include Files}
All public include files, such as \texttt{\textit{pkg}\_par} and
\texttt{\textit{pkg}\_err} files, will be found in \texttt{/star/include}.

The filenames of the Fortran include files are in lower case but they should
be specified in the program in the standard Starlink way. That is, for example:
\begin{terminalv}
INCLUDE 'SAE_PAR'
INCLUDE 'PAR_ERR'
\end{terminalv}
Links are then set up to the public include files in \texttt{/star/include}
using the appropriate \textit{pkg}\_dev script.
For example:
\begin{terminalv}
% par_dev
\end{terminalv}
creates links \texttt{PAR\_PAR} and \texttt{PAR\_ERR} to
\texttt{/star/include/par\_par} and \texttt{/star/include/par\_err}
respectively.
(\texttt{SAE\_PAR} is defined by \texttt{star\_dev}.)

For C programs, includes take the normal form:
\begin{terminalv}
#include "par_err.h"
\end{terminalv}
(\texttt{-I/star/include} is included in the
\htmlref{link scripts}{link_scripts}).

\subsection{\xlabel{ADAM_link_scripts}\xlabel{adam_link_scripts}
\label{link_scripts}ADAM Link Scripts}
Two shell scripts are provided to link ADAM programs.
\texttt{alink} is used to link A-tasks and A-task monoliths, and
\texttt{ilink} to link I-tasks.
The executable images produced may be run either from a suitable user-interface
such as
\xref{ICL}{sg5}{abstract},
or directly from the shell.

The following description uses \texttt{alink} in examples; \texttt{ilink} is
used in the same way.

To link an ADAM program, ensure that \texttt{/star/bin} is added to your
environment variable PATH, then type:
\begin{quote}
\texttt{\% alink} [{\texttt{-xdbx}}] {\textit{file}} [{\textit{arguments}}]
\end{quote}
Where:
\begin{description}
\item[{\texttt{-xdbx}}] optional (but must be the first argument if used),
overcomes some problems with debugging ADAM tasks, notably with \texttt{xdbx}
and \texttt{ups} on SunOS, by supplying dummy source files for required ADAM
system routines. It also has the effect of inserting a \texttt{-g} option into
the argument list. For more details, see Appendix \ref{link_det}.
\item[\textit{prog\_module}] specifies a file containing the task's main
subroutine. The filename should be of the form
{\textit{path}/\textit{name}\bf{.f}}, {\textit{path}/\textit{name}\bf{.c}},
{\textit{path}/\textit{name}\bf{.o}} or  {\textit{path}/\textit{name}}.
The \textit{path}/ component is optional but
\textit{name} must be the name of the program's main subroutine and will
be the name of the executable file produced in the current working directory.
If the file extension is \texttt{.f} or \texttt{.c}, the file will be compiled
appropriately; if none of the permitted extensions is given,
\texttt{.o} is appended.

Note that after a \texttt{.f} file has been compiled, on some platforms the
\texttt{.o} file will be retained in the current working directory so that
subsequent \texttt{alink}s with an unchanged \texttt{.f} file may specify the
name with no extension ( or \texttt{.o}).
The object files for any compiled C files will always be retained.
\item[\textit{arguments}] optional, is any additional arguments (library
specifications, compiler options \textit{etc.}) legal for the Fortran
compiler, or any C files to be compiled separately and linked in.
The list of Starlink libraries automatically included in the link is given in
Appendix~\ref{libs}; the method of including other Starlink libraries in ADAM
programs will be specified in the relevant Starlink User Note.

In most cases it will be:
\begin{quote}
\texttt{\% alink} {\textit{prog\_module} \texttt{`}\textit{pkg}\texttt{\_link\_adam`}}
\end{quote}
where \texttt{\textit{pkg}} is the specific software item name \textit{e.g.}
\texttt{ndf}.
\end{description}

Appendix \ref{link_det} gives details of the command used within
\texttt{alink/ilink} to compile and/or link the tasks.
This may assist users who wish to alter the standard behaviour.

By default, programs are statically linked with the Starlink libraries.
On platforms where shared libraries are installed, programs may be linked with
them by setting the environment variable ALINK\_FLAGS1 appropriately.
On currently supported systems it should be set to \texttt{-L/star/share}.
(See Appendix \ref{link_det} for more details.)

\subsection{\xlabel{interface_files}Interface Files}
Interface Files may be compiled by the \texttt{compifl} program which is
available in \texttt{/star/bin}.
Compiled Interface Files (\texttt{.ifc}s) must be produced on the platform on
which they are to be used.

Generally, the only things which will need changing when porting Interface
Files between operating systems are file and device names.
Case is significant on Unix and obviously the format of filenames is different
from VMS, for example.

\subsection{\xlabel{monoliths}\label{ADAM_monoliths}Monoliths}
The top-level routine for Unix A-task monoliths should be of the form:
\begin{small}
\begin{terminalv}
          SUBROUTINE TEST( STATUS )
          INCLUDE 'SAE_PAR'
          INCLUDE 'PAR_PAR'

          INTEGER STATUS

          CHARACTER*(PAR__SZNAM) NAME

          IF (STATUS.NE.SAI__OK) RETURN

*       Get the action name
          CALL TASK_GET_NAME( NAME, STATUS )

*       Call the appropriate action routine
          IF (NAME.EQ.'TEST1') THEN
            CALL TEST1(STATUS)
          ELSE IF (NAME.EQ.'TEST2') THEN
            CALL TEST2(STATUS)
          ELSE IF (NAME.EQ.'TEST3') THEN
            CALL TEST3(STATUS)
          END IF
          END
\end{terminalv}
\end{small}

To run such a monolith from a Unix shell, link the required action name to the
monolith, then execute the linkname (possibly via an alias). For example:

\begin{terminalv}
% ln -s $KAPPA_DIR/kappa add
% add
\end{terminalv}

Separate Interface Files are required for each action run from the shell
-- a monolithic Interface File is required for monoliths run from ICL.

\section{\xlabel{help_files}Help Files}
The Starlink portable help system (HLP) is used to provide help if ? or ?? is
typed in response to a parameter prompt.
Instructions for creating help libraries and navigating through them may be
found in
\xref{SUN/124}{sun124}{}.

Interface File entries specifying help libraries may be given as standard
pathnames \textit{e.g.}:
\begin{quote}
\begin{terminalv}
helplib /star/help/kappa/kappa.shl
\end{terminalv}
or
\begin{terminalv}
help %/star/help/kappa/kappa ADD Parameters IN1
\end{terminalv}
\end{quote}
Note that the file extension \texttt{.shl} is optional.

The system will also accept environment variables and \~{} in the help
library name, \textit{e.g.}:
\begin{quote}
\texttt{\$KAPPA\_DIR/kappahelp.shl} or \texttt{\$KAPPA\_HELP} or
\texttt{\~{}/help/myprog}
\end{quote}

For historical reasons, so that the same Interface File entry will work with
both Unix and VMS, the VMS form:
\begin{quote}
\texttt{KAPPA\_DIR:kappahelp.shl} or \texttt{KAPPA\_HELP:}
\end{quote}
will be accepted.

If the VMS form is used, the environment variable name will be forced to upper
case, and the filename to lower case for interpretation.
Unix-style specifications will be interpreted as given.

\section{\xlabel{miscellaneous_points}Miscellaneous Points}
\begin{itemize}
\item For Unix ADAM applications linked since ADAM V3.1, the ADAM\_USER
directory and global parameter file, \texttt{GLOBAL.sdf}, are created
automatically.
\item Compiled Interface Files (\texttt{.ifc}s) will only work for the platform
type on which they were created.
\item Monolith top-level routines must be written in the style described
in Section \ref{ADAM_monoliths} for use on Unix.
Monoliths are linked using \texttt{alink}.
\item ADAM tasks will usually default to interpreting environment variables
and `\~{}' in filenames used with HDS. This is not necessarily true of
names used with other ADAM facilities such as FIO and may be overridden by
user or programmer action for HDS filenames.
\end{itemize}

\section{\xlabel{references}References}
\emph{Note}: Only the first author is listed here.

\begin{latexonly}
\begin{tabular}{lll}
SG/4 & M.D.Lawden & ADAM -- The Starlink Software Environment.\\
SG/5 & J.A.Bailey & ICL -- The Interactive Command Language for ADAM.\\
SUN/101 & Jo Murray & Introduction to ADAM Programming.\\
SUN/124 & P.T.Wallace & HELP -- Interactive Help System.\\
SUN/134 & B.D.Kelly & ADAM -- Guide to Writing Instrumentation Tasks.\\
SUN/171 & P.M.Allan & MAG -- Access to Magnetic Tapes.\\
SUN/209 & P.M.Allan & CNF and F77 -- Mixed Language Programming.\\
SUN/217 & A.J.Chipperfield & Running Starlink Programs from IRAF CL.\\
SSN/4 & P.C.T.Rees  &EMS -- Error Message Service.\\
SSN/64 & A.J.Chipperfield & ADAM -- Organization of Application Packages.
\end{tabular}
\end{latexonly}




Key:
\begin{tabular}{ll}
SG & Starlink Guide.\\
SSN & Starlink System Note.\\
SUN & Starlink User Note.
\end{tabular}

\section{\xlabel{document_changes}Document Changes}
\subsection{SUN/144.13}
The sections on the
\htmlref{use}{link_scripts} (Section \ref{link_scripts})
and
\htmlref{details}{link_det} (Appendix \ref{link_det})
of the ADAM link scripts, \texttt{alink} and \texttt{ilink}, have been modified
to reflect some changes designed to allow easy switching between static and
dynamic linking with the Starlink libraries.

\subsection{SUN/144.14}
The document was re-formatted to include the standard copyright statement -
there are no changes in substance.

\subsection{SUN/144.15}
The ADAM and MESSYS libraries are removed from the list of
\htmlref{available libraries}{libs}.

Environment variable ADAM\_NOPROMPT is added to the list of
\htmlref{ADAM Environment Variables}{envars}.
\newpage
\appendix
\section{\xlabel{example_session}Example Session}
The following session shows the process of compiling, linking and running an
example program, derived from
\xref{SUN/101}{sun101}{},
on the Sun.

\begin{terminalv}
% source /star/etc/login
%
% ls
repdim2.f       repdim2.ifl     example.sdf
%
% cat repdim2.f
*   Program to report the dimensions of an NDF.
*   The CHR package is used to produce a nice output message.
*   See SUN/101, section 11.

      SUBROUTINE REPDIM2 (STATUS)
      IMPLICIT NONE
      INCLUDE 'SAE_PAR'
      INTEGER DIM(10), I, NCHAR, NDF1, NDIM, STATUS
      CHARACTER*100 STRING

*   Check inherited global status.
      IF (STATUS.NE.SAI__OK) RETURN

*   Begin an NDF context.
      CALL NDF_BEGIN

*   Get the name of the input NDF file and associate an NDF
*   identifier with it.
      CALL NDF_ASSOC ('INPUT', 'READ', NDF1, STATUS)

*   Enquire the dimension sizes of the NDF.
      CALL NDF_DIM (NDF1, 10, DIM, NDIM, STATUS)

*   Set the token 'NDIM' with the value NDIM.
      CALL MSG_SETI ('NDIM', NDIM)

*   Report the message.
      CALL MSG_OUT (' ', 'No. of dimensions is ^NDIM', STATUS)

*   Report the dimensions.
      NCHAR = 0
      CALL CHR_PUTC ('Array dimensions are ', STRING, NCHAR)
      DO  I = 1, NDIM
*      Add a `x' between the dimensions if there are more than one.
         IF (I.GT.1) CALL CHR_PUTC (' x ', STRING, NCHAR)
*      Add the next dimension to the string.
         CALL CHR_PUTI (DIM(I), STRING, NCHAR)
      ENDDO
      CALL MSG_OUT ('  ', STRING(1:NCHAR), STATUS)

*   End the NDF context.
      CALL NDF_END (STATUS)
      END
%
% cat repdim2.ifl
interface REPDIM2
   parameter      INPUT
      position    1
      prompt      'Input NDF structure'
      default     example
      association '->global.ndf'
   endparameter
endinterface
%
% star_dev
% alink repdim2.f `ndf_link_adam`
f77 -g -c repdim2.f
repdim2.f:
        repdim2:
dtask_applic.f:
        dtask_applic:
%
% repdim2
INPUT - Input NDF structure /@example/ >
No. of dimensions is 1
Array dimensions are 856
%
% ls ~/adam
repdim2.sdf     GLOBAL.sdf
%
% compifl repdim2
!! COMPIFL: Successful completion
%
% ls
example.sdf   repdim2.f     repdim2.ifl
repdim2       repdim2.ifc   repdim2.o
%
\end{terminalv}
Note that whether or not the file repdim2.o is retained depends upon the
compiler used.

\newpage
\section{\xlabel{link_script_details}\label{link_det}Link Script Details}
The link scripts firstly have to create a subroutine, DTASK\_APPLIC, which is
called by the ADAM fixed part and in turn calls the user's top-level
routine.
The difference between \texttt{alink} and \texttt{ilink} is just that the
template DTASK\_APPLIC for \texttt{alink} contains a call to close down the
parameter system after each invocation of the task unless the environment
variable ADAM\_TASK\_TYPE is set to `I'.

If the user's main routine is written in C, a temporary routine,
DTASK\_WRAP is created as a Fortran-callable wrapper for the user's
routine. DTASK\_APPLIC will call DTASK\_WRAP, which in turn calls the user's
routine.

During installation (as part of the DTASK library in the
\xref{PCS}{ssn29}{}
package), the actual compile/link command within \texttt{alink/ilink} is edited
depending upon the platform and setting of various environment variables.
The template command is:
\begin{terminalv}
F77 $FFLAGS -o $EXENAME \
$XDBX \
/star/lib/dtask_main.o \
dtask_applic.f \
$ALINK_FLAGS1 \
$ARGS \
-L/star/lib \
-lhdspar_adam \
-lpar_adam \
`dtask_link_adam` \
$ALINK_FLAGS2 \
Additional system libraries
\end{terminalv}
Notes:
\begin{itemize}
\item F77 is replaced by whatever the FC environment variable is when
\texttt{alink/ilink} is installed.
\item Any C source files specified will be compiled separately and linked in
the above Fortran command.  The C compiler command used will be of the form:
\begin{terminalv}
% CC -c $CFLAGS $CARGS -I/star/include
\end{terminalv}
where:
\begin{description}
\item[\texttt{CC}] is replaced by the CC environment variable when
\texttt{alink/ilink} script is installed.
\item[\texttt{\$CFLAGS}] is the value of the CFLAGS environment variable when
\texttt{alink/ilink} is invoked.
\item[\texttt{\$CARGS}] is any C files or \texttt{-I} or \texttt{-D} arguments
on the \texttt{alink/ilink} command line, plus, if the main routine is C,
\texttt{dtask\_wrap.c}\end{description}
\item \texttt{\$EXENAME} is the basename of the \textit{prog\_module}
argument of \texttt{alink/ilink} with any \texttt{.f}, \texttt{.c} or
\texttt{.o} suffix removed.
\item Environment variable FFLAGS may be used to specify any options which
must be included at the start of the command line.
\item \texttt{\$XDBX} is set to \texttt{-g} if the \texttt{-xdbx} argument
is given.
\item \texttt{\$ALINK\_FLAGS1} may be set to determine the type of linking
required. For instance:
\begin{terminalv}
% setenv ALINK_FLAGS1 -L/star/share
\end{terminalv}
would cause the linker to find the Starlink
shared libraries (on platforms where they are installed), thus producing a
dynamically linked executable.
\item \texttt{\$ARGS} is the \textit{prog\_module} argument (with \texttt{.o}
appended if the original extension was not \texttt{.f}, \texttt{.c} or
\texttt{.o})
or \texttt{\$EXENAME.o dtask\_wrap.o} if the main routine was in C, followed
by the remaining arguments unchanged except that any \texttt{.c} file
extensions are replaced by \texttt{.o}.
\item To speed up the link, \textit{pkg}\_link\_adam scripts are only used
selectively.
\texttt{dtask\_link\_adam} refers to DTASK, TASK and ERR libraries directly then
invokes \texttt{subpar\_link\_adam} which references the necessary
libraries directly, apart from HDS, HLP and PSX whose link\_adam scripts are
invoked.
\item \texttt{ALINK\_FLAGS2} may be useful in controlling the way system
libraries are accessed.
\item \texttt{Additional system libraries} A platform-dependent list of
required system libraries which are not searched automatically is added to
\texttt{alink/ilink} at installation time.
\item Other adjustments are made during installation if PCS is not being
installed in \texttt{/star}.
In particular, a -L option is added to include the newly installed libraries
before those in \texttt{/star/lib}.
Similarly with -I options in the C compilation.
\end{itemize}
The \texttt{-xdbx} argument is provided to overcome some awkward problems which
can arise when debugging ADAM applications. Usually it is sufficient to
include \texttt{-g} in the arguments of \texttt{alink/ilink} but sometimes,
notably when using \texttt{xdbx} and \texttt{ups} on SunOS, the debuggers do
not behave sensibly if required source files are missing so the \texttt{-xdbx}
argument should be used instead.
The effects of the argument are:
\begin{itemize}
\item A dummy source file for \texttt{dtask\_main}, the main routine of every
ADAM task, is created in the working directory.
The file contains an explanatory message to the user and the name of the
user's top-level subroutine (which may be helpful in selecting a breakpoint).
\item The source file of DTASK\_APPLIC is not deleted from the working
directory.
\item \texttt{-g} is inserted in the compile/link command.
\end{itemize}
\newpage
\section{\xlabel{available_libraries}\label{libs}Available Libraries}
\subsection*{ADAM System Libraries}
\begin{verse}
HDSPAR (DAT\_ASSOC \textit{etc.}). This library is named DATPAR in the VMS
release \\
SUBPAR \\
PARSECON \\
STRING \\
LEX \\
DTASK \\
TASK \\
MISC (Miscellaneous routines required for Unix.) \\
AMS (and its subsidiary libraries MSP and SOCK) \\
ATIMER
\end{verse}

\subsection*{Starlink Libraries Searched Automatically}
The following separate Starlink libraries will be searched automatically by
the ADAM link. The libraries used for ADAM may differ from the stand-alone
versions (see relevant documents for details).

\begin{verse}
PAR \\
ERR/MSG/EMS \\
HDS \\
CHR \\
PSX \\
HLP \\
CNF
\end{verse}

\subsection*{Libraries Not Searched Automatically}
The following libraries must be included by optional arguments on Unix.

\begin{verse}
AGI \\
ARY \\
AST \\
GKS (includes GKSPAR) \\
GNS \\
GRP \\
IDI \\
IMG \\
NDF \\
SGS (includes SGSPAR) \\
PGPLOT (includes PGPPAR) \\
\end{verse}

\newpage
\section{\xlabel{ADAM_environment_variables}\label{envars}ADAM Environment Variables}
For more complex operation of ADAM tasks, the user may make use of the following
environment variables:
\begin{description}
\item[HOME] Is expected to specify the user's home directory.
\item[ADAM\_USER]
ADAM\_USER may be set to define a directory other than \texttt{\$HOME/adam}
to hold the program parameter files \textit{etc.} (see Section \ref{adam_user}).
\item[ADAM\_IFL] Optionally specifies a search path of directories in
which the system is to look for Interface Files.
See the
\htmlref{Running from the Shell}{running_from_the_shell}
section for details.
\item[ADAM\_HELP] Specifies a search path of
directories in which the parameter system is to look for parameter help files
if a full pathname is not specified in the Interface File -- it is not
usually required.
\item[ADAM\_ABBRV]
If this environment variable is set, keywords used on the
command line may be abbreviated to the minimum unambiguous length.
Note that there could always be an undetectable ambiguity between logical or
special keywords and unquoted strings or names. To alleviate this problem
slightly, special keywords (PROMPT, RESET \textit{etc.}) must always be given
with a minimum of two characters. This environment variable is set by default.
\item[ADAM\_NOPROMPT] This will prevent the task from prompting either for
parameter values or to resolve an ambiguous keyword. Status PAR\_\_NOUSR will
be returned for a parameter prompt, and SUBPAR\_\_CMDSYER if an ambiguous
keyword is given on the command line.
\item[ADAM\_EXTN]
Specifies a comma-separated list of extensions to be removed from the end
of filenames after filename completion. By default \texttt{.sdf} will be
removed.
See the
\htmlref{Filename Completion}{filename_completion}
section for details.
\item[ADAM\_TASK\_TYPE]
If set to `I', this will prevent A-tasks and A-task monoliths
resetting active parameter values (and NULL states) after each invocation.
Most other aspects of parameter system closedown (such as updating associated
GLOBAL variables and unsetting dynamic defaults and MIN/MAX values) will still
occur.
This is of particular use for Graphical User Interfaces and is unlikely
to be set directly by users.
\item[ADAM\_EXIT] If this environment variable is set when an ADAM task
terminates, the calling process will exit with system status set to 1 if
the ADAM status was set, or 0 if the ADAM status was SAI\_\_OK.
\item[EMS\_PATH] Unix ADAM will now use the EMS subroutines for obtaining the
message associated with Starlink status values. EMS\_PATH may be used to
override the default search path for the message files
(see
\xref{SSN/4}{ssn4}{operating_system_specific_routines}
for details).
\item[PATH] In addition to its use by the system to find the required
executable file, the environment variable PATH is used by the parameter
system to find the pathname of the file being executed if it was invoked by
simply typing its name (not its pathname).
This is needed to discover the directory in which to look for the Interface
File if the ADAM\_IFL search is unsuccessful.
This process means that the use of links may cause confusion -- the name
and directory of the link will be used.
\item[HDS\_SHELL] The interpretation of names given as values for parameters
accessed via PAR or DAT routines will be handled by HDS.
The environment variable HDS\_SHELL
(see
\xref{SUN/92}{sun92}{HDS_SHELL_tuning_parameter})
will be effective.
If it is not set when the application starts, interpretation with SHELL=1
(\texttt{csh} or failing that \texttt{sh}) is selected
-- thus environment variables and `\~{}' are usually expanded.
Note that parameter system syntax will usually prevent the use of more general
shell expressions as names.
\item[ICL Environment Variables] See Appendix \ref{iclvars} for details of the
environment variables used by ICL.
\end{description}

\section{\xlabel{icl_environment_variables}\label{iclvars}ICL Environment Variables}
ICL's operation can be controlled by several (optional) environment
variables. The variables are:
\begin{description}
\item[ICL\_LOGIN\_SYS, ICL\_LOGIN\_LOCAL and ICL\_LOGIN]
these may be set to specify ICL command files to be obeyed,
in the above order, by ICL before the ICL prompt appears.
A default file extension of \texttt{.icl} is assumed. For example:
\begin{terminalv}
% setenv ICL_LOGIN ~/myprocs
\end{terminalv}
will cause file \texttt{myprocs.icl} in the user's home directory to be loaded
as ICL starts up.

ICL\_LOGIN\_SYS and ICL\_LOGIN\_LOCAL should be reserved for system use.
\xref{SSN/64}{ssn64}{}
describes how they are used at Starlink sites.
\item[EDITOR] If set, this will override the ICL default editor (vi).
For example:
\begin{terminalv}
% setenv EDITOR tpu
\end{terminalv}
\item[ICL\_HELPFILE] If set, this will override the default search path for the
ICL helpfile. The default search path is:
\begin{terminalv}
../help/icl/iclhelp.shl
\end{terminalv}
relative to each of the directories on the user's PATH.

This process for finding the default helpfile only operates if the default
helpfile is not defined -- \textit{i.e.}\ initially and after any
SET NOHELPFILE command.
\item[SHELL] Defines the shell which ICL will use to run Unix commands.
If SHELL is undefined, \texttt{csh} will be used.
\item[ICL\_TASK\_NAME] This environment variable is set by ICL to the name
by which it wants the task to register with the ADAM message system. Other
user-interfaces controlling ADAM tasks via the ADAM message system will use the
same mechanism. If the environment variable is not set, the task assumes it
is being run directly from a shell and does not register with the message
system.
\end{description}
In addition to the above, ICL also makes use of some of the environment
variables listed in Section \ref{envars}, notably ADAM\_USER and ADAM\_EXTN.

\newpage
\section{\xlabel{edit_keys}\label{edit_keys}Edit Keys}
In addition to the left/right arrows and delete key, the following keys may be
used in editing the input line:

\begin{tabular}{cl}
Key & Function\\
\texttt{CNTL/A} & Move to Start of Line\\
\texttt{CNTL/B} & Backward character (same as left arrow)\\
\texttt{CNTL/C} & Abort (see Section \ref{other_special_keys})\\
\texttt{CNTL/D} & Delete character, or list choices\\
\texttt{CNTL/E} & Move to End of Line\\
\texttt{CNTL/F} & Forward character (same as right arrow)\\
\texttt{CNTL/H} & Backward delete character (same as delete)\\
\texttt{CNTL/K} & Delete to End of Line\\
\texttt{CNTL/N} & Down history (same as down arrow)\\
\texttt{CNTL/P} & Up history (same as up arrow)\\
\texttt{CNTL/R} & Redisplay\\
\texttt{CNTL/U} & Delete Line\\
\texttt{CNTL/Z} & Suspend\\
\\
\texttt{ESC},\texttt{CNTL/D} & List choices\\
\texttt{ESC},\texttt{CNTL/H} & Backward delete word\\
\texttt{ESC},\texttt{B} & Backward word\\
\texttt{ESC},\texttt{D} & Forward Delete word\\
\texttt{ESC},\texttt{F} & Forward word\\
\end{tabular}

Note that some of these have changed from the undocumented features in ICL
prior to Version 3.1-6.

\end{document}
