\documentclass[twoside,11pt]{article}

% ? Specify used packages
% \usepackage{graphicx}        %  Use this one for final production.
% \usepackage[draft]{graphicx} %  Use this one for drafting.
% ? End of specify used packages

\pagestyle{myheadings}

% -----------------------------------------------------------------------------
% ? Document identification
% Fixed part
\newcommand{\stardoccategory}  {Starlink System Note}
\newcommand{\stardocinitials}  {SSN}
\newcommand{\stardocsource}    {ssn\stardocnumber}
\newcommand{\stardoccopyright}
{Copyright \copyright\ 2000 Council for the Central Laboratory of the Research Councils}

% Variable part - replace [xxx] as appropriate.
\newcommand{\stardocnumber}    {77.0}
\newcommand{\stardocauthors}   {A.J. Chipperfield}
\newcommand{\stardocdate}      {16 August 2001}
\newcommand{\stardoctitle}     {ADAM\\[2ex]
                                The Control Subsystem}
\newcommand{\stardocabstract}  {This document describes the way an ADAM task
interacts with the ADAM Message System
(\xref{AMS}{sun241}{}).
It is intended for system programmers.

The reader is expected to be familiar with the Starlink Software Environment
documented in
\xref{SG/4}{sg4}{},
\xref{SUN/144}{sun144}{}
and
\xref{SUN/134}{sun134}{}}
% ? End of document identification
% -----------------------------------------------------------------------------

% +
%  Name:
%     ssn.tex
%
%  Purpose:
%     Template for Starlink System Note (SSN) documents.
%     Refer to SUN/199
%
%  Authors:
%     AJC: A.J.Chipperfield (Starlink, RAL)
%     BLY: M.J.Bly (Starlink, RAL)
%     PWD: Peter W. Draper (Starlink, Durham University)
%
%  History:
%     17-JAN-1996 (AJC):
%        Original with hypertext macros, based on MDL plain originals.
%     16-JUN-1997 (BLY):
%        Adapted for LaTeX2e.
%     13-AUG-1998 (PWD):
%        Converted for use with LaTeX2HTML version 98.2 and
%        Star2HTML version 1.3.
%      1-FEB-2000 (AJC):
%        Add Copyright statement in LaTeX
%     {Add further history here}
%
% -

\newcommand{\stardocname}{\stardocinitials /\stardocnumber}
\markboth{\stardocname}{\stardocname}
\setlength{\textwidth}{160mm}
\setlength{\textheight}{230mm}
\setlength{\topmargin}{-2mm}
\setlength{\oddsidemargin}{0mm}
\setlength{\evensidemargin}{0mm}
\setlength{\parindent}{0mm}
\setlength{\parskip}{\medskipamount}
\setlength{\unitlength}{1mm}

% -----------------------------------------------------------------------------
%  Hypertext definitions.
%  ======================
%  These are used by the LaTeX2HTML translator in conjunction with star2html.

%  Comment.sty: version 2.0, 19 June 1992
%  Selectively in/exclude pieces of text.
%
%  Author
%    Victor Eijkhout                                      <eijkhout@cs.utk.edu>
%    Department of Computer Science
%    University Tennessee at Knoxville
%    104 Ayres Hall
%    Knoxville, TN 37996
%    USA

%  Do not remove the %begin{latexonly} and %end{latexonly} lines (used by
%  LaTeX2HTML to signify text it shouldn't process).
%begin{latexonly}
\makeatletter
\def\makeinnocent#1{\catcode`#1=12 }
\def\csarg#1#2{\expandafter#1\csname#2\endcsname}

\def\ThrowAwayComment#1{\begingroup
    \def\CurrentComment{#1}%
    \let\do\makeinnocent \dospecials
    \makeinnocent\^^L% and whatever other special cases
    \endlinechar`\^^M \catcode`\^^M=12 \xComment}
{\catcode`\^^M=12 \endlinechar=-1 %
 \gdef\xComment#1^^M{\def\test{#1}
      \csarg\ifx{PlainEnd\CurrentComment Test}\test
          \let\html@next\endgroup
      \else \csarg\ifx{LaLaEnd\CurrentComment Test}\test
            \edef\html@next{\endgroup\noexpand\end{\CurrentComment}}
      \else \let\html@next\xComment
      \fi \fi \html@next}
}
\makeatother

\def\includecomment
 #1{\expandafter\def\csname#1\endcsname{}%
    \expandafter\def\csname end#1\endcsname{}}
\def\excludecomment
 #1{\expandafter\def\csname#1\endcsname{\ThrowAwayComment{#1}}%
    {\escapechar=-1\relax
     \csarg\xdef{PlainEnd#1Test}{\string\\end#1}%
     \csarg\xdef{LaLaEnd#1Test}{\string\\end\string\{#1\string\}}%
    }}

%  Define environments that ignore their contents.
\excludecomment{comment}
\excludecomment{rawhtml}
\excludecomment{htmlonly}

%  Hypertext commands etc. This is a condensed version of the html.sty
%  file supplied with LaTeX2HTML by: Nikos Drakos <nikos@cbl.leeds.ac.uk> &
%  Jelle van Zeijl <jvzeijl@isou17.estec.esa.nl>. The LaTeX2HTML documentation
%  should be consulted about all commands (and the environments defined above)
%  except \xref and \xlabel which are Starlink specific.

\newcommand{\htmladdnormallinkfoot}[2]{#1\footnote{#2}}
\newcommand{\htmladdnormallink}[2]{#1}
\newcommand{\htmladdimg}[1]{}
\newcommand{\hyperref}[4]{#2\ref{#4}#3}
\newcommand{\htmlref}[2]{#1}
\newcommand{\htmlimage}[1]{}
\newcommand{\htmladdtonavigation}[1]{}

\newenvironment{latexonly}{}{}
\newcommand{\latex}[1]{#1}
\newcommand{\html}[1]{}
\newcommand{\latexhtml}[2]{#1}
\newcommand{\HTMLcode}[2][]{}

%  Starlink cross-references and labels.
\newcommand{\xref}[3]{#1}
\newcommand{\xlabel}[1]{}

%  LaTeX2HTML symbol.
\newcommand{\latextohtml}{\LaTeX2\texttt{HTML}}

%  Define command to re-centre underscore for Latex and leave as normal
%  for HTML (severe problems with \_ in tabbing environments and \_\_
%  generally otherwise).
\renewcommand{\_}{\texttt{\symbol{95}}}

% -----------------------------------------------------------------------------
%  Debugging.
%  =========
%  Remove % on the following to debug links in the HTML version using Latex.

% \newcommand{\hotlink}[2]{\fbox{\begin{tabular}[t]{@{}c@{}}#1\\\hline{\footnotesize #2}\end{tabular}}}
% \renewcommand{\htmladdnormallinkfoot}[2]{\hotlink{#1}{#2}}
% \renewcommand{\htmladdnormallink}[2]{\hotlink{#1}{#2}}
% \renewcommand{\hyperref}[4]{\hotlink{#1}{\S\ref{#4}}}
% \renewcommand{\htmlref}[2]{\hotlink{#1}{\S\ref{#2}}}
% \renewcommand{\xref}[3]{\hotlink{#1}{#2 -- #3}}
%end{latexonly}
% -----------------------------------------------------------------------------
% ? Document specific \newcommand or \newenvironment commands.
% ? End of document specific commands
% -----------------------------------------------------------------------------
%  Title Page.
%  ===========
\renewcommand{\thepage}{\roman{page}}
\begin{document}
\thispagestyle{empty}

%  Latex document header.
%  ======================
\begin{latexonly}
   CCLRC / \textsc{Rutherford Appleton Laboratory} \hfill \textbf{\stardocname}\\
   {\large Particle Physics \& Astronomy Research Council}\\
   {\large Starlink Project\\}
   {\large \stardoccategory\ \stardocnumber}
   \begin{flushright}
   \stardocauthors\\
   \stardocdate
   \end{flushright}
   \vspace{-4mm}
   \rule{\textwidth}{0.5mm}
   \vspace{5mm}
   \begin{center}
   {\Large\textbf{\stardoctitle}}
   \end{center}
   \vspace{5mm}

% ? Heading for abstract if used.
   \vspace{10mm}
   \begin{center}
      {\Large\textbf{Abstract}}
   \end{center}
% ? End of heading for abstract.
\end{latexonly}

%  HTML documentation header.
%  ==========================
\begin{htmlonly}
   \xlabel{}
   \begin{rawhtml} <H1> \end{rawhtml}
      \stardoctitle
   \begin{rawhtml} </H1> <HR> \end{rawhtml}

   \begin{rawhtml} <P> <I> \end{rawhtml}
   \stardoccategory\ \stardocnumber \\
   \stardocauthors \\
   \stardocdate
   \begin{rawhtml} </I> </P> <H3> \end{rawhtml}
      \htmladdnormallink{CCLRC / Rutherford Appleton Laboratory}
                        {http://www.cclrc.ac.uk} \\
      \htmladdnormallink{Particle Physics \& Astronomy Research Council}
                        {http://www.pparc.ac.uk} \\
   \begin{rawhtml} </H3> <H2> \end{rawhtml}
      \htmladdnormallink{Starlink Project}{http://www.starlink.ac.uk/}
   \begin{rawhtml} </H2> \end{rawhtml}
   \htmladdnormallink{\htmladdimg{source.gif} Retrieve hardcopy}
      {http://www.starlink.ac.uk/cgi-bin/hcserver?\stardocsource}\\

%  HTML document table of contents.
%  ================================
%  Add table of contents header and a navigation button to return to this
%  point in the document (this should always go before the abstract \section).
  \label{stardoccontents}
  \begin{rawhtml}
    <HR>
    <H2>Contents</H2>
  \end{rawhtml}
  \htmladdtonavigation{\htmlref{\htmladdimg{contents_motif.gif}}
        {stardoccontents}}

% ? New section for abstract if used.
  \section{\xlabel{abstract}Abstract}
% ? End of new section for abstract

\end{htmlonly}

% -----------------------------------------------------------------------------
% ? Document Abstract. (if used)
%  ==================
\stardocabstract
% ? End of document abstract

% -----------------------------------------------------------------------------
% ? Latex Copyright Statement
%  =========================
\begin{latexonly}
\newpage
\vspace*{\fill}
\stardoccopyright
\end{latexonly}
% ? End of Latex copyright statement

% -----------------------------------------------------------------------------
% ? Latex document Table of Contents (if used).
%  ===========================================
  \newpage
  \begin{latexonly}
    \setlength{\parskip}{0mm}
    \tableofcontents
    \setlength{\parskip}{\medskipamount}
    \markboth{\stardocname}{\stardocname}
  \end{latexonly}
% ? End of Latex document table of contents
% -----------------------------------------------------------------------------
\cleardoublepage
\renewcommand{\thepage}{\arabic{page}}
\setcounter{page}{1}

% ? Main text
\section{Introduction}
ADAM tasks consist of user code wrapped up in the `ADAM fixed part'.
The top level of the user's code is a subroutine with the same name as the
task and a single argument \texttt{STATUS} conforming to
\xref{the normal Starlink inherited status strategy}
{sun104}{inherited_status_checking}.
The ADAM fixed part is provided by the task-building scripts,
\xref{\texttt{alink} and \texttt{ilink}}{sun144}{adam_link_scripts},
and handles the interface with the outside world. The two main parts of the
ADAM fixed part are:
\begin{description}
\item[The control subsystem]
provided by the DTASK and TASK libraries and the ADAM message system,
\xref{AMS}{sun241}{}).
This allows a task to be run directly from the shell, or to
be controlled from a user-interface (or another task) and to control
subsidiary tasks. It makes use of the TASK library which
maintains information about the task in its COMMON blocks and provides user
facilities for writing complete instrumentation systems with control tasks
running subsidiary tasks. For more details on this, see
\xref{ADAM -- Guide to Writing Instrumentation Tasks}{sun134}{}\latex{
(SUN/134)}.
\item[The parameter subsystem] provided by the SUBPAR, PARSECON and LEX
libraries. This allows the task to obtain parameter
values from a variety of sources, to output parameter values, to remember
previous values and to share `GLOBAL' parameter values with other tasks.
 \end{description}
All the subroutine libraries involved are released as the Starlink Software
Item
\xref{PCS}{ssn29}{}.

This document describes the control subsystem and the way it interacts with the
message system and the parameter system.

\section{\label{the_adam_main_routine}The ADAM Main Routine}
When a task is first started, \texttt{MAINTASK} (in \texttt{dtask\_main.f})
finds out if it is being run via the ADAM message system or directly from the
shell, by inspecting the environment variable \texttt{ICL\_TASK\_NAME}.

If \texttt{ICL\_TASK\_NAME} is not set, it is assumed that the task is being
run direct from the shell -- signal handling is disabled and
\texttt{DTASK\_DCLTASK()} is called. The parameter system is activated, the
user's code is run through once and the parameter system de-activated, updating
GLOBAL parameters if required. The program then exits. Any output messages
or parameter prompts will go directly to \texttt{stdout} and replies to
parameter prompts are read from \texttt{stdin} -- the message system will take
no part in the proceedings.
This very simple running of a task is not discussed further in this document.

If \texttt{ICL\_TASK\_NAME} is set, it is assumed that the task is being run
via the ADAM message system -- signal handling is enabled
(\texttt{DTASK\_SETSIG()}) and \texttt{DTASK\_DTASK()} is called to handle
messages.

\section{\label{dtask_applic}DTASK\_APPLIC}
\texttt{DTASK\_APPLIC()} is called by any of the DTASK routines which need to
activate the user's application code.

The code for \texttt{DTASK\_APPLIC} is built into the ADAM task linking
scripts, \texttt{alink} and \texttt{ilink}, and is modified by them to call
the user's top-level subroutine.

\texttt{DTASK\_APPLIC} copies information about the current action into the
TASK common blocks, runs the user's code, runs down the parameter system
if required and retrieves values, such as the action sequence number, the
action value string, the delay before next action entry and the application
request which may have been set by the user's code. These are passed back to
the ADAM fixed part for appropriate action.

The code for \texttt{DTASK\_APPLIC} differs between \texttt{alink} and
\texttt{ilink} in that the process of closing down the parameter system, thus
updating `GLOBAL' parameters and resetting all parameters to the `ground'
state, does not occur with an \texttt{ilink}ed task.
(It may also be prevented with an \texttt{alink}ed task by setting the
environment variable \texttt{ADAM\_TASK\_TYPE} to `I' in the task's process
when the task is run.)

As can be seen from the above, although the user's code will probably differ
markedly in structure between A-tasks, A-task monoliths, I-tasks and control
tasks, as far as the ADAM fixed part is concerned there is very little
difference between them.
(There are further differences in the way the parameter subsystem handles
things. These are triggered by the structure of the task's
\xref{interface module}{sun115}{abstract}.)

\section{\label{receiving_control_messages}Receiving Control Messages}
\texttt{DTASK\_DTASK()} firstly calls \texttt{DTASK\_PRCNAM()} to get the name
by which the task is to register with the ADAM message system.
This will be the name set in
\texttt{ICL\_TASK\_NAME} by the controlling task -- \texttt{DTASK\_PRCNAM()}
does not involve the message system.
The task then registers with this name by calling \texttt{AMS\_INIT()}
(from \texttt{DTASK\_INIT()}).

A loop is then entered, which continues until a bad \texttt{STATUS} is found.
Within the loop most errors will be handled, but an MSP error probably means
that no more messages can be sent or received so the task will exit.

Within the loop:
\begin{itemize}
\item The path for SUBPAR output is first set to 0 -- \textit{i.e.} the task
is not handling a message from another task.
There should be no output but if there is it will go to \texttt{stdout}.
\item \texttt{AMS\_RECEIVE()} with infinite timeout waits for a message --
The message may come from a controlling task or user interface, or may be
generated internally, by a timeout for example.
Certain control messages such as connection requests are handled invisibly
by \texttt{AMS\_RECEIVE()} but it returns when it receives a normal message
(message function \texttt{MESSYS\_\_MESSAGE}), a new transaction is started
and the message status (\texttt{MSGSTATUS}) indicates the type of message.
\item The message is handled. This may involve sending and receiving further
messages as part of the same transaction.
\item The transaction is closed by sending a final acknowledgement with an
appropriate message status.
\end{itemize}

\section{Message Types}
In a normal message (message function MESSYS\_\_MESSAGE) received by an ADAM
task, the message status \texttt{MSGSTATUS} indicates the action required.
\begin{description}
\item[\texttt{MSGSTATUS = SAI\_\_OK}] indicates the message is a `GSOCC'
message, the message context, \texttt{CONTEXT} indicates which one of:
\begin{description}
\item[\texttt{GET}]      Requesting a parameter value from the task.
\item[\texttt{SET}]      Setting a parameter value for the task.
\item[\texttt{OBEY}]     Requesting that an action is obeyed.
\item[\texttt{CANCEL}]   Requesting that an action in a waiting state is
cancelled.
\item[\texttt{CONTROL}]  Requesting that a control function be performed.

Permitted control functions are:
\begin{description}
\item[\texttt{DEFAULT}] to set the task's current working directory.
\item[\texttt{SETENV}] to set the value of an environment variable in the
task's process.
\item[\texttt{PAR\_RESET}] to reset all parameters, or parameters of a named
action in a monolith, back to the `ground' state.
For a discussion of parameter states, see
\xref{SUN/114}{sun114}{parameter_states_}.
\end{description}
\emph{N.B. \texttt{DEFAULT} and \texttt{SETENV} are required because tasks are
run in separate processes, therefore (under Unix anyway) changing the current
working directory or environment variables of the controlling process will
not alter them for any existing subsidiary tasks.}
\end{description}
See
\htmlref{Handling GSOCC Messages}{handling_gsocc_messages}\latex{ (Section
\ref{handling_gsocc_messages})}
for more details.
\item[\texttt{MSGSTATUS} not \texttt{SAI\_\_OK}] may be:
\begin{description}
\item[\texttt{MESSYS\_\_EXTINT}] This should not happen in a task -- the
facility is for use by user interfaces.
An error report is made and the event ignored.
\item[\texttt{MESSYS\_\_RESCHED}]   A timer has expired in this task and
\texttt{AMS\_REMSG()} has been called. \texttt{DTASK\_TIMEOUT()} is called
to handle the message. It checks that the specified action is active and that
the reschedule was expected. If all is OK, the action is obeyed
(\texttt{DTASK\_OBEY()}).
\item[\texttt{MESSYS\_\_ASTINT}]  An AST has fired in this task and
\texttt{AMS\_ASTINT()} or \texttt{AMS\_ASTMSG()} has been called.
\texttt{DTASK\_ASTINT()} checks the action and obeys it if OK
(\texttt{DTASK\_OBEY()}).
\item[\texttt{MESSYS\_\_KICK}] Subroutine \texttt{AMS\_KICK()} has been
called by this task to run one of its actions. \texttt{DTASK\_KICK()} checks
the action and obeys it if OK (\texttt{DTASK\_OBEY()}).
\item[\texttt{Other > 0}]   May be from a subsidiary task.
See
\htmlref{Messages from Subsidiary Tasks}
{messages_from_subsidiary_tasks}\latex{ (Section~\ref{messages_from_subsidiary_tasks})} for more details.
\item[\texttt{MSGSTATUS < 0}]  An MSP error, set STATUS to terminate loop.
\end{description}
\end{description}

\section{\label{handling_gsocc_messages}Handling GSOCC Messages}
These are handled by \texttt{DTASK\_GSOC()} in the first instance. In addition
to the transaction identification, \texttt{PATH} and \texttt{MESSID}, the
\texttt{CONTEXT}, \texttt{NAME} and \texttt{VALUE} components of the received
message are passed.

Firstly the \texttt{PATH} and \texttt{MESSID} from the message are set for
SUBPAR (\texttt{SUBPAR\_PUTPATH()}) in case it wants to reply with a message or
request a parameter value as a result of this message, then a DTASK routine
appropriate to the \texttt{CONTEXT} is called.

When the required action for the \texttt{CONTEXT} is complete, the transaction
is ended by calling (\texttt{DTASK\_COMSHUT()}) which flushes any error
messages on the stack to the master task (using
\xref{\texttt{ERR\_CLEAR()}}{ssn4}{ERR_CLEAR})
before sending a final acknowledgment.
(The master task must therefore be prepared to handle any
\texttt{MESSYS\_\_INFORM} messages prior to the final acknowledgement in any
\texttt{CONTEXT}.)
The final acknowledgement will have an appropriate message status and the
message value will depend upon the status and the \texttt{CONTEXT}.

\subsection{\label{get_context}\texttt{GET} Context}
\texttt{GET} is handled by \texttt{DTASK\_GET()}.
\texttt{NAME} contains the name of the required parameter. If the task is a
\xref{monolith}{sun144}{introduction}, the parameter name must be of the form
\texttt{ACTION:PARNAME} to specify the particular action within the monolith.

The value of the named parameter is obtained, by searching the
\xref{vpath}{sun115}{the_vpath_field}
if the parameter is not already active.

Note that there are some restrictions on the vpath search:
\begin{itemize}
\item Dynamic will usually have no effect because any
dynamic defaults will be cancelled initially and if the parameter was reset.
\item Prompting is disabled so any attempt to prompt will return status
\texttt{PAR\_\_NOUSR}.
\end{itemize}

The transaction is then closed (\texttt{DTASK\_COMSHUT()}). Assuming the
status is \texttt{SAI\_\_OK}, the message value of the final acknowledgement is
the parameter value as a character string.  In the event of an error, error
messages are reported and the message status set appropriately.

\subsection{\label{set_context}\texttt{SET} Context}
\texttt{SET} is handled by \texttt{DTASK\_SET()}.
\texttt{NAME} contains the name of the required parameter as for
\htmlref{\texttt{GET}}{get_context}.
\texttt{VALUE} is a character string which the parameter subsystem can resolve
into a value for the specified parameter.
The given value is put into the parameter's internal storage and the parameter
is made active (\texttt{SUBPAR\_CMDPAR()}).
The transaction is then closed (\texttt{DTASK\_COMSHUT()}). If an error
occurred, error messages will be reported and an appropriate status set in
the final acknowledgement.

\subsection{\label{obey_and_cancel_context}
\texttt{OBEY} and \texttt{CANCEL} Context}
\texttt{OBEY} and \texttt{CANCEL} are further checked in
\texttt{DTASK\_GSOC()}.

\texttt{NAME} contains the name of the required action (or task within a
monolith) and \texttt{VALUE} is a string specifying parameter
values in a format acceptable to the parameter system.

For \texttt{OBEY} the action must not be active and for \texttt{CANCEL} the
action must be active; otherwise the transaction is terminated (in
\texttt{DTASK\_COMSHUT()}) by calling \texttt{AMS\_REPLY()} with a suitable
bad message status.

If the \texttt{OBEY}/\texttt{CANCEL} is valid, \texttt{SUBPAR\_FINDACT()}
is called to set up the parameter system for the specified action and the
\texttt{VALUE} string is processed (\texttt{SUBPAR\_CMDLINE()}) to set the
value of any given parameters.

\subsubsection{\texttt{OBEY}}
The new action is added to the active list and the list of
subsidiary tasks for it is cleared. \texttt{AMS\_REPLY()} is called with
message status \texttt{DTASK\_\_ACTSTART} to send an initial acknowledgement
of the \texttt{OBEY} message .

If that fails the action and/or transaction are
closed (\texttt{DTASK\_ACTSHUT()} or \texttt{DTASK\_COMSHUT()}); if it is
OK, \texttt{DTASK\_OBEY()} is called.

\texttt{DTASK\_OBEY()} calls \texttt{DTASK\_APPLIC()} with context
\texttt{OBEY} and that in turn calls the user's top-level routine.

On return from \texttt{DTASK\_APPLIC()} if \texttt{STATUS} is not
\texttt{SAI\_\_OK}, the message status, \texttt{MESSTATUS}, for the final
acknowledgement message is set to the bad status value;
if \texttt{STATUS} is
\texttt{SAI\_\_OK}, the \texttt{REQUEST} returned from the application is
checked
(\texttt{DTASK\_ACT\_SCHED()}) to see whether the action is to be terminated
or is to be
\htmlref{rescheduled}{rescheduling}.
If the user's code has not changed it, the  \texttt{REQUEST} will be
\texttt{ACT\_\_END}.

If a reschedule is requested, \texttt{DTASK\_ACT\_SCHED()} will set it up;
otherwise the action is ended and the communications transaction closed by
sending a reply to the task which initiated the \texttt{OBEY}.
\texttt{MESSTATUS} for the reply depends on the \texttt{STATUS} and
\texttt{REQUEST} returned from \texttt{DTASK\_APPLIC()}.

If a reschedule was not requested, \texttt{MESSTATUS} will depend upon the
value of \texttt{REQUEST} as
follows:
\begin{center}
\texttt{
\begin{tabular}{|l|l|} \hline
REQUEST & MESSTATUS \\
\hline
ACT\_\_END & DTASK\_\_ACTCOMPLETE \\
ACT\_\_UNIMP & DTASK\_\_UNIMP \\
ACT\_\_INFORM & DTASK\_\_ACTINFORM \\
SAI\_\_OK & DTASK\_\_IVACTSTAT \\
DTASK\_\_SYSNORM & DTASK\_\_IVACTSTAT \\
ACT\_\_CANCEL & DTASK\_\_IVACTSTAT \\
\textnormal{Other}& REQUEST \\
\hline
\end{tabular}
}
\end{center}
Error messages may also be reported.

\subsubsection{\texttt{CANCEL}}
\texttt{CANCEL} is handled by calling \texttt{DTASK\_CANCEL()}.

Details of the OBEY transaction which started this action are obtained from
the COMMON blocks (\texttt{DTASK\_CMN}) then \texttt{DTASK\_APPLIC()} is called
with context \texttt{CANCEL} and that in turn calls the user's code.
See
\xref{SUN/134}{sun134}{}
for examples of how the user's code might handle a
\texttt{CANCEL}

On return from \texttt{DTASK\_APPLIC} the status and \texttt{REQUEST} returned
from the application are checked (\texttt{DTASK\_ACT\_SCHED()}) to see whether
the action is to be terminated or is to continue rescheduling.
In any case, an acknowledgement is sent to the task which requested the
\texttt{CANCEL}.
If the action has ended, an acknowledgement is also sent to the task which
issued the \texttt{OBEY}.

\subsection{\label{control_context}\texttt{CONTROL} Context}
\texttt{CONTROL} is handled by \texttt{DTASK\_CONTROL()}.
The message name component, passed as \texttt{ACTION} defines the particular
control function required, \texttt{'DEFAULT'}, \texttt{'SETENV'} or
\texttt{'PAR\_RESET'}.
The required action is performed.

If a problem occurs in the process, an appropriate bad status value is set
and error report is made.

The transaction is then closed (\texttt{DTASK\_COMSHUT()}).

\subsubsection{\texttt{DEFAULT}}
The message value specifies a new `current working directory' required for the
task. If the value is blank, no change is required; otherwise the change is
made by calling the appropriate system routine.

Whether or not a change was made, the `new' current working directory is
obtained and returned as the message value in final reply message.
\texttt{CONTROL DEFAULT} can therefore be used to enquire the task's current
working directory without changing it.

\subsubsection{\texttt{SETENV}}
The message value specifies a new value for an environment variable for the
task. It has the form: `\texttt{VARIABLE = NEWVALUE}', where \texttt{VARIABLE}
is the name of the environment variable and \texttt{NEWVALUE} is the required
value. The string may be quoted -- quotes will be stripped in the usual way.
If the string is unquoted, leading blanks will also be stripped from
\texttt{NEWVALUE}.

The environment variable is set by calling the appropriate system routine.

\subsubsection{\texttt{PAR\_RESET}}
This causes the task's parameters to be reset to the `ground' state.
The message value may be used to specify the name of an individual task in a
monolith; if it doesn't, parameters for all tasks in the monolith are reset.
Resetting is done by calling \texttt{SUBPAR\_DEACT()} with argument
\texttt{TTYPE} set to `R' -- MIN, MAX and dynamic default values are also
cancelled but GLOBAL values are not updated.

\section{\label{messages_from_subsidiary_tasks}Messages from Subsidiary Tasks}
\texttt{DTASK\_SUBSID()} is called to check it. Given the \texttt{PATH} and
\texttt{MESSID} of the message, the Active Subsidiary Task Action Block (see
the TASK library) is searched and the associated action in this task found (or
an error).
Communications to the parent task for the associated action are set for
SUBPAR (\texttt{SUBPAR\_PUTPATH()}), then the particular message is handled
depending upon the \texttt{MSGSTATUS}.
\begin{description}
\item[\texttt{MESSYS\_\_INFORM}] is relayed on to this task's parent (using
\texttt{SUBPAR\_WRITE()}).
\item[\texttt{MESSYS\_\_PARAMREQ}] is handled by \texttt{TASK\_ASKPARAM()}.
It relays the parameter request on to this task's parent (using
\texttt{SUBPAR\_REQUEST()} which waits for the \texttt{MESSYS\_\_PARAMREP}
reply and returns the value).
\texttt{TASK\_ASKPARAM()} then returns the value in a
\texttt{MESSYS\_\_PARAMEREP} to the subsidiary task.
\item[\texttt{MESSYS\_\_SYNC}] causes this task to
\htmlref{synchronise}{synchronisation}\latex{ (see
Section~\ref{synchronisation})}
(in \texttt{SUBPAR\_SYNC()}) with its parent and then acknowledge the original
message with a \texttt{MESSYS\_\_SYNCREP} to the subsidiary task (using
\texttt{AMS\_REPLY()}). See

\end{description}
The above are said to be `transparent' messages. If the message is not
one of these, the message information is put (\texttt{TASK\_PUT\_MESSINFO()})
into the Current D-task Action Block in case the associated action requires it
later.

If there is a failure in any of the above attempts to relay messages, the
subsidiary task is removed from the Active Subsidiary Task Action Block
and the action in this task obeyed (\texttt{DTASK\_OBEY()}) again.
\begin{description}
\item[\texttt{MESSYS\_\_TRIGGER}] causes the associated action in this task to
be obeyed (\texttt{DTASK\_OBEY()}) as if it were initiated by its parent.
\item[Other \texttt{MSGSTATUS > 0}] is assumed to mean that action is complete
in the subsidiary task -- the subsidiary task action is cleared from the
Active Subsidiary Task Action Block and, again \texttt{DTASK\_OBEY()} is
called.
Failure to relay and other terminations are therefore treated the same.
\end{description}

\section{\label{rescheduling}Re-scheduling}
An action being obeyed or cancelled can request that it is re-entered
(re-scheduled) upon some future event by calling \texttt{TASK\_PUT\_REQUEST()}
and optionally \texttt{TASK\_PUT\_DELAY()}. See
\xref{SUN/134}{sun134}{}
for details.

Any values of the request and delay which have been set are returned to the
calling routine (\texttt{DTASK\_OBEY} or \texttt{DTASK\_CANCEL} by
\texttt{DTASK\_APPLIC} and \texttt{DTASK\_ACT\_SCHED} is called to check if
a reschedule is required and set it up if required.

Reschedule requests may be:
\begin{description}
\item[\texttt{ACT\_\_STAGE}] Re-schedule immediately (in fact after 10ms).
\item[\texttt{ACT\_\_WAIT}] Re-schedule after \texttt{DELAY} ms.
\item[\texttt{ACT\_\_ASTINT}] Re-schedule on AST (with timeout of
\texttt{DELAY}ms).
\item[\texttt{ACT\_\_MESSAGE}] Re-schedule on completion of an action in a
subsidiary task (with timeout of \texttt{DELAY}ms).
\end{description}
For \texttt{ACT\_\_ASTINT} and \texttt{ACT\_\_MESSAGE}, \texttt{DELAY} may be
\texttt{MESSYS\_\_INFINITE} indicating that no timer should be set.

As each reschedule is set, the action sequence number, initially 0, is
incremented.

Where a timeout is required, an ATIMER timer is set by:
\begin{quote} \begin{verbatim}
CALL DTASK_RESCHED( ACTPTR, ACTCNT, SCHEDTIME, STATUS )
\end{verbatim} \end{quote}

\texttt{ACTPTR} specifies the action to be rescheduled, and \texttt{ACTCT}
the timer identifier.
The timer is set to call \texttt{DTASK\_CHDLR()} after \texttt{SCHEDTIME}
millisecs. \texttt{DTASK\_CHDLR()} is written in C and is a straight-through
call to \texttt{DTASK\_ASTHDLR()} which calls \texttt{AMS\_RESMSG()} to
send a reschedule message (message status \texttt{MESSYS\_\_RESCHED}) to this
task.

When \texttt{DTASK\_DTASK()} receives the  reschedule message, it calls
\texttt{DTASK\_TIMEOUT()} which de-constructs the passed value into an action
pointer and count, and checks that the specified action reschedule was
expected.

If that checks OK, the action is obeyed (\texttt{DTASK\_OBEY()}); if not, it is
ignored.

\section{\label{synchronisation}Synchronisation}
If the task needs to output directly to the user's terminal, to switch it to
`graphics' mode for example, it is necessary to ensure that the user-interface
has completed output of previously sent text messages before the graphics
output is sent. This is achieved by the user's code calling
\xref{\texttt{MSG\_SYNC()}}{sun104}{MSG_SYNC}
which sends a \texttt{MESSYS\_SYNC} message to the master task and waits for a
\texttt{MESSYS\_SYNCREP} message to be returned. As all messages are queued by
the ADAM message system, we can be sure that all earlier message have been
handled by the time the task receives the reply. \emph{Note that this only
works where no other tasks can send messages to the user-interface in the
meantime.}

Where a control task lies between a subsidiary task and the user-interface,
the messages are simply relayed.

\section{Use of the TASK Library}
The TASK library maintains two lists in COMMON: the Current D-task Action Block
and the Active Subsidiary Task Action Block.
It also maintains the AST interrupt flag, indicating if
\texttt{TASK\_ASTSIGNAL()} has been called.

The Current D-task Action Block holds details of the current D-task action,
\textit{i.e.} the one on whose behalf the current call to ACT has been made.
Items held are:
\begin{itemize}
\item Action pointer for current action
\item Message path resulting in current entry
\item Message id resulting in current entry
\item Context in message resulting in entry
\item Status in message resulting in entry
\item Context for current action
\item Name code for current action
\item Sequence number for current action
\item Delay before next entry
\item Request for rescheduling
\item Current action name
\item Action name in message
\item Value string for current action
\end{itemize}

The Active Subsidiary Task Action Block holds details of active actions that
have been initiated in subsidiary tasks. If an incoming message corresponds to
one of the actions described in this common block, it will result in an entry
to ACT. Items held are (for each subsidiary action):
\begin{itemize}
\item Action pointer for initiating action (in this task)
\item Message path for initiated action
\item Message id for initiated action
\end{itemize}
\vfill
\appendix
\section{Typical A-task Message Sequence}
\begin{center}
\begin{tabular}{|p{65mm}|c|p{75mm}|}
\hline
User Interface & & Task\\
\hline
Register with message system & &\\
Load task & &\\
& & Register with message system\\
& & Await message\\
Send connection request & $>$ &\\
& $<$ & Acknowledge connection request (automatically)\\
& & Wait for next message\\
Send GSOCC OBEY & $>$ &\\
& $<$ & Acknowledge GSOC OBEY \\
& & Obey user's code \\
\hline
\end{tabular}

The user's code may, potentially, generate any number of:

\begin{tabular}{|p{65mm}|c|p{75mm}|}
\hline
& $<$ & Reply with \texttt{MESSYS\_\_INFORM}
(output from \texttt{MSG\_OUT()}, \texttt{ERR\_FLUSH()} \textit{etc.})\\
Display message value to user & &\\
\hline
\end{tabular}

or:

\begin{tabular}{|p{65mm}|c|p{75mm}|}
\hline
& $<$ & Reply with \texttt{MESSYS\_\_PARAMREQ}
(Request a prompt for a parameter value)\\
Prompt User & & \\
Reply with \texttt{MESSYS\_\_PARAMREP} & $>$ & \\
& & Set parameter value\\
\hline
\end{tabular}

or:

\begin{tabular}{|p{65mm}|c|p{75mm}|}
\hline
& $<$ & Reply with \texttt{MESSYS\_\_SYNC} (Output from \texttt{MSG\_SYNC})\\
Prompt user & & \\
Reply with \texttt{MESSYS\_\_SYNCREP} &$>$ & \\
\hline
\end{tabular}

and finally, from the ADAM fixed part:

\begin{tabular}{|p{65mm}|c|p{75mm}|}
\hline
& $<$ & Reply with terminating message status
(\texttt{DTASK\_\_ACTCOMPLETE} is good)\\
Terminate transaction (automatically) & & \\
Display final status if bad. & & \\
\hline
\end{tabular}
\end{center}
% ? End of main text
\end{document}
