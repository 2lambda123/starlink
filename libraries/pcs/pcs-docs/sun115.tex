\documentclass[twoside,11pt]{article}

% ? Specify used packages
% \usepackage{graphicx}        %  Use this one for final production.
% \usepackage[draft]{graphicx} %  Use this one for drafting.
% ? End of specify used packages

\pagestyle{myheadings}

% -----------------------------------------------------------------------------
% ? Document identification
% Fixed part
\newcommand{\stardoccategory}  {Starlink User Note}
\newcommand{\stardocinitials}  {SUN}
\newcommand{\stardocsource}    {sun\stardocnumber}
\newcommand{\stardoccopyright} 
{Copyright \copyright\ 2002 Council for the Central Laboratory of the Research Councils}

% Variable part - replace [xxx] as appropriate.
\newcommand{\stardocnumber}    {115.3}
\newcommand{\stardocauthors}   {A J Chipperfield, B D Kelly, S L Wright}
\newcommand{\stardocdate}      {3 September 2002}
\newcommand{\stardoctitle}     {ADAM\\[2ex]
                                Interface Module Reference Manual}
\newcommand{\stardocversion}   {}
\newcommand{\stardocmanual}    {Programmer's Manual}
\newcommand{\stardocabstract}  {
ADAM Interface Modules provide an interface between ADAM application programs 
and the rest of the system.
\\
This document describes in detail the facilities available with ADAM Interface
Modules and the rules for using them. It is intended as a reference manual and
should shed light on some of the finer points of the ADAM parameter system.
Readers requiring an introduction to Interface Modules should read
\xref{SG/4}{sg4}{}.
}
% ? End of document identification

% -----------------------------------------------------------------------------

% +
%  Name:
%     sun.tex
%
%  Purpose:
%     Template for Starlink User Note (SUN) documents.
%     Refer to SUN/199
%
%  Authors:
%     AJC: A.J.Chipperfield (Starlink, RAL)
%     BLY: M.J.Bly (Starlink, RAL)
%     PWD: Peter W. Draper (Starlink, Durham University)
%
%  History:
%     17-JAN-1996 (AJC):
%        Original with hypertext macros, based on MDL plain originals.
%     16-JUN-1997 (BLY):
%        Adapted for LaTeX2e.
%        Added picture commands.
%     13-AUG-1998 (PWD):
%        Converted for use with LaTeX2HTML version 98.2 and
%        Star2HTML version 1.3.
%      1-FEB-2000 (AJC):
%        Add Copyright statement in LaTeX
%     {Add further history here}
%
% -

\newcommand{\stardocname}{\stardocinitials /\stardocnumber}
\markboth{\stardocname}{\stardocname}
\setlength{\textwidth}{160mm}
\setlength{\textheight}{230mm}
\setlength{\topmargin}{-2mm}
\setlength{\oddsidemargin}{0mm}
\setlength{\evensidemargin}{0mm}
\setlength{\parindent}{0mm}
\setlength{\parskip}{\medskipamount}
\setlength{\unitlength}{1mm}

% -----------------------------------------------------------------------------
%  Hypertext definitions.
%  ======================
%  These are used by the LaTeX2HTML translator in conjunction with star2html.

%  Comment.sty: version 2.0, 19 June 1992
%  Selectively in/exclude pieces of text.
%
%  Author
%    Victor Eijkhout                                      <eijkhout@cs.utk.edu>
%    Department of Computer Science
%    University Tennessee at Knoxville
%    104 Ayres Hall
%    Knoxville, TN 37996
%    USA

%  Do not remove the %begin{latexonly} and %end{latexonly} lines (used by 
%  LaTeX2HTML to signify text it shouldn't process).
%begin{latexonly}
\makeatletter
\def\makeinnocent#1{\catcode`#1=12 }
\def\csarg#1#2{\expandafter#1\csname#2\endcsname}

\def\ThrowAwayComment#1{\begingroup
    \def\CurrentComment{#1}%
    \let\do\makeinnocent \dospecials
    \makeinnocent\^^L% and whatever other special cases
    \endlinechar`\^^M \catcode`\^^M=12 \xComment}
{\catcode`\^^M=12 \endlinechar=-1 %
 \gdef\xComment#1^^M{\def\test{#1}
      \csarg\ifx{PlainEnd\CurrentComment Test}\test
          \let\html@next\endgroup
      \else \csarg\ifx{LaLaEnd\CurrentComment Test}\test
            \edef\html@next{\endgroup\noexpand\end{\CurrentComment}}
      \else \let\html@next\xComment
      \fi \fi \html@next}
}
\makeatother

\def\includecomment
 #1{\expandafter\def\csname#1\endcsname{}%
    \expandafter\def\csname end#1\endcsname{}}
\def\excludecomment
 #1{\expandafter\def\csname#1\endcsname{\ThrowAwayComment{#1}}%
    {\escapechar=-1\relax
     \csarg\xdef{PlainEnd#1Test}{\string\\end#1}%
     \csarg\xdef{LaLaEnd#1Test}{\string\\end\string\{#1\string\}}%
    }}

%  Define environments that ignore their contents.
\excludecomment{comment}
\excludecomment{rawhtml}
\excludecomment{htmlonly}

%  Hypertext commands etc. This is a condensed version of the html.sty
%  file supplied with LaTeX2HTML by: Nikos Drakos <nikos@cbl.leeds.ac.uk> &
%  Jelle van Zeijl <jvzeijl@isou17.estec.esa.nl>. The LaTeX2HTML documentation
%  should be consulted about all commands (and the environments defined above)
%  except \xref and \xlabel which are Starlink specific.

\newcommand{\htmladdnormallinkfoot}[2]{#1\footnote{#2}}
\newcommand{\htmladdnormallink}[2]{#1}
\newcommand{\htmladdimg}[1]{}
\newcommand{\hyperref}[4]{#2\ref{#4}#3}
\newcommand{\htmlref}[2]{#1}
\newcommand{\htmlimage}[1]{}
\newcommand{\htmladdtonavigation}[1]{}

\newenvironment{latexonly}{}{}
\newcommand{\latex}[1]{#1}
\newcommand{\html}[1]{}
\newcommand{\latexhtml}[2]{#1}
\newcommand{\HTMLcode}[2][]{}

%  Starlink cross-references and labels.
\newcommand{\xref}[3]{#1}
\newcommand{\xlabel}[1]{}

%  LaTeX2HTML symbol.
\newcommand{\latextohtml}{\LaTeX2\texttt{HTML}}

%  Define command to re-centre underscore for Latex and leave as normal
%  for HTML (severe problems with \_ in tabbing environments and \_\_
%  generally otherwise).
\renewcommand{\_}{\texttt{\symbol{95}}}

% -----------------------------------------------------------------------------
%  Debugging.
%  =========
%  Remove % on the following to debug links in the HTML version using Latex.

% \newcommand{\hotlink}[2]{\fbox{\begin{tabular}[t]{@{}c@{}}#1\\\hline{\footnotesize #2}\end{tabular}}}
% \renewcommand{\htmladdnormallinkfoot}[2]{\hotlink{#1}{#2}}
% \renewcommand{\htmladdnormallink}[2]{\hotlink{#1}{#2}}
% \renewcommand{\hyperref}[4]{\hotlink{#1}{\S\ref{#4}}}
% \renewcommand{\htmlref}[2]{\hotlink{#1}{\S\ref{#2}}}
% \renewcommand{\xref}[3]{\hotlink{#1}{#2 -- #3}}
%end{latexonly}
% -----------------------------------------------------------------------------
% ? Document specific \newcommand or \newenvironment commands.
% ? End of document specific commands
% -----------------------------------------------------------------------------
%  Title Page.
%  ===========
\renewcommand{\thepage}{\roman{page}}
\begin{document}
\thispagestyle{empty}

%  Latex document header.
%  ======================
\begin{latexonly}
   CCLRC / \textsc{Rutherford Appleton Laboratory} \hfill \textbf{\stardocname}\\
   {\large Particle Physics \& Astronomy Research Council}\\
   {\large Starlink Project\\}
   {\large \stardoccategory\ \stardocnumber}
   \begin{flushright}
   \stardocauthors\\
   \stardocdate
   \end{flushright}
   \vspace{-4mm}
   \rule{\textwidth}{0.5mm}
   \vspace{5mm}
   \begin{center}
   {\Huge\textbf{\stardoctitle \\ [2.5ex]}}
   {\LARGE\textbf{\stardocversion \\ [4ex]}}
   {\Huge\textbf{\stardocmanual}}
   \end{center}
   \vspace{5mm}

% ? Add picture here if required for the LaTeX version.
%   e.g. \includegraphics[scale=0.3]{filename.ps}
% ? End of picture

% ? Heading for abstract if used.
   \vspace{10mm}
   \begin{center}
      {\Large\textbf{Abstract}}
   \end{center}
% ? End of heading for abstract.
\end{latexonly}

%  HTML documentation header.
%  ==========================
\begin{htmlonly}
   \xlabel{}
   \begin{rawhtml} <H1> \end{rawhtml}
      \stardoctitle\\
      \stardocversion\\
      \stardocmanual
   \begin{rawhtml} </H1> <HR> \end{rawhtml}

% ? Add picture here if required for the hypertext version.
%   e.g. \includegraphics[scale=0.7]{filename.ps}
% ? End of picture

   \begin{rawhtml} <P> <I> \end{rawhtml}
   \stardoccategory\ \stardocnumber \\
   \stardocauthors \\
   \stardocdate
   \begin{rawhtml} </I> </P> <H3> \end{rawhtml}
      \htmladdnormallink{CCLRC / Rutherford Appleton Laboratory}
                        {http://www.cclrc.ac.uk} \\
      \htmladdnormallink{Particle Physics \& Astronomy Research Council}
                        {http://www.pparc.ac.uk} \\
   \begin{rawhtml} </H3> <H2> \end{rawhtml}
      \htmladdnormallink{Starlink Project}{http://www.starlink.ac.uk/}
   \begin{rawhtml} </H2> \end{rawhtml}
   \htmladdnormallink{\htmladdimg{source.gif} Retrieve hardcopy}
      {http://www.starlink.ac.uk/cgi-bin/hcserver?\stardocsource}\\

%  HTML document table of contents. 
%  ================================
%  Add table of contents header and a navigation button to return to this 
%  point in the document (this should always go before the abstract \section). 
  \label{stardoccontents}
  \begin{rawhtml} 
    <HR>
    <H2>Contents</H2>
  \end{rawhtml}
  \htmladdtonavigation{\htmlref{\htmladdimg{contents_motif.gif}}
        {stardoccontents}}

% ? New section for abstract if used.
  \section{\xlabel{abstract}Abstract}
% ? End of new section for abstract
\end{htmlonly}

% -----------------------------------------------------------------------------
% ? Document Abstract. (if used)
%  ==================
\stardocabstract
% ? End of document abstract

% -----------------------------------------------------------------------------
% ? Latex Copyright Statement
%  =========================
\begin{latexonly}
\newpage
\vspace*{\fill}
\stardoccopyright
\end{latexonly}
% ? End of Latex copyright statement

% -----------------------------------------------------------------------------
% ? Latex document Table of Contents (if used).
%  ===========================================
  \newpage
  \begin{latexonly}
    \setlength{\parskip}{0mm}
    \tableofcontents
    \setlength{\parskip}{\medskipamount}
    \markboth{\stardocname}{\stardocname}
  \end{latexonly}
% ? End of Latex document table of contents
% -----------------------------------------------------------------------------

\cleardoublepage
\renewcommand{\thepage}{\arabic{page}}
\setcounter{page}{1}

% ? Main text

\section{Introduction\xlabel{introduction}}
This document gives a description of Interface Modules as used by
application programs (tasks) running under ADAM with the parameter system
based on the SUBPAR library.
It describes Interface Modules and some related aspects of the parameter 
system in fine detail and is intended as a reference document -- 
users requiring an introduction to the parameter system and Interface Modules 
should read \xref{SG/4}{sg4}{} first.

For the purpose of the examples it is assumed that a user interface
similar to ICL, the Interactive Command Language, is being used.
Note that the actual use of keywords, prompts, {\em etc.}\ is 
determined by the user interface, which may or may not support the
various features. 

Useful associated documents are:
\begin{description}
\item[\xref{SG/4}{sg4}{}] ADAM --- The Starlink Software Environment
\item[\xref{SUN/134}{sun134}{}] ADAM --- How to Write Instrumentation Tasks
\item[\xref{SUN/92}{sun92}{}] HDS --- Hierarchical Data System
\item[\xref{SG/5}{sg5}{}] ICL --- The Interactive Command Language
\item[\xref{SUN/104}{sun104}{}] MSG and ERR --- Message and Error Reporting
Systems
\end{description}
This document applies to Unix ADAM Version 2.0 (VMS ADAM Version 2.1) and 
later.

\section{The Interface Module\xlabel{the_interface_module}}

\subsection{Description\xlabel{description}}

The {\em Interface Module}\/ is a collection of information which acts as a
go-between for the application program on one hand, and the rest of the
system on the other. It acts as an interface (hence the name) between
the program and system so that neither has to know any details about
the working of the other.This means that either
side of the interface could be completely changed, without having to
change anything on the other side.
Only the Interface Module itself may have to be changed. 

The main functions of the Interface Module are to:
\begin{itemize}
\item Specify the program's parameters and define a search path for
parameter values.
\item Specify actions within instrumentation tasks.
\item Specify the format of the command line for the program.
\item Specify the messages and prompts which the user sees.
\end{itemize}

\subsection{Different Forms of an Interface Module -- Compilation
\xlabel{different_forms_of_an_interface_module}}

An Interface Module can exist in two forms. When an application
program is being developed, the information making up the Interface
Module is normally put into an ordinary text file, where it can be
easily inspected and edited.
The file should have the same name as the program, but with a file 
extension of {\bf .ifl}.

When the system interface for a program has become relatively stable, the 
Interface Module can be saved in compiled form --
this leads to a more efficient running of the system. 
The compiled form has the same name as the source file, but a file extension of
{\bf .ifc}.

Strictly speaking, the source file is the {\em Interface File} although that 
term is often used to mean either source or compiled forms of the
Interface Module.

The program {\bf compifl} is provided to compile an Interface File. 
Perform the usual startup procedure for ADAM program development 
then type:
\begin{quote} \begin{verbatim}
compifl filename
\end{verbatim} \end{quote}
in response to the command language prompt.
\texttt{filename} may optionally include the \texttt{.ifl} extension.
\texttt{filename.ifl} will be compiled into \texttt{filename.ifc}.

\subsection{Interface Module Search Path
\xlabel{interface_module_search_path}}

When an ADAM program is first loaded, it searches for the required
Interface Module, reads it and uses the information therein to set up the
common blocks used by the parameter system to control its behaviour.
The program will accept either source or compiled form of the Interface Module.

By default, the program will look for its Interface Module in the directory
where the executable file was found. It looks firstly for a compiled file,
\texttt{name.ifc} (where {\em name} is the name of the program's executable 
file); if that is not found, it looks for a source file, \texttt{name.ifl} 

This behaviour can be altered by defining a list of directories in which
the program will search before using the default.
The process is fully described in Appendix \ref{searchpath}.

\subsection{Basic Structure of the Interface File
\xlabel{basic_structure_of_the_interface_file}}

\subsubsection{Single Program}

An Interface File for a single program consists of a number of {\em
Specification Fields} contained between INTERFACE and ENDINTERFACE
statements.
There are four possible types of
Specification Field under ADAM\footnote{Excluding the unused Program 
Specification ref. Appendix \ref{program}}, the {\em Parameter
Specification}\/, the {\em Help Library Specification}\/, the {\em Message 
Specification}\/ and the {\em Action Specification}.
They are all optional (depending upon the application).

Specification Fields may consist of a number of subfields as shown in the
example below.
{\em Examples in this document will use `.' to represent possible lines not 
shown.}
\begin{quote} \begin{verbatim}
INTERFACE programname
   HELPLIB     HELP_TOP_LEVEL_SPEC
        .
   PARAMETER   parametername_1
      parfieldname   fieldvalue(s)
            .
   ENDPARAMETER
   PARAMETER   parametername_2
            .
   ENDPARAMETER
         .
   ACTION      actionname
      actfieldname   fieldvalue(s)
            .
   ENDACTION
         .
   MESSAGE     messagename
      TEXT     'string'
   ENDMESSAGE
      .
ENDINTERFACE
\end{verbatim} \end{quote}
\texttt{programname} is a character string giving the name of the Interface 
Module. 
It must be the same as the name of the program and unique in any given ADAM 
system.


\subsubsection{Monoliths}
For efficiency, a group of related programs can all be 
linked into a single executable file called a {\em monolith} (see
\xref{SG/4}{sg4}{}).
If the monolith is to be run from ICL, the individual Interface Files must 
be concatenated into a single file for the whole monolith and the monolithic 
Interface File must have the same filename as the executable file.
The structure of a monolithic Interface File is as follows:
\begin{quote} \begin{verbatim}
MONOLITH monname
   INTERFACE programname_1
      .
      .
   ENDINTERFACE
   INTERFACE programname_2
      .
      .
   ENDINTERFACE
ENDMONOLITH
\end{verbatim} \end{quote}
where \texttt{programname\_1}, \texttt{programname\_2}, {\em etc.}\ are the names of 
the individual programs. They must comply with the rules for action names
(see Appendix \ref{names}).

When monoliths are run directly from a Unix shell, the individual program
Interface Modules are used.

\subsubsection{Syntax}
Details of Interface File parsing are given in Appendix \ref{parsing}, but the
following points should be sufficient for now.
\begin{itemize}
\item The file is interpreted as a sequence of {\em tokens}, usually separated
by comma, space or newline.
\item Tokens may be:
\begin{itemize}
\item reserved words
\item constants (number, character, logical or !)
\item names ({\em i.e.}\ everything else).
\end {itemize}
\item A character constant is a string of characters enclosed in single quotes.
Where a token may be a character constant, a name ({\em i.e.}\ a 
single-token, unquoted string) or even a reserved word is usually also allowed.
The term {\em character string} is used in this document where this situation
pertains.
An exception to this is where the token represents a parameter value (a
default value, for example) -- then character constants must be enclosed in 
quotes so that names can be distinguished from strings.
\item Case is generally not significant.
\item Extra white space is ignored (apart from within quoted 
strings) and it is recommended that blank lines and indentation are used for 
clarity.
\item Comments may be introduced by \texttt{\#} (apart from within
quoted strings).
\item STARLSE (see SUN/105) contains VAX Language Sensitive Editor templates 
for Interface Files.
\end{itemize}

\section{Parameter Storage and the CURRENT Value\xlabel{parameter_storage}
\label{current}}

Each ADAM application program has zero or more {\em program parameters}.
A program parameter is a value which the program gets
from the `outside world', or which the program wishes to make available
outside itself. The value may be a primitive value (scalar or array) or a file 
or device name for example.

Before describing the way parameters are specified in the Interface File, it 
will be useful to say a little about the way in which parameter values are 
stored.

If a primitive value is supplied for parameters of a standard HDS type
(see Section \ref{type}),
storage of the defined type is created and, if possible, the value is 
converted to the storage type and stored.
For other types, storage of the type of the supplied value is created.

If a name is supplied as a value for a parameter, the name is stored.
For parameters of a standard HDS type, it is assumed to be the name of an HDS 
object of a suitable type and the storage type of the parameter is effectively
the type of the named object.

The stored value of a parameter is known as the {\em current value}. It will
normally\footnote{The exception is `internal' parameters (see Section 
\ref{vpath}) which are stored in memory.} be in an HDS file known as the 
{\em parameter file}.
A component of the right type, size and shape is created to hold the given 
parameter value -- there is a special structure to hold values which are names.
Each program or monolith has its own parameter file which is retained between 
sessions. The parameter file name is \texttt{name.sdf} (where \texttt{name} is the
name of the executable file) and, by default, it will be found in subdirectory
\texttt{adam} of the user's login directory.

Note that the current value is usually thought of as the last-used value, 
however:
\begin{itemize}
\item The given value may have been rejected by the parameter system or program
after having been stored. This will be the case if it fails a range check, for
example.
\item The current value may have been written into the parameter file by
external means, the ICL SETPAR command, for example.
\end{itemize}

\section{The Parameter Specification\xlabel{the_parameter_specification}}

\subsection{Introduction\xlabel{introduction2}}

Each program parameter must have a {\em Parameter Specification} in the
Interface File. The Parameter Specification is of the form: 
\begin{quote} \begin{verbatim}
PARAMETER parametername
   parfieldname fieldvalue(s)
         .
ENDPARAMETER
\end{verbatim} \end{quote}
where \texttt{parametername} is a character string giving the program parameter 
name which is used in the program (in a call to DAT\_ASSOC or 
PAR\_GETnx {\em etc.}). Case is not significant in parameter names.
For a discussion of valid parameter names, see Appendix \ref{names}.

The possible subfields in the Parameter Specification are described in detail
below -- they are all optional and, apart from
TYPE, which must appear before DEFAULT, RANGE or IN fields, the order
is unimportant.

\subsection{Summary of Parameter Specification Subfields
\xlabel{summary_of_parameter_specification_subfields}}

The Parameter Specification subfields are described in detail in the following
sections. This is a classified list to show the features available.

\subsubsection*{Data Type and Access Fields}
\begin{center}
\begin{tabular}{|l|l|l|} \hline
Field & Specified Attribute & Default\\
\hline
TYPE & Type of the parameter & univ \\
PTYPE & Parameter is a Device Parameter & Normal Parameter \\
ACCESS & Access required to the parameter & update \\
\hline
\end{tabular}
\end{center}

\subsubsection*{Command Line Control Fields}
\begin{center}
\begin{tabular}{|l|l|l|} \hline
Field & Specified Attribute & Default\\
\hline
POSITION & Command line parameter position allocated & \\
KEYWORD & User's name for the parameter & parameter
name \\
\hline
\end{tabular}
\end{center}

\subsubsection*{Value Control Fields}
\begin{center}
\begin{tabular}{|l|l|l|} \hline
Field & Specified Attribute & Default\\
\hline
DEFAULT & Static default value & \\
ASSOCIATION & Associated global parameter & \\
VPATH & Search path for parameter value & prompt\\
RANGE & Range of values the parameter may take & \\
IN & Set of values the parameter may take & \\
\hline
\end{tabular}
\end{center}

\subsubsection*{Prompt Control Fields}
\begin{center}
\begin{tabular}{|l|l|l|} \hline
Field & Specified Attribute & Default\\
\hline
PROMPT & String to be used as parameter prompt & parameter
name \\ 
PPATH & Search path for prompt default value & 'dynamic,default' \\
HELP & Single-line parameter help message & \\
HELPKEY & Source of multi-line parameter help information & \\
\hline
\end{tabular}
\end{center}

\subsection{The TYPE Field
\xlabel{the_type_field}\label{type}}

This field specifies the type of the parameter. 
This field is used by the system to determine the type of storage to be
used for the parameter and its default values, {\em etc}. 
If a primitive value of a different type is supplied,
it will be converted to the specified type for storage if possible. 
If conversion is not possible, an error is reported.

It is important to distinguish between the type of the parameter and the type 
of the value to be obtained from it by the program.

The field is of the form:
\begin{quote} \begin{verbatim}
TYPE datatype
\end{verbatim} \end{quote}
where \texttt{datatype} is a character string giving the required type of the
parameter.
\texttt{datatype} may be:
\begin{itemize}
\item A `standard' HDS primitive type ({\em i.e.}\ \_INTEGER, \_REAL, \_DOUBLE, \_LOGICAL,
\_CHAR).
\item LITERAL -- formerly used to specify that the parameter is of type \_CHAR
and to force the parameter system to accept unquoted character strings as
character values rather than HDS object names. 
This is now the standard behaviour for parameters of type \_CHAR so datatype
LITERAL is no longer required.
\item A `user-defined' type (IMAGE, SPECTRUM {\em etc.}).
\item UNIV -- used by convention when the parameter type is unimportant. 
Actually this is treated the same as other `user-defined' types.
Parameter storage corresponding with the type of the value given will be
created.
\item A `facility-recommended' type (TAPE for MAG, FILENAME for FIO {\em etc.}).
These are used by convention -- actually any user-defined type will do.
\end{itemize}
If the TYPE field is omitted, type UNIV is assumed.

Notes:
\begin{enumerate}
\item Internally only the `standard' primitive types (\_LOGICAL, \_INTEGER, 
\_REAL, \_CHAR, \_DOUBLE) are differentiated -- all other types are treated
the same.
\item Only values of the standard types, or names, can be given in the 
Interface File or as parameter values at run time.
\item The dimensionality of parameters is not declared in the 
Interface File, but is defined by the value given at run time.
The given dimensionality must be acceptable to the program
({\em e.g.}\ when it does a call to PAR\_GETnx).
\item The PAR\_GETnx/\_PUTnx routines only GET/PUT values of the 
standard HDS types --
other types have to be handled by calls to the appropriate facility\_ASSOC
routine (DAT\_ASSOC for example).
Where primitive values are being handled, conversion between the value type
and the storage type will be performed if necessary and possible.
\end{enumerate}

\subsubsection*{Example}
Suppose there is a program DISP that draws an image on an
image display device. 
It has three parameters:
\begin{description}
\item[IMAGE] the dataset containing the image to be drawn, 
\item[DISPLAY] the name of the image display device, and
\item[LIMITS] an integer array of limiting values.
\end{description}
The Interface File might be:
\begin{quote} \begin{verbatim}
INTERFACE DISP
   .
   PARAMETER IMAGE
         .
      TYPE IMAGE
         .
   ENDPARAMETER
   PARAMETER DISPLAY
         .
      TYPE IMAGEDISPLAY
         .
   ENDPARAMETER
   PARAMETER LIMITS
         .
      TYPE _INTEGER
         .
   ENDPARAMETER
      .
ENDINTERFACE
\end{verbatim} \end{quote}
If the user specified a value of \texttt{[1.2,511.8,255.6,512.1]}
for the LIMITS array, {\em i.e.}\ a \_REAL array where type \_INTEGER is
specified for the parameter, the system knows that this is a valid 
conversion and can perform it for the user ({\em i.e.}\ \texttt{[1,511,255,512]} 
is stored).

Similarly, the program could try to obtain a \_REAL array from the
parameter storage (by calling PAR\_GET1R)\@. In that case, the array would be
converted back to \_REAL although the numbers supplied by the user would have
been truncated ({\em i.e.}\ \texttt{[1.0,511.0,255.0,512.0]} would be supplied 
to the program).

\subsection{The PTYPE Field\xlabel{the_ptype_field}}

Most parameters are associated with values of one type or another, {\em e.g.}\
primitive data types, or data-system objects. These are known as {\em normal
parameters}.

However, there is a class of parameters which are intimately associated with
some type of device ({\em e.g.}\ a magnetic tape drive, or a graphics terminal).
These are known as {\em device parameters}.
Device parameters are typically used in calls such as GKS\_ASSOC.
A parameter is defined as a device parameter by the PTYPE field.
\begin{quote} \begin{verbatim}
PTYPE  parameter-type
\end{verbatim} \end{quote}
where \texttt{parameter-type} is a character string which can currently only be
DEVICE.

The precise type of the device can be indicated by the TYPE field.
The specification for parameter DISPLAY in our example above would then be:
\begin{quote} \begin{verbatim}
PARAMETER DISPLAY
      .
   PTYPE DEVICE
   TYPE GRAPHICS
      .
ENDPARAMETER
\end{verbatim} \end{quote}
The value of a device parameter is
a name which can be translated to define the device to be used.
For details of possible values, see the appropriate facility description,
SGS, GKS, FIO {\em etc}.

{\em
Currently PTYPE is not used by the system -- only the TYPE is of significance
and there anything other than a standard HDS type will have the same effect.}

\subsection{The ACCESS Field\xlabel{the_access_field}}

This field specifies the mode of access which the program
requires to the parameter.
It is used to restrict access to parameters handled by PAR routines or by
DAT\_ASSOC.

Note that the field has no effect for facilities such as NDF, FIO and GKS
which only use the parameter system to get the name of a file or device which
is accessed by other means. The access mode applies to the parameter itself,
not to any file or device, {\em etc.}\ that the parameter refers to.

The field has the form:
\begin{quote} \begin{verbatim}
ACCESS mode
\end{verbatim} \end{quote}
where \texttt{mode} is a character string which may be:
\begin{description}
\item[READ] which gives read access.
\item[WRITE] which gives write access.
\item[UPDATE] which gives read and write access.
\end{description}

Thus if a program does not modify the parameter,
READ should be specified in the Interface File.
If the value is to be written or modified, then WRITE or UPDATE 
should be specified.

If the ACCESS field is omitted, UPDATE is assumed.

\subsubsection*{Example}
The access fields for the parameters in our DISP example could be:
\begin{quote} \begin{verbatim}
INTERFACE DISP
   .
   PARAMETER IMAGE
         .
      TYPE IMAGE
      ACCESS READ
         .
   ENDPARAMETER
   PARAMETER DISPLAY
         .
      PTYPE DEVICE
      TYPE IMAGEDISPLAY
         .
   ENDPARAMETER
   PARAMETER LIMITS
         .
      TYPE _INTEGER
      ACCESS READ
         .
   ENDPARAMETER
      .
ENDINTERFACE
\end{verbatim} \end{quote}

\subsection{The POSITION Field\xlabel{the_position_field}}

Defines the {\em parameter position} on the command line at which values for 
this parameter may be given.

There are a limited number (See Appendix \ref{consts}) of
command line positions available to be allocated. Not every parameter need
be allocated a position, but the positions allocated must form a contiguous
set starting at 1.
Each position may have only one parameter allocated to it.

Note that, for ADAM I-tasks, the POSITION field may be overridden by an 
individual action NEEDS list (see the ACTION specification, Section 
\ref{action}).

The field is of the form:
\begin{quote} \begin{verbatim}
POSITION n
\end{verbatim} \end{quote}
where \texttt{n} is an integer constant which specifies the allocated position.

If the POSITION field is omitted, no command line position is allocated to this
parameter.

\subsubsection*{Example}
Again taking our DISP example, if the positions were specified as follows:
\begin{quote} \begin{verbatim}
INTERFACE DISP
         .
   PARAMETER IMAGE
         .
      POSITION 1
         .
   ENDPARAMETER
   PARAMETER DISPLAY
         .
      POSITION 2
         .
   ENDPARAMETER
   PARAMETER LIMITS
         .
      POSITION 3
         .
   ENDPARAMETER   
      .
ENDINTERFACE
\end{verbatim} \end{quote}
and the user gives the command:
\begin{quote} \begin{verbatim}
disp iue.image.swp2099 ikon [1,511,255,512]
\end{verbatim} \end{quote}
Then the HDS object \texttt{iue.image.swp2099} would be associated with parameter
IMAGE, XWINDOWS would be used as the DISPLAY and LIMITS would take the value
\texttt{[1,511,255,512]}.

\subsection{The KEYWORD Field\xlabel{the_keyword_field}}

This field defines the name by which the parameter is known to 
the person running the program and which can be used on the command line to 
specify a value for the parameter. It will also be used to refer to the
parameter in prompts and error messages. Case is not significant in keywords --
they are always displayed in upper case.

The field is of the form:
\begin{quote} \begin{verbatim}
KEYWORD  name
\end{verbatim} \end{quote}
where \texttt{name} is a valid keyword\footnote{See Appendix \ref{names} for a
discussion of valid keyword names.}.

This completes the separation between the program's
view of parameters, and the user's view of them. It is possible for the
programmer to re-write the program using completely different parameter
names, but the user's view could be kept the same by just changing the 
parameter names in the Interface File. Similarly, the
user's view of the program can be changed by just changing the
KEYWORD fields.

If the KEYWORD field is not specified for a parameter, then the system will
use the parameter name as the keyword.

\subsubsection*{Example}
If the names of the parameters in the DISP example were changed,
the user's view could be preserved by specifying keywords as follows
({\em Note: the example is not intended as an
illustration of common practice -- it is only to illustrate the
principle}\/):
\begin{quote} \begin{verbatim}
INTERFACE DISP
   .
   PARAMETER P1
         .
      POSITION 1
      KEYWORD IMAGE
         .
   ENDPARAMETER
   PARAMETER P2
         .
      POSITION 2
      KEYWORD DISPLAY
         .
   ENDPARAMETER
   PARAMETER P3
         .
      POSITION 3
      KEYWORD LIMITS
         .
   ENDPARAMETER
      .
ENDINTERFACE
\end{verbatim} \end{quote}
The user could then specify:
\begin{quote} \begin{verbatim}
disp display=ikon image=iue.image.swp2099 limits=[1,511,255,512]
\end{verbatim} \end{quote}
Note that keyword-style parameter specifiers are ignored in calculating
{\em position}.
Therefore, with the above Interface File, the same effect would be produced
by the command:
\begin{quote} \begin{verbatim}
disp display=ikon iue.image.swp2099 limits=[1,511,255,512]
\end{verbatim} \end{quote}

\subsubsection*{Logical Keywords}
If a parameter is of type \_LOGICAL, its keyword alone may be used on
the command line to specify the value TRUE for the parameter and the 
keyword prefixed by NO to specify the value FALSE.

For example, assume a program TEST with a logical parameter, keyword
SWITCH, then:
\begin{quote}
\texttt{test switch} is equivalent to \texttt{test switch=TRUE}
\end{quote}
and
\begin{quote}
\texttt{test noswitch} is equivalent to \texttt{test switch=FALSE}
\end{quote}

\subsection{The DEFAULT Field
\xlabel{the_default_field}\label{default}}

This field is used to specify a {\em static default} value for a parameter. 
The static default value is one of the options available to be used as the 
parameter value.

The field is of the form:
\begin{quote} \begin{verbatim}
DEFAULT default_value
\end{verbatim} \end{quote}
where \texttt{default\_value} may be:
\begin{itemize}
\item A constant (number, character or logical).
\item A name.
\item An array. (Currently only 1-D arrays may be given, and the run-time 
syntax for arrays, involving square brackets, cannot be used -- they must be 
specified as a sequence of constants and may be enclosed in parentheses.)
\item The null value (!).
\end{itemize}
The field is terminated by the next reserved-word token, therefore names which
correspond with a reserved word cannot be given unless enclosed in quotes.

Note that DEFAULT cannot be specified before the TYPE field has been declared 
and the value specified must be appropriate for the specified type. 
A character constant (quoted string) may be given as \texttt{default\_value} 
for a non-primitive type -- the string within the quotes will be used as the
`name' when required. This can be useful if non-standard strings are to be 
given as defaults ({\em e.g.}\ lists for the GRP facility handling parameters
declared as TYPE NDF).

\subsubsection*{Example}
\begin{quote} \begin{verbatim}
INTERFACE DISP
      .
   PARAMETER IMAGE
         .
      TYPE IMAGE
      DEFAULT IUE.IMAGE.SWPTEST
         .
   ENDPARAMETER
   PARAMETER DISPLAY
         .
      PTYPE DEVICE
      TYPE IMAGEDISPLAY
      DEFAULT XWINDOWS
         .
   ENDPARAMETER
   PARAMETER LIMITS
         .
      TYPE _INTEGER
      DEFAULT (1,512,1,512)
         .
   ENDPARAMETER
      .
ENDINTERFACE
\end{verbatim} \end{quote}

\subsection{The ASSOCIATION Field
\xlabel{the_association_field}\label{association}}

This field is used to specify that another parameter is to be associated
with this parameter.
The value of the associated parameter is one of the options available to be
used as the value of this parameter.
Furthermore, the value of a parameter can be used to automatically update
the value of its associated parameter on successful completion of the
program.

This behaviour is primarily of relevance to ADAM A-tasks. I-tasks 
can obtain parameter values from an association, but do not write 
parameters on completion. 

The field is of the form:
\begin{quote} \begin{verbatim}
ASSOCIATION  association-specification
\end{verbatim} \end{quote}
where \texttt{association-specification} is a character string consisting of two 
parts -- an association operator followed by a parameter specifier.

There are three association operators:
\begin{itemize}
\item \texttt{<- } The associated parameter may be used by the program (see
Sections \ref{vpath} and \ref{ppath}) but will not be updated by it.
\item \texttt{ ->} The associated parameter will be set to the value of the
program parameter if:
\begin{itemize}
\item The program has ended successfully, {\em i.e.}\ STATUS = SAI\_\_OK.
\item The program parameter has a value and has not been cancelled.
\item The program is an A-task or A-task monolith.
\end{itemize}
\item \texttt{<->} combines \texttt{<-} and \texttt{->}.
\end{itemize}

Currently, only {\em global} parameters may be associated with program
parameters  -- they are stored in the GLOBAL data structure and provide a
common pool of parameter values for programs in an integrated system.
Global parameters can be created by the association write
mechanism or by an ICL CREATEGLOBAL or SETGLOBAL command. They may also be
read into ICL variables by the GETGLOBAL command.
It is necessary to be careful when multi-tasking in ADAM because 
an error will occur if two programs attempt to update the GLOBAL data 
structure simultaneously.

The specifier of a global parameter is of the form:
\begin{quote} \begin{verbatim}
GLOBAL.parameter_name
\end{verbatim} \end{quote}
If the ASSOCIATION field is omitted, no association is made.

\subsubsection*{Example}
\begin{quote} \begin{verbatim}
INTERFACE DISP
   .
   PARAMETER IMAGE
         .
      ASSOCIATION <->GLOBAL.IMAGE
         .
   ENDPARAMETER
      .
ENDINTERFACE
\end{verbatim} \end{quote}
This gives the parameter DISP.IMAGE read and write access to the global 
parameter GLOBAL.\-IMAGE.

\subsection{The VPATH Field
\xlabel{the_vpath_field}\label{vpath}}

This field is used to specify the way in which a value is obtained for
a parameter if the user does not specify a value (on the command line, or by
an ICL SEND SET command, for example) before it is required by the program.
In that case, there are several potential sources for parameter values and
different applications will want to use different sources for their various
parameters.

The VPATH field is the means whereby this
order of searching for parameter values is specified.

The field is of the form:
\begin{quote} \begin{verbatim}
VPATH  value-resolution-path
\end{verbatim} \end{quote}
where \texttt{value-resolution-path} is a character string which gives a path
used for searching for a parameter value.
It consists of a set of {\em path specifiers}. If there are two or more
path specifiers, they must be separated by commas and the string must be
enclosed in single quotes.

The valid specifiers are:
\begin{description}
\item[CURRENT] Use the current (last-used) value of the parameter.
(See Section \ref{current} for more information on the current value.)
\item[DYNAMIC] Use the dynamic default value specified by the program.
Dynamic defaults are set by the program calling a subroutine such as
PAR\_DEFnx or DAT\_DEF specifying the required values.
\item[DEFAULT] Use the static default specified in the Interface Module
(see Section \ref{default}).
\item[GLOBAL] Use the value of the associated parameter (see Section
\ref{association}, The ASSOCIATION Field).
\item[PROMPT]  Prompt the user and obtain a value. (See also Section
\ref{ppath}, The PPATH Field.)
There is an implied PROMPT at the end of every VPATH.
{\em Note that PROMPT will always give a result to the program, even if it is
a bad status, therefore PROMPT only makes sense as the last thing on the path.}
\item[NOPROMPT] Give up trying to get a value and return status PAR\_\_NULL.
This only makes sense as the last specifier in the path and will prevent a
prompt being issued as a last resort if the value-resolution-path is exhausted.
\item[INTERNAL] This specifier can only be used as the first and only
specifier on the VPATH. 
A \texttt{value-resolution-path} of 'DYNAMIC,CURRENT,NOPROMPT' is
implied\footnote{Old-style D-task parameters should have VPATH 'INTERNAL' 
specified as D-tasks cannot prompt for values.}.
`Internal' storage will be used to hold the parameter values.
Scalar values are stored in memory to give enhanced performance; array values 
are stored in the parameter file.
\end{description}

If the value for a parameter has not been specified when it is required,
the system looks at the VPATH specification, picks out the first path 
specifier and tries to find a value from this source. 
If a value is not found, the next path
specifier is extracted and another search is made. This process continues
until a value is found, the specifier is NOPROMPT or the path specification 
runs out.
If the path specification is exhausted, the user is prompted for a value.

Note that the VPATH field is only used the first time a parameter value is
obtained. If the parameter is cancelled before another attempt to `get' a
value, a prompt will be issued, even if NOPROMPT is put on the VPATH.
If the parameter is not cancelled, the existing value will be returned again.

If the VPATH field is omitted, VPATH PROMPT is assumed.

\subsubsection*{Example}
Suppose the parameters of DISP are specified as follows:
\begin{quote} \begin{verbatim}
INTERFACE DISP
   .
   PARAMETER IMAGE
         .
      VPATH GLOBAL
      ASSOCIATION <->GLOBAL.IMAGE
         .
   ENDPARAMETER
   PARAMETER DISPLAY
         .
      VPATH 'CURRENT,DEFAULT'
      DEFAULT XWINDOWS
         .
   ENDPARAMETER
   PARAMETER LIMITS
         .
      VPATH DYNAMIC
         .
   ENDPARAMETER
      .
ENDINTERFACE
\end{verbatim} \end{quote}
Then, if the user just types:
\begin{quote} \begin{verbatim}
disp
\end{verbatim} \end{quote}
({\em i.e.}\ does not give any parameter values), the following actions occur:

When the program asks for the value of the IMAGE parameter, no value has 
been given so the first part of the VPATH specification is extracted (GLOBAL).
This path specifier tells the system to look for a value of the associated
parameter (GLOBAL.IMAGE).
If it has a value, this value is taken as the value of IMAGE;
if not, the resolution path is exhausted so the user will be prompted for a 
value for IMAGE.

When the program asks for a value of DISPLAY, the specifier CURRENT
is extracted from the VPATH.
If DISPLAY has a current value, the same value is used again; if not, the 
static default value (XWINDOWS) specified in the Interface File is taken.

Similarly, when the program asks for the value of LIMITS, the system 
attempts to use the dynamic default suggested by the program. If no
dynamic default has been set, the user is prompted for a value for LIMITS.

\subsubsection*{Run-time Modification of VPATH Action}
The user can modify the action of VPATH at run time by using 
the special command-line keywords PROMPT and/or RESET.
\begin{description}
\item[PROMPT] causes the VPATH to be ignored and a prompt to be issued
for any required parameters which have not already been given values.
\item[RESET] makes the system ignore CURRENT on the VPATH.
See also Section \ref{ppath} for the effect of RESET on prompts.
\end{description}
For example, assuming the Interface File above,
\begin{quote} \begin{verbatim}
disp prompt
\end{verbatim} \end{quote}
causes all parameters to be prompted for, irrespective of VPATH.
\begin{quote} \begin{verbatim}
disp reset
\end{verbatim} \end{quote}
causes the system to ignore `CURRENT' in the VPATH for parameter DISPLAY,
and use the DEFAULT value.

\subsection{The RANGE Field
\xlabel{the_range_field}\label{range}}

This field is used to specify a range of permitted values for the parameter.
The RANGE values will also be used if MIN or MAX is specified as the parameter
value in the absence of any dynamic minimum or maximum values set by the
program.

RANGE can only be specified for parameters of standard primitive type other
than \_LOGICAL, and is only relevant if the parameter value is a scalar.
A RANGE field will also constrain any dynamic minimum or maximum values which
the program may set.

A parameter cannot have both a RANGE and an IN field and RANGE cannot be 
specified before TYPE has been declared. 

Range checking is carried out
when a program attempts to get the value of a parameter or set a minimum
or maximum value. No range checking occurs when putting a value. 
For \_CHAR parameters, all values are converted to upper case for checking and 
the ASCII collating sequence is used.

If the constraints are violated, the system reports the error and, if the
violation was on getting a parameter value, prompts for
another value unless the parameter had VPATH INTERNAL, in which case status 
SUBPAR\_\_OUTRANGE is returned.

The field is of the form:
\begin{quote} \begin{verbatim}
RANGE min, max
\end{verbatim} \end{quote}
where \texttt{min} and \texttt{max} are two character or number
constants specifying the minimum and maximum values for the parameter -- the 
values must be convertible to the type of the parameter.

If \texttt{min > max}, the value must not lie between them. The values themselves
are always acceptable.

\subsubsection*{Example}
\begin{quote} \begin{verbatim}
PARAMETER EXPONENT
      .
   TYPE _REAL
   RANGE 0.001, 1000.0
      .
ENDPARAMETER
\end{verbatim} \end{quote}

\subsection{The IN Field\xlabel{the_in_field}}

This field is used to specify a set of values which the parameter may take.
IN can only be specified for parameters of standard primitive type other
than \_LOGICAL, and is only relevant if the parameter value is a scalar.

A parameter cannot have both a RANGE and an IN field and IN cannot be specified
before TYPE has been declared. 

Checking is carried out when a program attempts to get the value of a 
parameter.

No checking occurs when putting a value.
For \_CHAR parameters the check is case independent.

If the constraints are violated, the system reports the error and prompts for
another value unless the parameter had VPATH INTERNAL, in which case status 
SUBPAR\_\_OUTRANGE is returned.

The field is of the form:
\begin{quote} \begin{verbatim}
IN set-of-values
\end{verbatim} \end{quote}
where \texttt{set-of-values} consists of a list of character or number
constants specifying the valid values for the parameter. 
The specified values must all be convertible to the type of the parameter.
The set of values is terminated by the next reserved-word token, therefore
any required string which corresponds with a reserved word must be enclosed in
quotes.

\subsubsection*{Example}
\begin{quote} \begin{verbatim}
PARAMETER FILTER
      .
   TYPE '_CHAR'
   IN 'R', 'I', 'J'
      .
ENDPARAMETER
\end{verbatim} \end{quote}

\subsection{The PROMPT Field\xlabel{the_prompt_field}}

This field is used to specify a prompt string for the parameter\footnote{Use
of the prompt string is actually a function of the user interface.
The effect with ICL, SMSICL, DCL and Unix shells is described here.}.

The field is of the form:
\begin{quote} \begin{verbatim}
PROMPT  text
\end{verbatim} \end{quote}
where text is a character string ({\em n.b. it must be enclosed in quotes
if it consists of more than one token}\/).

If the PROMPT field is omitted, the parameter keyword is used as the prompt.

\subsubsection*{Example}
\begin{quote} \begin{verbatim}
INTERFACE DISP
   .
   PARAMETER IMAGE
         .
      VPATH PROMPT
      PROMPT 'Image to be displayed'
         .
   ENDPARAMETER
   PARAMETER DISPLAY
         .
      VPATH PROMPT
      PROMPT 'Image display device'
         .
   ENDPARAMETER
      .
ENDINTERFACE
\end{verbatim} \end{quote}
Suppose that the user has given the command:
\begin{quote} \begin{verbatim}
disp image=iue.image.swp2099
\end{verbatim} \end{quote}
The program will take the specified value as the value of the IMAGE parameter.
When it requires the DISPLAY parameter value, none has been specified on
the command line so, as the VPATH field specifies that the user is
to be prompted for a value, the system generates a prompt line such 
as:
\begin{quote} \begin{verbatim}
DISPLAY - Image display device >
\end{verbatim} \end{quote}
to which the user has to reply with a suitable value. (The prompt may
also have a suggested value for the parameter, obtained via the PPATH
field, see below).

\subsection{The PPATH Field
\xlabel{the_ppath_field}\label{ppath}}

This field specifies a path used for searching for a {\em suggested value}
\footnote{Use of the suggested value is actually a function of the user 
interface.
The effect with ICL, SMSICL, DCL and Unix shells is described here.}.
The suggested value will be displayed as part of the prompt string and used as 
the parameter value if the user responds to the prompt by hitting the carriage
return key.
Hitting the TAB key will make the suggested value available in the 
terminal input buffer where it can be edited with the normal line editing 
commands.

The field is of the form:
\begin{quote} \begin{verbatim}
PPATH  value-resolution-path
\end{verbatim} \end{quote}
where \texttt{value-resolution-path} is a character string 
consisting of a set of {\em path specifiers}. If there are two or more
path specifiers, they must be separated by commas and the string must be
enclosed in single quotes.

The valid specifiers are:
\begin{description}
\item[CURRENT] Use the current value of the parameter.
(See Section \ref{current} for more information on the current value.)
\item[DYNAMIC] Use the dynamic default value specified by the program.
Dynamic defaults are set by the program calling a subroutine such as
PAR\_DEFnx or DAT\_DEF specifying the required values.
\item[DEFAULT] Use the static default value specified in the Interface Module
(see Section \ref{default}).
\item[GLOBAL] Use the value of the associated global parameter (see Section
\ref{association}, The ASSOCIATION Field).
\end{description}
If the suggested value for a parameter is required, the system looks at the
PPATH specification, picks out the first path specifier and tries to
find a value from this source. If a value is not found, the next path
specifier is extracted and another search is made. This process
continues until a value is found, or until the path specification runs
out. 
The value is then passed, together with the prompt string,
in a parameter request message to the user interface.

If the PPATH field is omitted, or fails to give a value, 
\texttt{'DYNAMIC,DEFAULT'} is used.
If this fails to give a value, there is no suggested value.
\subsubsection*{Example}
\begin{quote} \begin{verbatim}
INTERFACE DISP
   .
   PARAMETER IMAGE
         .
      PROMPT 'Image to be displayed'
      PPATH 'CURRENT,DEFAULT'
      DEFAULT IUE.IMAGE.SWPTEST
         .
   ENDPARAMETER
   PARAMETER DISPLAY
         .
      PROMPT 'Image display device'
      PPATH GLOBAL
      ASSOCIATION <-GLOBAL.DEVICE
         .
   ENDPARAMETER
   PARAMETER LIMITS
         .
      PROMPT 'Pixel limits'
      PPATH 'DYNAMIC,DEFAULT'
      DEFAULT 1,511,255,512
         .
   ENDPARAMETER
      .
ENDINTERFACE
\end{verbatim} \end{quote}
Assuming that:
\begin{enumerate}
\item There is no current value for IMAGE.
\item GLOBAL.DEVICE has the value XWINDOWS.
\item The program has not specified a dynamic default for LIMITS.
\end{enumerate}
Then, with the above Interface File, if the user just types:
\begin{quote} \begin{verbatim}
disp
\end{verbatim} \end{quote}
({\em i.e.}\ does not give any parameter values), and the VPATH field
results in the user being prompted, the three prompts would be something like:
\begin{quote} \begin{verbatim}
IMAGE - Image to be displayed /@IUE.IMAGE.SWPTEST/ >
DISPLAY - Image display device /@XWINDOWS/ >
LIMITS - Pixel limits /[1,511,255,512]/ >
\end{verbatim} \end{quote}
{\em Note the `@' in the suggested values is inserted by the system to avoid
ambiguity between names and character strings.}

\subsubsection*{Run-time Modification of PPATH Action}
The user can modify the operation of the PPATH at run-time by using the
special command-line keywords RESET and/or ACCEPT.
\begin{description}
\item[RESET] Causes CURRENT to be ignored on the PPATH.
\item[ACCEPT (or $\backslash$)] causes the suggested value to be used for any 
required parameters which would otherwise be prompted for.
The user may also respond to a prompt with
\texttt{$\backslash$}. The suggested value will be taken for the prompted parameter and
for any remaining parameters.
It is also possible to force ACCEPT for individual parameters by putting
\texttt{keyword=ACCEPT} or \texttt{keyword=$\backslash$} on the command line (where
\texttt{keyword} is the keyword of the parameter).

If there is no suggested value, the prompt will be issued anyway.
\end{description}
{\em E.g.}\ with the Interface File above, the command:
\begin{quote} \begin{verbatim}
disp reset accept
\end{verbatim} \end{quote}
will cause the static default value, rather than the current value, 
to be used as the suggested value for IMAGE, and that value to be used as the
parameter value without prompting.

\subsection{The HELPKEY Field\xlabel{the_helpkey_field}}

This field is used to specify a help file and a module within it at which
{\em multi-line} help text for the parameter may be found\footnote{
The precise effect of the HELPKEY field will depend upon the user interface 
in use.
The effect with ICL, SMSICL, DCL and Unix shells is described here.}.
Text read from a help file is known as {\em multi-line} help although it could
in fact be only one line.

The field is of the form:
\begin{quote} \begin{verbatim}
helpkey help_specifier
\end{verbatim} \end{quote}
where \texttt{help\_specifier} is a character string of the form:
\begin{quote} \begin{verbatim}
filename key1 key2 etc...
\end{verbatim} \end{quote}
specifying that the text to be displayed may be found in the help file 
\texttt{filename} and the module defined by \texttt{ key1 key2 etc...} (but see
also the HELPLIB Specification, Section \ref{helplib}).
For more information on multi-line help and a discussion of help filenames, 
see Appendix \ref{help}.

\texttt{help\_specifier}  may also be \texttt{*} or \texttt{'*'}, in which case a
default value of:
\begin{quote} \begin{verbatim}
interface_name PARAMETERS parameter_name
\end{verbatim} \end{quote}
will be used, where \texttt{interface\_name} is the name specified in the
current INTERFACE field.
(Use of this default implies that there is an associated HELPLIB Specification.)

If the user is prompted for a parameter value, and responds with `??',
the specified text is displayed at the terminal and the user will be left in 
the help system and prompted for further topics of interest. 

If the user responds with `?' and there is no HELP field specified for the
parameter the effect is the same except that the user 
will be re-prompted for the parameter value immediately (except from SMSICL).

\subsubsection*{Example}
The previous example is therefore better written:
\begin{quote} \begin{verbatim}
INTERFACE DISP
   .
   PARAMETER IMAGE
         .
      HELP 'This is the name of an image dataset to be displayed'
         .
   ENDPARAMETER
      .
      .
   PARAMETER LIMITS
         .
      HELPKEY 'DISP_DIR:DISP DISP PARAMETERS LIMITS'
         .
   ENDPARAMETER
      .
ENDINTERFACE
\end{verbatim} \end{quote}

\subsection{The HELP Field\xlabel{the_help_field}}

This field is used to specify some help text for a parameter\footnote{
The precise effect of the HELP field will depend upon the user interface in use.
The effect with ICL, SMSICL, DCL and Unix shells is described here.}.

This field is of the form:
\begin{quote} \begin{verbatim}
HELP help-specifier
\end{verbatim} \end{quote}
where \texttt{help-specifier} is a character string giving help information about 
the parameter, it may be either a single line of text to be displayed
({\em single-line help}) or a string of the form:
\begin{quote} \begin{verbatim}
%filename key1 key2 etc...
\end{verbatim} \end{quote}
specifying that multi-line help may be found in the help file 
\texttt{filename} and the module defined by \texttt{ key1 key2 etc...}

{\em Note that the preferred way of specifying multi-line help
is using the HELPKEY field and that single-line help is not generally
considered very useful.}

If the user is prompted for a parameter value, and responds with `?',
the specified text is displayed at the terminal and the user is re-prompted.
(Except in the case where SMSICL outputs multi-line help -- then the user
is left in the help system and prompted for further topics of interest.
When the user exits from the help system, the parameter prompt will
reappear.)

If the user responds with `??' and there is no HELPKEY
field specified for this parameter, the effect is the same except that
if multi-line help is output the user will 
be left in the help system and prompted for further topics.

\subsubsection*{Example}
\begin{quote} \begin{verbatim}
INTERFACE DISP
   .
   PARAMETER IMAGE
         .
      HELP 'This is the name of an image dataset to be displayed'
         .
   ENDPARAMETER
      .
      .
   PARAMETER LIMITS
         .
      HELP '%disp_dir:disp disp parameters limits'
         .
   ENDPARAMETER
      .
ENDINTERFACE
\end{verbatim} \end{quote}

\section{The HELPLIB Specification\xlabel{the_helplib_specification}
\label{helplib}}

This specification may be used to specify a `top level' for all subsequent 
HELPKEY fields until another HELPLIB specification is found. The specification
may appear anywhere in an Interface File.

The field is of the form:
\begin{quote} \begin{verbatim}
HELPLIB help_top_level_spec
\end{verbatim} \end{quote}
where \texttt{help\_top\_level\_spec} is a character string to be inserted (with an 
intervening space) in front of any string specified in subsequent HELPKEY
fields as the Interface File is parsed --
it may be a blank character constant to nullify the 
effect of an earlier HELPLIB field.

Thus, assuming that the help file for the program DISP is 
DISP\_DIR:DISP and that help text for each parameter is held in subtopic
{\em parameter-name} of subtopic PARAMETERS of topic DISP, then
the Interface File for program DISP could be:
\begin{quote} \begin{verbatim}
INTERFACE DISP
# Specify the top level of help
   HELPLIB 'DISP_DIR:DISP DISP PARAMETERS'
   PARAMETER IMAGE
         .
      HELPKEY 'IMAGE'
         .
   ENDPARAMETER

# Specify top level help for use with default helpkey
   HELPLIB DISP_DIR:DISP
   PARAMETER DISPLAY
         .
      HELPKEY *
         .
   ENDPARAMETER

# Just as an example, use the blank specifier
   HELPLIB ''
   PARAMETER LIMITS
         .
      HELPKEY 'DISP_DIR:DISP DISP PARAMETERS LIMITS'
         .
   ENDPARAMETER
      .
ENDINTERFACE
\end{verbatim} \end{quote}
{\em Note that the use of multiple HELPLIB fields here is purely to show the
effects -- there would normally be only one HELPLIB specification in
an Interface File.}

\section{The MESSAGE Specification\xlabel{the_message_specification}}

If a program uses the ADAM message or error systems (MSG or ERR, see
\xref{SUN/104}{sun104}{}), then a message text is required. 
The Message Specification can be used to specify text to be used in preference
to text supplied by the program in the call to the MSG or ERR routine.
Each of the Message Parameter names used in the program can have
a Message Specification in the Interface File. The Message Specification
is of the form: 
\begin{quote} \begin{verbatim}
MESSAGE parametername
  TEXT  string
ENDMESSAGE
\end{verbatim} \end{quote}
where \texttt{parametername} is a character string specifying the Message 
Parameter name which is used in the call to an ERR or MSG subroutine.
The name must be different from any other Program or Message Parameter name
specified for the program.

\texttt{sting} may by anything acceptable to the ERR or MSG routine 
({\em i.e.}\ it may include tokens {\em etc.}).

If the Message Specification for a given Message Parameter is omitted, the
text specified in the MSG or ERR routine will be used.

\subsubsection*{Example}
\begin{quote} \begin{verbatim}
MESSAGE APP_OUTMESS
   TEXT 'Calculated value is ^VALUE'
ENDMESSAGE
\end{verbatim} \end{quote}

\section{The ACTION Specification\xlabel{the_action_specification}
\label{action}}

The concept of ACTION is specific to instrumentation tasks (see
\xref{SUN/134}{sun134}{}).
The task is capable of obeying or cancelling a number of distinct commands
known as actions.
It does this, for example, in response to the ICL command:
\begin{quote} \begin{verbatim}
SEND taskname context actionname parameter_list
\end{verbatim} \end{quote}
where \texttt{context} is \texttt{OBEY} or \texttt{CANCEL}.

Actions have to be declared in the Interface File, following the parameter
declarations. The Action Specification is of the form: 
\begin{quote} \begin{verbatim}
ACTION actionname
   fieldname fieldvalue
   context specifications
      .
ENDACTION
\end{verbatim} \end{quote}
Where \texttt{actionname} is a character string specifying a valid action name.
The same rules should be applied to action names as are applied to parameter
names (see Appendix \ref{names}).

\subsection{The Context Specification and NEEDS Field
\xlabel{the_context_specification_and_needs_field}\label{needs}}

The Context Specifications allow command-line parameter positions to be 
allocated differently for different actions and contexts of the task.
The Context Specification is of the form:
\begin{quote} \begin{verbatim}
   context
      NEEDS parametername_1
      NEEDS parametername_2
       .
   ENDcontext
\end{verbatim} \end{quote}
where \texttt{context} is \texttt{OBEY} or \texttt{CANCEL}.

The order of the NEEDS fields within the Context Specification defines
the order in which parameters may be specified on the command line
for this particular action and context.

If no NEEDS list is specified, positions will default to those specified
in the Parameter Specification.

\subsubsection*{Example}
\begin{quote} \begin{verbatim}
ACTION FIRST
   OBEY
      NEEDS PAR1
      NEEDS PAR2
      NEEDS PAR3
   ENDOBEY
   CANCEL
      NEEDS PAR2
   ENDCANCEL
ENDACTION
\end{verbatim} \end{quote}
For historical reasons, NEEDS specifications can have RANGE or IN constraints 
but these have no effect.

\subsubsection*{Example}
\begin{quote} \begin{verbatim}
ACTION GET_FILTER
   OBEY
      NEEDS FILTER IN 'R', 'I', 'J'
      NEEDS DEAD_TIME RANGE 1.0, 5.5
   ENDOBEY
      .
ENDACTION
\end{verbatim} \end{quote}

\subsection{The Action KEYWORD Field\xlabel{the_action_keyword_field}}

This field specifies the name by which the action is known to 
the person running the I-task.

The field is of the form:
\begin{quote} \begin{verbatim}
KEYWORD  name
\end{verbatim} \end{quote}
where \texttt{name} is a character string specifying a valid keyword.
The same rules should be applied to action keywords as are applied to parameter
keywords (see Appendix \ref{names}).

\subsubsection*{Example}
\begin{quote} \begin{verbatim}
INTERFACE FILTASK
      .
   ACTION GET_FILTER
      KEYWORD FIND_FILTER
      OBEY
         NEEDS FILTER IN 'R', 'I', 'J'
         NEEDS DEAD_TIME RANGE 1.0, 5.5
      ENDOBEY
         .
   ENDACTION
      .
ENDINTERFACE
\end{verbatim} \end{quote}
This field is used to specify the word that the user uses to communicate 
with a particular action. Thus the user could specify:
\begin{quote} \begin{verbatim}
ICL> send filtask obey find_filter
\end{verbatim} \end{quote}
As can be seen, this completes the separation between the program's
view of actions and the user's view of them. It is possible for the
programmer to re-write the program using completely different action
names, but the user command could be kept the same by just changing the 
ACTION statements in the Interface File. Similarly the
user's view of the program can be changed by just changing the
action KEYWORD fields. 

If the KEYWORD field is not specified for an action, then the system will
use the action name as the keyword.

\subsection{The HELP Field\xlabel{the_help_field}}

This field is used to specify some help text for an action. No current 
user interfaces make use of this.

The field is of the form:
\begin{quote} \begin{verbatim}
HELP help-text
\end{verbatim} \end{quote}
where \texttt{help-text} is a character string giving help information about the
action.
\subsubsection*{Example}
\begin{quote} \begin{verbatim}
ACTION GET_FILTER
   KEYWORD FIND_FILTER
   HELP 'This causes the filter to be selected'
   OBEY
      NEEDS FILTER IN 'R', 'I', 'J'
      NEEDS DEAD_TIME RANGE 1.0, 5.5
   ENDOBEY
      .
ENDACTION
\end{verbatim} \end{quote}

\section{ENTRIES FOR MENUS\xlabel{entries_for_menus}}

There are two fields which are provided for possible future 
menu-style user interfaces. These can be used in both ACTION and 
PARAMETER specifications and are:
\begin{quote} \begin{verbatim}
MENU menuname
MENUCOORDS xcoord,ycoord
\end{verbatim} \end{quote}
where \texttt{menuname} is a character string specifying the name to be shown on
the screen, and \texttt{XCOORD} and \texttt{YCOORD} are integers giving the 
coordinates for the position of the menu name. 
These quantities are not used by the parameter system, but merely stored. 

\appendix
\newpage

\section{Parsing the Interface File\xlabel{parsing_the_interface_file}
\label{parsing}}

Interface File parsing is case-insensitive, the file is
interpreted as a sequence of {\em tokens}, where a token is a sequence of
characters which are either all alphanumeric (with \texttt{.;:+-()\_[]"'<>} being
honorary alphanumerics) or all non-alphanumeric ({\em i.e.}\ anything other
than alphanumeric and `white'). 
Tokens are thus terminated by anything of the opposite class or by a `white'
character. 
For this purpose, `white' characters are space, tab, newline, comma or 
non-printable.
Apart from their role as delimiters, `white' characters are not significant.

The one exception to this is that, as a special case, a token may be a
quoted string, {\em i.e.}\ a character string consisting of a set of characters
enclosed in single quotes. 
The first quote must be the first
character of the token and the token is terminated by the next isolated
({\em i.e.}\ not \texttt{''}) quote or the end of line.
Thus two consecutive quotes in a quoted string will be interpreted as a single
quote to be contained in the string.

Tokens may be one of:
\begin{itemize}
\item A reserved word. The reserved words are: 
\begin{quote}
ACCESS,  
ACTION,  
ASSOCIATION,\\ 
CANCEL,\\ 
DEFAULT,\\ 
ENDACTION,  
ENDCANCEL,  
ENDINTERFACE,  
ENDMESSAGE,\\  
ENDMONOLITH,  
ENDOBEY,  
ENDPARAMETER,  
EPATH,\\ 
HELP,  
HELPKEY,  
HELPLIB,\\
IN,  
INTERFACE,\\ 
KEYWORD,\\ 
MENU,  
MENUCOORDS,  
MESSAGE,  
MONOLITH,\\ 
NEEDS,\\ 
OBEY,\\ 
PARAMETER,  
POSITION,  
PPATH,  
PROGRAM,  
PROMPT,  
PTYPE,\\ 
RANGE,\\ 
TEXT,  
TYPE,\\ 
VPATH
\end{quote}
\item A number constant (a valid Fortran number).
\item A character constant (quoted string).
\item A logical constant (Y, YES, T, TRUE, N, NO, F or FALSE, regardless of 
case).
\item The constant !.
\item A name. Anything which is not one of the above is classified as a name
token.
Name tokens can include a full file specification. Where such tokens are
used as default values, case is preserved. Other names, such as parameter names
and keywords are converted to upper case.
\end{itemize}
Fields (but not tokens) may be split across lines if necessary.

Everything following \texttt{\#} on a line (except within a character
string) will be ignored.
Thus  \texttt{\#} can be used to introduce comments.

\subsection{Parameter, Action and Keyword Names
\xlabel{parameter_action_and_keyword_names}\label{names}}

The question of what is a valid parameter, action or keyword name is rather
complicated.
It is affected by Interface File parsing, the parameter system, HDS and the
particular user interface in use.
Appendix \ref{consts} gives the maximum lengths allowed and, to avoid problems,
it is recommended that names begin with a letter and continue with {\em true}
alphanumeric characters or underscore.

Parameter, action and keyword names are converted to upper case for storage.
If they are displayed, in prompts or error messages {\em etc.}, they will
appear in upper case.

\subsection{Error Reporting\xlabel{error_reporting}}

If an error is detected during the Interface File compilation phase, it is 
reported and, in most cases, the system is set into a state which is likely 
to enable it to continue. 
No further errors will be reported until a token acceptable in the new state
is found. At the end of compilation a message giving the number of errors
found will be reported.
For some errors, where recovery is unlikely, compilation will stop immediately.

Unless compilation is aborted, a program reading the Interface File at run
time, will go on and attempt to run with the information it has found and
\texttt{compifl} will go on to write the \texttt{.ifc} file which could be read
by the program later.
In both cases the information obtained by the program will probably be 
incomplete and result in some unexpected defaults being used.

\newpage

\section{Interface Module Search Path, ADAM\_IFL
\xlabel{interface_module_search_path}\label{searchpath}}

\subsection{Introduction\xlabel{introduction3}}

By default, ADAM application programs  attempt to find their Interface Module 
in the directory containing the executable file. A compiled ({\bf .ifc}) file
will be used in preference to a source ({\bf .ifl}) file.
To allow users to have different versions of the Interface Module for a given
program without having multiple copies of the executable file, a search path 
mechanism has been implemented. 
This will be particularly useful for private tailoring of the Interface 
Modules of released software.

The search path is entirely the responsibility of the user --
no value is defined in the standard system setup as there is a slight overhead
in having one defined when it is not required.

If the search path is not defined, or an Interface Module is not found using 
it, the program will attempt to use the default.

\subsection{Implementation for Unix\xlabel{implementation_for_unix}}

When a program is loaded, its name is obtained using system routine GETARG and
ignoring any path component of the name. 
If environment variable ADAM\_IFL is defined, it is
assumed to specify a search path as a list of directory names separated by
semi-colons.
Each directory in turn is searched for a file with the same name 
as the program and with extension {\bf .ifc} or, failing that, {\bf .ifl}. 
If such a file is found, it is used as the Interface Module.

\subsection{Implementation for VAX VMS\xlabel{implementation_for_vax_vms}}

When a program is loaded, it attempts to translate the logical name ADAM\_IFL 
using the tables defined by the SYSTEM logical name LNM\$FILE\_DEV (Normally
PROCESS, JOB, GROUP and SYSTEM in that order).

ADAM\_IFL may be a search path and, if it is defined, it is used as the 
filespec, together with a default filespec of `filename.IF\%' (where
`filename' is the filename of the executable image and `\%' is any single
character), in a call to LIB\$FIND\_FILE.

If such a file is found, the type is checked and, if the type is not .IFC
or .IFL, the routine continues searching until no more files matching 
the specification are found.
{\em Note that a .IFC file will be found before a .IFL file in any given
directory.}

Notes: 
\begin{enumerate}
\item It is recommended that, if required, ADAM\_IFL be defined as a JOB 
logical name.
It cannot be a PROCESS logical name if the program is to be run in a 
subprocess, and GROUP logical names may cause confusion as they remain set 
between sessions.
\item If ADAM\_IFL includes filenames, they will override the name of the 
executable image.
\end{enumerate}

\newpage

\section{Parameter Specification for Output Parameters
\xlabel{parameter_specification_for_output_parameters}\label{output}}

Users whose programs have called a PAR\_PUT routine are often surprised
to be prompted for `a parameter value'.
This appendix is an attempt to clarify what is happening and what to do about
it.

When the parameter system is asked for the value of a parameter of primitive
type, the question it asks of the user via the Interface File) is not 
actually ``What is the value of this parameter?'' but ``Where is the value of 
this parameter stored?''. 
If the parameter system is provided with an HDS object name, the 
correspondence between the question and the answer is obvious.
However, if a primitive constant is provided, the correspondence is not so
clear.
What is happening is that providing a primitive constant actually
means ``Store the given value in the program's parameter file\footnote{Or, in 
the case of scalar internal parameters, in memory},
creating a component of appropriate name, type and size if necessary, and tell 
the parameter system where it is.''.
The program side of the parameter system then gets the value from where it
is stored and delivers it to the program.

When a parameter value is to be output, the parameter system asks exactly the
same question. 
Remembering that the prompt is actually for a location and not a value,
the reason for the prompt is clearer.

It may be that output is required into some specific HDS object in which case
its name may be specified in the normal way. 
(The object must exist and be of appropriate type and size.)

More often, the user just wants the parameter value to be put into the
program's parameter file (possibly to update the GLOBAL parameter file 
eventually).
There are various ways of specifying the VPATH in order to obtain this effect.

It is possible to be prompted and respond with a primitive value
of an appropriate type.
This will create a parameter file component of the correct type, in
the same way as when a parameter value is `got', and tell the system that
the parameter file is to be used for the `put'.
(The actual value doesn't matter as it will be overwritten by the `put';
the size and shape of the component will be changed if necessary. However,
it is more efficient to specify the correct size if possible.)

If you do not want to be prompted for the parameter, the simplest way is
to use the DEFAULT field.
\begin{quote} \begin{verbatim}
PARAMETER parametername
      .
   ACCESS WRITE
   VPATH DEFAULT
   DEFAULT value
      .
ENDPARAMETER
\end{verbatim} \end{quote}
where \texttt{value} is a primitive value of an appropriate type. Again the
size does not matter (apart from efficiency), it will be changed if 
necessary.

\newpage
\section{Help Files for Multi-line Help
\xlabel{help_files_for_multiline_help}\label{help}}

On Unix, the multi-line parameter help system uses the Starlink Portable HELP 
System (see \xref{SUN/124}{sun124}{}).
If the specified file name has no extension or extension
\texttt{.shl} or \texttt{.hlb}, \texttt{.shl} is assumed; anything else is an error.

On VMS, either Portable HELP or the VMS help system
may be used.
If the given filename has extension \texttt{.shl}, Portable HELP is used:
if no file extension, or \texttt{.hlb}, is specified,
the system will first look for a file with the given filename and extension 
{\bf .shl}.
If the file is found it will be used with the Portable Help System.
In all other cases, the VMS help system is used.

In the interests of allowing the same Interface Modules to work on both VMS
and Unix, both systems will handle names starting with (or wholly) environment 
variables (logical names) in either system's style.

{\em E.g.}  
\begin{quote}
\texttt{\$environment\_variable/relative\_pathname}\/
\end{quote}
or
\begin{quote}
\texttt{logical\_name:filename}
\end{quote}

\newpage
\section{Parameter System Constants\xlabel{parameter_system_constants}
\label{consts}}

The common blocks used in ADAM application programs to hold information read 
from the Interface Module and required by the parameter system, impose certain 
limits on the size and number of parameters names, defaults and constraints 
{\em etc.} Further constraints are imposed by workspace requirements for the
Interface File parsing system.

Current values for important limits are published here, together with
the symbolic names used for the constants where appropriate.

\subsection*{Maximum Length of Names (characters)}
\begin{center}
\begin{tabbing}
xxxxxxxxxxxxxxxxxxxxxxxxxxxxxxxxxxxxxxxxxx \= xxxx \= xxxxxxxxxxxxxxxx \kill
Parameter names   \> 15 \> SUBPAR\_\_NAMELEN \\ 
Parameter keywords \> 15 \> SUBPAR\_\_NAMELEN \\
Action and program names   \> 15 \> SUBPAR\_\_NAMELEN \\ 
Action keywords   \> 15 \> SUBPAR\_\_NAMELEN \\ 
Menu names   \> 15 \> SUBPAR\_\_NAMELEN \\
Monolith names \> 80 \\
\end{tabbing} \end{center}

\subsection*{Maximum Lengths of Strings (characters)}
\begin{center}
\begin{tabbing}
xxxxxxxxxxxxxxxxxxxxxxxxxxxxxxxxxxxxxxxxxx \= xxxx \= xxxxxxxxxxxxxxxx \kill
Length of parameter value string \> 132 \\
Length of PROMPT string   \> 80 \\
Length of HELP string   \> 132 \\
Length of HELPKEY string \> 132 \\
Length of action HELP string \> 132 \\
\end{tabbing} \end{center}

\subsection*{Maximum Numbers of Items}
\begin{center}
\begin{tabbing}
xxxxxxxxxxxxxxxxxxxxxxxxxxxxxxxxxxxxxxxxxx \= xxxx \= xxxxxxxxxxxxxxxx \kill
Number of parameters \> 1500 \> SUBPAR\_\_MAXPAR \\
Number of actions \> 300 \> SUBPAR\_\_MAXACT \\
Number of menus \> 300 \> SUBPAR\_\_MAXACT \\
Number of command line parameters \> 50 \\
Size of NEEDS storage \> 300 \> SUBPAR\_\_MAXNEEDS \\
Size of storage for constraints/defaults \> 500 \> SUBPAR\_\_MAXLIMS \\
\end{tabbing} \end{center}

Note: There are SUBPAR\_\_MAXLIMS positions of each type available for
constraints/default storage. One position of the relevant type is used 
for every value given in IN, RANGE and DEFAULT fields or specified as a dynamic
default by the program.

\subsection*{Maximum Parsing Workspace}
\begin{center}
\begin{tabbing}
xxxxxxxxxxxxxxxxxxxxxxxxxxxxxxxxxxxxxxxxxx \= xxxx \= xxxxxxxxxxxxxxxx \kill
Length of line (characters) \> 132 \> PARSE\_\_BUFSIZ \\
Length of token (characters) \> 132 \> PARSE\_\_BUFSIZ \\
Number of tokens on a line\> 32 \\
\end{tabbing} \end{center}

\newpage
\section{Obsolete Fields\xlabel{obsolete_fields}}

Several fields which were relevant to the original Starlink Software
Environment (SSE) are not used with ADAM.
The fields are described here so that the ideas behind them may be preserved 
and because they may still be found in old software and documentation.

\subsection{The Program Specification
\xlabel{the_program_specification}\label{program}}

The ADAM application program to be run is determined by the user interface
when the appropriate command is issued.
The program then finds the corresponding Interface Module (see Appendix
\ref{searchpath}).
The SSE worked in a different way -- the user interface found the Interface
Module first and the Interface Module defined which executable image
was to be run.
The Program Specification fields controlled the name of the program to be
executed when a request to run an application is received, and where to find 
the executable image.

There are 2 Program Specification fields: PROGRAM and EPATH.

\subsubsection{The PROGRAM field}
This Field Specification gives the name of application program to be executed
when the user invokes this Interface Module.
It is still accepted by the system but has no effect.

The field is of the form:
\begin{quote} \begin{verbatim}
PROGRAM progname
\end{verbatim} \end{quote}
where \texttt{progname} is a character string (or reserved word) which is
the name of the executable image without the '.EXE' file extension.

\subsubsection*{Example}
\begin{quote} \begin{verbatim}
PROGRAM 'DECONV'
\end{verbatim} \end{quote}

\subsubsection{The EPATH field}
This field specification provided the search path for locating the
directory in which the executable image of the application program was to be 
found.
It is still accepted by the system but has no effect.

The field is of the form:
\begin{quote} \begin{verbatim}
EPATH dpath
\end{verbatim} \end{quote}
where \texttt{dpath} is a character constant listing the order of search through
subdirectories. The directory specifications in the path are separated
by semi-colons. A null path indicates the current directory. Note that
logical names must end with a colon. 

\subsubsection*{Example}
\begin{quote} \begin{verbatim}
EPATH 'DISK$USER1:[SLW.JUNK];SAI_LAPPLIC:'
\end{verbatim} \end{quote}

\subsection{The RPATH field\xlabel{the_rpath_field}}

This field provided a search path for data system object names.
It is no longer accepted by the system.
The field specifier was of the form:
\begin{quote} \begin{verbatim}
RPATH resolution-path
\end{verbatim} \end{quote}
where resolution-path was a character string giving a list of resolution path
specifiers separated by semi-colons.
Each specifier was a data-system object name.

\subsubsection*{Example}
\begin{quote} \begin{verbatim}
INTERFACE DISP
      .
   PARAMETER DISPLAY
         .
      RPATH 'TEMP.DEVICES;"SAI_USER:DEVICES";"LSAIDIR:DEVICES"'
         .
   ENDPARAMETER
      .
ENDINTERFACE
\end{verbatim} \end{quote}
Then if the parameter DISPLAY was given the value ARGS, the system 
attempted to use the object TEMP.DEVICES.ARGS then, failing that 
SAI\_USER:DEVICES.ARGS and so on.
(Note that ADAM no longer uses a `DEVICE dataset' object when opening graphics
devices.)

\newpage
\section{Changes in this Document\xlabel{changes_in_this_document}}
\subsection{SUN115.2}
Apart from the format, there has been little change in this document. 
Some points are clarified and references to out-of-date documents replaced or
removed. The move from DCL to Unix is also reflected.

The main significant changes are as follows:
\begin{description}
\item[The DEFAULT field] (Section \ref{default}
Quoted strings may now be given as defaults for
non-primitive types -- the string within the quotes will be taken as the 
`name'.
\item[NEEDS] (Section \ref{needs})
The changes introduced with ADAM V2.0 whereby the only effect of
NEEDS is to define command line positions, are reflected. There are
corresponding changes to the RANGE and IN descriptions.
\item[RANGE] (Section \ref{range})
The interaction between the dynamic minimum/maximum system and the RANGE is
described.
\item[ACCEPT] (Sections \ref{vpath} and \ref{ppath})
The ACCEPT keyword now only affects parameters for which a prompt would
otherwise have been issued. 
Parameters may also now be forced to ACCEPT by the \texttt{keyword=$\backslash$} parameter 
specifier or by typing `\texttt{$\backslash$}' in response to a prompt.
\item[Suggested Values] The term {\em prompt value} has been replaced by
{\em suggested value} which is now the preferred term.
\item[Help Files] Appendix \ref{help} has been added to give details of the 
filenames which may be given with the HELP, HELPKEY and HELPLIB fields.
\item[Output Parameters] Appendix \ref{output} has been altered to describe
the simpler method of avoiding prompts for output parameters now that the
type, shape and size of parameter file components will be changed automatically
if necessary when a parameter value is `put'.
\end{description}
\subsection{SUN/115.3}
Minor re-formatting and typo correction.
\end{document}
