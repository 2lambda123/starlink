\documentclass[twoside,11pt]{article}

% ? Specify used packages
% \usepackage{graphicx}        %  Use this one for final production.
% \usepackage[draft]{graphicx} %  Use this one for drafting.
% ? End of specify used packages

\pagestyle{myheadings}

% -----------------------------------------------------------------------------
% ? Document identification
% Fixed part
\newcommand{\stardoccategory}  {Starlink User Note}
\newcommand{\stardocinitials}  {SUN}
\newcommand{\stardocsource}    {sun\stardocnumber}
\newcommand{\stardoccopyright} 
{Copyright \copyright\ 2000 Council for the Central Laboratory of the Research Councils}

% Variable part - replace [xxx] as appropriate.
\newcommand{\stardocnumber}    {130.5}
\newcommand{\stardocauthors}   {David Terrett\\Nicholas Eaton}
\newcommand{\stardocdate}      {7 July 1995}
\newcommand{\stardoctitle}     {GWM --- X Graphics Window Manager}
\newcommand{\stardocversion}   {Version 1.5-1}
\newcommand{\stardocmanual}    {Programmers' Manual}
\newcommand{\stardocabstract}  {This note describes how to create graphics
windows on an X display that do not disappear when a program terminates.}

% ? End of document identification
% -----------------------------------------------------------------------------

% +
%  Name:
%     sun.tex
%
%  Purpose:
%     Template for Starlink User Note (SUN) documents.
%     Refer to SUN/199
%
%  Authors:
%     AJC: A.J.Chipperfield (Starlink, RAL)
%     BLY: M.J.Bly (Starlink, RAL)
%     PWD: Peter W. Draper (Starlink, Durham University)
%
%  History:
%     17-JAN-1996 (AJC):
%        Original with hypertext macros, based on MDL plain originals.
%     16-JUN-1997 (BLY):
%        Adapted for LaTeX2e.
%        Added picture commands.
%     13-AUG-1998 (PWD):
%        Converted for use with LaTeX2HTML version 98.2 and
%        Star2HTML version 1.3.
%      1-FEB-2000 (AJC):
%        Add Copyright statement in LaTeX
%     {Add further history here}
%
% -

\newcommand{\stardocname}{\stardocinitials /\stardocnumber}
\markboth{\stardocname}{\stardocname}
\setlength{\textwidth}{160mm}
\setlength{\textheight}{230mm}
\setlength{\topmargin}{-2mm}
\setlength{\oddsidemargin}{0mm}
\setlength{\evensidemargin}{0mm}
\setlength{\parindent}{0mm}
\setlength{\parskip}{\medskipamount}
\setlength{\unitlength}{1mm}

% -----------------------------------------------------------------------------
%  Hypertext definitions.
%  ======================
%  These are used by the LaTeX2HTML translator in conjunction with star2html.

%  Comment.sty: version 2.0, 19 June 1992
%  Selectively in/exclude pieces of text.
%
%  Author
%    Victor Eijkhout                                      <eijkhout@cs.utk.edu>
%    Department of Computer Science
%    University Tennessee at Knoxville
%    104 Ayres Hall
%    Knoxville, TN 37996
%    USA

%  Do not remove the %begin{latexonly} and %end{latexonly} lines (used by 
%  LaTeX2HTML to signify text it shouldn't process).
%begin{latexonly}
\makeatletter
\def\makeinnocent#1{\catcode`#1=12 }
\def\csarg#1#2{\expandafter#1\csname#2\endcsname}

\def\ThrowAwayComment#1{\begingroup
    \def\CurrentComment{#1}%
    \let\do\makeinnocent \dospecials
    \makeinnocent\^^L% and whatever other special cases
    \endlinechar`\^^M \catcode`\^^M=12 \xComment}
{\catcode`\^^M=12 \endlinechar=-1 %
 \gdef\xComment#1^^M{\def\test{#1}
      \csarg\ifx{PlainEnd\CurrentComment Test}\test
          \let\html@next\endgroup
      \else \csarg\ifx{LaLaEnd\CurrentComment Test}\test
            \edef\html@next{\endgroup\noexpand\end{\CurrentComment}}
      \else \let\html@next\xComment
      \fi \fi \html@next}
}
\makeatother

\def\includecomment
 #1{\expandafter\def\csname#1\endcsname{}%
    \expandafter\def\csname end#1\endcsname{}}
\def\excludecomment
 #1{\expandafter\def\csname#1\endcsname{\ThrowAwayComment{#1}}%
    {\escapechar=-1\relax
     \csarg\xdef{PlainEnd#1Test}{\string\\end#1}%
     \csarg\xdef{LaLaEnd#1Test}{\string\\end\string\{#1\string\}}%
    }}

%  Define environments that ignore their contents.
\excludecomment{comment}
\excludecomment{rawhtml}
\excludecomment{htmlonly}

%  Hypertext commands etc. This is a condensed version of the html.sty
%  file supplied with LaTeX2HTML by: Nikos Drakos <nikos@cbl.leeds.ac.uk> &
%  Jelle van Zeijl <jvzeijl@isou17.estec.esa.nl>. The LaTeX2HTML documentation
%  should be consulted about all commands (and the environments defined above)
%  except \xref and \xlabel which are Starlink specific.

\newcommand{\htmladdnormallinkfoot}[2]{#1\footnote{#2}}
\newcommand{\htmladdnormallink}[2]{#1}
\newcommand{\htmladdimg}[1]{}
\newcommand{\hyperref}[4]{#2\ref{#4}#3}
\newcommand{\htmlref}[2]{#1}
\newcommand{\htmlimage}[1]{}
\newcommand{\htmladdtonavigation}[1]{}

\newenvironment{latexonly}{}{}
\newcommand{\latex}[1]{#1}
\newcommand{\html}[1]{}
\newcommand{\latexhtml}[2]{#1}
\newcommand{\HTMLcode}[2][]{}

%  Starlink cross-references and labels.
\newcommand{\xref}[3]{#1}
\newcommand{\xlabel}[1]{}

%  LaTeX2HTML symbol.
\newcommand{\latextohtml}{\LaTeX2\texttt{HTML}}

%  Define command to re-centre underscore for Latex and leave as normal
%  for HTML (severe problems with \_ in tabbing environments and \_\_
%  generally otherwise).
\renewcommand{\_}{\texttt{\symbol{95}}}

% -----------------------------------------------------------------------------
%  Debugging.
%  =========
%  Remove % on the following to debug links in the HTML version using Latex.

% \newcommand{\hotlink}[2]{\fbox{\begin{tabular}[t]{@{}c@{}}#1\\\hline{\footnotesize #2}\end{tabular}}}
% \renewcommand{\htmladdnormallinkfoot}[2]{\hotlink{#1}{#2}}
% \renewcommand{\htmladdnormallink}[2]{\hotlink{#1}{#2}}
% \renewcommand{\hyperref}[4]{\hotlink{#1}{\S\ref{#4}}}
% \renewcommand{\htmlref}[2]{\hotlink{#1}{\S\ref{#2}}}
% \renewcommand{\xref}[3]{\hotlink{#1}{#2 -- #3}}
%end{latexonly}
% -----------------------------------------------------------------------------
% ? Document specific \newcommand or \newenvironment commands.

\newcommand{\aster}{$*$}
\begin{htmlonly}
  \newcommand{\aster}{*}
\end{htmlonly}

%+
%  Name:
%     SST.TEX

%  Purpose:
%     Define LaTeX commands for laying out Starlink routine descriptions.

%  Language:
%     LaTeX

%  Type of Module:
%     LaTeX data file.

%  Description:
%     This file defines LaTeX commands which allow routine documentation
%     produced by the SST application PROLAT to be processed by LaTeX and
%     by LaTeX2html. The contents of this file should be included in the
%     source prior to any statements that make of the sst commnds.

%  Notes:
%     The style file html.sty provided with LaTeX2html needs to be used.
%     This must be before this file.

%  Authors:
%     RFWS: R.F. Warren-Smith (STARLINK)
%     PDRAPER: P.W. Draper (Starlink - Durham University)

%  History:
%     10-SEP-1990 (RFWS):
%        Original version.
%     10-SEP-1990 (RFWS):
%        Added the implementation status section.
%     12-SEP-1990 (RFWS):
%        Added support for the usage section and adjusted various spacings.
%     8-DEC-1994 (PDRAPER):
%        Added support for simplified formatting using LaTeX2html.
%     {enter_further_changes_here}

%  Bugs:
%     {note_any_bugs_here}

%-

%  Define length variables.
\newlength{\sstbannerlength}
\newlength{\sstcaptionlength}
\newlength{\sstexampleslength}
\newlength{\sstexampleswidth}

%  Define a \tt font of the required size.
\latex{\newfont{\ssttt}{cmtt10 scaled 1095}}
\html{\newcommand{\ssttt}{\tt}}

%  Define a command to produce a routine header, including its name,
%  a purpose description and the rest of the routine's documentation.
\newcommand{\sstroutine}[3]{
   \goodbreak
   \rule{\textwidth}{0.5mm}
   \vspace{-7ex}
   \newline
   \settowidth{\sstbannerlength}{{\Large {\bf #1}}}
   \setlength{\sstcaptionlength}{\textwidth}
   \setlength{\sstexampleslength}{\textwidth}
   \addtolength{\sstbannerlength}{0.5em}
   \addtolength{\sstcaptionlength}{-2.0\sstbannerlength}
   \addtolength{\sstcaptionlength}{-5.0pt}
   \settowidth{\sstexampleswidth}{{\bf Examples:}}
   \addtolength{\sstexampleslength}{-\sstexampleswidth}
   \parbox[t]{\sstbannerlength}{\flushleft{\Large {\bf #1}}}
   \parbox[t]{\sstcaptionlength}{\center{\Large #2}}
   \parbox[t]{\sstbannerlength}{\flushright{\Large {\bf #1}}}
   \begin{description}
      #3
   \end{description}
}

%  Format the description section.
\newcommand{\sstdescription}[1]{\item[Description:] #1}

%  Format the usage section.
\newcommand{\sstusage}[1]{\item[Usage:] \mbox{}
\\[1.3ex]{\raggedright \ssttt #1}}

%  Format the invocation section.
\newcommand{\sstinvocation}[1]{\item[Invocation:]\hspace{0.4em}{\tt #1}}

%  Format the arguments section.
\newcommand{\sstarguments}[1]{
   \item[Arguments:] \mbox{} \\
   \vspace{-3.5ex}
   \begin{description}
      #1
   \end{description}
}

%  Format the returned value section (for a function).
\newcommand{\sstreturnedvalue}[1]{
   \item[Returned Value:] \mbox{} \\
   \vspace{-3.5ex}
   \begin{description}
      #1
   \end{description}
}

%  Format the parameters section (for an application).
\newcommand{\sstparameters}[1]{
   \item[Parameters:] \mbox{} \\
   \vspace{-3.5ex}
   \begin{description}
      #1
   \end{description}
}

%  Format the examples section.
\newcommand{\sstexamples}[1]{
   \item[Examples:] \mbox{} \\
   \vspace{-3.5ex}
   \begin{description}
      #1
   \end{description}
}

%  Define the format of a subsection in a normal section.
\newcommand{\sstsubsection}[1]{ \item[{#1}] \mbox{} \\}

%  Define the format of a subsection in the examples section.
\newcommand{\sstexamplesubsection}[2]{\sloppy
\item[\parbox{\sstexampleslength}{\ssttt #1}] \mbox{} \vspace{1.0ex}
\\ #2 }

%  Format the notes section.
\newcommand{\sstnotes}[1]{\item[Notes:] \mbox{} \\[1.3ex] #1}

%  Provide a general-purpose format for additional (DIY) sections.
\newcommand{\sstdiytopic}[2]{\item[{\hspace{-0.35em}#1\hspace{-0.35em}:}]
\mbox{} \\[1.3ex] #2}

%  Format the implementation status section.
\newcommand{\sstimplementationstatus}[1]{
   \item[{Implementation Status:}] \mbox{} \\[1.3ex] #1}

%  Format the bugs section.
\newcommand{\sstbugs}[1]{\item[Bugs:] #1}

%  Format a list of items while in paragraph mode.
\newcommand{\sstitemlist}[1]{
  \mbox{} \\
  \vspace{-3.5ex}
  \begin{itemize}
     #1
  \end{itemize}
}

%  Define the format of an item.
\newcommand{\sstitem}{\item}

%% Now define html equivalents of those already set. These are used by
%  latex2html and are defined in the html.sty files.
\begin{htmlonly}

%  sstroutine.
   \newcommand{\sstroutine}[3]{
      \subsection{#1\xlabel{#1}-\label{#1}#2}
      \begin{description}
         #3
      \end{description}
   }

%  sstdescription
   \newcommand{\sstdescription}[1]{\item[Description:]
      \begin{description}
         #1
      \end{description}
      \\
   }

%  sstusage
   \newcommand{\sstusage}[1]{\item[Usage:]
      \begin{description}
         {\ssttt #1}
      \end{description}
      \\
   }

%  sstinvocation
   \newcommand{\sstinvocation}[1]{\item[Invocation:]
      \begin{description}
         {\ssttt #1}
      \end{description}
      \\
   }

%  sstarguments
   \newcommand{\sstarguments}[1]{
      \item[Arguments:] \\
      \begin{description}
         #1
      \end{description}
      \\
   }

%  sstreturnedvalue
   \newcommand{\sstreturnedvalue}[1]{
      \item[Returned Value:] \\
      \begin{description}
         #1
      \end{description}
      \\
   }

%  sstparameters
   \newcommand{\sstparameters}[1]{
      \item[Parameters:] \\
      \begin{description}
         #1
      \end{description}
      \\
   }

%  sstexamples
   \newcommand{\sstexamples}[1]{
      \item[Examples:] \\
      \begin{description}
         #1
      \end{description}
      \\
   }

%  sstsubsection
   \newcommand{\sstsubsection}[1]{\item[{#1}]}

%  sstexamplesubsection
   \newcommand{\sstexamplesubsection}[2]{\item[{\ssttt #1}] #2}

%  sstnotes
   \newcommand{\sstnotes}[1]{\item[Notes:] #1 }

%  sstdiytopic
   \newcommand{\sstdiytopic}[2]{\item[{#1}] #2 }

%  sstimplementationstatus
   \newcommand{\sstimplementationstatus}[1]{
      \item[Implementation Status:] #1
   }

%  sstitemlist
   \newcommand{\sstitemlist}[1]{
      \begin{itemize}
         #1
      \end{itemize}
      \\
   }
%  sstitem
   \newcommand{\sstitem}{\item}

\end{htmlonly}

%  End of "sst.tex" layout definitions.
%.

% ? End of document specific commands
% -----------------------------------------------------------------------------
%  Title Page.
%  ===========
\renewcommand{\thepage}{\roman{page}}
\begin{document}
\thispagestyle{empty}

%  Latex document header.
%  ======================
\begin{latexonly}
   CCLRC / \textsc{Rutherford Appleton Laboratory} \hfill \textbf{\stardocname}\\
   {\large Particle Physics \& Astronomy Research Council}\\
   {\large Starlink Project\\}
   {\large \stardoccategory\ \stardocnumber}
   \begin{flushright}
   \stardocauthors\\
   \stardocdate
   \end{flushright}
   \vspace{-4mm}
   \rule{\textwidth}{0.5mm}
   \vspace{5mm}
   \begin{center}
   {\Huge\textbf{\stardoctitle \\ [2.5ex]}}
   {\LARGE\textbf{\stardocversion \\ [4ex]}}
   {\Huge\textbf{\stardocmanual}}
   \end{center}
   \vspace{5mm}

% ? Add picture here if required for the LaTeX version.
%   e.g. \includegraphics[scale=0.3]{filename.ps}
% ? End of picture

% ? Heading for abstract if used.
   \vspace{10mm}
   \begin{center}
      {\Large\textbf{Abstract}}
   \end{center}
% ? End of heading for abstract.
\end{latexonly}

%  HTML documentation header.
%  ==========================
\begin{htmlonly}
   \xlabel{}
   \begin{rawhtml} <H1> \end{rawhtml}
      \stardoctitle\\
      \stardocversion\\
      \stardocmanual
   \begin{rawhtml} </H1> <HR> \end{rawhtml}

% ? Add picture here if required for the hypertext version.
%   e.g. \includegraphics[scale=0.7]{filename.ps}
% ? End of picture

   \begin{rawhtml} <P> <I> \end{rawhtml}
   \stardoccategory\ \stardocnumber \\
   \stardocauthors \\
   \stardocdate
   \begin{rawhtml} </I> </P> <H3> \end{rawhtml}
      \htmladdnormallink{CCLRC / Rutherford Appleton Laboratory}
                        {http://www.cclrc.ac.uk} \\
      \htmladdnormallink{Particle Physics \& Astronomy Research Council}
                        {http://www.pparc.ac.uk} \\
   \begin{rawhtml} </H3> <H2> \end{rawhtml}
      \htmladdnormallink{Starlink Project}{http://www.starlink.rl.ac.uk/}
   \begin{rawhtml} </H2> \end{rawhtml}
   \htmladdnormallink{\htmladdimg{source.gif} Retrieve hardcopy}
      {http://www.starlink.rl.ac.uk/cgi-bin/hcserver?\stardocsource}\\

%  HTML document table of contents. 
%  ================================
%  Add table of contents header and a navigation button to return to this 
%  point in the document (this should always go before the abstract \section). 
  \label{stardoccontents}
  \begin{rawhtml} 
    <HR>
    <H2>Contents</H2>
  \end{rawhtml}
  \htmladdtonavigation{\htmlref{\htmladdimg{contents_motif.gif}}
        {stardoccontents}}

% ? New section for abstract if used.
  \section{\xlabel{abstract}Abstract}
% ? End of new section for abstract
\end{htmlonly}

% -----------------------------------------------------------------------------
% ? Document Abstract. (if used)
%  ==================
\stardocabstract
% ? End of document abstract

% -----------------------------------------------------------------------------
% ? Latex Copyright Statement
%  =========================
\begin{latexonly}
\newpage
\vspace*{\fill}
\stardoccopyright
\end{latexonly}
% ? End of Latex copyright statement

% -----------------------------------------------------------------------------
% ? Latex document Table of Contents (if used).
%  ===========================================
  \newpage
  \begin{latexonly}
    \setlength{\parskip}{0mm}
    \tableofcontents
    \setlength{\parskip}{\medskipamount}
    \markboth{\stardocname}{\stardocname}
  \end{latexonly}
% ? End of Latex document table of contents
% -----------------------------------------------------------------------------

\cleardoublepage
\renewcommand{\thepage}{\arabic{page}}
\setcounter{page}{1}

\section{Introduction}

X was designed to support ``graphical user interfaces'' (GUIs) and there are two
properties of X that make an X window fundamentally different from a
traditional graphics device.
Firstly, windows ``belong'' to applications programs and are deleted when the
application exits unlike the picture on an image display which remains there
until replaced by something else. 

Secondly, applications
programs are expected respond to ``events'' occurring in their windows;
things like key presses, mouse movements, etc. One event type that all
applications must handle is the ``window expose'' event which occurs whenever
part of a window that was invisible---because another application's window was
obscuring it for example---becomes visible. The application is responsible for
restoring the contents of the newly exposed part of the window but only an
application designed as an X application can do this; an application that is
using X via a graphics package such as GKS (\xref{SUN/83}{sun83}{})
or IDI (\xref{SUN/65}{sun65}{}) but is otherwise a
conventional application that knows nothing of X, cannot.

The X graphics window manager (not to be confused with the Window
Manager which allows you to move windows around the screen, iconize them etc.)
makes a window on an X windows display behave like a
traditional graphics device by making the lifetime of the window independent
of any applications program and by handling window expose events. Applications
still send plotting commands directly to the window; they don't have to go via
a ``server'' process, so there is no adverse impact on performance. All
communication between the window manager and the application is via the X
server and the graphics window manager does not have to run on the same 
machine as the application.
Indeed the manager and applications can even be running on different operating
systems. 

\section{Creating a window\xlabel{xmakeCommand}} 

The simplest way to create a window (in terms of understanding what is
happening) is to use the {\tt xmake} command and specify the properties that
you want the window to have. There are other, more convenient, ways to create
windows but they are easier to explain after describing the {\tt xmake} command.
\begin{quote}{\tt
xmake {\em name} {\em options}}
\end{quote}
creates a window with the name {\em name}. Case matters in window names so {\tt
mywindow}, {\tt MYWINDOW} and {\tt MyWindow} are all different.

The following options, which can be abbreviated, are recognized:
\begin{description}

\item[{\tt -background {\em colour}}] The background colour of the window; {\em
colour} is the name of a colour recognized by X windows. Examples:
\begin{quote}{\tt
xmake test -background black\\
xmake test -background yellow}
\end{quote}
The default is {\tt Black}.

\item[{\tt -bg {\em colours}}] The same as {\tt -background}.

\item[{\tt -borderwidth}] Specifies the width of the window border in
pixels. The
border width is ignored by many window managers but the value specified
is used to calculate the window position so should match the actual
border width used by the window manager. The default is 10 which is what
the Motif window manager uses. 

If the window appears to
be incorrectly position when a geometry specification of {\tt -geometry
-0-0} is used then changing the border width may correct the problem.

\item[{\tt -colours {\em number}}] The number of colour table entries to be
allocated to the window; {\em number} should be an integer greater than or
equal to 2. 

On a server with a writable colour table, the requested number of colour table
entries are reserved for the exclusive use of the window; if there are 
insufficient free entries in the default colour map, a private colour map
is created and the entries allocated from that.
On a server with a fixed colour table, entries can
be shared between windows and any number up to the maximum supported by the
server can be requested. 

Examples:
\begin{quote}{\tt
xmake test -colours 2\\
xmake test -col 127}
\end{quote}
The default depends on the properties of the X server.

\item[{\tt -fg {\em colours}}] The same as {\tt -foreground}.

\item[{\tt -foreground {\em colour}}] The foreground colour of the window; {\em
colour} is the name of a colour or RGB specification in X windows format.
Case does not matter but colour names cannot be abbreviated. 
Examples:
\begin{quote}{\tt
xmake test -foreground white\\
xmake test -foreground SpringGreen\\
xmake test -foreground \#FF00FF}
\end{quote}
The default is {\tt White}.

\item[{\tt -geometry {\em specification}}\label{geometryOpt}] 
The size and position of the window;
{\em specification} is a window geometry specification of the form:
\begin{quote}{\tt
{\em width}x{\em height}+/-{\em xorigin}+/-{\em yorigin}}
\end{quote}
({\tt x} is the letter ``x'' and {\tt +/-} is either a plus sign or a
minus sign.)
All dimensions are in pixels and the origin is the position of the top left hand
corner of the window relative to the top left hand corner of the screen
if the origin specification is a positive number, or the position of the
bottom right corner of the window relative to the bottom right of the
screen if it is negative. All
the components of the geometry specification are optional. Examples:
\begin{quote}{\tt
xmake test -geometry 780x512\\
xmake test -geom 600x400+300+300\\
xmake test -geom +512+368}
\end{quote}
To position the window in the bottom right corner of the screen:
\begin{quote}{\tt
xmake test -geom -0-0}
\end{quote}
To position the window in the top right corner:
\begin{quote}{\tt
xmake test -geom -0+0}
\end{quote}
The default is {\tt 780x512} (this leaves it up to the window manager to
decide where to place the window). 

The geometry specification is only a hint to the window manager and may be
ignored.

\item[{\tt -iconic}] Create the window as an icon.

\item[{\tt -interactive}] Allows the window size to be set interactively. When
the window appears on the screen its size can be modified before the window
creation is completed and the window size fixed. Resize the window with the
facilities provided by the window manager that you are using 
and then click the mouse on the
interior of the window to indicate that the desired size has been set.

This option is ignored if {\tt -iconic} is specified.

\item[{\tt -nointeractive}] The opposite of {\tt -interactive}. This is
the default.

\item[{\tt -nooverlay}] Do not allocate an overlay plane. This option
would be required if a default overlay colour had been set using the ``X
defaults'' file \htmlref{(see below)}{xdefaults} but a window without an 
overlay plane was required.

\item[{\tt -ovcolour {\em colour}}]
Set the colour of the overlay plane. If not specified the overlay plane is set
to the foreground colour. Example:
\begin{quote}{\tt
xmake test -ovcolour Green}
\end{quote}

\item[{\tt -overlay}] Allocate an overlay plane
There is no default for this option. The default is not to
allocate an overlay plane. Note that overlay planes effectively consume half
the available colours on the X server so an overlay should be specified only
when required.

\item[{\tt -title {\em title}}] The window title that is displayed in the
window's title bar;
{\em title} is a character string. If it contains spaces or other
special characters it must be enclosed in quotes 
Example:
\begin{quote}{\tt
xmake test -title Test\_Window}
\end{quote}
The default title is ``{\tt{GWM Window - }{\em window name}}''

\end{description}

\label{xdefaults}\xlabel{xdefaults}There are defaults 
for all these options but they are not 
necessarily appropriate. You can provide your own defaults by putting
them in your ``X defaults'' file ({\tt\~{}/.Xdefaults}) 
and to have different defaults for different window names. If you 
edit your X defaults file, the changes won't have any effect until they
have been loaded into the resource database in your X server. This is normally
done by the window manager when you first log in to the server but can 
be done explicitly with the command:
\begin{quote}{\tt
xrdb -merge \$HOME/.Xdefaults}
\end{quote}

Rather than attempting to explain in detail what is a powerful but quite
complicated system here is an example that shows how to control the properties
of GWM windows. Suppose your X defaults file contains the following:
\begin{quote}{\tt
Gwm.mono*colours: 2\\
Gwm*colours: 128}
\end{quote}
then a window created with the name ``mono'' will have 2 colours allocated to
it and windows with any other names will have 128 colours allocated.

When a graphics package creates a window for you this is the only mechanism
available for specifying the properties of the window. For example, GKS will
create a window if it doesn't already exist and each of the four available
workstation types of X windows can be assigned different properties as follows:
\begin{quote}{\tt
Gwm*foreground: Black\\
Gwm*background: White\\
Gwm.GKS\_3800*colours: 2\\
Gwm.GKS\_3800*geometry: 512x380\\
Gwm.GKS\_3801*colours: 8\\
Gwm.GKS\_3801*geometry: 780x512\\
Gwm.GKS\_3802*colours: 128\\
Gwm.GKS\_3802*geometry: 780x512\\
Gwm.GKS\_3803*colours: 128\\
Gwm.GKS\_3803*interactive: True}
\end{quote}
This defines workstation type 3800 as being a small window with only two
colours, suitable for black and white line plots, 3801 is a bit bigger and has
8 colours suitable for colour line plots, 3802 is the same size but with more
colours for image display applications and 3803 is the same but the window size
is set interactively.

A complete list of the resources used and their default values are shown in 
Table~\ref{resources}.
\begin{htmlonly}
\subsection{GWM X resources}
\end{htmlonly}
\begin{table}\caption{GWM X resources\label{resources}}\tt
\[\begin{tabular}{|l|l|}\hline
\rm Resource name &\rm Default\\ \hline
background &Black\\
borderWidth &10\\
colours & 0\\
foreground &White\\
geometry &780x512\\
overlay &False\\
overlayColour &\\
iconic &False\\
interactive &False\\
title &\\\hline 
\end{tabular}\]
\end{table}

\section{Using windows}

Programs that use GWM (usually via a graphics package such as GKS or IDI) will
normally create a new window as required. However, to plot into an existing
window you need to know how the graphics package generates the window name. For
example, GKS  uses a window with the name {\tt GKS\_{\em{nnnn}}} where {\em
nnnn} is the workstation type (the connection identifier is ignored) or
whatever that name translates to if treated as an environment variable.

Consult the documentation for the graphics package in question for full
details.

\section{Resizing windows}
Once a window has been created, any resizing does not alter the size of the
plotting area seen by applications programs; all that happens is either part of
the plotting area becomes invisible or a blank border appears around the
plotting area.

You should avoid resizing a window while an application is attached (i.e.
plotting into) the window as the application may not be able to handle the
resulting change in the picture position correctly.

\section{Deleting windows\xlabel{xdestroyCommand}}
Windows are deleted from the display with the command:
\begin{quote}{\tt
xdestroy {\em name}}
\end{quote}
where {\em name} is the window name. There are no command options.

Resetting the X server (i.e. by logging out) will also delete all windows 
and all window manager processes will exit.

\section{The FORTRAN interface}

This section describes the subroutines that can be called from a FORTRAN
program to interface with the Graphics Window Manager. All these routines
use the {\em inherited status strategy} which means that if the {\tt STATUS}
argument is not set to the value {\tt SAI\_\_OK} on entry the routine will
exit without performing any action. If an error occurs during the execution
of a routine the {\tt STATUS} will be set and an error will be reported
using the error message service 
(EMS---\xref{SUN/104}{sun104}{}).

\subsection{Summary of GWM calls}

\newenvironment{routinelist}{\begin{list}{}{\setlength{\leftmargin}{2cm}
                             \setlength{\parsep}{\smallskipamount}}}{\end{list}}
\newcommand{\routine}[2]{\item\hspace{-1cm}\htmlref{#1}{#2}\\}

\begin{routinelist}
\routine{GWM\_CLOSE( STATUS )}{GWM_CLOSE}
   Close the X client-server connection.
\routine{GWM\_CRWIN( WNAME, STATUS )}{GWM_CRWIN}
   Create a GWM window.
\routine{GWM\_DSWIN( WNAME, STATUS )}{GWM_DSWIN}
   Destroy a GWM window.
\routine{GWM\_EXIST( WNAME, EXISTS, STATUS )}{GWM_EXIST}
   Inquire if a GWM window of the given name exists.
\routine{GWM\_GETCI( WNAME, IDIM, INDEXS, NCOLS, STATUS )}{GWM_GETCI}
   Inquire the number of colours and colour indices.
\routine{GWM\_OPEN( DISPLY, USEDEF, STATUS )}{GWM_OPEN}
   Establish the X client server connection.
\routine{GWM\_WSETC( OPTION, VALUE, STATUS )}{GWM_WSETC}
   Set a character string window option.
\routine{GWM\_WSETI( OPTION, VALUE, STATUS )}{GWM_WSETI}
   Set an integer window option.
\routine{GWM\_WSETL( OPTION, VALUE, STATUS )}{GWM_WSETL}
   Set a logical window option.
\end{routinelist}

\subsection{Making the X connection}
The first thing to do when using the FORTRAN interface is to call
the routine \texttt{\htmlref{GWM\_OPEN}{GWM_OPEN}}
which establishes the connection between
the client and the server. In this context the client is the cpu running
the program and the server is the cpu displaying the image (X allows
these cpu's to be different). The first argument passed to {\tt GWM\_OPEN}
defines this connection, however most applications need not worry about
the details of the connection and can use the default (the option of
specifying the connection path is included for completeness). The default
connection is established with the following call
\begin{verbatim}
      CALL GWM_OPEN( ' ', .TRUE., STATUS )
\end{verbatim}
where the second argument is set to true to indicate that the default
connection is to be used.

The connection is ended with a call to 
\texttt{\htmlref{GWM\_CLOSE}{GWM_CLOSE}}.
These two routines
({\tt GWM\_OPEN} and {\tt GWM\_CLOSE}) should bracket all other routines
called from this interface.

As well as establishing the X connection {\tt GWM\_OPEN} also sets up
an error handler to deal with any non-fatal X errors. Once this has been
set up any such errors are reported using EMS. When {\tt GWM\_CLOSE} is
called the previously active error handler is re-established.

\subsection{Creating a GWM window}
The routine \texttt{\htmlref{GWM\_CRWIN}{GWM_CRWIN}}
is used to create a GWM window on the
server. The following example shows how a window with the name 'XWINDOWS'
is created (remember GWM window names are case sensitive).
\begin{verbatim}
      CALL GWM_CRWIN( 'XWINDOWS', STATUS )
\end{verbatim}
On its own this call will create a window with the default characteristics
of shape, colour etc. To select different characteristics calls to the
routines \texttt{\htmlref{GWM\_WSETx}{GWM_WSETC}} are made before 
the call to {\tt GWM\_CRWIN}.

Each required window characteristic is selected with a separate call to
one of the {\tt GWM\_WSETx} routines, where 'x' represents the type of
the characteristic, 'I' for integer, 'L' for logical and 'C' for character
string. The following table lists the characteristics and their types.

\begin{center}
\begin{tabular}{|l|l|}   \hline
Characteristic & Type \\ \hline
BACKGROUND     & C    \\
BORDERWIDTH    & I    \\
COLOURS        & I    \\
FOREGROUND     & C    \\
HEIGHT         & I    \\
ICONIC         & L    \\
INTERACTIVE    & L    \\
NOINTERACTIVE  & L    \\
NOOVERLAY      & L    \\
OVCOLOUR       & C    \\
OVERLAY        & L    \\
TITLE          & C    \\ 
WIDTH          & I    \\
XORIGIN        & I    \\
XSIGN          & I    \\
YORIGIN        & I    \\
YSIGN          & I    \\ \hline
\end{tabular}
\end{center}

Most of these characteristics are directly comparable to the options
of the {\tt xmake} command. The only major difference is the 
\texttt{\htmlref{-geometry}{geometryOpt}}
option which here is split into its four components WIDTH, HEIGHT, XORIGIN
and YORIGIN. The XSIGN and YSIGN characteristic is used to override the sign of
XORIGIN and YORIGIN values; setting XSIGN or YSIGN
to any value forces the corresponding origin value to have the same
sign. This is only required when setting one of the
origin characteristics to -0 in order to position the window at the
bottom or right hand side of the screen.

The following example shows the use of most the options to create a window.

\begin{small}
\begin{verbatim}
      *   Set up the size and position of the window
            CALL GWM_WSETI( 'WIDTH', 256, STATUS )
            CALL GWM_WSETI( 'HEIGHT', 192, STATUS )
            CALL GWM_WSETI( 'XORIGIN', 200, STATUS )
            CALL GWM_WSETI( 'YORIGIN', 0, STATUS )
            CALL GWM_WSETI( 'YSIGN', -1, STATUS)

      *   Allocate 128 colours to this window
            CALL GWM_WSETI( 'COLOURS', 128, STATUS )

      *   Define the foreground and background colours
            CALL GWM_WSETC( 'FOREGROUND', 'White', STATUS )
            CALL GWM_WSETC( 'BACKGROUND', 'DarkSlateGrey', STATUS )

      *   Give the window a title
            CALL GWM_WSETC( 'TITLE', 'Small_Window', STATUS )

      *   Create the GWM window
            CALL GWM_CRWIN( 'XWINDOWS', STATUS )
\end{verbatim}
\end{small}

When defining a logical characteristic, such as the INTERACTIVE option,
passing a true value in \texttt{\htmlref{GWM\_WSETL}{GWM_WSETL}}
will activate the option,
whereas passing a false value is equivalent to accepting the default.

Once the window has been created all the options are reset to their
default values. If a second window is to be created all the necessary
characteristics have to be re-defined before the next call to
{\tt GWM\_CRWIN}.

A GWM window will remain on the server after the program which created
it has terminated unless it is explicitly deleted using 
\texttt{\htmlref{GWM\_DSWIN}{GWM_DSWIN}}.

\subsection{Inquiries}
The routine \texttt{\htmlref{GWM\_EXIST}{GWM_EXIST}}
can be used to inquire if a GWM window of
a given name exists on the server. A logical argument is returned with
a value of true signifying that the window exists on the server, and
a value of false indicating that it does not.

The routine \texttt{\htmlref{GWM\_GETCI}{GWM_GETCI}}
returns the number of colours and the list
of colour indices (pen numbers) allocated to a GWM window. The colour
indices are returned in an array dimensioned by the application. If the
array is not large enough to contain all the allocated colour indices
then the array is filled up to its maximum size and the remaining indices
are not returned. The argument specifying the number of colours does
however return the actual number of colours allocated, and not the number
of colours returned in the array. Consider making the inquiry for
the window created in the previous example:

\begin{small}
\begin{verbatim}
      *   Define the array to receive the colour indices
            INTEGER IDIM
            PARAMETER ( IDIM = 64 )
            INTEGER INDEXS( IDIM ), NCOLS

      *   Inquire the colour indices of the GWM window
            CALL GWM_GETCI( 'XWINDOWS', IDIM, INDEXS, NCOLS, STATUS )
\end{verbatim}
\end{small}

In this case the number of colours returned (NCOLS) will be 128, but
only the first 64 colour indices will be returned in the INDEXS array.

\subsection{Linking programs with GWM}
Stand alone programs are linked with GWM by including
{\tt `gwm\_link`} on the {\tt f77} command line.

Adam programs are linked with GMW by including {\tt `gwm\_link\_adam`} on the
{\tt alink} command line.

\newpage
\subsection{Subroutine Specifications}

\begin{small}
\sstroutine{
   GWM\_CLOSE
}{
   Close the X client-server connection
}{
   \sstdescription{
      Close the X client-server connection established by GWM\_OPEN.
      The standard error handler is restored.
   }
   \sstinvocation{
      CALL GWM\_CLOSE( STATUS )
   }
   \sstarguments{
      \sstsubsection{
         STATUS = INTEGER (Given and Returned)
      }{
         The global status.
      }
   }
}
\sstroutine{
   GWM\_CRWIN
}{
   Create a GWM window
}{
   \sstdescription{
      Create a GWM window. A window with the given name is created on
      the current X display. The window is created with the default
      characteristics unless they have been set using the GWM\_WSETx
      routines. After the window has been created any set characteristics
      are reset to their default values.
   }
   \sstinvocation{
      CALL GWM\_CRWIN( WNAME, STATUS )
   }
   \sstarguments{
      \sstsubsection{
         WNAME = CHARACTER{\aster}({\aster}) (Given)
      }{
         The window name.
      }
      \sstsubsection{
         STATUS = INTEGER (Given and Returned)
      }{
         The global status.
      }
   }
}
\sstroutine{
   GWM\_DSWIN
}{
   Destroy a GWM window
}{
   \sstdescription{
      Destroy a GWM window. The GWM window having the given name on
      the current X display is destroyed.
   }
   \sstinvocation{
      CALL GWM\_DSWIN( WNAME, STATUS )
   }
   \sstarguments{
      \sstsubsection{
         WNAME = CHARACTER{\aster}({\aster}) (Given)
      }{
         The window name.
      }
      \sstsubsection{
         STATUS = INTEGER (Given and Returned)
      }{
         The global status.
      }
   }
}
\sstroutine{
   GWM\_EXIST
}{
   Inquire if a GWM window of the given name exists
}{
   \sstdescription{
      Inquire if a GWM window of the given name exists. The current X
      display is searched for a GWM window with the given name. The
      result of the search is returned in the EXISTS argument.
   }
   \sstinvocation{
      CALL GWM\_EXIST( WNAME, EXISTS, STATUS )
   }
   \sstarguments{
      \sstsubsection{
         WNAME = CHARACTER{\aster}({\aster}) (Given)
      }{
         The window name.
      }
      \sstsubsection{
         EXISTS = LOGICAL (Returned)
      }{
         True if a GWM window of the given name exists, otherwise false.
      }
      \sstsubsection{
         STATUS = INTEGER (Given and Returned)
      }{
         The global status.
      }
   }
}
\sstroutine{
   GWM\_GETCI
}{
   Inquire the number of colours and the colour indices
}{
   \sstdescription{
      Inquire the number of colours and the colour indices allocated
      to the given window.
   }
   \sstinvocation{
      CALL GWM\_GETCI( WNAME, IDIM, INDEXS, NCOLS, STATUS )
   }
   \sstarguments{
      \sstsubsection{
         WNAME = CHARACTER{\aster}({\aster}) (Given)
      }{
         The window name.
      }
      \sstsubsection{
         IDIM = INTEGER (Given)
      }{
         The declared size of the INDEXS array.
      }
      \sstsubsection{
         INDEXS = INTEGER(IDIM) (Returned)
      }{
         Array containing the colour indices allocated to the window.
         If the number of allocated colours (NCOLS) is greater than
         the array size (IDIM) only the first IDIM indices are returned.
      }
      \sstsubsection{
         NCOLS = INTEGER (Returned)
      }{
         The number of colours allocated to the window.
      }
      \sstsubsection{
         STATUS = INTEGER (Given and Returned)
      }{
         The global status.
      }
   }
}
\sstroutine{
   GWM\_OPEN
}{
   Establish the X client-server connection
}{
   \sstdescription{
      Establish the X client-server connection. The display name
      specifies the node on which the server is running. Most
      applications will use the default device in which case the
      logical argument USEDEF should be set to true and the display
      name is ignored. An error handler is established which reports
      errors via EMS. This routine has to be called before any of
      the other GWM FORTRAN interface routines. The connection is
      terminated by the routine GWM\_CLOSE.
   }
   \sstinvocation{
      CALL GWM\_OPEN( DISPLY, USEDEF, STATUS )
   }
   \sstarguments{
      \sstsubsection{
         DISPLY = CHARACTER{\aster}({\aster}) (Given)
      }{
         Display name specifying the network node name and display
         number of the workstation. The format of the name will be
         {\tt "}hostname:number$<$.screen$>${\tt "} if the transport mechanism is
         TCP/IP or {\tt "}hostname::number$<$.screen$>${\tt "} if the transport
         mechanism is DECNET. The {\tt "}hostname{\tt "} is the name of the
         host machine to which the display is physically connected.
         The {\tt "}number{\tt "} is the number of the server on the host machine.
         A single CPU can have one or more servers which are usually
         numbered starting with zero. On a multiple-screen workstation
         the optional {\tt "}screen{\tt "} number indicates the screen to use.
         Examples are {\tt "}cpu:0{\tt "}, {\tt "}cpu::0{\tt "}, {\tt "}cpu:0.1{\tt "} or {\tt "}cpu::0.1{\tt "}.
         If USEDEF is true the display name is ignored.
      }
      \sstsubsection{
         USEDEF = LOGICAL (Given)
      }{
         If true use the default display, otherwise use the display
         specified by the DISPLY argument.
      }
      \sstsubsection{
         STATUS = INTEGER (Given and Returned)
      }{
         The global status.
      }
   }
}
\sstroutine{
   GWM\_WSETC
}{
   Set a character string window option
}{
   \sstdescription{
      The window options are used to control the characteristics of
      the GWM window and to override the default values. These must
      be set before the window is created with GWM\_CRWIN. The
      {\tt '}TITLE{\tt{'}} option is an example of an character string option.
   }
   \sstinvocation{
      CALL GWM\_WSETC( OPTION, VALUE, STATUS )
   }
   \sstarguments{
      \sstsubsection{
         OPTION = CHARACTER{\aster}({\aster}) (Given)
      }{
         The option name.
      }
      \sstsubsection{
         VALUE = CHARACTER{\aster}({\aster}) (Given)
      }{
         The option value.
      }
      \sstsubsection{
         STATUS = INTEGER (Given and Returned)
      }{
         The global status.
      }
   }
}
\sstroutine{
   GWM\_WSETI
}{
   Set an integer window option
}{
   \sstdescription{
      The window options are used to control the characteristics of
      the GWM window and to override the default values. These must
      be set before the window is created with GWM\_CRWIN. The
      {\tt '}COLOURS{\tt{'}} option is an example of an integer option.
   }
   \sstinvocation{
      CALL GWM\_WSETI( OPTION, VALUE, STATUS )
   }
   \sstarguments{
      \sstsubsection{
         OPTION = CHARACTER{\aster}({\aster}) (Given)
      }{
         The option name.
      }
      \sstsubsection{
         VALUE = INTEGER (Given)
      }{
         The option value.
      }
      \sstsubsection{
         STATUS = INTEGER (Given and Returned)
      }{
         The global status.
      }
   }
}
\sstroutine{
   GWM\_WSETL
}{
   Set a logical window option
}{
   \sstdescription{
      The window options are used to control the characteristics of
      the GWM window and to override the default values. These must
      be set before the window is created with GWM\_CRWIN. A logical
      option has two values, true or false. A true value means select
      the option, a false value is equivalent to accepting the default.
      The {\tt '}INTERACTIVE{\tt{'}} option is an example of a logical option.
   }
   \sstinvocation{
      CALL GWM\_WSETL( OPTION, VALUE, STATUS )
   }
   \sstarguments{
      \sstsubsection{
         OPTION = CHARACTER{\aster}({\aster}) (Given)
      }{
         The option name.
      }
      \sstsubsection{
         VALUE = LOGICAL (Given)
      }{
         The option value, true or false.
      }
      \sstsubsection{
         STATUS = INTEGER (Given and Returned)
      }{
         The global status.
      }
   }
}
\end{small}
\newpage
\section{Graphics Driver Interface}

This section describes the routines that are used by low level graphics 
software to interact with GWM windows. A working knowledge of Xlib
programming is assumed.
\begin{quote}{\em
The routines described here are for the use of ``system'' software only
and all this information in this section is subject to change without
notice. Therefore before making any use of it you should consult the 
\htmladdnormallink{Starlink project staff}{http://star-www.rl.ac.uk/people.html}
at RAL.}
\end{quote}
These return a status value rather than reporting errors; symbolic names
for the status values can be found in {\tt gwm\_err.h}. Note that these are
{\em not} EMS error codes.

\subsection{Anatomy of a Window}
A GWM window consists of the following components:
\begin{itemize}
\item A window
\item A pixmap for saving the contents of the window
\item On an X display with a writable colour table, a set of allocated
colour cells.
\item Optionally, an allocated bit plane for use as an overlay plane
(only supported on pseudo colour displays).
\end{itemize}
Various items of information are stored as properties of the window so
that any applications which knows the id of the window can read and
write them. The {\tt GWM\_Set{\em Xxxx}} and {\tt GWM\_Get{\em Xxxx}}
routines can be used to access them. The properties are:
\begin{description}

\item[{\tt GWM\_background}] The background colour of the window (a character 
string) 

\item[{\tt GWM\_colour\_table}] An array of integers that lists the colour
cells allocated to the window. The first element of the array 
defines the background colour.

\item[{\tt GWM\_foreground}] The foreground colour of the window (a character 
string) 

\item[{\tt GWM\_name}] The window name (a character string). This enables a
window id to be translated into a GWM window name.

\item[{\tt GWM\_ov\_mask}] A bit mask that which plane is the overlay plane.
The mask has a 1 in every bit position that should be used when writing
to the window and a 0 in the bit position corresponding to the overlay
plane. A window with no overlay plane has a mask of all 1s.

\item[{\tt GWM\_pixmap}] The id of the pixmap.

\item[{\tt GWM\_x\_offset GMW\_y\_offset}] The position of the top left corner
of the pixmap relative to the top left corner of the window. Used by the
refresh process when copying the pixmap to the window and allows the
contents of the window to be scrolled.

\item[{\tt GWM\_x\_ov\_offset GWM\_y\_ov\_offset}] As above but for the overlay
plane.

\end{description}

\subsection{Drawing in the Window}

Everything drawn in the window must also be drawn in the pixmap so that
the window is refreshed properly. The easiest way to do this is to draw
into the pixmap first and then add the x and y offsets to all the
absolute coordinates and then draw into the window. The graphics
context must have its plane mask set to the value of the overlay mask
property (so as to avoid altering the overlay plane) and the foreground
set to one of the numbers in the colour table array. If the drawing
operation results in anything being drawn with the background colour
then the background should also be set to one of the numbers in the
colour table array, usually the first.

If the drawing operation involves a large amount of data it may be more
efficient to only draw into the pixmap and then copy the pixmap to the
window with {\tt XCopyArea}. The destination position should be the x
and y offset values and again, the overlay plane must be protected by
the plane mask in the gc.

\subsection{Drawing in the Overlay Plane}
Drawing in the overlay plane is just the same as drawing into the window
except that the plane mask must be set to the complement of the overlay
mask
property and the foreground and background colours must be all 1s and
all 0s respectively\footnote{Of course in actual fact only the bit that
corresponds to the overlay plane matters since the rest are masked by
the plane mask but all 1s and all 0s is the easiest value to use}. The
overlay offset values must be added to the coordinates when drawing into
the window.

\subsection{The Colour Table}

On a display with a writeable colour table, the colour table property is a list
of the colour cells allocated to the refresh process and must not be
altered.

On a display with a fixed colour table the colour table property will
list all possible pixel values---since the colour cells can't be changed
they can be used by all windows---and the colour table can be used to
map ``virtual'' pixel values to the real pixel values used to write to
the window and may be altered. For example, GKS uses the colour table to 
map colour indices onto pixel values.

If a window has an overlay plane, the colour of the overlay plane is set
by setting the colour cells that correspond to values listed in the
colour tables ORed with the bit position of the overlay plane (ie. the
complement of the plane mask).

\subsection{Scrolling}
Scrolling of the window and overlay planes is achieved by setting the
values of the offset and overlay offset properties. The refresh process
is notified of the changes and updates the destination positions it uses
when copying the pixmap. However, it does not update the window when the
property values change so the application that changes the offsets must
copy the window contents to the new position itself.

\subsection{Examples}

Some example programs that illustrate how to perform various operations on a
GWM window can be found in the gwm examples directory 
({\tt \$STARLINK\_DIR/share/gwm}
on Starlink systems).

\newpage
\subsection{Subroutine Specifications}

\begin{small}
\sstroutine{
   GWM\_CreateWindow
}{
   Create a window
}{
   \sstdescription{
      A GWM window is created according to the specification in the
      argument list and the display id of the X connection and the name
      of the window returned.
   }
   \sstinvocation{
      status = GWM\_CreateWindow( argc, argv, \&display, \&name);
   }
   \sstarguments{
      \sstsubsection{
         int argc (given)
      }{
         Count of number of arguments
      }
      \sstsubsection{
         char {\aster}argv[] (given)
      }{
         xmake arguments
      }
      \sstsubsection{
         Display {\aster}display (returned)
      }{
         display id
      }
      \sstsubsection{
         char name[] (returned)
      }{
         window name
      }
   }
}
\sstroutine{
   GWM\_DestroyWindow
}{
   Destroy a window
}{
   \sstdescription{
      The X display is searched for the named window and if found, the
      window name property is removed from the root window and the window
      and its associated pixmap destroyed.
   }
   \sstinvocation{
      status = GWM\_DestroyWindow( display, name);
   }
   \sstarguments{
      \sstsubsection{
         Display {\aster}display (given)
      }{
         Display id
      }
      \sstsubsection{
         char name[] (given)
      }{
         Window name
      }
   }
}
\sstroutine{
   GWM\_FindWindow
}{
   Find a window
}{
   \sstdescription{
      The X server is searched for a GWM window with the specified name
      and the id of the window returned.
   }
   \sstinvocation{
      status = GWM\_FindWindow( display, name, \&win);
   }
   \sstarguments{
      \sstsubsection{
         Display {\aster}display (given)
      }{
         Display id
      }
      \sstsubsection{
         char name[] (given)
      }{
         Window name
      }
      \sstsubsection{
         Window win (returned)
      }{
         Window id
      }
   }
}
\sstroutine{
   GWM\_GetBgCol
}{
   Get background colour
}{
   \sstdescription{
      The value of the GWM\_background property is fetched from the window.
   }
   \sstinvocation{
      status = GWM\_GetBgCol(display, win\_id, \&bg);
   }
   \sstarguments{
      \sstsubsection{
         Display {\aster}display (given)
      }{
         Display id
      }
      \sstsubsection{
         Window win\_id (given)
      }{
         Window id
      }
      \sstsubsection{
         char {\aster}bg (returned)
      }{
         Pointer to background colour specification
      }
   }
}
\sstroutine{
   GWM\_GetColTable
}{
   Get window{\tt '}s colour table
}{
   \sstdescription{
      The GWM\_colour\_table property is fetched from the window and
      pointer to it and the number of values it contains returned.
   }
   \sstinvocation{
      status = GWM\_GetColTable( display, win\_id, \&table, \&size);
   }
   \sstarguments{
      \sstsubsection{
         Display {\aster}display (given)
      }{
         Display id
      }
      \sstsubsection{
         Window win\_id (given)
      }{
         Window id
      }
      \sstsubsection{
         long {\aster}table (returned)
      }{
         Pointer to Colour table
      }
      \sstsubsection{
         unsigned long size (returned)
      }{
         Number of colour table entries
      }
   }
}
\sstroutine{
   GWM\_GetFgCol
}{
   Get foreground colour
}{
   \sstdescription{
      The value of the GWM\_foreground property is fetched from the window.
   }
   \sstinvocation{
      status = GWM\_GetFgCol(display, win\_id, \&fg);
   }
   \sstarguments{
      \sstsubsection{
         Display {\aster}display (given)
      }{
         Display id
      }
      \sstsubsection{
         Window win\_id (given)
      }{
         Window id
      }
      \sstsubsection{
         char {\aster}fg (returned)
      }{
         Pointer to foreground colour specification
      }
   }
}
\sstroutine{
   GWM\_GetOvMask
}{
   Get overlay mask
}{
   \sstdescription{
      The GMW\_ov\_mask property is fetched from the window.
   }
   \sstinvocation{
      status = GWM\_GetOvMask( display, win\_id, \&mask);
   }
   \sstarguments{
      \sstsubsection{
         Display {\aster}display (given)
      }{
         Display id
      }
      \sstsubsection{
         Window win\_id (given)
      }{
         Window id
      }
      \sstsubsection{
         unsigned long mask
      }{
         Overlay plane mask
      }
   }
}
\sstroutine{
   GWM\_GetOvScroll
}{
   Get overlay scroll offset
}{
   \sstdescription{
      The values of the GWM\_x\_ov\_offset and GWM\_y\_ov\_offset properties
      are fetched from the window.
   }
   \sstinvocation{
      status = GWM\_GetOvScroll( display, win\_id, \&xoffset, \&yoffset);
   }
   \sstarguments{
      \sstsubsection{
         Display {\aster}display (given)
      }{
         Display id
      }
      \sstsubsection{
         Window win\_id (given)
      }{
         Window id
      }
      \sstsubsection{
         int xoffset (returned)
      }{
         Overlay scroll offset in x
      }
      \sstsubsection{
         int yoffset (returned)
      }{
         Overlay scroll offset in y
      }
   }
}
\sstroutine{
   GWM\_GetPixmap
}{
   Get pixmap id
}{
   \sstdescription{
      The value of the GWM\_pixmap property is fetched from the window.
   }
   \sstinvocation{
      status = GWM\_GetPixmap( display, win\_id, \&pix\_id);
   }
   \sstarguments{
      \sstsubsection{
         Display {\aster}display (given)
      }{
         Display id
      }
      \sstsubsection{
         Window win\_di (given)
      }{
         Window id
      }
      \sstsubsection{
         Pixmap pix\_id (returned)
      }{
         Pixmap id
      }
   }
}
\sstroutine{
   GWM\_GetScroll
}{
   Get scroll offset
}{
   \sstdescription{
      The values of the GWM\_x\_offset and GWM\_y\_offset properties are
      fetched from the window.
   }
   \sstinvocation{
      status = GWM\_GetScroll(display, win\_id, \&xoffset, \&yoffset);
   }
   \sstarguments{
      \sstsubsection{
         Display {\aster}display (given)
      }{
         Display id
      }
      \sstsubsection{
         Window win\_id (given)
      }{
         Window id
      }
      \sstsubsection{
         int xoffset (returned)
      }{
         Scroll offset in x
      }
      \sstsubsection{
         int yoffset (returned)
      }{
         Scroll offset in y
      }
   }
}
\sstroutine{
   GWM\_SetBgCol
}{
   Set background colour property
}{
   \sstdescription{
      The value of the GWM\_background property is set on the window.
   }
   \sstinvocation{
      status = GWM\_SetBgCol(display, win\_id, bg);
   }
   \sstarguments{
      \sstsubsection{
         Display {\aster}display (given)
      }{
         Display id
      }
      \sstsubsection{
         Window win\_id (given)
      }{
         Window id
      }
      \sstsubsection{
         char {\aster}bg (given)
      }{
         Background colour specification
      }
   }
}
\sstroutine{
   GWM\_SetColTable
}{
   Set colour table
}{
   \sstdescription{
      The GWM\_colour\_table window property is replaced with the new
      array of values.
   }
   \sstinvocation{
      status = GWM\_SetColTable( display, win\_id, table, size);
   }
   \sstarguments{
      \sstsubsection{
         Display {\aster}display (given)
      }{
         Display id
      }
      \sstsubsection{
         Window win\_id (given)
      }{
         Window id
      }
      \sstsubsection{
         int {\aster}table (given)
      }{
         Pointer to new colour table array
      }
      \sstsubsection{
         unsigned long size (given)
      }{
         Number of entries in colour table
      }
   }
}
\sstroutine{
   GWM\_SetFgCol
}{
   Set foreground colour property
}{
   \sstdescription{
      The value of the GWM\_foreground property is set on the window.
   }
   \sstinvocation{
      status = GWM\_SetFgCol(display, win\_id, fg);
   }
   \sstarguments{
      \sstsubsection{
         Display {\aster}display (given)
      }{
         Display id
      }
      \sstsubsection{
         Window win\_id (given)
      }{
         Window id
      }
      \sstsubsection{
         char {\aster}fg (given)
      }{
         Foreground colour specification
      }
   }
}
\sstroutine{
   GWM\_SetOvScroll
}{
   Set overlay scroll offset
}{
   \sstdescription{
      The values of the GWM\_x\_ov\_offset and GWM\_y\_ov\_offset window
      properties are replaced by the new values.
   }
   \sstinvocation{
      status = GWM\_SetOvScroll( display, win\_id, xoffset, yoffset);
   }
   \sstarguments{
      \sstsubsection{
         Display {\aster}display (given)
      }{
         Display id
      }
      \sstsubsection{
         Window win\_id (given)
      }{
         Window id
      }
      \sstsubsection{
         int xoffset (given)
      }{
         New overlay offset value in x
      }
      \sstsubsection{
         int yoffset (given)
      }{
         New overlay offset value in y
      }
   }
}
\sstroutine{
   GWM\_SetPixmap
}{
   Set pixmap id
}{
   \sstdescription{
      The value of the GWM\_pixmap window property is replaced with the
      new value.
   }
   \sstinvocation{
      status = GWM\_SetPixmap( display, win\_id, pix\_id);
   }
   \sstarguments{
      \sstsubsection{
         Display {\aster}display (given)
      }{
         Display id
      }
      \sstsubsection{
         Window win\_id (given)
      }{
         Window id
      }
      \sstsubsection{
         Pixmap pix\_id (given)
      }{
         New pixmap id
      }
   }
}
\sstroutine{
   GWM\_SetScroll
}{
   Set offset offset
}{
   \sstdescription{
      The GWM\_x\_offset and GWM\_y\_offset window properties
      are replaced by the new values.
   }
   \sstinvocation{
      status = GWM\_SetScroll( display, win\_id, xoffset, yoffset);
   }
   \sstarguments{
      \sstsubsection{
         Display {\aster}display (given)
      }{
         Display id
      }
      \sstsubsection{
         Window win\_id (given)
      }{
         Window id
      }
      \sstsubsection{
         int xoffset (given)
      }{
         New window scroll offset in x
      }
      \sstsubsection{
         int yoffset (given)
      }{
         New window scroll offset in y
      }
   }
}
\end{small}
\end{document}
