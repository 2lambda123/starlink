\documentclass[11pt,nolof]{starlink}

% -----------------------------------------------------------------------------
% ? Document identification
%------------------------------------------------------------------------------
\stardoccategory    {Starlink User Note}
\stardocinitials    {SUN}
\stardocsource      {sun\stardocnumber}
\stardocnumber      {150.10}
\stardocauthors     {D.S. Berry}
\stardocdate        {20th October 2012}
\stardoctitle     {GRP \\ [1ex]
                                Routines for Managing Groups of Objects}
\stardocversion     {Version 3.7}
\stardocmanual      {Programmer's Manual}
\stardocabstract{
GRP provides facilities for handling groups of objects which can be described by
character strings (e.g.\ file names, astronomical objects, numerical values,
etc). Facilities are provided for storing and retrieving such character strings
within groups, and for passing groups of character strings between applications
by means of a text file. It also allows the creation of groups in which each
element is a modified version of an element in another group.
}
% ? End of document identification
% -----------------------------------------------------------------------------

\providecommand{\noteroutine}[3]{{\small \textbf {#1}{#2}} \nopagebreak \\
                             \hspace*{3em} \emph{{#3}} \\[1.5ex]}

% ? End of document specific commands
% -----------------------------------------------------------------------------
%  Title Page.
%  ===========
\begin{document}
\scfrontmatter

\section {Introduction}

Applications often have to manage groups of strings in which each string
represents some object. Typical examples of the contents of such groups may be:

\begin{itemize}
\item Names of astronomical objects.
\item File names.
\item Lists of wavelengths of spectral lines.
\end{itemize}

The GRP package provides facilities for managing the storage and retrieval of
strings within such groups. Applications may explicitly specify strings to be
stored within a group, or instead may specify the name of a text file from which
to read such strings. The contents of a group may be written out to a text file,
providing an easy means of passing groups of objects between applications.

\subsection{\label{SEC:FAC}Facilities Provided by GRP}
The concept of a group within GRP may be compared with an ``array'' structure
within a high level programming language such as Fortran. Arrays are used for
storing many values in a single object, with each value associated with an
integer ``index''. A value can be stored in, or retrieved from, a particular
element of an array by specifying the index for that element. The facilities
provided by the GRP package are similar. GRP allows character strings to be
stored in, or retrieved from, one of several ``groups'' (in this sense, a
``group'' is the GRP equivalent of a character array). Each element within the
group has an associated index, and different groups are distinguished by
different ``identifiers'' (similar to the way that different arrays are
distinguished by having different names).

GRP also provides the following features not available through the use of
Fortran arrays:

\begin{enumerate}

\item In order to store values in a normal character array, you must assign an
explicit character string to each element of the array. The GRP package provides
similar facilities for storing explicit strings, but also provides facilities
for reading the values to be stored from a text file. This is known as \emph{indirection}; instead of providing a set of strings to be stored, you provide
the name of a file which contains the strings to to be stored. For instance, if
you were prompted for a list of data files to be processed, you could then
respond either with the explicit name of each data file, or with the name of a
text file containing a list of the names of the data files.

\item \label{ITEM:MOD} An alternative method for specifying values to be stored
in a group is by modification of the values already stored in \emph{another}
group. For instance, if values are to be assigned to elements 1 to 3, the actual
strings stored would be obtained by taking the values stored in elements 1 to 3
of another group and applying some editing to them. The same editing is used for
each element; it can include the addition of a suffix and/or a prefix, and the
substitution of one sub-string with another. A typical use of this facility
could be to specify a set of output files by describing the editing and the names
of the corresponding input files. Thus, if a group describing the input files
contained the three strings:

\begin{small}
\begin{terminalv}
OBJ1.DAT
OBJ2.DAT
OBJ3.DAT
\end{terminalv}
\end{small}

you might specify the output files using a string such as:

\begin{small}
\begin{terminalv}
A_*|DAT|TXT|
\end{terminalv}
\end{small}

This would cause the addition of the
prefix ``\textbf{A\_}'' and substitution of ``\textbf{TXT}'' for ``\textbf{DAT}'',
resulting in the output file names:

\begin{small}
\begin{terminalv}
A_OBJ1.TXT
A_OBJ2.TXT
A_OBJ3.TXT
\end{terminalv}
\end{small}

\item In the above example, a group of values was obtained by copying them from
another group, and then applying some specified editing. This technique of
editing values can be combined with other ways of specifying the original
values. For instance, to apply the same editing as above to each of the values
stored in the text file \texttt{OBJECTS.LIS} the following response could be given:

\begin{small}
\begin{terminalv}
 A_{^OBJECTS.LIS}|DAT|TXT|
\end{terminalv}
\end{small}

If the same editing is to be applied to a list of literal values typed in
directly at the keyboard, then a response such as the following could be given:

\begin{small}
\begin{terminalv}
A_{OBJ1.DAT,OBJ2.DAT,OBJ3.DAT}|DAT|TXT|
\end{terminalv}
\end{small}

\item Groups have dynamic size (i.e.\ they expand in size as required to make
room for new values), which is often more convenient than the fixed size of an
array specified in its declaration. Thus if an application wants to process
many data files, and chooses to store the file names in a GRP group, then no
limit is imposed on the number of files which the user may specify.

\item The contents of a group may be written out to a text file using a single
subroutine call. This, together with the ability to read a group's contents
from a text file, provides a convenient means of passing groups of objects
between applications. For instance, an application may produce a text file
holding the names of all the output data files it has created. A later
application can then read this text file to obtain the names of the data files
which it is to process.

\item Elements within a group can be copied in a single call to another group.
Duplicate names may also be purged from a group in a single call.

\item GRP stores information about how each value within a group group was
obtained (i.e.\ whether it was given explicitly, or by indirection, or by
modification). The names of indirection files are stored, as are the identifiers
for groups used as the basis for ``modified'' elements, and all this information
is available to the calling application.

\end{enumerate}

GRP is a general purpose library which makes no attempt to
attach any particular meaning or properties to the strings stored in a group.
It is expected that other, more specialized libraries will be written
which use GRP to handle specific types of strings (eg coordinate values, names
of data files, etc). Such packages will provide additional features to handle
the objects stored by GRP (eg the creation, opening and closing of data files).

\subsection{\label{SEC:EXAMC}An Example of a Simple GRP Application}
The facilities of GRP are particularly useful for processing lists of text
strings provided in response to a prompt. The user of an application can specify
the strings literally, or can specify the name of a text file containing the
strings, or can specify the editing to be used to derive them by modification of
the strings stored somewhere else.

\subsubsection{Example Code}
Here is a simple example of the use of GRP which illustrates the ideas of
indirection and modification. In this example, each stored string corresponds to
the name of a file but obviously an application could apply other
interpretations. The user is prompted for a set of input file names and then
prompted again for a set of output file names. Each input file is processed in
some way (by routine PROC) to produce the corresponding output file. The source
code that follows is not intended to provide all the information necessary to
write GRP applications, but simply to give a feeling for the way GRP works:
\begin{small}
\begin{terminalv}
      SUBROUTINE GRP_TEST( STATUS )                             [1]
      IMPLICIT NONE

*  Include definitions of global constants.
      INCLUDE 'SAE_PAR'                                         [2]
      INCLUDE 'GRP_PAR'                                         [3]

*  Declare local variables.
      INTEGER STATUS, IGRP1, IGRP2, SIZE1, SIZE2, ADDED, I
      CHARACTER*(GRP__SZNAM) INFIL, OUTFIL
      LOGICAL FLAG

*  Check inherited global status.
      IF ( STATUS .NE. SAI__OK ) RETURN                         [4]

*  Create a new (empty) group to contain the names of the
*  input files.
      CALL GRP_NEW( 'Input files', IGRP1, STATUS )              [5]

*  Prompt the user for a group of input file names and place
*  them in the group just created.
      CALL GRP_GROUP( 'IN_FILES', GRP__NOID, IGRP1, SIZE1,      [6]
     :                ADDED, FLAG, STATUS )

*  Create a second group to hold output file names.
      CALL GRP_NEW( 'Output files', IGRP2, STATUS )             [7]

*  Prompt the user for a group of output file names, giving
*  the chance to specify them by modification of the input
*  file names. Place the output file names in the new group
*  just created.
      CALL GRP_GROUP( 'OUT_FILES', IGRP1, IGRP2, SIZE2,         [8]
     :                ADDED, FLAG, STATUS )

*  Report an error and abort if the number of output files
*  does not equal the number of input files.
      IF( SIZE2 .NE. SIZE1 .AND. STATUS .EQ. SAI__OK ) THEN     [9]
         STATUS = SAI__ERROR
         CALL ERR_REP( 'GRP_TEST_ERR1',
     :           'Incorrect number of output files specified',
     :           STATUS )
         GO TO 999
      END IF

*  Loop round each input file.
      DO I = 1, SIZE1                                           [10]

*  Retrieve the input file name with index given by I.
         CALL GRP_GET( IGRP1, I, 1, INFIL, STATUS )             [11]

*  Retrieve the output file name with index given by I.
         CALL GRP_GET( IGRP2, I, 1, OUTFIL, STATUS )

*  Process the data.
         CALL PROC( INFIL, OUTFIL, STATUS )                     [12]

*  Do the next input file.
      END DO

*  Delete the groups created by this application.
 999  CONTINUE
      CALL GRP_DELET( IGRP1, STATUS )                           [13]
      CALL GRP_DELET( IGRP2, STATUS )

      END
\end{terminalv}
\end{small}

\textbf{Programming notes:}

\begin{enumerate}

\item The example is actually an ADAM A-task, and so consists of a subroutine with
a single argument giving the inherited status value. See
\xref{SUN/101}{sun101}{} for further
details about writing ADAM A-tasks. A ``stand-alone'' version of the GRP package
is available which can be used with non-ADAM applications.

\item The first INCLUDE statement is used to define standard ``symbolic
constants'', such as the values SAI\_\_OK and SAI\_\_ERROR which are used in
this routine. Starlink software makes widespread use of such constants, which should always
be defined in this way rather than by using actual numerical values. The file
SAE\_PAR is almost always needed, and should be included as standard in every
application.

\item The second INCLUDE statement performs a similar function to the first, but
defines symbolic constants which are specific to the GRP package. Such constants
are recognizable by the fact that they start with the five characters
``GRP\_\_'' (such as GRP\_\_NOID and GRP\_\_SZNAM used in the above example).
Note the double underscore ``\_\_'' which distinguishes them from subroutine
names.

\item The value of the STATUS argument is checked. This is because the
application uses the Starlink error handling strategy (see
\xref{SUN/104}{sun104}{}), which
requires that a subroutine should do nothing unless its STATUS argument is set
to the value SAI\_\_OK on entry. Here, we simply return without action if STATUS
has the wrong value.

\item An identifier for a new group is now obtained. The variable IGRP1 is
returned holding an integer value which the GRP package uses to recognise the
group just created. Initially, there are no names stored in the group. A string
is stored which should be used to give a description of the type of objects
stored within the group (in this case the string \emph{``Input files''} is
used).

\item The user is now prompted for a value for the parameter IN\_FILES, and
replies with a string, which GRP splits up into separate elements, each element
being either a literal file name or the name of a text file containing other
file names. As there are no other groups defined at this point, it is not
possible to specify the files names using ``modification'' (as described in item
\ref{ITEM:MOD} in section \ref{SEC:FAC}). For this reason, the second argument
(which would normally specify the group to use as the basis for modification) is
given the value GRP\_\_NOID. This is a special identifier value used to indicate
a ``null'' group. The names supplied by the user are stored in the group created
by the previous call to \htmlref{GRP\_NEW}{GRP_NEW}, and the number of names is returned in
argument SIZE1.
Note, no permanent association exists between the \emph{group}
identified by IGRP1 and the \emph{parameter} IN\_FILES (which is one reason why
\htmlref{GRP\_GROUP}{GRP_GROUP} is not called GRP\_ASSOC). The parameter value may be cancelled (for
instance using PAR\_CANCL) without effecting the contents of the
group.

\item A second group is now created to hold the output file names. The two
groups are distinguished by the fact that they have different identifiers
(stored in IGRP1 and IGRP2).

\item \label {ITEM:OUT} The user is prompted again, this time for a value for
the parameter OUT\_FILES and the names obtained are stored in the second group
just created. Again the user can give literal files names and/or the names of
text files holding other file names. Now that there are two groups, it is
possible to use modification to specify the output files. The identifier for the
group containing the input files names is given as the second argument of
\htmlref{GRP\_GROUP}{GRP_GROUP}, telling the GRP system that \emph{if} the user specifies output file
names using modification (which may not be the case of course), then the output
file names are to be derived by modifying the input file names stored in the
first group.

\item \label {ITEM:CHK} In this particular application it is deemed necessary to
have equal numbers of input and output files, but GRP imposes no restrictions on the
number of strings which can be supplied when responding to a prompt from
\htmlref{GRP\_GROUP}{GRP_GROUP}. It is therefore necessary to check the that the two groups contain
the same number of elements. A more sophisticated application might seek user
intervention to determine how to proceed at this point (either by requesting
extra output file names or by ignoring some). Note, if some other error has already been detected (as shown by
STATUS having a value other than SAI\_\_OK), then the check on the number of
input and output files is irrelevant.

\item Having stored the input file names in one group and the output file names
in another, the application now loops through each pair of input and output file
names in turn. An ``index'' I is used to distinguish between different
elements within a group. The input file name with index I is retrieved from the
first group and the output file name with the same index is retrieved from the
second group.

\item Note, there is a limit to the size of the character string which can be
stored in a GRP group. This size is given by the symbolic constant GRP\_\_SZNAM.

\item A routine is now called which uses the two file names; a typical routine
may take data out of the input file, process it and store the results in the
output file. The file handling itself would be done within the routine PROC.
This example actually makes no assumptions about what the strings stored in the
two groups represent (the descriptive strings stored when the two groups were
created are of no significance in this application). Although, for clarity, it
has been assumed that the strings correspond to file names, they could just as
easily have been wavelengths, the names of astronomical objects, calender dates,
or just about anything else.

\item Finally, the groups created by this application are deleted. This is
particularly important in applications which run as subroutines within a wider
context (such as ADAM applications). There is a limited number of groups
available, and if applications forget to delete the groups they have created,
then the possibility of exceeding the limit then exists. Note, the groups should
be deleted even if the application aborts because of an error, so the statement
labelled 999 (to which a jump is made if an error is detected) comes \emph{before} the calls to \htmlref{GRP\_DELET}{GRP_DELET}.

\end{enumerate}

\subsubsection{Examples of Possible User Responses}
If the example above was run, the user could respond in several ways to the
prompts for parameters IN\_FILES and OUT\_FILES. The following paragraphs
illustrate the use of indirection and modification in this context.

\paragraph {Indirection.}
When prompted for IN\_FILES the user could reply with the following text:

\begin{small}
\begin{terminalv}
IC_1575_RAW, IC_4320_RAW, ^NGC_OBJECTS.LIS
\end{terminalv}
\end{small}

This would cause the GRP package to place the two strings \texttt{IC\_1575\_RAW}
and \texttt{IC\_4320\-\_RAW} in two elements of the first group and then search
for a file called \texttt{NGC\_OBJECTS.LIS}. If this file contains the following
two lines of text:

\begin{small}
\begin{terminalv}
NGC_5128_RAW, NGC_2534_RAW
^OTHERS.LIS
\end{terminalv}
\end{small}

then the strings \texttt{NGC\_5128\_RAW} and \texttt{NGC\_2534\_RAW} would be added
to the same group, and a search made for the file \texttt{OTHERS.LIS}. If this
file, in turn, contained the two lines:

\begin{small}
\begin{terminalv}
NGC_1947_FLAT
NGC_3302_FLAT
\end{terminalv}
\end{small}

then the final group would consist of the six strings:

\begin{small}
\begin{terminalv}
IC_1575_RAW
IC_4320_RAW
NGC_5128_RAW
NGC_2534_RAW
NGC_1947_FLAT
NGC_3302_FLAT
\end{terminalv}
\end{small}

This illustrates the use of indirection as a means of specifying the strings to
be stored in a group, and shows it being combined with the specification of
literal strings, and indirections being nested.

Strings stored in a text file can be edited ``on the fly'' before being stored
in a group. For instance, the user could give the following response to a prompt
for IN\_FILES:

\begin{small}
\begin{terminalv}
^NGC_OBJECTS.LIS|_RAW|_CAL|
\end{terminalv}
\end{small}

This would cause the GRP package to read the values stored in text file \texttt{NGC\_OBJECTS.LIS}, replacing all occurrences of the string ``\_RAW'' with the
string ``\_CAL''. If the file \texttt{NGC\_OBJECTS.LIS} contained the same values
as before, then the editing would also be applied to the values stored in the
file \texttt{OTHERS.LIS}.

\paragraph {Modification.}
As an example of the use of ``modification'', let's suppose that the user
responds to the prompt for OUT\_FILES with the string:

\begin{small}
\begin{terminalv}
A_*2|RAW|FLAT|
\end{terminalv}
\end{small}

This would cause the application to generate six strings, based on the
six strings held in the first group (see programming note \ref{ITEM:OUT}).
The names are generated as follows
\begin{enumerate}
\item First, a copy of the six names in the first group is made and stored in
the second group.
\item Next, any occurrence of the string ``RAW'' within any of these
names is replaced with the string ``FLAT''. This
leaves the second group holding the names:

\begin{small}
\begin{terminalv}
IC_1575_FLAT
IC_4320_FLAT
NGC_5128_FLAT
NGC_2534_FLAT
NGC_1947_FLAT
NGC_3302_FLAT
\end{terminalv}
\end{small}

\item Next, each name is substituted in turn for the ``$*$'' character in the
string to the left of the first ``$\mid$'' character. Thus each name is prefixed by
``A\_'' and suffixed by ``2''. The second group finally holds the names:

\begin{small}
\begin{terminalv}
A_IC_1575_FLAT2
A_IC_4320_FLAT2
A_NGC_5128_FLAT2
A_NGC_2534_FLAT2
A_NGC_1947_FLAT2
A_NGC_3302_FLAT2
\end{terminalv}
\end{small}

\end{enumerate}

Modification can be combined with indirection and/or the specification of
literal strings. For instance, the user could have replied to the prompt for
OUT\_FILES with the string:

\begin{small}
\begin{terminalv}
NEW_FILE,A_*2|RAW|FLAT|,^LIST.DAT
\end{terminalv}
\end{small}

This would have caused the second group to contain not only the six names
described above, but also the additional names \textbf{NEW\_FILE} and any names
read from the text file LIST.DAT. In this case, the number of output files would
have exceeded the number of input files and the check described in programming
note \ref{ITEM:CHK} would fail.

\section{Names}

Each string stored within a group is referred to in the rest of this document as
a \emph{name}. The term is used purely for convenience and is not meant to imply
anything about what the strings actually represent. In some cases the strings
may easily be thought of as \emph{names} (for instance, in the example
in the previous section the strings were the \emph{names} of
files), but this may not always be so.

The maximum length of a name is given by the symbolic constant GRP\_\_SZNAM
(which is currently 255). Leading spaces, and spaces within names are
significant, but trailing spaces are ignored. Completely blank names are
allowed, and null names (i.e.\ names of zero length) are treated as blank names.

\subsection{\label{SEC:CASE}Case Sensitivity}
Names are always \emph{stored} in the form in which they are supplied. However,
when names are \emph{retrieved} from a group they will be converted to upper case
if the group has been designated as a case insensitive group. When groups are
first created they are considered to be case sensitive (i.e.\ names are retrieved
from the group in the same form in which they were stored). They can be made
case \emph{in}-sensitive by calling routine \htmlref{GRP\_SETCS}{GRP_SETCS}. Any comparisons involving
names held in case insensitive groups are performed without reference to case.

\section{Groups and Group Identifiers}

Applications can maintain information about many independent groups
simultaneously. The maximum number of simultaneously active groups is currently
set to 500 and is given by the symbolic constant GRP\_\_MAXG. There is no
limit on the number of names that a single group can contain (other than those
imposed by system quotas).

The GRP system distinguishes between different groups by use of an {identifier}
system. When a new group is created, it is assigned a ``GRP identifier'' or
``group identifier''. This value is used to specify which group is to be acted
upon by subsequent GRP calls. The symbolic constant GRP\_\_NOID can be given in
certain places where a GRP identifier would normally be given, in order to
specify a ``null'' group. Note, identifiers in the GRP package cannot be
``cloned'' to produce alternative, independent channels for accessing the same
group. If two integer variables contain identifiers for a single group, then
the integer values stored in the two variables will always be equal.

\section {\label{SEC:GREXP}Group Expressions}
One of the most useful routines within GRP is \htmlref{GRP\_GROUP}{GRP_GROUP}. This routine appends
names to the end of a previously created group using a ``group expression''
obtained from the environment via a named parameter which can be of any type. The
routine \htmlref{GRP\_GRPEX}{GRP_GRPEX} also performs this function, except that the group expression
is provided by the calling application, rather than being obtained through the
parameter system.

This section describes the syntax of group expressions.

\subsection{Elements and Delimiters}
Group expressions may contain several ``delimiter'' characters (usually a comma
although this can be changed, see section \ref{SEC:CC}) and the substrings
delimited by these characters are referred to as ``elements''. If there are no
delimiters in a group expression, then the group expression consists of a single
element. For instance, the group expression:

\begin{small}
\begin{terminalv}
NEW_FILE,A_*2|RAW|FLAT|,^LIST.DAT
\end{terminalv}
\end{small}

consists of the three elements \textbf{NEW\_FILE},
\textbf{A\_$*$2$\mid$RAW$\mid$FLAT$\mid$} and \textbf{\wedge LIST.DAT}. Note, delimiter characters are ignored if they occur within
matching ``nesting characters'' (see section \ref{SEC:NEST}). For instance,
nesting prevents the group expression:

\begin{small}
\begin{terminalv}
FLATFIELD(100:200,20:220),OBJECT
\end{terminalv}
\end{small}

being split into three elements instead of two (i.e.\ the first comma does not
act as a delimiter because it occurs within a nest formed by matching parentheses).

Each element of a group expression may be a literal name (eg \textbf{NEW\_FILE} in
the previous example), or an ``indirection element'' or a ``modification
element''. An indirection element specifies a text file from which further names
are to be read (eg \textbf{\wedge LIST.DAT} in the previous example). A
modification element specifies an existing group of names which are to be used
as the basis for the new names (eg \textbf{A\_$*$2$\mid$RAW$\mid$FLAT$\mid$} in the
previous example). These are described in more detail below.

\subsection{Editing of Names}
Each element in a group expression will give rise to one or more names
(depending on whether the element consists of a literal name, an indirection
element or a modification element). These names may be edited before being
stored in a group by including certain ``editing strings'' within the text of
the element. The general format of an element with editing strings included is:

\begin{small}
\begin{terminalv}
prefix{kernel}suffix|old|new|
\end{terminalv}
\end{small}

The \emph{kernel} string can be a single element, or can be a full group
expression. Processing of the element proceeds as follows:

\begin{enumerate}

\item The kernel is first expanded to give a list of literal names. This may
involve reading names from files, copying names from another group, etc,
depending on the exact nature of the kernel. The characters which mark the start
and end of the kernel are known as the opening and closing kernel delimiters.
They are usually set to be ``\{'' and ``\}'', but can be changed if needed.

\item Each name is checked to see if it contains the \emph{old} string. If it
does, all occurrences of the \emph{old} string are replaced by the \emph{new}
string. The character which delimits the \emph{old} and \emph{new} strings is
known as the ``separator'' character and is usually a ``$\mid$'' character,
but can be
changed if needed. This substitution will be case sensitive if the group to
which the names are to be added has been designated as case sensitive (see
section \ref{SEC:CASE}). If no substitutions are to be performed then the old
and new strings, together with the three separator characters, should be
omitted.

\item The \emph{prefix} string is added to the start of each name, and the \emph{suffix} string is appended to the end of each name. Either or both of these
strings may be null (i.e.\ of zero length).

\end{enumerate}

The names which result from this processing are then added to a group. If there
is no ambiguity about where the kernel starts and finishes (for instance if the
prefix and suffix are both omitted, and the kernel consists of a single element)
then the kernel does not need to be enclosed within kernel delimiters. The
contents of the kernel can be any group expression. In particular, the kernel
can contain other nested kernels with their own associated editing strings.

Let's look at some examples:

\begin{small}
\begin{terminalv}
A_{TOM,DICK,HARRY}_B
\end{terminalv}
\end{small}

This will give rise to the three names \textbf{A\_TOM\_B}, \textbf{A\_DICK\_B} and
\textbf{A\_HARRY\_B}.

\begin{small}
\begin{terminalv}
^FILE.LIS|_OLD|_NEW|
\end{terminalv}
\end{small}

This will read names from the text file \texttt{FILE.LIS} (see the description of
indirection elements below), and replace all occurrences of the string ``\textbf{\_OLD}'' within the names with the string ``\textbf{\_NEW}''.

\begin{small}
\begin{terminalv}
WW,{A,B_{ONE,TWO,THREE}|T|Z|,C}KK|_Z|_Y|
\end{terminalv}
\end{small}

This is a complex example and needs looking at carefully. Looking at it at the
highest level, it can be thought of as:

\begin{small}
\begin{terminalv}
WW,{kernel}KK|_Z|_Y|
\end{terminalv}
\end{small}

where \texttt{kernel} is the group expression:

\begin{small}
\begin{terminalv}
A,B_{ONE,TWO,THREE}|T|Z|,C
\end{terminalv}
\end{small}

The first and third elements in this inner group expression are simple literal names
and give rise to the two names \textbf{A} and \textbf{C}. The second element specifies
that the three names \textbf{ONE}, \textbf{TWO} and \textbf{THREE} are to be edited by
replacement of the letter T by the letter Z, and the addition of the prefix \textbf{B\_}. After editing, these three names become \textbf{B\_ONE}, \textbf{B\_ZWO} and
\textbf{B\_ZHREE}. So the total group specified by this inner kernel is:

\begin{small}
\begin{terminalv}
A
B_ONE
B_ZWO
B_ZHREE
C
\end{terminalv}
\end{small}

We can now go back and look at the full group expression in the form:

\begin{small}
\begin{terminalv}
WW,{kernel}KK|_Z|_Y|
\end{terminalv}
\end{small}

The first element specifies the single name \textbf{WW}. The second element
specifies that each of the names arising from the expansion of the inner kernel
(i.e.\ the names listed above) should be edited by replacing \textbf{\_Z} with \textbf{\_Y}, and then appending the suffix \textbf{KK}. Thus the final group contains:

\begin{small}
\begin{terminalv}
WW
AKK
B_ONEKK
B_YWOKK
B_YHREEKK
CKK
\end{terminalv}
\end{small}

\subsection{Indirection Elements}
An indirection element consists of an ``indirection character'' (usually
``\texttt{\wedge}'' (up arrow) although this can be changed, see section \ref{SEC:CC})
followed by the name of a text file. For instance, the group expression:

\begin{small}
\begin{terminalv}
^raw_data
\end{terminalv}
\end{small}

would cause GRP to search for a file called \texttt{raw\_data}.

The specified file is read to obtain further names to be added to the group.
Each line in the file is processed as if it were a separate group expression,
and so may contain any combination of literal names, modification elements or
further indirection elements. It is thus possible to get several levels of
indirection, in which a literal name is specified within a text file, which is
itself specified within an indirection element contained within another text
file, etc. GRP imposes a limit of 7 levels of indirection, primarily to
safe-guard against ``run-away'' indirection which happens (for instance) when a
file specifies \emph{itself} within an indirection element.

Indirection elements are always considered to be case sensitive, even if the
group has been designated case insensitive. This is because file names on
certain operating systems (eg UNIX) are always considered case sensitive, and so
problems would arise while accessing indirection files if GRP was to consider
them case \emph{in}sensitive.

The file name can contain shell meta-characters (references to
environment variables for instance) which will be expanded before the
file is used.

\subsection{\label{SEC:MOD}Modification Elements}
A modification element causes GRP to generate a set of names by copying the
names from another group. These new names can then be modified using the
facilities for editing names described above. The application specifies which
group is to be used as the basis for the new names. A special character (usually
a ``$*$'' character, but this can be changed if required) is used as a token to
represent all the names in the basis group. Thus:

\begin{small}
\begin{terminalv}
*|_DS|_BK|
\end{terminalv}
\end{small}

would cause all the names in the basis group to be modified by replacing the
string \textbf{\_DS} with the string \textbf{\_BK}. The basis names can also be
modified by the addition of a prefix and suffix. Following the description of
name editing given above, you may expect the format to be (for instance):

\begin{small}
\begin{terminalv}
Hello_{*}_Goodbye
\end{terminalv}
\end{small}

in which the token character takes on the role of the kernel. This does in fact
work, but in this case the opening and closing kernel delimiters (``\{'' and
``\}'') can be omitted because there is no ambiguity about where the kernel
starts and finishes. Thus a simpler form would be:

\begin{small}
\begin{terminalv}
Hello_*_Goodbye
\end{terminalv}
\end{small}

The addition of a prefix and suffix can be combined with substitution as
usual. For instance, the element:

\begin{small}
\begin{terminalv}
A*B|C|D|
\end{terminalv}
\end{small}

would cause all occurrences of the letter C within the names of the basis group
to be replaced with D, followed by the addition of the prefix A and the suffix
B.

If a ``null'' group is specified as the basis group (i.e.\ the group identifier
is given as GRP\_\_NOID), then there are no
names on which to base the new names and the token character is treated as a
literal name. That is, if the user gave the group expression

\begin{small}
\begin{terminalv}
A_*2|RAW|FLAT|
\end{terminalv}
\end{small}

and the application had specified a null group as the basis for modification
elements, the the specified editing would be applied to the \emph{literal name}
``$*$'', resulting in the single literal name ``\textbf{A\_$*$2}'' being added to the
group.

\subsection{\label{SEC:NEST}Nesting Within Group Expressions}

There is sometimes a clash of interests to be resolved when deciding on the best
choice for the character which delimits elements within a group expression. The
default delimiter character is the comma, but this character can sometimes be
useful \emph{within} an element, for instance when specifying a set of indices.
For instance, if the user gave the group expression:

\begin{small}
\begin{terminalv}
A(1,2),B(3,10)
\end{terminalv}
\end{small}

in which each element is a literal name corresponding to an array element, it
would be wrong to split this up using the commas as delimiters into the four
strings ``\textbf{A(1}'', ``\textbf{2)}'', ``\textbf{B(3}'' and ``\textbf{10)}''.

To get round this particular problem, GRP ignores delimiters which occur within
matching ``nesting characters''. There are two nesting characters, the ``open
nest'' character (usually set to ``('') and the ``close nest'' character
(usually set to ``)''). Thus in the above example, the commas occurring within
the parentheses would not be treated as delimiters, resulting in the group
expression being split into the two elements \textbf{A(1,2)} and \textbf{B(3,10)}. The
characters to use as the opening and closing nest characters may be set by the
calling application (see section \ref{SEC:CC}).

\subsection{\label{SEC:FLAG}Flagging a Group Expression}
GRP allows a group expression to be flagged by terminating it with a ``flag''
character (usually a minus sign although this can be changed, see section
\ref{SEC:CC}). If the last character in the group expression is a flag
character, then the FLAG argument of routine \htmlref{GRP\_GROUP}{GRP_GROUP} is returned true. The
flag character is stripped off the group expression before it is split up into
elements, so the flag character itself does not get included in any of the names
stored in the group.

A typical use of this facility might be to allow the user to request a further
prompt for more names. For instance, in the example of section \ref{SEC:EXAMC},
the user may wish to specify more input file names than will fit on a single
line. To allow this, the call to \htmlref{GRP\_GROUP}{GRP_GROUP} would be replaced with the
following:
\begin{small}
\begin{terminalv}
*  Loop round, prompting the user for group expressions
*  until one is found without a minus sign at the end, or an
*  error occurs.
      FLAG = .TRUE.
      DO WHILE( FLAG .AND. STATUS .EQ. SAI__OK )
         CALL GRP_GROUP( 'IN_FILES', GRP__NOID, IGRP1, SIZE1,
     :                   ADDED, FLAG, STATUS )

*  Cancel the parameter association to get a new group
*  expression on the next call to GRP_GROUP.
         CALL PAR_CANCL( 'IN_FILES', STATUS )

      END DO
\end{terminalv}
\end{small}

The user could then request a further prompt by appending a minus sign to the
end of the group expression, as follows:

\begin{small}
\begin{terminalv}
NEW_FILE,A_*2|RAW|FLAT|,^LIST.DAT-
\end{terminalv}
\end{small}

The names obtained at each prompt are appended to the end of the group, which
expands as necessary.

Note, if the final element in a group expression is an indirection element, the
flag character may be placed at the end of the last record in the indicated text
file. For instance, instead of giving:

\begin{small}
\begin{terminalv}
^LIST.DAT-
\end{terminalv}
\end{small}

where the file LIST.DAT contains the single record

\begin{small}
\begin{terminalv}
RED,GREEN,BLUE
\end{terminalv}
\end{small}

a user could ``hard-wire'' a flag character on to the end of LIST.DAT so that
it contains:

\begin{small}
\begin{terminalv}
RED,GREEN,BLUE-
\end{terminalv}
\end{small}

\subsection{Comments Within Group Expressions}
It is often useful to mix comments with names, particularly within a text file.
All group expressions (whether obtained from the environment or from a text
file or as an argument) are truncated if a ``comment'' character is found (usually `\#'' but this
can be changed, see section \ref{SEC:CC}). Anything occurring after such a
character is ignored. In a text file, the comment is assumed to extend from the
comment character to the end of the line, so a new group expression may be
given on the next line. Note, blank lines are \emph{not} ignored. Each blank line
within a text file will result in a blank name being added to the group.

\subsection{Escaping the Special Characters in a Group Expression}
If is possible to specify that a given character be used as an ``escape
character'' within group expressions. This facility is normally
suppressed, but an application can choose to switch it on by assigning a
value to the ESCAPE control character associated with a group (see
section \ref{SEC:CC}). If this is done, any special meaning associated
with a character within a group expression is ignored if the character is
preceeded by an escape character. The escape characters themselves are
not included in the resulting names if they preceed any of the other
``special'' control characters. Note, escape characters which do not
preceed another control character \emph{are} included in the resulting
names.

For instance, the group expression:

\begin{small}
\begin{terminalv}
* | A
\end{terminalv}
\end{small}

would normally result in an error because the ``\texttt{|}'' character
would be taken as the start of an incomplete specification for some
editing to apply to the preceeding text (assuming the application has not
changed the default editing behaviour). If, in fact, the user wants this
string to be accepted as a literal string (maybe representing a Unix
piping operation for instance), then the ``\texttt{|}'' should be escaped.
Assuming the application chooses to use the backslash character
``\texttt{\textbackslash}'' as the escape character, then this can be done by entering
the following group expression:

\begin{small}
\begin{terminalv}
* \| A
\end{terminalv}
\end{small}

The ``\texttt{\textbackslash}'' character results in the ``\texttt{|}'' character being
treated as part of the required string, rather than as the start of an
editing specification. The string returned to the application is then
``\texttt{* | A}'' (note, the escape character has been removed). Any
escape characters which do not preceed special characters are included
literaly in the returned string. So, for instance, if the group
expression was:

\begin{small}
\begin{terminalv}
\* \| A
\end{terminalv}
\end{small}

the string ``\texttt{\textbackslash * | A}'' would be returned to the application.

All escape characters within a section of a group expression can be
ignored by using the special strings ``$<$!!'' and ``!!$>$'' to mark the
start and end of the section.

\subsection{The Order of Names Within a Group}
Names are stored within a group in the order in which they are specified in the
group expression. For instance, if the file F1.DAT contained the following two
records:

\begin{small}
\begin{terminalv}
A,^F2
B,C
\end{terminalv}
\end{small}

and the file F2.DAT contained the following three records:

\begin{small}
\begin{terminalv}
D
E
F
\end{terminalv}
\end{small}

then the group expression:

\begin{small}
\begin{terminalv}
X,^F1.DAT,Y
\end{terminalv}
\end{small}

would result in the names being added to the group in the following order

\begin{small}
\begin{terminalv}
 X
 A
 D
 E
 F
 B
 C
 Y
\end{terminalv}
\end{small}

The contents of the two indirection files have been inserted at the position at
which the corresponding indirection element occurred. Names resulting from the
expansion of modification elements are similarly inserted into the list at the
position at which the modification element occurred. The modified names are
stored in the same order as the names within the group upon which the
modification was based. For example, if the above group is used as the basis for
modification, then the group expression:

\begin{small}
\begin{terminalv}
U,*_2,V
\end{terminalv}
\end{small}

would give rise to the group:

\begin{small}
\begin{terminalv}
U
X_2
A_2
D_2
E_2
F_2
B_2
C_2
Y_2
V
\end{terminalv}
\end{small}


\section{Using GRP Routines}

Many of the points described below are illustrated by the example application in
section \ref{SEC:EXAMC}. Further details can be found in the subroutine
specifications in appendix \ref{APP:SPEC}.

\subsection{Symbolic Constants and Status Values}
The GRP package has associated with it various symbolic constants defining such
things as the required length of various character variables, an invalid GRP
identifier value, etc. These values consist of a name of up to 5 characters
prefixed by ``GRP\_\_''  (note the \emph{double} underscore), and can be made
available to an application by including the line:

\begin{small}
\begin{terminalv}
INCLUDE 'GRP_PAR'
\end{terminalv}
\end{small}

Another set of symbolic constants is made available by the
statement:

\begin{small}
\begin{terminalv}
INCLUDE 'GRP_ERR'
\end{terminalv}
\end{small}

These values have the same format of those contained in GRP\_PAR, but define
error conditions which can be generated within the GRP package. Applications can
compare the STATUS argument with these values to check for specific error
conditions. These values are described in appendix \ref {APP:ERRORS}.

\subsection{Creating and Deleting Groups}
New groups can be created by calling \htmlref{GRP\_NEW}{GRP_NEW} which returns an identifier for
the new group. The group is initially empty but names can be added immediately
using any of the methods described in section \ref{SEC:ADDN}. The calling
application must provide a ``type'' for the group when calling \htmlref{GRP\_NEW}{GRP_NEW}. This
type is not used directly by the GRP system, but is provided as a means for
applications to differentiate between different \emph{types} of groups (for
instance, groups holding file names and groups holding names of astronomical
objects). No restrictions are placed on the strings which can be used for
group types. The type of a group can be recalled at any time using routine
\htmlref{GRP\_GTYPE}{GRP_GTYPE}, and a group can be given a new type by calling \htmlref{GRP\_PTYPE}{GRP_PTYPE}.

New groups are also created by the routines \htmlref{GRP\_COPY}{GRP_COPY}, \htmlref{GRP\_REMOV}{GRP_REMOV} and
\htmlref{GRP\_PURGE}{GRP_PURGE}, but they differ from \htmlref{GRP\_NEW}{GRP_NEW} in that the created group is based on
a previously existing group. \htmlref{GRP\_COPY}{GRP_COPY} creates a new group containing a copy of
a subsection of an old group, \htmlref{GRP\_REMOV}{GRP_REMOV} creates a new group containing a copy
of the whole of an old group, but excluding all occurrences of a given name, and
\htmlref{GRP\_PURGE}{GRP_PURGE} creates a new group containing a purged copy of the whole of an old
group (i.e.\ there are no duplicate entries in the new group). The new groups
inherit the type string, control characters and case sensitivity of the old
groups, but do not inherit any other attributes (such as owner/slave
relationships - see section \ref {SEC:OWN}).

The routine \htmlref{GRP\_DELET}{GRP_DELET} deletes a group, thus freeing the GRP identifier and the
internal resources used by the group. As well as deleting the group identified
by the specified GRP identifier, it also deletes any groups \emph{associated}
with the specified group (by means of an owner-slave relationship as described
in section \ref{SEC:OWN}).

To avoid running out of group identifiers, applications should always delete all
groups which they have created (using \htmlref{GRP\_DELET}{GRP_DELET}) before terminating, even if an
error status exists.

\subsection{\label{SEC:ADDN}Storing Names in a Group}
There are several routines which can be used to store names in a group.

The routine \htmlref{GRP\_PUT}{GRP_PUT} stores a set of names starting at a given index within a
group. Any previous names with the same indices are over-written, and the group
is extended as necessary. Note, an application must supply the \emph{literal}
names to be stored, as modification and indirection elements cannot be used with
\htmlref{GRP\_PUT}{GRP_PUT}. For instance, if the string ``\wedge LIST.DAT'' was stored in a group
using \htmlref{GRP\_PUT}{GRP_PUT}, the \wedge character would \emph{not} cause names to be read
from the file LIST.DAT. Instead the string would be stored in the group exactly
as supplied.

The routine \htmlref{GRP\_PUT1}{GRP_PUT1} is like GRP\_PUT except that it
stores just a single name, and so has a simpler interface.

The routine \htmlref{GRP\_GROUP}{GRP_GROUP} obtains a group expression from the environment using a
specified parameter and expands it into a list of literal names. These names are
appended to the end of a group, which must previously have been created. An
existing group may be specified as the basis group for any modification elements
included in the group expression. If such a group is not supplied, all elements
are stored literally as names. \htmlref{GRP\_GROUP}{GRP_GROUP} returns the total number of names in
the group, together with the number of names which it has added. It also returns
a logical argument indicating if the group expression was ``flagged'' (see
section \ref{SEC:FLAG}).

The routine \htmlref{GRP\_GRPEX}{GRP_GRPEX} performs the same function as \htmlref{GRP\_GROUP}{GRP_GROUP} except
that the group expression is supplied as an argument instead of being
obtained through the parameter system. This routine can thus be used in
stand-alone applications.

The following code fragment shows an example of the use of \htmlref{GRP\_GRPEX}{GRP_GRPEX} and
\htmlref{GRP\_PUT}{GRP_PUT}. It is assumed that the text file FRIENDS.LIS exists and contains the
single record ``FRED,BERT''.

\begin{small}
\begin{terminalv}
      INTEGER IGRP, STATUS, SIZE, ADDED
      LOGICAL FLAG
      CHARACTER * ( GRP__SZNAM ) NAMES( 3 )

      ...

      CALL GRP_NEW( 'A test group', IGRP, STATUS )

      CALL GRP_GRPEX( '^FRIENDS.LIS', GRP__NOID, IGRP, SIZE, ADDED, FLAG,
     :                STATUS )

      CALL GRP_GRPEX( 'HAROLD', GRP__NOID, IGRP, SIZE, ADDED, FLAG,
     :                STATUS )
      ...

      NAMES( 1 ) = 'TOM'
      NAMES( 2 ) = 'DICK'
      NAMES( 3 ) = 'HARRY'

      ...

      CALL GRP_PUT( IGRP, 3, NAMES, 2, STATUS )
\end{terminalv}
\end{small}

The call to \htmlref{GRP\_NEW}{GRP_NEW} creates a new, empty group. The first call to \htmlref{GRP\_GRPEX}{GRP_GRPEX}
read the two names from the file FRIENDS.LIS and stores them with the following
indices:

\begin{small}
\begin{terminalv}
1 - FRED
2 - BERT
\end{terminalv}
\end{small}

Both SIZE and ADDED are returned equal to two. The second call to \htmlref{GRP\_GRPEX}{GRP_GRPEX}
appends the name \textbf{HAROLD} to the group, so that the group becomes:

\begin{small}
\begin{terminalv}
1 - FRED
2 - BERT
3 - HAROLD
\end{terminalv}
\end{small}

SIZE is returned equal to 3 and ADDED equal to 1. The call to \htmlref{GRP\_PUT}{GRP_PUT}
then stores the name \textbf{TOM} at index 2 (over-writing \textbf{BERT}), and \textbf{DICK} at index 3 (over-writing \textbf{HAROLD}). The group is then extended and
\textbf{HARRY} is stored at index 4 so that the group becomes:

\begin{small}
\begin{terminalv}
1 - FRED
2 - TOM
3 - DICK
4 - HARRY
\end{terminalv}
\end{small}

\subsection{Retreiving Names from a Group}
\htmlref{GRP\_GET}{GRP_GET} will retrieve a set of names from a group starting at a given index. An
error is reported if a range of indices extending beyond the current size of
the group is specified. If a character variable which receives a name  has a
declared length shorter than the stored name, then the name is truncated to the
length of the character variable, but no error is reported.

A group can be searched for a specified name by calling \htmlref{GRP\_INDEX}{GRP_INDEX}, which
returns the index of the name within the specified group. If the name does
not exist within the group, the returned index is set to zero but no error is
reported.

The routines \htmlref{GRP\_LIST}{GRP_LIST} and \htmlref{GRP\_LISTF}{GRP_LISTF}
each create a text file containing a
subset of the names stored in a specified group. The names are written one per
record, and a comment is stored as the first record. The difference between
\htmlref{GRP\_LIST}{GRP_LIST} and \htmlref{GRP\_LISTF}{GRP_LISTF} is that the former
obtains the name of the text file to
be created through the parameter system, whereas the latter requires the calling
application to provide it. These routines are useful for passing groups of names
between successive applications. For instance, the first application may create
a text file holding a group of names using \htmlref{GRP\_LIST}{GRP_LIST}, and the second application
may then call \htmlref{GRP\_GROUP}{GRP_GROUP}, allowing the file created by the previous application
to be specified within an indirection element.

A single name can also be retrieved from a group using routine \htmlref{GRP\_INFOC}{GRP_INFOC},
giving the value ``NAME'' for the argument ITEM (see below).

\subsection{Retrieving Attributes of Names}
Each name has associated with it various attributes, which can be retrieved
using \htmlref{GRP\_INFOC}{GRP_INFOC} (for attributes which take character values) and \htmlref{GRP\_INFOI}{GRP_INFOI}
(for attributes which take integer values). In both cases, the particular
attribute required is specified by the argument ITEM. Attributes of a name
are carried round with the name if it is copied using \htmlref{GRP\_COPY}{GRP_COPY}, \htmlref{GRP\_REMOV}{GRP_REMOV} or
\htmlref{GRP\_PURGE}{GRP_PURGE}. The following attributes are currently used:

\begin{description}
\item [MODGRP] - An integer which has the value GRP\_\_NOID unless the name was
specified using a modification element. Otherwise, it is the identifier for the
group which was used as the basis for the modification element (see section
\ref{SEC:MOD}).

\item [MODIND] - An integer value which is zero unless the name was specified
using a modification element. Otherwise, it is the index of the name within
the basis group (i.e.\ the group identified by the MODGRP attribute) which was
modified in order to generate the specified name.

\item [DEPTH] - An integer value which is zero unless the name was specified
using an indirection element. Otherwise, it is the number of levels of
indirection at which the literal name was specified. For instance, consider the
case where a prompt
issued by \htmlref{GRP\_GROUP}{GRP_GROUP} is responded to with the group expression

\begin{small}
\begin{terminalv}
^FILE1.DAT
\end{terminalv}
\end{small}

and the file FILE1.DAT contains the single record ``NAME1,\wedge FILE2.DAT'',
and the file FILE2.DAT contained the single record ``NAME2''. The DEPTH
attribute would have a value of 1 for name \textbf{NAME1} indicating that the name
was given within a file specified directly within the group expression, and a
value of 2 for name \textbf{NAME2} indicating that the name was given within a file
which was itself given within another file.

\item [FILE] - A character value which is returned blank unless the name was
specified using an indirection element. Otherwise, it is the name of the file in
which the name was literally stored. Thus, in the above example it would be
FILE1.DAT for name \textbf{NAME1} and FILE2.DAT for \textbf{NAME2}.

\item [NAME] - A character value equal to the name itself.

\end{description}

\subsection{Deleting Names}
Names can be deleted from a group using \htmlref{GRP\_REMOV}{GRP_REMOV}, which deletes all
occurrences of a specified name within a group, creating a new group to hold the
results (the input group is left unchanged).

Names with a specified range of indices can be deleted from a group using
\htmlref{GRP\_COPY}{GRP_COPY}, as in the following example:

\begin{small}
\begin{terminalv}
CALL GRP_COPY( IGRP, 3, 5, .TRUE., ITEMP, STATUS )
CALL GRP_DELET( IGRP, STATUS )
IGRP = ITEMP
\end{terminalv}
\end{small}

The call to \htmlref{GRP\_COPY}{GRP_COPY} creates a copy of the group identified by IGRP,
excluding the names with indices between 3 and 5. The copy is stored in a new
group, the identifier for which is returned in ITEMP. The names which used to
have indices greater than 5 are shuffled down to fill up the gap (name 6 in
the old group becomes name 3 in the new group, etc). The above example then
deletes the original group, and copies the new group identifier into the
variable previously used to store the identifier to the old group.

Names can also be deleted using \htmlref{GRP\_SETSZ}{GRP_SETSZ} which reduces the size of a group.
Thus, if a group with size of 6 is reduced to a size of 4, then the names with
indices 5 and 6 are deleted.

Routine \htmlref{GRP\_PURGE}{GRP_PURGE} deletes duplicate names within a group, creating a new group
to hold the results.

\subsection{\label{SEC:CC}Changing the Characters Used to Control the Syntax of Group
Expressions}

A set of ``control characters'' are used to indicate various items of
syntax within
a group expression. Each group is created with a set of default control
characters, but these can be changed by subsequent calls to \htmlref{GRP\_SETCC}{GRP_SETCC}. Each
group has its own set of control characters which can be different to those of
other groups, and they can be inspected using routine
\htmlref{GRP\_GETCC}{GRP_GETCC}. When calling
these routines, specific control characters are represented by the following
names:

\begin{description}
\item [INDIRECTION] - The character used to indicate that the string which
follows is not a literal name but is the name of a text file from which further
names are to be read. The INDIRECTION control character defaults to
``\wedge''.

\item [COMMENT] - The character used to indicate that the remainder of the group
expression (or record within a text file) is a comment and is to be ignored.
The COMMENT control character defaults to ``\#''.

\item [DELIMITER] - The character used to delimit elements within a group
expression. The DELIMITER control character defaults to ``,''.

\item [NAME\_TOKEN] - The character used within a modification element as a
token for each name in the basis group. The NAME\_TOKEN character defaults to
``$*$''.

\item [SEPARATOR] - The character used to separate old and new substitution
strings when editing a group of names. The SEPARATOR character defaults to
``$\mid$''.

\item [OPEN\_NEST] - The character used to open a nest within a group
expression, within which any DELIMITER characters are ignored. The OPEN\_NEST
character defaults to ``(''.

\item [CLOSE\_NEST] - The character used to close a nest within a group
expression, within which any DELIMITER characters are ignored. The CLOSE\_NEST
character defaults to ``)''.

\item [FLAG] - The character which can be used to flag selected group
expressions by appending it to the end of the group expression. The presence of
the flag is communicated to the calling application by means of the FLAG
argument of routines \htmlref{GRP\_GROUP}{GRP_GROUP} and \htmlref{GRP\_GRPEX}{GRP_GRPEX}. The FLAG character defaults
to ``-''.

\item [OPEN\_KERNEL] - The character used to open a kernel within an element.
The OPEN\_KERNEL character defaults to ``\{''.

\item [CLOSE\_KERNEL] - The character used to close a kernel within an element.
The CLOSE\_KERNEL character defaults to ``\}''.

\item [NULL] - This is a character which can be assigned to any other control
character to suppress use of that control character (see below). The NULL
character defaults to ``\%''.

\item [ESCAPE] - This is a character which can be used to ``escape'' another
control character within a group expression. If a GRP control character is
encountered within a group expression, its special meaning is ignored
and it is treated literally if it is preceeded by an escape character.
Such escape characters are removed before storing the corresponding
strings in the returned group (escape character which do not preceed
another control character are not removed). Escape characters can
themselves be escaped. By default, the escape character is equal to the
null character (i.e.\ escaping is not available by default).

\end{description}

\subsubsection {Suppressing the Use of Selected Control Characters}
Applications may sometimes want to suppress the use of certain control
characters. For instance, an application which stores lines of text within a
group may want to suppress the division of each group expression into separate
elements, so that the whole group expression is stored as a single name within
the group. One way to do this would be to set the DELIMITER control character to
a character which is guaranteed not to occur within the supplied text. However,
such a character may not exist. A more flexible approach is to assign the value
of the NULL control character to the DELIMITER control character. For instance,
if the NULL and DELIMITER characters are both set to ``\%'', then the group
expression is not split up into elements delimited by ``\%'' signs. Note, any
``\%'' signs in the group expression will be included in the name stored in the
group without any change. The NULL character can be assigned to any of the other
control characters to suppress the use of those control character. The only
restriction is that if one of the two control characters OPEN\_NEST and
CLOSE\_NEST is assigned the NULL value, then the other one must also be assigned
the NULL value. The same restriction applies to the OPEN\_KERNEL and
CLOSE\_KERNEL control characters.

\subsection{\label{SEC:OWN}Creating Associations Between Groups}

Each group may be ``owned'' by another group. This idea of ownership may be
used to
establish associations between groups. Many groups may be associated together
into a sort of ``super-group'', or group of groups, by making one group the
owner of another, which in turn is the owner of another, and so on. These
associations are provided for the benefit of the calling application, which can
impose any meaning it likes on the associations. The only significance of these
associations within the GRP package is that when a group is deleted (using
\htmlref{GRP\_DELET}{GRP_DELET}), all other groups in the same ``super-group'' (i.e.\ all groups which
have an association, either directly, or indirectly, with the specified group)
are also deleted.

An example of the use of these associations is to provide a means of attaching
supplemental information to a group describing the global properties of the
group. For instance, consider an application which uses groups to store
information about spiral galaxies. Group A may contain the names \textbf{1232},
\textbf{628}, \textbf{5364}, and group B may contain \textbf{81}, \textbf{31}. It may be
necessary to store supplemental information describing each of these groups, for
instance the catalogue to which the numbers in groups A and B refer, the
morphological type, and so. A new group C could be created holding the names
\textbf{NGC} and \textbf{ScI}, giving the catalogue and type of the galaxies in group
A. This group could then be associated with group A. Similarly a group D could
be created holding the names \textbf{Messier} and \textbf{Sb}, and associated with
group B.

Four routines are included in the GRP package for handling these
associations;
\htmlref{GRP\_SOWN}{GRP_SOWN} establishes one group as owner of another,
\htmlref{GRP\_OWN}{GRP_OWN} returns the
identifier of the group which owns a specified group,
\htmlref{GRP\_SLAVE}{GRP_SLAVE} returns the
identifier of the group which is owned by (i.e.\ is a slave of) a specified group,
and \htmlref{GRP\_HEAD}{GRP_HEAD} returns the
group at the head of an owner-slave chain. Groups are ``free'' (i.e.\ are not
owned by any other group) when they are created. A group which has had an owner
established for it can be made free again by calling \htmlref{GRP\_SOWN}{GRP_SOWN} giving a null
group (GRP\_\_NOID) as the ``owner''. A group may not be owned by more than one
group, and may itself own a maximum of one group. It is thus possible to
establish ``chains'' of owner-slave associations (but not ``tree structures'').
All the groups in such a chain form a ``super-group'' of inter-dependant groups.
When an attempt is made to delete any group within such a super-group, all the
other groups in the same super-group are also deleted whether higher or lower in
the chain.

The owner-slave relationship is strictly one way; a group cannot own its own
owner, or equivalently be a slave of its own slave. If an attempt is
made to set up such an association using \htmlref{GRP\_SOWN}{GRP_SOWN}, then an error is reported,
and STATUS is returned set to GRP\_\_SLAVE. Thus if group A is owned by group B,
and group B is owned by group C, group C may not be owned by either group A or
group B.

\section{Compiling and Linking}
To compile and link an ADAM application with the GRP package, the following
commands should be used (see \xref{SUN/144}{sun144}{}):

\begin{small}
\begin{terminalv}
% grp_dev
% alink adamprog.f `grp_link_adam`
% grp_dev remove
\end{terminalv}
\end{small}

Note the use of \emph{opening} apostrophes (`) rather than the more common
closing apostrophe ('). To compile and link a stand-alone application with
the GRP package, the following commands should be used:

\begin{small}
\begin{terminalv}
% grp_dev
% f77 prog.f -o prog `grp_link`
% grp_dev remove
\end{terminalv}
\end{small}

This produces an executable image called \textbf{prog}. The \texttt{grp\_dev}
command creates soft links to the include files GRP\_PAR and GRP\_ERR
within the current directory. These are removed by the \texttt{grp\_dev remove} command.

The ADAM and stand-alone versions of the GRP\_ system differ, in that those
routines which use ADAM facilities (\emph{i.e.}\ those listed in
\ref{SEC:ADAM}) are not available in the stand-alone version.

\appendix
\section{Alphabetical List of Routines}

% List of routines:
% =================

\noteroutine{\htmlref{GRP\_DELET}{GRP_DELET}}{( IGRP, STATUS )}
            {Delete a group}
\noteroutine{\htmlref{GRP\_COPY}{GRP_COPY}}{( IGRP1, INDXLO, INDXHI, REJECT, IGRP2, STATUS )}
            {Create a new group containing a copy of a section of an old group}
\noteroutine{\htmlref{GRP\_GET}{GRP_GET}}{( IGRP, INDEX, SIZE, NAMES, STATUS )}
            {Retrieve names from a group}
\noteroutine{\htmlref{GRP\_GETCC}{GRP_GETCC}}{( IGRP, CCLIST, CC, STATUS )}
            {Return control characters for a given group}
\noteroutine{\htmlref{GRP\_GETCS}{GRP_GETCS}}{( IGRP, SENSIT, STATUS )}
            {Determine if a group is case sensitive or not}
\noteroutine{\htmlref{GRP\_GROUP}{GRP_GROUP}}{( PARAM, IGRP1, IGRP2, SIZE, ADDED, FLAG, STATUS )}
            {Add names obtained from the environment into a group}
\noteroutine{\htmlref{GRP\_GRPEX}{GRP_GRPEX}}{( GRPEXP, IGRP1, IGRP2, SIZE, ADDED, FLAG, STATUS )}
            {Add names specified by a given group expression into a group}
\noteroutine{\htmlref{GRP\_GRPSZ}{GRP_GRPSZ}}{( IGRP, SIZE, STATUS )}
            {Return the current size of a group}
\noteroutine{\htmlref{GRP\_GTYPE}{GRP_GTYPE}}{( IGRP, TYPE, STATUS )}
            {Return the type string stored with a group}
\noteroutine{\htmlref{GRP\_HEAD}{GRP_HEAD}}{( IGRP1, IGRP2, STATUS )}
            {Find the group at the head of an owner-slave chain}
\noteroutine{\htmlref{GRP\_INDEX}{GRP_INDEX}}{( NAME, IGRP, START, INDEX, STATUS )}
            {Find the index of a given name within a group}
\noteroutine{\htmlref{GRP\_INFOC}{GRP_INFOC}}{( IGRP, INDEX, ITEM, VALUE, STATUS )}
            {Return an item of character information relating to a specific name}
\noteroutine{\htmlref{GRP\_INFOI}{GRP_INFOI}}{( IGRP, INDEX, ITEM, VALUE, STATUS )}
            {Return an item of integer information relating to a specific name}
\noteroutine{\htmlref{GRP\_LIST}{GRP_LIST}}{( PARAM, INDXLO, INDXHI, COMNT, IGRP, STATUS )}
            {Write all the names in a group out to a text file specified by the
environment}
\noteroutine{\htmlref{GRP\_LISTF}{GRP_LISTF}}{( FILENM, INDXLO, INDXHI, COMNT, IGRP, STATUS )}
            {Write all the names in a group out to a text file specified by the
calling application}
\noteroutine{\htmlref{GRP\_MSG}{GRP_MSG}}{( TOKEN, IGRP, INDEX )}
            {Assign a group element to a message token}
\noteroutine{\htmlref{GRP\_NEW}{GRP_NEW}}{( TYPE, IGRP, STATUS )}
            {Create a new, empty group}
\noteroutine{\htmlref{GRP\_OWN}{GRP_OWN}}{( IGRP1, IGRP2, STATUS )}
            {Return the identifier of the group which ``owns'' a specified
group}
\noteroutine{\htmlref{GRP\_PTYPE}{GRP_PTYPE}}{( IGRP, TYPE, STATUS )}
            {Store a new type string with a group}
\noteroutine{\htmlref{GRP\_PURGE}{GRP_PURGE}}{( IGRP1, IGRP2, STATUS )}
            {Remove duplicate entries from a group, putting the results in a new
group}
\noteroutine{\htmlref{GRP\_PUT}{GRP_PUT}}{( IGRP, SIZE, NAMES, INDEX, STATUS )}
            {Store a list of explicit names in a group}
\noteroutine{\htmlref{GRP\_PUT1}{GRP_PUT1}}{( IGRP, NAME, INDEX, STATUS )}
            {Store a single name in a group}
\noteroutine{\htmlref{GRP\_REMOV}{GRP_REMOV}}{( IGRP1, NAME, IGRP2, STATUS )}
            {Remove all occurrences of a given name from a group, putting the results in a new group}
\noteroutine{\htmlref{GRP\_SETCC}{GRP_SETCC}}{( IGRP, CCLIST, CC, STATUS )}
            {Set up new control characters for a group}
\noteroutine{\htmlref{GRP\_SETCS}{GRP_SETCS}}{( IGRP, SENSIT, STATUS )}
            {Establish a group as case sensitive or case insensitive}
\noteroutine{\htmlref{GRP\_SETSZ}{GRP_SETSZ}}{( IGRP, SIZE, STATUS )}
            {Reduce the size of a group}
\noteroutine{\htmlref{GRP\_SLAVE}{GRP_SLAVE}}{( IGRP1, IGRP2, STATUS )}
            {Return the identifier of the group which is owned by a
specified group}
\noteroutine{\htmlref{GRP\_SOWN}{GRP_SOWN}}{( IGRP1, IGRP2, STATUS )}
            {Establish one group as ``owner'' of another group}
\noteroutine{\htmlref{GRP\_VALID}{GRP_VALID}}{( IGRP, VALID, STATUS )}
            {Check if a group identifier is valid}

\section{Classified List of Routines}

\subsection{Creating and Deleting Groups}

\noteroutine{\htmlref{GRP\_DELET}{GRP_DELET}}{( IGRP, STATUS )}
            {Delete a group}
\noteroutine{\htmlref{GRP\_COPY}{GRP_COPY}}{( IGRP1, INDXLO, INDXHI, REJECT, IGRP2, STATUS )}
            {Create a new group containing a copy of a section of an old group}
\noteroutine{\htmlref{GRP\_NEW}{GRP_NEW}}{( TYPE, IGRP, STATUS )}
            {Create a new, empty group}
\noteroutine{\htmlref{GRP\_PURGE}{GRP_PURGE}}{( IGRP1, IGRP2, STATUS )}
            {Remove duplicate entries from a group, putting the results in a new
group}
\noteroutine{\htmlref{GRP\_REMOV}{GRP_REMOV}}{( IGRP1, NAME, IGRP2, STATUS )}
            {Remove all occurrences of a given name from a group, putting the results in a new group}

\subsection{Enquiring Group Attributes}

\noteroutine{\htmlref{GRP\_GETCC}{GRP_GETCC}}{( IGRP, CCLIST, CC, STATUS )}
            {Return control characters for a given group}
\noteroutine{\htmlref{GRP\_GETCS}{GRP_GETCS}}{( IGRP, SENSIT, STATUS )}
            {Determine if a group is case sensitive or not}
\noteroutine{\htmlref{GRP\_GRPSZ}{GRP_GRPSZ}}{( IGRP, SIZE, STATUS )}
            {Return the current size of a group}
\noteroutine{\htmlref{GRP\_GTYPE}{GRP_GTYPE}}{( IGRP, TYPE, STATUS )}
            {Return the type string stored with a group}
\noteroutine{\htmlref{GRP\_HEAD}{GRP_HEAD}}{( IGRP1, IGRP2, STATUS )}
            {Find the group at the head of an owner-slave chain}
\noteroutine{\htmlref{GRP\_OWN}{GRP_OWN}}{( IGRP1, IGRP2, STATUS )}
            {Return the identifier of the group which ``owns'' a specified
group}
\noteroutine{\htmlref{GRP\_SLAVE}{GRP_SLAVE}}{( IGRP1, IGRP2, STATUS )}
            {Return the identifier of the group which is owned by a
specified group}
\noteroutine{\htmlref{GRP\_VALID}{GRP_VALID}}{( IGRP, VALID, STATUS )}
            {Check if a group identifier is valid}


\subsection{Setting Group Attributes}

\noteroutine{\htmlref{GRP\_DELET}{GRP_DELET}}{( IGRP, STATUS )}
            {Delete a group}
\noteroutine{\htmlref{GRP\_PTYPE}{GRP_PTYPE}}{( IGRP, TYPE, STATUS )}
            {Store a new type string with a group}
\noteroutine{\htmlref{GRP\_SETCC}{GRP_SETCC}}{( IGRP, CCLIST, CC, STATUS )}
            {Set up new control characters for a group}
\noteroutine{\htmlref{GRP\_SETCS}{GRP_SETCS}}{( IGRP, SENSIT, STATUS )}
            {Establish a group as case sensitive or case insensitive}
\noteroutine{\htmlref{GRP\_SETSZ}{GRP_SETSZ}}{( IGRP, SIZE, STATUS )}
            {Reduce the size of a group}
\noteroutine{\htmlref{GRP\_SOWN}{GRP_SOWN}}{( IGRP1, IGRP2, STATUS )}
            {Establish one group as ``owner'' of another group}


\subsection{Enquiring Name Attributes}

\noteroutine{\htmlref{GRP\_INDEX}{GRP_INDEX}}{( NAME, IGRP, START, INDEX, STATUS )}
            {Find the index of a given name within a group}
\noteroutine{\htmlref{GRP\_INFOC}{GRP_INFOC}}{( IGRP, INDEX, ITEM, VALUE, STATUS )}
            {Return an item of character information relating to a specific name}
\noteroutine{\htmlref{GRP\_INFOI}{GRP_INFOI}}{( IGRP, INDEX, ITEM, VALUE, STATUS )}
            {Return an item of integer information relating to a specific name}


\subsection{Retreiving Names}

\noteroutine{\htmlref{GRP\_GET}{GRP_GET}}{( IGRP, INDEX, SIZE, NAMES, STATUS )}
            {Retrieve names from a group}
\noteroutine{\htmlref{GRP\_INFOC}{GRP_INFOC}}{( IGRP, INDEX, ITEM, VALUE, STATUS )}
            {Return an item of character information relating to a specific name}
\noteroutine{\htmlref{GRP\_LIST}{GRP_LIST}}{( PARAM, INDXLO, INDXHI, COMNT, IGRP, STATUS )}
            {Write all the names in a group out to a text file specified by the
environment}
\noteroutine{\htmlref{GRP\_LISTF}{GRP_LISTF}}{( FILENM, INDXLO, INDXHI, COMNT, IGRP, STATUS )}
            {Write all the names in a group out to a text file specified by the
calling application}
\noteroutine{\htmlref{GRP\_MSG}{GRP_MSG}}{( TOKEN, IGRP, INDEX )}
            {Assign a group element to a message token}


\subsection{Storing Names}

\noteroutine{\htmlref{GRP\_COPY}{GRP_COPY}}{( IGRP1, INDXLO, INDXHI, REJECT, IGRP2, STATUS )}
            {Create a new group containing a copy of a section of an old group}
\noteroutine{\htmlref{GRP\_GROUP}{GRP_GROUP}}{( PARAM, IGRP1, IGRP2, SIZE, ADDED, FLAG, STATUS )}
            {Add names obtained from the environment into a group}
\noteroutine{\htmlref{GRP\_GRPEX}{GRP_GRPEX}}{( GRPEXP, IGRP1, IGRP2, SIZE, ADDED, FLAG, STATUS )}
            {Add names specified by a given group expression into a group}
\noteroutine{\htmlref{GRP\_PUT}{GRP_PUT}}{( IGRP, SIZE, NAMES, INDEX, STATUS )}
            {Store a list of explicit names in a group}
\noteroutine{\htmlref{GRP\_PUT1}{GRP_PUT1}}{( IGRP, NAME, INDEX, STATUS )}
            {Store a single name in a group}



\subsection{Deleting Names}

\noteroutine{\htmlref{GRP\_COPY}{GRP_COPY}}{( IGRP1, INDXLO, INDXHI, REJECT, IGRP2, STATUS )}
            {Create a new group containing a copy of a section of an old group}
\noteroutine{\htmlref{GRP\_PURGE}{GRP_PURGE}}{( IGRP1, IGRP2, STATUS )}
            {Remove duplicate entries from a group, putting the results in a new
group}
\noteroutine{\htmlref{GRP\_REMOV}{GRP_REMOV}}{( IGRP1, NAME, IGRP2, STATUS )}
            {Remove all occurrences of a given name from a group, putting the results in a new group}
\noteroutine{\htmlref{GRP\_SETSZ}{GRP_SETSZ}}{( IGRP, SIZE, STATUS )}
            {Reduce the size of a group}


\subsection{Creation and Control of Identifiers}

\noteroutine{\htmlref{GRP\_DELET}{GRP_DELET}}{( IGRP, STATUS )}
            {Delete a group}
\noteroutine{\htmlref{GRP\_COPY}{GRP_COPY}}{( IGRP1, INDXLO, INDXHI, REJECT, IGRP2, STATUS )}
            {Create a new group containing a copy of a section of an old group}
\noteroutine{\htmlref{GRP\_NEW}{GRP_NEW}}{( TYPE, IGRP, STATUS )}
            {Create a new, empty group}
\noteroutine{\htmlref{GRP\_PURGE}{GRP_PURGE}}{( IGRP1, IGRP2, STATUS )}
            {Remove duplicate entries from a group, putting the results in a new
group}
\noteroutine{\htmlref{GRP\_REMOV}{GRP_REMOV}}{( IGRP1, NAME, IGRP2, STATUS )}
            {Remove all occurrences of a given name from a group, putting the results in a new group}
\noteroutine{\htmlref{GRP\_VALID}{GRP_VALID}}{( IGRP, VALID, STATUS )}
            {Check if a group identifier is valid}

\subsection{\label{SEC:ADAM}ADAM Parameter System Routines}


\noteroutine{\htmlref{GRP\_GROUP}{GRP_GROUP}}{( PARAM, IGRP1, IGRP2, SIZE, ADDED, FLAG, STATUS )}
            {Add names obtained from the environment into a group}
\noteroutine{\htmlref{GRP\_LIST}{GRP_LIST}}{( PARAM, INDXLO, INDXHI, COMNT, IGRP, STATUS )}
            {Write all the names in a group out to a text file specified by the
environment}


\subsection{\label{SEC:ADAM}Debugging Routines}

\noteroutine{\htmlref{GRP\_ALARM}{GRP_ALARM}}{( IGRP, EVENT, STATUS )}
            { Notify the user of an event in the life of a watched group}
\noteroutine{\htmlref{GRP\_WATCH}{GRP_WATCH}}{( IGRP, STATUS )}
            { Watch for events in the life of a specified group}

\newpage
\section{\label{APP:SPEC}Routine Descriptions}

% Routine descriptions:
% =====================

\sstroutine{
   GRP\_ALARM
}{
   Notify the user of an event in the life of a watched group
}{
   \sstdescription{
      This routine is called internally by the GRP library when a
      significant event happens in the life of a group that is being
      watched as a result of a call to GRP\_WATCH. It is not intended
      to be called by the user. It\texttt{'}s main purpose is to provide a break
      point target for debugging the creation and destruction of GRP
      groups.
   }
   \sstinvocation{
      CALL GRP\_ALARM( IGRP, EVENT, STATUS )
   }
   \sstarguments{
      \sstsubsection{
         IGRP = INTEGER (Given)
      }{
         An identifier for the created group. GRP\_\_NOID is returned if
         an error occurs.
      }
      \sstsubsection{
         EVENT = CHARACTER $*$ ( $*$ ) (Given)
      }{
         This will be \texttt{"}CREATE\texttt{"} or \texttt{"}DELETE\texttt{"}.
      }
      \sstsubsection{
         STATUS = INTEGER (Given and Returned)
      }{
         The global status.
      }
   }
   \sstnotes{
      \sstitemlist{

         \sstitem
         This routine attempts to run even if the global status value
         indicates an error has already occurred.
      }
   }
}
\sstroutine{
   GRP\_COPY
}{
   Copy a section of an existing group to a new group
}{
   \sstdescription{
      A new group is created by copying a section of an existing group
      specified by the range of indices supplied in INDXLO and INDXHI.
      The whole group is used if INDXLO and INDXHI are both zero.
      The output group can be formed in one of two ways:

      1) All names from the input group are copied to the output group
      except for those with indices in the given range.

      2) Only those names from the input group which have indices within
      the given range are copied to the output group.

      The method used is determined by the argument REJECT. Note, a
      name with a given index in the input group will have a different
      index in the output group if INDXLO is not 1 (or zero). The new
      group inherits the type, control characters and case sensitivity
      flag of the old group, but does not inherit any owner/slave
      relationships (see routine \htmlref{GRP\_SOWN}{GRP_SOWN}). If the input group is no
      longer required, it should be deleted using \htmlref{GRP\_DELET}{GRP_DELET}.
   }
   \sstinvocation{
      CALL GRP\_COPY( IGRP, INDXLO, INDXHI, REJECT, IGRP2, STATUS )
   }
   \sstarguments{
      \sstsubsection{
         IGRP1 = INTEGER (Given)
      }{
         A GRP identifier for the input group.
      }
      \sstsubsection{
         INDXLO = INTEGER (Given)
      }{
         The lowest index to reject or to copy.
      }
      \sstsubsection{
         INDXHI = INTEGER (Given)
      }{
         The highest index to reject or to copy.
      }
      \sstsubsection{
         REJECT = LOGICAL ( Given)
      }{
         If reject is .TRUE., then names in the given range are
         rejected.  Otherwise, names in the given range are copied.
      }
      \sstsubsection{
         IGRP2 = INTEGER (Returned)
      }{
         A GRP identifier for the created group. Returned equal to
         GRP\_\_NOID if an error occurs.
      }
      \sstsubsection{
         STATUS = INTEGER (Given and Returned)
      }{
         The global status.
      }
   }
}
\sstroutine{
   GRP\_DELET
}{
   Delete a group from the GRP system
}{
   \sstdescription{
      This routine releases the identifier and internal resources used
      by the specified group so that they can be used for another
      group.  There is a limited number of groups available for use
      within an application, so each group should be deleted when it is
      no longer needed to avoid the possibility of reaching the limit.

      Note, any parameter association used to establish the group is NOT
      cancelled.

      This routine also deletes any groups which are related to the
      supplied group by means of \texttt{"}owner-slave\texttt{"} relationships
      established by calls to \htmlref{GRP\_SOWN}{GRP_SOWN}. All groups in the same
      \texttt{"}owner-slave\texttt{"} chain are deleted whether higher up or lower down
      than the supplied group.

      This routine attempts to execute even if STATUS is bad on entry,
      although no further error report will be made if it subsequently
      fails under these circumstances.
   }
   \sstinvocation{
      CALL GRP\_DELET( IGRP, STATUS )
   }
   \sstarguments{
      \sstsubsection{
         IGRP = INTEGER (Given and Returned)
      }{
         A GRP identifier for the group. Returned equal to GRP\_\_NOID.
      }
      \sstsubsection{
         STATUS = INTEGER (Given and Returned)
      }{
         The global status.
      }
   }
}
\sstroutine{
   GRP\_GET
}{
   Returns a set of names contained in a group
}{
   \sstdescription{
      The names with indices between INDEX and INDEX$+$SIZE-1 (inclusive)
      contained in the given group are returned. An error is reported if
      the bounds of the group are exceeded, and STATUS is returned
      equal to GRP\_\_OUTBN. If the group is case insensitive (as
      established by a call to \htmlref{GRP\_SETCS}{GRP_SETCS}) then the names are converted
      to upper case before being returned.
   }
   \sstinvocation{
      CALL GRP\_GET( IGRP, INDEX, SIZE, NAMES, STATUS )
   }
   \sstarguments{
      \sstsubsection{
         IGRP = INTEGER (Given)
      }{
         A GRP identifier for the group.
      }
      \sstsubsection{
         INDEX = INTEGER (Given)
      }{
         The lowest index for which the corresponding name is required.
      }
      \sstsubsection{
         SIZE = INTEGER (Given)
      }{
         The number of names required.
      }
      \sstsubsection{
         NAMES( SIZE ) = CHARACTER $*$ ( $*$ ) (Returned)
      }{
         The names held at the given positions in the group. The
         corresponding character variables should have declared length
         specified by the symbolic constant GRP\_\_SZNAM. If the declared
         length is shorter than this, the returned names may be
         truncated, but no error is reported.
      }
      \sstsubsection{
         STATUS = INTEGER (Given and Returned)
      }{
         The global status.
      }
   }
}
\sstroutine{
   GRP\_GETCC
}{
   Returns requested control characters for the specified group
}{
   \sstdescription{
      Each group has associated with it several \texttt{"}control characters\texttt{"}
      which are the characters used to indicate various items of syntax
      within a group expression. These control characters can be
      changed at any time by calling \htmlref{GRP\_SETCC}{GRP_SETCC}. This routine returns
      the current values of a list of these control character.  The
      individual characters are described in GRP\_\_SETCC.
   }
   \sstinvocation{
      CALL GRP\_GETCC( IGRP, CCLIST, CC, STATUS )
   }
   \sstarguments{
      \sstsubsection{
         IGRP = INTEGER (Given)
      }{
         A GRP identifier for the group for which the control characters
         are required.
      }
      \sstsubsection{
         CCLIST = CHARACTER $*$ ( $*$ ) (Given)
      }{
         A comma separated list of control character names to be
         returned. See routine \htmlref{GRP\_SETCC}{GRP_SETCC} for a description of these
         names.
      }
      \sstsubsection{
         CC = CHARACTER $*$ ( $*$ ) (Returned)
      }{
         A character variable to receive the requested list of control
         characters. The control characters are stored at adjacent
         indices within this character variable, starting at index 1.
         The characters are stored in the same order that they are
         specified in CCLIST. An error is reported if the character
         variable is not long enough to receive all the requested
         control characters.
      }
      \sstsubsection{
         STATUS = INTEGER (Given and Returned)
      }{
         The global status.
      }
   }
}
\sstroutine{
   GRP\_GETCS
}{
   Determine the case sensitivity of a group
}{
   \sstdescription{
      Checks whether a group is currently case sensitive, or case
      insensitive (see routine \htmlref{GRP\_SETCS}{GRP_SETCS}).
   }
   \sstinvocation{
      CALL GRP\_GETCS( IGRP, SENSIT, STATUS )
   }
   \sstarguments{
      \sstsubsection{
         IGRP = INTEGER (Given)
      }{
         A GRP identifier for the group.
      }
      \sstsubsection{
         SENSIT = LOGICAL (Returned)
      }{
         Returned .TRUE. if the group is case sensitive and .FALSE.
         otherwise.
      }
      \sstsubsection{
         STATUS = INTEGER (Given and Returned)
      }{
         The global status.
      }
   }
}
\sstroutine{
   GRP\_GROUP
}{
   Append a list of names obtained from the environment
   to a previously created group
}{
   \sstdescription{
      A group expression is obtained from the environment
      using the supplied parameter name. The expression is parsed to
      produce a list of names which are appended to the end of the
      group identified by IGRP2. Note, no permanent association between
      the parameter and the group exists. The parameter value can be
      cancelled without effecting the contents of the group. If an error
      occurs while parsing the group expression, the user is re-prompted
      for a new group expression.

      If the group expression contains any modification elements, then
      the list of names added to the output group is based on the group
      identified by IGRP1. If IGRP1 is invalid (equal to the symbolic
      constant GRP\_\_NOID for instance), then any elements with the
      syntax of a modification element are stored in the output group
      as a single literal name.

      If the last character read from the group expression (or from a
      text file if the last element of the group expression is an
      indirection element) is equal to the current \texttt{"}flag\texttt{"} character for
      the group IGRP2 (see routine \htmlref{GRP\_SETCC}{GRP_SETCC}), then argument FLAG is
      returned set to .TRUE. Otherwise, it is returned set to .FALSE.
      The calling application can use this flag for any purpose (eg it
      may use it to indicate that the user wants to give more names).
      Note, the flag character itself is not included in the returned
      group.
   }
   \sstinvocation{
      CALL GRP\_GROUP( PARAM, IGRP1, IGRP2, SIZE, ADDED, FLAG, STATUS )
   }
   \sstarguments{
      \sstsubsection{
         PARAM = CHARACTER $*$ ( $*$ ) (Given)
      }{
         The ADAM parameter with which to associate the group
         expression. This may be of any type.
      }
      \sstsubsection{
         IGRP1 = INTEGER (Given)
      }{
         A GRP identifier for the group to be used as the basis for any
         modification elements which may be contained within the group
         expression obtained from the environment. This can be set to
         the symbolic constant GRP\_\_NOID if modification elements are
         to be treated as literal names.
      }
      \sstsubsection{
         IGRP2 = INTEGER (Given)
      }{
         A GRP identifier for the group to which the new names are to
         be appended.
      }
      \sstsubsection{
         SIZE = INTEGER (Returned)
      }{
         The number of names in the returned group. It is returned equal
         to 1 if an error status exists on entry. If an error occurs
         during execution of this routine, then SIZE is returned equal
         to the size of the group on entry (unless the group has zero
         size on entry, in which case it is returned equal to 1).
      }
      \sstsubsection{
         ADDED = INTEGER (Returned)
      }{
         The number of names added to the group as a result of the
         current call to this routine.
      }
      \sstsubsection{
         FLAG = LOGICAL (Returned)
      }{
         .TRUE. if the last character in the group expression is equal
         to the current flag character for group IGRP2. Note, if this
         is the case, then the flag character itself is not included
         in the returned group. FLAG is returned .FALSE. if the last
         character is not a flag character. Returned .FALSE. if an
         error occurs.
      }
      \sstsubsection{
         STATUS = INTEGER (Given and Returned)
      }{
         The global status.
      }
   }
   \sstnotes{
      \sstitemlist{

         \sstitem
         A null value (!) can be given for the parameter to indicate
         that no more names are to be specified. The corresponding error
         is annulled before returning unless no names have been added to
         the group.
      }
   }
}
\sstroutine{
   GRP\_GRPEX
}{
   Append a list of names contained within a supplied group
   expression to a previously created group
}{
   \sstdescription{
      The supplied group expression is expanded to produce a list of
      names which are appended to the end of the group identified by
      IGRP2.

      If the group expression contains any modification elements, then
      the list of names added to the output group is based on the group
      identified by IGRP1. If IGRP1 is invalid (equal to the symbolic
      constant GRP\_\_NOID for instance), then any elements with the
      syntax of a modification element are stored in the output group
      as a single literal name.

      If the last character read from the group expression (or from a
      text file if the last element of the group expression is an
      indirection element) is equal to the current \texttt{"}flag\texttt{"} character for
      the group IGRP2 (see routine \htmlref{GRP\_SETCC}{GRP_SETCC}), then argument FLAG is
      returned set to true. Otherwise, it is returned set to false. The
      calling application can use this flag for any purpose (eg it may
      use it to indicate that the user wants to give more names). Note,
      the flag character itself is not included in the returned group.
   }
   \sstinvocation{
      CALL GRP\_GRPEX( GRPEXP, IGRP1, IGRP2, SIZE, ADDED, FLAG, STATUS )
   }
   \sstarguments{
      \sstsubsection{
         GRPEXP = CHARACTER $*$ ( $*$ ) (Given)
      }{
         A group expression. This should not be longer than GRP\_\_SZGEX.
         If it is, the surplus characters will be ignored.
      }
      \sstsubsection{
         IGRP1 = INTEGER (Given)
      }{
         A GRP identifier for the group to be used as the basis for
         any modification elements which may be contained within the
         supplied group expression. This can be set to the symbolic
         constant GRP\_\_NOID if modification elements are to be treated
         as literal names.
      }
      \sstsubsection{
         IGRP2 = INTEGER (Given)
      }{
         A GRP identifier for the group to which the new names are to
         be appended.
      }
      \sstsubsection{
         SIZE = INTEGER (Returned)
      }{
         The number of names in the returned group. It is returned equal
         to 1 if an error status exists on entry. If an error occurs
         during execution of this routine, then SIZE is returned equal
         to the size of the group on entry (unless the group has zero
         size on entry, in which case it is returned equal to 1).
      }
      \sstsubsection{
         ADDED = INTEGER (Returned)
      }{
         The number of names added to the group by this routine.
      }
      \sstsubsection{
         FLAG = LOGICAL (Returned)
      }{
         .TRUE. if the last character in the group expression is equal
         to the current flag character for group IGRP2. Note, if this
         is the case, then the flag character itself is not included
         in the returned group. FLAG is returned .FALSE. if the last
         character is not a flag character. Returned .FALSE. if an
         error occurs.
      }
      \sstsubsection{
         STATUS = INTEGER (Given and Returned)
      }{
         The global status.
      }
   }
}
\sstroutine{
   GRP\_GRPSZ
}{
   Returns the number of names in a group
}{
   \sstdescription{
      This routine returns the number of names in a group.
   }
   \sstinvocation{
      CALL GRP\_GRPSZ( IGRP, SIZE, STATUS )
   }
   \sstarguments{
      \sstsubsection{
         IGRP = INTEGER (Given)
      }{
         A GRP identifier for the group.
      }
      \sstsubsection{
         SIZE = INTEGER (Returned)
      }{
         The number of names in the group. Returned equal to one if an
         error occurs.
      }
      \sstsubsection{
         STATUS = INTEGER (Given and Returned)
      }{
         The global status.
      }
   }
}
\sstroutine{
   GRP\_GTYPE
}{
   Retrieve the type string stored with a group
}{
   \sstdescription{
      The type string specified when the group was created is retrieved.
   }
   \sstinvocation{
      CALL GRP\_GTYPE( IGRP, TYPE, STATUS )
   }
   \sstarguments{
      \sstsubsection{
         IGRP = INTEGER (Given)
      }{
         A GRP group identifier.
      }
      \sstsubsection{
         TYPE = CHARACTER $*$ ( $*$ ) (Returned)
      }{
         The group type. The maximum allowable type length is given by
         the symbolic constant GRP\_\_SZTYP. If the supplied variable is
         too short, the type is truncated but no error is reported. If
         an error occurs, the string is returned blank.
      }
      \sstsubsection{
         STATUS = INTEGER (Given and Returned)
      }{
         The global status.
      }
   }
}
\sstroutine{
   GRP\_HEAD
}{
   Finds the group which is at the head of an owner-slave chain
}{
   \sstdescription{
      This routine climbs the chain of owners starting at the group
      identified by IGRP1, until a group is found which has no owner.
      The identifier issued for this group is returned in IGRP2. If
      the group identified by IGRP1 has no owner, then IGRP2 is returned
      equal to IGRP1.
   }
   \sstinvocation{
      CALL GRP\_HEAD( IGRP1, IGRP2, STATUS )
   }
   \sstarguments{
      \sstsubsection{
         IGRP1 = INTEGER (Given)
      }{
         A group identifier.
      }
      \sstsubsection{
         IGRP2 = INTEGER (Returned)
      }{
         The identifier for the group which is at the head of the
         owner-slave chain. Returned equal to GRP\_\_NOID if an error
         occurs.
      }
      \sstsubsection{
         STATUS = INTEGER (Given and Returned)
      }{
         The global status.
      }
   }
}
\sstroutine{
   GRP\_INDEX
}{
   Searches for a given name and if found, returns its index
}{
   \sstdescription{
      The group is searched for the given name, starting at the name
      with index given by START, and continuing to the end of the
      group.  If it is found then the corresponding index within the
      group is returned. If it is not found, the index is set to zero,
      but no error status is generated. The search is case sensitive
      unless \htmlref{GRP\_SETCS}{GRP_SETCS} has been called to indicate that the group is
      case insensitive.  If the section of the group searched contains
      the name more than once then the lowest index is returned.
   }
   \sstinvocation{
      CALL GRP\_INDEX( NAME, IGRP, START, INDEX, STATUS )
   }
   \sstarguments{
      \sstsubsection{
         NAME = CHARACTER $*$ ( $*$ ) (Given)
      }{
         The name to be searched for.
      }
      \sstsubsection{
         IGRP = INTEGER (Given)
      }{
         A GRP identifier for the group to be searched.
      }
      \sstsubsection{
         START = INTEGER (Given)
      }{
         The lowest index to be checked.
      }
      \sstsubsection{
         INDEX = INTEGER (Returned)
      }{
         The index of the name within the group. This number is greater
         than or equal to START if the name is found, and zero if it is
         not found.
      }
      \sstsubsection{
         STATUS = INTEGER (Given and Returned)
      }{
         The global status.
      }
   }
}
\sstroutine{
   GRP\_INFOC
}{
   Retrieve an item of character information about a name
}{
   \sstdescription{
      This routine returns an item of character information about a
      single name from a group. The item can be any one of those
      described under argument ITEM.
   }
   \sstinvocation{
      CALL GRP\_INFOC( IGRP, INDEX, ITEM, VALUE, STATUS )
   }
   \sstarguments{
      \sstsubsection{
         IGRP = INTEGER (Given)
      }{
         A GRP group identifier.
      }
      \sstsubsection{
         INDEX = INTEGER (Given)
      }{
         An index within the group specified by IGRP.  If the supplied
         value is outside the bounds of the group, then a blank value
         is returned for VALUE, and an error is reported.
      }
      \sstsubsection{
         ITEM = CHARACTER $*$ ( $*$ ) (Given)
      }{
         The name of an item of information. This can be any of the
         following (abbreviations are not allowed):

         NAME  -  The name itself. If the group is case insensitive
         (as established by a call to routine \htmlref{GRP\_SETCS}{GRP_SETCS}) then the name
         is returned in upper case.

         FILE  -  The text file within which the name was explicitly
         given. If the name was not specified within a file then FILE
         is returned blank.
      }
      \sstsubsection{
         VALUE = CHARACTER $*$ ( $*$ ) (Returned)
      }{
         The requested item of information. If the supplied character
         variable is too short, the string is truncated. If an error
         occurs a blank value is returned.
      }
      \sstsubsection{
         STATUS = INTEGER (Given and Returned)
      }{
         The global status.
      }
   }
}
\sstroutine{
   GRP\_INFOI
}{
   Retrieve an item of integer information about a name
}{
   \sstdescription{
      This routine returns an item of integer information about a
      single name from a group. The item can be any one of those
      described under argument ITEM.

      It can also return the current number of active GRP identifiers.
   }
   \sstinvocation{
      CALL GRP\_INFOI( IGRP, INDEX, ITEM, VALUE, STATUS )
   }
   \sstarguments{
      \sstsubsection{
         IGRP = INTEGER (Given)
      }{
         A GRP group identifier.
      }
      \sstsubsection{
         INDEX = INTEGER (Given)
      }{
         An index within the group specified by IGRP.  If the supplied
         value is outside the bounds of the group, then a \texttt{"}null\texttt{"} value
         is returned for VALUE as described below, and an error is
         reported.
      }
      \sstsubsection{
         ITEM = CHARACTER $*$ ( $*$ ) (Given)
      }{
         The name of an item of information. This can be any of the
         following (abbreviations are not allowed):

         MODGRP  -  If the name was specified by means of a
         modification element, then the the GRP identifier of the group
         used as a basis for the modification element is returned in
         VALUE. If the name was not specified by a modification
         element, then GRP\_\_NOID is returned.  If INDEX is outside the
         bounds of the group, then a value of GRP\_\_NOID is returned.

         MODIND  -  If the name was specified by means of a
         modification element, then the index of the original name
         (upon which the returned name was based) is returned in VALUE.
         This is an index into the group identified by MODGRP. If
         MODGRP is returned equal to GRP\_\_NOID, then MODIND will be
         zero.

         DEPTH  -  The number of levels of indirection at which the
         name was specified is returned in VALUE. Names given
         explicitly within a group expression have a DEPTH value of
         zero. Names given explicitly within a DEPTH zero indirection
         element have DEPTH 1. Names given explicitly within a DEPTH 1
         indirection element, have DEPTH 2, etc. If INDEX is out of
         bounds, then zero is returned.

         NGRP - The number of GRP group identifiers currently in use.
         The values of the IGRP and INDEX arguments are ignored.

         ACTIVE - Like NGRP, except that in addition to returning the
         number of GRP group identifiers currently in use, the numerical
         identifier values are also listed on standard output (using
         MSG\_OUT).

         ALLSIMPLE - A value of 1 is returned if all elements within the
         group were specified explicitly (i.e.\ by a literal name rather
         than by indirection or modification). Otherwise a value of zero
         is returned. The value of the INDEX argument is ignored.
      }
      \sstsubsection{
         VALUE = INTEGER (Returned)
      }{
         The requested item of information.
      }
      \sstsubsection{
         STATUS = INTEGER (Given and Returned)
      }{
         The global status.
      }
   }
}
\sstroutine{
   GRP\_LIST
}{
   Write names to a text file specified by the environment
}{
   \sstdescription{
      A text file is created with a name obtained from the environment
      using the supplied parameter. The supplied comment is written to
      the file as the first record (so long as it is not blank), using
      the groups current comment character (see routine \htmlref{GRP\_SETCC}{GRP_SETCC}) .
      All the names stored within the specified group section are then
      written to the file, one name per record. If the group is case
      insensitive (as set up by a call to routine \htmlref{GRP\_SETCS}{GRP_SETCS}) then the
      names are written out in upper case, otherwise they are written
      out as supplied.
   }
   \sstinvocation{
      CALL GRP\_LIST( PARAM, INDXLO, INDXHI, COMNT, IGRP, STATUS )
   }
   \sstarguments{
      \sstsubsection{
         PARAM = CHARACTER $*$ ( $*$ ) (Given)
      }{
         The parameter to be used to get the name of the text file to
         be created.
      }
      \sstsubsection{
         INDXLO = INTEGER (Given)
      }{
         The low index limit of the group section. If both INDXLO and
         INDXHI are zero, then the entire group is used.
      }
      \sstsubsection{
         INDXHI = INTEGER (Given)
      }{
         The high index limit of the group section.
      }
      \sstsubsection{
         COMNT = CHARACTER $*$ ( $*$ ) (Given)
      }{
         A comment line to form the first record in the file. The text
         is prefixed with the group\texttt{'}s current comment character before
         being written to the file.
      }
      \sstsubsection{
         IGRP = INTEGER (Given)
      }{
         The GRP identifier for the group to be listed.
      }
      \sstsubsection{
         STATUS = INTEGER (Given and Returned)
      }{
         The global status.
      }
   }
}
\sstroutine{
   GRP\_LISTF
}{
   Write names to a specified text file
}{
   \sstdescription{
      A text file is created with the specified name.  The supplied
      comment is written to the file as the first record (so long as it
      is not blank), using the groups current comment character (see
      routine \htmlref{GRP\_SETCC}{GRP_SETCC}). All the names stored within the specified
      group section are then written to the file, one name per record.
      If the group is case insensitive (as set up by a call to routine
      \htmlref{GRP\_SETCS}{GRP_SETCS}) then the names are written out in upper case,
      otherwise they are written out as supplied.

      The routine \htmlref{GRP\_LIST}{GRP_LIST} can be used if the file name is to be
      obtained through the parameter system.
   }
   \sstinvocation{
      CALL GRP\_LISTF( FILENM, INDXLO, INDXHI, COMNT, IGRP, STATUS )
   }
   \sstarguments{
      \sstsubsection{
         FILENM = CHARACTER $*$ ( $*$ ) (Given)
      }{
         The name of the text file to be created.
      }
      \sstsubsection{
         INDXLO = INTEGER (Given)
      }{
         The low index limit of the group section. If both INDXLO and
         INDXHI are zero, then the entire group is used.
      }
      \sstsubsection{
         INDXHI = INTEGER (Given)
      }{
         The high index limit of the group section.
      }
      \sstsubsection{
         COMNT = CHARACTER $*$ ( $*$ ) (Given)
      }{
         A comment line to form the first record in the file. The text
         is prefixed with a the groups current comment character before
         being written to the file.
      }
      \sstsubsection{
         IGRP = INTEGER (Given)
      }{
         The GRP identifier for the group to be listed.
      }
      \sstsubsection{
         STATUS = INTEGER (Given and Returned)
      }{
         The global status.
      }
   }
}
\sstroutine{
   GRP\_MSG
}{
   Assign an element of a group to a message token
}{
   \sstdescription{
      The routine assigns a specified element of a GRP group to a message
      token for use in constructing messages with the ERR\_ and MSG\_
      routines (see SUN/104).
   }
   \sstinvocation{
      CALL GRP\_MSG( TOKEN, IGRP, INDEX )
   }
   \sstarguments{
      \sstsubsection{
         TOKEN = CHARACTER $*$ ( $*$ ) (Given)
      }{
         Name of the message token.
      }
      \sstsubsection{
         IGRP = INTEGER (Given)
      }{
         A GRP identifier for the group.
      }
      \sstsubsection{
         INDEX = INTEGER (Given)
      }{
         The index of the element to assign to the message token.
      }
   }
   \sstnotes{
      \sstitemlist{

         \sstitem
         This routine has no inherited status argument. It will always
         attempt to execute, and no error will be reported if it should
         fail (although the message token will be assigned a blank string).
      }
   }
}
\sstroutine{
   GRP\_NEW
}{
   Create a new empty group
}{
   \sstdescription{
      A new empty group is created and an identifier to it is returned
      in IGRP. The string supplied in TYPE is stored with the group,
      and should be used to store a description of the contents of the
      group (see also routines \htmlref{GRP\_GTYPE}{GRP_GTYPE} and \htmlref{GRP\_PTYPE}{GRP_PTYPE}).

      The created group has the default control characters described in
      routine \htmlref{GRP\_SETCC}{GRP_SETCC}, and has no \texttt{"}owner\texttt{"} group (see \htmlref{GRP\_OWN}{GRP_OWN}).
   }
   \sstinvocation{
      CALL GRP\_NEW( TYPE, IGRP, STATUS )
   }
   \sstarguments{
      \sstsubsection{
         TYPE = CHARACTER $*$ ( $*$ ) (Given)
      }{
         A descriptive string to be associated with the group. The
         maximum length for a TYPE string is given by GRP\_\_SZTYP.
         Supplied characters beyond this length are ignored.
      }
      \sstsubsection{
         IGRP = INTEGER (Returned)
      }{
         An identifier for the created group. GRP\_\_NOID is returned if
         an error occurs.
      }
      \sstsubsection{
         STATUS = INTEGER (Given and Returned)
      }{
         The global status.
      }
   }
}
\sstroutine{
   GRP\_OWN
}{
   Returns the identifier of the group which owns the specified
   group
}{
   \sstdescription{
      If the group identified by IGRP1 has had an \texttt{"}owner\texttt{"} group
      established for it by a call to \htmlref{GRP\_SOWN}{GRP_SOWN}, then the identifier of
      the owner group is returned in IGRP2. Otherwise, the value
      GRP\_\_NOID is returned (but no error is reported).
   }
   \sstinvocation{
      CALL GRP\_OWN( IGRP1, IGRP2, STATUS )
   }
   \sstarguments{
      \sstsubsection{
         IGRP1 = INTEGER (Given)
      }{
         An identifier for the slave group whose owner is to be
         returned. If GRP\_\_NOID is supplied, then IGRP2 is returned
         holding GRP\_\_NOID and no error is reported.
      }
      \sstsubsection{
         IGRP2 = INTEGER (Returned)
      }{
         An identifier for the group which owns the group identified by
         IGRP1. Returned equal to GRP\_\_NOID if an error occurs.
      }
      \sstsubsection{
         STATUS = INTEGER (Given and Returned)
      }{
         The global status.
      }
   }
}
\sstroutine{
   GRP\_PTYPE
}{
   Associate a new type string with a group
}{
   \sstdescription{
      The given type string is stored with the group, replacing the
      previous value.  The maximum length of the type string is given
      by symbolic constant GRP\_\_SZTYP. If the supplied type string is
      longer than this, the title is truncated.
   }
   \sstinvocation{
      CALL GRP\_PTYPE( IGRP, TYPE, STATUS )
   }
   \sstarguments{
      \sstsubsection{
         IGRP = INTEGER (Given)
      }{
         A GRP identifier for the group.
      }
      \sstsubsection{
         TYPE = CHARACTER $*$ ( $*$ ) (Given)
      }{
         The group type.
      }
      \sstsubsection{
         STATUS = INTEGER (Given and Returned)
      }{
         The global status.
      }
   }
}
\sstroutine{
   GRP\_PURGE
}{
   Purge duplicate entries from a group
}{
   \sstdescription{
      This routine creates a new group based on a given existing group.
      The contents of the existing group are copied to the new group,
      but any duplicated names are only included once. The check for
      duplication is case sensitive unless the group has been declared
      case insensitive by a call to \htmlref{GRP\_SETCS}{GRP_SETCS}. The new group inherits
      the type, control characters case sensitivity flag of the old
      group, but does not inherit any owner/slave relationships (see
      routine \htmlref{GRP\_SOWN}{GRP_SOWN}).

      Note, indices determined from the old group will in general not
      point to the same name in the new group. The old group should be
      deleted using \htmlref{GRP\_DELET}{GRP_DELET} if it is no longer required.
   }
   \sstinvocation{
      CALL GRP\_PURGE( IGRP1, IGRP2, STATUS )
   }
   \sstarguments{
      \sstsubsection{
         IGRP1 = INTEGER (Given)
      }{
         The GRP identifier for an existing group.
      }
      \sstsubsection{
         IGRP2 = INTEGER (Returned)
      }{
         A GRP identifier for the created group. This group is a purged
         form of the group identified by IGRP1. A value of GRP\_\_NOID is
         returned if an error occurs.
      }
      \sstsubsection{
         STATUS = INTEGER (Given and Returned)
      }{
         The global status.
      }
   }
}
\sstroutine{
   GRP\_PUT
}{
   Put a given set of literal names into a group
}{
   \sstdescription{
      The given names are stored in the group in the form in which they
      are supplied (including any control characters). They overwrite
      any previous names stored at the specified indices. The group is
      extended if the range of indices extends beyond the current size
      of the group. The names can be appended to the end of the group
      by giving INDEX a value of zero or one greater than the current
      size of the group. An error is reported if the names are added
      beyond the end of the group (i.e.\  if adding the names would
      result in a gap within the group for which no names would be
      defined).
   }
   \sstinvocation{
      CALL GRP\_PUT( IGRP, SIZE, NAMES, INDEX, STATUS )
   }
   \sstarguments{
      \sstsubsection{
         IGRP = INTEGER (Given)
      }{
         A GRP identifier for the group.
      }
      \sstsubsection{
         SIZE = INTEGER (Given)
      }{
         The size of the NAMES array.
      }
      \sstsubsection{
         NAMES( SIZE ) = CHARACTER $*$ ( $*$ ) (Given)
      }{
         The names to be stored in the group. The first name is stored
         at the index given by INDEX, the last is stored at index
         INDEX$+$SIZE-1.
      }
      \sstsubsection{
         INDEX = INTEGER (Given)
      }{
         The index at which to store the first name. A value of zero
         causes the names to be appended to the end of the group.
      }
      \sstsubsection{
         STATUS = INTEGER (Given and Returned)
      }{
         The global status.
      }
   }
}
\sstroutine{
   GRP\_PUT1
}{
   Put a single name into a group
}{
   \sstdescription{
      The given name is stored in the group in the form in which it is
      supplied (including any control characters). It overwrites
      any previous name stored at the specified index. The name can be
      appended to the end of the group by giving INDEX a value of zero
      or one greater than the current size of the group. An error is
      reported if the name is added beyond the end of the group (i.e.\
      if adding the name would result in a gap within the group for
      which no names would be defined).
   }
   \sstinvocation{
      CALL GRP\_PUT1( IGRP, NAME, INDEX, STATUS )
   }
   \sstarguments{
      \sstsubsection{
         IGRP = INTEGER (Given)
      }{
         A GRP identifier for the group.
      }
      \sstsubsection{
         NAME = CHARACTER $*$ ( $*$ ) (Given)
      }{
         The name to be stored in the group.
      }
      \sstsubsection{
         INDEX = INTEGER (Given)
      }{
         The index at which to store the name. A value of zero
         causes the names to be appended to the end of the group.
      }
      \sstsubsection{
         STATUS = INTEGER (Given and Returned)
      }{
         The global status.
      }
   }
}
\sstroutine{
   GRP\_REMOV
}{
   Remove all occurrences of a given name from a group
}{
   \sstdescription{
      A new group is created by copying the contents of an existing
      group, excluding any occurrences of a specified name. Note, a name
      with a given index in the input group will in general have a
      different index in the output group. The new group inherits the
      type, control characters and case sensitivity flag of the old
      group, but does not inherit any owner/slave relationships (see
      routine \htmlref{GRP\_SOWN}{GRP_SOWN}). If the input group is no longer required, it
      should be deleted using \htmlref{GRP\_DELET}{GRP_DELET}.
   }
   \sstinvocation{
      CALL GRP\_REMOV( IGRP1, NAME, IGRP2, STATUS )
   }
   \sstarguments{
      \sstsubsection{
         IGRP1 = INTEGER (Given)
      }{
         A GRP identifier for the input group.
      }
      \sstsubsection{
         NAME = CHARACTER $*$ ( $*$ )(Given)
      }{
         The name to be removed. Leading blanks are significant, and
         case is also significant unless the group has been marked as
         case insensitive by calling \htmlref{GRP\_SETCS}{GRP_SETCS}.
      }
      \sstsubsection{
         IGRP2 = INTEGER (Returned)
      }{
         A GRP identifier for the created group. Returned equal to
         GRP\_\_NOID if an error occurs.
      }
      \sstsubsection{
         STATUS = INTEGER (Given and Returned)
      }{
         The global status.
      }
   }
}
\sstroutine{
   GRP\_SAME
}{
   Determine if two GRP identifiers refer to the same group
}{
   \sstdescription{
      The routines returns a flag indicating if the two supplied GRP
      identifiers refer to the same group.
   }
   \sstinvocation{
      CALL GRP\_SAME( IGRP1, IGRP2, SAME, STATUS )
   }
   \sstarguments{
      \sstsubsection{
         IGRP1 = INTEGER (Given)
      }{
         The first GRP identifier.
      }
      \sstsubsection{
         IGRP2 = INTEGER (Given)
      }{
         The second GRP identifier.
      }
      \sstsubsection{
         SAME = LOGICAL (Returned)
      }{
         Returned set to .TRUE. if the two identifiers refer to the
         same group, and .FALSE. otherwise.
      }
      \sstsubsection{
         STATUS = INTEGER (Given and Returned)
      }{
         The global status.
      }
   }
}
\sstroutine{
   GRP\_SETCC
}{
   Sets requested control characters for the specified group
}{
   \sstdescription{
      Each group has associated with it several \texttt{"}control characters\texttt{"}
      which are the characters used to indicate various items of syntax
      within a group expression. These characters are listed and
      described in the \texttt{"}Notes\texttt{"} section below. The control characters are
      given default values when the group is created, but can be changed
      at any time by calling this routine.

      Checks for particular control characters may be suppressed by
      assigning the NULL character to them. The NULL character is
      itself a control character which may be assigned a value using
      this routine. Some control characters form pairs, and an error is
      reported if only one member of a pair is assigned the NULL value.
      These pairs are OPEN\_NEST and CLOSE\_NEST, and OPEN\_KERNEL and
      CLOSE\_KERNEL.

      If a blank value for argument CCLIST is supplied, then the default
      control characters described in the \texttt{"}Notes\texttt{"} section are
      re-established.

      An error is reported if any two control characters are the same.
      The exception to this is that any number of control characters
      may have the same value as the NULL control character. If any
      error occurs, the control characters are left unaltered.
   }
   \sstinvocation{
      CALL GRP\_SETCC( IGRP, CCLIST, CC, STATUS )
   }
   \sstarguments{
      \sstsubsection{
         IGRP = INTEGER (Given)
      }{
         A GRP identifier for the group to which the control characters
         refer.
      }
      \sstsubsection{
         CCLIST = CHARACTER $*$ ( $*$ ) (Given)
      }{
         A comma separated list of names specifying which control
         character are to be altered. Unambiguous abbreviations may be
         used. A blank value causes all control characters to be reset
         to the default values.
      }
      \sstsubsection{
         CC = CHARACTER $*$ ( $*$ ) (Given)
      }{
         A list of the new control characters, in the same order as the
         corresponding names in CCLIST. Note, if CCLIST contains N
         names, then the first N characters are used out of the string
         specified by CC. If the total length (including any trailing
         blanks) of CC is less than N, then an error is reported.
      }
      \sstsubsection{
         STATUS = INTEGER (Given and Returned)
      }{
         The global status.
      }
   }
   \sstnotes{
      The following names are used to refer to the individual control
      characters:
      \sstitemlist{

         \sstitem
         INDIRECTION: (Default \texttt{"}$\wedge$\texttt{"}) The name given to the character used
         to indicate that an element of a group expression is the name of a
         text file from which further elements should be read.

         \sstitem
         COMMENT: (Default \texttt{"}\#\texttt{"}) The name given to the character used to
         introduce comments in group expressions.

         \sstitem
         DELIMITER: (Default \texttt{"},\texttt{"}) The name given to the character used
         to delimit elements within group expressions. Note, delimiters
         within group expressions are ignored if they occur within matched
         nesting characters (see OPEN\_NEST and CLOSE\_NEST below).

         \sstitem
         NAME\_TOKEN: (Default \texttt{"}$*$\texttt{"}) The name given to the character used
         as a token for input names in a modification element.

         \sstitem
         SEPARATOR: (Default \texttt{"}$|$\texttt{"}) The name given to the character used
         to separate the substitution strings within a modification
         element.

         \sstitem
         OPEN\_NEST: (Default \texttt{"}(\texttt{"}) The name given to the character used
         to open a \texttt{"}nest\texttt{"} within a group expression. Any delimiter
         characters occurring within matched nesting characters are ignored.

         \sstitem
         CLOSE\_NEST: (Default \texttt{"})\texttt{"}) The name given to the character used
         to close a \texttt{"}nest\texttt{"} within a group expression.

         \sstitem
         FLAG: (Default \texttt{"}-\texttt{"}) The name given to a character which can
         be appended to the end of a group expression in order to
         \texttt{"}flag\texttt{"} that expression. The interpretation of this flag is left
         up to the application.

         \sstitem
         OPEN\_KERNEL: (Default \texttt{"}\{\texttt{"}) The name given to the character used
         to open a \texttt{"}kernel\texttt{"} within a group expression.

         \sstitem
         CLOSE\_KERNEL: (Default \texttt{"}\}\texttt{"}) The name given to the character used
         to close a \texttt{"}kernel\texttt{"} within a group expression.

         \sstitem
         NULL: (Default \texttt{"}\%\texttt{"}) The name given to the character which can
         be assigned to other control characters to suppress checks for
         those control characters.  If this is changed, any other characters
         currently set to the null character are also changed to the new NULL
         character.

         \sstitem
         ESCAPE: (Default to the NULL character) The name given to the
         character which can be used to escape control characters within
         a group expression.
      }
   }
}
\sstroutine{
   GRP\_SETCS
}{
   Establish the case sensitivity of a group
}{
   \sstdescription{
      When a group is created using \htmlref{GRP\_NEW}{GRP_NEW}, it is initially case
      sensitive. This routine can be called to make it case
      insensitive, or to make it case sensitive again.

      All names stored within a case sensitive group are used in the
      same form as they were supplied by the environment or
      application. If the group is case insensitive the upper case
      equivalent is used when any reference is made to a name stored
      within the group, and all comparisons between strings (such as
      performed within routines \htmlref{GRP\_INDEX}{GRP_INDEX} and \htmlref{GRP\_PURGE}{GRP_PURGE} for instance)
      related to that group are performed without reference to case.

      Note, names are always stored in the form they are given.  Any
      case conversion takes place when the names are read out of the
      group, not when they are put into the group. This means that
      groups can be changed at any time from being case insensitive to
      being case sensitive (or vice-versa).
   }
   \sstinvocation{
      CALL GRP\_SETCS( IGRP, SENSIT, STATUS )
   }
   \sstarguments{
      \sstsubsection{
         IGRP = INTEGER (Given)
      }{
         A GRP identifier for the group.
      }
      \sstsubsection{
         SENSIT = LOGICAL (Given)
      }{
         If .TRUE. then the group is made case sensitive. If .FALSE.,
         then the group is made case insensitive.
      }
      \sstsubsection{
         STATUS = INTEGER (Given and Returned)
      }{
         The global status.
      }
   }
}
\sstroutine{
   GRP\_SETSZ
}{
   Reduce the size of a group
}{
   \sstdescription{
      This routine sets the size of the given group to the specified
      value. The new size must be less than or equal to the old size.
      The names with indices greater than the new size are lost. Other
      names remain unaltered.
   }
   \sstinvocation{
      CALL GRP\_SETSZ( IGRP, SIZE, STATUS )
   }
   \sstarguments{
      \sstsubsection{
         IGRP = INTEGER (Given)
      }{
         A GRP identifier for the group.
      }
      \sstsubsection{
         SIZE = INTEGER (Given)
      }{
         The new group size. If a negative value is given, then zero
         is used.
      }
      \sstsubsection{
         STATUS = INTEGER (Given and Returned)
      }{
         The global status.
      }
   }
}
\sstroutine{
   GRP\_SHOW
}{
   List contents ofa group to the screen
}{
   \sstdescription{
      The contents of the supplied group are listed to the screen, with
      one name on each line. If the group has a non-blank type string, it
      is dislayed first, prefixed with the groups current comment character
      (see routine GRP\_SETCC). If SLAVES is .TRUE., then any
      slave group owned by the supplied group is also displayed. This is
      recursive - any slave group owned by a displayed slave group is also
      displayed.
   }
   \sstinvocation{
      CALL GRP\_SHOW( IGRP, SLAVES, STATUS )
   }
   \sstarguments{
      \sstsubsection{
         IGRP = INTEGER (Given)
      }{
         The GRP identifier for the group to be displayed.
      }
      \sstsubsection{
         SLAVES = LOGICAL (Given)
      }{
         If TRUE., then the entire chain of slaved groups owned by the
         specified group are also displayed.
      }
      \sstsubsection{
         STATUS = INTEGER (Given and Returned)
      }{
         The global status.
      }
   }
   \sstnotes{
      \sstitemlist{

         \sstitem
         If the group is case insensitive (as set up by a call to
         routine GRP\_SETCS) then the names are written out in upper case,
         otherwise they are written out as supplied.
      }
   }
}
\sstroutine{
   GRP\_SLAVE
}{
   Returns the identifier of the group which is owned by the specified
   group
}{
   \sstdescription{
      If the group identified by IGRP1 has had a \texttt{"}slave\texttt{"} group
      established for it by a call to GRP\_SOWN, then the identifier of
      the slave group is returned in IGRP2. Otherwise, the value
      GRP\_\_NOID is returned (but no error is reported).
   }
   \sstinvocation{
      CALL GRP\_SLAVE( IGRP1, IGRP2, STATUS )
   }
   \sstarguments{
      \sstsubsection{
         IGRP1 = INTEGER (Given)
      }{
         An identifier for the owner group whose slave is to be
         returned.
      }
      \sstsubsection{
         IGRP2 = INTEGER (Returned)
      }{
         An identifier for the group which is owned by the group identified
         by IGRP1. Returned equal to GRP\_\_NOID if an error occurs.
      }
      \sstsubsection{
         STATUS = INTEGER (Given and Returned)
      }{
         The global status.
      }
   }
}
\sstroutine{
   GRP\_SOWN
}{
   Establish one group as the owner of another group
}{
   \sstdescription{
      This routine establishes one specified group as \texttt{"}owner\texttt{"} of
      another specified group. An error is reported if the group is
      already owned by another group. An error is also reported if the
      \texttt{"}owner\texttt{"} group already owns a slave.

      This routine may also be used to cancel an \texttt{"}owner-slave\texttt{"}
      relationship, by specifying IGRP2 as GRP\_\_NOID. The group
      identified by IGRP1 then becomes a \texttt{"}free\texttt{"} group (i.e.\ has no
      owner). An error is reported if IGRP1 identifies a group which is
      already free.
   }
   \sstinvocation{
      CALL GRP\_SOWN( IGRP1, IGRP2, STATUS )
   }
   \sstarguments{
      \sstsubsection{
         IGRP1 = INTEGER (Given)
      }{
         The identifier of the \texttt{"}slave\texttt{"} group which is to have an owner
         established for it. An error is reported if an invalid
         identifier is given.
      }
      \sstsubsection{
         IGRP2 = INTEGER (Given)
      }{
         The identifier of the group which is to be established as
         the owner of the group identified by IGRP1.
      }
      \sstsubsection{
         STATUS = INTEGER (Given and Returned)
      }{
         The global status.
      }
   }
   \sstnotes{
      \sstitemlist{

         \sstitem
         There is a restriction on the use of owner-slave
         relationships, namely that a slave cannot own its own owner,
         either directly or indirectly. That is, if group A is owned by
         group B, and group B is owned by group C, then group C cannot be
         owned by either group B, or group A. An error is reported, if an
         attempt is made to set up such a relationship.

         \sstitem
         When a group is deleted using \htmlref{GRP\_DELET}{GRP_DELET}, all other groups in
         the same owner/slave chain, whether higher up or lower down, are
         also deleted. If a group is to be deleted without deleting all
         other related groups, then the group must be established as a
         \texttt{"}free\texttt{"} group (i.e.\ no owner) by calling this routine with IGRP2
         set to GRP\_\_NOID, before calling \htmlref{GRP\_DELET}{GRP_DELET}.
      }
   }
}
\sstroutine{
   GRP\_VALID
}{
   Determine if a group identifier is valid
}{
   \sstdescription{
      Argument VALID is returned .TRUE. if the group identified by IGRP
      is valid, and is returned .FALSE. otherwise.
   }
   \sstinvocation{
      CALL GRP\_VALID( IGRP, VALID, STATUS )
   }
   \sstarguments{
      \sstsubsection{
         IGRP = INTEGER (Given)
      }{
         A GRP identifier.
      }
      \sstsubsection{
         VALID = LOGICAL (Returned)
      }{
         The status of the group.
      }
      \sstsubsection{
         STATUS = INTEGER (Given and Returned)
      }{
         The global status.
      }
   }
}
\sstroutine{
   GRP\_WATCH
}{
   Watch for events in the life of a specified group
}{
   \sstdescription{
      This routine causes messages to be reported when a group with the
      given integer identifier is created or destroyed. It will also
      list the integer identifiers for all currently active groups.

      Each event in the life of the group is repoirted by routine
      GRP\_ALARM, and so a debugger break point can be set there to
      investigate such events.
   }
   \sstinvocation{
      CALL GRP\_WATCH( IGRP, STATUS )
   }
   \sstarguments{
      \sstsubsection{
         IGRP= INTEGER (Given)
      }{
         The identifier for the group to be watched. If this is zero,
         the identifiers of all currently active groups are listed on
         standard output. NOTE, in the C interface this should be an
         integer, not a pointer to a Grp structure.
      }
      \sstsubsection{
         STATUS = INTEGER (Given and Returned)
      }{
         The global status.
      }
   }
   \sstnotes{
      \sstitemlist{

         \sstitem
         This routine attempts to run even if the global status value
         indicates an error has already occurred.
      }
   }
}

\section{\label{APP:ERRORS}GRP Error Status Values}
This appendix lists the STATUS values generated within the GRP package. Note,
STATUS values generated by the packages listed in appendix \ref{APP:PACK} may
also be returned.

\begin{description}
\item [GRP\_\_BADCC] - An ambiguous or un-recognised control character name has
been supplied.
\item [GRP\_\_BADIT] - An un-recognised item of information has been requested.
\item [GRP\_\_BADME] - A modification element has been given which has an
illegal format.
\item [GRP\_\_DEEP] - The maximum depth of indirection has been exceeded.
\item [GRP\_\_EMPTY] - The specified group is empty.
\item [GRP\_\_FIOER] - A Fortran I/O error has occurred.
\item [GRP\_\_FREE] - The group to be made free is already free.
\item [GRP\_\_INTER] - An internal GRP error has occurred.
\item [GRP\_\_INVID] - An invalid group identifier was given.
\item [GRP\_\_NOLUN] - No free logical unit numbers available.
\item [GRP\_\_NOMOR] - No more groups can be created.
\item [GRP\_\_NULNM] - A zero length name was given.
\item [GRP\_\_OUTBN] - A group has been indexed outside its bounds.
\item [GRP\_\_OWNED] - The group to be enslaved is already owned.
\item [GRP\_\_OWNER] - An attempt has been made to enslave a group to a group
which already owns another group.
\item [GRP\_\_SHORT] - A character variable supplied as an argument was too
short.
\item [GRP\_\_SLAVE] - An attempt has been made to make a group a slave of its
own slave.
\item [GRP\_\_SZINC] - An attempt has been made to increase the size of a
group using \htmlref{GRP\_SETSZ}{GRP_SETSZ}.
\end{description}

\section{\label{APP:PACK}Packages Called From Within GRP}
The GRP package makes use of subroutines from the PSX
library (see \xref{SUN/121}{sun121}{}), the CHR library (see
\xref{SUN/40}{sun40}{}) and the ERR library (see
\xref{SUN/104}{sun104}{}).

\section{Acknowledgements}

Peter Draper is thanked for his work on porting the original IRH\_ package to
UNIX, and Rodney Warren-Smith is thanked for his extensive suggestions for
improvements to the GRP\_ package.

\section{Changes and New Features in V3.7}
The following changes have occurred in the GRP\_ system since version 3.6:

\begin{enumerate}
\item Addition of routine GRP\_SAME.
\item Addition of routine GRP\_SHOW.
\item Addition of routine GRP\_ALARM.
\item Addition of routine GRP\_WATCH.
\end{enumerate}

\section{Changes and New Features in V3.6}
The following changes have occurred in the GRP\_ system since version 3.5:

\begin{enumerate}
\item Addition of routine GRP\_SLAVE.
\end{enumerate}

\section{Changes and New Features in V3.5}
The following changes have occurred in the GRP\_ system since version 3.4:

\begin{enumerate}
\item Addition of routine GRP\_MSG.
\end{enumerate}

\section{Changes and New Features in V3.4}
The following changes have occurred in the GRP\_ system since version 3.3:

\begin{enumerate}
\item The GRP\_INFOI routine has a new option to return a flag indicating
if all elements in a group were supplied literally (i.e.\ not by
indirection or modification).
\end{enumerate}

\section{Changes and New Features in V3.3}
The following changes have occurred in the GRP\_ system since version 3.2:

\begin{enumerate}
\item The GRP\_PUT1 routine has been added.
\end{enumerate}

\section{Changes and New Features in V3.2}
The following changes have occurred in the GRP\_ system since version 3.1:

\begin{enumerate}
\item The GRP\_INFOI routine can now return the total number of active
GRP identifiers.
\item Shell meta-characters (references to environment variables, for
instance) may now be included within file names which specify
indirection elements.

\end{enumerate}

\section{Changes and New Features in V3.1}
The following changes have occurred in the GRP\_ system since version 3.0:

\begin{enumerate}

\item Sections of a group expression can now be marked as ``verbatim text''
by beginning the section with the string ``<!!'' and ending it with
``!!>''. All control characters within such a section are treated as
literal characters (the delimiting strings themselves are thrown away).
\end{enumerate}

\section{Changes and New Features in V3.0}
The following changes have occurred in the GRP\_ system since version 2.0:

\begin{enumerate}

\item The maximum number of groups which can be active at any one time
has been increased from 100 to 500.

\item An escape control character can now be defined which allows other
control characters within a group expression to be treated literally
(i.e.\ their usual special meaning is suppressed and they are trated like a
normal text character if they are preceeded by an escape character). By
default, no escape character is defined.

\item If the NULL control character is changed using GRP\_SETCC, then any
other control characters which are equal to the old NULL character are
changed to the new NULL character, so long as the same call to GRP\_SETCC
does not explicitly set their value to something else.

\end{enumerate}

\section{Changes and New Features in V2.0}
The following changes have occurred in the GRP\_ system since version 1.1:
\begin{enumerate}
\item The facility for editing names which was previously restricted to use
within modification elements, has been made available for use with any form
of element. See section \ref{SEC:GREXP}.
\item Null names (i.e.\ names with zero length) are now accepted and treated as
blank names.
\end{enumerate}

\section{Changes and New Features in V1.1}
The following changes have occurred in the GRP\_ system since version 1.0:
\begin{enumerate}
\item A new routine \htmlref{GRP\_HEAD}{GRP_HEAD} has been introduced to simplify the task of
finding the head of an owner-slave chain.
\item Routines \htmlref{GRP\_GROUP}{GRP_GROUP} and \htmlref{GRP\_GRPEX}{GRP_GRPEX}
\begin{itemize}
\item Argument FLAG is now returned false if an error occurs.
\item If an error occurs, then argument SIZE is returned equal to the supplied
size of the group identified by argument IGRP2. If the group has zero size on
entry then argument SIZE is returned equal to one.
\end{itemize}
\end{enumerate}

\end{document}
